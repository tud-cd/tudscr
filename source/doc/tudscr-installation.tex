\chapter{%
  Weiterführende Installationshinweise%
  \label{sec:install:ext}%
  \index{Installation|(}%
}
\Attention{%
  Hier werden unterschiedliche Varianten erläutert, wie \TUDScript in der 
  Version~\vTUDScript{} genutzt werden kann, falls eine frühere Variante als 
  \textbf{lokale Nutzerinstallation} verwendet wurde.
}

\bigskip\noindent
Bis zur Version~v2.01 wurde \TUDScript ausschließlich über das \Forum zur 
lokalen Nutzerinstallation angeboten. In erster Linie hat das historische 
Hintergründe und hängt mit der Entstehungsgeschichte von \TUDScript zusammen. 
Eine lokale Nutzerinstallation bietet einen~-- eher zu vernachlässigenden~-- 
Vorteil. Treten bei der Verwendung von \TUDScript Probleme auf, können diese im 
Forum gemeldet und diskutiert werden. Ist für ein solches Problem tatsächlich 
eine Fehlerkorrektur respektive Aktualisierung von \TUDScript nötig, kann diese 
schnell und unkompliziert über das \GitHubRepo<releases/latest> bereitgestellt 
und durch den Anwender sofort genutzt werden.

Dies hat allerdings für alle Anwender, welche das Forum relativ wenig oder gar 
nicht besuchen, den großen Nachteil, dass Sie nicht von Aktualisierungen, 
Verbesserungen und Fehlerkorrekturen neuer Versionen profitieren können. Auch 
alle nachfolgenden Bugfixes und Aktualisierungen des \TUDScript-Bundles müssen 
durch den Anwender manuell durchgeführt werden. Daher wird die Verbreitung via 
\CTAN<pkg/tudscr> präferiert, sodass \TUDScript stets in der aktuellen Version 
verfügbar ist~-- eine durch den Anwender aktuell gehaltene \DistributionGeneral 
vorausgesetzt. Der einzige Nachteil bei diesem Ansatz ist, dass die Verbreitung 
eines Bugfixes und die anschließende Bereitstellung durch die verwendete 
\DistributionGeneral für gewöhnlich bis zu zwei Tagen dauert.

Alle gängigen \DistributionGeneral* durchsuchen im Regelfall zuerst das lokale 
\Path{texmf}"=Nutzerverzeichnis nach Klassen sowie Paketen und erst daran 
anschließend den \Path{texmf}"~Pfad der \DistributionGeneral selbst. Dabei 
spielt es keine Rolle, in welchem Pfad die neuere Version einer Klasse oder 
eines Paketes liegt. Sobald im Nutzerverzeichnis die gesuchte Datei gefunden 
wurde, wird die Suche beendet.
\Attention{%
  In der Konsequenz bedeutet dies, dass sämtliche Aktualisierungen für 
  \TUDScript über die genutzte \DistributionGeneral \textbf{nicht} zum Tragen 
  kommen, falls eine lokale Nutzerinstallation vorhanden ist.
}

Deshalb wird Anwendern empfohlen, eine gegebenenfalls vorhandene lokale 
Nutzerinstallation von \TUDScript zu deinstallieren, falls diese nicht 
\emph{bewusst} installiert wurde. Das Vorgehen für eine Deinstallation wird in 
\autoref{sec:local:uninstall} erläutert. Nach dieser können Updates des 
\TUDScript-Bundles durch die Aktualisierungsfunktion der jeweils eingesetzten 
\DistributionGeneral erfolgen. 

Wie das \TUDScript-Bundle trotzdem als lokale Nutzerversion installiert oder 
aktualisiert werden kann, ist in \autoref{sec:local:install} beziehungsweise 
\autoref{sec:local:update} zu finden. Der Anwender sollte in diesem Fall 
allerdings genau wissen, was er damit bezweckt, da er in diesem Fall für die 
Aktualisierung von \TUDScript selbst verantwortlich ist.

\minisec{Nutzung der veralteten Schriftfamilien}
%
\ChangedAt{v2.06}%
Soll ein Dokument noch mit den veralteten Schriftfamilien \Univers und \DIN 
gesetzt werden, so ist eine lokale Installation der Type1-Schriften notwendig, 
welche in \autoref{sec:install:fonts} beschrieben wird. Zusätzlich sei auf die 
beiden Optionen \Option||{tudscrver=2.05} und \Option||{cdoldfont} hingewiesen. 

Für die Nutzung der \OpenSans ist keine lokalle Installation der Schriften 
erforderlich. Es wird lediglich das Paket \Package||{opensans} benötigt, 
welches über die verwendete \DistributionGeneral installiert werden kann und 
via das \CTAN<pkg/opensans> bereitgestellt wird.



\section{%
  Lokale Deinstallation des \TUDScript-Bundles%
  \label{sec:local:uninstall}%
}
%
Über die Kommandozeile beziehungsweise das Terminal kann mit
%
\begin{quoting}
\Path{kpsewhich --all tudscrbase.sty}
\end{quoting}
%
überprüft werden, ob eine lokale Nutzerinstallation von \TUDScript vorhanden 
ist. Es werden alle Pfade ausgegeben, in denen die gesuchte Datei gefunden 
wird. Erscheint nur der Pfad der \DistributionGeneral, ist die 
\TUDScript-Version selbiger aktiv und der Anwender kann mit dem 
\TUDScript-Bundle arbeiten.

Wird \emph{nur} das lokale Nutzerverzeichnis oder gar kein Verzeichnis 
gefunden, so wird wahrscheinlich eine veraltete \DistributionGeneral verwendet. 
In diesem Fall wird eine Aktualisierung dieser \emph{unbedingt} empfohlen. 
Sollte dies nicht möglich sein, \emph{muss} \TUDScript als lokale Nutzerversion 
installiert (\autoref{sec:local:install}) beziehungsweise~-- falls ein Pfad 
ausgegeben wurde~-- aktualisiert (\autoref{sec:local:update}) werden.

Sollte neben dem Pfad der \DistributionGeneral noch mindestens ein weiterer 
Pfad angezeigt werden, so ist eine lokale Nutzerversion installiert. In diesem 
Fall hat der Anwender drei Möglichkeiten:
%
\begin{enumerate}
\item Entfernen der lokalen Nutzerinstallation (skriptbasiert)
\item Entfernen der lokalen Nutzerinstallation (manuell)
\item Aktualisierung der lokalen Nutzerversion (\autoref{sec:local:update})
\end{enumerate}
%
Um die lokale Nutzerinstallation zu entfernen, kann für Windows
\GitHubDownload*<uninstall>{tudscr_uninstall.bat} sowie für unixartige 
Betriebssysteme \GitHubDownload*<uninstall>{tudscr_uninstall.sh} verwendet 
werden. Nach der Ausführung des jeweiligen Skriptes kann mit dem zu Beginn 
gezeigten Aufruf in der Kommandozeile respektive Terminal geprüft werden, ob 
die Deinstallation erfolgreich war. Wird immer noch mindestens ein lokaler Pfad 
ausgegeben, sollte \TUDScript manuelle deinstalliert werden, was nachfolgend 
beschrieben wird.

Nur die Deinstallation aller lokalen Nutzerinstallationen von \TUDScript 
ermöglicht die Verwendung der jeweils aktuellen Version über die 
\DistributionGeneral. Hierfür ist~-- unter der Annahme, dass das automatisierte 
Deinstallieren mithilfe der zuvor genannten Skripte zur Deinstallation nicht 
erfolgreich war~-- etwas Handarbeit durch den Anwender vonnöten. Der in der 
Kommandozeile respektive im Terminal mit
%
\begin{quoting}
\Path{kpsewhich --all tudscrbase.sty}
\end{quoting}
%
gefundene~-- zum Ordner der \DistributionGeneral \emph{zusätzliche}~-- 
Pfad hat die folgende Struktur:
%
\begin{quoting}
\Path{\PName{Installationspfad}/tex/latex/tudscr/tudscrbase.sty}
\end{quoting}
%
Um die Nutzerinstallation vollständig zu entfernen, muss als erstes zu 
\Path{\PName{Installationspfad}} navigiert werden. Anschließend ist in diesem 
Pfad Folgendes durchzuführen:
%
\settowidth\tudscrdim{\Path{tex/latex/tudscr/}~}%
\begin{description}[labelwidth=\tudscrdim,labelsep=.5em]
\item[\Path{tex/latex/tudscr/}]\Path{.cls}- und \Path{.sty}"~Dateien löschen
\item[\Path{tex/latex/tudscr/}]Ordner \Path{logo} vollständig löschen
\item[\Path{doc/latex/}] Ordner \Path{tudscr} vollständig löschen
\item[\Path{source/latex/}] Ordner \Path{tudscr} vollständig löschen
\end{description}
%
Zum Abschluss ist in der Kommandozeile beziehungsweise im Terminal der Befehl 
\Path{texhash} aufzurufen. Damit wurde die lokale Nutzerversion entfernt und es 
wird von nun an die Version von \TUDScript genutzt, welche durch die verwendete 
\DistributionGeneral bereitgestellt wird.



\section{%
  Lokale Installation des \TUDScript-Bundles%
  \label{sec:local:install}%
  \index{Installation!Nutzerinstallation|(}%
}
%
\Attention{%
  Eine lokale Nutzerinstallation des \TUDScript-Bundles sollte ausschließlich 
  durch Anwender ausgeführt werden, die zwingende Gründe hierfür haben und sich 
  der daraus resultierenden Konsequenzen sowie möglicher Problem bewusst sind. 
  Um einen reibungslosen Installationsprozess zu gewährleisten, sollte zuvor 
  unbedingt die \DistributionGeneral aktualisiert werden.
}

Dazu werden die Archive \GitHubDownload*{TUD-Script_\vTUDScript_Windows.zip} 
für Windows beziehungsweise \GitHubDownload*{TUD-Script_\vTUDScript_Unix.zip} 
für unixoide Betriebssysteme bereitgestellt. Bei der Ausführung des jeweiligen 
Installationsskriptes aus dem Archiv werden alle Dateien in das lokale 
Nutzerverzeichnis der jeweiligen \DistributionGeneral installiert. 

Alternativ zur Nutzung der bereitgestellten Installationsskripte kann auch der 
Inhalt des TDS"~Archivs \GitHubDownload*{tudscr_\vTUDScript.zip} in das lokale 
\Path{texmf}"=Nutzerverzeichnis beziehungsweise in ein anderes der 
\DistributionGeneral bekanntes TDS"~Verzeichnis kopiert und abschließend in der 
Kommandozeile respektive im Terminal \Path{texhash} aufgerufen werden. 



\section{%
  Lokales Update des \TUDScript-Bundles%
  \label{sec:local:update}%
}
\subsection{Update des \TUDScript-Bundles ab Version~v2.02}
%
Ein Update und die lokale Installation unterscheiden sich ab der Version~v2.06 
nicht voneinander, das Vorgehen ist absolut identisch zu der Beschreibung in 
\autoref{sec:local:install}.
\Attention{%
  Die lokale Aktualisierung auf Version~\vTUDScript{} funktioniert allerdings 
  nur, wenn \TUDScript bereits mindestens in der Version~v2.02 entweder als 
  lokale Nutzerversion oder über die \DistributionGeneral installiert ist.%
}



\subsection{Update des \TUDScript-Bundles ab Version~v2.00}
%
Mit der Version~v2.02 gab es einige tiefgreifende Änderungen. Deshalb wird für 
vorausgehende Versionen~-- sprich v2.00 und v2.01~-- kein dediziertes Update 
angeboten. Die Aktualisierung kann durch den Anwender entweder~-- wie in 
\autoref{sec:local:install} erläutert~-- mit einer skriptbasierten oder mit 
einer manuellen Neuinstallation erfolgen.%
\index{Installation!Nutzerinstallation|)}%



\subsection{Update des \TUDScript-Bundles von Version v1.0}
%
Ist \TUDScript in der veralteten \emph{Version~v1.0} installiert, so wird vor 
der Aktualisierung dringlichst zu einem vollständigen Entfernen dieser Version 
geraten. Andernfalls werden nach einem Update bei der Verwendung massive 
Probleme und Fehler auftreten. Zur Deinstallation werden die Skripte 
\GitHubDownload*<uninstall>{tudscr_uninstall.bat} respektive
\GitHubDownload*<uninstall>{tudscr_uninstall.sh} bereitgestellt. Die aktuelle 
Version~\vTUDScript{} kann nach der vollständigen Deinstallation aller 
veralteten Versionen wie in \autoref{sec:local:install} beschrieben installiert 
werden.

Im Vergleich zur \emph{Version~v1.0} hat sich an der Benutzerschnittstelle 
nicht sehr viel verändert, ein Umstieg auf die Version~\vTUDScript{} dürfte 
keine Schwierigkeiten bereiten. Treten dabei Probleme auf, kann gegebenenfalls 
das Paket \Package||'full'{tudscrcomp} genutzt werden, welches eine Nutzung 
alter und ursprünglich nicht mehr vorgesehener Befehle sowie Optionen 
ermöglicht. Einige dieser sind jedoch obsolet und werden nicht mehr 
bereitgestellt. Aufgeführt sind diese in \autoref{sec:cessations}. Sollten 
trotz aller Hinweise dennoch Fehler oder Probleme beim Umstieg auf die neue 
\TUDScript-Version auftreten, ist eine Meldung im \GitHubRepo<issues> oder im 
\Forum die beste Möglichkeit, um Hilfe zu erhalten.



\section{%
  Installation veralteter Schriftfamilien%
  \label{sec:install:fonts}%
  \index{Installation!Schriftinstallation|(}%
}
%
\ChangedAt{%
  v2.02:Installationsroutine der Type1-Schriften angepasst;%
  v2.04:Installationsskripte verbessert und robuster gestaltet sowie für 
        portable Distributionen \TeXLive' und \MiKTeX' erweitert;%
  v2.06:\OpenSans erfordert \emph{keine} Schriftinstallation;%
}%
\ToDo{komplett raus?}[v2.07]
Bis Anfang des Jahres~2018 nutzte das \TUDCD als Hausschrift nicht \OpenSans 
sondern die Schriftfamilien \Univers und \DIN. Diese lassen sich weiterhin mit 
\TUDScript verwenden, um alte Dokumente kompilieren zu können. Hierfür sei auf 
die Optionen \Option||{tudscrver=2.05} und \Option||{cdoldfont} hingewiesen. 
Da es sich bei diesen um lizenzierte Schriften handelt, müssen diese beim 
Universitätsmarketing auf \href{https://tu-dresden.de/cd}{Anfrage} mit dem 
Hinweis auf die Verwendung von \Logo{LaTeX} bestellt und nach Erhalt der 
notwendigen Archive \File*{Univers_PS.zip} und \File*{DIN_Bd_PS.zip} für 
Windows (\autoref{sec:install:win}) beziehungsweise unixoide Betriebssysteme 
(\autoref{sec:install:unix}) installiert werden. Ohne einen triftigen Grund 
sollte jedoch in jedem Fall die \OpenSans genutzt werden, insbesondere für neu 
erstellte Dokumente.

Das \TUDScript-Bundle unterstützt besagte Schriften auch im OpenType-Format, 
welche ebenfalls über das Universitätsmarketing auf 
\href{https://tu-dresden.de/cd}{Anfrage} bestellt werden müssen. Die in den 
Archiven \File*{Univers_8_TTF.zip} und \File*{DIN_TTF.zip} enthaltenen 
Schriften lassen sich~-- sobald diese für das Betriebssystem installiert 
wurden~-- mit dem Paket \Package||{fontspec} verwenden. In \fullref{sec:fonts} 
sind weitere Hinweise zur Verwendung dieses Paketes zu finden.

Im \GitHubRepo<releases> sind die zur Schriftinstallation nötigen 
\GitHubRepo[Skripte für \TUDScript]<releases/fonts> ebenso zu finden wie die
\GitHubRepo[Skripte für das Bundle von Klaus~Bergmann]<releases/oldfonts>. Die 
unterschiedlichen Installationsskripte begründen sich insbesondere dadurch, 
dass bei der Installation für das \TUDScript-Bundle sowohl die Metriken als 
auch das Kerning der Schriften für Fließtext und den Mathematikmodus angepasst 
werden. Sollen die so verbesserten Schriften für die Klassen von Klaus~Bergmann 
verwendet werden, kann dies mit dem Paket \Package||{fix-tudscrfonts} erfolgen, 
was allerdings das Ergebnis der erzeugten Ausgabe beeinflusst, weshalb die 
Installationsskripte in unterschiedlichen Varianten weiterhin vorgehalten 
werden.

\minisec{Notwendige Pakete und Skripte für die Schriftinstallation}
%
Zur Schriftinstallation sind zum einen die Pakete \Package*{fontinst} sowie
\Package*{cmbright}, \Package*{hfbright}, \Package*{cm-super} und
\Package*{iwona} von \emph{essentieller} Bedeutung und daher \emph{zwingend} 
notwendig. Zum anderen werden die Skripte \Path{tftopl}, \Path{pltotf} und 
\Path{vptovf} benötigt, welche bei \TeXLive oder \MacTeX über 
\Bundle*{fontware} respektive bei \MiKTeX über \Bundle*{miktex-texware} sowie 
\Bundle*{miktex-fonts} bereitgestellt werden. 

\minisec{Anmerkung zu \Logo{MiKTeX}}
%
Vor der Installation der Schriften für \TUDScript sollte unbedingt ein Update 
von \MiKTeX durchgeführt werden. Außerdem ist es sehr ratsam, die Installation 
in der Mehrbenutzervariante mit Administratorrechten durchzuführen, da die 
Einzelbenutzervariante relativ unregelmäßig und nicht immer nachvollziehbar zu 
Problemen führen kann. Weiterhin sollten die zuvor genannten, zusätzlich 
notwendigen Pakete und Skripte~-- falls nicht vorhanden~-- über den 
\MiKTeX"=Paketmanager hinzugefügt werden.

Das Installationsskript scheitert außerdem bei einigen Anwendern~-- aufgrund 
eingeschränkter Nutzerzugriffsrechte~-- beim Eintragen der Schriften in die 
Map"~Datei. Dies muss gegebenenfalls durch den Anwender über die Kommandozeile 
%
\begin{quoting}
\Path{initexmf -{}-edit-config-file updmap}
\end{quoting}
%
erfolgen. In der sich öffnenden Datei sollte sich der Eintrag 
%
\begin{quoting}
\Path{Map~tudscr.map}
\end{quoting}
%
befinden. Ist dies nicht der Fall, muss diese Zeile manuell eingetragen und die 
Datei anschließend gespeichert werden. Danach ist in der Kommandozeile noch 
folgenden Aufruf auszuführen:
%
\begin{quoting}
\Path{initexmf~-{}-mkmaps}
\end{quoting}


\minisec{Anmerkung zu \Logo{TeXLive} und \Logo{MacTeX}}
%
Sollte keine Vollinstallation von \TeXLive durchgeführt worden sein, müssen die 
zuvor genannten, erforderlichen Pakete und Skripte zur Schriftinstallation über 
den \TeXLive"=Paketmanager manuell hinzugefügt werden.

Sind nach einem fehlerfreien Durchlauf des Installationsskriptes die Schriften 
dennoch nicht verfügbar, so muss die Synchronisierung aller Schriftdateien 
angestoßen werden. Daran anschließend müssen die Map"~Datei und die 
dazugehörigen Schriftdateien registriert werden. Die hierfür notwendigen 
Aufrufe lauten:
%
\begin{quoting}
\Path{updmap-sys -{}-syncwithtrees}\newline
\Path{updmap-sys -{}-enable Map=tudscr.map}\newline
\Path{updmap-sys -{}-force}
\end{quoting}
%
Sind die Schriften danach immer noch nicht verfügbar, so wurden bestimmt schon 
weitere Schriften auf dem System \emph{lokal} installiert. In diesem Fall 
sollte der Vorgang nochmals für eine lokale Schriftinstallation mit 
%
\begin{quoting}
\Path{updmap -{}-syncwithtrees}\newline
\Path{updmap -{}-enable Map=tudscr.map}\newline
\Path{updmap -{}-force}
\end{quoting}
%
ausgeführt werden. 

\Attention{%
  Dadurch wird allerdings der Befehl \Path{updmap-sys} von nun an wirkungslos.
}
Nach einer systemweiten Installation neuer Schriften~-- beispielsweise bei der 
Aktualisierung der \DistributionGeneral~-- müssen diese über den manuellen 
Aufruf von \Path{updmap} zukünftig durch den Anwender lokal bei 
\TeXLive respektive \MacTeX registriert werden.



\subsection{%
  Installation der Type1-Schriften unter Windows%
  \label{sec:install:win}%
}
%
Zur Installation der Schriften des \CDs für das \TUDScript-Bundle ist das 
Archiv \GitHubDownload*<fonts>{TUD-Script_fonts_Windows.zip} vorgesehen. Dieses 
ist sowohl für \TeXLive als auch \MiKTeX nutzbar und enthält~-- bis auf die 
jeweiligen Schriftarchive selbst~-- alle benötigten Dateien. Diese sollten nach 
dem Entpacken des Archivs in das gleiche Verzeichnis kopiert werden. Vor der 
Verwendung des Skriptes \File*{tudscr_fonts_install.bat} ist sicherzustellen, 
dass sich \emph{alle} der folgenden Dateien im selben Verzeichnis befinden:
%
\settowidth\tudscrdim{\File*{tudscr_fonts_install.zip}~}%
\begin{description}[labelwidth=\tudscrdim,labelsep=.5em]
  \item[\File*{tudscr_fonts_install.bat}]Installationsskript
  \item[\File*{Univers_PS.zip}]Archiv mit Schriftdateien für \Univers
  \item[\File*{DIN_Bd_PS.zip}]Archiv mit Schriftdateien für \DIN
  \item[\File*{tudscr_fonts_install.zip}]Archiv mit Metriken für die
    Schriftinstallation via \Package*{fontinst}
\end{description}
%
Beim Ausführen des Installationsskriptes werden alle Schriften standardmäßig in 
ein lokales Nutzerverzeichnis installiert. Wird das Skript über das Kontextmenü 
mit Administratorrechten ausgeführt, erfolgt die Installation in einem Pfad, 
der \emph{für alle Nutzer} gültig und lesbar ist.



\subsection{%
  Installation der Type1-Schriften unter Linux und OS~X%
  \label{sec:install:unix}%
}
%
Zur Installation der Schriften des \CDs für das \TUDScript-Bundle ist das 
Archiv \GitHubDownload*<fonts>{TUD-Script_fonts_Unix.zip} vorgesehen. Dieses 
ist sowohl für \TeXLive als auch \MacTeX nutzbar und enthält~-- bis auf die 
jeweiligen Schriftdateien selbst~-- alle benötigten Dateien. Diese sollten nach 
dem Entpacken des Archivs in das gleiche Verzeichnis kopiert werden. Vor der 
Verwendung des Skriptes \File*{tudscr_fonts_install.sh} ist sicherzustellen, 
dass sich \emph{alle} der folgenden Dateien im selben Verzeichnis befinden:
%
\settowidth\tudscrdim{\File*{tudscr_fonts_install.zip}~}%
\begin{description}[labelwidth=\tudscrdim,labelsep=.5em]
  \item[\File*{tudscr_fonts_install.sh}]Installationsskript
    (Terminal: \Path{bash\:tudscr_fonts_install.sh})
  \item[\File*{Univers_PS.zip}]Archiv mit Schriftdateien für \Univers
  \item[\File*{DIN_Bd_PS.zip}]Archiv mit Schriftdateien für \DIN
  \item[\File*{tudscr_fonts_install.zip}]Archiv mit Metriken für die
    Schriftinstallation via \Package*{fontinst}
\end{description}
%
\minisec{Anmerkung zu Linux und OS~X}
%
\Attention{%
  Nach dem Entpacken eines Release-Archivs im passenden Pfad\footnote{%
    beispielsweise \Path{cd~"\$HOME/Downloads/\PName{Unterordner}"}%
  } \textbf{muss das Skript zwingend} mit \Path{bash~\PName{Skript}.sh} im 
  Terminal in diesem Pfad mit den benötigten Dateien aufgerufen werden.
}
Dabei werden alle Schriften standardmäßig in das lokale Nutzerverzeichnis 
(\Path{\$TEXMFHOME}) installiert. Wird das Skript mit \Path{sudo} verwendet, 
erfolgt die Installation \emph{für alle Nutzer} in den lokalen Systempfad 
(\Path{\$TEXMFLOCAL}).

Es ist unbedingt darauf zu Achten, das beim Ausführen des Skriptes das Terminal 
im richtigen Verzeichnis aktiv ist. Bei den meisten unixoiden Betriebssystemen 
ist es problemlos möglich, das Terminal aus der Benutzeroberfläche heraus über 
das Kontextmenü im gewünschten Pfad zu öffnen. Geht dies nicht, so muss nach 
dem Öffnen des Terminals mit dem Befehl \Path{cd} erst zum entsprechenden 
Pfad~-- exemplarisch \Path{cd~"\$HOME/Downloads/\PName{Unterordner}"}~-- 
navigiert werden. Ein beispielhafter Aufruf im Terminal könnte also lauten:
%
\begin{quoting}
\Path{cd~"\$HOME/Downloads/TUD-Script_fonts_Unix"{}\,\OPValue{ENTER}}\newline
\Path{bash tudscr_fonts_install.sh\,\OPValue{ENTER}}
\end{quoting}



\subsection{Probleme bei der Installation der Type1-Schriften}
%
Wird Windows verwendet, kann es unter Umständen vorkommen, dass notwendige 
Befehlsaufrufe für das Installationsskript nicht ausgeführt werden können. In 
diesem Fall ist der Pfad zu den benötigten Dateien, welche normalerweise unter 
\Path{\%SystemRoot\%\textbackslash System32} zu finden sind, nicht in der 
Umgebungsvariable \Path{PATH} enthalten. Einen Hinweis zur Problemlösung ist 
in diesem \Forum[Beitrag im \TUDForum]<359> zu finden.

Treten bei der Installation wider Erwarten Probleme auf, so ist zur Lösung eine 
Logdatei zu erstellen. Hierfür sollte unter \textbf{Windows} das Skript, 
welches Probleme verursacht, \emph{nicht} aus der Kommandozeile oder dem 
Explorer heraus sondern über \emph{Windows PowerShell} ausgeführt werden. 
Hierfür ist die Eingabe von \enquote{PowerShell} im Startmenü von Windows mit 
einem nachfolgenden Öffnen mittels \OPValue{ENTER}"~Taste ausreichend. 
Danach muss mit \Path{cd} zum Ordner des Skriptes navigiert und dieses mit 
\Path{.\textbackslash\PName{Skript}.bat|Tee-Object -file \PName{Skript}.log} 
ausgeführt werden. Ein Aufruf aus der PowerShell"~Konsole könnte lauten:
%
\begin{quoting}[rightmargin=0pt]
\Path{%
  cd~"\$env:USERPROFILE\textbackslash{}Downloads\textbackslash{}%
  TUD-Script_fonts_Windows"{}\,\OPValue{ENTER}%
}\newline%
\Path{%
  .\textbackslash{}tudscr_fonts_install.bat%
  |Tee-Object -file fonts_install.log\,\OPValue{ENTER}%
}%
\end{quoting}
%
Für \textbf{unixartige Systeme} ist der Aufruf 
\Path{bash \PName{Skript}.sh > \PName{Skript}.log} aus dem Terminal heraus zu 
verwenden. Ein exemplarische Verwendung könnte lauten:
%
\begin{quoting}
\Path{cd~"\$HOME/Downloads/TUD-Script_fonts_Unix"{}\,\OPValue{ENTER}}\newline
\Path{bash tudscr_fonts_install.sh > fonts_install.log\,\OPValue{ENTER}}%
\end{quoting}
%
Die so erstellte Logdatei kann \emph{mit einer kurzen Fehlerbeschreibung} 
entweder im \Forum gepostet oder per E"~Mail an \mailto{\TUDScriptContact}
gesendet werden.


\subsection{%
  Installationshinweise für portable Installationen%
  \label{sec:install:portable}%
}
%
Prinzipiell ist die Installation der Type1-Schriften des \CDs bei der Nutzung 
von \TeXLive' respektive \MiKTeX' äquivalent zur nicht"~portablen Variante, 
welche in \autoref{sec:install:fonts} beschrieben wird. Alle dort gegebenen 
Hinweise sollten sorgfältig berücksichtigt werden. Der durch das jeweilige 
Installationsskript voreingestellte Installationspfad sollte für gewöhnlich 
nicht geändert werden. Geschieht dies dennoch, so sollte dieser sich 
logischerweise auf dem externen Speichermedium 
\Path{\PName{Laufwerksbuchstabe}:\textbackslash} befinden.

\minisec{\TeXLive'}
%
Das folgende Vorgehen wurde mit Windows getestet. Empfehlungen für die portable 
Installation für unixoide Betriebssysteme können an \mailto{\TUDScriptContact} 
gesendet werden.
\begin{enumerate}
\item Installation von \TeXLive' in 
  \Path{\PName{Laufwerksbuchstabe}:\textbackslash LaTeX\textbackslash texlive}
\item Die Datei \Path{tl-tray-menu.exe} im Installationspfad öffnen
\item Das Kontextmenü von \TeXLive' mit einem Rechtsklick auf das entsprechende 
  Symbol im Infobereich der Taskleiste öffnen und entweder über die grafische 
  Oberfläche (\emph{Package~Manager}) oder die Kommandozeile 
  (\emph{Command~Prompt}) ein Update durchführen
\item Über das Kontextmenü die Kommandozeile ausführen und in dieser das Skript 
  für die Installation der Schriften \File*{tudscr_fonts_install.bat} 
  starten. Dabei gegebenenfalls zuvor in den Pfad des Skriptes 
  wechseln~-- exemplarisch:
  \begin{quoting}[leftmargin=1.5em,rightmargin=0pt]
  \Path{%
    cd~/d~\%USERPROFILE\%\textbackslash{}Downloads%
    \textbackslash{}TUD-Script_fonts_Windows\,\OPValue{ENTER}
  }\newline%
  \Path{tudscr_fonts_install.bat}\,\OPValue{ENTER}
  \end{quoting}
  Unter Umständen meldet das Skript fehlende Pakete. Dieses müssen über durch 
  den Anwender über den \emph{\Logo{TeXLive}~Manager} installiert werden.
  \Attention{%
    Ein Ausführen des Skriptes ohne die über \TeXLive' geöffnete Kommandozeile 
    führt zu Fehlern.
  }%
\end{enumerate}

\minisec{\MiKTeX'}
%
\begin{enumerate}
\item Installation von \MiKTeX' in 
  \Path{\PName{Laufwerksbuchstabe}:\textbackslash LaTeX\textbackslash MiKTeX}%
  \footnote{%
    Der Pfad darf \emph{nicht} auf der obersten Verzeichnisebene 
    \Path{\PName{Laufwerksbuchstabe}:\textbackslash} liegen.
  }%
\item Die Datei \Path{miktex-portable.cmd} im Installationspfad öffnen
\item Das Kontextmenü von \MiKTeX' mit einem Rechtsklick auf das entsprechende 
  Symbol im Infobereich der Taskleiste öffnen und ein Update durchführen
\item Über das Kontextmenü die Kommandozeile ausführen und in dieser das Skript 
  für die Installation der Schriften \File*{tudscr_fonts_install.bat} starten.
  Dabei gegebenenfalls zuvor in den Pfad des Skriptes wechseln~-- exemplarisch:
  \begin{quoting}[leftmargin=1.5em,rightmargin=0pt]
  \Path{%
    cd~/d~\%USERPROFILE\%\textbackslash{}Downloads%
    \textbackslash{}TUD-Script_fonts_Windows\,\OPValue{ENTER}
  }\newline
  \Path{tudscr_fonts_install.bat}\,\OPValue{ENTER}
  \end{quoting}
  Bei diesem Schritt werden möglicherweise die Pakete \Package*{fontinst}, 
  \Package*{cmbright} und \Package*{iwona} automatisch nachinstalliert.
  \Attention{%
    Ein Ausführen des Skriptes ohne die über \MiKTeX' geöffnete Kommandozeile 
    führt zu Fehlern.
  }%
\item Bei der erstmaligen Verwendung von \TUDScript werden alle benötigten 
  Pakete von \MiKTeX' installiert, falls die automatische Nachinstallation 
  aktiviert ist und diese noch nicht sind. Dies betrifft die Bundle 
  \Bundle{tudscr} und \Bundle{koma-script} als auch die Pakete 
  \Package||{etoolbox}, \Package||{xpatch}, \Package||{trimspaces} und
  \Package||{xcolor} sowie \Package*{mptopdf}.
\end{enumerate}
\index{Installation!Schriftinstallation|)}%
\index{Installation|)}%
