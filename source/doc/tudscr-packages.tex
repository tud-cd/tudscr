\chapter{%
  Benötigte, unterstützte und empfehlenswerte Pakete%
  \label{sec:packages}%
  \index{Pakete|?(}%
}
\section{%
  Notwendige und ergänzende Pakete für \TUDScript%
  \label{sec:packages:needed}%
}

Die im Folgenden aufgezählten Pakete werden von den \TUDScript-Klassen zwingend 
benötigt und spätestens zum Ende der Präambel geladen. Möchten Sie eines der 
Pakete selber nutzen, kann dieses wie gewohnt mit
\Macro{usepackage|\OPName{Paketoptionen}\MPName{Paket}} eingebunden werden. 
Alternativ lassen sich gewünschte Optionen~-- wie in \autoref{sec:tips:options} 
beschrieben~-- einfach bereits \emph{vor} dem Laden der Dokumentklasse an das 
entsprechende Paket übergeben, falls bei dessen nachfolgender Beschreibung 
nichts anderweitiges angegeben ist.

\ToDo{nur koma-script und nicht Pakete aufführen... außerhalb DeclarePackages?}
\ToDo{Link zu Bundle, wenn kein Label für Paket/Klasse?}
\begin{DeclarePackages}
\itempkg{}(\Bundle{koma-script})
  <
    \Class*{koma-class},\Class{scrbook},\Class{scrreprt},\Class{scrartcl},
    \Package{typearea},\Package{scrlayer-scrpage},\Package{scrletter},
    \Package{scrbase},\Package{scrextend}
  >
  \index{KOMA-Script@\KOMAScript|!}%
  Die zentrale Grundlage für \TUDScript bilden~-- zusammenfassend in diesem 
  Dokument gegebenenfalls als \Class*{koma-class} bezeichnet~-- die drei 
  Hauptklassen \Class{scrbook}, \Class{scrreprt} sowie \Class{scrartcl} aus dem 
  \scrguide[\KOMAScript-Bundle]. Das Paket \Package{scrbase} erlaubt die 
  Definition von Optionen oder Schlüsseln, die sich auch noch nach dem Laden 
  einer Klasse oder eines Paketes aus dem \TUDScript-Bundle mit den Befehlen 
  \Macro{TUDoption} und \Macro{TUDoptions} ändern lassen. Weiterhin werden die 
  \PageStyle{tudheadings}"=Seitenstile der \TUDScript-Klassen mithilfe des 
  Paketes \Package{scrlayer-scrpage} bereitgestellt. Wenn dieses nicht~-- mit 
  beliebigen Optionen~-- durch den Anwender geladen wird, erfolgt dies 
  automatisch am Ende der Präambel. Für die Festlegung des Satzspiegels wird
  das Paket \Package{typearea} genutzt.
\itempkg{opensans,mathastext,iwona}[v2.06]
  \index{Schriftart|!}%
  Das Paket \Package{opensans} stellt die Schriftfamilie \OpenSans sowohl für 
  den Fließtext als auch den mathematischen Satz zur Verfügung. Es enthält alle 
  nötigen Schriftschnitte sowohl im Type1- als auch im OpenType-Format. Da die 
  Schriftfamilie in der aktuellen Version keine mathematischen Glyphen 
  bereitstellt, werden die Pakete \Package{mathastext} und \Package{iwona} 
  zusätzlich genutzt, um zumindest einen halbwegs erträglichen mathematischen 
  Satz mit \OpenSans zu ermöglichen.
%  Werden dabei zusätzliche Symbole benötigt, wird empfohlen, auf das Paket 
%  \Package{amssymb} zu verzichten und anstelle dessen \Package{mdsymbol} zu 
%  laden.
  \ToDo{roboto-mono, hinweis zu mdsymbol raus}[v2.07]
  \ToDo{mweights raus, Text in Hinweise verschieben?!}[v2.07]
\itempkg{mweights}
  \index{Schriftstärke}%
  In \Lettering{LaTeX} existieren die Schriftfamilien für Serifenschriften 
  (\Macro{rmfamily}), serifenlose Schriften (\Macro{sffamily}) sowie die 
  Schreibmaschinenschriften (\Macro{ttfamily}). Deren Schriftstärke wird für 
  gewöhnlich mit den Befehlen \Macro{mddefault} und \Macro{bfdefault} 
  einheitlich festgelegt. Bei der Verwendung unterschiedlicher Schriftpakete 
  kann es unter Umständen zu Problemen bei den Schriftstärken kommen. Diese 
  Paket erlaubt die individuelle Definition der Schriftstärke für jede der 
  drei Schriftfamilien.
\itempkg{geometry}
  \index{Satzspiegel}%
  Das Paket wird zum Festlegen der Seitenränder respektive des Satzspiegels 
  verwendet.
  \Attention{%
    Ein Weiterreichen zusätzlicher Optionen an das Paket wird dringlich nicht 
    empfohlen.
  }
\itempkg{graphicx}
  \index{Abbildungen}%
  Zum Einbinden des Logos der \TnUD im Kopf sowie aller weitere Abbildungen und 
  Logos wird \Macro{includegraphics} genutzt.
\itempkg{xcolor}
  \index{Farben}%
  Damit werden die Farben des \CDs zur Verwendung im Dokument definiert. 
  Genaueres ist bei der Beschreibung von \Package'ref'{tudscrcolor} zu finden. 
\itempkg{iftex,etoolbox,xpatch,letltxmacro}
  Diese Pakete stellen viele Funktionen zum Testen und zur Ablaufkontrolle 
  bereit. Weiterhin wird das Manipulieren vorhandener Makros ermöglicht.
  \ToDo{kvsetkeys, environ, nicht mehr verwenden, Doku raus!}[v2.07]
  \ToDo{text von environ to xparse?}[v2.07]
\itempkg{kvsetkeys}
  Das von \Package{scrbase} geladene Paket \Package{keyval} macht das 
  Definieren von Klassen- und Paketoptionen sowie Parametern nach dem 
  Schlüssel"=Wert"=Prinzip möglich. Mit diesem Paket kann das Verhalten für 
  unbekannte Schlüssel festgelegt werden.
\itempkg{environ}
  \index{Befehlsdeklaration}%
  Es wird eine verbesserte Deklaration von Umgebungen ermöglicht, bei der auch 
  beim Abschluss der Umgebung auf die übergebenen Parameter zugegriffen werden 
  kann. 
\itempkg{trimspaces}
  Bei mehreren Eingabefeldern für den Anwender werden die Argumente mithilfe 
  dieses Paketes um eventuell angegebene, unnötige Leerzeichen befreit.
\end{DeclarePackages}


\minisec{%
  Durch \TUDScript direkt unterstütze Pakete%
  \index{Kompatibilität!Pakete}%
}

Einige der nachfolgend beschriebenen Pakete werden durch \TUDScript direkt 
unterstützt und erweitern dessen Funktionalität. Dies sind namentlich 
\Package{hyperref}, \Package{multicol}, \Package{quoting}, \Package{ragged2e} 
und \Package{crop} sowie \Package{isodate} respektive \Package{datetime2}. 
Weitere Informationen dazu ist den nachfolgenden Beschreibung des jeweiligen 
Paketes zu entnehmen.



\section{%
  Empfehlenswerte Pakete%
  \label{sec:packages:recommended}%
}

In diesem \autorefname wird eine Vielzahl an Paketen kurz vorgestellt, welche 
sich für mich persönlich bei der Arbeit mit \Lettering{LaTeX} bewährt haben. 
Einige davon werden außerdem im Tutorial \Tutorial{treatise} in ihrer Anwendung 
beschrieben. Für detaillierte Informationen sollte deren Dokumentation
(Kommandozeile/Terminal:~\Path{texdoc\,\PName{Paketname}}) zu Rate gezogen 
werden.



\subsection{Pakete zur Verwendung in jedem Dokument}

Die hier vorgestellten Pakete gehören meiner Meinung nach in die Präambel eines 
jeden Dokumentes. Die Dokumentsprache sollte in jedem Fall mit \Package{babel} 
oder \Package{polyglossia} definiert werden~-- auch wenn dies Englisch ist. Für 
deutschsprachige Dokumente ist für eine annehmbare Worttrennung beim Einsatz 
von \Format{pdfLaTeX} das Paket \Package{hyphsubst} unbedingt zu verwenden.

\begin{DeclarePackages}
\itempkg{fontenc,fontspec}[v2.02]
  \index{Zeichensatzkodierung}%
  \ChangedAt*{%
    v2.02:OpenType-Schriften mit Paket \Package{fontspec} verwendbar;%
    v2.04:Unterstützung des Paketes \Package{fontspec} verbessert;%
  }%
  Die Zeichensatzkodierung des Ausgabefonts sollte immer festgelegt werden. Für 
  \Format{pdfLaTeX} ist die Ausgabe als 7"~bit"~kodierte Schrift in der 
  Voreinstellung gewählt, was unter anderem dazu führt, dass keine echten
  Umlaute im erzeugten PDF"~Dokument verwendet werden. Um auf 8"~bit"~Schriften
  zu schalten, ist \Macro*{usepackage|\OPValue{T1}\MPValue{fontenc}} zu nutzen.
  
  Für die Unicode"=Textsatzsysteme \Format{LuaLaTeX} oder \Format{XeLaTeX} 
  sollte stattdessen das Paket \Package{fontspec} verwendet werden. Damit 
  können Systemschriften im OpenType-Format und einer beliebigen 
  Zeichensatzkodierung eingebunden werden, womit sich die Auswahl der 
  verwendbaren Schriften stark erweitert. Das Paket wird durch \TUDScript 
  unterstützt.
\itempkg{babel,polyglossia}
  \index{Sprachunterstützung}%
  \index{Bezeichner}%
  Mit dem Paket \Package{babel} erfolgt die Einstellung der im Dokument 
  verwendeten Sprache(n). Bei mehreren angegebenen Sprachen ist die zuletzt 
  geladene die Hauptsprache des Dokumentes. Die gewünschten Sprachen sollten 
  nicht als Paketoption sondern als Klassenoption und gesetzt werden, damit 
  auch andere Pakete auf die Spracheinstellungen zugreifen können. Für 
  deutschsprachige Dokumente sind die beiden Optionen \Option*{ngerman} für die 
  neue respektive \Option*{german} für die alte deutsche Rechtschreibung zu 
  verwenden. 
  
  Mit dem Laden von \Package{babel} und der dazugehörigen Sprachen werden 
  sowohl die Trennungsmuster als auch die sprachabhängigen Bezeichner angepasst.
  Von einer Verwendung der obsoleten Pakete \Package*{german} beziehungsweise 
  \Package*{ngerman} anstelle von \Package{babel} wird abgeraten. Für 
  \Format{LuaLaTeX} und \Format{XeLaTeX} kann das Paket \Package{polyglossia} 
  genutzt werden.
\itempkg{microtype}
  \index{Typografie}%
  Dieses Paket sorgt für optischen Randausgleich (protrusion,~margin~kerning) 
  und das Nivellieren der Wortzwischenräume (font~expansion) im Dokument.
  \ToDo{Hinweis zu Optionen (final, spacing, kerning, tracking?)}
\itempkg{hyphsubst,dehyph-exptl}
  \index{Worttrennung|!}%
  Die möglichen Trennstellen von Wörtern wird von \Lettering{LaTeX} mithilfe 
  eines Algorithmus berechnet. Dieser wird für deutschsprachige Texte mit dem 
  Paket \Package{hyphsubst} entscheidend verbessert. \Format{LuaLaTeX} und 
  \Format{XeLaTeX} nutzen diese besseren Trennungsmuster automatisch, für 
  \Format{pdfLaTeX} müssen diese mit folgendermaßen eingebunden werden:
\begin{quoting}[rightmargin=0pt]
\begin{Code}
\usepackage[ngerman=ngerman-x-latest]{hyphsubst}
\end{Code}
\end{quoting}
  In \autoref{sec:tips:hyphenation} wird genauer auf das Zusammenspiel von 
  \Package{hyphsubst} und \Package{babel} sowie \Package{fontenc} eingegangen, 
  ein Blick dahin wird dringend empfohlen. Zusätzlich werden dort weitere 
  Hinweise für eine verbesserte Worttrennung gegeben.
\end{DeclarePackages}


\minisec{Pakete zur situativen Verwendung}

Die nachfolgenden Pakete sollten nicht zwangsweise in jedem Dokument geladen 
werden sondern nur, falls dies auch tatsächlich notwendig ist. Zur besseren 
Übersicht wurde versucht, diese thematisch passend zu gruppieren. Daraus lässt 
sich keinerlei Wertung bezüglich ihrer Nützlichkeit oder meiner persönlichen 
Wertschätzung ableiten.



\subsection{%
  Typografie und Layout%
  \index{Typografie}%
}

Neben dem zuvor beschriebenem Paket \Package{microtype}, welches verantwortlich 
für mikrotypografische Feinheiten ist, existieren weitere Pakete, die vorrangig 
die Makrotypografie adressieren.

\begin{DeclarePackages}[Typografie]
\itempkg{setspace}
  \index{Zeilenabstand|?}%
  Die Vergrößerung des Zeilenabstandes wird:
  \begin{enumerate}[itemindent=0pt,labelwidth=*,labelsep=1em,label=\Roman*.]
  \item viel zu häufig und völlig unnötig gefordert und
  \item schließlich auch noch zu groß gewählt.
  \end{enumerate}
  Die Forderung nach Erhöhung des Zeilenabstandes~-- eigentlich Vergrößerung 
  des Zeilendurchschusses~-- kommt aus den Zeiten der Textverarbeitung mit der 
  Schreibmaschine. Ein einzeiliger Zeilenabstand bedeutet, dass die 
  Unterlängen der oberen Zeile genau auf der Höhe der Oberlängen der folgenden 
  Zeile liegen und somit kein Zeilendurchschuss (kompresser~Satz) vorhanden 
  ist. Ein anderthalbzeiliger Zeilenabstand erzielte hier somit einen halbwegs 
  akzeptablen Satz. Eine Erhöhung des Zeilendurchschusses bei der Verwendung 
  von \Lettering{LaTeX} ist an und für sich nicht notwendig. Sinnvoll ist dies 
  nur, wenn im Fließtext serifenlose Schriften zum Einsatz kommen, um die damit 
  verbundene schlechte Lesbarkeit etwas zu verbessern.
  
  Ist die Erhöhung des Zeilendurchschusses wirklich notwendig, sollte das Paket 
  \Package{setspace} genutzt werden. Dieses stellt den Befehl 
  \Macro{setstretch|\MPName{Faktor}} zur Verfügung, mit dem der Zeilenabstand 
  respektive Zeilendurchschuss angepasst werden kann. Der Wert des Faktors ist 
  standardmäßig auf~1 gestellt und sollte maximal bis~1.25 vergrößert werden. 
  Der Befehl \Macro{onehalfspacing} aus diesem Paket setzt diesen Wert auf eben 
  genau~1.25. Allerdings ist hier anzumerken, dass die Vergrößerung des 
  Zeilenabstandes~-- so wie ich es mir angelesen habe~-- aus der Sicht eines 
  Typographen keine Spielerei ist sondern vielmehr allein der Lesbarkeit des 
  Textes dient und möglichst gering ausfallen sollte.
  
  Ziel ist es, beim Lesen nach dem Beenden einer Zeile das Auffinden der neuen 
  Zeile zu vereinfachen. Bei Serifen ist dies durch die Betonung der Grundlinie 
  sehr gut möglich. Bei serifenlosen Schriften~-- wie der im \TUDCD verwendeten 
  \OpenSans~-- ist dies schwieriger und ein erweiterter Abstand der Zeilen kann 
  hierbei hilfreich sein. Jedoch sollte nicht nach dem Motto 
  \enquote{viel hilft viel} verfahren werden. Für dieses Dokument wurde 
  \Macro{setstretch|\MPValue{\baselinestretch}} für den Zeilenabstand gewählt. 
  Weitere Tipps sind in \autoref{sec:tips:headings} und 
  \autoref{sec:tips:headline} zu finden.
\itempkg{multicol}
  \index{Satzspiegel!mehrspaltig|!}%
  \index{Satzspiegel!zweispaltig|!}%
  Hiermit kann jeglicher beliebiger Inhalt in zwei oder mehr Spalten ausgegeben 
  werden, wobei~-- im Gegensatz zum normalen zweispaltigen Satz über die
  \KOMAScript-Option \Option{twocolumn}~-- für beliebig viele Spalten ein 
  automatischer Ausgleich dieser erfolgt. Es wird innerhalb der Umgebungen 
  \Environment{abstract} und \Environment{tudpage} unterstützt, sofern es 
  geladen wird.
\itempkg{isodate,datetime2}
  \index{Datum|?}%
  \ToDo{Vorrang für datetime2?!}
  Mit \Macro{printdate|\MPName{Datum}} formatiert das Paket \Package{isodate} 
  die Ausgabe eines Datums automatisch in ein spezifiziertes Format. Die 
  Datumsangabe kann dabei im deutschen, englischen oder ISO"~Format erfolgen. 
  Alternativ kann auch das Paket \Package{datetime2} genutzt werden, welches 
  für \Macro{DTMDate|\MPName{ISO-Datum}} die Angabe im ISO"~Format erfordert. 
  Wird eines der Pakete geladen, werden alle Datumsfelder, welche durch die 
  \TUDScript-Klassen definiert wurden,%
  \footnote{%
    \Macro{date}, \Macro{dateofbirth} und \Macro{defensedate} sowie aus 
    \Package{tudscrsupervisor} \Macro{duedate} und \Macro{issuedate}%
  }
  im durch das jeweilige Paket definierten Ausgabeformat ausgegeben.
\itempkg{quoting}
  \index{Zitate}%
  \Lettering{LaTeX} bietet von Haus aus \emph{zwei} verschiedene Umgebungen~-- 
  \Environment{quote} und \Environment{quotation}~-- für Zitate und ähnliches 
  an. Allerdings werden durch beide Umgebungen die \KOMAScript-Option 
  \Option{parskip=\PMisc} ignoriert. Mit der Umgebung \Environment{quoting} aus 
  dem gleichnamigen Paket lässt sich dieses Problem umgehen. Wird das Paket 
  geladen, wird diese innerhalb der \Environment{abstract}"~Umgebung verwendet.
\itempkg{csquotes}
  \index{Zitate}%
  Das Paket stellt unter anderem den Befehl \Macro{enquote|\MPName{Zitat}} zur 
  Verfügung, welcher Anführungszeichen in Abhängigkeit der gewählten Sprache 
  setzt. Zusätzlich werden weitere Kommandos und Optionen für die spezifischen 
  Anforderungen des Zitierens bei wissenschaftlichen Arbeiten angeboten. 
  Außerdem wird es durch \Package{biblatex} unterstützt und sollte zumindest 
  bei dessen Verwendung geladen werden.
\itempkg{selnolig}(\Application'CTAN:delig'{DeLig|})
  \index{Ligaturen}%
  \ToDo{\Package{ligtype}}[v2.07]
  Wird eine Schriftart mit Ligaturen verwendet, sollten diese für einen guten 
  Textsatz bei bestimmten Wörtern~-- insbesondere in deutschen Texten aufgrund 
  der vielen Komposita~-- aufgelöst werden.%
  \footnote{%
    Das sind \enquote{ff}, \enquote{fi}, \enquote{fl}, \enquote{ffi}, und 
    \enquote{ffl} bei den meisten \Lettering{LaTeX}"~Schriften.%
  }
  Mit \Format{LuaLaTeX} als Textsatzsystem kann das Paket \Package{selnolig} 
  für dieses Unterfangen verwendet werden, welches sowohl Muster für das 
  Auflösen von Ligaturen als auch Trennstellen für Wörter bereitstellt. 
  Das Java"~Programm \Application{DeLig} trennt anhand eines Wörterbuches 
  problematische Ligaturen alternativ dazu durch direktes Einfügen von 
  \PValue{"\textbar} in die \Path{.tex}"~Datei selbst, wofür das Paket 
  \Package{babel} verwendet werden muss.
\itempkg{ragged2e}
  \index{Worttrennung}%
  Das Paket verbessert den Flattersatz, indem für diesen die Worttrennung 
  aktiviert wird.
\itempkg{fnpct}[v2.05]
  \index{Fußnoten|?}%
  Diese Paket sorgt zum einen für das Einhalten der richtigen Reihenfolge von 
  Satzzeichen und Fußnoten und zum anderen wird das typografisch korrekte 
  Setzen mehrerer, nacheinander folgender Fußnoten unterstützt. 
\itempkg{noindentafter}[v2.02]
  Mit diesem Paket lassen sich automatische Absatzeinzüge für selbst zu 
  bestimmende Befehle und Umgebungen unterdrücken.
\itempkg{xspace,xpunctuate}
  \index{Befehlsdeklaration}%
  Mit \Package{xspace} kann bei der Definition eigener Makros der Befehl 
  \Macro{xspace} genutzt werden. Dieser setzt ein gegebenenfalls notwendiges 
  Leerzeichen automatisch. In \autoref{sec:tips:xspace} ist die Definition 
  eines solchen Befehls exemplarisch ausgeführt. Das Paket \Package{xpunctuate} 
  erweitert \Package{xspace} um die Beachtung von Interpunktionen.
\itempkg{ellipsis}
  In \Lettering{LaTeX} folgen den Befehlen für Auslassungspunkte (\Macro{dots} 
  und \Macro{textellipsis}) \emph{immer} ein Leerzeichen. Dies kann unter 
  Umständen unerwünscht sein. Mit dem Paket \Package{ellipsis} wird das 
  nachfolgende Leerzeichen~-- im Gegensatz zum Standardverhalten~-- nur 
  gesetzt, wenn ein Satzzeichen und kein Buchstabe folgt. Zusätzliche Hinweise 
  zur Verwendung sind in \autoref{sec:tips:dots} zu finden.
\end{DeclarePackages}



\subsection{%
  Rechtschreibung%
  \index{Rechtschreibung}%
}

Die meisten \Lettering{LaTeX}"=Entwicklungsumgebungen enthalten eine 
integrierte und/oder konfigurierbare Rechtschreibkontrolle. Dennoch gibt 
es einige wenige Pakete, welche sich diesem Thema widmen. Diese sind jedoch 
ausschließlich mit \Format{LuaLaTeX} nutzbar.

\begin{DeclarePackages}[Rechtschreibung]
\itempkg{lua-check-hyphen}
  \index{Worttrennung}%
  Hiermit lassen sich mit \Format{LuaLaTeX} Trennstellen am Zeilenende zur 
  Prüfung markieren. Zum Thema der \textit{korrekten Worttrennung} sei außerdem 
  auf \autoref{sec:tips:hyphenation} verwiesen.
\itempkg{spelling}
  Wird \Format{LuaLaTeX} als Textsatzsystem verwendet, wird mit diesem Paket 
  der reine Textanteil aus dem \Lettering{LaTeX}-Dokument extrahiert~-- wobei 
  Makros und aktive Zeichen entfernt werden~-- und in eine separate Textdatei 
  geschrieben. Anschließend kann diese Datei mit einer externen Software zur 
  Rechtschreibprüfung wie \Application{GNU~Aspell}, \Application{Hunspell} oder 
  \Application{LanguageTool} analysiert und falsch geschriebene Wörter im 
  PDF"~Dokument hervorgehoben werden.
\end{DeclarePackages}



\subsection{Schriften und Sonderzeichen}

\begin{DeclarePackages}
\itempkg{lmodern,cfr-lm}
  \index{Schriftart}%
  Soll mit den klassischen \Lettering{LaTeX}"=Standardschriften gearbeitet 
  werden, empfiehlt sich die Verwendung des Paketes \Package{lmodern}. Dieses 
  verbessert die Darstellung der Computer~Modern sowohl am Bildschirm als auch 
  beim finalen Druck. Das experimentelle Paket \Package{cfr-lm} liefert 
  zusätzliche Schriftschnitte für die Latin~Modern~Schriftfamilie.
\itempkg{newtxtext,newtxmath}
  \index{Schriftart}%
  \index{Mathematiksatz}%
  Es werden einige alternative Schriften sowohl für den Fließtext 
  (\textit{Times} und \textit{Helvetica}) als auch den Mathematikmodus 
  bereitgestellt.
\itempkg{libertine,libgreek}
  \index{Schriftart}%
  Das Paket stellt die Schriften Linux~Libertine und Linux~Biolinum zur 
  Verfügung. Um diese Schriftart auch für den Mathematikmodus verwenden zu 
  können, sollte zusätzlich das Paket \Package{newtxmath} mit 
  \Macro*{usepackage|\OPValue{libertine}\MPValue{newtxmath}} in der 
  Präambel eingebunden werden. Das Paket \Package{libgreek} enthält griechische 
  Lettern für Linux~Libertine.
\itempkg{relsize}
  \index{Schriftgröße}%
  Die Größe einer Textauszeichnung kann relativ zur aktuellen Schriftgröße 
  gesetzt werden.
\itempkg{textcomp}
  \index{Sonderzeichen}%
  Es werden zusätzliche Symbole und Sonderzeichen wie beispielsweise das 
  Promille- oder Eurozeichen sowie Pfeile für den Fließtext zur Verfügung 
  gestellt.
\end{DeclarePackages}

\index{Mathematiksatz}%
Wird für den mathematischen Satz \OpenSans genutzt (\Option{cdmath=true}), wird 
vordefiniert das Paket \Package{mathastext} geladen und es sollten keine 
zusätzlichen Einstellungen vorgenommen werden. Wird darauf verzichtet 
(\Option{cdmath=false}), können serifenlose Mathematikschriften mit folgenden 
Paketen aktiviert werden.

\begin{DeclarePackages}
\itempkg{sansmathfonts,sansmath}
  Sollten die normalen \Lettering{LaTeX}"~Schriften Computer~Modern verwendet 
  werden, lässt sich dieses Paket zum serifenlosen Setzen mathematischer 
  Ausdrücke nutzen. Ein alternatives Paket mit der gleichen Zielstellung ist 
  \Package{sansmath}
\itempkg{sfmath}
  Diese Paket verfolgt ein ähnliches Ziel, kann jedoch im Gegensatz zu 
  \Package{sansmath} nicht nur für Computer~Modern sondern mit der 
  entsprechenden Option auch für Latin~Modern, Helvetica und 
  Computer~Modern~Bright verwendet werden.
\end{DeclarePackages}



\subsection{%
  Mathematiksatz%
  \index{Mathematiksatz|(}%
}

Dies sind Pakete, die Umgebungen und Befehle für den Mathematiksatz sowie das 
Setzen von Einheiten und Zahlen im Allgemeinen anbieten.
\ToDo{Hinweise zu mathswap, icomma und ziffer raus?!}[v2.07]
\ToDo{Hinweis auf fixdif}[v2.07]

\begin{DeclarePackages}
\itempkg{mathtools,amsmath}
  Das De"~facto"~Standard"~Paket für einen erweiterten mathematischen Satz ist 
  \Package{amsmath}, welches verschiedene Befehle und Umgebungen für dieses 
  Unterfangen bereitstellt. Das Paket \Package{mathtools} für dessen Befehle 
  und Umgebungen kleinere Bugfixes.
\itempkg{bm}
  Das Paket bietet mit \Macro{bm} eine Alternative zu \Macro{boldsymbol} im 
  \href{http://tex.stackexchange.com/q/3238}{Mathematiksatz}. Sollten bei
  der Nutzung des Paketes Fehler auftreten, sei auf dessen Dokumentation und 
  insbesondere auf das Makro \Macro||{bmmax} verwiesen.
\itempkg{ionumbers}
  \index{Zifferngruppierung}%
  Die korrekte Formatierung von Zahlen ist manchmal ein Ärgernis bei der 
  Nutzung von \Lettering{LaTeX}. Insbesondere, wenn in einem deutschsprachigen 
  Dokument Daten im englischsprachigen Format verwendet werden, kommt es zu 
  Problemen. Mit diesem Paket können einfach Eingabe- und Ausgabeformat von 
  Zahlen definiert werden, um beispielsweise externe Datensätze einzulesen und 
  korrekt formatiert auszugeben. Für das identische Unterfangen wird im 
  \TUDScript-Bundle das Paket \Package{mathswap}~-- mit zugegebenermaßen 
  deutlich weniger Funktionalitäten~-- bereitgestellt.
\itempkg{icomma}
  Wird im Mathematikmodus nach dem Komma ein Leerzeichen gesetzt, wird dies 
  dementsprechnd ausgegeben. Beim Verfassen eines Dokumentes muss folglich 
  beachtet werden, ob es sich bei einem zu setzenden Komma um eine gewöhnliche 
  Interpunktion ($z=f(x,y)$)~-- wo ein gewisser Abstand nach dem Komma durchaus 
  gewünscht ist~-- oder ein Dezimaltrennzeichen ($t=1,\!2$) handelt.
\itempkg{ziffer}
  Für deutschsprachige Dokumente wird das Komma als Dezimaltrennzeichen 
  zwischen zwei Ziffern definiert. Folgt dem Komma keine Ziffer, wird 
  jederzeit der obligatorische Freiraum gesetzt, was meiner Meinung nach 
  besser als das Verhalten von \Package{icomma} ist.
\end{DeclarePackages}

Für das typografisch korrekte Setzen von Einheiten~-- ein halbes Leerzeichen 
zwischen Zahl und \emph{aufrecht} gesetzter Einheit~-- gibt es zwei gut 
nutzbare Pakete.

\begin{DeclarePackages}[Einheiten]
\itempkg{units}
  Dies ist ein einfaches und sehr zweckdienliches Paket zum Setzen von 
  Einheiten und für die meisten Anforderungen völlig ausreichend.
\itempkg{siunitx}
  Dieses Paket ist in seinem Umfang im Vergleich deutlich erweitert. Neben 
  Einheiten können zusätzlich auch Zahlen typografisch korrekt gesetzt werden. 
  Die Ausgabe lässt sich in vielerlei Hinsicht an individuelle Bedürfnisse 
  anpassen. Für deutschsprachige Dokumenten sollte die Lokalisierung angegeben 
  werden. Mehr dazu in \autoref{sec:tips:siunitx}.
\end{DeclarePackages}

Weitere Hinweise zur mathematischen Typografie werden in \autoref{sec:tut} 
gegeben.%
\index{Mathematiksatz|)}%



\subsection{%
  Listen%
  \index{Listen|?}%
}

\begin{DeclarePackages}[Listen|?]
\itempkg{enumitem}
  Das Paket \Package{enumitem} erweitert die standardmäßig eher rudimentären 
  Funktionalitäten der \Lettering{LaTeX}"=Umgebungen für Aufzählungen 
  \Environment{itemize}, \Environment{enumerate} und \Environment{description}.
  Durch die Bereitstellung optionaler Parameter im Schlüssel"=Wert"=Stil wird 
  die individuelle Anpassung dieser sowohl allgemein für alle als auch für 
  einzelne Auflistungen ermöglicht. Eine von mir sehr häufig genutzte Funktion 
  ist beispielsweise die Entfernung des zusätzlichen Abstand zwischen den 
  einzelnen Einträgen einer Liste mit \Macro{setlist|\MPValue{noitemsep}}.
\end{DeclarePackages}



\subsection{%
  Verzeichnisse aller Art%
  \index{Verzeichnisse|?}%
}

Neben dem Erstellen des eigentlichen Dokumentes sind für eine wissenschaftliche 
Arbeit meist auch allerhand Verzeichnisse gefordert. Fester Bestandteil ist 
dabei das Literaturverzeichnis, auch ein Abkürzungs- und Formelzeichen- 
beziehungsweise Symbolverzeichnis werden häufig gefordert. Gegebenenfalls wird 
auch noch ein Glossar benötigt. Hier werden die passenden Pakete vorgestellt. 
Sollen im Dokument komplette Quelltexte oder auch nur Auszüge daraus erscheinen 
und für diese auch gleich ein entsprechendes Verzeichnis generiert werden, so 
sei auf das Paket \Package'full'{listings} verwiesen.

\begin{DeclarePackages}[Verzeichnisse|?]
\itempkg{biblatex}
  \index{Literaturverzeichnis}%
  Das Paket kann als legitimer Nachfolger zu \Lettering{BibTeX} gesehen werden. 
  Ähnlich dazu bietet \Package{biblatex} die Möglichkeit, Literaturdatenbanken 
  einzubinden und verschiedene Stile der Referenzierung und Darstellung des 
  Literaturverzeichnisses auszuwählen. 
  
  Mit \Package{biblatex} ist die Anpassung eines bestimmten Stiles wesentlich 
  besser umsetzbar als mit \Lettering{BibTeX}. Wird \Application{biber} für die 
  Sortierung des Literaturverzeichnisses genutzt, ist die Verwendung einer 
  Literaturdatenbank mit UTF"~8"=Eingabekodierung problemlos möglich. In 
  Verbindung mit \Package{biblatex} wird die zusätzliche Nutzung des Paketes 
  \Package{csquotes} sehr empfohlen.
\itempkg{glossaries,nomencl}
  \index{Glossar}%
  \index{Abkürzungsverzeichnis}%
  \index{Formelzeichenverzeichnis}%
  \index{Symbolverzeichnis}%
  Dies ist ein sehr mächtiges Paket zum Erstellen eines Glossars sowie 
  Abkürzungs- und Symbolverzeichnisses. Die mannigfaltige Anzahl an Optionen 
  ist zu Beginn eventuell etwas abschreckend. Insbesondere wenn Verzeichnisse 
  für Abkürzungen \emph{und} Formelzeichen respektive Symbole notwendig sind, 
  sollte dieses Paket in Erwägung gezogen werden.
  
  Alternativ dazu kann für ein Symbolverzeichnis auch lediglich eine einfache 
  Auflistung mit dem Paket \Package{nomencl} erzeugt werden.
\itempkg{acro,acronym}
  \index{Abkürzungsverzeichnis}%
  Soll lediglich ein Abkürzungsverzeichnis erstellt werden, ist dieses Paket 
  die erste Wahl. Es stellt Befehle zur Definition von Abkürzungen sowie zu 
  deren Verwendung im Text und zur sortierten Ausgabe eines Verzeichnisses 
  bereit. Alternativ dazu kann das Paket \Package{acronym} verwendet werden. 
  Die Sortierung des Abkürzungsverzeichnisses muss hier allerdings manuell 
  durch den Anwender erfolgen.
\end{DeclarePackages}



\subsection{%
  Gleitobjekte%
  \index{Gleitobjekte|?}%
}

Es werden Pakete für die Beeinflussung von Aussehen, Beschriftung und 
Positionierung von Gleitobjekten vorgestellt. Unter \autoref{sec:tips:floats} 
sind außerdem Hinweise zur manuellen Manipulation der Gleitobjektplatzierung zu 
finden.

\ToDo[doc]{Hinweise/Erläuterungen \Package{float}, \Package{keyfloat}}
\ToDo[doc]{Hinweise/Erläuterungen zu \Package{widows-and-orphans}}
\ToDo[doc]{Hinweise/Erläuterungen zu \Package{lua-widow-control}}
\ToDo[doc]{Hinweise/Erläuterungen zu \Package{hvfloat}?}
%\documentclass[%
%  captions=tableabove
%]
%{scrreprt}
%\usepackage{caption}
%\usepackage{keyfloat}
%%\captionsetup{tableposition=above}
%%\captionsetup[table]{position=above}
%\begin{document}
%\begin{keytable}{c=keytable}
%\centering
%\fbox{this would be some content for a table}
%\end{keytable}
%\end{document}
\begin{DeclarePackages}[Gleitobjekte|?]
\itempkg{caption}
  Mit der Option \InlineDeclaration{\Option{captions=\PMisc}} bieten die 
  \KOMAScript-Klassen bereits einige Möglichkeiten zum Formatieren der 
  Beschriftungen für Gleitobjekte. Sollten diese nicht ausreichen, lässt sich 
  dieses Paket nutzen, wobei in diesem Fall auf mögliche Überschneidungen der 
  Einstellungen geachtet werden sollte.
\itempkg{subcaption}
  Diese Paket kann zum einfachen Setzen von Unterabbildungen oder "~tabellen 
  mit den entsprechenden Beschriftungen genutzt werden. Das häufig angeführte 
  Paket \Package*{subfig} ist keine gute Alternative hierzu, da es nicht mehr 
  gepflegt wird und es im Zusammenspiel mit anderen Paketen des Öfteren zu 
  Problemen kommt. Sollte der Funktionsumfang von \Package{subcaption} nicht 
  ausreichen, kann anstelle dessen das Paket \Package{floatrow} verwendet 
  werden, welches ähnliche Funktionalitäten wie \Package*{subfig} bereitstellt.
\itempkg{floatrow}
  Mit diesem Paket können global wirksame Einstellungen und Formatierungen für 
  \emph{alle} Gleitobjekte eines Dokumentes über die Paketoptionen oder mit 
  \Macro{floatsetup|\MPValue{\dots}} vorgenommen werden. So lässt sich unter 
  anderem die verwendete Schrift innerhalb der Umgebungen \Environment{figure} 
  und \Environment{table} mit \Macro{floatsetup|\MPValue{font=\dots}} 
  einstellen. Mit 
\begin{quoting}[rightmargin=0pt]
\begin{Code}
\floatsetup[figure]{capposition=bottom}
\floatsetup[table]{capposition=top}
\end{Code}
\end{quoting}
  lässt sich automatisch das typografisch korrekte Setzen von 
  Abbildungs\emph{unterschriften} sowie Tabellen\emph{überschriften} 
  erzwingen~-- unabhängig von der Position des Befehls zur Beschriftung 
  \Macro{caption} innerhalb der jeweiligen Gleitobjektumgebung. Wird das 
  Verhalten wie empfohlen eingestellt, sollte für eine gute vertikale 
  Platzierung der Tabellenüberschriften zusätzlich die \KOMAScript-Option 
  \Option{captions=tableheading} genutzt werden.
\itempkg{placeins}
  \index{Gleitobjekte!Platzierung}%
  Mit diesem Paket kann die Ausgabe von Gleitobjekten vor Kapiteln und wahlweise
  Abschnitten erzwungen werden.
\itempkg{flafter}
  \index{Gleitobjekte!Platzierung}%
  Dieses Paket erlaubt die frühestmögliche Platzierung von Gleitobjekten im 
  ausgegeben Dokument erst an der Stelle ihres Auftretens im Quelltext. Diese 
  werden dementsprechend nie vor ihrer Definition am Anfang der Seite 
  erscheinen.
\end{DeclarePackages}



\subsection{%
  Tabellen%
  \index{Tabellen|?}%
}

Für den Tabellensatz werden standardmäßig die Umgebungen \Environment{tabbing} 
und \Environment{tabular} respektive \Environment{tabular*} bereitgestellt, 
welche in ihrer Funktionalität für einen qualitativ hochwertigen Tabellensatz 
meist nicht ausreichen. Deshalb werden hier zusätzliche Pakete vorgestellt.
\ToDo{Hinweis auf \Package*{tabularray}}[v2.07]

\begin{DeclarePackages}[Tabellen|?]
\itempkg{array}
  Dieses Paket ermöglicht mit \Macro{newcolumntype} das Erstellen neuer 
  Spaltentypen und die erweiterte Definition von Tabellenspalten
  (\PValue{>\MPValue{\dots}}\PName{Spaltentyp}\PValue{<\MPValue{\dots}}), 
  wobei mithilfe sogenannter \enquote{Hooks} vor und nach Einträgen innerhalb 
  einer Spalte gezielt Anweisungen gesetzt werden können. Die Zeilenhöhe lässt 
  sich mit \Macro{extrarowheight} ändern. 
\itempkg{booktabs}
  Für einen guten Tabellensatz mit \Lettering{LaTeX} gibt es bereits zahlreiche 
  \href{http://userpage.fu-berlin.de/latex/Materialien/tabsatz.pdf}{Tipps} im 
  Internet zu finden. Zwei Regeln sollten dabei definitiv beachtet werden:
  \begin{enumerate}[itemindent=0pt,labelwidth=*,labelsep=1em,label=\Roman*.]
  \item keine vertikalen Linien
  \item keine doppelten Linien
  \end{enumerate}
  Das Paket \Package{booktabs} ist für den Satz von hochwertigen Tabellen eine 
  große Hilfe und stellt mit den Befehlen \Macro{toprule}, \Macro{bottomrule} 
  sowie \Macro{midrule} und \Macro{cmidrule} unterschiedliche horizontale 
  Linien bereit.
\itempkg{widetable}
  Mit der Umgebung \Environment{tabular*} kann eine Tabelle mit einer 
  definierten Breite gesetzt werden. Dieses Paket stellt die zusätzliche 
  Umgebung \Environment{widetable} zur Verfügung, die als Alternative genutzt 
  werden kann und eine symmetrische Tabelle erzeugt.
\itempkg{tabularx}
  Auch mit diesem Paket lässt sich die Gesamtbreite einer Tabelle festlegen. 
  Dafür wird der Spaltentyp \PValue{X} definiert, welcher beliebig innerhalb 
  des Argumentes \MPName{Spalten} bei der Umgebung 
  \InlineDeclaration{\Environment{tabularx|\MPName{Breite}\MPName{Spalten}}} 
  angegeben werden kann. Die \PValue{X}"~Spalten ähneln denen vom 
  Typ~\PValue{p}\MPName{Breite}, wobei die Breite dieser aus der gewünschten 
  Tabellengesamtbreite abzüglich des benötigten Platzes der gegebenenfalls 
  vorhandenen Standardspalten \PValue{lcr} etc. automatisch berechnet wird.
\itempkg{tabulary}
  Dies ist ein weiteres Paket zur automatischen Berechnung von Spaltenbreiten. 
  Der zur Verfügung stehende Platz~-- gewünschte Gesamtbreite abzüglich der 
  notwendigen Breite für Standardspalten~-- wird jedoch nicht wie von 
  \Package{tabularx} auf alle Spalten gleichmäßig verteilt sondern für die 
  Spaltentypen~\PValue{LCRJ} anhand ihres Zellinhaltes in der Umgebung 
  \InlineDeclaration{\Environment{tabulary|\MPName{Breite}\MPName{Spalten}}} 
  gewichtet vergeben.
\itempkg{longtable,xltabular,ltxtable,ltablex}[v2.06:\Package{xltabular}]
  Sollen mehrseitige Tabellen mit Seitenumbruch erstellt werden, ist das Paket
  \Package{longtable} das Mittel der ersten Wahl. Zum Setzen einer mehrseitigen 
  \Environment{tabularx}"~Tabelle ist das Paket \Package{xltabular} 
  empfehlenswert, außerdem verfolgen die Pakete \Package{ltablex} und 
  \Package{ltxtable} das gleiche Ziel. Alternativ dazu lässt sich auch 
  \Package{tabu} nutzen.
\itempkg{multirow}
  Es wird der Befehl \Macro{multirow} definiert, der das Zusammenfassen von 
  mehreren Zeilen in einer Spalte ermöglicht~-- ähnlich zum Makro 
  \Macro{multicolumn} für Spalten.
\itempkg{tabularborder}
  Bei Tabellen wird zwischen Spalten automatisch ein horizontaler Abstand 
  (\Length{tabcolsep}) gesetzt~-- besser gesagt jeweils vor und nach einer 
  Spalte. Dies geschieht auch \emph{vor} der ersten und \emph{nach} der letzten 
  Spalte. Dieser zusätzliche Platz an den äußeren Rändern kann störend wirken, 
  insbesondere wenn die Tabelle über die komplette Textbreite gesetzt wird. Mit 
  dem Paket \Package{tabularborder} kann dieser Platz automatisch entfernt 
  werden.
  
  Dies funktioniert allerdings nur mit der \Environment{tabular}"~Umgebung. 
  Die Umgebungen aus den Paketen \Package{tabularx}, \Package{tabulary} und 
  \Package{tabu} werden nicht unterstützt. Wie der Abstand manuell bei den 
  zugehörigen Tabellen entfernt werden kann, ist unter \autoref{sec:tips:table} 
  zu finden.
\itempkg{tabu}
  [
    v2.02:\Package{tabu} nur bedingt empfehlenswert;
    v2.06:\Package{tabu} in Version~v2.10 nur sehr eingeschränkt nutzbar;
  ]
  Dieses Paket versucht, viele der zuvor genannten Funktionalitäten zu 
  implementieren und weitere bereitzustellen. Dafür werden die Umgebungen 
  \Environment{tabu} und \Environment{longtabu} definiert. Es lässt sich 
  alternativ zum Paket \Package{tabularx} verwenden und ist insbesondere 
  anstelle des Paketes \Package{ltxtable} eine Möglichkeit zur Erstellung 
  umbrechbarer Tabellen.
  \ToDo[doc]{Hinweise/Erläuterungen zu \Package{tabularray}?}
  \ToDo[doc]{\Package{tabu} ganz raus?!}
  
  \Attention{%
    Durch Änderungen am \Lettering{LaTeXe}"~Kernel wurde das Paket leider 
    unbrauchbar. Zwar wurden die allernötigsten Anpassungen durch das 
    \GitHubRepo<tabu-fixed/tabu>[\Lettering{LaTeX3}"~Projektteam] 
    vorgenommen, damit das Paket in der aktuellen Version~v2.9 zumindest 
    rudimentär lauffähig ist, allerdings sind viele der Funktionalitäten nicht 
    mehr nutzbar. Weiterhin wären seit einigen Jahren mehrere Bugfixes 
    notwendig. Der originäre Autor hatte außerdem die Änderung der 
    \href{https://groups.google.com/d/topic/comp.text.tex/xRGJTC74uCI}{%
      Benutzerschnittstelle in einer zukünftigen Version%
    }
    angekündigt, anscheinend aber mittlerweile die Pflege komplett eingestellt. 
    Ein Großteil der ursprünglich sehr guten Funktionalitäten kann momentan 
    nicht mehr genutzt werden, weshalb es aktuell nicht empfehlenswert ist.
  }
\end{DeclarePackages}



\subsection{%
  Grafiken%
  \index{Grafiken|?}%
}

Abbildungen für wissenschaftliche Arbeiten sollten als Vektorgrafiken erstellt 
werden, um die Skalierbarkeit und hohe Druckqualität zu gewährleisten. 
Bestenfalls folgen diese auch dem Stil der dazugehörigen Arbeit.%
\footnote{%
  Für qualitativ hochwertige Dokumente sollten übernommene Abbildungen nicht 
  direkt kopiert oder gescannt sondern im gewünschten Zielformat neu erstellt 
  und mit einer Referenz auf die Quelle eingebunden werden.%
}
Für das Erstellen eigener Vektorgrafiken, welche sowohl das Layout als auch die 
\Lettering{LaTeX}"~Schriften des Hauptdokumentes nutzen, gibt es zwei mögliche 
Ansätze. Die Grafiken lassen sich entweder~-- ähnlich wie das Dokument 
selbst~-- \enquote{programmieren} oder mit einem Zeichenprogramm erstellen, 
welches die Ausgabe oder das Weiterreichen von Text an \Lettering{LaTeX} 
unterstützt. Für beide Varianten sollen hier die wichtigsten Pakete vorgestellt 
werden. Wie diese zu verwenden sind, ist den dazugehörigen Paketdokumentationen 
zu entnehmen. 

\begin{DeclarePackages}[Grafiken|?]
\itempkg{tikz}
  Dies ist ein sehr mächtiges Paket für das Programmieren von Vektorgrafiken 
  und sehr häufig~-- insbesondere bei Einsteigern~-- die erste Wahl bei der 
  Verwendung eines der Formate \Format{pdfLaTeX}, \Format{LuaLaTeX} oder 
  \Format{XeLaTeX}.
\itempkg{pstricks}
  Diese Paket stellt die zweite Variante zum Programmieren von Grafiken dar. 
  Da hiermit auf PostScript (direkt) zugegriffen werden kann, existieren 
  \emph{noch} mehr Möglichkeiten bei der Erstellung eigener Grafiken, wovon
  die bereitgestellten Befehle rege Gebrauch machen.
  
  Daraus resultiert allerdings der Nachteil, dass die mit \Package{pstricks} 
  erstellten Grafiken nicht direkt in eine PDF"~Datei kompiliert werden können. 
  Vielmehr müssen die gewünschten Grafiken zunächst über den konsekutiven 
  Aufruf des Pfades \Path{latex\,>\,dvips\,>\,ps2pdf} in PDF"~Dateien gewandelt 
  werden. Diese lassen sich von \Format{pdfLaTeX}, \Format{LuaLaTeX} oder 
  \Format{XeLaTeX} anschließend als Abbildungen einbinden. Um dieses Vorgehen 
  zu vereinfachen, lassen sich folgende Pakete nutzen, welche die Inhalte aus 
  den \Environment{pspicture}"~Umgebungen extrahieren und separat über den 
  genannten Pfad kompilieren:
  \ToDo{\Package{luapstricks}}[v2.07]
  \begin{DeclarePackages}
  \itempkg{pst-pdf}
    Dieses Paket stellt Methoden für den Export von PostSript"~Grafiken in 
    PDF"~Datien bereit. Die einzelnen Aufrufe zur Kompilierung von DVI über 
    PostScript zu PDF müssen durch den Anwender manuell beziehungsweise über 
    die Ausgaberoutinen der verwendeten Entwicklungsumgebung durchgeführt 
    werden.
  \itempkg{auto-pst-pdf,pdftricks2}
    Das Paket automatisiert den Export mit \Package{pst-pdf}. Hierfür muss 
    \Format{pdfLaTeX} oder \Format{LuaLaTeX} respektive \Format{XeLaTeX} durch 
    den Parameter \Path{-{}-shell-escape} mit erweiterten Schreibrechten 
    ausgeführt werden. Bitte beachten Sie bezüglich möglicher Probleme die 
    Hinweise in \autoref{sec:tips:auto-pst-pdf}. Eine Alternative dazu ist das 
    Paket \Package{pdftricks2}.
  \end{DeclarePackages}
\end{DeclarePackages}

Im Tutorial \Tutorial{treatise} wird für \Package{pstricks} und \Package{tikz} 
jeweils ein Beispiel gegeben. Um bei der Erstellung von Grafiken mit einem der 
beiden Paketen nicht bei jeder Änderung das komplette Dokument kompilieren zu 
müssen, können diese in separate Dateien ausgelagert werden. Hierfür sind die 
beiden Pakete \Package{standalone} oder \Package{subfiles} sehr nützlich.

Für das Zeichnen einer Grafik mit einem Bildbearbeitungsprogramm, welches die 
Weiterverarbeitung durch \Lettering{LaTeX} erlaubt, möchte ich auf die freien 
Programme \Application{LaTeXDraw} und \Application{Inkscape} verweisen. 
Insbesondere das zuletzt genannte Programm ist sehr empfehlenswert. 

\begin{DeclarePackages}[Grafiken|?]
\itempkg{svg}
  Mit diesem Paket können \emph{automatisiert} alle notwendigen Schritte zum 
  Einfügen einer mit der Anwendung \Application{Inkscape} erstellten 
  SVG"~Grafik in ein \Lettering{LaTeX}-Dokument durchgeführt werden. Weitere 
  Hinweise hierzu sind in \autoref{sec:tips:svg} zu finden.
\end{DeclarePackages}



\subsection{%
  Querverweise und Lesezeichen%
  \index{Querverweise|?}%
  \index{Lesezeichen|?}%
}

Für das Erzeugen von Querverweisen auf bestimmte Gliederungsebenen, Tabellen, 
Abbildungen oder auch Gleichungen muss für diese besagten Elemente zunächst mit 
\Macro{label|\MPName{Label}} ein \emph{eindeutiges} Label erzeugt werden, auf 
welches im Dokument entweder mit \Macro{ref} oder nach dem Laden von 
\Package{hyperref} besser noch mit \Macro{autoref} referenziert werden kann. In 
\autoref{sec:tips:references} sind diesbezüglich weitere Informationen zu 
finden.

\begin{DeclarePackages}[Querverweise|?,Lesezeichen|?]
\itempkg{hyperref,bookmark}
  Für das Erstellen insbesondere von Hyperlinks aber auch erweiterten 
  Querverweisen sowie Lesezeichen~-- auch Outline-Einträge~-- in einem 
  PDF"~Dokument kann das Paket \Package{hyperref} genutzt werden. Wird 
  es geladen, sind außerdem die Option \Option{tudbookmarks} sowie der Befehl 
  \Macro{tudbookmark} nutzbar, welche von den \TUDScript-Klassen bereitgestellt 
  werden. Das Paket \Package{bookmark} erweitert nochmals die Funktionalität 
  für Lesezeichen, beispielsweise um die Möglichkeit zur Festlegung der Ebene 
  eines Outline-Eintrags sowie der Anpassung von Schriftfarbe- und "~stil.
  Beide Pakete sollten~-- bis auf sehr wenige, \emph{explizit dokumentierte} 
  Ausnahmen wie exemplarisch das Paket \Package{glossaries}~-- als letztes in 
  der Präambel eingebunden werden.
\itempkg{varioref}
  Hiermit können für Seitenverweise~-- nicht ausschließlich aber insbesondere 
  auf die aktuelle, vorhergehende oder nachfolgende sowie im doppelseitigen 
  Satz gegenüberliegende Seite~-- anpassbare Textbausteine anstelle der bloßen 
  Seitenzahl verwendet werden.
\itempkg{cleveref}
  \ToDo{Hinweis auf \Package{zref} und \Package{zref-clever}}
  Dieses Paket vereint die Vorzüge der automatischen Benennung referenzierter 
  Objekte mit dem Befehl \Macro{autoref} aus dem Paket \Package{hyperref} und 
  der Verwendung von \Package{varioref}.
\end{DeclarePackages}



\subsection{%
  Aufteilung des Hauptdokumentes in Unterdateien%
  \index{Gliederung}%
}

Um während des Entwurfes eines Dokumentes die Zeitdauer für das Kompilieren zu 
verkürzen, kann dieses in Unterdokumente gegliedert werden. Dadurch wird es 
möglich, nur den momentan bearbeiteten Dokumentteil~-- respektive die aktuelle 
\Lettering{TikZ}- oder PSTricks-Grafik~-- zu kompilieren. Die meiner 
Meinung nach besten Pakete für dieses Unterfangen werden folgend vorgestellt.

\begin{DeclarePackages}
\itempkg{standalone}<\Class{standalone}>
  \ChangedAt*{v2.02:Bugfix für Verwendung der Klasse \Class{standalone}}%
  Das Paket ist für das Erstellen eigenständiger (Unter)"~Dokumente gedacht, 
  welche später in ein Hauptdokument eingebunden werden können. Jedes dieser 
  Teildokumente benötigt eine eigene Präambel. Optional lassen sich die 
  Präambeln der Unterdokumente automatisch in ein Hauptdokument einbinden. 
  Zusätzlich wird eine Klasse bereitgestellt, mit der eingenständige Elemente 
  wie beispielsweise zugeschnittene Grafiken erstellt werden können.
\itempkg{subfiles}[v2.02]
  Dieses Paket wählt einen etwas anderen Ansatz als \Package{standalone}. Es 
  ist von Anfang an dafür gedacht, ein dediziertes Hauptdokument zu verwenden. 
  Die darin mit \Macro{subfiles} eingebundenen Unterdateien nutzen bei der 
  autarken Kompilierung dessen Präambel.
\end{DeclarePackages}

Unabhängig davon, ob Sie eines der beiden Pakete nutzen oder alles in einem 
Dokument belassen, ist es ratsam, eigens definierte Befehle, Umgebungen und 
ähnliches in ein separates Paket auszulagern. Dafür müssen Sie lediglich eine 
leere Textdatei erzeugen und diese unter \Path{mypreamble.sty}~-- oder einem 
beliebigen anderen Namen mit der Dateiendung \Path{.sty}~-- im gleichen Ordner 
wie das Hauptdokument speichern. Anschließend können Sie in dieser Datei ihre 
Deklarationen vornehmen und diese mit dem gewählten Namen ohne die Dateiendung
\Macro*{usepackage|\MPValue{mypreamble}} in das Dokument einbinden. Dies hat 
den Vorteil, dass das Hauptdokument zum einen übersichtlich bleibt und Sie zum 
anderen Ihre persönliche Präambel generisch wachsen lassen und für andere 
Dokumente wiederverwenden können, wobei es dann sicherlich ratsam wäre, das 
Paket zentral in Ihrem lokalen \Path{texmf}"~Pfad abzulegen. 



\subsection{Die kleinen und großen Helfer\dots}

Hier taucht alles auf, was nicht in die vorherigen Kategorien eingeordnet 
werden konnte.

\begin{DeclarePackages}
\itempkg{marginnote}
  \index{Randnotizen}%
  Mit dem Befehl \Macro{marginpar} lassen sich Randnotizen erzeugen. Diese sind 
  in \Lettering{LaTeX} spezielle Gleitobjekte, wodurch selbige nicht immer 
  direkt an der ursprünglich intendierten Stelle am Blattrand gesetzt werden. 
  Das Paket \Package{marginnote} stellt den Befehl \Macro{marginnote} für 
  nicht"~gleitende Randnotizen zur Verfügung. Eine Alternative dazu ist Paket 
  \Package{mparhack}.
\itempkg{todonotes}
  \index{Randnotizen}%
  Mit \Package{todonotes} können noch offene Aufgaben in unterschiedlicher 
  Formatierung am Blattrand oder im direkt Fließtext ausgegeben werden. Aus 
  allen Anmerkungen lässt sich eine Liste aller offenen Punkte erzeugen.
  \ToDo{\Package{fixme} erwähnen}[v2.07]
\itempkg{xparse}
  \index{Befehle}%
  \index{Umgebungen}%
  \index{Befehlsdeklaration}%
  Dieses mächtige Paket entstammt dem \Lettering{LaTeX3}"~Projekt und bietet 
  für die Erstellung eigener Befehle und Umgebungen einen alternativen Ansatz 
  zu den bekannten \Lettering{LaTeX}"=Deklarationsbefehlen \Macro*{newcommand} 
  und \Macro*{newenvironment} sowie deren Derivaten. Mit \Package{xparse} wird 
  es möglich, obligatorische und optionale Argumente an beliebigen Stellen 
  innerhalb des Befehlskonstruktes zu definieren. Auch die Verwendung anderer 
  Zeichen als eckige Klammern für die Spezifizierung eines optionalen 
  Argumentes ist möglich.
  \ToDo{xparse im Kernel!}[v2.07]
\itempkg{keyval,xkeyval,pgfkeys,expkv,l3keys}[v2.02]
  \index{Befehle}%
  \index{Umgebungen}%
  \index{Parameter}%
  \index{Befehlsdeklaration}%
  Von \Package{scrbase} wird das Paket \Package{keyval} geladen, um Optionen 
  oder Funktionsargumente mit einer Schlüssel"=Wert"=Syntax deklarieren zu 
  können. Diese Funktionalität kann auch für zusätzlich vom Anwender definierte 
  Makros genutzt werden, um innerhalb eines (optionalen) Argumentes mehrere 
  Parameter nutzen zu können.
  
  Soll gänzlich auf die Funktionalitäten von \Package{scrbase} verzichtet 
  werden~-- beispielsweise zur Entwicklung eigener Pakete~-- und dennoch eine 
  Verarbeitung von Optionen im Schlüssel"=Wert"=Format gewünscht sein, kann das 
  Paket \Package{xkeyval} zum Einsatz kommen, welches viele Möglichkeiten zur 
  Deklaration unterschiedlicher Typen von Schlüsseln bereitstellt. Alternativ 
  dazu kann auch die \Package{expkv}"~Familie respektive \Package{pgfkeys} oder 
  für \Lettering{LaTeX3} auch \Package{l3keys} genutzt werden.
\itempkg{scrlfile}[v2.05]
  \index{Kompatibilität!Pakete}%
  Dieses \KOMAScript-Paket erlaubt es, auf das Laden von Klassen oder Paketen 
  direkt davor oder danach zu reagieren, um beispielsweise Paketabhängigkeiten 
  aufzulösen oder nach dem Laden eines bestimmten Paketes gezielt Befehle 
  anzupassen. Mehr dazu ist im \scrguide zu finden.
\itempkg{calc}
  \index{Berechnungen}%
  Normalerweise lassen sich Berechnungen im Dokument lediglich mit 
  Low"~Level"~\Lettering{TeX}"=Primitiven durchführen. Dieses Paket stellt eine 
  einfachere Syntax für Rechenoperationen der vier Grundrechenarten zur 
  Verfügung. Zusätzlich werden neue Befehle zur Bestimmung der Höhe und Breite 
  bestimmter Textauszüge definiert.
\itempkg{mwe,blindtext}[v2.02]
  \index{Minimalbeispiel}%
  Mit dem Paket \Package{mwe} lassen sich sehr einfach Minimalbeispiele 
  erzeugen, die sowohl Blindtexte als auch Abbildungen enthalten sollen. Werden 
  lediglich Textabschnitte etc. benötigt, ist das Laden von \Package{blindtext} 
  ausreichend. Weiterhin können mit der Umgebung \Environment{filecontents} 
  gegebenfalls zusätzlich benötigte Textdateien~-- beispielsweise eine einfache 
  \File*{.bib}"~Literaturdatei~-- erzeugt werden.
\itempkg{pdfpages}
  Das Paket ermöglicht die Einbindung von einzelnen oder mehreren PDF"~Dateien.
\itempkg{crop}
  \index{Beschnittzugabe}%
  \index{Schnittmarken}%
  Hiermit können eine Beschnittzugabe sowie Schnittmarken~-- beispielsweise für 
  Poster~-- erzeugt werden. Hierzu ist in \autoref{sec:tips:crop} ein Beispiel 
  zu finden.
\itempkg{pagecolor}
  \index{Farben}%
  Mit dem Paket lässt sich die Hintergrundfarbe der Seiten im Dokument ändern.
\itempkg{afterpage}
  Der Befehl \Macro{afterpage|\MPValue{\dots}} kann genutzt werden, um den 
  Inhalt aus dessen Argument direkt nach der Ausgabe der aktuellen Seite 
  auszuführen.
\itempkg{filemod}
  \ToDo{raus, text in Hinweise verschieben (pdffilemoddate)}[v2.07]
  Wird entweder \Format{pdfLaTeX} oder \Format{LuaLaTeX} als Textsatzsystem 
  eingesetzt, können mit diesem Paket das Änderungsdatum zweier Dateien 
  miteinander verglichen und in Abhängigkeit davon definierbare Aktionen 
  ausgeführt werden.
\itempkg{listings}[v2.02]
  \index{Quelltextdokumentation}%
  Dieses Paket eignet sich hervorragend zur Quelltextdokumentation in 
  \Lettering{LaTeX}. Es bietet die Möglichkeit, externe Quelldateien einzulesen 
  und darzustellen sowie die Syntax in Abhängigkeit der verwendeten 
  Programmiersprache hervorzuheben. Zusätzlich lässt sich ein Verzeichnis mit 
  allen eingebundenen sowie direkt im Dokument angegebenen Quelltextauszügen 
  erstellen. Wird \Package{listings} in Dokumenten mit UTF"~8"=Eingabekodierung 
  verwendet, sollten direkt nach dem Laden des Paketes in der Präambel folgende 
  Einstellungen vorgenommen werden:
\begin{quoting}[rightmargin=0pt]
\begin{Code}
\lstset{%
  inputencoding=utf8,extendedchars=true,
  literate=%
    {ä}{{\"a}}1 {ö}{{\"o}}1 {ü}{{\"u}}1
    {Ä}{{\"A}}1 {Ö}{{\"O}}1 {Ü}{{\"U}}1
    {ß}{{\ss}}1 {~}{{\textasciitilde}}1
    {»}{{\guillemetright}}1 {«}{{\guillemetleft}}1
}
\end{Code}
\end{quoting}
\itempkg{selinput,inputenc}
  \index{Eingabekodierung}%
  Die Eingabekodierung ist standardmäßig auf \PValue{utf8} festgelegt, was von 
  aktuellen \Lettering{LaTeX}"=Entwicklungsumgebung unterstützt wird. Dies 
  sollte im Normalfall auch beibehalten werden. Gegebenenfalls sind aber ältere 
  Dokumente in einer anderen Eingabekodierung gespeichert. Lässt sich diese 
  nicht ändern, ist es mit \Format{pdfLaTeX} möglich, eine andere~-- in der 
  \hyperref[sec:tips:ide]{Entwicklungsumgebung (\autoref*{sec:tips:ide})} 
  vom Anwender eingestellte~-- Eingabekodierungen zu nutzen. Diese kann mit dem 
  Paket \Package{selinput} (automatisch) für \Lettering{LaTeX} festgelegt 
  werden:
\begin{quoting}[rightmargin=0pt]
\begin{Code}
\usepackage{selinput}
\SelectInputMappings{adieresis={ä},germandbls={ß}}
\end{Code}
\end{quoting}
  Alternativ dazu lässt sich mit dem Paket \Package{inputenc} mit
\begin{quoting}[rightmargin=0pt]
\begin{Code}[escapechar=§]
\usepackage§\OPName{Eingabekodierung}§{inputenc}
\end{Code}
\end{quoting}
  die Eingabekodierung manuell einstellen. Das Paket \Package{fontenc} sollte 
  in jedem Fall \emph{zuvor} geladen werden.
\end{DeclarePackages}



\subsection{Bugfixes}

\begin{DeclarePackages}
\itempkg{scrhack}
  \index{Kompatibilität!Pakete}%
  Das Paket behebt Kompatibilitätsprobleme der \KOMAScript-Klassen mit den 
  Paketen \Package{hyperref}, \Package{float}, \Package{floatrow} und
  \Package{listings}. Es ist durchaus empfehlenswert, jedoch sollte unbedingt 
  die Dokumentation beachtet werden.
\itempkg{mparhack}
  \index{Randnotizen}%
  Zur Behebung falsch gesetzter Randnotizen wird ein Bugfix für den Befehl 
  \Macro{marginpar} bereitgestellt. Alternativ dazu lässt sich auch das Paket 
  \Package{marginnote} verwenden.
\itempkg{morewrites,scrwfile}
  Falls der Fehler
\begin{quoting}[rightmargin=0pt]
\begin{Code}
No room for a new \write
\end{Code}
\end{quoting}
  erscheint, kann dieser sehr wahrscheinlich mit dem Paket \Package{morewrites} 
  behoben werden. Alternativ dazu lässt sich auch das Paket 
  \Package{scrwfile} nutzen. Mehr dazu in \autoref{sec:tips:write}.
\itempkg{fix-cm}
  \index{Schriftart}%
  \index{Schriftgröße}%
  Sollte bei einer Schriftgrößenänderung eine oder mehrere Warnungen der Form
\begin{quoting}[rightmargin=0pt]
\begin{Code}
Font shape `T1/cmr/m/n' in size <...> not available
\end{Code}
\end{quoting}
  erscheinen, so sollte das Paket \Package{fix-cm} \emph{vor} der Klasse 
  geladen werden. Siehe dazu auch die Hinweise in \autoref{sec:tips:fontsize}.
\end{DeclarePackages}
\index{Pakete|?)}%
