\chapter{Versionsänderungen}
%%\begin{Entity}{\Bundle{tudscr}}
\section{%
  Kompatibilitätseinstellungen zu früheren Versionen%
  \index{Kompatibilität|!(}%
  \index{Änderungen|?(}%
}

Bei der Entwicklung von \TUDScript lässt es sich nicht immer vermeiden, dass 
Verbesserungen sowie Korrekturen an den Klassen und Paketen zu Änderungen am 
Ergebnis der Ausgabe führen, insbesondere bei Umbruch und Layout. Für bereits
archivierte Dokumente, welche mit einer früheren Version erstellt wurden ist 
dies unter Umständen jedoch eher unerwünscht.

\begin{Declaration}
  {\Option{tudscrver=\PMisc}}
  (last)
  [v2.03]
\printdeclarationlist

Mit dieser Option wird es möglich, auf das (Umbruch"~)Verhalten einer älteren 
respektive früheren Version von \TUDScript umzuschalten, um nach der 
Kompilierung das erwartete Ergebnis zu erhalten. Neue Möglichkeiten, die sich 
nicht auf den Umbruch oder das Layout auswirken, sind bei aktivierter 
Kompatibilität zu einer älteren Version dennoch verfügbar. 

Bei der Angabe einer unbekannten Version als Wert wird eine Warnung ausgegeben 
und \Option{tudscrver=first} angenommen. Mit \Option{tudscrver=last} wird die 
jeweils aktuell verfügbare Version ausgewählt und folglich auf die zukünftige 
Kompatibilität des Dokumentes zu der aktuell genutzten Version verzichtet. 
Dieses Verhalten entspricht der Voreinstellung. Es ist zu beachten, dass die 
Nutzung von \Option{tudscrver} nur als Klassenoption möglich ist.
\begin{DeclareValues}{\Option{tudscrver}}
\itemval{2.02,first}
  \ChangedAt{%
    v2.03:Satzspiegel im \CD geändert, Logo von \DDC im Fußbereich 
    wird ohne vergrößerten Seitenrand verwendet
  }%
  Der Satzspiegel (\seeplain{\Option{cdgeometry}}) im Layout des 
  \CDs wurde mit der Version~v2.03 leicht geändert. Der obere Seitenrand wurde 
  verkleinert, der untere im gleichen Maße vergrößert. Der verfügbare 
  Textbereich blieb folglich unverändert. Bei der Aktivierung des \DDC-Logos im 
  Seitenfußbereich (\seeplain{\Option{ddcfoot}}) wird der 
  identische Satzspiegel genutzt. Dieses Verhalten lässt sich mit 
  \Option{tudscrver=2.02} deaktivieren.
\itemval{2.03}
  \index{Leerraum}%
  \index{Schriftgröße}%
  Seit der Version~v2.04 werden mehrere Längen für vertikalen Leerraum in 
  Abhängigkeit der gewählten Schriftgröße definiert. Diese Funktionalität lässt 
  sich mit \Option{tudscrver=2.03} deaktivieren, wobei hierfür lediglich die 
  Option \Option{relspacing=false} gesetzt wird. 
\item[\PValue{2.04}]
  \index{Satzspiegel}%
  \ChangedAt{%
    v2.05:Satzspiegeleinstellungen für die jeweilige ISO/DIN"~Klasse 
    des Papierformates identisch%
  }%
  Mit der Version~v2.05 werden die vorgegebenen Einstellungen zum Satzspiegel 
  anhand der B"~ISO/DIN"~Reihe vorgenommen. Damit sind für alle Papierformate 
  einer spezifischen ISO/DIN"~Klasse die Seitenränder identisch. Mit der Wahl 
  \Option{tudscrver=2.04} ist der Satzspiegel von der A"~ISO/DIN"~Reihe 
  abhängig, sodass die B- und C"~Papierformate der gleichen Klasse größere 
  Seitenränder erhalten, als die D- und A"~Formate.
\itemval{2.05}
  \ChangedAt{%
    v2.06:Griechische Lettern standardmäßig kursiv;%
    v2.06:Überschriften nutzen immer \OpenSans, wenn Layout des \CDs aktiviert;%
    v2.06:Abstände des vertikalen Leerraums in Abhängigkeit von der 
      Schriftgröße verbessert;%
  }%
  Mit dem Wechsel der Hausschrift zu \OpenSans ergaben sich einige Änderungen 
  bezüglich der Erscheinung des \CDs. Mit dieser Kompatibilitätseinstellung 
  werden die alten Schriftfamilien \Univers und \DIN 
  (\Option{cdoldfont=true}, \Option{headings=light}, \Option{ttfont=lmodern})
  aktiviert. Weiterhin wirkt sich für diese Schriftfamilien im Mathematikmodus 
  die Option \Option{slantedgreek=true} lediglich auf die griechischen 
  Majuskeln aus.
  
  Wenn das Layout des \CDs nicht mit der Option \Option{cd=false} deaktiviert 
  wurde, wird ab der Version~v2.06 für Überschriften \OpenSans verwendet, auch 
  wenn für den Fließtext die Verwendung der Schriften des \CDs mit der 
  Einstellung \Option{cdfont=false} deaktiviert wurde. Weiterhin werden die 
  Abstände des vertikalen Leerraums (Option \Option{relspacing=true}) nicht 
  mehr in diskreten Stufen gesetzt, womit ein besseres Ergebnis für nicht 
  vordefinierte Schriftgrößen (Option \Option(\Bundle{koma-script}){fontsize}) 
  erzielt wird.
\itemval{2.06}
  Dies ist Kompatibilitätseinstellung für \vTUDScript*{\TUDScript} und wird für 
  zukünftige Änderungen bereits vorgehalten. Soll ein mit dieser Version 
  erzeugtes Dokument auch mit einer späteren Version von \TUDScript nach einem 
  \Lettering{LaTeX}"~Lauf das gleiche Ausgabeergebnis liefern, muss dies mit 
  dieser Einstellung angegeben werden.
\itemval{last}
  Es werden keine Kompatibilitätseinstellungen für das Dokument vorgenommen. 
  Mit einer späteren Version von \TUDScript kann ein anderes Umbruchverhalten 
  innerhalb des Dokumentes auftreten. Dies ist die Standardeinstellung.
\end{DeclareValues}
\end{Declaration}
\index{Änderungen|?)}%
\index{Kompatibilität|!)}%



\section{%
  Veraltete sowie vollständig entfernte Optionen und Befehle%
  \index{Änderungen|!(}%
  \label{sec:cessations}%
}

Einige Optionen und Befehle waren während der Weiterentwicklung von \TUDScript
in ihrer ursprünglichen Form nicht mehr umsetzbar oder wurden~-- unter anderem 
aus Gründen der Kompatibilität zu anderen Paketen~-- schlichtweg verworfen. 
Dennoch besteht für die meisten entfallenen Direktiven eine Möglichkeit, deren 
Funktionalität ohne größere Aufwände mit \vTUDScript*{\TUDScript} darzustellen. 
Ist dies der Fall, wird hier entsprechend kurz darauf hingewiesen.



\begin{Cessations}{v2.00}
\begin{Cessation}
  [\Option{cd=\PMisc}]
  {\Option{cd=alternative}}
\begin{Cessation}
  [\Option{cdtitle=\PMisc}]
  {\Option{cdtitle=alternative}}
\begin{Cessation}
  {\Length{titlecolwidth}}
\begin{Cessation}
  {\Term{authortext}}
\printdeclarationlist

\ToDo[imp]{Indexeinträge im Fließtext automatisch unterdrücken?!}[v2.07]%
Die alternative Titelseite ist komplett aus dem \TUDScript-Bundle entfernt 
worden. Dementsprechend entfallen auch die dazugehörigen Optionen sowie Länge 
und Bezeichner.
\end{Cessation}
\end{Cessation}
\end{Cessation}
\end{Cessation}

\begin{Cessation}
  {\Option{color=\PBoolean}}
  <\Option{cd}>
\printdeclarationlist

Die Einstellungen der farbigen Ausprägung des Dokumentes erfolgt über die 
Option \Option*{cd}.
\end{Cessation}

\begin{Cessation}
  {\Option{tudfonts=\PBoolean}}
  <\Option{cdfont}>
\begin{Cessation}
  {\Option{tudfoot=\PBoolean}}
  <\Option{cdfoot}>
\printdeclarationlist

Die~-- wesentlich erweiterte~-- Option zur Schrifteinstellung sowie die Option 
für den Seitenfuß sind umbenannt worden, um dem Namensschema von \TUDScript zu 
entsprechen.
\end{Cessation}
\end{Cessation}

\begin{Cessation}
  {\Option{headfoot=\PMisc}}
  '\seeplain{\KOMAScript-Optionen \Option*{headinclude} \& \Option*{footinclude}}'
\printdeclarationlist

Diese Option war für \TUDScript in der \emph{Version~v1.0} notwendig, um die 
parallele Verwendung der Pakete \Package*{typearea} und \Package*{geometry} zu 
ermöglichen. Die Erstellung des Satzspiegels wurde komplett überarbeitet. 
Mittlerweile werden an das Paket \Package*{geometry} die Einstellungen der 
\KOMAScript-Optionen \Option*(\Package{typearea}){headinclude} und 
\Option*(\Package{typearea}){footinclude} direkt 
weitergereicht, sodass diese Option nicht mehr benötigt wird und entfernt wurde.
\end{Cessation}

\begin{Cessation}
  {\Option{partclear=\PBoolean}}
  <\Option{cleardoublespecialpage}>
\begin{Cessation}
  {\Option{chapterclear=\PBoolean}}
  <\Option{cleardoublespecialpage}>
\printdeclarationlist

Beide Optionen sind in der neuen Option \Option*{cleardoublespecialpage} 
aufgegangen, womit ein konsistentes Layout erreicht wird. Die ursprünglichen 
Optionen entfallen. 
\end{Cessation}
\end{Cessation}

\begin{Cessation}
  {\Option{abstracttotoc=\PBoolean}}
  <\Option{abstract}>
\begin{Cessation}
  {\Option{abstractdouble=\PBoolean}}
  <\Option{abstract}>
\printdeclarationlist

Beide Optionen wurden in die Option \Option*{abstract} integriert und sind 
deshalb überflüssig.
\end{Cessation}
\end{Cessation}

\begin{Cessation}
  {\Macro{logofile|\MPName{Dateiname}}}
  <\Macro{headlogo}>
\printdeclarationlist

Der Befehl wurde in \Macro*{headlogo} umbenannt, wobei die Funktionalität 
weiterhin bestehen bleibt.
\end{Cessation}

\begin{Cessation}
  {\Option{bookmarks=\PBoolean}}
  <\Option{tudbookmarks}>
\printdeclarationlist

Die Option wurde umbenannt, um Überschneidungen mit \Package*{hyperref} zu 
vermeiden.
\end{Cessation}

\begin{Cessation}
  {\Length{signatureheight}}
\printdeclarationlist

Die Höhe für die Zeile der Unterschriften wurde dehnbar gestaltet, eine etwaige 
Anpassung durch den Anwender ist nicht vonnöten.
\end{Cessation}

\begin{Cessation}
  {\Macro{location|\MPName{Ort}}}
  <\Macro{place}>
\printdeclarationlist

In Anlehnung an andere \Lettering{LaTeX}"~Pakete und "~Klassen wurde 
dieser Befehl in \Macro*{place} umbenannt.
\end{Cessation}

\begin{Cessation}
  {\Term{titlecoldelim}}
  <\Macro{titledelimiter}>
\printdeclarationlist

Das Trennzeichen für Bezeichnungen beziehungsweise beschreibende Texte und dem 
eigentlichen Feld auf der Titelseite ist nicht mehr sprachabhängig und wurde 
umbenannt.
\end{Cessation}

\begin{Cessation}
  {\Macro{confirmationandrestriction}}
  <\Macro{declaration}>
\begin{Cessation}
  {\Macro{restrictionandconfirmation}}
  <\Macro{declaration}>
\printdeclarationlist

Die beiden Befehle entfallen, stattdessen sollte entweder der Befehl 
\Macro*{declaration} oder die Umgebung \Environment*{declarations} zusammen 
mit den Befehlen \Macro*{confirmation} und \Macro*{blocking} verwendet 
werden, wobei sich diese in der Umgebung in beliebiger Reihenfolge anordnen 
lassen.
\end{Cessation}
\end{Cessation}


\minisec{\taskname}

\begin{Entity}{\Package{tudscrsupervisor}}
Die Umgebung zur Erstellung einer Aufgabenstellung für eine wissenschaftliche 
Arbeit wurde in das Paket \Package{tudscrsupervisor} ausgelagert. Dieses muss 
für die Verwendung der Umgebung \Environment*{task} und der daraus abgeleiteten 
standardisierten Form zwingend geladen werden.

\begin{Cessation}
  {\Option{cdtask=\PMisc}}
  <\Environment{task}>
\begin{Cessation}
  {\Option{taskcompact=\PBoolean}}
\begin{Cessation}
  {\Length{taskcolwidth}}
\printdeclarationlist

Die Klassenoption \Option*'none'{cdtask} ist entfernt worden, alle Einstellungen 
erfolgen direkt über das optionale Argument der Umgebung \Environment*{task}. 
Die Variante eines kompakten Kopfes mit der Option \Option*'none'{taskcompact} 
wird nicht mehr bereitgestellt. Die Möglichkeit zur manuellen Festlegung der 
Spaltenbreite für den Kopf der Aufgabenstellung mit \Length*'none'{taskcolwidth} 
wurde aufgrund der verbesserten automatischen Berechnung entfernt.
\end{Cessation}
\end{Cessation}
\end{Cessation}

\begin{Cessation}
  {\Macro{studentid|\MPName{Matrikelnummer}}}
  <\Macro{matriculationnumber}>
\begin{Cessation}
  {\Macro{enrolmentyear|\MPName{Matrikeljahr}}}
  <\Macro{matriculationyear}>
\begin{Cessation}
  {\Macro{submissiondate|\MPName{Datum}}}
  <\Macro{date}>
\begin{Cessation}
  {\Macro{birthday|\MPName{Geburtsdatum}}}
  <\Macro{dateofbirth}>
\begin{Cessation}
  {\Macro{birthplace|\MPName{Geburtsort}}}
  <\Macro{placeofbirth}>
\begin{Cessation}
  {\Macro{startdate|\MPName{Ausgabedatum}}}
  <\Macro{issuedate}>
\printdeclarationlist

Alle Befehle wurden umbenannt und sind jetzt neben der Aufgabenstellung auch 
für die Titelseite im \CD nutzbar.
\end{Cessation}
\end{Cessation}
\end{Cessation}
\end{Cessation}
\end{Cessation}
\end{Cessation}

\begin{Cessation}
  {\Term{studentidname}}
  <\Term{matriculationnumbername}>
\begin{Cessation}
  {\Term{enrolmentname}}
  <\Term{matriculationyearname}>
\begin{Cessation}
  {\Term{submissiontext}}
  <\Term{datetext}>
\begin{Cessation}
  {\Term{birthdaytext}}
  <\Term{dateofbirthtext}>
\begin{Cessation}
  {\Term{birthplacetext}}
  <\Term{placeofbirthtext}>
\begin{Cessation}
  {\Term{supervisorIIname}}
  <\Term{supervisorothername}>
\begin{Cessation}
  {\Term{defensetext}}
  <\Term{defensedatetext}>
\begin{Cessation}
  {\Term{starttext}}
  <\Term{issuedatetext}>
\begin{Cessation}
  {\Term{duetext}}
  <\Term{duedatetext}>
\printdeclarationlist

Die Bezeichner wurden in Anlehnung an die dazugehörigen Befehlsnamen umbenannt.
\end{Cessation}
\end{Cessation}
\end{Cessation}
\end{Cessation}
\end{Cessation}
\end{Cessation}
\end{Cessation}
\end{Cessation}
\end{Cessation}

\begin{Cessation}
  {\Macro{tasks|\MPName{Ziele}\MPName{Schwerpunkte}}}
  <\Macro{taskform}>
\begin{Cessation}
  {\Term{focustext}}
  <\Term{focusname}>
\begin{Cessation}
  {\Term{objectivestext}}
  <\Term{objectivesname}>
\printdeclarationlist

Dieser Befehl wurde in \Macro*{taskform} umbenannt und in der Funktionalität 
erweitert. Die Namen der darin verwendeten Bezeichner wurden ebenfalls leicht 
abgewandelt.
\end{Cessation}
\end{Cessation}
\end{Cessation}
\end{Entity}
\end{Cessations}



\begin{Cessations}{v2.02}
  [Umbenennung einiger Befehle für Kompatibilität mit anderen Paketen]
\begin{Cessation}
  {\Length{chapterheadingvskip}}
  <\Option{pageheadingsvskip}>
\printdeclarationlist

Die vertikale Positionierung von Überschriften wurde aufgeteilt. Zum einen kann 
diese für Titel"~, Teile- und Kapitelseiten (\Option*{chapterpage=true}) über 
die Option \Option*{pageheadingsvskip} geändert werden. Für den Titelkopf
(\Option*(\Bundle{koma-script}){titlepage=false}) sowie Kapitelüberschriften 
(\Option*{chapterpage=false}) kann dies mit \Option*{headingsvskip} 
unabhängig davon erfolgen.
\end{Cessation}

\begin{Cessation}
  {\Macro{degree|\OPName{Abk.}\MPName{Grad}}}
  <\Macro{graduation}>
\begin{Cessation}
  {\Term{degreetext}}
  <\Term{graduationtext}>
\begin{Cessation}
  {\Macro{restriction|\OPList{Firma}}}
  <\Macro{blocking}>
\begin{Cessation}
  {\Term{restrictionname}}
  <\Term{blockingname}>
\begin{Cessation}
  {\Term{restrictiontext}}
  <\Term{blockingtext}>
\printdeclarationlist

Die beiden Befehle wurden zur Erhöhung der Kompatibilität aufgrund möglicher 
Konflikte mit anderen Paketen umbenannt, die jeweils dazugehörigen Bezeichner 
dahingehend angepasst.
\end{Cessation}
\end{Cessation}
\end{Cessation}
\end{Cessation}
\end{Cessation}

\begin{Declaration*}
  {\Environment{tudpage|\OPList{Sprache}}}
\begin{Cessation}
  {\Environment{tudpage/head=\PMisc}}
  <\Environment{tudpage/pagestyle}>
\begin{Cessation}
  {\Environment{tudpage/foot=\PMisc}}
  <\Environment{tudpage/pagestyle}>
\printdeclarationlist

Die Funktionalität beider Parameter wurde durch den Parameter 
\Environment*{tudpage/pagestyle} ersetzt.
\end{Cessation}
\end{Cessation}
\end{Declaration*}


\minisec{Änderungen im Paket \Package{tudscrsupervisor}}

\begin{Entity}{\Package{tudscrsupervisor}}
Im Paket \Package{tudscrsupervisor} gab es ein paar kleinere Anpassungen.
\begin{Cessation}
  {\Macro{branch|\MPName{Studienrichtung}}}
  <\Macro{discipline}>
\begin{Cessation}
  {\Term{branchname}}
  <\Term{disciplinename}>
\printdeclarationlist

Für die Aufgabenstellung (\Environment*{task} und \Macro*{taskform}) wurden 
dieser Befehl sowie der dazugehörige Bezeichner umbenannt.
\end{Cessation}
\end{Cessation}

\begin{Cessation}
  {\Macro{contact|\MPName{Kontaktperson(en)}}}
  <\Macro{contactperson}>
\begin{Cessation}
  {\Term{contactname}}
  <\Term{contactpersonname}>
\begin{Cessation}
  {\Macro{phone|\MPName{Telefonnummer}}}
  <\Macro{telephone}>
\begin{Cessation}
  {\Macro{email|\MPName{E-Mail-Adresse}}}
  <\Macro{emailaddress}>
\printdeclarationlist

Alle hier aufgeführten Befehle und Bezeichner zur Erstellung eines Aushangs 
(\Environment*{notice} und \Macro*{noticeform}) wurden umbenannt.
\end{Cessation}
\end{Cessation}
\end{Cessation}
\end{Cessation}
\end{Entity}
\end{Cessations}



\begin{Cessations}{v2.03}
\begin{Cessation}
  {\Option{geometry=\PBoolean}}
  <\Option{cdgeometry}>
\printdeclarationlist

Diese Option wurde zur Konsistenz sowie dem Vermeiden eines möglichen 
Konfliktes mit einer späteren \KOMAScript-Version umbenannt. Die Funktionalität 
bleibt bestehen.
\end{Cessation}

\begin{Cessation}
  {\Option{cdfonts=\PBoolean}}
  <\Option{cdfont}>
\begin{Cessation}
  {\Option{din=\PBoolean}}
  <\Option{cdfont}>
\printdeclarationlist

Die Option \Option*{cdfont} wurde erweitert und fungiert als zentrale 
Schnittstelle zur Schrifteinstellung. 
\end{Cessation}
\end{Cessation}

\begin{Cessation}
  {\Option{sansmath=\PBoolean}}
  <\Option{cdmath}>
\printdeclarationlist

Diese Option wurde aus Gründen der Konsistenz umbenannt. Zusätzlich wurde die 
Funktionalität erweitert.
\end{Cessation}

\begin{Cessation}
  {\Option{barfont=\PMisc}}
  <\Option{cdhead}>
\begin{Cessation}
  {\Option{widehead=\PBoolean}}
  <\Option{cdhead}>
\printdeclarationlist

Die Optionen \Option*'none'{barfont} und \Option*'none'{widehead} wurden in der 
Option \Option*{cdhead} zusammengefasst.
\end{Cessation}
\end{Cessation}

\begin{Declaration*}
  {\Environment{tudpage|\OPList{Sprache}}}
\begin{Cessation}
  {\Environment{tudpage/color=\PName{Farbe}}}
\printdeclarationlist

Der Parameter \Environment*'none'{tudpage/color=\PSet{Farbe}} der 
\Environment*{tudpage}"~Umgebung wurde ersatzlos entfernt.
\end{Cessation}
\end{Declaration*}
\end{Cessations}



\begin{Cessations}{v2.04}
\begin{Cessation}
  {\Option{fontspec=\PBoolean}}
\printdeclarationlist

Anstatt diese Option zu aktivieren, kann einfach das Paket \Package{fontspec} 
in der Dokumentpräambel geladen werden. Dadurch lassen sich anschließend 
zusätzliche Pakete nutzen, welche auf die Verwendung von \Package{fontspec} 
angewiesen sind. Sollte diese Option dennoch genutzt werden, müssen alle auf 
das Paket \Package{fontspec} aufbauenden Einstellungen durch den Anwender mit 
\Macro*(\Bundle{koma-script}){AfterPackage|\MPValue{fontspec}\MPValue{\dots}} 
verzögert werden. In \fullref{sec:fonts} sind weitere Hinweise zur Verwendung 
des Paketes \Package{fontspec} zu finden.
\end{Cessation}
\end{Cessations}



\begin{Cessations}{v2.05}
\begin{Cessation}
  {\Length{headingsvskip}}
  <\Option{headingsvskip}>
\begin{Cessation}
  {\Length{pageheadingsvskip}}
  <\Option{pageheadingsvskip}>
\printdeclarationlist

Die vertikale Positionierung von speziellen Überschriften erfolgt nicht mehr 
über diese beiden Längen sondern über die Optionen \Option*{headingsvskip} 
sowie \Option*{pageheadingsvskip}.
\end{Cessation}
\end{Cessation}


\begin{Cessation}
  {\Length{footlogoheight}}
  <\Option{footlogoheight}>
\printdeclarationlist

Auch die Höhe der Logos im Fuß der \PageStyle*{tudheadings}"=Seitenstile wird 
von nun an mit der Option \Option*{footlogoheight} und nicht mehr mit dieser 
Länge festgelegt.
\end{Cessation}
\end{Cessations}



\begin{Cessations}{v2.06}
\begin{Declaration}
  {\Option{cdoldfont=\PMisc}}
  (false)
  [v2.06]
\printdeclarationlist

Mit der Version~v2.06 wird \OpenSans als Hausschrift verwendet. Um weiterhin 
ältere Dokumente mit den Schriften \Univers und \DIN erzeugen zu können, wird 
diese Option bereitgestellt.
\Attention{%
  Sie lässt sich ausschließlich als Klassenoption~-- oder für die Pakete 
  \Package*{tudscrfonts} und \Package*{fix-tudscrfonts} als Paketoption~-- 
  nutzen.
}
Die Verwendung mit \Macro*{TUDoption} oder \Macro*{TUDoptions} ist nicht 
möglich. Wurden mit \Option{cdoldfont} die alten Schriftfamilien aktiviert, 
wirken allerdings alle Einstellungen der Option \Option*{cdfont}~-- auch nach 
dem Laden~-- weiterhin wie gewohnt.
\begin{DeclareValues}{\Option{cdoldfont}}
\itemval{false}
  Das Verhalten ist äquivalent zu \Option*{cdfont=false}, die Hausschrift ist 
  nicht aktiv.
\itemval{true}
  Es werden die alten Hausschriften \Univers für den Fließtext sowie \DIN für 
  Überschriften der obersten Gliederungsebenen bis einschließlich 
  \Macro*{subsubsection} genutzt. Die Schriftstärke lässt sich mit 
  \Option*{cdfont=true} respektive \Option*{cdfont=heavy} anpassen.
\end{DeclareValues}

Für die \TUDScript-Klassen sowie die vom Paket \Package*{fix-tudscrfonts} 
unterstützten Dokumentklassen kann die für eine Gliederungsebenen verwendete 
Schriftart angepasst werden.
\begin{DeclareValues}{\Option{cdoldfont}}
\itemval{nodin}
  Für Überschriften wird \Univers anstatt \DIN verwendet.
\itemval{din}
  Mit dieser Einstellung wird \DIN in Überschriften genutzt. 
\itemval{onlydin}
  Überschriften werden in \DIN gesetzt, für den Fließtext wird nicht \Univers 
  sondern die \Lettering{LaTeX}"=Standardschriften respektive die eines geladenen 
  Schriftpaketes verwendet.
\end{DeclareValues}
\end{Declaration}

\begin{Cessation}
  {\Macro{ifdin|\MPName{Dann-Teil}\MPName{Sonst-Teil}}}
\printdeclarationlist

Auch dieser Befehl wird lediglich definiert, falls die zuvor beschriebene 
Option \Option{cdoldfont} aktiviert wurde. Es wird geprüft, ob aktuell die 
Schriftfamilie \DIN aktiv ist und dementsprechend entweder \MPName{Dann-Teil} 
oder \MPName{Sonst-Teil} ausgeführt. 
\end{Cessation}

\begin{Cessation}
  [\Option{cdfont=\PMisc}]
  {\Option{cdfont=din}}
  'entfällt, \seeplain{\Option{cdoldfont=din}}'
\begin{Cessation}
  [\Option{cdfont=\PMisc}]
  {\Option{cdfont=nodin}}
  'entfällt, \seeplain{\Option{cdoldfont=nodin}}'
\printdeclarationlist

Die Einstellungen für Überschriften sind mit der Umstellung auf \OpenSans nicht 
mehr notwendig. Zur Verwendung der alten Schriftfamilien muss die Option 
\Option{cdoldfont} aktiviert werden.
\end{Cessation}
\end{Cessation}

\begin{Cessation}
  {\Option{clearcolor=\PBoolean}}
  <\Option{cleardoublespecialpage}>
\printdeclarationlist

Diese Option wurde zur Vereinfachung und Vereinheitlichung der 
Benutzerschnittstelle in \Option*{cleardoublespecialpage=\PMisc} integriert.
\end{Cessation}

\begin{Cessation}
  {\Macro{univln}}
\begin{Cessation}
  {\Macro{textuln|\MPName{Text}}}
\begin{Cessation}
  {\Macro{univrn}}
\begin{Cessation}
  {\Macro{texturn|\MPName{Text}}}
\begin{Cessation}
  {\Macro{univbn}}
\begin{Cessation}
  {\Macro{textubn|\MPName{Text}}}
\begin{Cessation}
  {\Macro{univxn}}
\begin{Cessation}
  {\Macro{textuxn|\MPName{Text}}}
\begin{Cessation}
  {\Macro{univls}}
\begin{Cessation}
  {\Macro{textuls|\MPName{Text}}}
\begin{Cessation}
  {\Macro{univrs}}
\begin{Cessation}
  {\Macro{texturs|\MPName{Text}}}
\begin{Cessation}
  {\Macro{univbs}}
\begin{Cessation}
  {\Macro{textubs|\MPName{Text}}}
\begin{Cessation}
  {\Macro{univxs}}
\begin{Cessation}
  {\Macro{textuxs|\MPName{Text}}}
\begin{Cessation}
  {\Macro{dinbn}}
\begin{Cessation}
  {\Macro{textdbn|\MPName{Text}}}
\printdeclarationlist

Diese veralteten Befehle zur expliziten Auswahl eines Schriftschnittes werden 
nur noch bereitgestellt, falls die Option \Option{cdoldfont} aktiviert 
wurde. Es wird auch für den Einsatz der alten Schriftfamilien empfohlen, 
stattdessen die Befehle \Macro*{cdfont} oder \Macro*{textcd|\MPName{Text}} zu 
nutzen, welche in \fullref{sec:text} dokumentiert sind.
\end{Cessation}
\end{Cessation}
\end{Cessation}
\end{Cessation}
\end{Cessation}
\end{Cessation}
\end{Cessation}
\end{Cessation}
\end{Cessation}
\end{Cessation}
\end{Cessation}
\end{Cessation}
\end{Cessation}
\end{Cessation}
\end{Cessation}
\end{Cessation}
\end{Cessation}
\end{Cessation}


\minisec{Auszeichnungen in Überschriften}

Die Überschriften aller Gliederungsebenen von \Macro*{part} bis einschließlich 
\Macro*{subsubsection} werden in Majuskeln der Schrift \DIN gesetzt, wenn dies 
mit \Option*{cdoldfont=true} respektive \Option*{cdoldfont=onlydin} 
aktiviert wurde. Hierfür wird das Paket \Package{textcase} zusammen mit den alten 
Schriftfamilien geladen und intern \Macro*(\Package{textcase}){MakeTextUppercase} 
verwendet. Sollen bestimmte Minuskeln erhalten bleiben, ist innerhalb des 
obligatorischen Argumentes eines Gliederungsbefehls 
\Macro*(\Package{textcase}){NoCaseChange} zu verwenden.
%
\begin{Example}
In einem Kapitel wird ein einzelnes Wort in Minuskeln geschrieben:
\begin{Code}
\chapter{Überschrift mit \NoCaseChange{kleinem} Wort}
\end{Code}
\end{Example}
%
Die Schrift \DIN durfte laut \CD ausschließlich mit Majuskeln verwendet werden, 
weshalb das beschriebene Vorgehen lediglich im \emph{Ausnahmefall} anzuwenden 
ist. Bei einer manuellen Nutzung der Schrift sind ebenfalls Großbuchstaben zu 
verwenden, siehe hierzu die Anmerkung für \Macro'pageref'{textdbn}.
\end{Cessations}
\index{Änderungen|!)}%
%%\end{Entity}
