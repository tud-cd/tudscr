\chapter{Versionsänderungen}
\section{Kompatibilitätseinstellungen zu früheren Versionen}
Bei der Entwicklung von \TUDScript lässt es sich nicht immer vermeiden, dass 
Verbesserungen sowie Korrekturen an den Klassen und Paketen zu Änderungen am 
Ergebnis der Ausgabe führen, insbesondere bei Umbruch und Layout. Für bereits
archivierte Dokumente, welche mit einer früheren Version erstellt wurden ist 
dies jedoch bei einer erneuten Kompilierung unter Umständen eher unerwünscht.

\begin{Declaration}[v2.03]{\Option{tudscrver=\PName{Version}}}[last]
\printdeclarationlist%
\index{Kompatibilität|!}%
%
Mit dieser Option wird es möglich, auf das (Umbruch-)Verhalten einer älteren 
respektive früheren Version von \TUDScript umzuschalten, um nach der 
Kompilierung das erwartete Ergebnis zu erhalten. Neue Möglichkeiten, die sich 
nicht auf den Umbruch oder das Layout auswirken, sind auch für den Fall 
verfügbar, dass per Option die Kompatibilität zu einer älteren Version 
ausgewählt wurde. 

Bei der Angabe einer unbekannten Version als Wert wird eine Warnung ausgegeben 
und \Option{tudscrver=first} angenommen. Mit \Option{tudscrver=last} wird die 
jeweils aktuell verfügbare Version ausgewählt und folglich auf die zukünftige 
Kompatibilität des Dokumentes zu der aktuell genutzten Version verzichtet. 
Dieses Verhalten entspricht der Voreinstellung. Es ist zu beachten, dass die 
Nutzung von \Option{tudscrver} nur als Klassenoption möglich ist.
%
\begin{values}{\Option{tudscrver}}
\item[\PValue{first}/\PValue{2.02}]
  \ChangedAt{%
    v2.03:Satzspiegel im \CD geändert, Logo von \DDC im Fußbereich 
    wird ohne vergrößerten Seitenrand verwendet
  }
  Der Satzspiegel im Layout des \CDs (\seeref{\Option{cdgeometry}}) wurde in 
  der Version~v2.03 leicht geändert. Der obere Seitenrand wurde verkleinert, 
  der untere im gleichen Maße vergrößert. Der verfügbare Textbereich ist 
  folglich identisch. Bei der Aktivierung des \DDC-Logos im Fußbereich der 
  Seite (\seeref{\Option{ddcfoot}}) wird im Gegensatz zur Version~v2.02 der 
  gleiche Satzspiegel genutzt. Mit \Option{tudscrver=first} kann dieses 
  Verhalten deaktiviert werden.
\item[\PValue{2.03}]
  \index{Leerraum}%
  \index{Schriftgröße}%
  \ChangedAt{v2.04:Vertikaler Leerraum abhängig von verwendeter Schriftgröße}
  Seit der Version~v2.04 werden mehrere Längen in Abhängigkeit der gewählten 
  Schriftgröße definiert. Mit der Wahl \Option{tudscrver=2.03} wird diese 
  Funktionalität deaktiviert, wobei hierfür lediglich die \TUDScript-Option 
  \Option{relspacing=false} aktiviert wird. 
\item[\PValue{2.04}]
  \index{Satzspiegel}%
  \ChangedAt{%
    v2.05:Satzspiegeleinstellungen für die jeweilige ISO/DIN"~Klasse 
    des Papierformates identisch%
  }
  Mit der Version~v2.05 werden die vorgegebenen Einstellungen zum Satzspiegel 
  anhand der B"~ISO/DIN"~Reihe vorgenommen. Damit sind für alle Papierformate 
  einer spezifischen ISO/DIN"~Klasse die Seitenränder identisch. Mit der Wahl 
  \Option{tudscrver=2.04} ist der Satzspiegel von der A"~ISO/DIN"~Reihe 
  abhängig, sodass die B- und C"~Papierformate der gleichen Klasse größere 
  Seitenränder erhalten, als die D- und A"~Formate.
\item[\PValue{2.05}]
  \ChangedAt{v2.06:Griechische Buchstaben standardmäßig kursiv}
  Mit dem Wechsel der Hausschrift zu \OpenSans ergaben sich einige Änderungen 
  bezüglich der Erscheinung des \CDs. Mit dieser Kompatibilitätseinstellung 
  werden die alten Schriftfamilien \Univers und \DIN 
  (\Option{cdoldfont=true}, \Option{headings=light}, \Option{ttfont=lmodern})
  aktiviert. Weiterhin wirkt sich für diese Schriftfamilien im Mathematikmodus 
  die Option \Option{slantedgreek=true} lediglich auf die griechischen 
  Majuskeln aus.
\item[\PValue{2.06}]
  Dies ist Kompatibilitätseinstellung für \TUDScript~\vTUDScript{} und wird für 
  zukünftige Änderungen bereits vorgehalten. Soll ein mit der momentan 
  aktuellen Version erzeugtes Dokument auch mit einer späteren Version von 
  \TUDScript nach einem \hologo{LaTeX}"=Lauf das gleiche Ausgabeergebnis 
  liefern, muss dies mit \Option{tudscrver=2.06} angegeben werden.
\item[\PValue{last}]
  Es werden keine Kompatibilitätseinstellungen für das Dokument vorgenommen. 
  Mit einer späteren Version von \TUDScript kann ein anderes Umbruchverhalten 
  innerhalb des Dokumentes auftreten. Dies ist die Standardeinstellung.
\end{values}
\end{Declaration}



\section{Veraltete Optionen und Befehle in \TUDScript}
\index{Kompatibilität}%
\label{sec:obsolete}%
%
Einige Optionen und Befehle waren während der Weiterentwicklung von \TUDScript
in ihrer ursprünglichen Form nicht mehr umsetzbar oder wurden~-- unter anderem 
aus Gründen der Kompatibilität zu anderen Paketen~-- schlichtweg verworfen. 
Dennoch besteht für die meisten entfallenen Direktiven eine Möglichkeit, deren 
Funktionalität ohne größere Aufwände mit \TUDScript in der aktuellen Version 
\vTUDScript{} darzustellen. Ist dies der Fall, wird hier entsprechend kurz 
darauf hingewiesen.

\newcommand*\ChangesTo[1]{%
  Änderungen für \TUDScript~#1%
  \ChangedAt*{#1:Änderungen gegenüber der vorhergehenden Version}%
}

\subsection{\ChangesTo{v2.00}}
\begin{Obsolete}{v2.00}[\Option{cd=\PSet}]{\Option{cd=alternative}}
\begin{Obsolete}{v2.00}[\Option{cdtitle=\PSet}]{\Option{cdtitle=alternative}}
\begin{Obsolete}{v2.00}{\Length{titlecolwidth}}
\begin{Obsolete}{v2.00}{\Term{authortext}}
\printobsoletelist%
%
Die alternative Titelseite ist komplett aus dem \TUDScript-Bundle entfernt 
worden. Dementsprechend entfallen auch die dazugehörigen Optionen sowie Länge 
und Bezeichner.
\end{Obsolete}
\end{Obsolete}
\end{Obsolete}
\end{Obsolete}

\begin{Obsolete}{v2.00:\Option{cd}}{\Option{color=\PBoolean}}
\printobsoletelist%
%
Die Einstellungen der farbigen Ausprägung des Dokumentes erfolgt über die 
Option \Option*{cd}.
\end{Obsolete}

\begin{Obsolete}{v2.00:\Option{cdfont}}{\Option{tudfonts=\PBoolean}}
\printobsoletelist%
%
Die Option zur Schrifteinstellung ist wesentlich erweitert worden. Aus Gründen 
der Konsistenz wurde diese umbenannt.
\end{Obsolete}

\begin{Obsolete}{v2.00:\Option{cdfoot}}{\Option{tudfoot=\PBoolean}}
\printobsoletelist%
%
Ebenso wurde die Option \Option*{tudfoot} umbenannt, um dem Namensschema der 
restlichen Optionen von \TUDScript zu entsprechen.
\end{Obsolete}

\begin{Obsolete}{v2.00}{\Option{headfoot=\PSet}}{%
  \seeref{\KOMAScript-Optionen \Option*{headinclude} und \Option*{footinclude}}%
}
\printobsoletelist%
%
Diese Option war für \TUDScript in der \emph{Version~v1.0} notwendig, um die 
parallele Verwendung der beiden Pakete \Package*{typearea} und 
\Package*{geometry} zu ermöglichen. Die Erstellung des Satzspiegels wurde 
komplett überarbeitet. Mittlerweile werden an das Paket \Package*{geometry} die 
Einstellungen für die \KOMAScript"=Optionen \Option*{headinclude} und 
\Option*{footinclude} direkt weitergereicht, sodass die Option 
\Option*{headfoot} nicht mehr notwendig ist und deshalb entfernt wurde.
\end{Obsolete}

\begin{Obsolete}{v2.00:\Option{cleardoublespecialpage}}{%
  \Option{partclear=\PBoolean}%
}
\begin{Obsolete}{v2.00:\Option{cleardoublespecialpage}}{%
  \Option{chapterclear=\PBoolean}%
}
\printobsoletelist%
%
Beide Optionen sind in der neuen Option \Option*{cleardoublespecialpage} 
aufgegangen, womit ein konsistentes Layout erreicht wird. Die ursprünglichen 
Optionen entfallen. 
\end{Obsolete}
\end{Obsolete}

\begin{Obsolete}{v2.00:\Option{abstract}}{\Option{abstracttotoc=\PBoolean}}
\begin{Obsolete}{v2.00:\Option{abstract}}{\Option{abstractdouble=\PBoolean}}
\printobsoletelist%
%
Beide Optionen wurden in die Option \Option*{abstract} integriert und sind 
deshalb überflüssig.
\end{Obsolete}
\end{Obsolete}

\begin{Obsolete}{v2.00:\Macro{headlogo}}{%
  \Macro{logofile}[\Parameter{Dateiname}]%
}
\printobsoletelist%
%
Der Befehl \Macro*{logofile} wurde in \Macro*{headlogo} umbenannt, wobei die 
Funktionalität weiterhin bestehen bleibt.
\end{Obsolete}

\begin{Obsolete}{v2.00:\Option{tudbookmarks}}{\Option{bookmarks=\PBoolean}}
\printobsoletelist%
%
Die Option wurde umbenannt, um Überschneidungen mit \Package*{hyperref} zu 
vermeiden.
\end{Obsolete}

\begin{Obsolete}{v2.00}{\Length{signatureheight}}
\printobsoletelist%
%
Die Höhe für die Zeile der Unterschriften wurde dehnbar gestaltet, eine etwaige 
Anpassung durch den Anwender ist nicht vonnöten.
\end{Obsolete}

\begin{Obsolete}{v2.00:\Macro{titledelimiter}}{\Term{titlecoldelim}}%
\printobsoletelist%
%
Das Trennzeichen für Bezeichnungen beziehungsweise beschreibende Texte und dem 
eigentlichen Feld auf der Titelseite ist nicht mehr sprachabhängig und wurde 
umbenannt.
\end{Obsolete}

\begin{Obsolete}{v2.00:\Macro{declaration}}{\Macro{confirmationandrestriction}}
\begin{Obsolete}{v2.00:\Macro{declaration}}{\Macro{restrictionandconfirmation}}
\printobsoletelist%
%
Die beiden Befehle entfallen, stattdessen sollte entweder der Befehl 
\Macro*{declaration} oder die Umgebung \Environment*{declarations} zusammen mit 
den Befehlen \Macro*{confirmation} und \Macro*{blocking} verwendet werden, 
wobei sich diese in der Umgebung in beliebiger Reihenfolge anordnen lassen.
\end{Obsolete}
\end{Obsolete}

\begin{Obsolete}{v2.00:\Macro{place}}{\Macro{location}[\Parameter{Ort}]}
\printobsoletelist%
%
In Anlehnung an andere \hologo{LaTeX}"=Pakete und "~Klassen wurde 
\Macro*{location} in \Macro*{place} umbenannt.
\end{Obsolete}

\minisec{\taskname}
\begin{Bundle}{\Package{tudscrsupervisor}}
Die Umgebung für die Erstellung einer Aufgabenstellung für eine 
wissenschaftliche Arbeit wurde in das Paket \Package{tudscrsupervisor} 
ausgelagert. Dieses muss für die Verwendung der Umgebung \Environment*{task} 
und der daraus abgeleiteten standardisierten Form zwingend geladen werden.

\begin{Obsolete}{v2.00:\Environment{task}}{\Option{cdtask=\PSet}}
\begin{Obsolete}{v2.00}{\Option{taskcompact=\PBoolean}}
\begin{Obsolete}{v2.00}{\Length{taskcolwidth}}
\printobsoletelist%
%
Die Klassenoption \Option*{cdtask} ist komplett entfernt worden, alle 
Einstellungen, erfolgen direkt über das optionale Argument der Umgebung 
\Environment*{task}. Die Variante eines kompakten Kopfes mit der Option 
\Option*{taskcompact} wird nicht mehr bereitgestellt. Die Möglichkeit zur 
manuellen Festlegung der Spaltenbreite für den Kopf der Aufgabenstellung mit 
\Length*{taskcolwidth} wurde aufgrund der verbesserten automatischen Berechnung 
entfernt.
\end{Obsolete}
\end{Obsolete}
\end{Obsolete}

\begin{Obsolete}{v2.00:\Macro{taskform}}{%
  \Macro{tasks}[\Parameter{Ziele}\Parameter{Schwerpunkte}]%
}
\begin{Obsolete}{v2.00:\Term{focusname}}{\Term{focustext}}
\begin{Obsolete}{v2.00:\Term{objectivesname}}{\Term{objectivestext}}
\printobsoletelist%
%
Der Befehl \Macro*{tasks} wurde in \Macro*{taskform} umbenannt und in der 
Funktionalität erweitert. Die darin verendeten Bezeichner wurden ebenfalls 
leicht abgewandelt.
\end{Obsolete}
\end{Obsolete}
\end{Obsolete}

\begin{Obsolete}{v2.00:\Macro{matriculationnumber}}{%
  \Macro{studentid}[\Parameter{Matrikelnummer}]%
}
\begin{Obsolete}{v2.00:\Macro{matriculationyear}}{%
  \Macro{enrolmentyear}[\Parameter{Immatrikulationsjahr}]%
}
\begin{Obsolete}{v2.00:\Macro{date}}{\Macro{submissiondate}[\Parameter{Datum}]}
\begin{Obsolete}{v2.00:\Macro{dateofbirth}}{%
  \Macro{birthday}[\Parameter{Geburtsdatum}]%
}
\begin{Obsolete}{v2.00:\Macro{placeofbirth}}{%
  \Macro{birthplace}[\Parameter{Geburtsort}]%
}
\begin{Obsolete}{v2.00:\Macro{issuedate}}{%
  \Macro{startdate}[\Parameter{Ausgabedatum}]%
}
\printobsoletelist%
%
Alle Befehle wurden umbenannt und sind jetzt neben der Aufgabenstellung auch 
für die Titelseite im \CD nutzbar.
\end{Obsolete}
\end{Obsolete}
\end{Obsolete}
\end{Obsolete}
\end{Obsolete}
\end{Obsolete}

\begin{Obsolete}{v2.00:\Term{matriculationnumbername}}{\Term{studentidname}}
\begin{Obsolete}{v2.00:\Term{matriculationyearname}}{\Term{enrolmentname}}
\begin{Obsolete}{v2.00:\Term{datetext}}{\Term{submissiontext}}
\begin{Obsolete}{v2.00:\Term{dateofbirthtext}}{\Term{birthdaytext}}
\begin{Obsolete}{v2.00:\Term{placeofbirthtext}}{\Term{birthplacetext}}
\begin{Obsolete}{v2.00:\Term{supervisorothername}}{\Term{supervisorIIname}}
\begin{Obsolete}{v2.00:\Term{defensedatetext}}{\Term{defensetext}}
\begin{Obsolete}{v2.00:\Term{issuedatetext}}{\Term{starttext}}
\begin{Obsolete}{v2.00:\Term{duedatetext}}{\Term{duetext}}
\printobsoletelist%
%
Die Bezeichner wurden in Anlehnung an die dazugehörigen Befehlsnamen umbenannt.
\end{Obsolete}
\end{Obsolete}
\end{Obsolete}
\end{Obsolete}
\end{Obsolete}
\end{Obsolete}
\end{Obsolete}
\end{Obsolete}
\end{Obsolete}
\end{Bundle}


\subsection{\ChangesTo{v2.02}}
\begin{Obsolete}{v2.02:\Option{pageheadingsvskip}}{\Length{chapterheadingvskip}}
\printobsoletelist%
%
Die vertikale Positionierung von Überschriften wurde aufgeteilt. Zum einen kann 
diese für Titel"~, Teile- und Kapitelseiten (\Option*{chapterpage=true}) über 
die Option \Option*{pageheadingsvskip} geändert werden. Für Kapitelüberschriften
(\Option*{chapterpage=false}) sowie den Titelkopf (\Option*{titlepage=false}) 
kann dies unabhängig davon mit \Option*{headingsvskip} erfolgen.
\end{Obsolete}

\begin{Obsolete}{v2.02:\Macro{graduation}}{%
  \Macro{degree}[\OParameter{Abk.}\Parameter{Grad}]%
}
\begin{Obsolete}{v2.02:\Term{graduationtext}}{\Term{degreetext}}
\printobsoletelist%
%
Der Befehl wurde zur Erhöhung der Kompatibilität mit anderen Paketen umbenannt, 
der dazugehörige Bezeichner dahingehend angepasst.
\end{Obsolete}
\end{Obsolete}

\begin{Obsolete}{v2.02:\Macro{blocking}}{%
  \Macro{restriction}[\OLParameter{Firma}]%
}
\begin{Obsolete}{v2.02:\Term{blockingname}}{\Term{restrictionname}}
\begin{Obsolete}{v2.02:\Term{blockingtext}}{\Term{restrictiontext}}
\printobsoletelist%
%
Der Befehl wurde zur Erhöhung der Kompatibilität mit anderen Paketen umbenannt, 
die dazugehörigen Bezeichner dahingehend angepasst.
\end{Obsolete}
\end{Obsolete}
\end{Obsolete}

\begin{Obsolete}{}{\Environment{tudpage}[\OLParameter{Sprache}]}
\begin{Obsolete}{v2.02:\Key{\Environment{tudpage}}{pagestyle}}{%
  \Key{\Environment{tudpage}}{head=\PSet}
}
\begin{Obsolete}{v2.02:\Key{\Environment{tudpage}}{pagestyle}}{%
  \Key{\Environment{tudpage}}{foot=\PSet}
}
\printobsoletelist%
%
Diese beiden Parameter der Umgebung \Environment*{tudpage} wurden in ihrer 
Funktionalität durch den Parameter \Key*{\Environment{tudpage}}{pagestyle} 
ersetzt.
\end{Obsolete}
\end{Obsolete}
\end{Obsolete}



\minisec{Änderungen im Paket \Package{tudscrsupervisor}}
%
\begin{Bundle}{\Package{tudscrsupervisor}}
Im Paket \Package{tudscrsupervisor} gab es ein paar kleinere Anpassungen.
\begin{Obsolete}{v2.02:\Macro{discipline}}{%
  \Macro{branch}[\Parameter{Studienrichtung}]%
}
\begin{Obsolete}{v2.02:\Term{disciplinename}}{\Term{branchname}}
\printobsoletelist%
%
Für die Aufgabenstellung (\Environment*{task} und \Macro*{taskform}) wurden 
dieser Befehl sowie der dazugehörige Bezeichner umbenannt.
\end{Obsolete}
\end{Obsolete}

\begin{Obsolete}{v2.02:\Macro{contactperson}}{%
  \Macro{contact}[\Parameter{Kontaktperson(en)}]%
}
\begin{Obsolete}{v2.02:\Term{contactpersonname}}{\Term{contactname}}
\begin{Obsolete}{v2.02:\Macro{telephone}}{%
  \Macro{phone}[\Parameter{Telefonnummer}]%
}
\begin{Obsolete}{v2.02:\Macro{emailaddress}}{%
  \Macro{email}[\Parameter{E-Mail-Adresse}]%
}
\printobsoletelist%
%
Alle genannten Befehle und Bezeichner wurden für den \noticename{} umbenannt.
\end{Obsolete}
\end{Obsolete}
\end{Obsolete}
\end{Obsolete}
\end{Bundle}


\subsection{\ChangesTo{v2.03}}
\begin{Obsolete}{v2.03:\Option{cdgeometry}}{\Option{geometry=\PBoolean}}
\printobsoletelist%
%
Die Option \Option*{geometry} wurde zur Konsistenz sowie dem Vermeiden 
eines möglichen Konfliktes mit einer späteren \KOMAScript-Version umbenannt. 
Die Funktionalität bleibt bestehen.
\end{Obsolete}

\begin{Obsolete}{v2.03:\Option{cdfont}}{\Option{cdfonts=\PBoolean}}
\begin{Obsolete}{v2.03:\Option{cdfont}}{\Option{din=\PBoolean}}
\printobsoletelist%
%
Die Option \Option*{cdfont} wurde erweitert und fungiert als zentrale 
Schnittstelle zur Schrifteinstellung. 
\end{Obsolete}
\end{Obsolete}

\begin{Obsolete}{v2.03:\Option{cdmath}}{\Option{sansmath=\PBoolean}}
\printobsoletelist%
%
Die Option \Option*{sansmath} wurde aus Gründen der Konsistenz umbenannt. 
Zusätzlich wurde die Funktionalität erweitert.
\end{Obsolete}

\begin{Obsolete}{v2.03:\Option{cdhead}}{\Option{barfont=\PSet}}
\begin{Obsolete}{v2.03:\Option{cdhead}}{\Option{widehead=\PBoolean}}
\printobsoletelist%
%
Die Optionen \Option*{barfont} und \Option*{widehead} wurden in der Option 
\Option*{cdhead} zusammengefasst.
\end{Obsolete}
\end{Obsolete}

\begin{Obsolete}{}{\Environment{tudpage}[\OLParameter{Sprache}]}
\begin{Obsolete}{v2.03}{\Key{\Environment{tudpage}}{color=\PName{Farbe}}}
\printobsoletelist%
%
Der Parameter \Key*{\Environment{tudpage}}{color=\PValueName{Farbe}} der 
\Environment*{tudpage}"=Umgebung wurde ersatzlos entfernt.
\end{Obsolete}
\end{Obsolete}


\subsection{\ChangesTo{v2.04}}
\begin{Obsolete}{v2.04}{\Option{fontspec=\PBoolean}}%
\printobsoletelist%
%
Anstatt die Option \Option*{fontspec} zu aktivieren, kann einfach das Paket 
\Package{fontspec} in der Dokumentpräambel geladen werden. Dadurch können 
anschließend zusätzliche Pakete genutzt werden, welche auf die Verwendung von 
\Package{fontspec} angewiesen sind. Sollte die Option \Option*{fontspec} 
dennoch genutzt werden, müssen alle auf das Paket \Package{fontspec} 
aufbauenden Einstellungen durch den Anwender mit 
\Macro*{AfterPackage}[\PParameter{fontspec}\PParameter{\dots}] 
verzögert werden. In \fullref{sec:fonts} sind weitere Hinweise zur Verwendung 
des Paketes \Package{fontspec} zu finden.
\end{Obsolete}


\subsection{\ChangesTo{v2.05}}
\begin{Obsolete}{v2.05:\Option{pageheadingsvskip}}{\Length{pageheadingsvskip}}
\begin{Obsolete}{v2.05:\Option{headingsvskip}}{\Length{headingsvskip}}
\printobsoletelist%
%
Die vertikale Positionierung von speziellen Überschriften erfolgt nicht mehr 
über die Längen \Length*{headingsvskip} und \Length*{pageheadingsvskip} sondern 
über die Optionen \Option*{headingsvskip} sowie \Option*{pageheadingsvskip}.
\end{Obsolete}
\end{Obsolete}


\begin{Obsolete}{v2.05:\Option{footlogoheight}}{\Length{footlogoheight}}%
\printobsoletelist%
%
Auch die Höhe der Logos im Fußbereich der \PageStyle*{tudheadings}"=Seitenstile 
wird von nun an mit der Option \Option*{footlogoheight} und nicht mehr mit der 
Länge \Length*{footlogoheight} festgelegt.
\end{Obsolete}


\subsection{\ChangesTo{v2.06}}
\begin{Declaration}[v2.06]{\Option{cdoldfont}}[false]
\printdeclarationlist%
%
Mit der Version~v2.06 wird standardmäßig \OpenSans als Hausschrift verwendet. 
Um jedoch weiterhin ältere Dokumente mit den Schriften \Univers und \DIN 
erzeugen zu können, wird diese Option bereitgestellt.
\Attention{%
  Diese kann ausschließlich als Klassenoption~-- oder für die Pakete 
  \Package*{tudscrfonts} und \Package*{fix-tudscrfonts} als Paketoption~-- 
  genutzt werden.
} Eine späte Optionenwahl mit \Macro*{TUDoption} oder \Macro*{TUDoptions} ist 
nicht möglich. Wurden mit \Option{cdoldfont=true} die alten Schriftfamilien 
aktiviert, kann jedoch weiterhin die Option \Option*{cdfont} genutzt werden.
%
\begin{values}{\Option{cdoldfont}}
\item[false]
  Das Verhalten ist äquivalent zu \Option*{cdfont=false}, die Hausschrift ist 
  nicht aktiv.
\item[true]
  Es werden die alten Hausschriften \Univers für den Fließtext sowie \DIN für 
  Überschriften der obersten Gliederungsebenen bis einschließlich 
  \Macro*{subsubsection} genutzt. Die Schriftstärke lässt sich mit 
  \Option*{cdfont=true} respektive \Option*{cdfont=heavy} anpassen.
\end{values}
%
Für die \TUDScript-Klassen sowie die vom Paket \Package*{fix-tudscrfonts} 
unterstützten Dokumentklassen kann die für die Gliederungsebenen verwendete 
Schriftart angepasst werden.
%
\begin{values}{\Option{cdoldfont}}
\item[nodin]
  Für Überschriften wird \Univers anstatt \DIN verwendet.
\item[din]
  Mit dieser Einstellung wird die Schrift \DIN in den Überschriften verwendet. 
\item[onlydin]
  Hiermit werden nur die Überschriften in \DIN gesetzt, für den Fließtext kommt 
  nicht \Univers sondern die \hologo{LaTeX}"=Standardschriften respektive die 
  eines geladenen Schriftpaketes zum Einsatz.
\end{values}
\end{Declaration}

\begin{Obsolete}{v2.06}{%
  \Macro{ifdin}[\Parameter{Dann-Teil}\Parameter{Sonst-Teil}]
}%
\printobsoletelist%
%
Der Befehl \Macro*{ifdin} prüft, ob die Schriftfamilie \DIN aktiv ist und führt 
in diesem Fall \Parameter{Dann-Teil} aus, andernfalls \Parameter{Sonst-Teil}. 
\end{Obsolete}

\begin{Obsolete}{v2.06}[\Option{cdfont=\PSet}]{\Option{cdfont=din}}{%
  entfällt, \seeref{\Option{cdoldfont=din}}%
}
\begin{Obsolete}{v2.06}[\Option{cdfont=\PSet}]{\Option{cdfont=nodin}}{%
  entfällt, \seeref{\Option{cdoldfont=nodin}}%
}
\printobsoletelist%
%
Die Einstellungen für Überschriften sind mit der Umstellung auf \OpenSans nicht 
mehr notwendig. Für die Verwendung der alten Schriftfamilien \Univers und \DIN 
muss die Option \Option{cdoldfont} aktiviert werden.
\end{Obsolete}
\end{Obsolete}

\begin{Obsolete}{v2.06:\Option{cleardoublespecialpage}}{%
  \Option{clearcolor=\PBoolean}%
}
\printobsoletelist%
%
Die Option \Option*{clearcolor=\PBoolean} wurde zur Vereinheitlichung der 
Benutzerschnittstelle in \Option*{cleardoublespecialpage=\PSet} integriert.
\end{Obsolete}

\begin{Obsolete}{v2.06}{\Macro{univln}}
\begin{Obsolete}{v2.06}{\Macro{textuln}[\Parameter{Text}]}
\begin{Obsolete}{v2.06}{\Macro{univrn}}
\begin{Obsolete}{v2.06}{\Macro{texturn}[\Parameter{Text}]}
\begin{Obsolete}{v2.06}{\Macro{univbn}}
\begin{Obsolete}{v2.06}{\Macro{textubn}[\Parameter{Text}]}
\begin{Obsolete}{v2.06}{\Macro{univxn}}
\begin{Obsolete}{v2.06}{\Macro{textuxn}[\Parameter{Text}]}
\begin{Obsolete}{v2.06}{\Macro{univls}}
\begin{Obsolete}{v2.06}{\Macro{textuls}[\Parameter{Text}]}
\begin{Obsolete}{v2.06}{\Macro{univrs}}
\begin{Obsolete}{v2.06}{\Macro{texturs}[\Parameter{Text}]}
\begin{Obsolete}{v2.06}{\Macro{univbs}}
\begin{Obsolete}{v2.06}{\Macro{textubs}[\Parameter{Text}]}
\begin{Obsolete}{v2.06}{\Macro{univxs}}
\begin{Obsolete}{v2.06}{\Macro{textuxs}[\Parameter{Text}]}
\begin{Obsolete}{v2.06}{\Macro{dinbn}}
\begin{Obsolete}{v2.06}{\Macro{textdbn}[\Parameter{Text}]}
\printobsoletelist%
%
Wird die Option \Option{cdoldfont} nicht aktiviert, werden auch die Befehle zur 
expliziten Auswahl eines Schriftschnittes nicht mehr bereitgestellt. 
Stattdessen können \Macro*{cdfont} oder \Macro*{textcd}[\Parameter{Text}] 
genutzt werden, welche in \autoref{sec:text} zu finden sind.
\end{Obsolete}
\end{Obsolete}
\end{Obsolete}
\end{Obsolete}
\end{Obsolete}
\end{Obsolete}
\end{Obsolete}
\end{Obsolete}
\end{Obsolete}
\end{Obsolete}
\end{Obsolete}
\end{Obsolete}
\end{Obsolete}
\end{Obsolete}
\end{Obsolete}
\end{Obsolete}
\end{Obsolete}
\end{Obsolete}


\minisec{Auszeichnungen in Überschriften}
%
Für alle Gliederungsebenen bis einschließlich \Macro*{subsubsection} werden 
die Überschriften in Großbuchstaben der Schrift \DIN gesetzt, wenn diese mit 
den entsprechenden Einstellungen (\Option*{cdoldfont=true/onyldin}) aktiviert 
wurde. Hierfür wird intern \Macro*{MakeTextUppercase}(\Package{textcase})'none'
aus dem Paket \Package{textcase} genutzt, welches zusammen mit den alten 
Schriftfamilien geladen wird. Sollen bestimmte Kleinbuchstaben erhalten 
bleiben, ist \Macro*{NoCaseChange}(\Package{textcase})'none' zu verwenden.
%
\begin{Example}
In einem Kapitel wird ein einzelnes Wort in Minuskeln geschrieben:
\begin{Code}[escapechar=§]
\chapter{§Ü§berschrift mit \NoCaseChange{kleinem} Wort}
\end{Code}
\end{Example}
%
Die Schrift \DIN durfte laut \CD nur mit Majuskeln (Großbuchstaben) verwendet 
werden, weshalb das beschriebene Vorgehen lediglich im \emph{Ausnahmefall} 
anzuwenden ist. Die manuellen Nutzung sollte mit 
\Macro*{MakeTextUppercase}[%
  \PParameter{\Macro{textdbn}[\Parameter{Text}]}%
](\Package{textcase})'none' geschehen.
