\RequirePackage[ngerman=ngerman-x-latest]{hyphsubst}
\documentclass[english,ngerman,final,parskip=no]{tudscrman}
\usepackage{selinput}
\SelectInputMappings{adieresis={ä},germandbls={ß}}
\usepackage[T1]{fontenc}
\lstset{%
  inputencoding=utf8,extendedchars=true,
  literate=%
    {ä}{{\"a}}1 {ö}{{\"o}}1 {ü}{{\"u}}1
    {Ä}{{\"A}}1 {Ö}{{\"O}}1 {Ü}{{\"U}}1
    {~}{{\textasciitilde}}1 {ß}{{\ss}}1
}
\begin{document}
Die meisten Anwender der \TUDScript-Klassen sind Studenten oder angehörige der 
\TnUD, die ihre ersten Schritte mit \hologo{LaTeXe} beim Verfassen einer 
wissenschaftlichen Arbeit oder ähnlichem machen. Während der Einstiegsphase in 
\hologo{LaTeXe} kann ein Anfänger sehr schnell aufgrund der großen Anzahl an 
empfohlenen Pakete sowie der teilweise diametral zueinander stehenden Hinweise 
überfordert sein. Mit dem Tutorial \Tutorial{treatise} soll versucht werden, 
ein wenig Licht ins Dunkel zu bringen. Es erhebt jedoch keinerlei Anspruch, 
vollständig oder perfekt zu sein. Einige der darin vorgestellten Möglichkeiten 
lassen sich mit Sicherheit auch anders, einfacher und/oder besser lösen. 
Dennoch ist es gerade für Neulinge~-- vielleicht auch für den einen oder 
anderen \hologo{LaTeX}"=Veteran~-- als Leitfaden für die Erstellung einer 
wissenschaftlichen Arbeit gedacht.
%\include{tudscr_examples}
%\include{tudscr_installation}
%\include{tudscr_index}
\end{document}