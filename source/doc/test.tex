%\PassOptionsToClass{ToDo=no}{tudscrman3}
%\includeonly{%
%%  tudscr-preface,
%%  tudscr-introduction,
%  tudscr-mainclasses,
%%  tudscr-poster,
%%  tudscr-bundle,
%%  tudscr-examples,
%%  tudscr-packages,
%%  tudscr-hints,
%%  tudscr-installation,
%%  tudscr-obsolete,
%%  tudscr-additional,
%  tudscr-index,
%}
%%
%  TUD-Script -- Corporate Design of Technische Universitaet Dresden
% ----------------------------------------------------------------------------
%
%  Copyright (C) Falk Hanisch <hanisch.latex@outlook.com>, 2012-2017
%
% ----------------------------------------------------------------------------
%
%  This work may be distributed and/or modified under the conditions of the
%  LaTeX Project Public License, version 1.3c of the license. The latest
%  version of this license is in http://www.latex-project.org/lppl.txt and
%  version 1.3c or later is part of all distributions of LaTeX 2005/12/01
%  or later and of this work. This work has the LPPL maintenance status
%  "author-maintained". The current maintainer and author of this work
%  is Falk Hanisch.
%
% ----------------------------------------------------------------------------
%
%  Dieses Werk darf nach den Bedingungen der LaTeX Project Public Lizenz
%  in der Version 1.3c, verteilt und/oder veraendert werden. Die aktuelle
%  Version dieser Lizenz ist http://www.latex-project.org/lppl.txt und
%  Version 1.3c oder spaeter ist Teil aller Verteilungen von LaTeX 2005/12/01
%  oder spaeter und dieses Werks. Dieses Werk hat den LPPL-Verwaltungs-Status
%  "author-maintained", wird somit allein durch den Autor verwaltet. Der
%  aktuelle Verwalter und Autor dieses Werkes ist Falk Hanisch.
%
% ----------------------------------------------------------------------------
\input docstrip.tex
\begingroup
\catcode`\%=12 \catcode`\*=14
\gdef\processLineX%#1{*
  \ifcase\ifx%#10\else
         \ifx<#11\else
         \ifx!#12\else 3\fi\fi\fi\relax
    \expandafter\putMetaComment\or
    \expandafter\checkOption\or
    \expandafter\expandVariable\or
    \expandafter\removeComment\fi
  #1}
\endgroup
\def\expandVariable!#1\endLine{%
  \advance\codeLinesPassed\@ne
  \maybeMsg{x}%
  \edef\inLine{\csname #1\endcsname}%
  \let\do\putline@do
  \activefiles
}
\@@input tudscr-version.dtx
\keepsilent
\preamble

TUD-Script -- Corporate Design of Technische Universitaet Dresden

\TUD@Version

Copyright (C) Falk Hanisch <hanisch.latex@outlook.com>, 2012-2017

This file was generated from file(s) of the TUD-Script bundle.
----------------------------------------------------------------------------

This work may be distributed and/or modified under the conditions
of the LaTeX Project Public License, version 1.3c of the license.
The latest version of this license is in
    http://www.latex-project.org/lppl.txt
and version 1.3c or later is part of all distributions of
LaTeX 2005/12/01 or later and of this work.

This work has the LPPL maintenance status "author-maintained".

The current maintainer and author of this work is Falk Hanisch.

\endpreamble
\askforoverwritefalse
\usedir{tex/latex/tudscr}
\generate{%
  \usepreamble\defaultpreamble%
  \file{tudscrbase.sty}{%
    \from{tudscr-version.dtx}{package,base}%
    \from{tudscr-base.dtx}{package,base}%
  }
  \file{tudscrbook.cls}{%
    \from{tudscr-version.dtx}{class,book}%
    \from{tudscr-base.dtx}{class,book,load}%
    \from{tudscr-fonts.dtx}{class,book,option}%
    \from{tudscr-area.dtx}{class,book,option}%
    \from{tudscr-pagestyle.dtx}{class,book,option}%
    \from{tudscr-layout.dtx}{class,book,option}%
    \from{tudscr-title.dtx}{class,book,option}%
    \from{tudscr-frontmatter.dtx}{class,book,option}%
    \from{tudscr-comp.dtx}{class,book,option}%
    \from{tudscr-misc.dtx}{class,book,option}%
    \from{tudscr-misc.dtx}{class,book,execute}%
    \from{tudscr-base.dtx}{class,book,body}%
    \from{tudscr-fonts.dtx}{class,book,body}%
    \from{tudscr-fields.dtx}{class,book,body}%
    \from{tudscr-locale.dtx}{class,book,body}%
    \from{tudscr-area.dtx}{class,book,body}%
    \from{tudscr-pagestyle.dtx}{class,book,body}%
    \from{tudscr-layout.dtx}{class,book,body}%
    \from{tudscr-title.dtx}{class,book,body}%
    \from{tudscr-frontmatter.dtx}{class,book,body}%
    \from{tudscr-comp.dtx}{class,book,body}%
    \from{tudscr-misc.dtx}{class,book,body}%
  }
  \file{tudscrreprt.cls}{%
    \from{tudscr-version.dtx}{class,report}%
    \from{tudscr-base.dtx}{class,report,load}%
    \from{tudscr-fonts.dtx}{class,report,option}%
    \from{tudscr-area.dtx}{class,report,option}%
    \from{tudscr-pagestyle.dtx}{class,report,option}%
    \from{tudscr-layout.dtx}{class,report,option}%
    \from{tudscr-title.dtx}{class,report,option}%
    \from{tudscr-frontmatter.dtx}{class,report,option}%
    \from{tudscr-comp.dtx}{class,report,option}%
    \from{tudscr-misc.dtx}{class,report,option}%
    \from{tudscr-misc.dtx}{class,report,execute}%
    \from{tudscr-base.dtx}{class,report,body}%
    \from{tudscr-fonts.dtx}{class,report,body}%
    \from{tudscr-fields.dtx}{class,report,body}%
    \from{tudscr-locale.dtx}{class,report,body}%
    \from{tudscr-area.dtx}{class,report,body}%
    \from{tudscr-pagestyle.dtx}{class,report,body}%
    \from{tudscr-layout.dtx}{class,report,body}%
    \from{tudscr-title.dtx}{class,report,body}%
    \from{tudscr-frontmatter.dtx}{class,report,body}%
    \from{tudscr-comp.dtx}{class,report,body}%
    \from{tudscr-misc.dtx}{class,report,body}%
  }
  \file{tudscrartcl.cls}{%
    \from{tudscr-version.dtx}{class,article}%
    \from{tudscr-base.dtx}{class,article,load}%
    \from{tudscr-fonts.dtx}{class,article,option}%
    \from{tudscr-area.dtx}{class,article,option}%
    \from{tudscr-pagestyle.dtx}{class,article,option}%
    \from{tudscr-layout.dtx}{class,article,option}%
    \from{tudscr-title.dtx}{class,article,option}%
    \from{tudscr-frontmatter.dtx}{class,article,option}%
    \from{tudscr-comp.dtx}{class,article,option}%
    \from{tudscr-misc.dtx}{class,article,option}%
    \from{tudscr-misc.dtx}{class,article,execute}%
    \from{tudscr-base.dtx}{class,article,body}%
    \from{tudscr-fonts.dtx}{class,article,body}%
    \from{tudscr-fields.dtx}{class,article,body}%
    \from{tudscr-locale.dtx}{class,article,body}%
    \from{tudscr-area.dtx}{class,article,body}%
    \from{tudscr-pagestyle.dtx}{class,article,body}%
    \from{tudscr-layout.dtx}{class,article,body}%
    \from{tudscr-title.dtx}{class,article,body}%
    \from{tudscr-frontmatter.dtx}{class,article,body}%
    \from{tudscr-comp.dtx}{class,article,body}%
    \from{tudscr-misc.dtx}{class,article,body}%
  }
  \file{tudscrposter.cls}{%
    \from{tudscr-version.dtx}{class,poster}%
    \from{tudscr-base.dtx}{class,poster,load}%
    \from{tudscr-fonts.dtx}{class,poster,option}%
    \from{tudscr-area.dtx}{class,poster,option}%
    \from{tudscr-pagestyle.dtx}{class,poster,option}%
    \from{tudscr-layout.dtx}{class,poster,option}%
    \from{tudscr-title.dtx}{class,poster,option}%
    \from{tudscr-frontmatter.dtx}{class,poster,option}%
    \from{tudscr-comp.dtx}{class,poster,option}%
    \from{tudscr-misc.dtx}{class,poster,option}%
    \from{tudscr-misc.dtx}{class,poster,execute}%
    \from{tudscr-base.dtx}{class,poster,body}%
    \from{tudscr-fonts.dtx}{class,poster,body}%
    \from{tudscr-fields.dtx}{class,poster,body}%
    \from{tudscr-locale.dtx}{class,poster,body}%
    \from{tudscr-area.dtx}{class,poster,body}%
    \from{tudscr-pagestyle.dtx}{class,poster,body}%
    \from{tudscr-layout.dtx}{class,poster,body}%
    \from{tudscr-title.dtx}{class,poster,body}%
    \from{tudscr-frontmatter.dtx}{class,poster,body}%
    \from{tudscr-comp.dtx}{class,poster,body}%
    \from{tudscr-misc.dtx}{class,poster,body}%
  }
}
\generate{%
  \usepreamble\defaultpreamble%
  \file{tudscrsupervisor.sty}{%
    \from{tudscr-version.dtx}{package,supervisor}%
    \from{tudscr-supervisor.dtx}{package,supervisor}%
    \from{tudscr-fields.dtx}{package,supervisor}%
    \from{tudscr-locale.dtx}{package,supervisor}%
  }
}
\generate{%
  \usepreamble\defaultpreamble%
  \file{tudscrfonts.sty}{%
    \from{tudscr-version.dtx}{package,fonts}%
    \from{tudscr-fonts.dtx}{package,fonts,identify}%
    \from{tudscr-base.dtx}{package,fonts,load}%
    \from{tudscr-fonts.dtx}{package,fonts,option}%
    \from{tudscr-comp.dtx}{package,fonts,option}%
    \from{tudscr-misc.dtx}{package,fonts,execute}%
    \from{tudscr-fonts.dtx}{package,fonts,body}%
    \from{tudscr-comp.dtx}{package,fonts,body}%
    \from{tudscr-misc.dtx}{package,fonts,body}%
  }
}
\generate{%
  \usepreamble\defaultpreamble%
  \file{fix-tudscrfonts.sty}{%
    \from{tudscr-version.dtx}{package,fontsfix}%
    \from{tudscr-comp.dtx}{package,fontsfix,identify}%
    \from{tudscr-base.dtx}{package,fontsfix}%
    \from{tudscr-comp.dtx}{package,fontsfix,option}%
    \from{tudscr-comp.dtx}{package,fontsfix,body}%
  }
}
\generate{%
  \usepreamble\defaultpreamble%
  \file{tudscrcomp.sty}{%
    \from{tudscr-version.dtx}{package,comp,base}%
    \from{tudscr-comp.dtx}{package,comp,base,identify}%
  }
  \file{tudscrcomp-book.sty}{%
    \from{tudscr-version.dtx}{package,comp,book}%
    \from{tudscr-comp.dtx}{package,comp,book,identify}%
    \from{tudscr-comp.dtx}{package,comp,book,option}%
    \from{tudscr-misc.dtx}{package,comp,book,execute}%
    \from{tudscr-comp.dtx}{package,comp,book,body}%
  }
  \file{tudscrcomp-poster.sty}{%
    \from{tudscr-version.dtx}{package,comp,poster}%
    \from{tudscr-comp.dtx}{package,comp,poster,identify}%
    \from{tudscr-comp.dtx}{package,comp,poster,option}%
    \from{tudscr-misc.dtx}{package,comp,poster,execute}%
    \from{tudscr-comp.dtx}{package,comp,poster,body}%
  }
}
\generate{%
  \usepreamble\defaultpreamble%
  \file{tudscrcolor.sty}{%
    \from{tudscr-version.dtx}{package,color}%
    \from{tudscr-color.dtx}{package,color}%
  }
  \file{twocolfix.sty}{%
    \from{tudscr-twocolfix.dtx}{package}%
  }
  \file{mathswap.sty}{%
    \from{tudscr-mathswap.dtx}{package}%
  }
}
\generate{%
  \usepreamble\defaultpreamble%
  \file{tudscrmanual.cls}{%
    \from{tudscr-version.dtx}{class,manual,inherit}%
    \from{tudscr-manual.dtx}{class,manual}%
    \from{tudscr-base.dtx}{class,manual,inherit,load}%
    \from{tudscr-manual.dtx}{class,manual,option}%
    \from{tudscr-misc.dtx}{class,manual,inherit,execute}%
    \from{tudscr-manual.dtx}{class,manual,body}%
    \from{tudscr-fields.dtx}{class,manual,body}%
    \from{tudscr-locale.dtx}{class,manual,body}%
    \from{tudscr-texindy.dtx}{class,manual}%
  }
  \file{tudscrtutorial.sty}{%
    \from{tudscr-version.dtx}{package,tutorial}%
    \from{tudscr-manual.dtx}{package,tutorial,identify}%
    \from{tudscr-base.dtx}{package,tutorial,load}%
    \from{tudscr-manual.dtx}{package,tutorial,option}%
    \from{tudscr-misc.dtx}{package,tutorial,execute}%
    \from{tudscr-manual.dtx}{package,tutorial,body}%
    \from{tudscr-texindy.dtx}{package,tutorial}%
  }
}
\generate{%
  \usepreamble\defaultpreamble%
  \file{tudscrdoc.cls}{%
    \from{tudscr-version.dtx}{class,doc}%
    \from{tudscr-base.dtx}{class,doc}%
    \from{tudscr-doc.dtx}{class,doc,option}%
    \from{tudscr-locale.dtx}{class,doc}%
    \from{tudscr-manual.dtx}{class,doc}%
    \from{tudscr-doc.dtx}{class,doc,body}%
    \from{tudscr-texindy.dtx}{class,doc}%
  }
}
\endbatchfile


\RequirePackage[ngerman=ngerman-x-latest]{hyphsubst}
\documentclass[english,ngerman,ToDo=no]{tudscrman3}
\usepackage{selinput}\SelectInputMappings{adieresis={ä},germandbls={ß}}
\usepackage[T1]{fontenc}
\usepackage{blindtext}
\begin{document}

\begin{Declaration}[%
  v2.02!Wert \PValue{double} mit \PValue{multi} ersetzt,%
  v2.02!Wert \PValue{tocleveldown} neu,%
  v2.02!Wert \PValue{markboth} neu,%
  v2.04!Wert \PValue{tocmultiple} neu%
]{\Option{abstract}[\PSet]}%
\printdeclarationlist%
\index{Zusammenfassung|!(}%
\index{Zweispaltensatz}%
%
Diese Option wird bereits durch \KOMAScript{} für die Klassen \Class{scrartcl} 
und \Class{scrreprt} standardmäßig bereitgestellt. Für die Klasse 
\Class{scrbook} geschieht dies nicht. Dazu heißt es im Handbuch:
\end{Declaration}

\begin{Declaration}[%
  v2.02!\Macro{nextabstract} zur Trennung
]{\Environment{abstract}[\OLParameter{Sprache}]}
\begin{Declaration}[v2.02]{\Macro{nextabstract}\OLParameter{Sprache}}
\begin{Declaration}{\Key{\Environment{abstract}}{language}[\PName{Sprache}]}
\begin{Declaration}[v2.02]{%
  \Key{\Environment{abstract}}{markboth}[\PBName{Kolumnentitel}]%
}
\begin{Declaration}[v2.02]{%
  \Key{\Environment{abstract}}{pagestyle}[\PName{Seitenstil}]%
}
\begin{Declaration}{\Key{\Environment{abstract}}{columns}[\PName{Anzahl}]}
\begin{Declaration}{\Key{\Environment{abstract}}{option}[\PSet]}{%
  \see*{\Option{abstract}'ppage'}%
}
\printdeclarationlist%
\index{Zweispaltensatz}%
%
Die \Environment{abstract}-Umgebung dient speziell für die Ausgabe einer 
Zusammenfassung, entweder zu Beginn eines Dokumentes oder beispielsweise vor 
einem Teil oder Kapitel. Wird ein Titelkopf und keine Titelseite verwendet 
(\KOMAScript-Option \Option{titlepage}[false]), wird eine Zusammenfassung, die  
\emph{nicht} mit der Überschrift einer Gliederungsebene gesetzt wird, wie bei 
den \KOMAScript"=Klassen in einer \Environment{quotation}"=Umgebung gesetzt, um 
diese vom restlichen Fließtext abzuheben. Diese hat jedoch den Nachteil, dass 
in besagter Umgebung die \KOMAScript-Option \Option{parskip} nicht beachtet 
wird. Um dieses Problem zu beheben, kann das Paket \Package{quoting} geladen 
werden, wodurch stattdessen die Umgebung \Environment{quoting} verwendet wird.

Mit der zuvor erläuterten Option \Option{abstract} kann eingestellt werden, in 
welcher Gestalt die Zusammenfassung ausgegeben werden soll. Des Weiteren lässt 
sich jede \Environment{abstract}"=Umgebung individuell über weitere Parameter 
als optionales Argument anpassen. Damit lassen sich gegebenenfalls für eine 
bestimmte \Environment{abstract}"=Umgebung die globalen Einstellungen 
der Option \Option{abstract} lokal ändern und gezielt anpassen. 

Wird das Paket \Package{babel} durch den Anwender geladen, kann mit dem 
optionalen Parameter \Key{\Environment{abstract}}{language}[\PName{Sprache}] 
die Sprache innerhalb der \Environment{abstract}"=Umgebung geändert werden. 
Dafür muss die gewünschte Sprache bereits mit dem Laden von \Package{babel} 
entweder als Paketoption oder besser noch als Klassenoption angegeben worden 
sein. Dadurch werden innerhalb der Umgebung die Bezeichnung \Term{abstractname} 
und die Trennungsmuster sprachspezifisch angepasst. Die gewünschte Sprache kann 
auch ohne die Verwendung des Parameters \Key{\Environment{abstract}}{language} 
direkt als optionales Argument übergeben werden.

\ChangedAt{v2.02}
Mit \Key{\Environment{abstract}}{markboth} können die gesetzten Kolumnentitel 
beeinflusst werden. Wird \Key{\Environment{abstract}}{markboth}[false] 
angegeben, werden automatische respektive manuelle Kolumnentitel verwendet. Die 
Einstellung \Key{\Environment{abstract}}{markboth}[true] wiederum setzt diese 
für linke und rechte Seiten auf \Term{abstractname}. Außerdem lässt sich der 
Kolumnentitel mit \Key{\Environment{abstract}}{markboth}[\PName{Kolumnentitel}] 
auch direkt festlegen. So können die Kolumnen beispielsweise mit der Verwendung 
von \Key{\Environment{abstract}}{markboth}[\PParameter{}] auch gelöscht werden. 
Sollte \Key{\Environment{abstract}}{markboth} aktiviert werden, so wird in der
Umgebung automatisch der Seitenstil \PageStyle{headings} genutzt~-- falls eine 
Titelseite und kein Titelkopf (\KOMAScript-Option \Option{titlepage}[true]) 
verwendet wird. Mit dem Parameter \Key{\Environment{abstract}}{pagestyle} kann 
dieser auch manuell angegeben werden, die \PageStyle{tudheadings}"=Seitenstile 
werden dabei ebenfalls unterstützt werden.

Wurde das Paket \Package{multicol} geladen, kann mit dem Parameter 
\Key{\Environment{abstract}}{columns}[\PName{Anzahl}] die Zusammenfassung 
mehrspaltig gesetzt werden. Dem Parameter \Key{\Environment{abstract}}{option} 
können alle gültigen, bereits erläuterten Werte der Option \Option{abstract} 
übergeben werden. Die damit gemachten Einstellungen wirken sich~-- im Gegensatz 
zur Angabe als Klassenoption oder über die Variante der späten Optionenwahl%
\footnote{%
  \Macro{TUDoption}\PParameter{abstract}\Parameter{Einstellung} oder
  \Macro{TUDoptions}\PParameter{abstract=\PName{Einstellung}}
}~-- lediglich lokal auf die verwendete \Environment{abstract}"=Umgebung aus.

\ChangedAt{v2.02}
Sollen mehrere Zusammenfassungen im gleichen Stil erzeugt und die Einstellungen 
der Option \Option{abstract}[simple/multiple/fill/nofill] beachtet werden, so 
ist die \Environment{abstract}"=Umgebung nur einmal zu verwenden. Innerhalb 
dieser müssen die einzelnen Zusammenfassungen mit \Macro{nextabstract} 
voneinander getrennt werden. Der Befehl akzeptiert dabei im optionalen Argument 
alle Parameter, die auch von der \Environment{abstract}"=Umgebung selbst 
unterstützt werden. Das Minimalbeispiel in \fullref{sec:exmpl:dissertation} 
zeigt hierfür das notwendige Vorgehen.

Wird die \Environment{abstract}"=Umgebung innerhalb des Argumentes der Befehle 
\Macro{setpartpreamble} beziehungsweise \Macro{setchapterpreamble} verwendet, 
so wird die Überschrift~-- im Fall, dass nicht \Option{abstract}[false] gewählt 
ist~-- \emph{immer} in Textgröße und zentriert gesetzt.
\end{Declaration}
\end{Declaration}
\end{Declaration}
\end{Declaration}
\end{Declaration}
\end{Declaration}
\end{Declaration}

%\index[changelognew]{v2.02!Implementierung!
%  \hyperidx [tudscr:macros:abstract]{
%    \begingroup \ttfamily  abstract\endgroup
%    \begingroup ~\footnotesize (Umgebung)\endgroup
%  }:~bla
%}
%\index[changelognew]{v2.02!Implementierung!
%  \hyperidx [tudscr:macros:abstract]{
%    \begingroup \ttfamily  abstract\endgroup
%    \begingroup ~\footnotesize (Umgebung)\endgroup
%  }:~xxx
%}
%\index[changelognew]{v2.02!Implementierung!
%  \begingroup \ttfamily  abstract\endgroup
%  \begingroup ~\footnotesize (Umgebung)\endgroup
%  :~sss
%}


\Blindtext
\ChangedAt{v2.06}
\blindtext
\ChangedAt{v2.06!Bla}
\blindtext
\ChangedAt{v2.06!Kram!\Macro{Blubb}}
%\blindtext
%\makeatletter
%\edef\bla{@}
%\meaning\bla
%\meaning @
%\the\catcode`@
%\makeatother
%
%\edef\bla{@}
%\meaning\bla
%\meaning @
%\the\catcode`@
%
%\def\bla{aa@bb}

%\def\blubb{aaa!bbb!eee}
%\tudatreplace\blubb
%\meaning\blubb
%
%
%\def\blubb{aaa@bbb@eee}
%\tudatreplace\blubb
%\meaning\blubb

%\tudatreplace{}

%\meaning\bla


%\indexentry{\hyperidx [tudscr:options:abstract]{abstract@\begingroup 
%\ttfamily  
%abstract\endgroup |(declare}}{41}


%\manualhyperdef{aaa:bbb:"@foo}
%\Blindtext
%\index[macros]{\hyperidx[aaa:bbb:"@foo]{"@bla}|(declare}
%\Blindtext
%\index[macros]{\hyperidx[aaa:bbb:"@foo]{"@bla}|declare)}
%\Blindtext
%
%
%\index[macros]{abstract@\begingroup \ttfamily  abstract\endgroup|declare}

%\Macro{bla}
%
%\Macro{@blubb}
%
%\index{Sam@\textsl{Sam}}
%
%\index{Sam!dude}
%
%\index{Sam!aaa@\textbf{typ}}

\clearpage
\ToDo[imp]{see und Crossindex kombinieren, einträge prüfen}
% \SeeIndex[macros]{Befehle}{Makros,Parameter}
% -> \index{Makros|see{Befehle}}%
% -> \index{Parameter|see{Befehle}}%
% -> \CrossIndex{Befehle}{macros}%
% \SeeIndex[Änderungsliste:changes]{Änderungen}{Changelog,Version}
\CrossIndex{Optionen}{options}%
\CrossIndex{Befehle,Umgebungen}{macros}%
\CrossIndex{Bezeichner}{terms}%
\CrossIndex{%
  Seitenstile,Layout!Seitenstile,Schriftelemente,Farben,Layout!Farben%
}{elements}%
\CrossIndex{Längen}{misc}%
\CrossIndex{Zähler}{misc}%
\CrossIndex{Klassen,Pakete,Dateien}{files}%
\CrossIndex{Änderungen,Kompatibilität}[Änderungsliste]{changes}%
%
\index{LaTeX-Distribution@\protect\hologo{LaTeX}-Distribution|see{Distribution}}%
\index{MacTeX@\Distribution{Mac\protect\hologo{TeX}}|see{Distribution}}%
\index{TeX Live@\Distribution{\protect\hologo{TeX}~Live}|see{Distribution}}%
\index{MiKTeX@\Distribution{\protect\hologo{MiKTeX}}|see{Distribution}}%
%
\index{Abbildungen|see{Grafiken}}%
\index{Abschlussarbeit|see{Typisierung}}%
\index{Abstände|see{Absatzauszeichnung}}%
\index{Abstände|see{Längen}}%
\index{Abstände|see{Leerraum}}%
\index{Aktualisierung|see{Update}}%
\index{Aufzählungen|see{Listen}}%
\index{Autorenangaben|see{Titel}}%
\index{Bindekorrektur|see{Satzspiegel}}%
\index{Changelog|see{Änderungen}}%
\index{Cover|see{Umschlagseite}}%
\index{Dezimaltrennzeichen|see{Zifferngruppierung}}%
\index{doppelseitiger Satz|see{Satzspiegel}}%
\index{Dresden-concept-Logo@\DDC-Logo|see{Layout}}%
\index{Drittlogo|see{Layout}}%
\index{Fachreferent|see{Referent}}%
\index{Farben|see{Layout}}%
\index{Fußzeile|see{Layout}}%
\index{Installation|see{Update}}%
\index{Gliederung|see{Layout!Überschriften}}%
\index{Grafiken|see{Gleitobjekte}}%
\index{Kapitel|see{Layout}}%
\index{Klassenoptionen|see{Optionen}}%
\index{Kolumnentitel|see{Layout}}%
\index{Kopfzeile|see{Layout}}%
\index{Kurzfassung|see{Zusammenfassung}}%
\index{Layout!Seitenränder|see{Satzspiegel}}%
\index{Layout!Titel|see{Titel}}%
\index{Layout!Umschlagseite|see{Umschlagseite}}%
\index{Leerraum|see{Absatzauszeichnung}}%
\index{Leerseiten|see{Vakatseiten}}%
\index{Lokalisierung|see{Bezeichner}}%
\index{Makros|see{Befehle}}%
\index{Makros|see{Umgebungen}}%
\index{Mathematiksatz|see{Griechische Lettern}}%
\index{Mathematiksatz|see{Einheiten}}%
\index{Mathematiksatz|see{Zifferngruppierung}}%
\index{Nutzerinstallation|see{Installation}}%
\index{Outline-Eintrag|see{Lesezeichen}}%
\index{Paketoptionen|see{Optionen}}%
\index{Parameter|see{Befehle}}%
\index{Parameter|see{Umgebungen}}%
\index{Professor|see{Hochschullehrer}}%
\index{Querbalken|see{Layout}}%
\index{Seitenränder|see{Satzspiegel}}%
\index{Seitenstile|see{Layout}}%
\index{Silbentrennung|see{Worttrennung}}%
\index{Sprachunterstützung|see{Bezeichner}}%
\index{Sprachunterstützung|see{Worttrennung}}%
\index{Sprungmarken|see{Lesezeichen}}%
\index{Tabellen|see{Gleitobjekte}}%
\index{Tausendertrennzeichen|see{Zifferngruppierung}}%
\index{Teile|see{Layout}}%
\index{Titel!Umschlagseite|see{Umschlagseite}}%
\index{Überfüllung|see{Beschnittzugabe}}%
\index{Trennungsmuster|see{Worttrennung}}%
%\index{Vektorgrafiken|see{Grafiken}}%
\index{Vektorgrafiken|seeunverified{\Application{Inkscape}}}%
\index{Versalien|see{Schriftauszeichnung}}%
\index{Version|see{Änderungen}}%
\index{zweiseitiger Satz|see{Satzspiegel}}%
\index{zweispaltiger Satz|see{Satzspiegel}}%
\index{Zweitlogo|see{Layout}}%
%
\bookmarksetup{startatroot}
\PrintIndex
\PrintChanges

%\printindex
%\printindex[macros]
\end{document}

