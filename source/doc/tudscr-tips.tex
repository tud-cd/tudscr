\chapter{%
  Praktische Tipps \& Tricks%
  \label{sec:tips}%
}
\section{%
  \hologo{LaTeX}-Editoren%
  \label{sec:tips:editor}%
}
\ToDo[doc]{%
  Hinweis zu \Macro*{input glyphtounicode} und \Macro*{pdfgentounicode=1}; 
  Suche in source Dateien nach fontenc\textbar hyphsubst
}[v2.06]%
Hier werden die gängigsten Editoren zum Bearbeiten von \hologo{LaTeX}"~Dateien 
genannt. Ich persönlich nutze seit langem \Application{\hologo{TeX}studio}, da 
dieser viele Unterstützungs- und Assistenzfunktionen bietet. Neben diesen gibt 
es noch weitere, gut nutzbare \hologo{LaTeX}-Editoren. Unabhängig von der 
Auswahl des Editors, sollte dieser auf jeden Fall eine Unterstützung von 
Unicode~(UTF"~8) enthalten:
%
\begin{itemize}
\item \Application{\hologo{TeX}maker}
\item \Application{Kile}
\item \Application{\hologo{TeX}works}
\item \Application{\hologo{TeX}lipse}~-- Plug"~in für \Application{Eclipse}
\item \Application{\hologo{TeX}nicCenter}
\item \Application{WinEdt}
\item \Application{LEd}~-- früher \hologo{LaTeX}~Editor
\item \Application{\hologo{LyX}}~-- grafisches Front"~End für \hologo{LaTeX}
\end{itemize}
%
Für \Application{\hologo{TeX}studio} wird im \GitHubRepo'<releases> das Archiv 
\GitHubDownload<TeXstudio>{tudscr4texstudio.zip} zur Erweiterung der 
automatischen Befehlsvervollständigung für das \TUDScript-Bundle 
bereitgestellt. Die darin enthaltenen Dateien müssen unter Windows in 
\Path{\%APPDATA\%\bsc texstudio} beziehungsweise unter unixoiden 
Betriebssystemen in \Path{.config/texstudio} eingefügt werden.

Möchten Sie das grafische \hologo{LaTeX}"~Frontend~\Application{\hologo{LyX}} 
für das Erstellen eines Dokumentes mit den \TUDScript-Klassen nutzen, so werden 
dafür spezielle Layout"~Dateien benötigt, um die Klassendateien verwenden zu 
können. Diese sind zusammen mit einem \Application{\hologo{LyX}}"~Dokument als 
Archiv \GitHubDownload<LyX>{tudscr4lyx.zip} im \GitHubRepo'<releases> 
verfügbar. Um die Layout"~Dateien zu aktivieren, müssen diese in den passenden 
Unterordner im \Application{\hologo{LyX}}"=Installationspfad kopiert werden. 
Dieser ist bei Windows
\Path{\%PROGRAMFILES(X86)\%\bsc LyX~2.1\bsc Resources\bsc layouts}
beziehungsweise bei unixoiden Betriebssystemen \Path{/usr/share/lyx/layouts}.
Anschließend muss LyX über den Menüpunkt \emph{Werkzeuge} neu konfiguriert 
werden. 



\section{%
  Literaturverwaltung in \hologo{LaTeX}%
  \index{Literaturverzeichnis|!}%
}

\Hint{v2.02}%
Die simpelste Variante, eine \hologo{LaTeX}"=Literaturdatenbank zu verwalten, 
ist dies mit dem Editor manuell zu erledigen. Wesentlich komfortabler ist es 
jedoch, die Referenzverwaltung mit einer darauf spezialisierten Anwendung zu 
bewerkstelligen. Dafür gibt es zwei sehr gute Programme:
%
\begin{itemize}
\item \Application{Citavi}
\item \Application{JabRef}
\end{itemize}
%
Das Programm \Application{Citavi} ermöglicht den Import von bibliografischen 
Informationen aus dem Internet. Allerdings sind diese teilweise unvollständig 
oder mangelhaft. Mit \Application{JabRef} hingegen muss die Literaturdatenbank 
manuell erstellt werden, wobei sich einzelne Einträge aus \File*{.bib}"~Dateien 
importieren lassen. Beide Anwendungen unterstützen den Export beziehungsweise 
die Erstellung von Datenbanken im Stil von \Package{biblatex}. Für 
\Application{JabRef} muss diese durch den Anwender explizit aktiviert 
werden.\footnote{Optionen/Einstellungen/Erweitert/BibLaTeX"~Modus} 
Zur Verwendung der beiden Programme in Verbindung mit \Package{biblatex} und 
\Application{biber} gibt es ein gutes Tutorial unter diesem
\hrfn{http://www.suedraum.de/latex/stammtisch/degenkolb_latex_biblatex_folien-final.pdf}{Link}.



\section{%
  Worttrennungen in deutschsprachigen Texten%
  \label{sec:tips:hyphenation}%
  \index{Worttrennung|!}%
  \index{Trennungsmuster}%
}

\Hint{v2.02}%
Die möglichen Trennstellen von Wörtern werden von \hologo{LaTeX} mithilfe 
eines Algorithmus berechnet. Dieser ist jedoch in seiner ursprünglichen Form 
für die englische Sprache konzipiert worden. Für deutschsprachige Texte wird 
die Worttrennung~-- insbesondere bei zusammengeschriebenen Wörtern~-- mit dem 
Paket \Package{hyphsubst} entscheidend verbessert, welches ein um vielerlei 
Trennungsmuster ergänztes Wörterbuch aus dem Paket \Package{dehyph-exptl} 
nutzt. Weiterhin muss eines der beiden Sprachpakete \Package{babel} oder
\Package{polyglossia} geladen werden. 

Für \Format{pdfLaTeX} ist zusätzlich das Paket \Package{fontenc} beispielsweise 
mit der \PValue{T1}~Schriftkodierung erforderlich, damit auch Wörter mit 
Umlauten richtig getrennt werden. Bei der Verwendung von \Format{LuaLaTeX} 
oder \Format{XeLaTeX} werden~-- in Verbindung mit einem der zwei genannten 
Sprachpakete~-- die besseren Trennungsmuster automatisch aktiviert. Der Beginn 
einer Dokumentpräambel könnte folgendermaßen aussehen:
%
\begin{quoting}[rightmargin=0pt]
\begin{Code}
\documentclass[ngerman,§\PName{Klassenoptionen}§]§\MPName{Dokumentklasse}§
\usepackage{iftex}
\iftutex
  \usepackage{fontspec}
\else
  \usepackage[T1]{fontenc}
  \usepackage[ngerman=ngerman-x-latest]{hyphsubst}
\fi
\usepackage{babel}
\end{Code}
\end{quoting}
%
Eine Anmerkung noch zur Trennung von Wörtern mit Bindestrichen. Normalerweise 
sind die beiden von \hologo{LaTeX} verwendeten Zeichen für Bindestrich und 
Trennstrich identisch. Leider wird der Trennungsalgorithmus von \hologo{LaTeX} 
bei Wörtern, welche bereits einen Bindestrich enthalten, außer Kraft gesetzt. 
In der Folge werden~-- in der deutschen Sprache durchaus öfter anzutreffende~-- 
Wortungetüme wie die \enquote{Donaudampfschifffahrts-Gesellschafterversammlung} 
normalerweise nur direkt nach dem angegebenen Bindestrich getrennt. 

Allerdings gibt es die Möglichkeit, das genutzte Zeichen für den Trennstrich 
zu ändern. Dafür ist das Laden der \PValue{T1}"~Schriftkodierung mit dem Paket 
\Package{fontenc} zwingend erforderlich. Wenn von der verwendeten Schrift 
nichts anderes eingestellt ist, liegen sowohl Binde- als auch Trennstrich auf 
Position~\PValue{45} der Zeichentabelle. In der \PValue{T1}"~Schriftkodierung 
befindet sich auf der Position~\PValue{127} glücklicherweise für gewöhnlich das 
gleiche Zeichen noch einmal. Dies ist jedoch von der verwendeten Schrift 
abhängig. Wird der Ausdruck \Macro*{defaulthyphenchar}[\PValue{=127}] vor dem 
Laden des Paketes \Package{fontenc} verwendet, kann dieses Zeichen für den 
Trennstrich genutzt werden. 

Sollte trotz aller Maßnahmen dennoch einmal ein bestimmtes Wort falsch getrennt 
werden, so kann die Worttrennung dieses Wortes manuell und global geändert 
werden. Dies wird mit \Macro{hyphenation}[\MPValue{Sil-ben-tren-nung}] gemacht. 
Es ist zu beachten, dass dies für alle Flexionsformen des Wortes erfolgen 
sollte. Für eine lokale/temporäre Worttrennung kann mit Befehlen aus dem Paket 
\Package{babel} gearbeitet werden. Diese sind: 

\vskip\medskipamount\noindent
\begingroup
\newcommand*\listhyphens[2]{#1 & \PValue{#2} \tabularnewline}%
\begin{tabular}{@{}ll}
  \textbf{Beschreibung} & \textbf{Befehl} \tabularnewline
  \listhyphens{ausschließliche Trennstellen}{\bsc-}
  \listhyphens{zusätzliche Trennstellen}{"'-} 
  \listhyphens{Umbruch ohne Trennstrich}{"'"'}
  \listhyphens{Bindestrich ohne Umbruch}{"'\textasciitilde} 
  \listhyphens{Bindestrich, der weitere Trennstellen erlaubt}{"'=}
\end{tabular}
\endgroup



\section{%
  Lokale Änderungen von Befehlen und Einstellungen%
  \label{sec:tips:local}%
  \index{Befehle}%
  \index{Umgebungen}%
  \index{Befehlsdeklaration!Geltungsbereich}%
}

\Hint{v2.02}%
Ein zentraler Bestandteil von \hologo{LaTeX} ist die Verwendung von Gruppen 
oder Gruppierungen. Innerhalb dieser bleiben alle vorgenommenen Änderungen an 
Befehlen, Umgebungen oder Einstellungen lokal. Dies kann sehr nützlich sein, 
wenn beispielsweise das Verhalten eines bestimmten Makros einmalig oder 
innerhalb von selbst definierten Befehlen oder Umgebungen geändert werden, im 
Normalfall jedoch die ursprüngliche Funktionalität behalten soll.
\begin{Example}
\index{Schriftauszeichnung}%
Der Befehl \Macro{emph} wird von \hologo{LaTeX} für Hervorhebungen im Text 
bereitgestellt und führt normalerweise zu einer kursiven oder~-- falls kein 
Schriftschnitt mit echten Kursiven vorhanden ist~-- kursivierten oder auch 
geneigten Auszeichnung. Soll nun in einem bestimmten Abschnitt die Auszeichnung 
mit fetter Schrift erfolgen, kann der Befehl \Macro{emph} innerhalb einer 
Gruppierung geändert und verändert werden. Wird diese beendet, verhält sich der 
Befehl wie gewohnt.
\begin{Code}
In diesem Text wird genau ein \emph{Wort} hervorgehoben.

\begingroup
  \renewcommand*{\emph}[1]{\textbf{#1}}%
  In diesem Text wird genau ein \emph{Wort} hervorgehoben.
\endgroup

In diesem Text wird genau ein \emph{Wort} hervorgehoben.
\end{Code}
\end{Example}
Eine Gruppierung kann entweder mit \Macro*{begingroup} und \Macro*{endgroup} 
oder einfach mit einem geschweiften Klammerpaar \MPValue{\dots} definiert 
werden.



\section{%
  Bezeichnung der Gliederungsebenen durch \PackageRaw{hyperref}{\BooleanFalse}%
  \label{sec:tips:references}%
  \index{Querverweise}%
}

\Hint{v2.02}%
Das Paket \Package{hyperref} stellt für Querverweise unter anderem den Befehl 
\Macro{autoref}[\MPName{label}](\Package{hyperref}) zur Verfügung. Mit diesem 
wird~-- im Gegensatz zur Verwendung von \Macro{ref}~-- bei einer Referenz nicht 
nur die Nummerierung selber sondern auch das entsprechende Element wie Kapitel 
oder Abbildung vorangestellt. Zur Benennung des referenzierten Elementes wird 
sequentiell geprüft, ob das Makro \Macro*{\PName{Element}autorefname} oder 
\Macro*{\PName{Element}name} existiert. Zur Änderung der Bezeichnung eines 
Elementes, muss der entsprechende Bezeichner angepasst werden.
%
\begin{Example}
Bezeichnungen von Gliederungsebenen können folgendermaßen verändert werden.
\begin{Code}
\renewcaptionname{ngerman}{\sectionautorefname}{Unterkapitel}
\renewcaptionname{ngerman}{\subsectionautorefname}{Abschnitt}
\end{Code}
\end{Example}



\section{%
  URL-Umbrüche im Literaturverzeichnis mit \PackageRaw{biblatex}{\BooleanFalse}%
  \index{Literaturverzeichnis}%
}

\Hint{v2.02}%
Wird das Paket \Package{biblatex} verwendet, kann es unter Umständen dazu 
kommen, das eine URL nicht vernünftig umbrochen wird. Ist dies der Fall, 
können die Zähler \Counter{biburlnumpenalty}(\Package{biblatex}), 
\Counter{biburlucpenalty}(\Package{biblatex}) und 
\Counter{biburllcpenalty}(\Package{biblatex}) erhöht werden. Das Manipulieren 
eines Zähler kann mit \Macro*{setcounter}[\MPName{Zähler}] oder lokal mit 
\Macro*{defcounter}[\MPName{Zähler}] aus dem Paket \Package{etoolbox} erfolgen. 
Die möglichen Werte liegen zwischen 0~und~10\,000, wobei es bei höheren 
Zählerwerten zu mehr URL"~Umbrüchen an 
Ziffern~(\Counter{biburlnumpenalty}(\Package{biblatex})), 
Majuskeln~(\Counter{biburlucpenalty}(\Package{biblatex})) und 
Minuskeln~(\Counter{biburllcpenalty}(\Package{biblatex})) kommt. 
Genaueres in der Dokumentation zu \Package{biblatex}.



\section{%
  Zeilenabstände in Überschriften%
  \index{Zeilenabstand}%
  \index{Durchschuss}%
  \label{sec:tips:headings}%
}

Mit dem Paket \Package{setspace} kann der Zeilenabstand beziehungsweise der 
Durchschuss innerhalb des Dokumentes geändert werden. Sollte dieser erhöht 
worden sein, können die Abstände bei mehrzeiligen Überschriften als zu groß 
erscheinen. Um dies zu korrigieren kann mit dem Befehl 
\Macro{addtokomafont}[%
  \MPValue{disposition}\MPValue{%
    \Macro{setstretch}[\MPValue{1}](\Package{setspace})%
  }%
](\Bundle{koma-script})
der Zeilenabstand aller Überschriften auf einzeilig zurückgeschaltet werden. 
Soll dies nur für eine bestimmte Gliederungsebene erfolgen, so ist der 
Parameter \PValue{disposition} durch das dazugehörige Schriftelement zu 
ersetzen.



\section{%
  Warnung wegen zu geringer Höhe der Kopf-/Fußzeile%
  \label{sec:tips:headline}%
  \index{Zeilenabstand}%
  \index{Layout!Kopfzeile}%
  \index{Layout!Fußzeile}%
}

Wird das Paket \Package{setspace} verwendet, kann es passieren, dass nach der 
Änderung des Zeilenabstandes \emph{innerhalb} des Dokumentes eine oder beide 
der folgenden Warnungen erscheinen:
%
\begin{quoting}
\begin{Code}
scrlayer-scrpage Warning: \headheight to low.
scrlayer-scrpage Warning: \footheight to low.
\end{Code}
\end{quoting}
%
Dies liegt an dem durch den vergrößerten Zeilenabstand erhöhten Bedarf für die
Kopf- und Fußzeile, die Höhen können in diesem Fall direkt mit der Verwendung 
von \Macro{recalctypearea}(\Package{typearea}) angepasst werden. Allerdings 
ändert das den Satzspiegel im Dokument, was eine andere und durchaus 
berechtigte Warnung von \Package{typearea} zur Folge hat. Falls die Änderung 
des Durchschusses wirklich nötig ist, sollte dies in der Präambel des 
Dokumentes einmalig passieren. Dann entfallen auch die Warnungen.



\section{%
  Einrückung von Tabellenspalten verhindern%
  \label{sec:tips:table}%
  \index{Tabellen}%
}

Normalerweise wird in einer Tabelle vor \emph{und} nach jeder Spalte durch 
\hologo{LaTeX} etwas horizontaler Raum mit \Macro*{hskip}\Length{tabcolsep} 
eingefügt.%
\footnote{%
  Der Abstand zweier Spalten beträgt folglich \PValue{2}\Length{tabcolsep}.%
}
Dies geschieht auch \emph{vor} der ersten und \emph{nach} der letzten Spalte. 
Diese optische Einrückung an den äußeren Rändern kann unter Umständen stören, 
insbesondere bei Tabellen, die willentlich~-- beispielsweise mit den Paketen 
\Package{tabularx}, \Package{tabulary} oder auch \Package{tabu}~-- über die 
komplette Seitenbreite aufgespannt werden.

Das Paket \Package{tabularborder} versucht, dieses Problem automatisiert zu 
beheben, ist jedoch nicht zu allen \hologo{LaTeX}"~Paketen für den 
Tabellensatz kompatibel, unter anderem auch nicht zu den drei zuvor genannten. 
Allerdings lässt sich dieses Problem manuell lösen. 

Bei der Deklaration einer Tabelle kann mit~\PValue{@}\MPValue{\dots} vor und 
nach dem Spaltentyp angegeben werden, was anstelle von \Length{tabcolsep} vor 
beziehungsweise nach der eigentlichen Spalte eingeführt werden soll. Dies kann 
für das Entfernen der Einrückungen genutzt werden, indem an den entsprechenden 
Stellen~\PValue{@\MPValue{}} bei der Angabe der Spaltentypen vor der ersten und 
nach der letzten Tabellenspalte verwendet wird.
%
\begin{Example}
Eine Tabelle mit zwei Spalten, wobei bei einer die Breite automatisch berechnet 
wird, soll über die komplette Textbreite gesetzt werden. Dabei soll der Rand 
vor der ersten und nach der letzten entfernt werden.
\begin{Code}
\begin{tabularx}{\textwidth}{@{}lX@{}}
§\dots§ & §\dots§ \tabularnewline
§\dots§
\end{tabularx}
\end{Code}
\end{Example}



\section{%
  Unterdrückung des Einzuges eines Absatzes%
  \index{Absatzauszeichnung}%
}
\begin{Entity}{\Bundle{koma-script}}
Durch die \KOMAScript-Option \InlineDeclaration{\Option{parskip=\PMisc}} wird
die Absatzauszeichnung  festgelegt. Werden hierfür keine vertikalen Abstände 
(\Option{parskip=half}'none' oder \Option{parskip=true}'none' sondern~-- wie es 
aus typografischer Sicht zumeist sinnvoll ist~-- stattdessen Einzüge verwendet 
(\Option{parskip=false})'none', kann es vorkommen, dass ein ganz bestimmter 
Absatz~-- beispielsweise der nach einer zuvor genutzten Umgebung folgende~-- 
ungewollt eingerückt ist.
\end{Entity}
Dies kann sehr einfach manuell behoben werden, indem direkt zu Beginn des 
Absatzes das Makro \Macro{noindent}(\Bundle{latex-base}) aufgerufen wird. Soll 
das Einrücken von Absätzen nach ganz bestimmten Umgebungen oder Befehlen 
automatisiert unterbunden werden, ist das Paket \Package{noindentafter} zu 
empfehlen.



\section{%
  Unterbinden des Zurücksetzens von Fußnoten%
  \index{Fußnoten}%
  \index{Zähler}%
}

Mit \Macro{counterwithin} und \Macro{counterwithout} können 
\hologo{LaTeX}"~Zähler so umdefiniert werden, dass sie bei der Änderung eines 
anderen Zählers zurückgesetzt werden oder nicht. Mit der Angabe von 
\Macro*{counterwithout*}[\MPValue{footnote}\MPValue{chapter}] wird das 
Zurücksetzen des Fußnotenzählers durch neue Kapitel deaktiviert, womit sich 
über Kapitel fortlaufende Fußnoten realisieren lassen.



\section{%
  Warnung beim Erzeugen des Inhaltsverzeichnisses%
  \index{Inhaltsverzeichnis}%
}

\Hint{v2.02}%
Wird mit \Macro{tableofcontents}(\Bundle{koma-script}) das Inhaltsverzeichnis 
für ein Dokument mit einer dreistelligen Seitenanzahl erstellt, so erscheinen 
unter Umständen viele Warnungen mit der Meldung:
%
\begin{quoting}
\begin{Code}
overfull \hbox
\end{Code}
\end{quoting}
%
Die Seitenzahlen im Verzeichnis werden in einer Box mit einer festen Breite 
von~\PValue{1.55em} gesetzt, welche im Makro \Macro*{@pnumwidth} hinterlegt ist 
und im Zweifel vergrößert werden sollte. Dabei ist auch der rechte Rand für 
\Hint{v2.05}%
mehrzeilige Einträge im Verzeichnis \Macro*{@tocrmarg} zu vergrößern, welcher 
mit~\PValue{2.55em} voreingestellt ist. Die Werte sollten nur minimal geändert 
werden:
%
\begin{quoting}
\begin{Code}
\makeatletter
\renewcommand*{\@pnumwidth}{1.7em}\renewcommand*{\@tocrmarg}{2.7em}
\makeatother
\end{Code}
\end{quoting}



\section{%
  Leer- und Satzzeichen nach \hologo{LaTeX}-Befehlen%
  \label{sec:tips:xspace}%
  \index{Typografie}%
  \index{Befehlsdeklaration}%
}

Normalerweise \enquote{schluckt} \hologo{LaTeX} die Leerzeichen nach einem 
Makro ohne Argumente. Dies ist jedoch nicht immer~-- genau genommen in den 
seltensten Fällen~-- erwünscht. Für dieses Handbuch ist beispielsweise der 
Befehl \Macro*{TUD} definiert worden, um \enquote{\TUD{}} nicht ständig 
ausschreiben zu müssen. Um sich bei der Verwendung des Befehl innerhalb eines 
Satzes für den Erhalt eines folgenden Leerzeichens das Setzen der geschweiften 
Klammer nach dem Befehl zu sparen (\Macro*{TUD}[\MPValue{}]), kann 
\Macro{xspace}(\Package{xspace}) aus dem Paket \Package{xspace} genutzt werden. 
Damit wird ein folgendes Leerzeichen erhalten. Der Befehl \Macro*{TUD} ist wie 
folgt definiert:
%
\begin{quoting}
\begin{Code}
\newcommand*{\TUD}{Technische Universit\"at Dresden\xspace}
\end{Code}
\end{quoting}
%
Das Paket \Package{xpunctuate} erweitert die Funktionalität nochmals. Damit 
können auch Abkürzungen so definiert werden, dass ein versehentlicher Punkt 
ignoriert wird:
%
\begin{quoting}
\begin{Code}
\newcommand*{\zB}{z.\,B\xperiod}
\end{Code}
\end{quoting}



\section{%
  Finden von unbekannten \hologo{LaTeX}-Symbolen%
  \index{Symbole}%
}

Für \hologo{LaTeX} stehen jede Menge Symbole zur Verfügung, die allerdings 
nicht immer einfach zu finden sind. In der Zusammenfassung
\hrfn{http://mirrors.ctan.org/info/symbols/comprehensive/symbols-a4.pdf}%
{\File{symbols-a4.pdf}}
werden viele Symbole aus mehreren Paketen aufgeführt. Alternativ kann 
\hrfn{http://detexify.kirelabs.org/classify.html}{Detexify} verwendet werden. 
Auf dieser Webseite wird das gesuchte Symbol einfach gezeichnet, die dazu 
ähnlichsten werden zurückgegeben.



\section{%
  Setzen von Auslassungspunkten%
  \label{sec:tips:dots}%
  \index{Typografie}%
  \index{Befehlsdeklaration}%
}

\Hint{v2.02}%
Auslassungspunkte werden für \hologo{LaTeX} mit den Befehlen \Macro{dots} oder 
\Macro{textellipsis} gesetzt. Für gewöhnlich folgt diesen \emph{immer} ein 
Leerzeichen, was nicht in jedem Fall gewünscht ist. Abhilfe schafft das Paket 
\Package{ellipsis} mit der Option \Option*{xspace}(\Package{ellipsis}), wodurch 
bei der Verwendung dieser Befehle nur ein Leerzeichen gesetzt wird, wenn nicht 
direkt danach ein Interpunktionszeichen folgt.
%
\begin{quoting}
\begin{Code}
\usepackage[xspace]{ellipsis}
\end{Code}
\end{quoting}
%
Im Ursprung ist es für das Setzen englischsprachiger Texte gedacht, wo zwischen 
Auslassungspunkten und Satzzeichen ein Leerzeichen gesetzt wird. Im Deutschen 
ist dies anders:
%
\begin{quoting}
\enquote{%
  Um eine Auslassung in einem Text zu kennzeichnen, werden drei Punkte gesetzt. 
  Vor und nach den Auslassungspunkten wird jeweils ein Wortzwischenraum 
  gesetzt, wenn sie für ein selbständiges Wort oder mehrere Wörter stehen. Bei 
  Auslassung eines Wortteils werden sie unmittelbar an den Rest des Wortes 
  angeschlossen. Am Satzende wird kein zusätzlicher Schlusspunkt gesetzt. 
  Satzzeichen werden ohne Zwischenraum angeschlossen.%
}~[Duden, 23. Aufl.]
\end{quoting} 
%
Um dieses Verhalten zu erreichen, sollte noch Folgendes in der Präambel 
eingefügt werden:
%
\begin{quoting}
\begin{Code}
\let\ellipsispunctuation\relax
\newcommand*{\qdots}{[\dots{}]\xspace}
\end{Code}
\end{quoting}
%
Der Befehl \Macro*{qdots} wird definiert, um Auslassungspunkte in eckigen 
Klammern (\OPValue{\dots}) setzen zu können, wie sie für das Kürzen von 
wörtlichen Zitaten häufig verwendet werden.



\section{%
  Lokalisierung für das Setzen von Einheiten mit 
  \PackageRaw{siunitx}{\BooleanFalse}%
  \label{sec:tips:siunitx}%
  \index{Einheiten}%
}

Wenn \Package{siunitx} in einem deutschsprachigen Dokument genutzt soll
werden, muss zumindest die richtige Lokalisierung mit
\Macro{sisetup}[\MPValue{locale = DE}](\Package{siunitx}) angegeben werden. 
Sollen auch die Zahlen richtig formatiert sein, müssen weitere Einstellungen 
vorgenommen werden. Die meiner Meinung nach besten sind die folgenden.
%
\begin{quoting}
\begin{Code}
\sisetup{%
  locale = DE,%
  input-decimal-markers={,},input-ignore={.},%
  group-separator={\,},group-minimum-digits=3%
}
\end{Code}
\end{quoting}
%
Das Komma kommt als Dezimaltrennzeichen zum Einsatz. Des Weiteren werden Punkte 
innerhalb der Zahlen ignoriert und eine Gruppierung von jeweils drei Ziffern 
vorgenommen. Alternativ zu diesem Paket kann übrigens auch \Package{units} 
verwendet werden.



\section{%
  Fehler beim Laden eines Paketes mit Optionen%
  \label{sec:tips:options}%
  \index{Optionen|?}%
}

\Hint{v2.05}%
Es kann unter Umständen passieren, dass beim Laden eines Paketes mit bestimmten 
Optionen via \InlineDeclaration{%
  \Macro{usepackage}[\OPName{Paketoptionen}\MPName{Paket}](\Bundle{latex-base})%
} folgender Fehler ausgegeben wird:
%
\begin{quoting}
\begin{Code}
! LaTeX Error: Option clash for package <...>.
\end{Code}
\end{quoting}
%
Wahrscheinlich wird das angeforderte Paket bereits durch die verwendete 
Dokumentklasse oder ein anderes Paket geladen. Normalerweise genügt es, die 
gewünschten Optionen mit
\begin{quoting}
\begin{Entity}{\Bundle{latex-base}}
\InlineDeclaration{%
  \Macro{PassOptionsToPackage}[\MPName{Paketoptionen}\MPName{Paket}]%
}\newline%
\Macro*{documentclass}[\OPName{Klassenoptionen}\MPName{Klasse}]%
\end{Entity}
\end{quoting}
bereits vor dem Laden der Dokumentklasse an das betreffende Paket 
weiterzureichen.



\section{%
  Hinweise zur Verwendung von \PackageRaw{auto-pst-pdf}{\BooleanFalse}%
  \label{sec:tips:auto-pst-pdf}%
  \index{Grafiken}%
}

\Hint{v2.02}%
Bei der Verwendung von \Format{pdfLaTeX} liest das Paket \Package{auto-pst-pdf} 
die Präambel ein und erstellt anschließend über den PostScript-Pfad 
\Path{latex\,>\,dvips\,>\,ps2pdf} eine PDF"~Datei, welche lediglich alle in den 
vorhandenen \Environment{pspicture}(\Package{pstricks})"~Umgebungen erstellten 
Grafiken enthält. Mit dem Befehl \Macro{ifpdf}(\Package{iftex}) aus dem Paket 
\Package{iftex} lässt sich unterscheiden, ob die Ausgabe in eine PDF"~ oder 
DVI"~Datei erfolgt. Folgend wird abhängig vom Ausgabeformat unterschiedlicher 
Quelltext ausgeführt.
%
\begin{quoting}
\begin{Code}
\usepackage{iftex}
\end{Code}
\end{quoting}


\minisec{Die gleichzeitige Verwendung von \Package{floatrow}}

Das Paket \Package{floatrow} stellt Befehle bereit, mit denen die Beschriftung 
von Gleitobjekten sehr bequem gesetzt werden können. Diese Setzen ihren Inhalt 
vor der eigentlichen Ausgabe erst in einer Box, um deren Breite zu ermitteln. 
In Kombination mit \Package{auto-pst-pdf} führt das zu einer doppelten 
Erstellung der gewünschten Abbildung. Um dies zu vermeiden, müssen die durch 
\Package{floatrow} bereitgestellten Makros \enquote{unschädlich} gemacht 
werden. Die fraglichen Befehle akzeptieren allerdings bis zu drei optionale 
Argumente \emph{vor} den beiden obligatorischen, was bei \hologo{LaTeX} für die 
(Re"~)Definition von Befehlen normalerweise nicht vorgesehen ist. Deshalb wird 
das Paket \Package{xparse} geladen, mit welchem dies möglich wird. Genaueres 
dazu ist der dazugehörigen Paketdokumentation zu entnehmen. Mit folgendem 
Quelltextauszug lassen sich die Befehle des Paketes \Package{floatrow} zusammen 
mit der \Environment{pspicture}(\Package{pstricks})"~Umgebung wie gewohnt 
verwenden.
%
\begin{quoting}
\begin{Code}
\usepackage{floatrow}
\usepackage{xparse}
\ifpdf\else
  \RenewDocumentCommand{\fcapside}{ooo+m+m}{#4#5}
  \RenewDocumentCommand{\ttabbox}{ooo+m+m}{#4#5}
  \RenewDocumentCommand{\ffigbox}{ooo+m+m}{#4#5}
\fi
\end{Code}
\end{quoting}


\minisec{Die parallele Nutzung von \Package{tikz} und \Package{todonotes}}

Mit dem Paket \Package{tikz}~-- und auch allen anderen Paketen die 
selbiges nutzen wie beispielsweise \Package{todonotes}~-- gibt es in Verbindung 
mit \Package{auto-pst-pdf} ebenfalls Probleme. Lösen lässt sich dieses Dilemma, 
indem die fraglichen Pakete lediglich geladen werden, wenn \Format{pdfLaTeX} 
aktiv ist.
%
\begin{quoting}[rightmargin=0pt]
\begin{Code}
\ifpdf
  \usepackage{tikz}%\dots gegebenenfalls weitere auf tikz basierende Pakete
\fi
\end{Code}
\end{quoting}



\section{%
  Automatisiertes Einbinden von 
  \ApplicationRaw{Inkscape}{\BooleanFalse}-Grafiken%
  \label{sec:tips:svg}%
  \index{Grafiken}%
}

\Hint{v2.05}%
Das Einbinden von \Application{Inkscape}"=Grafiken in \hologo{LaTeX}"~Dokumente
ist Dank der entsprechenden Exportfunktion prinzipiell möglich. Allerdings muss 
normalerweise relativ viel manueller Aufwand betrieben werden. Die hierfür 
notwendigen Schritte sind auf \CTAN[pkg/svg-inkscape] erläutert. Ein daraus 
abgeleiteter und verbesserter Ansatz wird durch das Paket \Package{svg} 
bereitgestellt. Mit diesem Paket ist ein \textbf{automatisierter} Export der 
\Application{Inkscape}"=Grafiken bei der Kompilierung mit \Format*{LaTeX} 
und dem anschließenden Einbinden in das Dokument möglich. Für diesen Zweck wird 
der Befehl \Macro*{includesvg}[\OPName{Parameter}\MPName{SVG-Datei}] als 
zentrale Benutzerschnittstelle durch das Paket definiert.

Dabei erfolgt der externe Aufruf von \Application{Inkscape} über Kommandozeile 
respektive Terminal mit \Path{inkscape.exe}. Damit dieser auch tatsächlich 
erfolgen kann, ist die Ausführung einer \Format*{LaTeX}"~Engine mit dem 
Parameter \Path{-{}-shell-escape} zwingend notwendig. Außerdem muss der Pfad 
zur Datei \Path{inkscape.exe} dem System bekannt sein.%
\footnote{%
  Der Pfad zu \Path{inkscape.exe} in der Umgebungsvariable 
  \textbf{\texttt{PATH}} des Betriebssystems enthalten sein.%
}
Für weiterführende Informationen sei auf die Dokumentation des Pakets 
\Package{svg} verwiesen.



\section{%
  Änderung des Papierformates%
  \index{Papierformat}%
}

Es kann vorkommen, dass innerhalb eines Dokumentes kurzzeitig das Papierformat 
geändert werden soll, um beispielsweise eine Konstruktionsskizze in der 
digitalen PDF"~Datei einzubinden. Dabei ist es mit der \KOMAScript-Option 
\Option{paper=\PMisc}(\Package{typearea}) sowohl möglich, lediglich die 
Ausrichtung in ein Querformat zu ändern, als auch die Größe des Papierformates 
selber.
%
\begin{Example}
Ein Dokument im A4"~Format soll kurzzeitig auf ein A3"~Querformat geändert 
werden. Das folgende Minimalbeispiel zeigt, wie dies mit \KOMAScript-Mitteln 
über die Optionen \Option*{paper=landscape}(\Package{typearea}) und 
\Option*{paper=A3}(\Package{typearea}) geändert werden kann.
\begin{Code}
\documentclass[paper=A4,pagesize]{tudscrreprt}
\usepackage[T1]{fontenc}
\usepackage[ngerman]{babel}
\usepackage{blindtext}

\begin{document}
\chapter{Überschrift Eins}
\Blindtext

\cleardoublepage
\storeareas\PotraitArea% speichert den aktuellen Satzspiegel
\KOMAoptions{paper=A3,paper=landscape,DIV=current}
\chapter{Überschrift Zwei}
\Blindtext

\cleardoublepage
\PotraitArea% lädt den gespeicherten Satzspiegel
\chapter{Überschrift Drei}
\Blindtext
\end{document}
\end{Code}
\end{Example}



\section{%
  Beschnittzugabe und Schnittmarken%
  \label{sec:tips:crop}%
  \index{Beschnittzugabe|!}%
  \index{Schnittmarken|!}%
  \index{Papierformat}%
}

\Hint{v2.05}%
Beim Plotten von Postern oder anderen farbigen Druckerzeugnissen besteht 
oftmals das Problem, dass ein randloses Drucken nur schwer realisierbar ist. 
Deshalb wird zu oftmals damit beholfen, dass der Druck des fertigen Dokumentes 
auf einem größeren Papierbogen erfolgt und anschließend auf das gewünschte 
Zielformat zugeschnitten wird, womit das Problem des nicht bedruckbaren Randes 
entfällt. Dies kann über zwei verschiedene Wege realisiert werden.

Der einfachste Weg ist die Verwendung des Paketes \Package{crop}. Mit diesem 
kann das Dokument ganz normal im gewünschten Zielformat erstellt werden. Vor 
dem Druck wird dieses Paket geladen und einfach das gewünschte Format des 
Papierbogens angegeben. 
%
\begin{quoting}[rightmargin=0pt]
\begin{Code}
\RequirePackage{fix-cm}
\documentclass[%
  paper=A1,
  fontsize=36pt
]{tudscrposter}
\usepackage[T1]{fontenc}
§\dots§
\usepackage{graphicx}
\usepackage[b1,center,cam]{crop}
\begin{document}
§\dots§
\end{document}
\end{Code}
\end{quoting}
%
\ToDo{check index, remove default, maybe use InlineDeclaration}[v2.07]%
\begin{Entity}{\Package{geometry}}
Alternativ dazu kann für die \TUDScript-Klassen auf die Funktionalität des 
Paketes \Package{geometry} zurückgegriffen werden. Dieses Paket stellt den 
Befehl \Macro{geometry} bereit, in dessen Argument mit
\Key{\Macro{geometry}}{paper=\PSet{Papierformat}} das Papierformat festgelegt 
werden kann. Wird zusätzlich noch der Parameter 
\Key{\Macro{geometry}}{layout=\PSet{Zielformat}} angegeben, so wird damit das 
gewünschte Zielformat definiert. Dabei sollte mit 
\Key{\Macro{geometry}}{layoutoffset=\PLength} dieser Bereich gegebenenfalls 
etwas eingerückt werden. Die Angabe von 
\Key{\Macro{geometry}}{showcrop=\PBoolean} generiert außerdem noch visuelle 
Schnittmarken.
\end{Entity}
%
\begin{quoting}[rightmargin=0pt]
\begin{Code}
\RequirePackage{fix-cm}
\documentclass[%
  paper=A1,
  fontsize=36pt
]{tudscrposter}
\usepackage[T1]{fontenc}
§\dots§
\geometry{paper=b1,layout=a1,layoutoffset=1in,showcrop}
\begin{document}
§\dots§
\end{document}
\end{Code}
\end{quoting}
%
Für genauere Erläuterungen sowie weitere Einstellmöglichkeiten sei auf die 
Dokumentation von \Package{crop} beziehungsweise \Package{geometry} verwiesen.
Mit der \TUDScript-Option \Option{bleedmargin}(\Class{tudscrposter}) können
zusätzlich die farbigen Bereiche der \PageStyle{tudheadings}"~Seitenstile 
erweitert werden, um ein \enquote{Zuschneiden in die Farbe} zu ermöglichen.



\section{%
  Platzierung von Gleitobjekten%
  \label{sec:tips:floats}%
  \index{Gleitobjekte!Platzierung|?}%
  \index{Satzspiegel!zweispaltig}%
}

Mit den beiden Paketen \Package{flafter} sowie \Package{placeins} gibt es die 
Möglichkeit, den für \hologo{LaTeX} zur Verfügung stehenden Raum für die 
Platzierung von Gleitobjekten einzuschränken. Darüber hinaus kann diese auch 
durch die im Folgenden aufgezählten Befehle beeinflusst werden, wobei zwischen 
einspaltigem und zweispaltigen Layout (\Option{twocolumn}(\Package{typearea})) 
unterschieden wird. Die nachfolgend erläuterten Makros lassen sich sehr 
einfach mit \Macro*{renewcommand*}[\MPName{Makro}\MPName{Wert}] ändern. 

\begin{Declaration}
  {\Macro*{floatpagefraction}}
  (0\floatpagefraction)
\begin{Declaration}
  {\Macro*{dblfloatpagefraction}}
  (0\dblfloatpagefraction)
\printdeclarationlist

Der Wert gibt die relative Größe eines Gleitobjektes bezogen auf die Texthöhe 
(\Length*{textheight}) an, die mindestens erreicht sein muss, damit für dieses 
gegebenenfalls vor dem Beginn eines neuen Kapitels eine separate Seite erzeugt 
wird. Dabei wird einspaltiges (\Macro*{floatpagefraction}) und zweispaltiges 
(\Macro*{dblfloatpagefraction}) Layout unterschieden. Der Wert für beide 
Befehle sollte im Bereich von~\PValue{0.5\dots0.8} liegen.
\end{Declaration}
\end{Declaration}

\begin{Declaration}
  {\Macro*{topfraction}}
  (0\topfraction)
\begin{Declaration}
  {\Macro*{dbltopfraction}}
  (0\dbltopfraction)
\printdeclarationlist

Diese Werte geben den maximalen Seitenanteil für Gleitobjekte an, die am oberen 
Seitenrand platziert werden. Er sollte größer als \Macro*{floatpagefraction} 
respektive \Macro*{dblfloatpagefraction} (zweispaltiges Layout) sein und im 
Bereich von \PValue{0.5\dots0.8} liegen. 
\end{Declaration}
\end{Declaration}

\begin{Declaration}
  {\Macro*{bottomfraction}}
  (0\bottomfraction)
\printdeclarationlist

Dies ist der maximale Seitenanteil für Gleitobjekte, die am unteren Seitenrand 
platziert werden. Er sollte zwischen~\PValue{0.2} und~\PValue{0.5} betragen.
\end{Declaration}

\begin{Declaration}
  {\Macro*{textfraction}}
  (0\textfraction)
\printdeclarationlist

Dies ist der notwendige Mindestanteil an Fließtext auf einer Seite mit 
Gleitobjekten, damit diese nicht auf einer separaten Seite ausgegeben werden. 
Er sollte im Bereich von~\PValue{0.1}\dots\PValue{0.3} liegen.
\end{Declaration}

\begin{Declaration}{\Length*{textfloatsep}}[\the\textfloatsep]
\begin{Declaration}{\Length*{dbltextfloatsep}}[\the\dbltextfloatsep]
\begin{Declaration}{\Length*{intextsep}}[\the\intextsep]
\begin{Declaration}{\Length*{floatsep}}[\the\floatsep]
\begin{Declaration}{\Length*{dblfloatsep}}[\the\dblfloatsep]
\printdeclarationlist%
%
Die Längen \Length*{textfloatsep} und \Length*{dbltextfloatsep} (zweispaltig) 
werden zwischen Fließtext und dem ersten Gleitobjekt am Ende beziehungsweise 
dem letzten am Anfang einer Seite eingefügt. Die Länge \Length*{intextsep} 
bestimmt den Abstand oberhalb und unterhalb von Gleitobjekten, wenn die Ausgabe 
innerhalb des Fließtextes forciert wird. Mehrere Gleitobjekte nacheinander 
werden mit \Length*{textfloatsep} respektive \Length*{dbltextfloatsep} 
(zweispaltig) voneinander separiert.
\end{Declaration}
\end{Declaration}
\end{Declaration}
\end{Declaration}
\end{Declaration}

\begin{Declaration}
  {\Counter*{totalnumber}}
  (\arabic{totalnumber})
\begin{Declaration}
  {\Counter*{topnumber}}
  (\arabic{topnumber})
\begin{Declaration}
  {\Counter*{dbltopnumber}}
  (\arabic{dbltopnumber})
\begin{Declaration}
  {\Counter*{bottomnumber}}
  (\arabic{bottomnumber})
\printdeclarationlist[Zähler]

Außerdem gibt es noch Zähler, welche die maximale Anzahl an Gleitobjekten pro 
Seite insgesamt (\Counter*{totalnumber}) sowie am oberen (\Counter*{topnumber}) 
und am unteren Seitenrand (\Counter*{bottomnumber}) sowie im zweispaltigen 
Satz beide Spalten überspannend (\Counter*{dbltopnumber}) festlegen. Die Werte 
können mit \Macro*{setcounter}[\MPName{Zähler}\MPName{Wert}] geändert werden.
\end{Declaration}
\end{Declaration}
\end{Declaration}
\end{Declaration}

\begin{Declaration}
  {\Length*{@fptop}}
\begin{Declaration}
  {\Length*{@fpsep}}
\begin{Declaration}
  {\Length*{@fpbot}}
\begin{Declaration}
  {\Length*{@dblfptop}}
\begin{Declaration}
  {\Length*{@dblfpsep}}
\begin{Declaration}
  {\Length*{@dblfpbot}}
\printdeclarationlist[Längen]

Sind vor Beginn eines Kapitels noch Gleitobjekte verblieben, so werden diese 
von \hologo{LaTeX} normalerweise auf einer separaten Seite vertikal zentriert 
ausgegeben. Dabei bestimmen diese Längen jeweils den Abstand vor dem ersten 
Gleitobjekt zum oberen Seitenrand (\Length*{@fptop}, \Length*{@dblfptop}), 
zwischen den einzelnen Objekten (\Length*{@fpsep}, \Length*{@dblfpsep}) sowie 
zum unteren Seitenrand (\Length*{@fpbot}, \Length*{@dblfpbot}). Soll dies nicht 
geschehen, können die Längen durch den Anwender geändert werden.
\end{Declaration}
\end{Declaration}
\end{Declaration}
\end{Declaration}
\end{Declaration}
\end{Declaration}
%
\begin{Example}
Alle Gleitobjekte auf einer dafür speziell gesetzten Seite sollen direkt zu 
Beginn dieser ausgegeben werden. In der Dokumentpräambel lässt sich für dieses 
Unterfangen Folgendes nutzen:
\begin{Code}
\makeatletter
\setlength{\@fptop}{0pt}
\setlength{\@dblfptop}{0pt}% twocolumn
\makeatother
\end{Code}
\end{Example}



\section{%
  Fehler beim Schreiben von Hilfsdateien%
  \label{sec:tips:write}%
}

\Hint{v2.02}%
Für das Erstellen und Schreiben externer Hilfsdateien steht \hologo{LaTeX} nur 
eine begrenzte Anzahl sogenannter Ausgabe-Streams zur Verfügung. Allein für 
jedes zu erstellende Verzeichnis reserviert \hologo{LaTeX} selbst jeweils einen 
neuen Stream. Auch einige bereits zuvor in diesem Handbuch vorgestellte, sehr 
hilfreiche Pakete~-- wie beispielsweise \Package{hyperref}, \Package{biblatex}, 
\Package{glossaries}, \Package{todonotes} oder auch \Package{filecontents}~-- 
benötigen eigene Hilfsdateien und öffnen für das Erstellen dieser einen 
Ausgabe-Stream oder mehr. Lädt der Anwender mehrere, in eine Hilfsdatei 
schreibende Pakete, kann es zur folgender Fehlermeldung kommen:
%
\begin{quoting}
\begin{Code}
! No room for a new \write .
\end{Code}
\end{quoting}
%
Abhilfe schafft das Paket \Package{morewrites}, welches die Ausgabe der Streams 
in eine Hilfsdatei umleitet. Es sollte möglichst frühzeitig innerhalb der 
Präambel geladen werden. Nur sehr selten ist die Verwendung des Paketes nicht 
von Erfolg gekrönt. 

Für diesen Fall lässt sich das Paket \Package{scrwfile}(\Bundle{koma-script}) 
als Alternative nutzen, welches einige Änderungen am \hologo{LaTeXe}"~Kernel 
vornimmt. Das Ziel dieses Paketes ist es, zumindest die Anzahl der benötigten 
Hilfsdateien für das Schreiben \emph{aller Verzeichnisse} zu reduzieren und 
somit nur einen einzigen Stream hierfür zu verwenden. Auch dieses sollte gleich 
zu Beginn der Präambel eingebunden werden. Der Entwicklungsstand des Paketes 
ist als experimentell eingestuft, auch wenn es meiner Erfahrung nach fehlerfrei 
arbeitet. Sollten dennoch Probleme auftreten, ist die Anleitung des Paketes im 
\scrguide zu finden. 



\section{%
  Warnung bei der Schriftgrößenwahl%
  \label{sec:tips:fontsize}%
  \index{Schriftart}%
  \index{Schriftgröße}%
}

\Hint{v2.04}%
Die im Dokument verwendete Schriftgröße lässt sich bei den \KOMAScript-Klassen 
einfach über die Option~\Option{fontsize}(\Bundle{koma-script}) einstellen, 
wobei diese \emph{immer} als Klassenoption anzugeben ist. Werden mit der Option 
relativ großen oder kleinen Schriftgrößen angegeben, kann dabei die folgende 
Warnung auftreten:
%
\begin{quoting}[rightmargin=0pt]
\begin{Code}
LaTeX Font Warning: Font shape `...' in size <xx> not available
\end{Code}
\end{quoting}
%
Dies liegt darin begründet, dass zum Zeitpunkt des Ladens einer Klasse zunächst 
immer nach den Standardschriften der Computer~Modern gesucht wird, unabhängig 
davon, ob im Nachhinein ein anderes Schriftpaket Anwendung findet. Die 
ursprünglichen Schriften der Computer~Modern sind de"~facto nicht auf alle 
Größen skalierbar. Um die genannten Warnungen zu beseitigen, sollte \emph{vor} 
der Dokumentklasse das Paket \Package{fix-cm} geladen werden:
%
\begin{quoting}[rightmargin=0pt]
\begin{Code}
\RequirePackage{fix-cm}
\documentclass§\OPName{Klassenoptionen}\MPName{Klasse}§
\usepackage[T1]{fontenc}
§\dots§
\begin{document}
§\dots§
\end{document}
\end{Code}
\end{quoting}



\section{%
  Vermeiden des Skalierens einer PDF-Datei beim Druck%
}

\Hint{v2.04}%
Beim Erzeugen eines Druckauftrages einer PDF"~Datei kann es unter Umständen 
dazu führen, dass diese durch den verwendeten PDF"~Betrachter unnötigerweise 
vorher skaliert wird und dabei die Seitenränder vergrößert werden. Um dieses 
Verhalten für Dokumente, die mit \Format{pdfLaTeX} erzeugt werden, zu 
unterdrücken, gibt es zwei Möglichkeiten:
%
\begin{enumerate}
\item Wenn im Dokument ohnehin das Paket \Package{hyperref} verwendet wird, 
  ist der simple Aufruf von 
  \Macro{hypersetup}[\MPValue{pdfprintscaling=None}](\Package{hyperref})
  ausreichend.
\item Der Low"~Level"~Befehl
  \Macro*{pdfcatalog}[\MPValue{/ViewerPreferences<{}</PrintScaling/None>{}>}]
  besitzt das gleiche Verhalten und lässt sich auch ohne \Package{hyperref} 
  nutzen.
\end{enumerate}
%
Weitere Informationen sind unter \url{http://www.komascript.de/node/1897} 
zu finden.
