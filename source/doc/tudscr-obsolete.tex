\chapter{%
  Obsolete sowie vollständig entfernte Optionen und Befehle%
  \label{sec:obsolete}%
}
\section{%
  Veraltete Optionen und Befehle in \TUDScript%
  \index{Änderungen|!}%
  \index{Kompatibilität}%
}
Einige Optionen und Befehle waren während der Weiterentwicklung von \TUDScript
in ihrer ursprünglichen Form nicht mehr umsetzbar oder wurden~-- unter anderem 
aus Gründen der Kompatibilität zu anderen Paketen~-- schlichtweg verworfen. 
Dennoch besteht für die meisten entfallenen Direktiven eine Möglichkeit, deren 
Funktionalität ohne größere Aufwände mit \TUDScript in der aktuellen Version 
\vTUDScript{} darzustellen. Ist dies der Fall, wird hier entsprechend kurz 
darauf hingewiesen.

\NewDocumentCommand\ChangesTo{o m}{%
  \subsection{%
    Änderungen für \TUDScript~#2%
    \label{sec:obsolete:#2}%
    \index[changelog]{%
      #2!Allgemein@Allgemein: \hyperref[{sec:obsolete:#2}]{%
        Änderungen gegenüber der vorhergehenden Version%
      }%
    }%
    \IfValueT{#1}{\ChangedAt*{#2:#1}}%
  }%
}


\ChangesTo{v2.00}
%
\begin{Obsolete}{v2.00}{\Option{cd=alternative}}
\begin{Obsolete}{v2.00}{\Option{cdtitle=alternative}}
\begin{Obsolete}{v2.00}{\Length{titlecolwidth}}
\begin{Obsolete}{v2.00}{\Term{authortext}}
\printdeclarationlist%
%
\ToDo[doc]{keine Links im Index}[v2.07]
Die alternative Titelseite ist komplett aus dem \TUDScript-Bundle entfernt 
worden. Dementsprechend entfallen auch die dazugehörigen Optionen sowie Länge 
und Bezeichner.
\end{Obsolete}
\end{Obsolete}
\end{Obsolete}
\end{Obsolete}

\begin{Obsolete}{v2.00}{\Option{color=\PBoolean}}'\Option*{cd}'
\printdeclarationlist%
%
Die Einstellungen der farbigen Ausprägung des Dokumentes erfolgt über die 
Option \Option*{cd}.
\end{Obsolete}

\begin{Obsolete}{v2.00}{\Option{tudfonts=\PBoolean}}'\Option*{cdfont}'
\printdeclarationlist%
%
Die Option zur Schrifteinstellung ist wesentlich erweitert worden. Aus Gründen 
der Konsistenz wurde diese umbenannt.
\end{Obsolete}

\begin{Obsolete}{v2.00}{\Option{tudfoot=\PBoolean}}'\Option*{cdfoot}'
\printdeclarationlist%
%
Ebenso wurde die Option für den Seitenfuß umbenannt, um dem Namensschema der 
restlichen Optionen von \TUDScript zu entsprechen.
\end{Obsolete}

\begin{Obsolete}{v2.00}{\Option{headfoot=\PMisc}}(%
  \seeref{\KOMAScript-Optionen \Option*{headinclude}(\Package{typearea}) 
  und \Option*{footinclude}(\Package{typearea})}%
)
\printdeclarationlist%
%
Diese Option war für \TUDScript in der \emph{Version~v1.0} notwendig, um die 
parallele Verwendung der Pakete \Package*{typearea} und \Package*{geometry} zu 
ermöglichen. Die Erstellung des Satzspiegels wurde komplett überarbeitet. 
Mittlerweile werden an das Paket \Package*{geometry} die Einstellungen der 
\KOMAScript-Optionen \Option*{headinclude}(\Package{typearea}) und 
\Option*{footinclude}(\Package{typearea}) direkt weitergereicht, sodass diese 
Option nicht mehr notwendig ist und deshalb entfernt wurde.
\end{Obsolete}

\begin{Obsolete}{v2.00}{\Option{partclear=\PBoolean}}%
  '\Option*{cleardoublespecialpage}'%
\begin{Obsolete}{v2.00}{\Option{chapterclear=\PBoolean}}%
  '\Option*{cleardoublespecialpage}'%
\printdeclarationlist%
%
Beide Optionen sind in der neuen Option \Option*{cleardoublespecialpage} 
aufgegangen, womit ein konsistentes Layout erreicht wird. Die ursprünglichen 
Optionen entfallen. 
\end{Obsolete}
\end{Obsolete}

\begin{Obsolete}{v2.00}{\Option{abstracttotoc=\PBoolean}}'\Option*{abstract}'
\begin{Obsolete}{v2.00}{\Option{abstractdouble=\PBoolean}}'\Option*{abstract}'
\printdeclarationlist%
%
Beide Optionen wurden in die Option \Option*{abstract} integriert und sind 
deshalb überflüssig.
\end{Obsolete}
\end{Obsolete}

\begin{Obsolete}{v2.00}{\Macro{logofile}[\MPName{Dateiname}]}%
  '\Macro*{headlogo}'%
\printdeclarationlist%
%
Dieser Befehl wurde in \Macro*{headlogo} umbenannt, wobei die Funktionalität 
weiterhin bestehen bleibt.
\end{Obsolete}

\begin{Obsolete}{v2.00}{\Option{bookmarks=\PBoolean}}'\Option*{tudbookmarks}'
\printdeclarationlist%
%
Die Option wurde umbenannt, um Überschneidungen mit \Package*{hyperref} zu 
vermeiden.
\end{Obsolete}

\begin{Obsolete}{v2.00}{\Length{signatureheight}}
\printdeclarationlist%
%
Die Höhe für die Zeile der Unterschriften wurde dehnbar gestaltet, eine etwaige 
Anpassung durch den Anwender ist nicht vonnöten.
\end{Obsolete}

\begin{Obsolete}{v2.00}{\Term{titlecoldelim}}'\Macro*{titledelimiter}'
\printdeclarationlist%
%
Das Trennzeichen für Bezeichnungen beziehungsweise beschreibende Texte und dem 
eigentlichen Feld auf der Titelseite ist nicht mehr sprachabhängig und wurde 
umbenannt.
\end{Obsolete}

\begin{Obsolete}{v2.00}{\Macro{confirmationandrestriction}}%
  '\Macro*{declaration}'
\begin{Obsolete}{v2.00}{\Macro{restrictionandconfirmation}}%
  '\Macro*{declaration}'
\printdeclarationlist%
%
Die beiden Befehle entfallen, stattdessen sollte entweder der Befehl 
\Macro*{declaration} oder die Umgebung \Environment*{declarations} zusammen mit 
den Befehlen \Macro*{confirmation} und \Macro*{blocking} verwendet werden, 
wobei sich diese in der Umgebung in beliebiger Reihenfolge anordnen lassen.
\end{Obsolete}
\end{Obsolete}

\begin{Obsolete}{v2.00}{\Macro{location}[\MPName{Ort}]}'\Macro*{place}'
\printdeclarationlist%
%
In Anlehnung an andere \hologo{LaTeX}"~Pakete und "~Klassen wurde 
dieser Befehl in \Macro*{place} umbenannt.
\end{Obsolete}

\minisec{\taskname}
%
\begin{Entity}{\Package{tudscrsupervisor}}
Die Umgebung für die Erstellung einer Aufgabenstellung für eine 
wissenschaftliche Arbeit wurde in das Paket \Package{tudscrsupervisor} 
ausgelagert. Dieses muss für die Verwendung der Umgebung \Environment*{task} 
und der daraus abgeleiteten standardisierten Form zwingend geladen werden.

\begin{Obsolete}{v2.00}{\Option{cdtask=\PMisc}}'\Environment*{task}'
\begin{Obsolete}{v2.00}{\Option{taskcompact=\PBoolean}}
\begin{Obsolete}{v2.00}{\Length{taskcolwidth}}
\printdeclarationlist%
%
Die Klassenoption \Option*{cdtask}'none' ist komplett entfernt worden, alle 
Einstellungen, erfolgen direkt über das optionale Argument der Umgebung 
\Environment*{task}. Die Variante eines kompakten Kopfes mit der Option 
\Option*{taskcompact}'none' wird nicht mehr bereitgestellt. Die Möglichkeit zur 
manuellen Festlegung der Spaltenbreite für den Kopf der Aufgabenstellung mit 
\Length*{taskcolwidth}'none' wurde aufgrund der verbesserten automatischen 
Berechnung entfernt.
\end{Obsolete}
\end{Obsolete}
\end{Obsolete}

\begin{Obsolete}{v2.00}{\Macro{tasks}[\MPName{Ziele}\MPName{Schwerpunkte}]}%
  '\Macro*{taskform}'
\begin{Obsolete}{v2.00}{\Term{focustext}}'\Term*{focusname}'
\begin{Obsolete}{v2.00}{\Term{objectivestext}}'\Term*{objectivesname}'
\printdeclarationlist%
%
Dieser Befehl wurde in \Macro*{taskform} umbenannt und in der Funktionalität 
erweitert. Die Namen der darin verwendeten Bezeichner wurden ebenfalls leicht 
abgewandelt.
\end{Obsolete}
\end{Obsolete}
\end{Obsolete}

\begin{Obsolete}{v2.00}{\Macro{studentid}[\MPName{Matrikelnummer}]}%
  '\Macro*{matriculationnumber}'
\begin{Obsolete}{v2.00}{\Macro{enrolmentyear}[\MPName{Matrikeljahr}]}%
  '\Macro*{matriculationyear}'
\begin{Obsolete}{v2.00}{\Macro{submissiondate}[\MPName{Datum}]}'\Macro*{date}'
\begin{Obsolete}{v2.00}{\Macro{birthday}[\MPName{Geburtsdatum}]}%
  '\Macro*{dateofbirth}'
\begin{Obsolete}{v2.00}{\Macro{birthplace}[\MPName{Geburtsort}]}%
  '\Macro*{placeofbirth}'
\begin{Obsolete}{v2.00}{\Macro{startdate}[\MPName{Ausgabedatum}]}%
  '\Macro*{issuedate}'
\printdeclarationlist%
%
Alle Befehle wurden umbenannt und sind jetzt neben der \taskname{} auch für die 
Titelseite im \CD nutzbar.
\end{Obsolete}
\end{Obsolete}
\end{Obsolete}
\end{Obsolete}
\end{Obsolete}
\end{Obsolete}

\begin{Obsolete}{v2.00}{\Term{studentidname}}'\Term*{matriculationnumbername}'
\begin{Obsolete}{v2.00}{\Term{enrolmentname}}'\Term*{matriculationyearname}'
\begin{Obsolete}{v2.00}{\Term{submissiontext}}'\Term*{datetext}'
\begin{Obsolete}{v2.00}{\Term{birthdaytext}}'\Term*{dateofbirthtext}'
\begin{Obsolete}{v2.00}{\Term{birthplacetext}}'\Term*{placeofbirthtext}'
\begin{Obsolete}{v2.00}{\Term{supervisorIIname}}'\Term*{supervisorothername}'
\begin{Obsolete}{v2.00}{\Term{defensetext}}'\Term*{defensedatetext}'
\begin{Obsolete}{v2.00}{\Term{starttext}}'\Term*{issuedatetext}'
\begin{Obsolete}{v2.00}{\Term{duetext}}'\Term*{duedatetext}'
\printdeclarationlist%
%
Die Bezeichner wurden in Anlehnung an die dazugehörigen Befehlsnamen umbenannt.
\end{Obsolete}
\end{Obsolete}
\end{Obsolete}
\end{Obsolete}
\end{Obsolete}
\end{Obsolete}
\end{Obsolete}
\end{Obsolete}
\end{Obsolete}
\end{Entity}


\ChangesTo[%
  Umbenennung einiger Befehle für Kompatibilität mit anderen Paketen%
]{v2.02}
%
\begin{Obsolete}{v2.02}{\Length{chapterheadingvskip}}%
  '\Option*{pageheadingsvskip}'
\printdeclarationlist%
%
Die vertikale Positionierung von Überschriften wurde aufgeteilt. Zum einen kann 
diese für Titel"~, Teile- und Kapitelseiten (\Option*{chapterpage=true}) über 
die Option \Option*{pageheadingsvskip} geändert werden. Für den Titelkopf
(\Option*{titlepage=false}(\Bundle{koma-script})) sowie Kapitelüberschriften 
(\Option*{chapterpage=false}) kann dies mit \Option*{headingsvskip} unabhängig 
davon erfolgen.
\end{Obsolete}

\begin{Obsolete}{v2.02}{\Macro{degree}[\OPName{Abk.}\MPName{Grad}]}%
  '\Macro*{graduation}'
\begin{Obsolete}{v2.02}{\Term{degreetext}}'\Term*{graduationtext}'
\printdeclarationlist%
%
Der Befehl wurde zur Erhöhung der Kompatibilität mit anderen Paketen umbenannt, 
der dazugehörige Bezeichner dahingehend angepasst.
\end{Obsolete}
\end{Obsolete}

\begin{Obsolete}{v2.02}{\Macro{restriction}[\OPList{Firma}]}%
  '\Macro*{blocking}'
\begin{Obsolete}{v2.02}{\Term{restrictionname}}'\Term*{blockingname}'
\begin{Obsolete}{v2.02}{\Term{restrictiontext}}'\Term*{blockingtext}'
\printdeclarationlist%
%
Der Befehl wurde zur Erhöhung der Kompatibilität mit anderen Paketen umbenannt, 
die dazugehörigen Bezeichner dahingehend angepasst.
\end{Obsolete}
\end{Obsolete}
\end{Obsolete}

\begin{Declaration'}{\Environment{tudpage}[\OList]}
\begin{Obsolete}{v2.02}{\Key{\Environment{tudpage}}{head=\PMisc}}%
  '\Key*{\Environment{tudpage}}{pagestyle}'
\begin{Obsolete}{v2.02}{\Key{\Environment{tudpage}}{foot=\PMisc}}%
  '\Key*{\Environment{tudpage}}{pagestyle}'
\printdeclarationlist%
%
Die Funktionalität beider Parameter wurde durch den Parameter 
\Key*{\Environment{tudpage}}{pagestyle} ersetzt.
\end{Obsolete}
\end{Obsolete}
\end{Declaration'}



\minisec{Änderungen im Paket \Package{tudscrsupervisor}}
%
Im Paket \Package{tudscrsupervisor} gab es ein paar kleinere Anpassungen.
\begin{Entity}{\Package{tudscrsupervisor}}
\begin{Obsolete}{v2.02}{\Macro{branch}[\MPName{Studienrichtung}]}%
  '\Macro*{discipline}()'
\begin{Obsolete}{v2.02}{\Term{branchname}}'\Term*{disciplinename}'
\printdeclarationlist%
%
Für die \taskname{} wurden der Befehl sowie der dazugehörige Bezeichner 
umbenannt.
\end{Obsolete}
\end{Obsolete}

\begin{Obsolete}{v2.02}{\Macro{contact}[\MPName{Kontaktperson(en)}]}%
  '\Macro*{contactperson}'
\begin{Obsolete}{v2.02}{\Term{contactname}}'\Term*{contactpersonname}'
\begin{Obsolete}{v2.02}{\Macro{phone}[\MPName{Telefonnummer}]}%
  '\Macro*{telephone}'
\begin{Obsolete}{v2.02}{\Macro{email}[\MPName{E-Mail-Adresse}]}%
  '\Macro*{emailaddress}'
\printdeclarationlist%
%
Alle genannten Befehle und Bezeichner wurden für den \noticename{} umbenannt.
\end{Obsolete}
\end{Obsolete}
\end{Obsolete}
\end{Obsolete}
\end{Entity}


\ChangesTo{v2.03}
%
\begin{Obsolete}{v2.03}{\Option{geometry=\PBoolean}}'\Option*{cdgeometry}'
\printdeclarationlist%
%
Diese Option wurde zur Konsistenz sowie dem Vermeiden eines möglichen 
Konfliktes mit einer späteren \KOMAScript-Version umbenannt. Die Funktionalität 
bleibt bestehen.
\end{Obsolete}

\begin{Obsolete}{v2.03}{\Option{cdfonts=\PBoolean}}'\Option*{cdfont}'
\begin{Obsolete}{v2.03}{\Option{din=\PBoolean}}'\Option*{cdfont}'
\printdeclarationlist%
%
Die Option \Option*{cdfont} wurde erweitert und fungiert als zentrale 
Schnittstelle zur Schrifteinstellung. 
\end{Obsolete}
\end{Obsolete}

\begin{Obsolete}{v2.03}{\Option{sansmath=\PBoolean}}'\Option*{cdmath}'
\printdeclarationlist%
%
Diese Option wurde aus Gründen der Konsistenz umbenannt. Zusätzlich wurde die 
Funktionalität erweitert.
\end{Obsolete}

\begin{Obsolete}{v2.03}{\Option{barfont=\PMisc}}'\Option*{cdhead}'
\begin{Obsolete}{v2.03}{\Option{widehead=\PBoolean}}'\Option*{cdhead}'
\printdeclarationlist%
%
Die Optionen \Option*{barfont}'none' und \Option*{widehead}'none' wurden in der 
Option \Option*{cdhead} zusammengefasst.
\end{Obsolete}
\end{Obsolete}

\begin{Declaration'}{\Environment{tudpage}[\OPList{Sprache}]}
\begin{Obsolete}{v2.03}{\Key{\Environment{tudpage}}{color=\PSet{Farbe}}}
\printdeclarationlist%
%
Der Parameter \Key*{\Environment{tudpage}}{color=\PSet{Farbe}}'none' der 
\Environment*{tudpage}"~Umgebung wurde ersatzlos entfernt.
\end{Obsolete}
\end{Declaration'}


\ChangesTo{v2.04}
%
\begin{Obsolete}{v2.04}{\Option{fontspec=\PBoolean}}%
\printdeclarationlist%
%
Anstatt diese Option zu aktivieren, kann einfach das Paket \Package{fontspec} 
in der Dokumentpräambel geladen werden. Dadurch lassen sich anschließend 
zusätzliche Pakete nutzen, welche auf die Verwendung von \Package{fontspec} 
angewiesen sind. Sollte diese Option dennoch genutzt werden, müssen alle auf 
das Paket \Package{fontspec} aufbauenden Einstellungen durch den Anwender mit 
\Macro*{AfterPackage}[\MPValue{fontspec}\MPValue{\dots}](\Bundle{koma-script}) 
verzögert werden. In \fullref{sec:fonts} sind weitere Hinweise zur Verwendung 
des Paketes \Package{fontspec} zu finden.
\end{Obsolete}


\ChangesTo{v2.05}
%
\begin{Obsolete}{v2.05}{\Length{pageheadingsvskip}}'\Option*{pageheadingsvskip}'
\begin{Obsolete}{v2.05}{\Length{headingsvskip}}'\Option*{headingsvskip}'
\printdeclarationlist%
%
Die vertikale Positionierung von speziellen Überschriften erfolgt nicht mehr 
über diese beiden Längen sondern über die Optionen \Option*{headingsvskip} 
sowie \Option*{pageheadingsvskip}.
\end{Obsolete}
\end{Obsolete}


\begin{Obsolete}{v2.05}{\Length{footlogoheight}}'\Option*{footlogoheight}'
\printdeclarationlist%
%
Auch die Höhe der Logos im Fußbereich der \PageStyle*{tudheadings}"=Seitenstile 
wird von nun an mit der Option \Option*{footlogoheight} und nicht mehr mit 
dieser Länge festgelegt.
\end{Obsolete}


\ChangesTo{v2.06}
%
\begin{Declaration}[v2.06]{\Option{cdoldfont=\PMisc}}[false]
\printdeclarationlist%
%
Mit der Version~v2.06 wird standardmäßig \OpenSans als Hausschrift verwendet. 
Um jedoch weiterhin ältere Dokumente mit den Schriften \Univers und \DIN 
erzeugen zu können, wird diese Option bereitgestellt.
\Attention{%
  Diese kann ausschließlich als Klassenoption~-- oder für die Pakete 
  \Package*{tudscrfonts} und \Package*{fix-tudscrfonts} als Paketoption~-- 
  genutzt werden.
} Eine späte Optionenwahl mit \Macro*{TUDoption} oder \Macro*{TUDoptions} ist 
nicht möglich. Wurden mit \Option{cdoldfont=true} die alten Schriftfamilien 
aktiviert, kann jedoch weiterhin die Option \Option*{cdfont} wie gewohnt 
genutzt werden.
%
\begin{DeclarationValues}
\itemvalfalse
  Das Verhalten ist äquivalent zu \Option*{cdfont=false}, die Hausschrift ist 
  nicht aktiv.
\itemvaltrue*
  Es werden die alten Hausschriften \Univers für den Fließtext sowie \DIN für 
  Überschriften der obersten Gliederungsebenen bis einschließlich 
  \Macro*{subsubsection} genutzt. Die Schriftstärke lässt sich mit 
  \Option*{cdfont=true} respektive \Option*{cdfont=heavy} anpassen.
\end{DeclarationValues}
%
Für die \TUDScript-Klassen sowie die vom Paket \Package*{fix-tudscrfonts} 
unterstützten Dokumentklassen kann die für die Gliederungsebenen verwendete 
Schriftart angepasst werden.
%
\begin{DeclarationValues}
\itemval=nodin=
  Für Überschriften wird \Univers anstatt \DIN verwendet.
\itemval=din=
  Mit dieser Einstellung wird \DIN in Überschriften genutzt. 
\itemval=onlydin=
  Überschriften werden in \DIN gesetzt, für den Fließtext wird nicht \Univers 
  sondern die \hologo{LaTeX}"=Standardschriften respektive die eines geladenen 
  Schriftpaketes verwendet.
\end{DeclarationValues}
\end{Declaration}

\begin{Obsolete}{v2.06}{\Option{cdfont=din}}'\Option*{cdoldfont=din}'
\begin{Obsolete}{v2.06}{\Option{cdfont=nodin}}'\Option*{cdoldfont=nodin}'
\printdeclarationlist%
%
Die Einstellungen für Überschriften sind mit der Umstellung auf \OpenSans nicht 
mehr notwendig. Zur Verwendung der alten Schriftfamilien muss die Option 
\Option{cdoldfont} aktiviert werden.
\end{Obsolete}
\end{Obsolete}

\begin{Obsolete}{v2.06}{%
  \Macro{ifdin}[\MPName{Dann-Teil}\MPName{Sonst-Teil}]
}(nur definiert, falls \Option*{cdoldfont=true})
\printdeclarationlist%
%
Dieser Befehl prüft, ob aktuell die Schriftfamilie \DIN aktiv ist und führt in 
diesem Fall \MPName{Dann-Teil} aus, andernfalls \MPName{Sonst-Teil}. 
\end{Obsolete}



\minisec{Auszeichnungen in Überschriften}
%
Die Überschriften aller Gliederungsebenen von \Macro*{part} bis einschließlich 
\Macro*{subsubsection} werden in Majuskeln der Schrift \DIN gesetzt, wenn dies 
mit \Option*{cdoldfont=true} respektive \Option*{cdoldfont=onlydin} aktiviert 
wurde. Hierfür wird intern \Macro*{MakeTextUppercase}(\Package{textcase}) aus 
dem Paket \Package{textcase} genutzt, welches zusammen mit den alten 
Schriftfamilien geladen wird. Sollen bestimmte Minuskeln erhalten bleiben, ist 
\Macro*{NoCaseChange}(\Package{textcase}) zu verwenden.
%
\begin{Example}
In einem Kapitel wird ein einzelnes Wort in Minuskeln geschrieben:
\begin{Code}[escapechar=§]
\chapter{§Ü§berschrift mit \NoCaseChange{kleinem} Wort}
\end{Code}
\end{Example}
%
Die Schrift \DIN durfte laut \CD nur mit Majuskeln verwendet werden, weshalb 
das beschriebene Vorgehen lediglich im \emph{Ausnahmefall} anzuwenden ist. 
Die manuellen Nutzung sollte mit 
\Macro*{MakeTextUppercase}[%
  \MPValue{\Macro*{textdbn}[\MPName{Text}]}%
](\Package{textcase}) geschehen.

\begin{Obsolete}{v2.06}{\Macro{univln}}
\begin{Obsolete}{v2.06}{\Macro{textuln}[\MPName{Text}]}
\begin{Obsolete}{v2.06}{\Macro{univrn}}
\begin{Obsolete}{v2.06}{\Macro{texturn}[\MPName{Text}]}
\begin{Obsolete}{v2.06}{\Macro{univbn}}
\begin{Obsolete}{v2.06}{\Macro{textubn}[\MPName{Text}]}
\begin{Obsolete}{v2.06}{\Macro{univxn}}
\begin{Obsolete}{v2.06}{\Macro{textuxn}[\MPName{Text}]}
\begin{Obsolete}{v2.06}{\Macro{univls}}
\begin{Obsolete}{v2.06}{\Macro{textuls}[\MPName{Text}]}
\begin{Obsolete}{v2.06}{\Macro{univrs}}
\begin{Obsolete}{v2.06}{\Macro{texturs}[\MPName{Text}]}
\begin{Obsolete}{v2.06}{\Macro{univbs}}
\begin{Obsolete}{v2.06}{\Macro{textubs}[\MPName{Text}]}
\begin{Obsolete}{v2.06}{\Macro{univxs}}
\begin{Obsolete}{v2.06}{\Macro{textuxs}[\MPName{Text}]}
\begin{Obsolete}{v2.06}{\Macro{dinbn}}
\begin{Obsolete}{v2.06}{\Macro{textdbn}[\MPName{Text}]}
\printdeclarationlist%
%
Wird die Option \Option{cdoldfont} nicht aktiviert, werden auch die Befehle zur 
expliziten Auswahl eines Schriftschnittes nicht mehr bereitgestellt. 
Stattdessen können \Macro*{cdfont} oder \Macro*{textcd}[\MPName{Text}] 
genutzt werden, welche in \fullref{sec:text} zu finden sind.
\end{Obsolete}
\end{Obsolete}
\end{Obsolete}
\end{Obsolete}
\end{Obsolete}
\end{Obsolete}
\end{Obsolete}
\end{Obsolete}
\end{Obsolete}
\end{Obsolete}
\end{Obsolete}
\end{Obsolete}
\end{Obsolete}
\end{Obsolete}
\end{Obsolete}
\end{Obsolete}
\end{Obsolete}
\end{Obsolete}

\begin{Obsolete}{v2.06}{\Option{clearcolor=\PBoolean}}%
  '\Option*{cleardoublespecialpage}'
\printdeclarationlist%
%
Diese Option wurde zur Vereinfachung und Vereinheitlichung der 
Benutzerschnittstelle in \Option*{cleardoublespecialpage=\PMisc} integriert.
\end{Obsolete}



\section[%
  Das Paket \PackageRaw{tudscrcomp}{\BooleanFalse}
  -- Migration von anderen TUD-Klassen%
]{%
  Migration von anderen TUD-Klassen%
  \tudmarkuplabel{\Package{tudscrcomp}}%
  \index{Kompatibilität!\Class{tudbook}|(}%
  \index{Kompatibilität!\Class{tudmathposter}|(}%
}
\begin{Entity*}{\Package{tudscrcomp}}
\noindent\Attention{%
  Sollten Sie eine der bereits in \autoref{sec:tudclasses} genannten Klassen
  \Class{tudbook}|?| (inklusive des Paketes \Package{tudthesis}) sowie 
  \Class{tudletter}|?|, \Class{tudfax}|?|, \Class{tudhaus}|?| und 
  \Class{tudform}|?| oder auch \Class{tudmathposter}|?| beziehungsweise eine 
  der \TUDScript-Klassen in der \emph{Version~v1.0} nie genutzt haben, können 
  Sie dieses \autorefname ohne Weiteres überspringen. Sämtliche der hier 
  vorgestellten Optionen und Befehle sind in der aktuellen Version von 
  \TUDScript obsolet.
}

\bigskip\noindent
Zu Beginn der Entwicklung von \TUDScript diente die Klasse \Class{tudbook} als 
grundlegende Basis zur Orientierung. Ziel war es, sämtliche Funktionalitäten 
dieser Klasse beizubehalten und zusätzlich den vollen Funktionsumfang der 
\KOMAScript-Klassen nutzbar zu machen. Bei der kompletten Neuimplementierung 
der \TUDScript-Klassen wurde sehr viel verändert und verbessert. Ein Teil der 
implementierten Optionen und Befehle waren jedoch bereits in der 
\emph{Version~v1.0} von \TUDScript unerwünschte Relikte, mit denen lediglich 
die Kompatibilität zur \Class{tudbook}"~Klasse und ihren Derivaten 
gewährleistet werden sollte. Mit der Version~v2.00 wurden einige der unnötigen 
Befehle und Optionen aus Gründen der Konsistenz nur umbenannt, andere wiederum 
wurden vollständig entfernt oder über neue Befehle und Optionen in ihrer 
Funktionalität ersetzt und teilweise erweitert. 

Das Paket \Package{tudscrcomp} dient der Überführung von Dokumenten, welche
entweder mit der \Class{tudbook}"~Klasse, ihren Derivaten, 
der Klasse \Class{tudmathposter} oder mit \emph{\TUDScript~v1.0} 
erstellt wurden, auf die aktuelle Version \TUDScriptVersion. Es werden einige 
Optionen und Befehle bereitgestellt, welche von den zuvor genannten Klassen 
definiert werden, um das entsprechende Verhalten nachzuahmen. Damit soll vor 
allem die Kompatibilität bei einer Änderung der Dokumentklasse sichergestellt 
werden. Unter anderem wird dafür an das Paket \Package{tudscrcolor} die Option 
\Option{extended}(\Package{tudscrcolor}) übergeben. Ist dies nicht erwünscht, 
sollte das Paket mit der Option \Option{reduced}(\Package{tudscrcolor}) geladen 
werden.
%
\begin{quoting}
\Attention{%
  Die Intention des Paketes ist, Dokumente möglichst schnell und einfach auf 
  die \TUDScript-Klassen portieren zu können. Das Ausgabeergebnis kann jedoch 
  stark von dem ursprünglichen Dokument abweichen. Falls Sie das Paket 
  verwenden wollen, sollte es \textbf{direkt} nach der Dokumentklasse geladen 
  werden. Andernfalls kann es im Zusammenhang mit anderen Paketen zu Problemen 
  kommen. Für den Satz neuer Dokumente wird empfohlen, auf den Einsatz dieses 
  Paketes zu verzichten und stattdessen direkt die von \TUDScript 
  bereitgestellten Optionen und Befehle zu nutzen.
}
\end{quoting}
%
Es ist zu beachten, dass nicht alle Funktionalitäten der genannten Klassen 
portiert werden konnten. Die betroffenen Optionen und Befehle erzeugen in 
diesem Fall eine Warnung mit einem Hinweis, wie ein gleiches oder zumindest 
ähnliches Verhalten mit \TUDScript erreicht werden kann. Weiterhin wird im 
Folgenden skizziert, wie sich die einzelnen Funktionalität ohne eine Verwendung 
des Paketes \Package{tudscrcomp} mit den Mitteln von \TUDScript umsetzen 
lassen. 

\begin{Declaration}{\Macro{einrichtung}[\MPName{Fakultät}]}(%
  identisch zu \Macro*{faculty}[\MPName{Fakultät}]()%
)
\begin{Declaration}{\Macro{fachrichtung}[\MPName{Einrichtung}]}(%
  identisch zu \Macro*{department}[\MPName{Einrichtung}]()%
)
\begin{Declaration}{\Macro{institut}[\MPName{Institut}]}(%
  identisch zu \Macro*{institute}[\MPName{Institut}]()%
)
\begin{Declaration}{\Macro{professur}[\MPName{Lehrstuhl}]}(%
  identisch zu \Macro*{chair}[\MPName{Lehrstuhl}]()%
)
\printdeclarationlist%
%
Dies sind die deutschsprachigen Befehle für den Kopf im \CD.
\end{Declaration}
\end{Declaration}
\end{Declaration}
\end{Declaration}

\begin{Declaration}{\Option{serifmath}}(identisch zu \Option*{cdmath=false})
\printdeclarationlist%
%
Die Funktionalität wird durch die Option \Option*{cdmath} bereitgestellt.
\end{Declaration}

\begin{Declaration}{\Option{colortitle}}(identisch zu \Option*{cdtitle=color})
\begin{Declaration}{\Option{nocolortitle}}(identisch zu \Option*{cdtitle=true})
\printdeclarationlist%
%
Die Funktionalität wird durch die Option \Option*{cdtitle} bereitgestellt.
\end{Declaration}
\end{Declaration}

\begin{Declaration}{\Macro{moreauthor}[\MPName{Autorenzusatz}]}(%
  identisch zu \Macro*{authormore}[\MPName{Autorenzusatz}]()%
)
\printdeclarationlist%
%
Ursprünglich war der Befehl für das Unterbringen aller möglichen, zusätzlichen 
Autoreninformationen gedacht. Auch der Befehl \Macro*{authormore}() ist ein 
Relikt davon, welcher jedoch weiterhin für allgemeine oder nicht vordefinierte 
Angaben genutzt werden kann. Für spezielle Informationen auf dem Titel sind die 
Befehle \Macro*{emailaddress}, \Macro*{dateofbirth}, \Macro*{placeofbirth}, 
\Macro*{matriculationnumber} und \Macro*{matriculationyear} sowie 
\Macro*{course} und \Macro*{discipline} vorgesehen und sollten hierfür 
dementsprechend genutzt werden.
\end{Declaration}

\begin{Declaration}{\Option{ddcfooter}}(identisch zu \Option*{ddcfoot=true})
\printdeclarationlist%
%
Die Funktionalität wird durch die Option \Option*{ddcfoot} bereitgestellt.
\end{Declaration}

\begin{Declaration}{\Macro{tudfont}[\MPName{Scriftart}]}(%
  identisch zu \Macro*{cdfont}[\MPName{Scriftart}]%
)
\printdeclarationlist%
%
Die direkte Auswahl der Schriftart sollte mit \Macro*{cdfont} erfolgen. 
Zusätzlich gibt es den Befehl \Macro*{textcd}, mit dem die Auszeichnung 
eines bestimmten Textes in einer anderen Schriftart erfolgen kann, ohne die 
Dokumentschrift umzuschalten.
\end{Declaration}
\index{Kompatibilität!\Class{tudmathposter}|)}%



\subsection{Optionen und Befehle aus \ClassRaw{tudbook}{\BooleanFalse} \& Co.}
%
Die nachfolgenden Optionen, Umgebungen sowie Befehle werden~-- zumindest 
teilweise~-- von den Klassen \Class{tudbook}, \Class{tudletter}, 
\Class{tudfax}, \Class{tudhaus}, \Class{tudform} sowie dem Paket 
\Package{tudthesis} und \TUDScript in der \emph{Version~v1.0} definiert. Diese 
werden durch das Paket \Package{tudscrcomp} für \TUDScript~\vTUDScript{} zur 
Verfügung gestellt.

\begin{Declaration}{\Macro{submitdate}[\MPName{Datum}]}(%
  identisch zu \Macro*{date}[\MPName{Datum}]%
)
\printdeclarationlist%
%
Die Funktionalität wird durch den erweiterten Standardbefehl \Macro*{date} 
abgedeckt.
\end{Declaration}

\begin{Declaration}{\Macro{supervisorII}[\MPName{Name}]}(%
  identisch zur Verwendung von \Macro*{and} innerhalb von \Macro*{supervisor}%
)
\printdeclarationlist%
%
Es ist \Macro*{supervisor}[\MPValue{\PName{Name} \Macro*{and} \PName{Name}}]
statt \Macro*{supervisorII}[\MPName{Name}] zu verwenden.
\end{Declaration}

\begin{Declaration}{\Macro{submittedon}[\MPName{Bezeichnung}]}(%
  siehe \Term*{datetext}%
)
\begin{Declaration}{\Macro{supervisedby}[\MPName{Bezeichnung}]}(%
  siehe \Term*{supervisorname}%
)
\begin{Declaration}{\Macro{supervisedIIby}[\MPName{Bezeichnung}]}(%
  siehe \Term*{supervisorothername}%
)
\printdeclarationlist%
%
Zur Änderung der Bezeichnung der Betreuer sollten die sprachabhängigen 
Bezeichner wie in \autoref{sec:localization} beschrieben angepasst werden. Eine 
Verwendung der hier beschriebenen Befehle entfernt die Abhängigkeit der 
Bezeichner von der verwendeten Sprache.
\end{Declaration}
\end{Declaration}
\end{Declaration}

\begin{Declaration}{\Macro{dissertation}}
\printdeclarationlist%
%
Die Funktionalität kann durch die Befehle \Macro*{thesis}[\MPValue{diss}] und 
\Macro*{referee} sowie die Bezeichner \Term*{refereename} und 
\Term*{refereeothername} dargestellt werden.
\end{Declaration}

\begin{Declaration}{\Macro{chapterpage}}
\printdeclarationlist%
%
Durch diesen Befehl können Kapitelseiten konträr zur eigentlichen Einstellung 
aktiviert oder deaktiviert werden. Prinzipiell ist dies auch durch eine 
Änderung der Option \Option*{chapterpage} möglich. Allerdings wird davon 
abgeraten, da dies zu einem inkonsistenten Layout innerhalb des Dokumentes 
führt.
\end{Declaration}

\begin{Declaration}{\Environment{theglossary}[\OPName{Präambel}]}
\begin{Declaration}{\Macro{glossitem}[\MPName{Begriff}]}
\printdeclarationlist%
%
Die \Class{tudbook}"~Klasse stellt eine rudimentäre Umgebung für ein Glossar 
bereit. Allerdings gibt es dafür bereits zahlreiche und besser implementierte 
Pakete. Daher wird für diese Umgebung keine Portierung vorgenommen, sondern 
lediglich die ursprüngliche Definition übernommen. Allerdings sein an dieser 
Stelle auf wesentlich bessere Lösungen wie beispielsweise das Paket 
\Package{glossaries} oder~-- mit Abstrichen~-- das nicht ganz so umfangreiche 
Paket \Package{nomencl} verwiesen.
\end{Declaration}
\end{Declaration}
\index{Kompatibilität!\Class{tudbook}|)}%



\subsection{%
  Optionen und Befehle aus \ClassRaw{tudmathposter}{\BooleanFalse}%
  \index{Kompatibilität!\Class{tudmathposter}|(}%
}
\ChangedAt{v2.05:Unterstützung der Klasse \Class{tudmathposter}}%
Die Klasse~\Class{tudmathposter} wird~-- im Gegensatz zu den zuvor genannten 
Klassen von Klaus~Bergmann~-- weiterhin gepflegt und kann bedenkenlos zum 
Setzen von Postern im A0"~Format verwendet werden. Dennoch gab es vermehrt 
Anfragen bezüglich einer Posterklasse auf Basis der \TUDScript-Klassen, um 
beispielsweise die Schriftgröße oder auch das Papierformat einfach anpassen zu 
können. Um von \Class{tudmathposter} einen möglichst einfachen Übergang auf 
\Class{tudscrposter} zu gewährleisten, kann zusätzlich zu letzterer Klasse das 
Paket \Package{tudscrcomp} geladen werden, welches die nachfolgend erläuterten 
Anwenderbefehle bereitstellt.

Mit der Kombination von \Class{tudscrposter} und dem Paket \Package{tudscrcomp}
soll lediglich die Möglichkeit geschaffen werden, auf \Class{tudmathposter} 
basierende Dokumente zu überführen. Es ist nicht beabsichtigt, dass bei einem 
Umstieg das Ausgabeergebnis identisch ist. In jedem Fall sollte beim Einsatz 
der Klasse \Class{tudscrposter} beachtet werden, dass für diese eine explizite 
Wahl der Schriftgröße über die Option~\Option*{fontsize}(\Bundle{koma-script}) 
notwendig ist. Um kongruent zur Klasse \Class{tudmathposter} zu bleiben, ist 
eine Schriftgröße von \PValue{34\dots36pt} sinnvoll. Weitere Informationen dazu 
sind in \autoref{sec:fontsize} vorhanden. Weiterhin sollte für ein ähnliches 
Ausgabeergebnis die Absatzformatierung über die \KOMAScript-Option 
\Option*{parskip=half}(\Bundle{koma-script}) eingestellt werden. Ein blaues 
\DDC-Logo im Fußbereich lässt sich über \Option*{ddcfoot=blue} aktivieren.

\begin{Declaration}{\Option{loadpackages=\PBoolean}}
\printdeclarationlist%
%
Durch \Class{tudmathposter} werden normalerweise die Pakete \Package{calc}, 
\Package{textcomp} sowie \Package{tabularx} eingebunden, welche allerdings für 
die Funktionalität der Klasse selbst nicht (zwingend) benötigt werden. Deshalb 
wird bei der Nutzung von \Package{tudscrcomp} standardmäßig darauf verzichtet. 
Diese können bei Bedarf einfach in der Präambel geladen werden. Alternativ 
dazu 
lässt sich die Option \Option{loadpackages} nutzen, welche die Pakete am Ende 
der Präambel automatisch lädt.
\end{Declaration}

\begin{Declaration}{\Option{tudmathfoot=\PBoolean}}%
\printdeclarationlist%
%
Durch die Klasse \Class{tudmathposter} wird der Fußbereich zweispaltig jedoch 
asymmetrisch und ohne Überschriften innerhalb der beiden Spalten gesetzt. 
Dieses Verhalten lässt sich mit \Option{tudmathfoot=true} auswählen, wohingegen 
\Option{tudmathfoot=false} genutzt werden kann, um auf das Standardverhalten 
von \Class{tudscrposter} zu schalten.
\end{Declaration}

\begin{Declaration}{\Option{bluebg}}(%
  identisch zu \Option*{backcolor=true}(\Class{tudscrposter})%
)
\printdeclarationlist%
%
Mit der Klasse \Class{tudscrposter} lässt sich das Verhalten mit der Option 
\Option*{backcolor}(\Class{tudscrposter}) umsetzen.
\end{Declaration}

\begin{Declaration}{\Macro{email}[\MPName{E-Mail-Adresse}]}(identisch zu 
  \Macro*{emailaddress*}[\MPName{E-Mail-Adresse}]%
)
\begin{Declaration}{\Macro{telefon}[\MPName{Telefonnummer}]}(identisch zu 
  \Macro*{telephone}[\MPName{Telefonnummer}](\Class{tudscrposter})%
)
\begin{Declaration}{\Macro{fax}[\MPName{Telefaxnummer}]}(identisch zu 
  \Macro*{telefax}[\MPName{Telefaxnummer}](\Class{tudscrposter})%
)
\begin{Declaration}{\Macro{homepage}[\MPName{URL}]}(identisch zu 
  \Macro*{webpage*}[\MPName{URL}](\Class{tudscrposter})%
)
\printdeclarationlist%
%
Dies sind die von \Class{tudmathposter} definierten Befehle für die Felder im 
vordefinierten Fußbereich des Posters. Es ist dabei insbesondere zu beachten, 
dass eine angegebene E"~Mail"=Adresse sowie URL nicht automatisch formatiert 
werden.
\end{Declaration}
\end{Declaration}
\end{Declaration}
\end{Declaration}

\begin{Declaration}{%
  \Macro{topsection}[\OPName{Kurzform}\MPName{Überschrift}]%
}
\begin{Declaration}{%
  \Macro{topsubsection}[\OPName{Kurzform}\MPName{Überschrift}]%
}
\printdeclarationlist%
%
Der Grund für die Existenz dieser beiden Befehle bei \Class{tudmathposter} ist 
nicht ohne Weiteres nachvollziehbar. Beide entsprechen in ihrem Verhalten den 
Standardbefehlen \Macro*{section} und \Macro*{subsection}, setzen allerdings 
keinen vertikalen Abstand vor der erzeugten Überschrift. Auch wenn das aus 
typographischer Sicht unvorteilhaft ist, werden diese beiden Befehle 
bereitgestellt.
\end{Declaration}
\end{Declaration}

\begin{Declaration}{%
  \Macro{centersection}[\OPName{Kurzform}\MPName{Überschrift}]%
}
\begin{Declaration}{%
  \Macro{centersubsection}[\OPName{Kurzform}\MPName{Überschrift}]%
}
\begin{Declaration}{%
  \Macro{topcentersection}[\OPName{Kurzform}\MPName{Überschrift}]%
}
\begin{Declaration}{%
  \Macro{topcentersubsection}[\OPName{Kurzform}\MPName{Überschrift}]%
}
\printdeclarationlist%
%
Weiterhin werden auch noch eigene Makros zum Setzen zentrierter Überschriften 
definiert~-- ein simples Umdefinieren von \Macro*{raggedsection} wäre dafür im 
Normalfall absolut ausreichend. Und um die Sache vollständig zu machen, gibt es 
die zentrierten Überschriften auch noch ohne vorgelagerten, vertikalen Abstand.
\end{Declaration}
\end{Declaration}
\end{Declaration}
\end{Declaration}

\begin{Declaration}{\Macro{zweitlogofile}[\MPName{Dateiname}]}(%
  identisch zu \Macro*{headlogo}[\MPName{Dateiname}]
)
\begin{Declaration}{\Macro{institutslogofile}[\MPName{Dateiname}]}(%
  \seeref{\Macro*{footlogo}}%
)
\begin{Declaration}{\Macro{drittlogofile}[\MPName{Dateiname}]}(%
  \seeref{\Option*{ddc} und \Option*{ddcfoot}}%
)
\printdeclarationlist%
%
Für die Angabe von Logos für den Kopf- und Fußbereich existieren diese Befehle. 
Bei der Verwendung von \Macro{institutslogofile}[\MPName{Dateiname}] ist zu 
beachten, dass die angegebene Datei sehr weit rechts im Fußbereich des Posters 
gesetzt wird. Dabei kommt bei der Verwendung im Hintergrund der von \TUDScript 
für das Setzen von Logos im Fußbereich tatsächlich vorgesehene Befehl in der 
Form \Macro*{footlogo}[\MPValue{{{,}{,}{,}{,}{,}{,}{,}\PName{Dateiname}{,}}}]
zum Einsatz. Das Makro \Macro{drittlogofile} wird von \Class{tudmathposter} für 
die Angabe eines \DDC-Logos im rechten Seitenfuß bereitgestellt. Für die 
\TUDScript-Klassen gibt es hierfür die Optionen \Option*{ddc} beziehungsweise 
\Option*{ddcfoot}.
\end{Declaration}
\end{Declaration}
\end{Declaration}

\begin{Declaration}{\Macro{zweitlogo}[\MPName{Definition}]}(%
  keine Funktionalität, \seeref{\Macro*{headlogo}}
)
\begin{Declaration}{\Macro{institutslogo}[\MPName{Definition}]}(%
  keine Funktionalität, \seeref{\Macro*{footlogo}}
)
\begin{Declaration}{\Macro{drittlogo}[\MPName{Definition}]}(%
  keine Funktionalität, \seeref{\Option*{ddc} und \Option*{ddcfoot}}%
)
\printdeclarationlist%
%
Mit \Class{tudmathposter} kann der Anwender die Definition für das Einbinden 
diverser Logos selber vornehmen. Dies ist für \TUDScript nicht vorgesehen, 
die Makros geben lediglich eine Warnung aus. Im Bedarfsfall lassen sich die 
optionalen Parameter der korrelierenden Befehle nutzen. 
\end{Declaration}
\end{Declaration}
\end{Declaration}

\begin{Declaration}{\Macro{fusszeile}[\MPName{Inhalt}]}(%
  identisch zu \Macro*{footcontent}[\MPName{Inhalt}]
)
\begin{Declaration}{\Macro{footcolumn0}[\MPName{Inhalt}]}(%
  identisch zu \Macro*{footcontent}[\MPName{Inhalt}]
)
\begin{Declaration}{\Macro{footcolumn1}[\MPName{Inhalt}]}(%
  identisch zu \Macro*{footcontent}[\MPName{Inhalt}\OPValue{*}]
)
\begin{Declaration}{\Macro{footcolumn2}[\MPName{Inhalt}]}(%
  identisch zu \Macro*{footcontent}[\MPValue{*}\OPName{Inhalt}]
)
\printdeclarationlist%
%
Mit diesen Befehlen kann die Gestalt des Fußes angepasst werden, wobei entweder 
der Bereich über die gesamte Breite (\Macro{fusszeile}, \Macro{footcolumn0}) 
oder lediglich die linke (\Macro{footcolumn1}) respektive die rechte Spalte 
(\Macro{footcolumn2}) angepasst wird. Für zusätzliche Hinweise zur Anpassung 
des Fußbereichs~-- insbesondere für die Schriftformatierung~-- sollte die 
Beschreibung von \Macro*{footcontent}'full' zu Rate gezogen werden.
\end{Declaration}
\end{Declaration}
\end{Declaration}
\end{Declaration}

\begin{Declaration}{\Environment{farbtabellen}}
\begin{Declaration}{\Macro{blautabelle}}
\begin{Declaration}{\Macro{grautabelle}}
\printdeclarationlist%
%
Wird innerhalb der \Environment{farbtabellen}"~Umgebung eine Tabelle gesetzt, 
so werden die Zeilen alternierend farbig hervorgehoben. Standardmäßig sind 
hierfür leichte Blautöne eingestellt, was auch jederzeit mit dem Aufruf von 
\Macro{blautabelle} wiederhergestellt werden kann. Alternativ zu dieser 
Darstellung kann mit \Macro{grautabelle} auf leichte Grautöne umgestellt werden.
\end{Declaration}
\end{Declaration}
\end{Declaration}

\begin{Declaration}{\Macro{schnittrand}}
\printdeclarationlist%
%
Wird der Befehl \Macro{schnittrand} innerhalb der Präambel definiert, so wird 
dessen Inhalt als Längenwert interpretiert. Dieser wird verwendet, um den zuvor 
festgelegten Satzspiegel über die drei Parameter
\Key*{\Macro{geometry}}{paper=\PSet{Papierformat}}(\Package{geometry}),
\Key*{\Macro{geometry}}{layout=\PSet{Zielformat}}(\Package{geometry}) und 
\Key*{\Macro{geometry}}{layoutoffset=\PLength}(\Package{geometry}) des
Befehls \Macro*{geometry}(\Package{geometry}) aus dem Paket \Package*{geometry} 
zu setzen und das erzeugte Papierformat um den gegebenen Längenwert an allen 
Rändern zu vergrößern. Somit wird eine Beschnittzugabe hinzugefügt, \emph{ohne 
dabei die Seitenränder des Entwurfslayouts anzupassen}. Weitere Informationen 
hierzu sind in \fullref{sec:tips:crop} sowie im \GitHubRepo(tud-cd/tud-cd)<6>  
zu finden.
\end{Declaration}
\index{Kompatibilität!\Class{tudmathposter}|)}%
\end{Entity*}
