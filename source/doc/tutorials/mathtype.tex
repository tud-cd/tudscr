\documentclass[english,ngerman,cdfont=false]{tudscrartcl}
\iftutex
  \usepackage{fontspec}
\else
  \usepackage[T1]{fontenc}
  \usepackage[ngerman=ngerman-x-latest]{hyphsubst}
\fi

\usepackage{tudscrtutorial}
\lstset{%
  inputencoding=utf8,extendedchars=true,
  literate=%
    {ä}{{\"a}}1 {ö}{{\"o}}1 {ü}{{\"u}}1
    {Ä}{{\"A}}1 {Ö}{{\"O}}1 {Ü}{{\"U}}1
    {ß}{{\ss}}1 {~}{{\textasciitilde}}1
    {»}{{\guillemetright}}1 {«}{{\guillemetleft}}1
}

\usepackage{mathtools}
\usepackage{bookmark}

\begin{document}
\title{Ein Beitrag zum mathematischen Satz mit \hologo{LaTeX}}
\author{Falk Hanisch\TUDScriptContactTitle}
\date{07.11.2016}

\makeatletter
\begingroup%
  \def\and{, }%
  \let\thanks\@gobble%
  \let\footnote\@gobble%
  \hypersetup{%
    pdfauthor = {\@author},%
    pdftitle = {\@title},%
    pdfsubject = {Mathematiksatz in LaTeX},%
    pdfkeywords = {LaTeX, \TUDScript, Tutorial, Mathematiksatz},%
  }%
\endgroup%
\markright{\@title}
\makeatother

\StartTutorial[%
  Im mathematischen Satz sollten lediglich Formelzeichen für physikalische 
  Größen und Variablen sowie Funktions- und Operatorzeichen mit frei wählbarer 
  Bedeutung kursiv geschrieben werden. Dagegen werden Einheiten und ihre 
  Vorsätze, Zahlen, Funktions- und Operatorzeichen mit feststehender Bedeutung 
  sowie chemische Elemente und Verbindungen ebenso wie Indizes, welche ein 
  Formelzeichen näher beschreiben, aufrecht gesetzt. Dieses Tutorial zeigt, wie 
  mit einfachen Mitteln die Auszeichnung von Indizes bei Formelzeichen 
  typografisch korrekt erfolgen kann.
]
Bevor das eigentliche Tutorial beginnt, werden sowohl eine Dokumentklasse als 
auch die für jedes \hologo{LaTeX}"~Dokument meiner Meinung nach sinnvollen 
Pakete geladen.
%
\begin{Preamble}
\documentclass[ngerman]{tudscrartcl}% andere Klassen sind möglich
\usepackage{iftex}
\iftutex
  \usepackage{fontspec}
\else
  \usepackage[T1]{fontenc}
  \usepackage[ngerman=ngerman-x-latest]{hyphsubst}
\fi
\usepackage{babel}
\usepackage{microtype}

\end{Preamble}
%
Zusätzlich wird das Paket \Package{amsmath} geladen, welches unter anderem die 
in diesem Tutorial verwendeten Mathematikumgebungen \Environment{equation}, 
\Environment{gathered} und \Environment{aligned} zur Verfügung stellt.
%
\begin{Preamble}
\usepackage{amsmath}
\end{Preamble}
%
Zunächst wird ein Beispiel gegeben, das für die folgenden Ausführungen als 
Grundlage dient. 
%
\CodePreamble{}%
\begin{Trunk*}
\begin{equation*}
\begin{gathered}
M_{EM} = \frac{M_{Rad}}{i_g \cdot i_A} - M_{VM} \\
\textrm{für }
\begin{aligned}
0\leq M_{VM}\leq M_{VMmax} \\ 
M_{EMmin}\leq M_{EM}\leq M_{EMmax}
\end{aligned}
\end{gathered}
\end{equation*}

\end{Trunk*}
%
Ein guter Mathematiksatz ist in \hologo{LaTeX} durchaus Sisyphusarbeit. Wenn 
mikrotypografisch alles richtig gemacht werden soll, gibt es einiges zu 
beachten. Generell gilt, dass Variablen kursiv oder geneigt, Bezeichnungen und 
Konstanten aufrecht gesetzt werden. Um beschreibende Indizes formal richtig zu 
setzen, ist ohne weitere Maßnahmen die exzessive Nutzung der Befehle 
\Macro{mathrm}[\MPValue{\dots}] und \Macro{mathit}[\MPValue{\dots}] 
respektive \Macro{mathnormal}[\MPValue{\dots}] wohl oder übel notwendig. 
Aus dem vorhergehenden Beispiel wird mit ziemlich viel Anpassungsarbeit:
%
\CodePreamble{}%
\begin{Trunk*}
\begin{equation*}
\begin{gathered}
M_\mathrm{EM} = \frac{M_\mathrm{Rad}}{i_g \cdot i_A} - M_\mathrm{VM} \\
\textrm{für }
\begin{aligned}
0\leq M_\mathrm{VM}\leq M_\mathrm{VM_{max}} \\ 
M_\mathrm{EM_{min}}\leq M_\mathrm{EM}\leq M_\mathrm{EM_{max}}
\end{aligned}
\end{gathered}
\end{equation*}

\end{Trunk*}
%
Augenscheinlich ist dabei sehr viel Handarbeit notwendig. Allerdings lässt sich 
dies relativ gut vereinfachen. Zu diesem Zwecke wird für das Setzen von Indizes 
bei Formelzeichen der Befehl \Macro{ind}[\MPValue{\dots}] definiert. Danach 
können~-- wenn dies für nötig und sinnvoll erachtet wird~-- noch zusätzliche 
Befehle für häufig verwendete Ausdrücke definiert werden. Als Beispiel wird das 
schon eben genutzte Drehmoment \ensuremath{M} verwendet. Hierfür wäre folgende 
Definition möglich:%
\footnote{%
  Dabei sorgt \Macro*{kern.03em} für das Einfügen eins kleinen Abstandes 
  zwischen kursiver Variable und dem aufrechten Index. Abhängig von der 
  verwendeten Schriftart sollte dieser Abstand leicht angepasst werden. 
}
%
\CodeHook{\let\newcommand\renewcommand}
\begin{Trunk*}
\newcommand*{\ind}[1]{\ensuremath{\kern.03em_\mathrm{#1}}}
\newcommand*{\M}[1]{\ensuremath{M\ind{#1}}}

\end{Trunk*}
\CodePreamble{}
\begin{Trunk*}
\begin{equation*}
\begin{gathered}
\M{EM} = \frac{\M{Rad}}{i_g \cdot i_A} - \M{VM} \\
\textrm{für }
\begin{aligned}
0\leq \M{VM}\leq \M{VM_{max}} \\ 
\M{EM_{min}}\leq \M{EM}\leq \M{EM_{max}}
\end{aligned}
\end{gathered}
\end{equation*}

\end{Trunk*}
%
Ein gewünschter Nebeneffekt der vorhergehenden Definition ist, dass dieser 
Befehl aufgrund der Verwendung von \Macro{ensuremath}[\MPValue{\dots}] nun 
auch im Fließtext verwenden werden kann, wie hier für das induzierte Moment 
einer Verbrennungskraftmaschine~\M{VM_{ind}} 
(\Macro{M}[\MPValue{VM\_\MPValue{ind}}]).

Für noch mehr Bequemlichkeit bei der Nutzung kann eine Lösung gefunden werden, 
mit welcher~-- im Gegensatz zur Standardbefehlsdefinition in \hologo{LaTeX}~-- 
\textbf{nach} dem obligatorischen noch ein optionales Argument für einen 
zusätzlichen Subindex angegeben werden kann, um damit der natürlichen 
Schreibweise zu entsprechen. Es wird der Befehl \Macro{M} so definiert, das 
dieser entweder mit \Macro{M}[\MPName{Index}] oder in der Variante 
\Macro{M}[\MPName{Index}\OPName{Indexindex}] mit nachgelagertem 
optionalen Argument genutzt werden kann.
%
\CodeHook{\let\newcommand\renewcommand}
\begin{Trunk*}
\makeatletter
\renewcommand*{\M}[1]{%
  \kernel@ifnextchar[%]
    {\o@M{#1}}{\n@M{#1}}%
}
\newcommand*{\n@M}[1]{\ensuremath{M\ind{#1}}}
\newcommand*{\o@M}{}
\def\o@M#1[#2]{\ensuremath{M\ind{#1_{#2}}}}
\makeatother

\end{Trunk*}
%
Somit vereinfacht sich das zu Beginn vorgestellte Beispiel recht deutlich:
%
\begin{Trunk*}
\begin{equation*}
\begin{gathered}
\M{EM} = \frac{\M{Rad}}{i_g \cdot i_A} - \M{VM} \\
\textrm{für }
\begin{aligned}
0\leq \M{VM}\leq \M{VM}[max] \\ 
\M{EM}[min]\leq \M{EM}\leq \M{EM}[max]
\end{aligned}
\end{gathered}
\end{equation*}
\end{Trunk*}
%
Das Definieren von \Macro{M}[\MPName{Index}\OPName{Indexindex}] mit 
angehängtem optionalen Argument ist ehrlich gesagt nur ein wenig Spielerei und 
soll zeigen, wie dies prinzipiell mit \hologo{LaTeX} umgesetzt werden kann. 
Mit dem Paket \Package{xparse} könnte alternativ für die Befehlsdeklaration des 
optionalen \textbf{nach} dem obligatorischen Argument folgendermaßen genutzt 
werden:
\begin{Hint}
\NewDocumentCommand \M { m o } {%
  \ensuremath{M\ind{#1\IfValueT{#2}{_{#2}}}}%
}
\end{Hint}
%
\FinishTutorial
\ListOfToDo
\end{document}
