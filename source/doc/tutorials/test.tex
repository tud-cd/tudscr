\RequirePackage[ngerman=ngerman-x-latest]{hyphsubst}
\documentclass[%
  english,ngerman,%
  geometry=no,DIV=12,automark,%
]{tudscrartcl}
\usepackage{selinput}
\SelectInputMappings{adieresis={ä},germandbls={ß}}
\usepackage[T1]{fontenc}
\usepackage{lmodern}

\usepackage{tudscrman}
\lstset{%
  inputencoding=utf8,extendedchars=true,
  literate=%
    {ä}{{\"a}}1 {ö}{{\"o}}1 {ü}{{\"u}}1
    {Ä}{{\"A}}1 {Ö}{{\"O}}1 {Ü}{{\"U}}1
    {~}{{\textasciitilde}}1 {ß}{{\ss}}1
}
\TUDoptions{cdfont=false}
\KOMAoptions{headings=normal}

\usepackage{tudscrsupervisor}
\usepackage{array}
\usepackage{tabu}
\usepackage{tabularx}
\usepackage{booktabs}
\usepackage{units}
\AfterPackage*{hyperref}{%
  \usepackage[%
    automake,%
    acronym,%
    symbols,%
    nomain,%
    translate=babel,%
    nogroupskip,%
  ]{glossaries}
  \setStyleFile{\jobname-temp}
  \renewcommand*{\glsglossarymark}[1]{}
  \makeglossaries
}
\usepackage{csquotes}
\usepackage[backend=biber,style=alphabetic]{biblatex}
\usepackage{filecontents}
\addbibresource{treatise-temp.bib}

\usepackage{enumitem}

\usepackage{caption}
\usepackage{floatrow}
\renewcommand{\floatpagefraction}{0.7}

\ifpdf
  \usepackage{tikz}
  \usetikzlibrary{chains}
  \usetikzlibrary{decorations.markings}
  \tikzset{on grid}
\fi

\usepackage{pstricks,pst-node}

\makeatletter
\newcommand*\pcolumnfuzz[1]{\pretocmd{\@endpbox}{\hfuzz=#1}{}{}}
\makeatother

\usepackage[open,openlevel=0]{bookmark}[2011/12/02]

\begin{document}

\newcolumntype{Y}{>{\hspace{0pt}}X}
\newcolumntype{L}{>{\raggedright}Y}
\newcolumntype{C}{>{\centering}Y}
\newcolumntype{R}{>{\raggedleft}Y}
\newcommand*\tableexample[2]{%
  \begin{table}
  \begin{tabularx}{.9\textwidth}{@{}LCRY@{}}
  \toprule
  \textbf{Linksbündig} & \textbf{Zentriert} & 
  \textbf{Rechtsbündig} & \textbf{Blocksatz} \tabularnewline
  \midrule
  Ein linksbündiger Blindtext &
  Ein zentrierter Blindtext &
  Ein rechtsbündiger Blindtext &
  Ein im Blocksatz gesetzter Blindtext
  \tabularnewline
  \bottomrule
  \end{tabularx}
  \caption{#1}\label{#2}%
  \end{table}
}

\subsection{Gleitobjektlayout}
\label{sec:floatlayout}
%
Es wurde bis jetzt das prinzipielle Vorgehen bei der Nutzung von Gleitobjekten 
beschrieben. Allerdings gibt es auch noch einige typographische Aspekte, welche 
zu beachten sind. Zum einen sollte beachtet werden, dass die Beschriftung einer 
Abbildung immer unterhalb dieser erfolgen sollte, bei Tabellen hingegen eine 
Überschrift gesetzt wird. Hierfür muss der Anwender ohne die Verwendung eines 
zusätzlichen Paketes selber Sorge tragen, indem er den Befehl \Macro{caption} 
entweder vor oder nach dem eigentlichen Objekt in der Gleitumgebung verwendet. 
Der zweite Punkt ist die verwendete Schrift innerhalb der Gleitumgebungen. 

Damit sich der Inhalt dieser besser vom restlichen Fließtext abhebt und vom 
Leser direkt als nicht dazugehörig erkannt werden kann ist es ratsam, diesen in 
serifenloser Schrift zu setzen~-- die Verwendung einer Serifenschrift für den 
Fließtext vorausgesetzt. Das Paket \Package{floatrow} bietet die Möglichkeiten, 
diese beiden Punkte automatisiert umzusetzen. Außerdem wird zur Formatierung 
der Beschriftungen das Paket \Package{caption} benötigt. Sollten Sie noch keine 
Erfahrung mit dem Setzen von Tabellen mit \hologo{LaTeX} haben, so wäre zuvor 
ein Blick in \autoref{sec:tables} sehr sinnvoll. In \autoref{tab:tabular} ist 
zu sehen, wie der Inhalt einer \Environment{table}"=Gleitumgebung erscheint, 
wenn die Funktionalitäten der beiden Pakete nicht genutzt werden.

\tableexample{Eine Tabelle in einer Gleitumgebung}{tab:tabular}

Um die \emph{Formatierung} der Bezeichnungen von Gleitobjekten anzupassen, wird 
das Paket \Package{caption} benötigt.
%
\begin{Preamble}
\usepackage{caption}
\end{Preamble}
%
Diese stellt den Befehl \Macro{captionsetup} bereit. Mit diesem werden die 
Beschriftungen in Serifenlosen und das Label zur besseren Unterscheidung im 
fetten Schriftschnitt gesetzt.
%
\begin{Preamble*}
\captionsetup{font=sf,labelfont=bf,labelsep=space}
\end{Preamble*}
%
Um den Inhalt einer Gleitumgebung formatieren und die \emph{Platzierung} der 
Beschriftungen beeinflussen zu können, wird \Package{floatrow} geladen.
\begin{Preamble}
\usepackage{floatrow}
\end{Preamble}
%
Der Inhalt von Gleitobjekten wird~-- entgegen des normalen Verhaltens~-- durch 
\Package{floatrow} automatisch zentriert gesetzt. Zusätzlich soll der Inhalt 
noch die serifenlose Schriftfamilie verwenden. Dies lässt sich mit der Nutzung 
des Befehls \Macro{floatsetup} realisieren.
%
\begin{Preamble*}
\floatsetup{font=sf}
\end{Preamble*}
%
Ohne weitere Maßnahmen werden durch \Package{floatrow} alle Beschriftungen 
unterhalb der Gleitobjekte ausgegeben. Da das Ziel jedoch die Verwendung von 
Tabellenüberschriften und Abbildungsunterschriften ist, wird dies für Tabellen
angepasst.
%
\begin{Preamble*}
\floatsetup[table]{style=plaintop}
\end{Preamble*}
%
\ToDo{genau hier geht's weiter}
\tableexample{Eine Tabelle in einer Gleitumgebung}{tab:tabularA}
%
\captionsetup{singlelinecheck=false,format=hang,justification=raggedright}
\tableexample{Eine Tabelle in einer Gleitumgebung}{tab:tabularA}
%
\begin{table}
\ttabbox{%
  \caption{Eine Tabelle in einer Gleitumgebung}\label{tab:tabularB}%
}{%
  \begin{tabularx}{.9\textwidth}{@{}LCRY@{}}
  \toprule
  \textbf{Linksbündig} & \textbf{Zentriert} & 
  \textbf{Rechtsbündig} & \textbf{Blocksatz} \tabularnewline
  \midrule
  Ein linksbündiger Blindtext zur Demonstration einer L"~Spalte &
  Ein zentrierter Blindtext zur Demonstration einer C"~Spalte &
  Ein rechtsbündiger Blindtext zur Demonstration einer R"~Spalte &
  Ein im Blocksatz gesetzter Blindtext zur Demonstration einer Y"~Spalte
  \tabularnewline
  \bottomrule
  \end{tabularx}
}
\end{table}
%


\subsection{Untergleitobjekte}
\label{sec:subfloats}


\end{document}