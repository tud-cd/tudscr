\RequirePackage[ngerman=ngerman-x-latest]{hyphsubst}
\documentclass[%
  english,ngerman,%
  geometry=no,DIV=12,automark,%
]{tudscrartcl}
\usepackage{selinput}
\SelectInputMappings{adieresis={ä},germandbls={ß}}
\usepackage[T1]{fontenc}
\usepackage{lmodern}

\usepackage{tudscrman}
\lstset{%
  inputencoding=utf8,extendedchars=true,
  literate=%
    {ä}{{\"a}}1 {ö}{{\"o}}1 {ü}{{\"u}}1
    {Ä}{{\"A}}1 {Ö}{{\"O}}1 {Ü}{{\"U}}1
    {~}{{\textasciitilde}}1 {ß}{{\ss}}1
}
\TUDoptions{cdfont=false}
\KOMAoptions{headings=normal}

\usepackage{tudscrsupervisor}
\usepackage{tabu,tabularx,booktabs}
\usepackage{longtable}
\usepackage{ltxtable}
\usepackage{filecontents}

\usepackage{units}
\AfterPackage*{hyperref}{%
  \usepackage[%
    automake,%
    acronym,%
    symbols,%
    nomain,%
    translate=babel,%
    nogroupskip,%
    section=subsubsection,%
  ]{glossaries}
  \setStyleFile{\jobname-temp}
  \renewcommand*{\glsglossarymark}[1]{}
  \makeglossaries
}
\usepackage{csquotes}
\usepackage[backend=biber,style=alphabetic]{biblatex}
\usepackage{filecontents}
\addbibresource{treatise-temp.bib}

\usepackage{enumitem}

\renewcommand{\floatpagefraction}{0.7}

\ifpdf
  \usepackage{tikz}
  \usetikzlibrary{chains}
  \usetikzlibrary{decorations.markings}
  \tikzset{on grid}
\fi

\usepackage{pstricks,pst-node}
\usepackage{blindtext}
\begin{document}


\section{Tabellen}
\label{sec:tables}
Zum Thema \enquote{Tabellensatz mit \hologo{LaTeX}} sind bereits zahlreiche 
\hrfn{http://userpage.fu-berlin.de/latex/Materialien/tabsatz.pdf}{Leitfäden} im 
Internet zu finden. Deshalb werde ich meine Ausführungen zu diesem Punkt 
relativ kurz halten. Zwei Regeln sollten beim Satz von Tabellen in jedem Fall 
beachtet werden:
%
\begin{enumerate}[itemindent=0pt,labelwidth=*,labelsep=1em,label=\Roman*.]
\item keine vertikalen Linien
\item keine doppelten Linien
\end{enumerate}
%
Das Paket \Package{booktabs} ist für den Satz von hochwertigen Tabellen eine 
große Hilfe und stellt die Befehle \Macro{toprule}, \Macro{midrule} sowie
\Macro{cmidrule} und \Macro{bottomrule} für unterschiedliche horizontale Linien 
bereit. Außerdem existiert das Paket \Package{array}, welches mit dem Befehl 
\Macro{newcolumntype} die Definition eigener Spaltentypen sowie die Verwendung 
sogenannter \enquote{Hooks} vor und nach Einträgen innerhalb einer Spalte 
(\PValue{>\{\dots\}}\PName{Spaltentyp}\PValue{<\{\dots\}}) ermöglicht.

Für alle im Folgenden vorgestellten Umgebungen zum Setzen von Tabellen gilt, 
dass die Inhalte zeilenweise angegeben werden, wobei die Einträge für die 
einzelnen Spalten mit \PValue{\&} voneinander zu trennen sind. Das Beenden 
einer Tabellenzeile und der Wechsel zur nächsten erfolgt normalerweise mit 
\Macro{\bsc}. Da jedoch einige \hologo{LaTeX}-Pakete diesen Befehl innerhalb 
von Tabellen lokal ändern ist es zur Vermeidung unnötiger Fehler wesentlich 
sicherer, das Zeilenende ausschließlich mit \Macro{tabularnewline} zu setzen.

\subsection{Die Standardumgebung \Environment{tabular}}
Normalerweise gibt es vier unterschiedliche Spaltentypen. Die Spaltentypen 
\PValue{l}, \PValue{c} und \PValue{r} stehen für linksbündige, zentrierte und 
rechtsbündige Spalten, welche allerdings keinen Zeilenumbruch erlauben. Der 
Inhalt wird quasi wie in einer \Macro {mbox} gesetzt, wodurch die Spalten sehr 
breit werden und über den Seitenrand hinausragen können.

Der Spaltentyp~\PValue{p}\Parameter{Breite} hingegen legt die Spaltenbreite 
fest und setzt den Inhalt in jeder Zeile in eine \Macro{parbox}, wobei diese 
wird mit ihrer obersten Zeile an der Grundlinie ausgerichtet wird. Das Paket
\Package{array} stellt außerdem die Spaltentypen~\PValue{m}\Parameter{Breite} 
und ~\PValue{b}\Parameter{Breite} bereit, welche ebenfalls mit \Macro{parbox} 
gesetzt werden, die Ausrichtung an der Grundlinie jedoch zentriert respektive 
an der unteren Zeile der Box erfolgt. Es folgt ein kurzes Beispiel.
%
\begin{Hint}
\begin{tabular}{lcrp{40mm}}
\toprule
\textbf{Linksbündig} & \textbf{Zentriert} & 
\textbf{Rechtsbündig} & \textbf{Blocksatz} \tabularnewline
\midrule
a   & b   & c   & Dieser Text erscheint im Blocksatz \tabularnewline
aa  & bb  & cc  & Außerdem sind Zeilenumbrüche möglich\tabularnewline
aaa & bbb & ccc & Zeilenumbruchsvorschauansicht\tabularnewline
\bottomrule
\end{tabular}
\end{Hint}
%
Die Tabellenbreite ergibt sich aus der Breite der einzelnen Spalten. Bei dieser 
Umgebung liegt es allein beim Anwender, auf die korrekte Breite der Tabelle zu 
achten, damit diese nicht über die Seitenränder hinausragt. Das kann auf Dauer 
recht aufwändig werden. Das Festlegen der Gesamtbreite einer Tabelle durch den 
Anwender und das automatische Berechnen einiger oder aller Spaltenbreiten ist 
sicher die angenehmere Variante. Hierfür stehen die Pakete \Package{tabularx} 
oder \Package{tabu} zur Verfügung, welche nun kurz vorgestellt werden sollen.
Wie man das Problem des nicht umbrochenen Eintrags in der letzten Spalte der 
dritten Zeile lösen kann, wird ebenfalls dort erklärt.

\subsection{Die Tabellenumgebung \Environment{tabularx}}
Das Paket \Package{tabularx} stellt den Spaltentyp~\PValue{X} bereit, welcher
prinzipiell dem Spaltentyp~\PValue{p} entspricht. Auch für diesen wird eine 
\Macro{parbox} verwendet, allerdings wird deren Breite \emph{automatisch} 
berechnet. Die Umgebung \Environment{tabularx} erwartet vor der Angabe der 
Spaltentypen als obligatorisches Argument die gewünschte Breite der Tabelle. Zu 
beachten ist, dass Spalten vom Typ~\PValue{l},~\PValue{c}~und~\PValue{r} 
weiterhin ohne Zeilenumbruch gesetzt werden. Nur für Spalten vom Typ~\PValue{X} 
und deren Derivate wird aus dem verbliebenen Platz die Breite berechnet. Für 
die \Environment{tabularx}-Umgebung wird das vorherige Beispiel wiederholt.
%
\begin{Hint}
\begin{tabularx}{12cm}{lcrX}
\toprule
\textbf{Linksbündig} & \textbf{Zentriert} & 
\textbf{Rechtsbündig} & \textbf{Blocksatz} \tabularnewline
\midrule
a   & b   & c   & Dieser Text erscheint im Blocksatz \tabularnewline
aa  & bb  & cc  & Außerdem sind Zeilenumbrüche möglich\tabularnewline
aaa & bbb & ccc & Zeilenumbruchsvorschauansicht\tabularnewline
aaa & bbb & ccc & \hspace{0pt}Zeilenumbruchsvorschauansicht\tabularnewline
\bottomrule
\end{tabularx}
\end{Hint}
%
Die Breite der letzten Spalte wurde dabei aus der Angabe der Gesamtbreite mit 
\PValue{12cm} berechnet. Des Weiteren ist zu sehen, wie das Problem des 
Zeilenumbruchs behandelt werden kann. Normalerweise wird das erste Wort in 
einem Absatz von \hologo{LaTeX} \emph{nie} umbrochen. Das kann mit dem Einfügen 
eines horizontalen Abstandes mit dem Wert \PValue{0pt} umgangen werden. Die 
gefundene Lösung ist allerdings alles andere als elegant.

Mit den Möglichkeiten des Paketes \Package{array} ist das Problem relativ 
schnell gelöst. Es wird mit \Macro{newcolumntype} ein neuer Spaltentyp 
definiert. Das erste Argument von \Macro{newcolumntype} legt den Namen des 
Spaltentyps fest. Mit \PValue{>\Parameter{Definitionen}}\Parameter{Typ} wird im 
zweiten Argument das Ausführen von \PName{Definitionen} vor dem Setzen des 
eigentlichen Inhaltes in einer \Parameter{Typ}"~Spalte definiert.

Es wird ein neuer, auf der \PValue{X}"~Spalten basierender Typ~\PValue{Y} 
definiert, welcher zu Beginn der Spalte den Phantomabstand automatisch einfügt. 
Darauf basierend werden drei Spaltentypen für den links- und rechtsbündigen 
sowie zentrierten Textsatz mit einem möglichen Zeilenumbruch erstellt.
%
\InputHook{\renewcommand*{\newcolumntype}[2]{}}
\begin{Excerpt}
\newcolumntype{Y}{>{\hspace{0pt}}X}
\newcolumntype{L}{>{\raggedright}Y}
\newcolumntype{C}{>{\centering}Y}
\newcolumntype{R}{>{\raggedleft}Y}
\end{Excerpt}
%
Die Umgebung \Environment{tabularx} erwartet vor der Angabe der Spaltentypen 
als obligatorisches Argument die gewünschte Breite der Tabelle. Es ist dabei zu 
beachten, dass Spalten vom Typ~\PValue{l},~\PValue{c}~und~\PValue{r} weiterhin 
ohne Zeilenumbruch gesetzt werden. Nur für Spalten vom Typ~\PValue{X} und deren 
Derivate wird aus dem verbliebenen Platz die Breite berechnet. Nachfolgend wird 
die Tabelle gesetzt, das Ergebnis ist in \autoref{tab:tabularx} zu sehen.
%
\begin{Excerpt*}
\begin{table}
\begin{tabularx}{\textwidth}{@{}LCRY@{}}
\toprule
\textbf{Linksbündig} & \textbf{Zentriert} & 
\textbf{Rechtsbündig} & \textbf{Blocksatz} \tabularnewline
\midrule
Ein linksbündiger Blindtext zur Demonstration einer L"~Spalte &
Ein zentrierter Blindtext zur Demonstration einer C"~Spalte &
Ein rechtsbündiger Blindtext zur Demonstration einer R"~Spalte &
Ein im Blocksatz gesetzter Blindtext zur Demonstration einer Y"~Spalte
\tabularnewline
aaa & bbb & ccc & Zeilenumbruchsvorschauansicht\tabularnewline
\bottomrule
\end{tabularx}
\caption{Eine \texttt{tabularx}-Tabelle}\label{tab:tabularx}
\end{table}
\end{Excerpt*}
\InputExcerpt
%
Wahrscheinlich werden Sie sich über das Konstrukt~\PValue{@\{\}} vor der ersten 
und nach der letzten Spalte wundern. Normalerweise wird in einer Tabelle mit 
\Macro{hskip}\Macro{tabcolsep} vor \emph{und} nach jeder Spalte ein 
horizontaler Zwischenraum eingefügt.%
\footnote{Der Abstand zweier Spalten beträgt folglich 2\Macro{tabcolsep}.}
Mit \PValue{@\Parameter{Ausdruck}} kann dies verhindert und stattdessen  
\Parameter{Ausdruck} ausgeführt werden. Mit der Verwendung von \PValue{@\{\}} 
an den entsprechenden Stellen bei der Angabe der Spaltentypen wurde der vor der 
ersten und nach der letzten Tabellenspalte eingefügten Abstand unterdrückt.

\subsection{Die Tabellenumgebung \Environment{tabu}}
Das Paket \Package{tabu} bietet eine mächtige und komfortable Alternative zu 
\Package{tabularx}. Es kam in diesem Tutorial bereits für die Verzeichnisse von 
Abkürzungen und Symbolen in \autoref{sec:glossaries} zum Einsatz. Das Paket 
definiert ebenso einen Spaltentyp~\PValue{X}, welchem allerdings zusätzlich ein 
optionales Argument angehängt werden kann. Mit diesem lässt sich sowohl die 
Gewichtung der automatisch berechneten Spalten untereinander als auch die 
Positionierung zur Grundlinie sowie die Ausrichtung des Inhaltes ändern.
%

\begin{Hint}
\begin{tabu}{lcrp{40mm}}
\toprule
\textbf{Linksbündig} & \textbf{Zentriert} & 
\textbf{Rechtsbündig} & \textbf{Blocksatz} \tabularnewline
\midrule
a   & b   & c   & Dieser Text erscheint im Blocksatz \tabularnewline
aa  & bb  & cc  & Außerdem sind Zeilenumbrüche möglich\tabularnewline
aaa & bbb & ccc & Etwas Text, um dieses Feld zu füllen\tabularnewline
\bottomrule
\end{tabu}
\end{Hint}

\tabulinesep=_\dp\strutbox

\begin{Hint}
\begin{tabu}{lcrp{40mm}}
\toprule
\textbf{Linksbündig} & \textbf{Zentriert} & 
\textbf{Rechtsbündig} & \textbf{Blocksatz} \tabularnewline
\midrule
a   & b   & c   & Dieser Text erscheint im Blocksatz \tabularnewline
aa  & bb  & cc  & Außerdem sind Zeilenumbrüche möglich\tabularnewline
aaa & bbb & ccc & Etwas Text, um dieses Feld zu füllen\tabularnewline
\bottomrule
\end{tabu}
\end{Hint}

\begin{Excerpt*}
\begin{table}
\caption{Eine \texttt{tabu}-Tabelle}\label{tab:tabu}
\begin{tabu} to \textwidth {@{}X[2,l]X[2,c]X[2,r]X[3,j]@{}}
\toprule
\textbf{Linksbündig} & \textbf{Zentriert} & 
\textbf{Rechtsbündig} & \textbf{Blocksatz} \tabularnewline
\midrule
Ein linksbündiger Blindtext zur Demonstration einer X[l]"~Spalte &
Ein zentrierter Blindtext zur Demonstration einer X[c]"~Spalte &
Ein rechtsbündiger Blindtext zur Demonstration einer X[r]"~Spalte &
Ein im Blocksatz gesetzter Blindtext zur Demonstration einer X[j]"~Spalte
\tabularnewline
Ein linksbündiger Blindtext zur Demonstration einer X[l]"~Spalte &
Ein zentrierter Blindtext zur Demonstration einer X[c]"~Spalte &
Ein rechtsbündiger Blindtext zur Demonstration einer X[r]"~Spalte &
Ein im Blocksatz gesetzter Blindtext zur Demonstration einer X[j]"~Spalte
\tabularnewline
\bottomrule
\end{tabu}
\end{table}
\end{Excerpt*}
\InputExcerpt
\end{document}
