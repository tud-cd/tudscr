\documentclass[english,ngerman,class=scrreprt]{standalone}
\input{!myPath}
\input{\myPath Grafiken/myGraphicPreamble}

% zum Einbinden von Grafiken relativ zur Hauptdatei in pstricks
%\includegraphicspstricks[]{}
% zum Einbinden von Grafiken relativ zur Hauptdatei in TikZ
%\includegraphicstikz[]{}

\begin{document}

\settowidth{\myGlength}{Stabilisierungsi}

\tikzstyle{inner box}=[%
	text width=\myGlength,align=center,rectangle,minimum height=8ex,draw
]
\tikzstyle{inner label}=[%
	align=center,font=\tabfloatfont\scriptsize
]

\tikzstyle{inner box chain}=[%
	every node/.style={on chain}
]
\tikzstyle{inner box chain below}=[%
	inner box chain,node distance=90\myGunit,continue chain=going below
]
\tikzstyle{inner box chain right}=[%
	inner box chain,node distance=350\myGunit,continue chain=going right
]
\tikzstyle{inner box chain above}=[%
	inner box chain,node distance=180\myGunit,continue chain=going above
]

\begin{tikzpicture}
	\begin{scope}[start chain]
		\begin{scope}[inner box chain below]
			\node(NE)[inner box]{\hyperref[in:Navigationsebene]{Navigations\-ebene}};
			\node(NB)[inner label]{gewählte Fahrtroute\\zeitlicher Ablauf};
			\node(BE)[inner box]{\hyperref[in:Bahnfuehrungsebene]{Bahnführungs\-ebene}};
			\node(BS)[inner label]{gewählte Führungsgrößen:\\Sollspur, Sollgeschwindigkeit};
			\node(SE)[inner box]{\hyperref[in:Stabilisierungsebene]{Stabilisierungs\-ebene}};
		\end{scope}
		\begin{scope}[inner box chain right]
			\node(LQ)[inner box]{Längs- und Querdynamik};
			\node(FO)[inner box]{Fahrbahn\-oberfläche};
		\end{scope}
		\begin{scope}[inner box chain above]
			\node(FR)[inner box]{Fahrraum\\\smallskip{\scriptsize Straße und\\\vspace{-1.5ex}Verkehrssituation}};
			\node(SN)[inner box]{Straßennetz};
		\end{scope}
	\end{scope}
		
	\begin{scope}[inner label,minimum size=0pt]
		\draw [pstarrow->] (FO) -| ++(130,-120) to node [above]{Istspur, Istgeschwindigkeit} ++(-960,0) |- (SE);
		\draw [pstarrow->] (FR) -| ++(140	,-330) to node [above]{Bereich sicherer Führungsgrößen} ++(-980,0) |- (BE);
		\draw [pstarrow->] (SN) -| ++(150	,-540) to node [above]{mögliche Fahrtroute} ++(-1000,0) |- (NE);
	\end{scope}
	
	\begin{scope}[inner label]
		\draw              (NE) to (NB)
					[pstarrow->] (NB) to (BE);
		\draw  						 (BE) to (BS)
					[pstarrow->] (BS) to (SE);
		\draw [pstarrow->] (SE) to  node[above]{Stellgrößen}
				  											node[below]{\parbox{4.5em}{\centering\hspace{0pt}Lenken Gasgeben Bremsen}}(LQ);
		\draw [pstarrow->] (LQ) to  node[above]{Regelgrößen}
				  											node[below]{\parbox{4.5em}{\centering\hspace{0pt}Fahrzeugbewegung}}(FO);
		\draw [pstarrow->] (LQ)+(245,0) |- (FR);
		\draw [pstarrow->] (LQ)+(245,0) |- (SN);
	\end{scope}
	
	\begin{scope}[very thick,rounded corners=50\myGunit]
		\draw (-120,-440) rectangle (120,120);
		\draw ( 230,-440) rectangle (470,-240);
		\draw ( 580,-440) rectangle (820,120);
	\end{scope}
	
	\begin{scope}[font=\tabfloatfont\bfseries]
		\node at (0,80) {Fahrer};
		\node at (350,-280) {Fahrzeug};
		\node at (700,80) {Umwelt};
	\end{scope}	
\end{tikzpicture}

\end{document}
