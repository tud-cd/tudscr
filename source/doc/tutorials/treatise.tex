\RequirePackage[ngerman=ngerman-x-latest]{hyphsubst}
\documentclass[%
  english,ngerman,%
  geometry=no,DIV=12,automark,%
]{tudscrartcl}
\usepackage{selinput}
\SelectInputMappings{adieresis={ä},germandbls={ß}}
\usepackage[T1]{fontenc}
\usepackage{lmodern}

\usepackage{tudscrman}
\lstset{%
  inputencoding=utf8,extendedchars=true,
  literate=%
    {ä}{{\"a}}1 {ö}{{\"o}}1 {ü}{{\"u}}1
    {Ä}{{\"A}}1 {Ö}{{\"O}}1 {Ü}{{\"U}}1
    {~}{{\textasciitilde}}1 {ß}{{\ss}}1
}
\TUDoptions{cdfont=false}
\KOMAoptions{headings=normal}

\usepackage{tudscrsupervisor}
\usepackage{array}
\usepackage{tabu}
\usepackage{tabularx}
\usepackage{booktabs}
\usepackage{units}
\AfterPackage*{hyperref}{%
  \usepackage[%
    automake,%
    acronym,%
    symbols,%
    nomain,%
    translate=babel,%
    nogroupskip,%
  ]{glossaries}
  \setStyleFile{\jobname-temp}
  \renewcommand*{\glsglossarymark}[1]{}
  \makeglossaries
}
\usepackage{csquotes}
\usepackage[backend=biber,style=alphabetic]{biblatex}
\usepackage{filecontents}
\addbibresource{treatise-temp.bib}

\usepackage{enumitem}

\usepackage{caption}
\usepackage{floatrow}
\renewcommand{\floatpagefraction}{0.7}

\ifpdf
  \usepackage{tikz}
  \usetikzlibrary{chains}
  \usetikzlibrary{decorations.markings}
  \tikzset{on grid}
\fi

\usepackage{pstricks,pst-node}

\makeatletter
\newcommand*\pcolumnfuzz[1]{\pretocmd{\@endpbox}{\hfuzz=#1}{}{}}
\makeatother

\usepackage[open,openlevel=0]{bookmark}[2011/12/02]

\begin{document}
\title{%
  Ein Anwenderleitfaden für das Erstellen einer wissenschaftlichen Abhandlung%
}
\author{Falk Hanisch\thanks{\noexpand\href{mailto:\tudscrmail}{\tudscrmail}}}
\date{03.11.2014}
\makeatletter
\begingroup%
  \def\and{, }%
  \let\thanks\@gobble%
  \let\footnote\@gobble%
  \hypersetup{%
    pdfauthor = {\@author},%
    pdftitle = {\@title},%
    pdfsubject = {Tutorial für \hologo{LaTeXe}},%
    pdfkeywords = {LaTeX, \TUDScript, Tutorial, Anwenderleitfaden},%
  }%
\endgroup%
\makeatother
\StartTutorial[%
  \begin{abstract}\noindent
  Der Versuch, ein allumfassendes Tutorial für eine wissenschaftliche Arbeit 
  zur Verfügung zu stellen gleicht der beschwerlichen Suche nach einer 
  eierlegenden Wollmilchsau. Es ist quasi nicht möglich, alle möglichen 
  Anforderungen an eine wissenschaftliche Arbeit in einem Dokument abzudecken. 
  Dennoch soll hier versucht werden, einen Großteil der für gewöhnlich 
  auftretenden Erfordernisse zu bearbeiten.
  
  Dieses Tutorial hat \emph{nicht} die Intention, \hologo{LaTeX}-Einsteigern 
  sämtliche Grundlagen zu erläutern. Vielmehr wird davon ausgegangen, dass Sie 
  bereits erste Erfahrungen mit \hologo{LaTeXe} gesammelt haben. Dennoch wird 
  versucht, alle Schritte möglichst leicht nachvollziehbar zu gestalten. Sollte 
  Ihnen beim Lesen und Durcharbeiten des Tutorials etwas auf- oder missfallen, 
  so dürfen Sie mich gerne per E-Mail kontaktieren. Auch Anregungen und Wünsche 
  dürfen sie mir gegenüber gerne kommunizieren.
  
  Für absolute Neueinsteiger gibt es einige freie Tutorials, welche die ersten 
  Schritte mit \hologo{LaTeXe} stark erleichtern. Sehr empfehlenswert ist die 
  ausführliche \hrfn{http://www.fadi-semmo.de/latex/workshop/}{Workshop-Reihe} 
  von Fadi~Semmo. Außerdem stellt Nicola~L.~C.~Talbot sehr gute Tutorials für 
  \hrfn{http://www.dickimaw-books.com/latex/novices/}{\hologo{LaTeX}-Novizen} 
  sowie \hrfn{http://www.dickimaw-books.com/latex/thesis/}{Dissertationen} zur 
  freien Verfügung.
  
  In erster Linie ist dieser Leitfaden für Anwender gedacht, die für ihre
  wissenschaftliche Arbeit eine \TUDScript"=Dokumentklasse verwenden wollen. 
  Das vorgestellte Vorgehen kann jedoch~-- natürlich mit gewissen Abstrichen~-- 
  auch mit anderen Klassen, insbesondere denen aus dem \KOMAScript"=Bundle, 
  umgesetzt werden. Viele der hier verwendeten Optionen und Befehle aus dem 
  \TUDScript-Bundle werden nur sporadisch in ihrer Grundfunktion erläutert. 
  Eine detaillierte Erläuterung lässt sich jedoch jederzeit sehr einfach über 
  die farbigen Hyperlinks im \manualhyperref{}{\TUDScript-Handbuch} öffnen.
  Des Weiteren wird im Tutorial auf eine Vielzahl von Pakete verwiesen, deren 
  Dokumentation sich entweder über den für jedes Paket erzeugten Hyperlink auf 
  das \href{http://www.ctan.org/}{Comprehensive TeX Archive Network (CTAN)} 
  oder alternativ über die Kommandozeile respektive das Terminal mit dem Aufruf 
  \PValue{texdoc }\PName{Paket} direkt öffnen lässt.
  
  Der Anwenderleitfaden muss nicht zwingend vollständig nachvollzogen werden. 
  Dieser ist in einzelne Abschnitte untergliedert, damit Sie sich bestimmte 
  Aspekte erarbeiten können. Sollten Querbezüge zu den einzelnen Abschnitten 
  bestehen, werden diese auch genannt. Zu guter Letzt findet sich am Ende 
  dieses Dokumentes das komplette Tutorial als ausführbarer Quelltext. 
  \end{abstract}
]
\tableofcontents
\listoffigures
\listoftables



\section{Einleitung}
\label{sec:introduction}
Zu Beginn werden allerhand Pakete geladen. Bei einigen wird in der Einleitung 
nicht weiter darauf eingegangen, diese werden in den relevanten Abschnitten in 
diesem Tutorial genauer erläutert. 

Den Anfang macht das Paket \Package{hyphsubst}. Dieses wird für eine wesentlich 
verbesserte Worttrennung für die deutsche Sprache benötigt und muss bereits 
\emph{vor} der Klasse geladen werden, damit es problemlos funktioniert. Näheres 
können Sie in \autoref{sec:hyphenation} nachlesen.
%
\begin{Preamble}
\RequirePackage[ngerman=ngerman-x-latest]{hyphsubst}
\end{Preamble}
%
Beim Laden der Klasse mit \Macro{documentclass} sollten die im Dokument 
verwendeten Sprachen als Klassenoption angegeben werden, wobei die zuletzt 
angegebene als aktuelle Sprache aktiviert wird. Dadurch werden diese nicht nur 
an das Paket \Package{babel} sondern auch an andere Pakete weitergereicht, 
welche sprachspezifische Einstellungen vornehmen.
%
\begin{Preamble}
\documentclass[english,ngerman]{tudscrreprt}
\usepackage{babel}
\end{Preamble}
%
Bei der Verwendung von \hologo{LaTeXe} sollte zum einen die Eingabekodierung 
des erstellten Datei spezifiziert werden. Das Paket \Package{selinput} erkennt 
automatisch, welche Kodierung der genutzte Editor verwendet. Zum anderen werden 
die Schriften in der Ausgabe ebenfalls kodiert. Mit dem Paket \Package{fontenc} 
lässt sich die Schriftkodierung spezifizieren, wobei die europäischen Zeichen 
mit der Option~\Option{T1} aktiviert werden.
%
\begin{Preamble}
\usepackage{selinput}\SelectInputMappings{adieresis={ä},germandbls={ß}}
\usepackage[T1]{fontenc}

\end{Preamble}
%
Das Paket \Package{fixltx2e} behebt einige Fehler im \hologo{LaTeXe}-Kernel. 
In neuen Dokumenten kann es bedenkenlos geladen werden.
%
\begin{Preamble}
\usepackage{fixltx2e}

\end{Preamble}
%
Damit sind die allgemein notwendigen Pakete eingebunden. Es werden zwar weitere 
benötigt, diese werden allerdings in den einzelnen Abschnitten dieses Tutorials 
aufgeführt.



\section{Satzspiegel und Bindekorrektur}
Gleich zu Beginn und bevor das eigentliche Verfassen der Arbeit beginnt, sollte 
man sich Gedanken über das zu nutzenden Layout und den Satzspiegel machen, um 
bei der Finalisierung keine böse Überraschung bei Seitenumbrüchen oder der 
Position von Gleitobjekten zu erleben.

Zuallererst gilt zu entscheiden, ob das Dokument einseitig oder beidseitig 
gesetzt werden soll. Ist Letzteres der Fall, so sollte \Option{twoside} als 
Klassenoption angegeben werden. Im nächsten Schritt ist der zu verwendenden 
Satzspiegel festzulegen. Hierfür kann die Option \Option*{geometry} verwendet 
werden, welche im \TUDScript-Handbuch beschrieben wird. Normalerweise wird das 
Dokument im asymmetrischen Layout des \CDs gesetzt. Dieses Verhalten kann mit 
\Option*{geometry}[false] deaktiviert werden und der Satzspiegel wird durch das 
Paket \Package{typearea} nach typographischen Gesichtspunkten konstruiert.

Falls die Arbeit nach der Fertigstellung gebunden werden soll, so ist auf den 
notwendigen Binderand zu achten, quasi der Teil einer Seite, welcher durch die 
Bindung \enquote{verschwindet} und nicht mehr als sichtbarer Teil der Seite 
vorhanden ist. Als Faustregel gilt, dass die erforderliche Bindekorrektur in 
etwa der halben Höhe des Buchblocks entsprechen sollte. Dessen Höhe wiederum 
ist abhängig von der Anzahl der Seiten sowie der Papierdichte. Wird qualitativ 
höherwertiges Papier mit einer Dichte von \unit[100]{g/m²} verwendet, so 
entsprechen 100~Blatt in etwa einer Höhe von \unit[12]{mm}. Dementsprechend 
wäre bei diesem Beispiel eine Bindekorrektur \unit[6]{mm} notwendig. Diese 
kann mit der Klassenoption \Option*{BCOR}[6mm] eingestellt werden.



\section{Umschlagseite und Titel}
Umschlagseite und Titel sind sich in ihrer Gestalt sehr ähnlich. Allerdings 
gibt es ein paar kleine Unterschiede. Zum einen werden auf dem Cover weniger 
Informationen als auf der Titelseite ausgegeben. Zum anderen wird der Titel 
immer im Satzspiegel des restlichen Dokumentes ausgegeben, wohingegen die 
Umschlagseite ohne weitere Optionen im asymmetrischen Layout des \CDs der \TnUD 
erscheint. Wie dieses Verhalten geändert werden kann, ist im Handbuch für 
\Macro*{makecover} erläutert. Die resultierende Ausgabe des nachfolgenden 
Quelltextauszugs ist in \autoref{fig:title} zu sehen.
%
\begin{figure}
\IncludeStandalone{Title}[1,2]
\caption{Umschlagseite und Titel}
\label{fig:title}
\end{figure}
\begin{Trunk!}{Title}
\faculty{Juristische Fakultät}
\department{Fachrichtung Strafrecht}
\institute{Institut für Kriminologie}
\chair{Lehrstuhl für Kriminalprognose}
\title{%
  Entwicklung eines optimalen Verfahrens zur Eroberung des
  Geldspeichers in Entenhausen
}
\thesis{master}
\graduation[M.Sc.]{Master of Science}
\author{%
  Mickey Mouse
  \matriculationnumber{12345678}
  \dateofbirth{2.1.1990}
  \placeofbirth{Dresden}
  \course{Klinische Prognostik}
  \discipline{Individualprognose}
\and%
  Donald Duck\matriculationnumber{87654321}
  \dateofbirth{1.2.1990}
  \placeofbirth{Berlin}
  \course{Statistische Prognostik}
  \discipline{Makrosoziologische Prognosen}
}
\matriculationyear{2010}
\supervisor{Dagobert Duck \and Mac Moneysac}
\professor{Prof. Dr. Kater Karlo}
\date{10.09.2014}

\makecover
\maketitle

\end{Trunk!}



\section{Vor- und Nachspann}
In den folgenden Unterabschnitten werden Elemente vorgestellt, welche häufig 
als Bestandteil einer wissenschaftlichen (Abschluss"~)Arbeit gefordert werden, 
wobei meistens nur eine Teilmenge verlangt wird. Die Platzierung oder Position 
der vorgestellten Elemente innerhalb der Arbeit ist nicht eindeutig durch eine 
Norm oder dergleichen festgelegt. Vielmehr gibt es meist eine Richtlinie vom 
verantwortlichen Prüfungsamt oder eine konkrete Vorgabe des wissenschaftlichen 
Mitarbeiters respektive betreuenden Hochschullehrers.


\subsection{Aufgabenstellung}
\label{sec:task}
Das Erstellen der Aufgabenstellung einer Abschlussarbeit im \CD der \TnUD muss 
das Paket \Package*{tudscrsupervisor} geladen werden.
%
\begin{Preamble}
\usepackage{tudscrsupervisor}

\end{Preamble}
%
Dieses stellt die Umgebung \Environment*{task} und den Befehl \Macro*{taskform} 
bereit. Bei beiden Varianten wird zu Beginn eine Tabelle mit Informationen zum 
Autor erzeugt. Am Ende werden der oder die Betreuer der Arbeit sowie Professor 
und gegebenenfalls der Prüfungsausschussvorsitzende ausgegeben. Dazwischen kann 
in der \Environment*{task}-Umgebung ein beliebiger Inhalt gesetzt werden. Der 
Befehl \Macro*{taskform} hingegen erzeugt eine standardisierte Ausgabe, wobei 
das zweite obligatorische Argument innerhalb der \Environment{itemize}-Umgebung 
verwendet wird und somit \Macro{item} zu nutzen ist. Das Resultat des folgenden 
Quelltextes ist in \autoref{fig:task} zu sehen. 
%
\begin{figure}
\IncludeStandalone{Task}[1,2]
\caption{Aufgabenstellung in freier und standardisierter Form}
\label{fig:task}
\end{figure}
\begin{Hint!}{Task}
\faculty{Juristische Fakultät}
\department{Fachrichtung Strafrecht}
\institute{Institut für Kriminologie}
\chair{Lehrstuhl für Kriminalprognose}
\title{%
  Entwicklung eines optimalen Verfahrens zur Eroberung des
  Geldspeichers in Entenhausen
}
\thesis{master}
\graduation[M.Sc.]{Master of Science}
\author{%
  Mickey Mouse
  \matriculationnumber{12345678}
  \dateofbirth{2.1.1990}
  \placeofbirth{Dresden}
  \course{Klinische Prognostik}
  \discipline{Individualprognose}
\and%
  Donald Duck\matriculationnumber{87654321}
  \dateofbirth{1.2.1990}
  \placeofbirth{Berlin}
  \course{Statistische Prognostik}
  \discipline{Makrosoziologische Prognosen}
}
\matriculationyear{2010}
\issuedate{1.2.2015}
\duedate{1.8.2015}
\supervisor{Dagobert Duck \and Mac Moneysac}
\professor{Prof. Dr. Kater Karlo}
\chairman{Prof. Dr. Primus von Quack}

\newcommand\taskcontent{%
  Momentan ist das besagte Thema in aller Munde. Insbesondere wird es
  gerade in vielen~-- wenn nicht sogar in allen~-- Medien diskutiert.
  Es ist momentan noch nicht abzusehen, ob und wann sich diese Situation
  ändert. Eine kurzfristige Verlagerung aus dem Fokus der Öffentlichkeit
  wird nicht erwartet.
  
  Als Ziel dieser Arbeit soll identifiziert werden, warum das Thema
  gerade so omnipräsent ist und wie man diesen Effekt abschwächen
  könnte. Zusätzlich sollen Methoden entwickelt werden, wie sich ein
  ähnlicher Vorgang zukünftig vermeiden ließe.
}

\begin{task}
\smallskip
\par\noindent
\taskcontent
\end{task}

\taskform[pagestyle=empty]{\taskcontent}{%
  \item Recherche
  \item Analyse
  \item Entwicklung eines Konzeptes
  \item Anwendung der entwickelten Methodik
  \item Dokumentation und grafische Aufbereitung der Ergebnisse
}

\end{Hint!}
\begin{Trunk+}
\newcommand\taskcontent{%
  Momentan ist das besagte Thema in aller Munde. Insbesondere wird es
  gerade in vielen~-- wenn nicht sogar in allen~-- Medien diskutiert.
  Es ist momentan noch nicht abzusehen, ob und wann sich diese Situation
  ändert. Eine kurzfristige Verlagerung aus dem Fokus der Öffentlichkeit
  wird nicht erwartet.
  
  Als Ziel dieser Arbeit soll identifiziert werden, warum das Thema
  gerade so omnipräsent ist und wie man diesen Effekt abschwächen
  könnte. Zusätzlich sollen Methoden entwickelt werden, wie sich ein
  ähnlicher Vorgang zukünftig vermeiden ließe.
}
\taskform[pagestyle=empty]{\taskcontent}{%
  \item Recherche
  \item Analyse
  \item Entwicklung eines Konzeptes
  \item Anwendung der entwickelten Methodik
  \item Dokumentation und grafische Aufbereitung der Ergebnisse
}

\end{Trunk+}


\subsection{Zusammenfassung}
Häufig wird zu Beginn einer wissenschaftliche Arbeit die Motivation und der 
Inhalt dieser zusammengefasst, um den Leser die Thematik der Abhandlung 
vorzustellen. in den meisten Fällen wird diese dabei in deutscher und 
englischer Sprache verfasst. Hierfür stellt \KOMAScript{} bereits die Umgebung 
\Environment*{abstract} bereit. Vielfach wird der Wunsch geäußert, sowohl die 
deutsche als auch die englische Zusammenfassung auf derselben Seite zu setzen. 
Diese Variante kann mithilfe der \TUDScript-Klassen sehr einfach umgesetzt 
werden, wie der nachfolgende Quelltextauszug zeigt. Die resultierende Ausgabe 
ist in \autoref{fig:abstr} zu sehen.
%
\begin{figure}
\centering
\IncludeStandalone[width=.5\textwidth]{Abstract}
\caption{Zusammenfassung in deutscher und englischer Sprache}
\label{fig:abstr}
\end{figure}
\begin{Trunk!}{Abstract}
\TUDoption{abstract}{multiple,section}
\begin{abstract}
  Dies ist der deutschsprachige Teil der Zusammenfassung, in dem die
  Motivation sowie der Inhalt der nachfolgenden wissenschaftlichen
  Abhandlung kurz dargestellt werden.
\nextabstract[english]
  This is the english part of the summary, in which the motivation and
  the content of the following academic treatise are briefly presented.
\end{abstract}

\end{Trunk!}



\subsection{Erklärungen}
Für die meisten Abschlussarbeiten an der \TnUD wird vom Verfasser eine 
Selbstständigkeitserklärung verlangt. Für diese wird ein Standardtext 
bereitgestellt. Dieser kann mit dem Befehl \Macro*{confirmation} ausgegeben 
werden. Wurde das Thema in Kooperation mit einem Unternehmen bearbeitet, so 
wird zumeist auch ein Sperrvermerk gefordert, welcher mit \Macro*{blocking} 
erzeugt werden kann. Mit \Macro*{declaration} werden beide Erklärungen direkt 
nacheinander erzeugt. Die verwendete Überschrift und ein möglicher Eintrag in 
das Inhaltsverzeichnis können über die Option \Option*{declaration} reguliert 
werden. Eine mögliche Ausprägung der Erklärungen ist in \autoref{fig:decl} 
abgebildet.
%
\begin{figure}
\centering
\IncludeStandalone[width=.5\textwidth]{Declaration}
\caption{Selbstständigkeitserklärung und Sperrvermerk}
\label{fig:decl}
\end{figure}
\begin{Hint!}{Declaration}
\title{%
  Entwicklung eines optimalen Verfahrens zur Eroberung des
  Geldspeichers in Entenhausen
}
\author{Mickey Mouse\and Donald Duck}
\declaration[company=FIRMA]
\end{Hint!}
\begin{Trunk+}
\declaration[company=FIRMA]

\end{Trunk+}

\subsection{Inhalts-, Abbildungs-, und Tabellenverzeichnis}
Zum Inhaltsverzeichnis ist nicht allzu viel zu sagen. Dieses wird mit dem
\hologo{LaTeX}"=Standardbefehl \Macro{tableofcontents} erzeugt und führt die 
Gliederung des erstellten Dokumentes entsprechend der verwendeten Befehle 
(\Macro{part}, \Macro{addpart}, \Macro{chapter}, \Macro{addchap}, 
\Macro{section}, \Macro{addsec} etc.) auf. Wurde das Paket \Package{hyperref} 
geladen, so werden im Inhaltsverzeichnis PDF-Hyperlinks auf die einzelnen 
Abschnitte erzeugt.

Sowohl Abbildungen als auch Tabellen werden in \hologo{LaTeX} normalerweise 
mit speziellen Umgebungen~-- \Environment{figure} und \Environment{table}~-- 
eingebunden. Innerhalb dieser sogenannten Gleitumgebungen kann der Befehl 
\Macro{caption}\OParameter{Verzeichniseintrag}\Parameter{Bezeichnung} genutzt 
werden, um diesen eine Bezeichnung hinzuzufügen. Mit \Macro{listoffigures} 
beziehungsweise \Macro{listoftables} lassen sich Verzeichnisse erstellen, in 
denen alle Gleitobjekte des jeweiligen Typs ausgegeben werden, falls diese denn 
eine Bezeichnung hinzugefügt wurde. Sollen Abbildungen oder Tabellen außerhalb 
ihrer angestammten Gleitumgebung genutzt und benannt werden, kann dies mit dem 
Befehl \Macro{captionof}\Parameter{Typ}\OParameter{Verzeichniseintrag}
\Parameter{Bezeichnung} erfolgen. Weitere Informationen diesbezüglich sind der 
\KOMAScript"=Anleitung \scrguide zu entnehmen. In \autoref{sec:floats} wird 
außerdem genauer auf die Verwendung von Gleitumgebungen eingegangen.
%
\begin{Trunk}
\tableofcontents
\listoffigures
\listoftables

\end{Trunk}
\begin{Trunk+}
\printacronyms[style=acrotabu]
\printsymbols[style=symblongtabu]

\setchapterpreamble{%
  \renewcommand*\dictumwidth{.4\textwidth}%
  \dictum[Johann Wolfgang von Goethe]{%
    Es irrt der Mensch, solang er strebt.%
  }%
  \bigskip
}
\chapter{Einleitung}
Nachdem nun der Vorspann und~-- bis auf das Literaturverzeichnis am 
Ende des Dokumentes auf Seite~\pageref{sec:bibliography}~-- alle 
Verzeichnisse erfolgreich ausgegeben wurden, wird nun die Verwendung 
der weiteren Umgebungen und Befehle demonstriert, welche im Tutorial 
\texturn{treatise.pdf} vorgestellt wurden.

\end{Trunk+}


\subsection{Abkürzungs- und Symbolverzeichnis}
\label{sec:glossaries}
Für die Auszeichnung von Abkürzungen gibt es zwei sehr gute Pakete, die dieses 
Unterfangen stark vereinfachen. Die einfachere~-- jedoch nicht so mächtige~-- 
der beiden Varianten ist die Nutzung des Paketes \Package{acro}. Sollen nur 
Abkürzungen und gegebenenfalls eine sortierte Liste dieser gesetzt werden, ist 
dieses allerdings absolut ausreichend. Für ein Symbolverzeichnis lässt sich in 
dieser Variante das Paket \Package{nomencl} nutzen. Dieses bietet meiner 
Meinung nach jedoch keine großen Vorteile, stattdessen kann auch einfach eine 
Tabelle händisch erzeugt werden. Das Paket \Package{acro} ist sehr gut und 
ausführlich dokumentiert. Deshalb wird hier auf eine exemplarische Erläuterung 
verzichtet und stattdessen auf dessen Dokumentationen verwiesen.

Die andere Möglichkeit ist die Nutzung des Paketes \Package{glossaries}, das 
eine große Zahl an Einstellmöglichkeiten und Optionen besitzt, allerdings auch 
etwas Zeit für die Einarbeitung und Studium der Dokumentation benötigt. Der 
ursprüngliche Einsatzzweck dieses Paketes ist das Setzen eines fachsprachlichen 
oder technischen Glossars. Es bietet zusätzlich die Mittel zum Erzeugen eines 
Abkürzungs- sowie Symbolverzeichnis. Die Dokumentation von \Package{glossaries} 
lässt ebenfalls keine Wünsche offen. Dennoch soll folgend hier kurz erläutert 
werden, wie das Paket zu verwenden ist. Für weiterführende Beispiele sollte 
die Dokumentationen zu Rate gezogen werden.

Das Paket \Package{glossaries} sollte immer \emph{nach} \Package{hyperref} 
geladen werden. Entweder Sie achten explizit darauf oder Sie verwenden den 
Befehl \Macro{AfterPackage*} aus dem \KOMAScript-Bundle. Mit diesem können Sie 
die Ausführung von Quelltext bis zum Laden eines Paketes verzögern. Allerdings 
ist darauf zu achten, dass der Quelltext nur ausgeführt wird, wenn auch das 
avisierte Zielpaket geladen wird. Die für \Package{glossaries} verwendeten 
Optionen werden kurz erläutert.
%
\begin{Preamble+}
\AfterPackage*{hyperref}{%
\end{Preamble+}
\begin{Preamble}
\usepackage[%
\end{Preamble}
%
Das Programm \Application{makeindex} wird im Normalfall durch die genutzte 
\hologo{LaTeX}"=Distribution bereitgestellt und für das alphabetische Sortieren 
der erstellten Listen verwendet. Mit der Paketoption \Option{automake} erfolgt 
der automatische Aufruf von \Application{makeindex} mit den passenden 
Einstellungen für alle Verzeichnisse. Alternativ dazu kann die Option 
\Option{xindy} aktiviert werden, welche \Application{xindy} anstelle von 
\Application{makeindex} für das Sortieren verwendet. Diese Programm bietet 
unter anderem eine Unterstützung von Unicode sowie die Möglichkeit, nach 
sprachabhängigen Regeln zu sortieren. Allein für die deutsche Sprache gibt es 
beispielsweise zwei verschiedene Varianten~-- nach DIN und nach Duden~-- zum 
alphabetischen Sortieren. Allerdings wird das Programm \Application{xindy} 
lediglich mit der Distribution \Distribution{\hologo{TeX}~Live} jedoch nicht 
mit \Distribution{\hologo{MiKTeX}} geliefert. Wenn ohnehin die Distribution 
\Distribution{\hologo{TeX}~Live} verwendet wird, würde ich persönlich die 
Verwendung von \Application{xindy} vorziehen. Für das Erstellen der Glossare 
sollte das Perl"=Skript \Application{makeglossaries} verwendet werden, welches 
alle notwendigen Optionen an \Application{xindy} weiterleitet. Treten Probleme 
bei der Erzeugung der einzelnen Glossare auf, sollte die Dokumentation von 
\Package{glossaries} weiterhelfen können.
%
\begin{Preamble}
  automake,%
%  xindy,\%={language=german-din},\% mit Tex Live einfach verwendbar
\end{Preamble}
%
Die Optionen \Option{acronym} und \Option{symbols} erzeugen die Glossare 
beziehungsweise die Verzeichnisse für Abkürzungen und Symbole. Die Option  
\Option{nomain} wird verwendet, weil in diesem Tutorial kein zusätzliches 
technisches Glossar erzeugt werden soll.
%
\begin{Preamble}
    acronym,% Abkürzungen
    symbols,% Formelzeichen
    nomain,% kein Glossar
\end{Preamble}
%
Mit der Option \Option{translate}[babel] werden die Überschriften der Glossare 
in der Dokumentsprache gesetzt, mit \Option{nogroupskip} kann der automatische 
Abstand zwischen den Einträgen zur Gruppierung innerhalb eines Glossars 
entfernt werden. Die Option \Option{toc} fügt die Verzeichnisse dem 
Inhaltsverzeichnis hinzu, mit \Option{section} kann die Gliederungsebene für 
die Überschrift angegeben werden.
%
\begin{Preamble}
  translate=babel,%
  nogroupskip,%
  toc,%
  section=chapter,%
\end{Preamble}
%
Damit sind alle verwendeten Optionen erläutert. Schließlich sorgt der Befehl 
\Macro{makeglossaries} für das Erstellen der optionsabhängigen Stildateien für 
\Application{makeindex} respektive \Application{xindy} sowie das Erzeugen der 
benötigten Hilfsdateien.
%
\begin{Preamble}
]{glossaries}
\makeglossaries

\end{Preamble}
%
Damit wäre der erste Teil zur Initialisierung überstanden und wir können zum 
eigentlichen Problem kommen. Wie erstellt man nun ein Abkürzungs- und/oder 
Symbolverzeichnis?

\subsubsection{Abkürzungsverzeichnis}
Das Paket \Package{glossaries} stellt für die Definition von Abkürzungen einen 
speziellen Befehl bereit. Mit\Macro{newacronym}\LParameter\Parameter{Label}%
\Parameter{Abkürzung}\Parameter{Wortgruppe} wird eine Abkürzung definiert und 
kann später über \Parameter{Label} genutzt werden. Die möglichen optionalen 
Parameter können in der Dokumentation zu \Package{glossaries} nachgeschlagen 
werden. Für ein kleines Beispiel werden drei Abkürzungen erstellt\dots
%
\begin{Trunk*}
\newacronym{apsp}{APSP}{All-Pairs Shortest Path}
\newacronym{spsp}{SPSP}{Single-Pair Shortest Path}
\newacronym{sssp}{SSSP}{Single-Source Shortest Path}

\end{Trunk*}
%
\dots und diese in einer kurzen Textpassage mit \Macro{gls}\Parameter{Label} 
verwendet.
%
\begin{Trunk+}
\section{Die Verwendung von Abkürzungen}
\end{Trunk+}
\begin{Trunk*}
In der Graphentheorie wird häufig die Lösung des Problems des kürzesten
Pfades zwischen zwei Knoten gesucht. Dieses Problem wird häugig auch
mit \gls{spsp} bezeichnet. Es lässt sich auf die Variationen \gls{sssp}
und \gls{apsp} erweitern. Für die Lösung von \gls{spsp}, \gls{sssp} 
oder \gls{apsp} kommen unterschiedliche Algorithmen zum Einsatz.

\end{Trunk*}
%
Gut zu sehen ist, dass sich die Ausgabe der Abkürzung bei der ersten Verwendung 
mit \Macro{gls} von der zweiten~-- und jeder weiteren~-- unterscheidet. Das 
Verhalten lässt sich über verschiedene Stile mit \Macro{setacronymstyle} 
anpassen. Die Ausgabe einer Liste aller Abkürzungen erfolgt mit:
%
\begin{Hint}
\printacronyms
\end{Hint}
\begin{quoting}[rightmargin=0pt]
\vspace*{-\baselineskip}
\glsdisablehyper
\InputCode
\end{quoting}
%
Dabei werden die Abkürzungen in einer \Environment{description}"=Umgebung 
gesetzt. Dies ist absolut ausreichend. Mir persönlich ist die Darstellung in 
einer quasi-tabellarischen Form jedoch lieber. Dabei soll eigentliche Stil mit 
fettgedruckter Abkürzung beibehalten werden. Das \Package{glossaries}"=Paket 
stellt zwar auch eine Menge Stilen in Tabellenform bereit, allerdings nicht in 
dem gewünschten Stil. Deshalb wird nachfolgend gezeigt, wie ein eigener Stil in 
Tabellenform kreiert werden kann. 

Es bieten sich die beiden Tabellenumgebungen \Environment{tabularx} oder 
\Environment{tabu} an, bei denen die Spaltenbreite teilweise automatisch 
berechnet wird, um sich manuelle Formatierungsarbeiten zu sparen. Hier werden 
beide Varianten vorgestellt. Wie Sie Stile definieren, die auch Seitenumbrüche 
in der Tabelle zulassen, können Sie etwas später bei Symbolverzeichnissen in 
diesem \autorefname[sec:glossaries] erfahren. Falls Sie noch keine Erfahrungen 
mit dem Satz von Tabellen in \hologo{LaTeX} haben, lohnt sich vorher vielleicht 
ein Blick in \autoref{sec:tables}, um die verwendeten Befehle und Umgebungen zu 
verstehen.

\minisec{Eigener Stil mit \Environment{tabularx}}
Für das Definieren eines eigenen Glossarstils wird \Macro{newglossarystyle} 
verwendet, wobei für den neuen Stils die Umgebung \Environment{theglossary} 
umdefiniert wird. Als erstes wird \Environment{tabularx} verwendet, welches in 
\autoref{sec:tabularx} genauer vorgestellt wird.

Es werden drei Spalten definiert. Die erste und letzte Spalte sind schlicht 
linksbündig~(\PValue{l}). In diesen werden die Abkürzung selbst respektive die 
Seitenzahl eingetragen. Die Verwendung von~\PValue{@\{\}} führt dazu, dass der 
normalerweise vor der ersten und nach der letzten Spalte eingefügte Abstand von 
\Length{tabcolsep} entfällt. Die Breite der Spalte vom Typ~\PValue{X} wird 
automatisch berechnet. Aufgrund der Implementierung von \Environment{tabularx} 
kann man diese nicht als verschachtelte Umgebung verwenden sondern muss diese 
direkt in der Low"~Level"~Variante (\Macro{tabularx} und \Macro{endtabularx}) 
nutzen. Die Definition des neuen Stils \PValue{acrotabularx} wird nachfolgend 
ausgegeben, die weitergehende Erläuterung schließt sich daran an.
%
\CodeHook{\let\newglossarystyle\renewglossarystyle}
\begin{Preamble*}
\newglossarystyle{acrotabularx}{%
  \renewenvironment{theglossary}{%
    \tabularx{\linewidth}{@{}lXl@{}}%
  }{%
    \endtabularx\par\bigskip%
  }%
  \renewcommand*{\glossaryheader}{}%
  \renewcommand*{\glsgroupheading}[1]{}%
  \renewcommand*{\glsgroupskip}{}%
  \renewcommand*{\glossentry}[2]{%
    \glsentryitem{##1}% Entry number if required
    \glstarget{##1}{\sffamily\bfseries\glossentryname{##1}} &
    \glsentrydesc{##1} &
    ##2\tabularnewline
  }
}

\end{Preamble*}
%
Der Rest des Stils ist recht schnell erläutert. Zunächst wird auf Tabellenköpfe 
sowie Gruppierungen (Abstände und Überschriften) verzichtet. Schließlich ist 
der Befehl \Macro{glossentry} verantwortlich für die Formatierung der Einträge 
im Abkürzungsverzeichnis. Dieser wird intern durch \Package{glossaries} mit 
zwei Argumenten aufgerufen. Das erste enthält das entsprechende Label, das 
zweite ein kommaseparierte Liste der Seitenzahlen. Dabei stehen verschiedene 
Makros zur Auswahl, um anhand eines Labels die gewünschte Information zu 
extrahieren.%
\footnote{%
  bspw. mit \Macro{glossentryname} die Bezeichnung oder mit 
  \Macro{glsentrydesc} die dazugehörige Beschreibung%
}
Der Befehl \Macro{glossentry} so definiert, dass für jeden Eintrag eine Zeile 
in der Tabellen erzeugt wird, wo in der ersten Spalte die Abkürzung selbst, in 
der zweiten die Langform und in der dritten Spalte schließlich die Liste der 
Seiten, auf welchen die jeweilige Abkürzung mit \Macro{gls}\Parameter{Label} 
verwendet wurde, ausgegeben wird. Zum Abschluss die resultierende Ausgabe des 
neuen Stils.
%
\begin{Hint}
\printacronyms[style=acrotabularx]
\end{Hint}
\begin{quoting}[rightmargin=0pt]
\vspace*{-\baselineskip}
\InputCode
\end{quoting}
%
Sollte der Platz für die Erläuterungen wie in diesem Beispiel in der mittleren 
Spalte mehr als ausreichend sein und kein Zeilenumbruch benötigt werden, kann 
auch einfach eine \Environment{tabular}"=Umgebung mit einer \PValue{l}"~Spalte 
anstelle von \PValue{X} verwendet werden.

\minisec{Eigener Stil mit \Environment{tabu}}
Das Paket \Package{tabu} stellt mit \Environment{tabu} eine sehr komfortable 
und gut zu nutzende Umgebung bereit. Es wird in \autoref{sec:tabu} vorgestellt, 
wobei die dort gemachten Anmerkungen \emph{unbedingt} zu beachten sind. 

Ähnlich zu \Environment{tabularx} bietet auch die \Environment{tabu}-Umgebung 
einen \PValue{X}"~Spaltentyp. Allerdings gibt es hier eine Besonderheit. Für 
\Environment{tabularx}"=Tabellen muss generell eine feste Tabellenbreite 
angegeben werden. Die Breite der \PValue{X}"~Spalten wird anhand der 
angegebenen, gewünschten Gesamtbreite und dem bereits für andere Spalten vom 
Typ~\PValue{l},~\PValue{r}~und~\PValue{c} benötigten Platz berechnet. 

Für \Environment{tabu}"=Tabellen kann anstelle einer fest vorgegebenen Breite 
auch \PValue{spread~0pt} angegeben werden. Dadurch werden \PValue{X}"~Spalten 
anfänglich in ihrer natürlichen Breite gesetzt. Sobald jedoch die Gesamtbreite 
der Tabelle den zur Verfügung stehenden Platz bis zum Zeilenende überschreiten 
würde, werden die \PValue{X}"~Spalten automatisch umbrochen. Geschieht dies 
tatsächlich, gibt es beim Paket \Package{tabu} jedoch ein kleineres Problem. 
In umbrochenen Spalten setzt \Package{tabu} zu wenig vertikalen Freiraum am 
unteren Ende. Um dies zu beheben wird am Schluss jeder \PValue{X}"~Spalte mit
\PValue{X<\{\Macro{strut}\}} einfach der Befehl \Macro{strut} angehängt, der 
vertikalen Freiraum ober- und unterhalb der aktuellen Grundlinie einfügt. In 
\autoref{sec:tables} wird ein Ansatz aufgezeigt, wie man dies automatisiert 
ausmerzen kann. Der Rest des Stils ist identisch zu \PValue{acrotabularx}.
%
\CodeHook{\let\newglossarystyle\renewglossarystyle}
\begin{Preamble*}
\newglossarystyle{acrotabu}{%
  \renewenvironment{theglossary}{%
    \begin{tabu}spread 0pt{@{}lX<{\strut}l@{}}%
  }{%
    \end{tabu}\par\bigskip%
  }%
  \renewcommand*{\glossaryheader}{}%
  \renewcommand*{\glsgroupheading}[1]{}%
  \renewcommand*{\glsgroupskip}{}%
  \renewcommand*{\glossentry}[2]{%
    \glsentryitem{##1}% Entry number if required
    \glstarget{##1}{\sffamily\bfseries\glossentryname{##1}} &
    \glsentrydesc{##1} &
    ##2\tabularnewline
  }
}

\end{Preamble*}
%
\begin{Hint}
\printacronyms[style=acrotabu]
\end{Hint}
\begin{quoting}[rightmargin=0pt]
\vspace*{-\baselineskip}
\glsdisablehyper
\InputCode
\end{quoting}

\subsubsection{Symbolverzeichnis}
Für das Erzeugen eines Symbolverzeichnisses kann ebenfalls \Package{glossaries} 
verwendet werden. Allerdings muss dazu ein wenig mehr Aufwand getrieben werden, 
da das Paket hierfür keine dedizierte Schnittstelle bereitstellt. Wurde die 
Paketoption \Option{symbols} angegeben wird jedoch zumindest das notwendige 
Glossar erstellt.

Als erstes sollte ein gut nutzbarer Befehl zum Definieren eines neuen Symbols 
erstellt werden. In Anlehnung an den Befehl für Abkürzungen wird dieser 
\Macro{newsymbol} genannt. Dieser hat ein optionales und vier obligatorische 
Argumente, wobei das optionale Argument prinzipiell alle Schlüssel-Wert-Paare 
enthalten kann, die durch \Package{glossaries} akzeptiert werden. Welche davon 
letztlich auch Auswirkungen haben, hängt allerdings von der Gestaltung des 
Stils durch den Anwender ab. Der nachfolgend definierte Befehl hat folgende 
Gestalt:
%
\begin{quoting}[leftmargin=\parindent]
\Macro{newsymbol}\LParameter\Parameter{Label}\Parameter{Bezeichnung}%
\Parameter{Symbol}\Parameter{Einheit}
\end{quoting}
%
Mit \PName{Label} erfolgt die eindeutige Kennzeichnung des Symbols. Außerdem 
wird dies für die Sortierung verwendet, was unter Umständen etwas problematisch 
sein könnte. Eine manuelle Festlegung für den dazugehörigen Schlüssel durch den 
Anwender über das optionale Argument mit \PValue{sort=}\PName{Bezeichnung} ist 
eventuell sinnvoll. Nach dem \PName{Label} folgt die \PName{Bezeichnung} für 
das Formelzeichen beziehungsweise das \PName{Symbol} sowie die dazugehörige 
physikalische \PName{Einheit}.
%
\CodeHook{\let\newcommand\renewcommand}
\begin{Preamble*}
\newcommand*{\newsymbol}[5][]{%
  \newglossaryentry{#2}{%
    type=symbols,%
    description={},%
    name={#3},%
    symbol={\ensuremath{#4}},%
    user1={\ensuremath{\mathrm{#5}}},%
    sort={#2},%
    #1%
  }%
}

\end{Preamble*}
%
Da es sich zumeist um mathematische Symbole handelt, wird für das Symbol und 
die Einheit mit \Macro{ensuremath} sorge getragen, dass diese auch im 
Textmodus ohne Probleme verwendet werden können. Für ein kleines Beispiel 
werden fünf Formelzeichen definiert. Dabei wird der Befehl \Macro{nicefrac} aus 
dem Paket \Package{units} verwendet, welches in \autoref{sec:units} beschrieben 
ist.
%
\begin{Trunk*}
\newsymbol{l}{Länge}{l}{m}
\newsymbol{m}{Masse}{m}{kg}
\newsymbol{a}{Beschleunigung}{a}{\nicefrac{m}{s^2}}
\newsymbol{t}{Zeit}{t}{s}
\newsymbol{f}{Frequenz}{f}{s^{-1}}
\newsymbol{F}{Kraft}{F}{m \cdot kg \cdot s^{-2}=\nicefrac{J}{m}}

\end{Trunk*}
%
\dots und diese in einer kurzen Textpassage mit \Macro{gls}\Parameter{Label} 
verwendet.
%
\begin{Hint*}
Die Einheiten für die \gls{f} sowie die \gls{F} werden aus den 
SI"=Einheiten der Basisgrößen \gls{l}, \gls{m} und \gls{t} abgeleitet.
Und dann gibt es noch die Grundgleichung der Mechanik, welche für den
Fall einer konstanten Kraftwirkung in die Bewegungsrichtung einer
Punktmasse lautet:
$\gls{F} = \gls{m} \cdot \gls{a}$
\end{Hint*}
%
Damit wären zumindest die Symbole definiert. Allerdings ist noch nicht 
festgelegt, wie genau die Ausgabe des Symbolverzeichnisses aussehen soll. 
Momentan erzeugt der Befehl \Macro{printsymbols} jedenfalls kein sinnvolles 
Verzeichnis:
%
\begin{Hint}
\printsymbols
\end{Hint}
\begin{quoting}[rightmargin=0pt]
\vspace*{-\baselineskip}
\glsdisablehyper
\InputCode
\end{quoting}
%
Für dieses muss erst ein Stil definiert werden, was nachfolgend ähnlich zum 
Stil \PValue{acrotabularx} respektive \PValue{acrotabu} geschieht. Allerdings 
wird hier eine Variante gezeigt, mit der die Tabellen einen Seitenumbruch 
zulassen.

\minisec{Eigener Stil mit \Environment{tabularx} und \Environment{longtable}}
Soweit mir bekannt ist, lassen sich umbruchfähige Tabellen nicht direkt mit 
\Environment{tabularx} setzen. Vielmehr ist das Paket \Package{ltxtable} 
notwendig. Dieses wiederum verlangt, dass die zu setzende Tabelle in einer 
separaten Datei abgelegt wird. Möchte man diese dennoch im Hauptdokument 
belassen, kann man die \Environment{filecontents}"=Umgebung nutzen.

Leider ist mir kein Weg bekannt, diesen Prozess ohne wahnsinnig großen Aufwand 
für das Erstellen eines Glossars zu portieren. Für umbruchfähige Tabellen mit 
automatisch berechneten Spaltenbreiten kommt meines Wissens nach momentan nur 
das Paket \Package{tabu} infrage.

\minisec{Eigener Stil mit \Environment{tabu}}
Das Paket \Package{tabu} definiert die \Environment{longtabu}"=Umgebung, die 
wiederum auf \Environment{longtable} basiert und nachfolgend verwendet wird. 
Damit diese linksbündig gesetzt wird, muss vor dem obligatorischen Argument mit 
den Spaltendefinitionen noch das optionale Argument \OParameter{l} angegeben 
werden.
%
\CodeHook{\let\newglossarystyle\renewglossarystyle}
\begin{Preamble*}
\newglossarystyle{symblongtabu}{%
  \renewenvironment{theglossary}{%
    \begin{longtabu}spread 0pt[l]{ccX<{\strut}l}%
  }{%
    \end{longtabu}%
  }%
  \renewcommand*{\glossaryheader}{%
    \toprule
    \bfseries Symbol & \bfseries Einheit &
    \bfseries Name & \bfseries Seite(n)
    \tabularnewline\midrule\endhead%
    \bottomrule\endfoot%
  }%
  \renewcommand*{\glsgroupheading}[1]{}%
  \renewcommand*{\glsgroupskip}{}%
  \renewcommand*{\glossentry}[2]{%
    \glsentryitem{##1}% Entry number if required
    \glstarget{##1}{\glossentrysymbol{##1}} &
    \glsentryuseri{##1} &
    \glossentryname{##1} &
    ##2\tabularnewline%
  }%
}

\end{Preamble*}
%
Innerhalb von \Macro{newglossarystyle} wird \Macro{glossaryheader} für einen
Tabellenkopf definiert, wie er auch für eine \Environment{longtable}"=Umgebung 
erscheinen würde. In hier vorgestellten Fall werden Kopf beziehungsweise Fuß 
mit \Macro{endhead} respektive \Macro{endfoot} terminiert. Diese werden beim 
einem möglichen Seitenumbruch zu Beginn und am Ende auf jeder Seite gesetzt.

Damit man die Symbole auch wirklich sinnvoll nutzen kann, sollte das 
Erscheinungsbild der Einträge mit \Macro{defglsentryfmt} angepasst werden.
%
\begin{Preamble*}
\defglsentryfmt[symbols]{%
  \ifmmode%
    \glssymbol{\glslabel}%
  \else%
    \glsgenentryfmt~\glsentrysymbol{\glslabel}%
  \fi%
}
\end{Preamble*}
\begin{Preamble+}
}

\end{Preamble+}
%
Diese Definition führt dazu, dass bei der Verwendung eines Symbols mit
\Macro{gls}\Parameter{Label} im Text diesem die Bezeichnung vorangestellt 
wird, im Mathematikmodus allerdings allein das Symbol verwendet wird. Das 
nachfolgende Beispiel macht dies deutlich.
%
\begin{Trunk*}
Die Einheiten für die \gls{f} sowie die \gls{F} werden aus den
SI"=Einheiten der Basisgrößen \gls{l}, \gls{m} und \gls{t} abgeleitet.
Und dann gibt es noch die Grundgleichung der Mechanik, welche für den
Fall einer konstanten Kraftwirkung in die Bewegungsrichtung einer
Punktmasse lautet:
$\gls{F} = \gls{m} \cdot \gls{a}$

\end{Trunk*}
%
Das Symbolverzeichnis kann sich nun durchaus sehen lassen.
%
\begin{Hint}
\printsymbols[style=symblongtabu]
\end{Hint}
\begin{quoting}[rightmargin=0pt]
\newglossarystyle{symbtabu}{%
  \setglossarystyle{symblongtabu}%
  \renewenvironment{theglossary}{%
    \begin{tabu}spread 0pt{ccX<{\strut}l}%
  }{%
    \bottomrule\end{tabu}%
  }%
  \renewcommand*{\glossaryheader}{%
    \toprule
    \bfseries Symbol & \bfseries Einheit &
    \bfseries Name & \bfseries Seite(n)
    \tabularnewline\midrule%
  }%
}
\printsymbols[style=symbtabu]
\end{quoting}


\subsection{Literaturverzeichnis}
\label{sec:biblatex}
Für das Erstellen eines Literaturverzeichnisses wurde in der Vergangenheit fast 
ausschließlich \hologo{BibTeX} verwendet. Lieder wird auch heute immer noch 
darauf verwiesen, obwohl es seit einigen Jahren das Paket \Package{biblatex} 
gibt, welches insbesondere für neue Dokumente definitiv den Vorzug erhalten 
sollte. Auch die Umstellung älterer \hologo{BibTeX}"=Datenbanken ist mit 
wenigen Handgriffen realisierbar. Für \Package{biblatex} existieren eine Menge 
unterschiedlicher, vordefinierter Zitierstile, welche sich im Vergleich zu 
\hologo{BibTeX} auch wesentlich leichter an die individuellen Bedürfnisse 
anpassen lassen.

Ein weiterer Vorteil ist die Unterstützung von Datenbanken, welche eine 
UTF"~8"~Kodierung nutzen, wenn \Application{biber} zur Sortierung der Einträge 
verwendet wird. Welcher Stil und welches Backend zur Sortierung genutzt werden 
soll, lässt sich durch das optionale Argument beim Laden des Paketes festlegen. 
Damit die Zitierstile das optimale Ergebnis erzielen wird das vorherige Laden 
von \Package{csquotes} sehr empfohlen.
%
\begin{Preamble}
\usepackage{csquotes}
\usepackage[backend=biber,style=alphabetic]{biblatex}
\end{Preamble}
%
Die Erstellung einer Literaturdatenbank kann entweder von Hand oder mithilfe 
einer externen Anwendung erfolgen. Für die letztgenannte Variante sind die 
Programme \Application{Citavi} respektive \Application{JabRef} empfehlenswert. 
Auf eine Einführung in diese Anwendungen wird jedoch verzichtet. 

Die \Environment{filecontents}"=Umgebung kann verwendet werden, um innerhalb 
eines \hologo{LaTeX}"=Dokumentes externe Dateien direkt beim Kompilieren zu
erstellen. Damit wird nachfolgend für dieses Tutorial eine Literaturdatenbank 
\PValue{treatise-temp.bib} mit drei Einträgen manuell erzeugt. Die Umgebung 
gehört zu den Bordmitteln von \hologo{LaTeXe}. Das Paket \Package{filecontents} 
erweitert die Umgebung dahingehend, dass \emph{bereits existierende Dateien} 
überschrieben werden. Hier ist folglich Vorsicht geboten. Der große Vorteil ist 
jedoch, dass die erweiterte \Environment{filecontents}"=Umgebung~-- im 
Gegensatz zur Standardversion~-- die Dateien in der gleichen Eingabekodierung 
erzeugt, wie das verwendete Dokument. Diese Funktionalität wird für dieses 
Tutorial benötigt, weshalb auf das Laden des Paketes \Package{filecontents} 
nicht verzichtet werden kann. 
%
\begin{Preamble}
\usepackage{filecontents}
\end{Preamble}
%
Es werden folgend drei Einträge für die Literaturdatenbank definiert. Jeder 
Eintrag einer \PValue{.bib}-Datei beginnt mit \PValue{@}\PName{Eintragstyp}. 
Direkt danach ist für jeden Eintrag ein \emph{eindeutiges} \PParameter{Label} 
festzulegen. Anschließend können für unterschiedliche Felder die dazugehörige 
Werte eingetragen werden. Die verwendbaren Eintragstypen und die für diesen 
benötigten und zusätzlich nutzbaren Felder sind in der Dokumentation von 
\Package{biblatex} zu finden.
%
\begin{Preamble*}
\begin{filecontents}{treatise-temp.bib}
@book{goossens94,
  author    = {Goossens, Michel and Mittelbach, Frank
               and Samarin, Alexander},
  title     = {The LaTeX Companion},
  date      = {1994},
  publisher = {Addison-Wesley},
  location  = {Reading, Massachusetts},
  language  = {english},
}
@book{knuth84,
  author    = {Knuth, Donald E.},
  title     = {The \TeX book},
  date      = {1984},
  maintitle = {Computers \& Typesetting},
  volume    = {A},
  publisher = {Addison-Wesley},
  location  = {Reading, Massachusetts},
  language  = {english},
}
@manual{hanisch14,
  author    = {Hanisch, Falk},
  title     = {Ein \LaTeX"=Bundle für Dokumente
               im neuen Corporate Design 
               der Technischen Universität Dresden},
  date      = {2014},
  subtitle  = {Benutzerhandbuch},
  location  = {Dresden},
  language  = {german},
}
\end{filecontents}
\end{Preamble*}
%
Nachdem die Literaturdatenbank erstellt wurde, muss diese auch noch eingebunden 
werden. Dies geschieht mit:
%
\begin{Preamble}
\addbibresource{treatise-temp.bib}

\end{Preamble}
%
Im einfachsten Fall werden die Einträge mit \Macro{cite}\Parameter{macro} im 
Dokument referenziert. Für die Referenzierung werden durch \Package{biblatex} 
weitere Befehle angeboten.
%
\begin{Trunk+}
\section{Zitieren und das Literaturverzeichnis}
Das Literaturverzeichnis wir auf Basis der nachfolgend verwendeten 
Zitate erstellt und ist auf Seite~\pageref{sec:bibliography} zu finden.
\end{Trunk+}
\begin{Trunk*}
In diesem Textabschnitt werden die zwei bekannten \LaTeX-Bücher
\cite{knuth84} und \cite{goossens94} sowie das Anwenderhandbuch
\cite{hanisch14} zitiert.

\end{Trunk*}
%
Das Literaturverzeichnis wird mit \Macro{printbibliography} ausgegeben, wobei 
nicht alle Einträge der Literaturdatenbank sondern lediglich die tatsächlich 
referenzierten verwendet werden.
%
\begin{Hint}
\printbibliography
\end{Hint}
\begin{quoting}[rightmargin=0pt]
\makeatletter
\let\markboth\@gobbletwo
\let\markright\@gobble
\makeatother
\vspace*{-\baselineskip}
\InputCode
\end{quoting}



\section{Anfangszitat oder Schlauer Spruch}
\renewcommand*\dictumwidth{.4\textwidth}%
\dictum[Johann Wolfgang von Goethe]{Es irrt der Mensch, solang er strebt.}%
\bigskip\noindent
Oftmals möchte der Autor einer wissenschaftlichen Arbeit für das erste oder 
auch jedes Kapitel ein Zitat oder ähnliches voranstellen. Dies kann mit dem 
Befehl \Macro{dictum}\OParameter{Autor}\Parameter{Text} erfolgen. Damit wird 
der im obligatorischen Argument angegeben Ausspruch in einer \Macro{parbox} 
ausgegeben. Das optionale Argument kann für die Angabe des Autors verwendet 
werden. Soll das Ganze für einen Teil oder ein Kapitel erfolgen, sollte der 
Befehl \Macro{dictum} innerhalb von \Macro{setpartpreamble} beziehungsweise 
\Macro{setchapterpreamble} verwendet werden. Genaueres hierzu und zu den 
Möglichkeiten, die Gestalt des Zitats zu beeinflussen, ist in der \scrguide zu 
finden. Es folgt ein Beispiel für die Verwendung im Dokument.
%
\begin{Hint}
\setchapterpreamble{%
  \dictum[Johann Wolfgang von Goethe]{%
    Es irrt der Mensch, solang er strebt.%
  }%
  \bigskip
}
\chapter{Einleitung}

\end{Hint}



\section{Listen}
\label{sec:lists}
Für Auflistungen aller Art bietet \hologo{LaTeX} die drei Standardumgebungen 
\Environment{iteimze}, \Environment{enumerate} und \Environment{description}. 
Das häufigste Anliegen bei der Verwendung dieser Umgebungen ist das Reduzieren 
der Abstände zwischen den einzelnen, mit \Macro{item} gesetzten Punkten, welche 
sehr häufig als zu groß empfunden werden. Eine Aufzählung erscheint ohne 
weitere Anpassungen normalerweise so:
%
\CodePreamble{%
  Aufzählung mit einer \Environment{itemize}-Umgebung und Standardabständen:%
}
\begin{Hint*}
\begin{itemize}
\item erster Punkt
\item zweiter Punkt
\item dritter Punkt
\end{itemize}
\end{Hint*}
%
Dabei wirken die Abstände zwischen den einzelnen Punkten tatsächlich etwas 
überdimensioniert. Mit dem Paket \Package{enumitem} können die Umgebungen für 
Auflistungen einfach an die individuellen Bedürfnisse angepasst werden. Um 
diese ungewollten Abstände zwischen den Listenpunkten zu entfernen, kann der 
Schlüssel \Option{noitemsep} verwendet werden. Mit dem Befehl \Macro{setlist} 
werden~-- ohne zusätzliche Angaben im optionalen Argument~-- alle Umgebungen 
für Aufzählungen global geändert:
%
\begin{Preamble}
\usepackage{enumitem}
\end{Preamble}
\begin{Preamble*}
\setlist{noitemsep}

\end{Preamble*}
%
Das Ergebnis stellt sich folgendermaßen dar:
%
\begin{Trunk+}
\section{Eine Aufzählung mit angepassten vertikalen Abständen}
Ein kurzes Beispiel, wie eine Liste aussehen kann, nachdem die 
Möglichkeiten des Paketes \texttt{enumitem} genutzt und mit dem 
Befehl \texttt{\textbackslash setlist\{noitemsep\}} die vertikalen 
Abstände angepasst wurden.
\end{Trunk+}
\CodePreamble{%
  Aufzählung mit einer \Environment{itemize}-Umgebung ohne vertikale Abstände:%
}
\begin{Trunk*}
\begin{itemize}
\item erster Punkt
\item zweiter Punkt
\item dritter Punkt
\end{itemize}

\end{Trunk*}
%
Natürlich erlaubt das Paket noch weitergehende Einstellungen für die Umgebungen 
\Environment{itemize}, \Environment{enumerate} und \Environment{description}. 
Zusätzlich können diese für eigene Aufzählungsumgebungen geklont und den 
eigenen Bedürfnissen angepasst werden. Genaueres hierzu ist der Dokumentation 
des Paketes \Package{enumitem} entnehmen.



\section{Gleitumgebungen}
\label{sec:floats}
Die Positionierung von Abbildungen mit \hologo{LaTeX} kann zu Beginn für viele 
Anfänger durchaus frustrierend sein. Das liegt häufig am Missverständnis der 
beiden Standard"=Gleitobjektumgebungen für Tabellen (\Environment{table}) und 
Abbildungen (\Environment{figure} ). Diese sind in erster Linie zur Ergänzung 
des Fließtextes gedacht und sollten für das prinzipielle Verständnis des 
Geschrieben nicht notwendig sein. Das oft geforderte Verhalten, ein Gleitobjekt 
an einer ganz bestimmten und explizit festgelegten Position im Text zu setzen, 
ist nicht erforderlich, insbesondere weil dadurch der Lesefluss unnötig 
unterbrochen wird. Vielmehr ist es sinnvoll, Gleitobjekte entweder am Anfang 
oder Ende einer Seite zu platzieren, wo sie deutlich weniger stören. Allerdings 
sollte auf jedes Gleitobjekt im Fließtext über eine Referenz~-- beispielsweise 
mit dem Befehl \Macro{autoref} aus dem Paket \Package{hyperref}~-- Bezug 
genommen und gegebenenfalls eine kurze Erläuterung gegeben werden. Mehr dazu 
ist in \autoref{sec:references} zu finden.

Ein weitere Grund, \hologo{LaTeX} die Platzierung von Tabellen und Abbildungen 
vollständig zu überlassen, ist die Ungewissheit über den vorhandenen Platz auf 
der momentan erzeugten Seite. In nicht wenigen Fällen kann es passieren, dass 
das einzufügende Objekt zu groß für die aktuelle Seite ist, was einen wahrlich 
schlechten Seitenumbruch mit einer schlecht gefüllten Seite zur Folge hätte.
Die Verwendung einer Gleitumgebung für eine Abbildung wird im nachfolgenden 
Quelltextauszug exemplarisch gezeigt, das Ergebnis ist in \autoref{fig:example} 
zu sehen. Worauf man bei Gleitobjekten in jedem Fall achten sollte, dass die 
Verwendung von \Macro{label} immer erst \emph{nach} \Macro{caption} erfolgt, da 
der erzeugte Anker sich sonst nicht auf das Objekt bezieht.
%
\begin{Trunk+}
\section{Grafiken und Tabellen in Gleitumgebungen}
Es folgt die Demonstration von Gleitumgebungen, welche sowohl für 
Grafiken als auch Tabellen verwendet werden sollten. Im vorliegenden 
Beispiel kann unter Umständen der Eindruck entstehen, dass diese Seite 
etwas zu überladen mit Gleitobjekten ist. Dies liegt nicht an der 
Verwendung der Gleitobjekte sondern vielmehr am zu geringen Textvolumen 
und den eingeschränkten Möglichkeiten von \LaTeX{}, diese an geigneten 
Stellen zu platzieren. 

\subsection{Das Einbinden von Grafiken}
In \autoref{fig:example} wird dargestellt, wie eine Grafik im PDF-Format 
in ein Dokument eingebunden und auf diese verwiesen werden kann. 
Entscheidend dabei ist, den Anker für den Querverweis erst \emph{nach} 
der Beschriftung zu setzen.

\end{Trunk+}
\begin{Trunk}
\begin{figure}
\centering
\includegraphics{TUD-black}
\caption{Beispielgrafik}\label{fig:example}
\end{figure}

\end{Trunk}
\InputCode
%
In der Konsequenz ist dies hier ein Plädoyer, bei der Platzierung von Tabellen 
und Abbildungen vollständig auf \hologo{LaTeX} zu vertrauen. Anfangs kann der 
Anwender dies als Kontrollverlust empfinden. Sobald Sie jedoch Änderungen am 
Dokument vornehmen und eventuell in einem Kapitel einen Absatz ergänzen oder 
gar einen ganzen Abschnitt hinzufügen werden sie dankbar sein, sich in der 
Folge nicht mit der Neupositionierung sämtlicher Objekte herumschlagen zu 
müssen. Leidgeprüfte Anwender einschlägiger Textverarbeitungsprogramme können 
gewiss ein Lied darüber singen.



\subsection{Beeinflussung des Gleitprozesses}
Sobald das Dokument inhaltlich den finalen Zustand erreicht hat, haben Sie sich 
zu diesem Zeitpunkt schon mit Sicherheit an das standardmäßige Vorgehen von 
\hologo{LaTeX} gewöhnt und müssen respektive wollen gegebenenfalls nur noch bei 
wenigen Gleitobjekten~-- über das optionale Argument der Gleitumgebungen zur
individuellen Empfehlung für die Platzierung~-- nachjustieren. Mögliche Werte 
sind:
%
\begin{description}[labelindent=\parindent,leftmargin=*,style=nextline]
\item[\POParameter{h} (here)]
  An der Stelle, wo es im Quelltext angegeben wurde~-- falls genügend Platz 
  vorhanden ist
\item[\POParameter{t} (top)]
  Am oberen Ende der aktuellen oder der folgenden Seite
\item[\POParameter{b} (bottom)]
  Am unteren Ende der aktuellen Seite
\item[\POParameter{p} (page)]
  Auf einer separaten Seite für mindestens ein Gleitobjekt
\end{description}
%
Sie können eine, mehrere oder alle Optionen angeben, wobei die Reihenfolge 
keine Rolle spielt. Der Algorithmus arbeitet alle ihm zur Verfügung gestellten 
Optionen immer in der zuvor aufgezählten Reihenfolge ab, wobei diese nur als 
Empfehlung und nicht als Verpflichtung angesehen werden. Es gibt folglich keine 
Garantie, dass Ihr Vorschlag akzeptiert wird. Fügen sie dem optionalen Argument 
ein \PValue{!} an, so verhindern Sie, dass \hologo{LaTeX} weitere Optionen 
evaluiert. Das Gleitobjekt wird anhand der optionalen Parameter positioniert, 
selbst wenn dabei ein unschönes Seitenlayout entsteht. Kann das Gleitobjekt 
technisch unmöglich auf die angegebene Weise positioniert werden, wird dieses 
sowie alle folgenden aufgeschoben und am Ende des Abschnitts oder Kapitels 
angehängt. Dies ist ein Effekt, den man in den seltensten Fällen will. 

Außerdem können die Pakete \Package{flafter} sowie \Package{placeins} genutzt 
werden. Das erstgenannte verhindert das Auftreten von Gleitobjekten im Dokument 
vor ihrer Definition im Quelltext. In der Konsequent bedeutet dies, dass die 
Option \POParameter{t} Gleitobjekte nur am oberen Ende der nächsten Seite 
jedoch nicht auf der aktuellen zulässt. Mit dem zweiten Paket können Barrieren 
definiert werden, an welchen die Ausgabe aller noch in der Warteschlange 
befindlichen Gleitobjekte forciert wird~-- beispielsweise vor bestimmten 
Gliederungsüberschriften. Im Handbuch zu \TUDScript findet sich außerdem zum 
Thema \enquote{\manualhyperref{sec:tips:floats}{Platzierung von Gleitobjekten}} 
ein eigener Abschnitt mit weiterführenden Informationen.


\subsection{Abstellen des Gleitprozesses}
Oftmals wird verlangt, den Gleitprozess vollständig abzustellen. Dies bringt in 
den meisten Fällen einige Probleme mit sich. Die Folge sind sehr ein unruhiges 
Erscheinungsbild das Satzspiegels aufgrund schlecht gefüllter Seiten, außerdem 
viel Handarbeit mit hart kodierten Seitenumbrüchen sowie ein gegenüber von 
Umbruchänderungen äußerst anfälliges Dokument. Sollte es in \emph{Einzelfällen} 
dennoch erforderlich sein, dass eine Abbildung oder eine Tabelle nicht gleitet, 
ist die einzig logische Konsequenz, auf die Verwendung einer Gleitumgebung 
gänzlich zu verzichten. 

Hierfür eignen sich Umgebungen wie \Environment{center}, um etwas Abstand vor 
und nach dem Objekt zu erzeugen und \Environment{minipage}, um Grafik oder 
Tabelle und Beschriftung zusammenzuhalten. Die \KOMAScript-Klassen stellen für 
diese Anwendung das Makro \Macro{captionof} zur Verfügung. Damit lassen sich 
Bildunter- sowie Tabellenüberschriften auch ohne die gleitenden Umgebungen 
\Environment{figure} beziehungsweise \Environment{table} erzeugen. Alternativ 
kann das Paket \Package{float} verwendet werden, welches für Gleitumgebungen 
den Platzierungsparameter \POParameter{H} zur Verfügung stellt. Allerdings ist 
dies meiner Meinung nach alles andere als konsequent und sinnvoll.


\subsection{Gleitobjektlayout}
\label{sec:floatlayout}
\ToDo{floatrow}


\subsection{Untergleitobjekte}
\label{sec:subfloats}
\ToDo{subfloats}



\section{Tabellen}
\label{sec:tables}
Zum Thema \enquote{Tabellensatz mit \hologo{LaTeX}} sind bereits zahlreiche 
\hrfn{http://userpage.fu-berlin.de/latex/Materialien/tabsatz.pdf}{Leitfäden} im 
Internet zu finden. Deshalb werde ich meine Ausführungen zu diesem Punkt 
relativ kurz halten. Zwei Regeln sollten beim Satz von Tabellen in jedem Fall 
beachtet werden:
%
\begin{enumerate}[itemindent=0pt,labelwidth=*,labelsep=1em,label=\Roman*.]
\item keine vertikalen Linien
\item keine doppelten Linien
\end{enumerate}
%
Das Paket \Package{booktabs} ist für den Satz von hochwertigen Tabellen eine 
große Hilfe und stellt die Befehle \Macro{toprule}, \Macro{midrule} sowie
\Macro{cmidrule} und \Macro{bottomrule} für unterschiedliche horizontale Linien 
bereit.
%
\begin{Preamble}
\usepackage{booktabs}
\end{Preamble}
%
Außerdem existiert das Paket \Package{array}, welches mit dem Befehl 
\Macro{newcolumntype} die Definition eigener Spaltentypen sowie die Verwendung 
sogenannter \enquote{Hooks} vor und nach Einträgen innerhalb einer Spalte 
(\PValue{>\{\dots\}}\PName{Spaltentyp}\PValue{<\{\dots\}}) ermöglicht.
%
\begin{Preamble}
\usepackage{array}
\end{Preamble}
%
Für alle im Folgenden vorgestellten Umgebungen zum Setzen von Tabellen gilt, 
dass die Inhalte zeilenweise angegeben werden, wobei die Einträge für die 
einzelnen Spalten mit \PValue{\&} voneinander zu trennen sind. Das Beenden 
einer Tabellenzeile und der Wechsel zur nächsten erfolgt normalerweise mit 
\Macro{\bsc}. Da jedoch einige \hologo{LaTeX}-Pakete diesen Befehl innerhalb 
von Tabellen lokal ändern ist es zur Vermeidung unnötiger Fehler wesentlich 
sicherer, das Zeilenende ausschließlich mit \Macro{tabularnewline} zu setzen.


\subsection{Die Standardumgebung \Environment{tabular}}
Normalerweise gibt es vier unterschiedliche Spaltentypen. Die Spaltentypen 
\PValue{l}, \PValue{c} und \PValue{r} stehen für linksbündige, zentrierte und 
rechtsbündige Spalten, welche allerdings keinen Zeilenumbruch erlauben. Der 
Inhalt wird quasi wie in einer \Macro {mbox} gesetzt, wodurch die Spalten sehr 
breit werden und über den Seitenrand hinausragen können.

Der Spaltentyp~\PValue{p}\Parameter{Breite} hingegen legt die Spaltenbreite 
fest und setzt den Inhalt in jeder Zeile in eine \Macro{parbox}, wobei diese 
wird mit ihrer obersten Zeile an der Grundlinie ausgerichtet wird. Das Paket
\Package{array} stellt außerdem die Spaltentypen~\PValue{m}\Parameter{Breite} 
und ~\PValue{b}\Parameter{Breite} bereit, welche ebenfalls mit \Macro{parbox} 
gesetzt werden, die Ausrichtung an der Grundlinie jedoch zentriert respektive 
an der unteren Zeile der Box erfolgt. Es folgt ein Beispiel zur Verwendung der 
\Environment{tabular}-Umgebung.
%
\begingroup
\pcolumnfuzz{70pt}
\begin{Hint*}
\begin{tabular}{lcrp{33mm}}
\toprule
\textbf{Linksbündig} & \textbf{Zentriert} & 
\textbf{Rechtsbündig} & \textbf{Blocksatz} \tabularnewline
\midrule
a   & b   & c   & Dieser Text wird im Blocksatz gesetzt\tabularnewline
aa  & bb  & cc  & Auch Zeilenumbrüche sind vorhanden\tabularnewline
aaa & bbb & ccc & Worttrennungsmusterkontrolle\tabularnewline
\bottomrule
\end{tabular}
\end{Hint*}
\endgroup
%
Die Tabellenbreite ergibt sich aus der Breite der einzelnen Spalten. Bei dieser 
Umgebung liegt es allein beim Anwender, auf die korrekte Breite der Tabelle zu 
achten, damit diese nicht über die Seitenränder hinausragt. Das kann auf Dauer 
recht aufwändig werden. Das Festlegen der Gesamtbreite einer Tabelle durch den 
Anwender und das automatische Berechnen einiger oder aller Spaltenbreiten ist 
sicher die angenehmere Variante. Wie sich dies sehr komfortabel bewerkstelligen 
lässt wird~-- ebenso wie die Lösung des Problems des nicht umbrochenen Eintrags 
in der letzten Spalte der dritten Zeile der obigen Tabelle~-- im nachfolgenden
Abschnitt demonstriert.


\subsection{Tabellen mit variabler Spaltenbreite}
Hierfür stehen die Pakete \Package{tabularx} oder \Package{tabu} zur Verfügung, 
welche nun kurz vorgestellt werden sollen. Das Paket \Package{tabu} ist relativ 
neu und versucht, viele Funktionalitäten ganz unterschiedlicher Pakete~-- 
beispielsweise von \Package{tabularx}~-- für den Tabellensatz in sich zu 
vereinen. 
%
\begin{Preamble}
\usepackage{tabu}
\usepackage{tabularx}

\end{Preamble}

\subsubsection{Die Tabellenumgebung \Environment{tabularx}}
\label{sec:tabularx}
Das Paket \Package{tabularx} stellt den Spaltentyp~\PValue{X} bereit, welcher 
prinzipiell dem Spaltentyp~\PValue{p} entspricht. Auch für diesen wird eine 
\Macro{parbox} verwendet, allerdings wird deren Breite \emph{automatisch} 
berechnet. Die Umgebung \Environment{tabularx} erwartet vor der Angabe der 
Spaltentypen als obligatorisches Argument die gewünschte Breite der Tabelle. Zu 
beachten ist, dass Spalten vom Typ~\PValue{l},~\PValue{c}~und~\PValue{r} 
weiterhin ohne Zeilenumbruch gesetzt werden. Nur für Spalten vom Typ~\PValue{X} 
und deren Derivate wird aus dem verbliebenen Platz die Breite berechnet. Für 
die \Environment{tabularx}-Umgebung wird das vorherige Beispiel wiederholt.
%
\begingroup
\pcolumnfuzz{70pt}
\begin{Hint*}
\begin{tabularx}{11.7cm}{lcrX}
\toprule
\textbf{Linksbündig} & \textbf{Zentriert} & 
\textbf{Rechtsbündig} & \textbf{Blocksatz} \tabularnewline
\midrule
a    & b    & c    & Dieser Text wird im Blocksatz gesetzt\tabularnewline
aa   & bb   & cc   & Auch Zeilenumbrüche sind vorhanden\tabularnewline
aaa  & bbb  & ccc  & Worttrennungsmusterkontrolle\tabularnewline
aaaa & bbbb & cccc & \hspace{0pt}Worttrennungsmusterkontrolle
\tabularnewline
\bottomrule
\end{tabularx}
\end{Hint*}
\endgroup
%
Die Breite der letzten Spalte wurde dabei aus der Angabe der Gesamtbreite mit 
\PValue{11.7cm} berechnet. Des Weiteren ist zu sehen, wie das Problem des 
Zeilenumbruchs behandelt werden kann. Normalerweise wird das erste Wort in 
einem Absatz von \hologo{LaTeX} \emph{nie} umbrochen. Das kann mit dem Einfügen 
eines horizontalen Abstandes mit dem Wert \PValue{0pt} umgangen werden. Die 
gefundene Lösung ist allerdings alles andere als elegant.

Mit den Möglichkeiten des Paketes \Package{array} ist das Problem relativ 
schnell gelöst. Es wird mit \Macro{newcolumntype} ein neuer Spaltentyp 
definiert. Das erste Argument von \Macro{newcolumntype} legt den Namen des 
Spaltentyps fest. Mit \PValue{>\Parameter{Definitionen}}\Parameter{Typ} wird im 
zweiten Argument das Ausführen von \PName{Definitionen} vor dem Setzen des 
eigentlichen Inhaltes in einer \Parameter{Typ}"~Spalte definiert. Es wird ein 
neuer, auf der \PValue{X}"~Spalte basierender Typ~\PValue{Y} definiert, welcher 
zu Beginn der Spalte den Phantomabstand automatisch einfügt. Darauf basierend 
werden drei Spaltentypen für den links- und rechtsbündigen sowie zentrierten 
Textsatz mit einem möglichen Zeilenumbruch erstellt.
%
\begin{Trunk+}
\subsection{Tabellen als Gleitobjekte}
Auch Tabellen sollten in einer Gleitumgebung gesetzt werden. Nach diesem 
Absatz wird eine Tabelle mithilfe der \texttt{tabularx}-Umgebung 
erstellt. Zu sehen ist diese in \autoref{tab:tabularx}. Für diese werden 
zuvor neue Spaltentypen definiert.

\end{Trunk+}
\CodeHook{\renewcommand*{\newcolumntype}[2]{}}
\begin{Trunk*}
\newcolumntype{Y}{>{\hspace{0pt}}X}
\newcolumntype{L}{>{\raggedright}Y}
\newcolumntype{C}{>{\centering}Y}
\newcolumntype{R}{>{\raggedleft}Y}

\end{Trunk*}
%
Nachfolgend wird die Tabelle gesetzt mit den neuen Spaltentypen gesetzt. Achten 
Sie auf die Verwendung von \PValue{@\{\}} vor der ersten und nach der letzten 
Spalte. Nach dem Quelltextauszug wird dieses Konstrukt erläutert.
%
\begin{Trunk}
\begin{table}
\begin{tabularx}{\textwidth}{@{}LCRY@{}}
\toprule
\textbf{Linksbündig} & \textbf{Zentriert} & 
\textbf{Rechtsbündig} & \textbf{Blocksatz} \tabularnewline
\midrule
Ein linksbündiger Blindtext zur Demonstration einer L"~Spalte &
Ein zentrierter Blindtext zur Demonstration einer C"~Spalte &
Ein rechtsbündiger Blindtext zur Demonstration einer R"~Spalte &
Ein im Blocksatz gesetzter Blindtext zur Demonstration einer Y"~Spalte
\tabularnewline
\bottomrule
\end{tabularx}
\caption{Eine \texttt{tabularx}-Tabelle}\label{tab:tabularx}
\end{table}

\end{Trunk}
\InputCode
%
Normalerweise wird in einer Tabelle mit \Macro{hskip}\Macro{tabcolsep} vor 
\emph{und} nach jeder Spalte ein horizontaler Zwischenraum eingefügt.%
\footnote{Der Abstand zweier Spalten beträgt folglich 2\Macro{tabcolsep}.}
Mit \PValue{@\Parameter{Ausdruck}} kann dies verhindert und stattdessen  
\Parameter{Ausdruck} ausgeführt werden. Mit der Verwendung von \PValue{@\{\}} 
an den entsprechenden Stellen bei der Angabe der Spaltentypen wurde der vor der 
ersten und nach der letzten Tabellenspalte eingefügten Abstand unterdrückt.
Das Ergebnis ist in \autoref{tab:tabularx} zu sehen.


\subsubsection{Die Tabellenumgebung \Environment{tabu}}
\label{sec:tabu}
Das Paket \Package{tabu} bietet eine mächtige und komfortable Alternative zu 
\Package{tabularx}. Es kam in diesem Tutorial bereits für die Verzeichnisse von 
Abkürzungen und Symbolen in \autoref{sec:glossaries} zum Einsatz. Leider ist 
das Paket in er aktuellen Version~v2.8 mit Vorsicht zu genießen. Zum einen
wären für das Paket in der aktuellen Version seit geraumer Zeit ein paar 
kleinere Bugfixes notwendig, mehr dazu sehen Sie später. Außerdem wird sich die 
\hrfn{https://groups.google.com/d/topic/comp.text.tex/xRGJTC74uCI}{%
  Benutzerschnittstelle in einer zukünftigen Version
} sehr stark ändern. Sie sollten sich bewusst sein, dass mit der Version~v2.8
gesetzte Dokumente gegebenenfalls später angepasst werden müssen.

Nichtsdestotrotz soll hier die Verwendung der \Environment{tabu}-Umgebung 
gezeigt werden, insbesondere weil es für das Setzen umbruchfähiger Tabellen mit 
automatisch berechneten Spaltenbreiten momentan (fast) keine Alternative zu der 
Umgebung \Environment{longtabu} gibt. 

Dieses Paket definiert ebenso einen Spaltentyp~\PValue{X}, welchem allerdings 
zusätzlich ein optionales Argument angehängt werden kann. Mit diesem lässt sich 
sowohl die Gewichtung der automatisch berechneten Spalten untereinander als 
auch die Positionierung zur Grundlinie sowie die Ausrichtung des Inhaltes 
ändern.
%
\begingroup
\pcolumnfuzz{70pt}
\begin{Hint*}
\begin{tabu} to 11.7cm {lcrX}
\toprule
\textbf{Linksbündig} & \textbf{Zentriert} & 
\textbf{Rechtsbündig} & \textbf{Blocksatz} \tabularnewline
\midrule
a    & b    & c    & Dieser Text wird im Blocksatz gesetzt\tabularnewline
aa   & bb   & cc   & Auch Zeilenumbrüche sind vorhanden\tabularnewline
aaaa & bbbb & cccc & Worttrennungsmus\-terkontrolle\tabularnewline
aaa  & bbb  & ccc  & Worttrennungsmusterkontrolle\tabularnewline
\bottomrule
\end{tabu}
\end{Hint*}
\endgroup
%
Das Beispiel zeigt auch gleich das momentan meiner Meinung nach größte Problem 
des Paketes: Bei mehrzeiligen Zellen fehlt vertikale Zwischenraum zur nächsten 
Tabellenzeile. Nun hat man entweder die Möglichkeit, bei jeder Angabe einer 
Spalte im Tabellenkopf, welche eine \Macro{parbox} verwendet, mit dem Ausdruck 
\PValue{>\{\Macro{strut}\}} die fehlenden Unterlänge auszugleichen oder man 
definiert sich abermals einen entsprechenden Spaltentyp. Das ist für die 
\PValue{X}"~Spalten der Umgebung \Environment{tabu} etwas schwieriger, da 
\Macro{newcolumntype} normalerweise die Definition eines optionalen Argumentes 
nicht vorgesehen ist. Mit ein wenig Trickserei ist das dennoch möglich:
%
\begin{Trunk+}
Eine andere Variante zum Setzen einer Tabelle mit variabler Spaltenbreite 
und festgelegter Gesamtbreite ist in \autoref{tab:tabu} zu sehen. Diese 
würde in der Umgebung \texttt{tabu} gesetzt. Auch für diese wird zuerst 
ein neuer Spaltentyp definiert, welcher die Unzulänglichkeiten der 
Umgebung reduziert.

\end{Trunk+}
\CodeHook{\renewcommand*{\newcolumntype}[2]{}}
\begin{Trunk*}
\makeatletter
\newcolumntype{Z}{}
\renewcommand*\NC@rewrite@Z[1][]{%
  \NC@find>{\hspace{0pt}}X[#1]<{\@finalstrut\@arstrutbox}%
}
\makeatother

\end{Trunk*}
%
Anschließend kann der \PValue{Z}"~Spaltentyp äquivalent zu \PValue{X} genutzt 
werden. Zu beachten ist dabei insbesondere die Möglichkeit, die einzelnen 
Spalten in ihrer resultierenden Breite zueinander zu gewichten. Das Ergebnis 
ist in \autoref{tab:tabu} zu sehen.
%
\begin{Trunk}
\begin{table}
\caption{Eine \texttt{tabu}-Tabelle}\label{tab:tabu}
\begin{tabu} to \textwidth {@{}Z[2,l]Z[3,c]Z[2,r]Z[3,j]@{}}
\toprule
\textbf{Linksbündig} & \textbf{Zentriert} & 
\textbf{Rechtsbündig} & \textbf{Blocksatz} \tabularnewline
\midrule
Ein linksbündiger Blindtext zur Demonstration einer Z[l]"~Spalte &
Ein zentrierter Blindtext zur Demonstration einer Z[c]"~Spalte &
Ein rechtsbündiger Blindtext zur Demonstration einer Z[r]"~Spalte &
Ein im Blocksatz gesetzter Blindtext zur Demonstration einer Z"~Spalte
\tabularnewline
\bottomrule
\end{tabu}
\end{table}

\end{Trunk}
\InputCode



\section{Erstellen von Abbildungen}
\label{sec:figures}
Die einfachste Möglichkeit, in einer wissenschaftlichen Abhandlung Grafiken zu 
verwenden, ist sicherlich das Einbinden von externen Abbildungen mit dem Befehl 
\Macro{includegraphics}. Allerdings ist für eine wissenschaftliche Arbeit in 
höchster Qualität meiner Meinung nach das bloße Einfügen einer gescannten oder 
heruntergeladenen Grafik unzureichend. Vielmehr sollte diese zum einen als 
skalierbare Vektorgrafik vorliegen, um sowohl die höchste Druckqualität als 
auch die Wiederverwendbarkeit sicherzustellen, und zum anderen in ihrer 
Gestaltung dem verwendeten Layout des verfassten Dokumentes entsprechen. Dies 
betrifft sowohl die eingesetzten Farben und Schriften als auch die Stärke der 
verwendeten Linien.

Zum Zeichnen von Grafiken sind die freien Programme \Application{LaTeXDraw} und 
besonders \Application{Inkscape} empfehlenswert, da diese die Verwendung der 
Dokumentschriften innerhalb der erstellten Grafiken ermöglichen. Im Handbuch zu 
\TUDScript wird dies für \manualhyperref{sec:tips:svg}{\Application{Inkscape}} 
genauer erläutert. Eine andere Variante ist das \enquote{Programmieren} von 
Grafiken. Die zwei bekanntesten \hologo{LaTeX}-Pakete sind \Package{tikz}(pgf) 
und \Package{pstricks}(pstricks-base). Für diese beiden wird im Folgenden eine 
Abbildung exemplarisch erstellt, ohne dabei tiefer ins Detail gehen zu wollen. 
Ebenfalls wird darauf verzichtet, diese im Kopierbeispiel einzubinden. Für 
beide Pakete gibt es neben der umfangreichen Dokumentation zahlreiche Beispiele 
und Erläuterungen im Internet.


\subsection{Das Paket tikz}
Zur Verwendung von \Package{tikz}(pgf) ist nicht viel zu sagen. Einfach in der 
Dokumentpräambel einbinden und es kann losgehen. Für das nachfolgende Beispiel 
sind außerdem noch zwei weitere Programmbibliotheken für \Package{tikz}(pgf) 
notwendig.
%
\begin{Hint}
\usepackage{tikz}
\usetikzlibrary{chains}
\usetikzlibrary{decorations.markings}
\tikzset{on grid}
\end{Hint}
%
Ein kurze Anmerkungen zur Skalierung der Grafiken möchte ich mir erlauben. 
Normalerweise verwendet \Package{tikz}(pgf) Koordinaten im Zentimeter"=Raster. 
Ich persönlich bevorzuge der relativierte Skalierung auf die Textbreite des 
Dokumentes, wobei eine Einheit genau \unit[1]{\%} dieser entspricht. Das kann 
mit folgendem Quelltext erreicht werden:
%
\CodeHook{\renewcommand\newlength[1]{}}
\begin{Hint*}
\newlength{\tikzunit}
\setlength{\tikzunit}{.01\textwidth}
\tikzset{x=\tikzunit,y=\tikzunit}
\end{Hint*}
%
Nachfolgend wird eine Grafik innerhalb einer Gleitumgebung erstellt.
%
\begin{Hint}
\begin{figure}
\begin{tikzpicture}
  \tikzstyle{inner box}=[%
    text width=18\tikzunit,
    align=center,
    rectangle,
    inner sep=0pt,
    minimum height=8\tikzunit,
    font=\hspace{0pt},
    draw
  ]
  \tikzstyle{inner label}=[align=center, font=\scriptsize]
  \tikzstyle{inner box chain}=[every node/.style={on chain}]
  \tikzstyle{inner box chain below}=[%
    inner box chain, node distance=8\tikzunit,continue chain=going below
  ]
  \tikzstyle{inner box chain right}=[%
    inner box chain,node distance=35\tikzunit,continue chain=going right
  ]
  \tikzstyle{inner box chain above}=[%
    inner box chain,node distance=16\tikzunit,continue chain=going above
  ]
  \tikzstyle{pstarrow->}=[%
    decoration={markings,
      mark=at position 1 with {\arrow[xscale=1.5]{stealth}};
    },
    postaction={decorate},
    shorten >=0.7pt
  ]
  \newcommand\tikzparbox[2][9]{%
    \parbox{#1\tikzunit}{\centering\hspace{0pt}#2}%
  }
  \begin{scope}[start chain]
    \begin{scope}[inner box chain below]
      \node(NE)[inner box]{Navigations\-ebene};
      \node(NB)[inner label]{gewählte Fahrtroute\\zeitlicher Ablauf};
      \node(BE)[inner box]{{Bahnführungs\-ebene}};
      \node(BS)[inner label]{%
        gewählte Führungsgrößen:\\Sollspur, Sollgeschwindigkeit%
      };
      \node(SE)[inner box]{Stabilisierungs\-ebene};
    \end{scope}
    \begin{scope}[inner box chain right]
      \node(LQ)[inner box]{Längs- und Querdynamik};
      \node(FO)[inner box]{Fahrbahn\-oberfläche};
    \end{scope}
    \begin{scope}[inner box chain above]
      \node(FR)[inner box]{Fahrraum\\\smallskip{\scriptsize Straße 
      und\\\vspace{-1.5ex}Verkehrssituation}};
      \node(SN)[inner box]{Straßennetz};
    \end{scope}
  \end{scope}
  \begin{scope}[inner label,minimum size=0pt]
    \draw [pstarrow->] (FO) -| ++(13.5,-12) to
      node [above]{Istspur, Istgeschwindigkeit} ++(-97,0) |- (SE);
    \draw [pstarrow->] (FR) -| ++(14  ,-32) to 
      node [above]{Bereich sicherer Führungsgrößen} ++(-98,0) |- (BE);
    \draw [pstarrow->] (SN) -| ++(14.5  ,-52) to 
      node [above]{mögliche Fahrtroute} ++(-99,0) |- (NE);
  \end{scope}
  \begin{scope}[inner label]
    \draw              (NE) to (NB);
    \draw [pstarrow->] (NB) to (BE);
    \draw              (BE) to (BS);
    \draw [pstarrow->] (BS) to (SE);
    \draw [pstarrow->] (SE) to
      node[above] {\tikzparbox{Stell\-größen}}
      node[below] {\tikzparbox{Lenken Gasgeben Bremsen}}
    (LQ);
    \draw [pstarrow->] (LQ) to
      node[above]{\tikzparbox{Regel\-größen}}
      node[below]{\tikzparbox{Fahrzeugbewegung}}
    (FO);
    \draw [pstarrow->] (LQ)+(24,0) |- (FR);
    \draw [pstarrow->] (LQ)+(24,0) |- (SN);
  \end{scope}
  \begin{scope}[very thick,rounded corners=5\tikzunit]
    \draw (-12.5,-40) rectangle (12.5,14);
    \draw ( 22.5,-40) rectangle (47.5,-18);
    \draw ( 57.5,-40) rectangle (82.5,14);
  \end{scope}
  \begin{scope}[font=\bfseries]
    \node at (0,9) {Fahrer};
    \node at (35,-23) {Fahrzeug};
    \node at (70,9) {Umwelt};
  \end{scope}
\end{tikzpicture}
\caption{Eine mit TikZ erstellte Grafik}\label{fig:tikz}
\end{figure}
\end{Hint}
\InputCode


\subsection{Das Paket pstricks}
\begin{Hint}
\begin{figure}
\psset{%
  unit=.01\textwidth,%
  cornersize=absolute,%
  labelsep=.8ex,%
  linewidth=.4pt,%
  arrowscale=1.5,%
}
\begin{pspicture}(0,-2)(100,64)
\newcommand\fnodetext{}
\def\fnodetext(#1)#2#3{%
  \fnode[framesize=18 8](#1){#2}%
  \rput(#1){\parbox{17\psunit}{\centering\hspace{0pt}#3}}%
}
\newcommand\scriptbox[2][24]{%
  \parbox{#1\psunit}{\scriptsize\centering\hspace{0pt}#2}%
}
\rput(15,10){%
  \rput(0,49){\textbf{Fahrer}}
  \fnodetext(0,40){NE}{Navigations\-ebene}
  \fnodetext(0,24){BE}{Bahnführungs\-ebene}
  \fnodetext(0,08){SE}{Stabilisierungs\-ebene}
  \ncline{->}{NE}{BE}
  \ncput*{\scriptbox{gewählte Fahrtroute\\zeitlicher Ablauf}}
  \ncline{->}{BE}{SE}
  \ncput*{%
    \scriptbox{gewählte Führungsgrößen:\\Sollspur,~Sollgeschwindigkeit}%
  }
  \psframe[dimen=middle,linewidth=1.2pt,linearc=5](-12.5,0)(12.5,54)
}
\rput(50,10){%
  \rput(0,17){\textbf{Fahrzeug}}
  \fnodetext(0,8){FZ}{Längs- und\\Querdynamik}
  \psframe[dimen=middle,linewidth=1.2pt,linearc=5](-12.5,0)(12.5,22)
}
\rput(85,10){%
  \rput(0,49){\textbf{Umwelt}}
  \fnodetext(0,40){SN}{Straßennetz}
  \fnodetext(0,24){FR}{%
    Fahrraum\\\smallskip\scriptsize{Straße und Verkehrssituation}%
  }
  \fnodetext(0,08){FO}{Fahrbahn\-oberfläche}
  \psframe[dimen=middle,linewidth=1.2pt,linearc=5](-12.5,0)(12.5,54)
}

\ncline{->}{SE}{FZ}
\naput{\scriptbox[9]{Stell\-größen}}
\nbput{\scriptbox[9]{Lenken Gasgeben Bremsen}}
\ncline{->}{FZ}{FO}
\naput{\scriptbox[9]{Regel\-größen}}
\nbput{\scriptbox[9]{Fahrzeugbewegung}}

\psset{armA=15,armB=0,angleA=0,angleB=180}
\ncangles{->}{FZ}{FR}
\ncangles{->}{FZ}{SN}

\psset{angleA=180,angleB=0}
\ncloop[loopsize=12,arm=4.5]{<-}{SE}{FO}
\naput{\scriptbox{Istspur, Istgeschwindigkeit}}
\ncloop[loopsize=32,arm=5]{<-}{BE}{FR}
\naput{\scriptbox[30]{Bereich sicherer Führungsgrößen}}
\ncloop[loopsize=52,arm=5.5]{<-}{NE}{SN}
\naput{\scriptbox{mögliche Fahrtroute}}
\end{pspicture}
\caption{Eine mit pstricks erstellte Grafik}\label{fig:pstricks}
\end{figure}
\end{Hint}
\InputCode


\subsection{Die Verwendung von tikz und pstricks in einem Dokument}
\ToDo{hinweis auf ifpdf und auto-pst-pdf}

\subsection{Auslagern von Grafiken in separate Dateien}
\ToDo{standalone und subfiles}


\section{Mathematiksatz}
\ToDo{kurzer Abriss der Umgebungen?}


\section{Worttrennung}
\label{sec:hyphenation}
\ToDo{manuelle Angabe von Trennstellen global und lokal}
Die Option \Option{ngerman} führt dabei zur Verwendung der Trennmuster für die 
neue deutsche Rechtschreibung, wobei der Wert \PValue{ngerman-x-latest} die 
neuesten lädt. Für die alte Orthographie ist stattdessen die Option 
\Option{german} und der angepasste Wert zu verwenden.


\subsection{Worttrennungskorrektur für bestimmte Wörter}

\subsection{Worttrennung im Flattersatz}



\section{Typographie}
\ToDo{Erklärung}
\begin{Preamble}
\usepackage{microtype}

\end{Preamble}

\subsection{Abkürzungen}


\subsection{Einheiten}
\label{sec:units}
Mit \Package{units} lassen sich physikalische Einheiten typographisch korrekt 
setzen. Außerdem stellt diese Paket den Befehl \Macro{nicefrac} zur Verfügung, 
mit dem sich mathematische Brüche~-- insbesondere für den Fließtext~-- sehr 
ansehnlich darstellen lassen.
%
\begin{Preamble}
\usepackage{units}

\end{Preamble}




\section{Querverweise}
\label{sec:references}
%
Damit alle möglichen Querverweise in einem PDF-Dokument automatisch verlinkt 
werden, sollte das Paket \Package{hyperref} geladen werden. Um die erzeugten 
Links verträglich aussehen zu lassen, werden die Optionen \Option{colorlinks} 
sowie \Option{linkcolor}[blue] verwendet. Da \Package{hyperref} allerhand 
Veränderungen an vielen Standardbefehlen vornimmt, sollte dieses als letztes in 
der Präambel eingebunden werden. Nur Pakete, bei denen in der Dokumentation 
explizit darauf hingewiesen wird, dass diese nach \Package{hyperref} zu laden 
sind, sollten auch danach folgen. Eines dieser wenigen Pakete ist das in diesem 
Tutorial verwendete \Package{glossaries}. 
%
\begin{Preamble}
\usepackage[colorlinks,linkcolor=blue]{hyperref}
\end{Preamble}



\section{Quelltexte}
\ToDo{listings und fancyvrb}
%


\section{offene Punkte}
\ToDo{Zuordnen der Pakete}
\begin{Hint}
\usepackage{xparse}% §§§ bereits geladen
\usepackage{microtype}% §§§ bereits geladen
\usepackage{textcomp}% §§§ bereits geladen
\usepackage{setspace}% §§§ bereits geladen
\usepackage{csquotes}% §§§ bereits geladen
\usepackage{quoting}% §§§ bereits geladen
\usepackage{isodate}% §§§ bereits geladen
\usepackage{listings}% §§§ bereits geladen
\usepackage{xspace}% §§§ bereits geladen
\usepackage{scrhack}% §§§ bereits geladen
\end{Hint}

\begin{Hint}
\usepackage{caption}
\usepackage{floatrow}
\usepackage{mathtools}
\usepackage{units}
\usepackage{siunitx}
\usepackage{xpunctuate}
\usepackage{ellipsis}
\usepackage{varioref}
\usepackage{chngcntr}
\usepackage{bookmark}
\end{Hint}


Damit ist das Tutorial beendet.
\begin{Trunk+}
\printbibliography\label{sec:bibliography}
\end{Trunk+}

\FinishTutorial[%
  Um das im kopierten Beispiel erstellte Literaturverzeichnis in das Dokument 
  einbinden zu können, bedarf es dem einmaligen Aufruf von \Application{biber} 
  nach dem ersten Durchlauf von \hologo{pdfLaTeX}. Dies erfolgt mit dem Aufruf 
  \PValue{biber}~\PName{Dateiname}. Danach ist ein weiteres mal die Verwendung 
  von \PValue{pdflatex}~\PName{Dateiname} notwendig.

  Für das Erstellen von Abkürzungs- und Symbolverzeichnis sollte das mehrmalige 
  Ausführen von \PValue{pdflatex}~\PName{Dateiname} vollkommen ausreichen. In 
  diesem Fall werden die Einträge mit \Application{makeindex} sortiert. Soll 
  stattdessen \Application{xindy} die Sortierung durchführen, muss beim Laden 
  von \Package{glossaries} die entsprechende Paketoption aktiviert werden. Für
  diesen Fall sollte nach der Verwendung von \hologo{pdfLaTeX} der Aufruf des 
  Perl"=Skriptes \PValue{makeglossaries}~\PName{Dateiname} erfolgen, was 
  allerdings nur mit \Distribution{\hologo{TeX}~Live} ohne weiteres Zutun 
  möglich ist. Nutzer von \Distribution{\hologo{MiKTeX}} müssen das Sortieren 
  mit \Application{xindy} händisch anstoßen.
]

\ToDo[imp]{Einarbeiten von \File{paragraph.tex} in \File{treatise.tex}}[v2.02]
\ListOfToDo
\end{document}
