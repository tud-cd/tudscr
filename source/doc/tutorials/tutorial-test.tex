\RequirePackage[ngerman=ngerman-x-latest]{hyphsubst}
\documentclass[%
  english,ngerman,%
  geometry=no,DIV=12,automark,%
]{tudscrartcl}
\usepackage{selinput}
\SelectInputMappings{adieresis={ä},germandbls={ß}}
\usepackage[T1]{fontenc}
\usepackage{lmodern}

\usepackage{tudscrman2}
\lstset{%
  inputencoding=utf8,extendedchars=true,
  literate=%
    {ä}{{\"a}}1 {ö}{{\"o}}1 {ü}{{\"u}}1
    {Ä}{{\"A}}1 {Ö}{{\"O}}1 {Ü}{{\"U}}1
    {~}{{\textasciitilde}}1 {ß}{{\ss}}1
}
\usepackage{tudscrsupervisor}
\usepackage{tabu,booktabs}
\usepackage{units}
\AfterPackage*{hyperref}{%
  \usepackage[%
    automake,%
    acronym,%
    symbols,%
    nomain,%
    translate=babel,%
    nogroupskip,%
    section=subsubsection,%
  ]{glossaries}
  \setStyleFile{\jobname-temp}
  \renewcommand*{\glsglossarymark}[1]{}
  \makeglossaries
}
\usepackage{csquotes}
\usepackage[backend=biber,style=alphabetic]{biblatex}
\usepackage{filecontents}
\addbibresource{treatise-temp.bib}
\renewcommand{\floatpagefraction}{0.7}

\ifpdf
  \usepackage{tikz}
  \usetikzlibrary{chains}
  \usetikzlibrary{decorations.markings}
  \tikzset{on grid}
\else
  \newcommand*\pgfsyspdfmark[3]{}
\fi
\usepackage{pstricks,pst-node,pst-pdf}

\begin{document}
\TUDoptions{cdfont=no}
\KOMAoptions{headings=normal}
\title{%
  Ein Anwenderleitfaden für das Erstellen einer wissenschaftlichen Abhandlung%
}
\author{Falk Hanisch}
\date{10.09.2014}
\makeatletter
\begingroup%
  \def\and{, }%
  \let\thanks\@gobble%
  \let\footnote\@gobble%
  \hypersetup{%
    pdfauthor = {\@author},%
    pdftitle = {\@title},%
    pdfsubject = {Tutorial für \hologo{LaTeXe}},%
    pdfkeywords = {LaTeX, \TUDScript, Tutorial, Anwenderleitfaden},%
  }%
\endgroup%
\makeatother
\StartTutorial


\subsection{pstricks}

\begin{Tutorial*}
\begin{figure}
\psset{%
  unit=.01\textwidth,%
  cornersize=absolute,%
  labelsep=.8ex,%
  linewidth=.4pt,%
  arrowscale=1.5,%
}
\begin{pspicture}(0,-2)(100,64)
\newcommand\fnodetext{}
\def\fnodetext(#1)#2#3{%
  \fnode[framesize=18 8](#1){#2}%
  \rput(#1){\parbox{17\psunit}{\centering\hspace{0pt}#3}}%
}
\newcommand\scriptbox[2][24]{%
  \parbox{#1\psunit}{\scriptsize\centering\hspace{0pt}#2}%
}
\rput(15,10){%
  \rput(0,49){\textbf{Fahrer}}
  \fnodetext(0,40){NE}{Navigations\-ebene}
  \fnodetext(0,24){BE}{Bahnführungsebene}
  \fnodetext(0,08){SE}{Stabilisierungsebene}
  \ncline{->}{NE}{BE}
  \ncput*{\scriptbox{gewählte Fahrtroute\\zeitlicher Ablauf}}
  \ncline{->}{BE}{SE}
  \ncput*{\scriptbox{gewählte Führungsgrößen:\\Sollspur,~Sollgeschwindigkeit}}
  \psframe[dimen=middle,linewidth=1.2pt,linearc=5](-12.5,0)(12.5,54)
}
\rput(50,10){%
  \rput(0,17){\textbf{Fahrzeug}}
  \fnodetext(0,8){FZ}{Längs- und\\Querdynamik}
  \psframe[dimen=middle,linewidth=1.2pt,linearc=5](-12.5,0)(12.5,22)
}
\rput(85,10){%
  \rput(0,49){\textbf{Umwelt}}
  \fnodetext(0,40){SN}{Straßennetz}
  \fnodetext(0,24){FR}{%
    Fahrraum\\\smallskip\scriptsize{Straße und Verkehrssituation}%
  }
  \fnodetext(0,08){FO}{Fahrbahn\-oberfläche}
  \psframe[dimen=middle,linewidth=1.2pt,linearc=5](-12.5,0)(12.5,54)
}

\ncline{->}{SE}{FZ}
\naput{\scriptbox[9]{Stell\-größen}}
\nbput{\scriptbox[9]{Lenken Gasgeben Bremsen}}
\ncline{->}{FZ}{FO}
\naput{\scriptbox[9]{Regel\-größen}}
\nbput{\scriptbox[9]{Fahrzeugbewegung}}

\psset{armA=15,armB=0,angleA=0,angleB=180}
\ncangles{->}{FZ}{FR}
\ncangles{->}{FZ}{SN}

\psset{angleA=180,angleB=0}
\ncloop[loopsize=12,arm=4.5]{<-}{SE}{FO}
\naput{\scriptbox{Istspur, Istgeschwindigkeit}}
\ncloop[loopsize=32,arm=5]{<-}{BE}{FR}
\naput{\scriptbox[30]{Bereich sicherer Führungsgrößen}}
\ncloop[loopsize=52,arm=5.5]{<-}{NE}{SN}
\naput{\scriptbox{mögliche Fahrtroute}}
\end{pspicture}
\end{figure}
\end{Tutorial*}
\inputtutorial


\begin{itemize}
\item grafiken und untergrafiken?
\item tabellen
\item literaturverzteichnis
\item floatrow
\item mathematikmodus
\item tabbing-umgebung
\end{itemize}
\ListOfToDo
\ToDo[imp]{Einarbeiten von paragraph in treatise}[v2.02]

\FinishTutorial
\end{document}
