\setchapterpreamble{%
  \begin{abstract}
    Zusätzlich zu den bisher im Anwenderhandbuch vorgestellten Klassen und 
    Paketen werden im \TUDScript-Bundle weitere Paket bereitgestellt. Diese 
    sind nicht zwingend an die Verwendung einer der \TUDScript-Klassen 
    angewiesen sondern können prinzipiell mit jeder \hologo{LaTeX}"~Klasse 
    genutzt werden.%
  \end{abstract}
}
\chapter{%
  Zusätzliche Pakete im \TUDScript-Bundle%
  \label{sec:bundle}%
}
\section[%
  Das Paket \PackageRaw{tudscrcolor}{\BooleanFalse}
  -- Farben im \CD%
]{%
  Farben im \CD%
  \tudmarkuplabel{\Package{tudscrcolor}}%
  \index{Farben|(}%
  \index{Layout!Farben|?(}%
}
\begin{DeclareEntity}{\Package{tudscrcolor}}
\ToDo[doc]{als separates Kapitel}[v2.07]
Zur Verwendung der Farben des \CDs wird das Paket \Package{tudscrcolor} 
genutzt. Falls dieses nicht in der Präambel geladen wird~-- um beispielsweise 
zusätzliche Optionen aufzurufen~-- binden die \TUDScript-Klassen dieses 
automatisch ein.

Für das \CD sind mehrere Farben vorgesehen. Die prägnanteste aller ist die 
Hausfarbe \Color{HKS41}, danach folgen die Farben für Auszeichnungen der ersten
(\Color{HKS44}) und der zweiten Kategorie (\Color{HKS36}, \Color{HKS33}, 
\Color{HKS57}, \Color{HKS65}) sowie eine Ausnahmefarbe (\Color{HKS07}). 
Diese Farben dürfen sowohl in ihrer Grundform als auch in helleren Tönen mit 
einer Abstufung in 10\,\%"~Schritten verwendet werden. Das ohnehin verwendete 
Paket \Package{xcolor} stellt genau diese Funktionalität zur Verfügung. Jede 
der Farben kann sowohl mit \Color*{HKS\PName{Zahl}} als auch über ein Pseudonym 
\Color*{cd\PName{Farbe}} genutzt werden.
%
\begin{Example*}
Die Grundfarbe \Color{HKS44} soll in der auf 20\% reduzierten, helleren 
Abstufung genutzt werden. Innerhalb eines Befehls, der als Argument eine 
gültige Farbe erwartet, muss lediglich \PValue{HKS44!20} angegeben werden. 
Dies wird hier exemplarisch mit der folgenden \colorbox{HKS44!20}{%
  Box \Macro{colorbox}[\MPValue{HKS44!20}\MPValue{Box}]%
}
demonstriert.
\end{Example*}
%
Bei der farbigen Gestaltung des \CDs (\Option{cd=\PMisc}()'page') ist~-- je 
nach gewählter Ausprägung~-- der Hintergrund respektive Querbalken von 
Umschlagseite, Titel sowie Teilen in \Color{HKS41} und die Schrift auf selbigen 
in \Color{HKS41}[!30] gehalten. Der Hintergrund von Kapitelseiten erscheint 
gegebenenfalls in \Color{HKS41}[!10], die Schrift in \Color{HKS41}. Bei 
geringerem Farbeinsatz werden lediglich die Schriften der Gliederungsseiten 
auf \Color{HKS41} gesetzt.

Soll das Paket \Package{xcolor} mit speziellen Optionen genutzt werden, lassen 
sich diese \emph{vor} dem Laden der genutzten Dokumentklasse mit 
\Macro{PassOptionsToPackage}[\MPName{Paketoptionen}\MPValue{xcolor}] an dieses 
übergeben. Alternativ kann das Paket \Package{tudscrcolor} mit den gewünschten 
Optionen geladen werden, da dieses Paket unbekannte Einstellungen an 
\Package{xcolor} durchreicht.



\subsection{Generelle Farbdefinitionen}

\newcommand*\cdcolorcalc{}
\newcommand*\cdcolorname{}
\newcommand*\cdcolorvalue{}
\newcommand*\cdcolortext{}
\newcommand*\cdcolor[2][0]{%
  \ifhmode%
    \vskip\medskipamount%
  \else%
    \addvspace{\medskipamount}%
  \fi%
  \noindent%
  \begin{tikzpicture}[%
    every node/.style={%
      rectangle, minimum height=.1\linewidth, minimum width=7em%
    }%
  ]%
  \def\cdcolorcalc##1##2{%
    \pgfmathparse{100-##1*10}%
    \xdef\cdcolorname{HKS##2!\pgfmathresult}%
    \xdef\cdcolorvalue{\pgfmathresult}%
    \pgfmathparse{10+##1*10}%
  }%
  \foreach \x in {0,1,...,9}{%
    \cdcolorcalc{\x}{#2}%
    \ifnum\x<#1%
      \def\cdcolortext{white}%
    \else%
      \def\cdcolortext{black}%
    \fi%
    \node [fill=\cdcolorname,rotate=90] at (.\x\linewidth,0)%
      {\textcolor{\cdcolortext}{HKS#2!\pgfmathprintnumber\cdcolorvalue}};%
  }%
  \end{tikzpicture}%
  \par%
}


\minisec{Primäre Hausfarbe}

\begin{Declaration}
  {\Color{HKS41}}
  (Alternativ:\Color*{cddarkblue})
\printdeclarationlist
\cdcolor[6]{41}
\end{Declaration}


\minisec{Sekundäre Hausfarbe (Geschäftsausstattung)}

\begin{Declaration}
  {\Color{HKS92}}
  (Alternativ:\Color*{cdgray})
\printdeclarationlist
\cdcolor[4]{92}
\end{Declaration}


\minisec{Auszeichnungsfarbe 1.\,Kategorie}

\begin{Declaration}
  {\Color{HKS44}}
  (Alternativ:\Color*{cdblue})
\printdeclarationlist
\cdcolor[4]{44}
\end{Declaration}


\minisec{Auszeichnungsfarbe 2.\,Kategorie}

\begin{Declaration}
  {\Color{HKS36}}
  (Alternativ:\Color*{cdindigo})
\begin{Declaration}
  {\Color{HKS33}}
  (Alternativ:\Color*{cdpurple})
\begin{Declaration}
  {\Color{HKS57}}
  (Alternativ:\Color*{cddarkgreen})
\begin{Declaration}
  {\Color{HKS65}}
  (Alternativ:\Color*{cdgreen})
\printdeclarationlist
\cdcolor[4]{36}
\cdcolor[4]{33}
\cdcolor[2]{57}
\cdcolor{65}
\end{Declaration}
\end{Declaration}
\end{Declaration}
\end{Declaration}


\minisec{Ausnahmefarbe}

\begin{Declaration}
  {\Color{HKS07}}
  (Alternativ:\Color*{cdorange})
\printdeclarationlist
\cdcolor{07}
\end{Declaration}



\subsection{Zusätzliche Farbdefinitionen}

\begin{Declaration}
  {\Option{reduced}}
  ()
\printdeclarationlist

Das Paket \Package{tudscrcolor} definiert lediglich die zuvor beschriebenen 
Grundfarben \Color{HKS41}, \Color{HKS92}, \Color{HKS44}, \Color{HKS36}, 
\Color{HKS33}, \Color{HKS57}, \Color{HKS65} sowie \Color{HKS07}. 
Alle anderen farblichen Abstufungen können mit den beschrieben Möglichkeiten 
des Paketes \Package{xcolor} generiert werden.
\end{Declaration}

\begin{Declaration}
  {\Option{extended}}
\printdeclarationlist

Neben dem \TUDScript-Bundle existieren viele verschiedene Klassen und Pakete 
für das \CD, welche teilweise abweichende Farbdefinitionen nutzen. Durch die 
Paketoption \Option{extended} werden hierfür Farben nach dem Schema 
\Color*{HKS41K\PName{Zahl}} und \Color*{HKS41-\PName{Zahl}} definiert, wobei 
der angestellte Zahlenwert aus der 10er"~Reihe kommen muss.
\end{Declaration}



\subsection{%
  Umstellung des Farbmodells%
  \index{Farben!Farbmodell}%
}

Normalerweise verwendet \Package{tudscrcolor} das CMYK"~Farbmodell. Außerdem 
wird weiterhin noch der RGB"~Farbraum unterstützt. Eine Umschaltung des 
Farbmodells ist beispielsweise für gewisse Funktionen des Paketes 
\Package{tikz} notwendig.

\begin{Declaration}
  {\Option{RGB}}
\printdeclarationlist

Diese Option wird an das Paket \Package{xcolor} durchgereicht, wodurch bereits 
beim Laden des Paketes die globale Farbdefinition nicht nach dem 
CMYK"~Farbmodell sondern im RGB"~Farbraum erfolgt.
\end{Declaration}

\begin{Declaration}
  {\Macro{setcdcolors}[\MPName{Farbmodell}]}
\printdeclarationlist

Mit diesem Befehl kann innerhalb des Dokumentes das verwendete Farbmodell 
angepasst werden. Damit ist es möglich, lokal innerhalb einer Umgebung den 
Farbmodus zu ändern und so nur in bestimmten Situationen beispielsweise aus dem 
CMYK"~Farbmodell in den RGB"~Farbraum zu wechseln. Unterstützte Werte für 
\PName{Farbmodell} sind \PValue{CMYK} und \PValue{RGB} beziehungsweise 
\PValue{rgb}.
\end{Declaration}

\smallskip\noindent
\Attention{%
  Die Darstellung der Farben kann im jeweiligen Farbmodus 
  (\PValue{CMYK} oder \PValue{RGB}) je nach verwendeter Bildschirm"~, Drucker- 
  und Softwarekonfiguration verschieden ausfallen. Die Farbwerte entstammen dem 
  Handbuch zum \CD und sind lediglich Näherungswerte. Abweichungen vom 
  gedruckten HKS"~Farbregister und selbst ermittelten Werten sind technisch 
  nicht zu vermeiden.%
}%
\index{Farben|)}%
\index{Layout!Farben|?)}%
\end{DeclareEntity}



\section[%
  Das Paket \PackageRaw{tudscrfonts}{\BooleanFalse}
  -- Schriften im \CD%
]{%
  Schriften im \CD%
  \tudmarkuplabel{\Package{tudscrfonts}}%
}

\begin{DeclareEntity}{\Package{tudscrfonts}}
[
  v2.02:Neues Paket zur Nutzung Hausschriften ohne \TUDScript-Klassen;
  v2.04:Unterstützung der Klassen \Class{tudposter} und \Class{tudmathposter};
  v2.06:\OpenSans als neue Hausschrift standardmäßig aktiviert;
]
Das Paket \Package{tudscrfonts} stellt die Hausschriften des \TUDCDs für 
\hologo{LaTeX}"~Klassen bereit, welche \emph{nicht} zum \TUDScript-Bundle 
gehören. Das Paket unterstützt einen Großteil der normalerweise für die 
\TUDScript-Klassen bereitgestellten Optionen und Befehle für die 
Schriftauswahl. Um Dopplungen in der Dokumentation zu vermeiden, wird an dieser 
Stelle auf eine abermalige Erläuterung der im Paket \Package{tudscrfonts} 
verfügbaren Optionen und Befehle verzichtet. Diese werden im Folgenden 
lediglich noch einmal genannt, die dazugehörigen Erläuterungen sind in 
\fullref{sec:fonts} zu finden.

Die nutzbaren Paketoptionen sind für den Fließtext \Option{cdfont}~-- ohne die 
Einstellungsmöglichkeiten für den Querbalken des \CDs (\Option{cdhead})~-- und 
für die mathematischen Schriften \Option{cdmath} sowie \Option{slantedgreek}. 
Für die Auswahl der Schreibmaschinenschriften ist \Option{ttfont} verfügbar. 
Weiterhin wird die Option \Option{relspacing} bereitgestellt. Alle genannten 
Optionen können entweder direkt als Klassenoption oder als Paketoptionen im 
optionalen Argument von 
\Macro*{usepackage}[\OPName{Paketoptionen}\MPValue{tudscrfonts}] angegeben 
werden. Zusätzlich ist nach dem Laden des Paketes die Optionenwahl mit 
\Macro{TUDoption} respektive \Macro{TUDoptions} möglich.

Die in \autoref{sec:fonts} beschriebenen Textschalter und "~kommandos zur 
expliziten Auswahl einzelner Schnitte der Hausschrift des \CDs sowie die 
Befehle für griechische Lettern werden ebenso bereitgestellt.
\ChangedAt{v2.06}%
Soll nicht \OpenSans sondern die alten Schriftfamilien \Univers und \DIN 
genutzt werden, ist eine Installation dieser notwendig. Hinweise hierzu sind 
unter \autoref{sec:install:fonts} zu finden. Um diese zu aktivieren, sei auf 
\Option{tudscrver=2.05} und \Option{cdoldfont} verwiesen.

\ChangedAt{v2.04;v2.05}%
Ursprünglich war das Paket \Package{tudscrfonts} für die Verwendung zusammen 
mit einer der Klassen \Class{tudbook}, \Class{tudbeamer}, \Class{tudletter}, 
\Class{tudfax}, \Class{tudhaus}, \Class{tudform} und seit der Version~v2.04 
auch \Class{tudmathposter} sowie \Class{tudposter} vorgesehen. Allerdings 
traten bei der Verwendung des Paketes mit einer dieser Klassen einige kleinere 
Unzulänglichkeiten auf. Deshalb wird seit der Version~v2.05 empfohlen, für 
diese Klassen das Paket \Package{fix-tudscrfonts} zu verwenden.
\end{DeclareEntity}



\section[%
  Das Paket \PackageRaw{mathswap}{\BooleanFalse}
  -- Dezimal- und Tausendertrennzeichen%
]{%
  Formatierung von Dezimal- und Tausendertrennzeichen%
  \tudmarkuplabel{\Package{mathswap}}%
  \index{Mathematiksatz|(}%
  \index{Zifferngruppierung|(}%
}

\begin{DeclareEntity}{\Package{mathswap}}
Die Verwendung von Dezimal- und Tausendertrennzeichen im mathematischen Satz 
sind regional sehr unterschiedlich. In den meisten englischsprachigen Ländern 
wird der Punkt als Dezimaltrennzeichen und das Komma zur Zifferngruppierung 
verwendet, im restlichen Europa wird dies genau entgegengesetzt praktiziert.
Das Paket \Package{mathswap} soll dazu dienen, beliebige formatierte Zahlen in 
ihrer Ausgabe anzupassen. Dafür werden die Zeichen Punkt (\ .\ ) und Komma (\ 
,\ ) als aktive Zeichen im Mathematikmodus definiert.

Ähnliche Funktionalitäten werden bereits durch die Pakete \Package{icomma} und 
\Package{ziffer} bereitgestellt. Bei \Package{icomma} muss jedoch beim
Verfassen des Dokumentes durch den Autor beachtet werden, ob das verwendete
Komma einem Dezimaltrennzeichen entspricht ($t=1,\!2$) oder einem normalen 
Komma im Mathematiksatz ($z=f(x,y)$), wo ein gewisser Abstand nach dem Komma 
durchaus gewünscht ist. Das Paket \Package{ziffer} liefert dafür die gewünschte 
Funktionalität,%
\footnote{kein Leerraum nach Komma, wenn direkt danach eine Ziffer folgt}
ist allerdings etwas unflexibel, was den Umgang mit den Trennzeichen anbelangt.
Als Alternative zu diesem Paket kann außerdem \Package{ionumbers} verwendet 
werden.

Das Paket \Package{mathswap} sorgt dafür, dass Trennzeichen direkt vor einer 
Ziffer erkannt und nach bestimmten Vorgaben ersetzt werden. Sollte sich jedoch 
zwischen Trennzeichen und Ziffer Leerraum befinden, wird dieser als solcher
auch gesetzt. Für ein Beispiel zur Verwendung des Paketes sei auf das Tutorial 
\Tutorial{mathswap} in \autoref{sec:tut} hingewiesen.

\begin{Declaration}
  {\Macro{commaswap}[\MPName{Trennzeichen}]}
\begin{Declaration}
  {\Macro{dotswap}[\MPName{Trennzeichen}]}
\printdeclarationlist

Die beiden Befehle \Macro{commaswap} und \Macro{dotswap} sind die zentrale 
Benutzerschnittstelle des Paketes. Das Makro \Macro{commaswap} definiert das 
Trennzeichen oder den Inhalt, wodurch ein Komma ersetzt werden soll, auf 
welches direkt danach eine Ziffer folgt. Normalerweise setzt \hologo{LaTeX}
nach einem Komma im mathematischen Satz zusätzlich einen horizontalen Abstand.
Bei der Ersetzung durch \Macro{commaswap} entfällt dieser. Die Voreinstellung
für \Macro{commaswap} ist deshalb auf ein Komma (,) gesetzt. Mit dem Makro 
\Macro{dotswap} kann definiert werden, wodurch der Punkt im mathematischen 
Satz ersetzt werden soll, wenn auf diesen direkt anschließend eine Ziffer 
folgt. Da der Punkt im deutschsprachigem Raum zur Gruppierung von Ziffern 
genutzt wird, ist hierfür standardmäßig ein halbes geschütztes Leerzeichen 
definiert (\Macro*{,}).
\end{Declaration}
\end{Declaration}

\begin{Declaration}
  {\Macro{mathswapon}}
  [v2.02]
\begin{Declaration}
  {\Macro{mathswapoff}}
  [v2.02]
\printdeclarationlist

\ChangedAt*{v2.02:Funktionalität im Dokument umschaltbar}
Die Funktionalität von \Package{mathswap} kann innerhalb des Dokumentes mit 
diesen beiden Befehlen an- und abgeschaltet werden. Beim Laden des Paketes ist 
es standardmäßig aktiviert.%
\end{Declaration}
\end{Declaration}
\index{Mathematiksatz|)}%
\index{Zifferngruppierung|)}%
\end{DeclareEntity}



\section[%
  Das Paket \PackageRaw{twocolfix}{\BooleanFalse}
  -- Bugfix für den zweispaltigen Satz%
]{%
  Bugfix für den zweispaltigen Satz%
  \tudmarkuplabel{\Package{twocolfix}}%
  \index{Satzspiegel!zweispaltig|?}%
}

\begin{DeclareEntity}{\Package{twocolfix}}
Der \hologo{LaTeXe}"~Kernel enthält einen Fehler, der Kapitelüberschriften im
zweispaltigen Layout höher setzt, als im einspaltigen. Der 
\hrfn{http://www.latex-project.org/cgi-bin/ltxbugs2html?pr=latex/3126}{Fehler}
ist zwar schon länger bekannt, allerdings bisher noch nicht in den 
\hologo{LaTeXe}"~Kernel übernommen worden. Das Paket \Package{twocolfix} behebt 
das Problem. Eine Integration dieses Bugfixes in \KOMAScript{} wurde bereits 
bei Markus Kohm angefragt, jedoch von ihm bis jetzt 
\hrfn{http://www.komascript.de/node/1681}{nicht weiter verfolgt}.
\end{DeclareEntity}



\addsec*{Zukünftige Arbeiten}

Diese Dinge sollen langfristig in das \TUDScript-Bundle eingearbeitet werden:

%\chapter[%
%  Das Paket \PackageRaw{tudscrletter} -- Briefe im \CD%
%]{%
%  Briefe im \CD%
%  \tudmarkuplabel{\Package{tudscrletter}}%
%}
%\begin{DeclareEntity}{\Package{tudscrletter}}
\ToDo[imp]{%
  Paket \Package*{tudscrletter} für Briefe im \CD, 
  \url{https://komascript.de/node/2200}%
}[v2.07]
Es soll das Paket \Package*{tudscrletter}'none' für Briefe im \TUDCD entstehen. 
Auch Klassen für Fax und Hausmitteilungen sollen dabei abfallen.
%\end{DeclareEntity}

%\chapter[%
%  Das Paket \PackageRaw{tudscrbeamer} -- Präsentationen im \CD%
%]{%
%  Präsentationen im \CD%
%  \tudmarkuplabel{\Package{tudscrbeamer}}%
%}
%\begin{DeclareEntity}{\Package{tudscrbeamer}}
\ToDo[imp]{Paket \Package*{tudscrbeamer} für Präsentationen im \CD}[v2.08]
Mit \Package*{tudscrbeamer}'none' soll ein Paket entstehen, mit dem sich
\hologo{LaTeX}"~Beamer"=Präsentationen im Stil des \TUDCDs mit den 
Einstellungsmöglichkeiten von \TUDScript erstellen lassen. Dies ist jedoch als 
sehr langfristiges Projekt anzusehen und eine Umsetzung momentan nicht 
absehbar. Aktuell können hierfür die Pakete aus dem \GitHubRepo(tud-cd/tud-cd) 
genutzt werden.
%\end{DeclareEntity}

%\chapter[%
%  Das Paket \PackageRaw{tudscrlayout} -- Seitenstil und Satzspiegel im \CD%
%]{%
%  Seitenstil und Satzspiegel im \CD%
%  \tudmarkuplabel{\Package{tudscrlayout}}%
%}
%\begin{DeclareEntity}{\Package{tudscrlayout}}
\ToDo[imp]{%
  Paket \Package*{tudscrlayout} \GitHubRepo(tud-cd/tud-cd)<6> für Satzspiegel, 
  evtl. auch für die Klasse über \Option*{cdgeometry=forced}
}[v2.09]
Außerdem ist ein Paket \Package*{tudscrlayout}'none' vorstellbar, welches den 
durch das \CD vorgegebene Satzspiegel aktiviert ohne den Seitenstil selbst zu 
verwenden, um beispielsweise bereits mit dem Kopf der \TnUD bedrucktes Papier 
nutzen zu können. Ebenfalls wäre es denkbar, für andere Klassen die 
\PageStyle{tudheadings}"=Seitenstile verfügbar zu machen ohne dabei den 
Satzspiegel des \CDs umzusetzen. Dies hat jedoch nur geringe Priorität.
%\end{DeclareEntity}
