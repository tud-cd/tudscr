\chapter{Einleitung}
%
Zur fehlerfreien Verwendung der \TUDScript-Klassen in der Version~\vTUDScript{} 
werden sowohl die \KOMAScript-Klassen der Version~\vKOMAScript{} oder später 
als auch die Hausschrift des \CDs \OpenSans aus dem Paket \Package{opensans} 
zwingend benötigt. Zusätzlich müssen weitere Pakete verfügbar sein, welche 
unter \autoref{sec:packages:needed} aufgeführt sind. Beim Einsatz einer der 
Distributionen
\index{Distribution|?}%
\Distribution{\hologo{TeX}~Live}|?|,
\Distribution{Mac\hologo{TeX}}|?| oder 
\Distribution{\hologo{MiKTeX}}|?|
in der jeweils aktuellen Versionen sollte dies kein Problem darstellen. Wird 
jedoch eine Distribution verwendet, die \TUDScript in der Version~\vTUDScript{} 
nicht zur Verfügung stellt und eine Aktualisierung dieser nicht möglich sein, 
so sollte \autoref{sec:install:ext} konsultiert werden. In diesem Fall ist der 
Anwender selbst dafür verantwortlich, alle benötigten Pakete in der jeweils 
notwendigen Version bereitzustellen, wobei sämtliche Paketabhängigkeiten zu 
beachten sind. Dieses Vorgehen ist jedoch äußerst fehleranfällig, weshalb 
dringlich dazu geraten wird, eine aktuelle Distribution zu verwenden.



\section{Bestandteile und Nutzung des \TUDScript-Bundles}
%
\ChangedAt{v2.01:\TUDScript-Bundle über das \CTAN' veröffentlicht}
%
Das \TUDScript-Bundle wird über das \CTAN bereitgestellt und kann durch 
\hologo{LaTeX}"=Distributionen wie \Distribution{\hologo{TeX}~Live}, 
\Distribution{Mac\hologo{TeX}} oder auch \Distribution{\hologo{MiKTeX}} 
genutzt werden. Es ist hauptsächlich für das Erstellen wissenschaftlicher 
Texte sowie Abhandlungen gedacht und stellt hierfür zum einen die drei 
Hauptklassen \Class{tudscrbook}, \Class{tudscrreprt} sowie \Class{tudscrartcl} 
und zum anderen die Klasse \Class{tudscrposter} zur Verfügung, welche in 
\autoref{sec:mainclasses} beziehungsweise \autoref{sec:poster} vorgestellt 
werden. Das Paket \Package{tudscrsupervisor}~-- dokumentiert in 
\autoref{sec:supervisor}~-- kann zusammen mit den \TUDScript-Klassen genutzt 
werden, um Aufgabenstellungen, Aushänge oder Gutachten für studentische 
Arbeiten zu erstellen. Weiterhin existieren auch eigenständige Pakete, welche 
in \autoref{sec:bundle} beschrieben sind. 

Für die Verwendung des \TUDScript-Bundles ist~-- neben \KOMAScript{} mindestens 
in der Version~\vKOMAScript{} sowie den in \autoref{sec:packages:needed} 
aufgeführten \hologo{LaTeX}"~Paketen~-- seit der Version~v2.06 lediglich die 
Schrift \OpenSans vonnöten, welche durch das Paket \Package{opensans} zur 
Verfügung gestellt wird. Eine lokale Nutzerinstallation der Schriften~-- wie in 
vorherigen Versionen~-- ist nicht notwendig. Lediglich für den Fall, dass 
gezielt die alten Schriften \Univers und \DIN eingesetzt werden sollen, müssen 
diese auch installiert sein. Weitere Hinweise zu deren Installation sowie 
Aktivierung sind in \autoref{sec:install:fonts} zu finden.

\minisec{Anmerkung zu Windows}
Für Windows können zwei unterschiedliche \hologo{LaTeX}"=Distributionen genutzt 
werden. Die Vorteile von \Distribution{\hologo{TeX}~Live} im Vergleich zu 
\Distribution{\hologo{MiKTeX}} liegen zum einen in der Wartung durch mehrere 
Autoren sowie der etwas früheren Verfügbarkeit aller Updates über das \CTAN'. 
Zum anderen werden zusätzlich zu \hologo{LaTeX} ein \emph{Perl"~Interpreter} 
sowie \emph{Ghostscript} mitgeliefert, wodurch die Ad"~hoc"=Verwendung einiger 
Pakete wie beispielsweise \Package{glossaries} vereinfacht beziehungsweise 
verbessert wird. Für \Distribution{\hologo{MiKTeX}} müssen diese Programme 
gegebenenfalls manuell installiert werden. Demgegenüber entfällt für 
\Distribution{\hologo{MiKTeX}} die alljährliche Neuinstallation, welche bei 
\Distribution{\hologo{TeX}~Live} notwendig ist.

\minisec{Anmerkung zu Linux und OS~X}
Die Installation einer der \hologo{LaTeX}"=Distributionen 
\Distribution{\hologo{TeX}~Live} oder \Distribution{Mac\hologo{TeX}} sollte 
direkt über die angebotenen Pakete (\url{https://tug.org/texlive/} oder 
\url{https://tug.org/mactex/}) und nicht über \Path{apt-get install} erfolgen. 
Damit wird sichergestellt, dass die aktuelle Variante der jeweiligen 
Distribution genutzt wird.



\section{Zur Verwendung dieses Handbuchs}
\index{Optionen}%
\index{Umgebungen}%
\index{Befehle}%
%
Sämtliche neu definierten Optionen, Umgebungen und Befehle der Klassen und 
Pakete des \TUDScript-Bundles werden im Handbuch aufgeführt und beschrieben. Am 
Ende des Dokumentes befinden sich mehrere Indexe, die das Nachschlagen oder 
Auffinden dieser erleichtern sollen. Darin werden auch ausgewählte Optionen, 
Umgebungen und Befehle aufgeführt, welche nicht zu \TUDScript gehören und 
dennoch innerhalb dieses Handbuchs Erwähnung finden.

Die im Folgenden beschriebenen Optionen können~-- wie ein Großteil aller 
Einstellungen für \KOMAScript~-- in der Syntax des \Package{keyval}-Paketes 
als Schlüssel"=Wert"=Paare bei der Wahl der Dokumentklasse angegeben werden:
\Macro*{documentclass}[%
  \POParameter{\PName{Schlüssel}\PValue{=}\PName{Wert}}\Parameter{Klasse}%
]

Des Weiteren eröffnen die \KOMAScript-Klassen die Möglichkeit der späten 
Optionenwahl. Damit können Optionen nicht nur direkt beim Laden als sogenannte 
Klassenoptionen angegeben werden, sondern lassen sich auch noch innerhalb des 
Dokumentes nach dem Laden der Klasse ändern. Die \KOMAScript-Klassen sehen 
hierfür zwei Befehle vor. Mit 
\Macro{KOMAoptions}[\Parameter{Optionenliste}](\Package{koma-script})
lassen sich beliebig viele Schlüsseln jeweils genau einen Wert zuweisen, 
\Macro{KOMAoption}[%
  \Parameter{Option}\Parameter{Werteliste}%
](\Package{koma-script})
erlaubt das gleichzeitige Setzen mehrere Werte für genau einen Schlüssel. 
Für die von \TUDScript \emph{zusätzlich} zur Verfügung gestellten Optionen
werden äquivalent dazu die Befehle \Macro{TUDoptions}[\Parameter{Optionenliste}]
und \Macro{TUDoption}[\Parameter{Option}\Parameter{Werteliste}] definiert. 
Damit kann das Verhalten von Optionen im Dokument~-- innerhalb einer Gruppe 
auch lokal~-- geändert werden.

Die Voreinstellung jeder Option wird mit \enquote{Standardwert:\,\PName{Wert}} 
bei deren Beschreibung angeführt. Einige dieser Voreinstellungen sind nicht 
immer gleich sondern werden in Abhängigkeit der genutzten Benutzereinstellungen 
und Optionen gesetzt. Diese bedingten Voreinstellungen werden durch 
\enquote{%
  Standardwert:\,\PName{Wert}%
  \PValue{\,|\,}Bedingung:\,\PName{bedingter~Wert}%
}
angegeben. Wird ein Schlüssel durch den Benutzer \emph{ohne} eine Wertzuweisung 
genutzt, so wird~-- falls vorhanden~-- ein vordefinierter Säumniswert gesetzt, 
welcher in der Beschreibung aller Optionen durch die~\PValue{\emph{kursive}} 
Schreibweise innerhalb der Werteliste gekennzeichnet ist. In den meisten Fällen 
ist der Säumniswert eines Schlüssels \PValue{true}, er entspricht folglich der 
Angabe \PName{Schlüssel}\PValue{=true}. Mit der expliziten Wertzuweisung eines 
Schlüssels durch den Benutzer werden sowohl einfache als auch bedingte 
Voreinstellungen immer überschrieben. Die neben den Optionen neu eingeführten 
Umgebungen und Befehle der Klassen werden~-- gegebenenfalls zusammen mit den 
dazugehörigen optionalen Parametern~-- im gleichen Stil erläutert.



\section{Weitere Klassen und Pakete für das \CD}
Für das Erstellen von Dokumenten im \TUDCD mit \hologo{LaTeX} existiert eine 
\hrfn{https://tu-dresden.de/cd/vorlagen/druck/latex}{Vielzahl von Paketen}. Das 
\TUDScript-Bundle soll diese nicht ersetzen, jedoch längerfristig sämtliche 
Variationen vereinheitlichen und mit einer konsistenten Benutzerschnittstelle 
ausstatten. 

Momentan können ältere Dokumente, welche zuvor mit der Klasse 
\InlineDeclaration{\Class{tudbook}} und gegebenenfalls dem Paket 
\InlineDeclaration{\Package{tudthesis}} mit den Vorlagen von Klaus~Bergmann 
gesetzt wurden, auf eine der \TUDScript-Klassen \Class{tudscrbook} respektive 
\Class{tudscrreprt} oder \Class{tudscrartcl} migriert werden. Bereits 
existierende Poster basierend auf \InlineDeclaration{\Class{tudmathposter}} und 
\InlineDeclaration{\Class{tudposter}} können ebenfalls auf die 
\TUDScript-Klasse \Class{tudscrposter} umgestellt werden. In allen Fällen kann 
dabei das Paket \Package{tudscrcomp} für einen einfacheren Umstieg zum Einsatz 
kommen. 

Für alle weiteren Klassen des besagten Vorlagenbundles von Klaus~Bergmann
\InlineDeclaration{\Class{tudbeamer}}, \InlineDeclaration{\Class{tudletter}},
\InlineDeclaration{\Class{tudfax}}, \InlineDeclaration{\Class{tudhaus}} sowie
\InlineDeclaration{\Class{tudform}} werden \emph{momentan} durch \TUDScript
keine äquivalenten Klassen angeboten.
%
Eine Umsetzung des \CDs für die \Class{beamer}"~Klasse sowie für Briefe und 
Geschäftsschreiben auf Basis von \KOMAScript{} ist bis jetzt leider noch nicht 
mit \TUDScript realisiert worden, soll jedoch langfristig erfolgen. Für 
Präsentationen im \TUDCD existieren für die \Class{beamer}"~Klasse allerdings 
bereits mehrere Stile, die im \GitHubRepo(tud-cd/tud-cd) zu finden sind. 
Für das Erstellen von Briefen mit den \TUDScript-Klassen ließe sich das Paket 
\Package{scrletter}(\Package{koma-script}) nutzen.



\section{Schnelleinstieg}
Das Handbuch gliedert sich in drei Teile. In \autoref{part:main} ist die 
Dokumentation von \TUDScript zu finden. Hier werden alle neuen Optionen, 
Umgebungen und Befehle, die über die Funktionalität von \KOMAScript{} 
hinausgehen, erläutert. \autoref{part:additional} enthält zum einen einfache 
Minimalbeispiele, um den prinzipiellen Umgang und die Funktionalitäten von 
\TUDScript zu demonstrieren. Zum anderen werden hier auch ausführliche und 
dokumentierte Tutorials vor allem für \hologo{LaTeX}"=Neulinge angeboten. 
Insbesondere das Tutorial \Tutorial{treatise} ist mehr als einen Blick wert, 
wenn eine wissenschaftliche Arbeit mit \hologo{LaTeX} verfasst werden soll.
Abschließend werden verschiedenste Pakete vorgestellt, die nicht speziell für 
das \TUDScript-Bundle selber sondern auch für andere \hologo{LaTeX}-Klassen
verwendet werden können und demzufolge für jeden Anwender interessant sein 
könnten. Außerdem werden hier einige Tipps \& Tricks beim Umgang mit 
\hologo{LaTeX} beschrieben, um kleinere oder größere Probleme zu lösen.

Die Klassen \Class{tudscrbook}, \Class{tudscrreprt} und \Class{tudscrartcl} 
sind Wrapper-Klassen der bekannten und korrelierenden \KOMAScript-Klassen 
\Class{scrbook}(\Package{koma-script}), \Class{scrreprt}(\Package{koma-script}) 
sowie \Class{scrartcl}(\Package{koma-script}) und können einfach anstelle deren 
verwendet werden. Auf diesen basierende Dokumente können durch das Umstellen 
der Dokumentklasse einfach in das \TUDCD überführt werden. Bei Fragestellungen 
bezüglich Layout, Schriften oder ähnlichem ist in jedem Fall ein weiterer Blick
in das hier vorliegende Handbuch empfehlenswert.
