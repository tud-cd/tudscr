\chapter{Einleitung}
%
Zur fehlerfreien Verwendung der \TUDScript-Klassen in der Version~\vTUDScript{} 
werden sowohl die \KOMAScript-Klassen der Version~\vKOMAScript{} oder später 
als auch die Hausschrift des \CDs \OpenSans aus dem Paket \Package{opensans} 
zwingend benötigt. Zusätzlich müssen weitere Pakete verfügbar sein, welche 
unter \autoref{sec:packages:needed} aufgeführt sind. Beim Einsatz einer 
aktuellen Version der \Logo{LaTeX}"=Distributionen\index{Distribution|?} 
\Distribution|?|{\Logo{TeX}~Live}, \Distribution|?|{Mac\Logo{TeX}} oder 
\Distribution|?|{\Logo{MiKTeX}} sollte dies kein Problem darstellen. Wird 
jedoch eine \Logo{LaTeX}"=Distribution verwendet, die \TUDScript in der 
Version~\vTUDScript{} nicht zur Verfügung stellt und eine Aktualisierung dieser 
nicht möglich sein, so sollte \autoref{sec:install:ext} konsultiert werden. In 
diesem Fall ist der Anwender selbst dafür verantwortlich, alle benötigten 
Pakete in der jeweils notwendigen Version bereitzustellen, wobei sämtliche 
Paketabhängigkeiten zu beachten sind. Dieses Vorgehen ist jedoch äußerst 
fehleranfällig, weshalb dringlich dazu geraten wird, eine aktuelle 
\Logo{LaTeX}"=Distribution zu verwenden.



\section{Bestandteile und Nutzung des \TUDScript-Bundles}
%
\ChangedAt{v2.01:\TUDScript-Bundle über das \CTAN veröffentlicht}%
\index{Aktualisierung}%
Das \TUDScript-Bundle wird über das \CTAN bereitgestellt und kann 
durch \Logo{LaTeX}"=Distributionen wie \Distribution{\Logo{TeX}~Live}, 
\Distribution{Mac\Logo{TeX}} oder auch \Distribution{\Logo{MiKTeX}} genutzt 
werden. Es ist hauptsächlich für das Erstellen wissenschaftlicher Texte sowie 
Abhandlungen gedacht und stellt hierfür zum einen die drei Hauptklassen 
\Class{tudscrbook}, \Class{tudscrreprt} sowie \Class{tudscrartcl} und zum 
anderen die Klasse \Class{tudscrposter} zur Verfügung, welche in 
\autoref{sec:mainclasses} beziehungsweise \autoref{sec:poster} vorgestellt 
werden. Das Paket \Package{tudscrsupervisor}~-- in \autoref{sec:supervisor} 
dokumentiert~-- lässt sich zusätzlich in Verbindung mit diesen Klassen für die 
Erstellung von Aufgabenstellungen, Aushängen oder Gutachten zu studentischen 
Arbeiten nutzen. Weiterhin existieren auch eigenständige Pakete, welche in 
\autoref{sec:bundle} beschrieben sind. 

Für die Verwendung des \TUDScript-Bundles ist~-- neben \KOMAScript mindestens 
in der Version~\vKOMAScript{} sowie den in \autoref{sec:packages:needed} 
aufgeführten \Logo{LaTeX}"~Paketen~-- seit der Version~v2.06 lediglich die 
Schrift \OpenSans vonnöten, welche durch das Paket \Package{opensans} zur 
Verfügung gestellt wird. Eine lokale Nutzerinstallation der Schriften~-- wie in 
vorherigen Versionen~-- ist nicht notwendig. Lediglich für den Fall, dass 
gezielt die alten Schriften \Univers und \DIN eingesetzt werden sollen, müssen 
diese auch installiert sein. Weitere Hinweise zu deren Installation sowie 
Aktivierung sind in \autoref{sec:install:fonts} zu finden.

\minisec{Anmerkung zu Windows}
%
Unter Windows kann \Distribution{\Logo{TeX}~Live} beziehungsweise  
\Distribution{\Logo{MiKTeX}} als \Logo{LaTeX}"=Distribution genutzt werden. 
Die Vorteile von \Distribution{\Logo{TeX}~Live} liegen zum einen in der 
Wartung durch mehrere Autoren sowie der etwas früheren Verfügbarkeit aller 
Updates über das \CTAN. Zum anderen werden zusätzlich zu \Logo{LaTeX} ein 
\emph{Perl"~Interpreter} sowie \emph{Ghostscript} mitgeliefert, wodurch die 
Ad"~hoc"=Verwendung einiger Pakete wie beispielsweise \Package{glossaries} 
vereinfacht wird. Für \Distribution{\Logo{MiKTeX}} müssen diese externen 
Programme gegebenenfalls manuell installiert werden. Demgegenüber entfällt die 
alljährliche Neuinstallation, welche bei \Distribution{\Logo{TeX}~Live} 
notwendig ist. Weiterhin werden zuvor nicht installierte jedoch benötigte 
\Logo{LaTeX}"~Pakete im Bedarfsfall~-- eine aktive Internetverbindung 
vorausgesetzt~-- automatisch nachinstalliert.

\minisec{Anmerkung zu Linux und OS~X}
%
Die Installation einer der \Logo{LaTeX}"=Distributionen 
\Distribution{\Logo{TeX}~Live} oder \Distribution{Mac\Logo{TeX}} sollte 
direkt über die angebotenen Pakete (\url{https://tug.org/texlive/} oder 
\url{https://tug.org/mactex/}) und nicht über \Path{apt-get install} erfolgen. 
Damit wird sichergestellt, dass die aktuelle Variante der jeweiligen 
\Logo{LaTeX}"=Distribution genutzt wird.



\section{Zur Verwendung dieses Handbuchs}
%
Sämtliche neu definierten Optionen, Umgebungen und Befehle der Klassen und 
Pakete des \TUDScript-Bundles werden im Handbuch aufgeführt und beschrieben. Am 
Ende des Dokumentes befinden sich mehrere Indexe, die das Nachschlagen oder 
Auffinden dieser erleichtern sollen. Darin werden auch ausgewählte Optionen, 
Umgebungen und Befehle aufgeführt, welche nicht zu \TUDScript gehören und 
dennoch innerhalb dieses Handbuchs Erwähnung finden.

Die im Folgenden beschriebenen Optionen können~-- wie ein Großteil aller 
Einstellungen für \KOMAScript~-- in der Syntax des \Package{keyval}-Paketes 
als Schlüssel"=Wert"=Paare bei der Wahl der Dokumentklasse angegeben werden:
\Macro{documentclass|\OPValue{\PName{Schlüssel}=\PName{Wert}}\MPName{Klasse}}

Des Weiteren eröffnen die \KOMAScript-Klassen die Möglichkeit der späten 
Optionenwahl. Damit können Optionen nicht nur direkt beim Laden als sogenannte 
Klassenoptionen angegeben werden, sondern lassen sich auch noch innerhalb des 
Dokumentes nach dem Laden der Klasse ändern. Die \KOMAScript-Klassen sehen 
hierfür zwei Befehle vor. Mit \Macro{KOMAoptions|\MPName{Optionenliste}}
lassen sich beliebig viele Schlüsseln jeweils genau einen Wert zuweisen, 
\Macro{KOMAoption|\MPName{Option}\MPName{Werteliste}} erlaubt das gleichzeitige 
Setzen mehrere Werte für genau einen Schlüssel. Für die von \TUDScript 
\emph{zusätzlich} zur Verfügung gestellten Optionen werden äquivalent dazu die 
Befehle \Macro{TUDoptions|\MPName{Optionenliste}} und 
\Macro{TUDoption|\MPName{Option}\MPName{Werteliste}} definiert. Damit kann das 
Verhalten von Optionen im Dokument~-- innerhalb einer Gruppe auch lokal~-- 
geändert werden.

Bei der Beschreibung aller Optionen sind direkt neben dieser deren jeweilige 
Standardwerte mit \mbox{\enquote*{Voreinstellung: \PName*{Wert}}} angeführt. 
Einige dieser sind nicht immer gleich sondern werden in Abhängigkeit der 
genutzten Benutzereinstellungen und Optionen gesetzt. Diese bedingten 
Standardwerte werden mit 
\mbox{\enquote*{%
  Voreinstellung: \PName*{Wert}\,\textbar\,Bedingung: \PName*{bedingter Wert}%
}}
angegeben. Wird ein Schlüssel durch den Benutzer \emph{ohne} eine Wertzuweisung 
genutzt, so wird~-- falls vorhanden~-- ein vordefinierter Säumniswert gesetzt, 
welcher in der Beschreibung aller Optionen durch die~\PValue{\emph{kursive}} 
Schreibweise innerhalb der Werteliste gekennzeichnet ist. In den meisten Fällen 
ist der Säumniswert eines Schlüssels \PValue{true}, er entspricht folglich der 
Angabe \PName{Schlüssel}\PValue{=true}. Mit der expliziten Wertzuweisung eines 
Schlüssels werden sowohl einfache als auch bedingte Voreinstellungen in jedem 
Fall überschrieben. Die neben den Optionen neu eingeführten Umgebungen und 
Befehle der Klassen werden~-- gegebenenfalls zusammen mit den dazugehörigen 
optionalen Parametern~-- im gleichen Stil erläutert.



\section{Schnelleinstieg}
%
Das Handbuch gliedert sich in drei Teile. In \autoref{part:main} ist die 
Dokumentation von \TUDScript zu finden. Hier werden alle neuen Optionen, 
Umgebungen und Befehle, welche über die Funktionalität von \KOMAScript 
hinausgehen, erläutert. \autoref{part:additional} enthält zum einen einfache 
Minimalbeispiele, um den prinzipiellen Umgang und die Funktionalitäten von 
\TUDScript zu demonstrieren. Zum anderen werden hier auch ausführliche und 
dokumentierte Tutorials vor allem für \Logo{LaTeX}"=Neulinge angeboten. 
Insbesondere das Tutorial \Tutorial{treatise} ist mehr als einen Blick wert, 
wenn eine wissenschaftliche Arbeit mit \Logo{LaTeX} verfasst werden soll.
Abschließend werden verschiedenste Pakete vorgestellt, die nicht speziell für 
das \TUDScript-Bundle selber sondern auch für andere \Logo{LaTeX}"~Klassen
verwendet werden können und demzufolge für jeden Anwender interessant sein 
könnten. Außerdem werden hier einige Tipps \& Tricks beim Umgang mit 
\Logo{LaTeX} beschrieben, um kleinere oder größere Probleme zu lösen.

Die Klassen \Class{tudscrbook}, \Class{tudscrreprt} und \Class{tudscrartcl} 
sind Wrapper-Klassen der bekannten und korrelierenden \KOMAScript-Klassen 
\Class{scrbook}, \Class{scrreprt} sowie \Class{scrartcl} und können einfach 
anstelle deren verwendet werden, wobei ein Blick in das \scrguide sehr zu 
empfehlen ist~-- insbesondere wenn Sie noch keine oder nur wenig Erfahrung im 
Umgang mit den genannten Klassen haben. Auf diesen basierende Dokumente können 
durch das Umstellen der Dokumentklasse einfach in das \TUDCD überführt werden. 
Bei Fragestellungen bezüglich Layout, Schriften oder ähnlichem ist in jedem 
Fall ein weiterer Blick in das hier vorliegende Handbuch empfehlenswert.



\section{Identifikation von \TUDScript}
%
\begin{Entity}{\Bundle{tudscr}}
Im \TUDScript-Bundle gibt es neben den Klassen selbst auch noch zusätzliche 
Pakete. Ein Teil dieser Pakete~-- genauer \Package{tudscrsupervisor} und 
\Package{tudscrcomp}~-- sind ausschließlich mit den \TUDScript-Klassen nutzbar.
Andere wiederum~-- die beiden Pakete \Package{tudscrfonts}~(Schriften) und 
\Package{tudscrcolor}~(Farben) für Belange des \CDs sowie die davon vollkommen 
unabhängigen Pakete \Package{mathswap} und \Package{twocolfix}~-- können mit 
allen existierenden \Logo{LaTeX}"~Klassen genutzt werden. Sämtliche Klassen 
und Pakete aus dem \TUDScript-Bundle enthalten die folgenden Befehle, welche 
diese als dessen Bestandteil identifizieren.

\begin{Declaration}
  {\Macro{TUDScript}}
  [v2.04]
\printdeclarationlist
%
Diese Anweisung setzt das Logo respektive die Wortmarke \enquote{\TUDScript{}} 
in serifenloser Schrift und mit leichter Sperrung des in Versalien gesetzten 
Teils. Dieser Befehl wird von allen Klassen und Paketen des \TUDScript-Bundles 
definiert.
\end{Declaration}

\begin{Declaration}
  {\Macro{TUDScriptClassName}}
  [v2.04]
\printdeclarationlist
%
Die Bezeichnung der jeweiligen, im Dokument verwendeten \TUDScript-Klasse ist 
im Makro \Macro{TUDScriptClassName} abgelegt. Soll also in Erfahrung gebracht 
werden, ob~-- und wenn ja, welche~-- \TUDScript-Klasse verwendet wird, so kann 
einfach auf diese Anweisung getestet werden. \KOMAScript stellt zusätzlich 
noch die beiden Anweisungen \Macro||{KOMAClassName} und \Macro||{ClassName} 
bereit, welche den Namen der zugrundeliegenden \KOMAScript-Klasse sowie die 
durch diese ersetzte Standardklasse enthalten.
\end{Declaration}

\begin{Declaration}
  {\Macro{TUDScriptVersion}}
  [v2.04]
\begin{Declaration}
  {\Macro{TUDScriptVersionNumber}}
  [v2.05]
\printdeclarationlist
%
In \Macro{TUDScriptVersion} ist die Hauptversion von \TUDScript in der Form
\begin{quoting}
\PName{Datum}~\PName{Version}~\PValue{TUD-Script}
\end{quoting}
abgelegt. Die Version ist für alle Klassen und Pakete des \TUDScript-Bundles
gleich und kann nach dem Laden einer Klasse oder eines Paketes abgefragt 
werden. Beispielsweise wurde diese Anleitung mit \enquote{\TUDScriptVersion{}} 
erstellt.

Eventuell will der Anwender auf die aktuell verwendete Version von \TUDScript 
prüfen, um gegebenenfalls eigene Anpassungen in Abhängigkeit der verwendeten 
Version vorzunehmen. Hierfür kann \Macro{TUDScriptVersionNumber} verwendet 
werden. Darin ist alleinig die Versionsnummer enthalten. Die für das Handbuch 
verwendete Version lautet \enquote{\TUDScriptVersionNumber{}}.
\end{Declaration}
\end{Declaration}
\end{Entity}



\ToDo{Deklarationen tud\dots ins Paket tudscrcomp verschieben}
\section{%
  Weitere Klassen und Pakete für das \CD%
  \label{sec:tudclasses}%
}
%
Für das Erstellen von Dokumenten im \TUDCD mit \Logo{LaTeX} existiert eine 
\hrfn{https://tu-dresden.de/cd/vorlagen/druck/latex}{Vielzahl von Paketen}. Das 
\TUDScript-Bundle soll diese nicht ersetzen, jedoch längerfristig sämtliche 
Variationen vereinheitlichen und mit einer konsistenten Benutzerschnittstelle 
ausstatten. 

Momentan können ältere Dokumente, welche zuvor mit der Klasse 
\InlineDeclaration{\Class{tudbook}} und gegebenenfalls dem Paket 
\InlineDeclaration{\Package{tudthesis}} mit den Vorlagen von Klaus~Bergmann 
gesetzt wurden, auf eine der \TUDScript-Klassen \Class{tudscrbook} respektive 
\Class{tudscrreprt} oder \Class{tudscrartcl} migriert werden. Bereits 
existierende Poster basierend auf \InlineDeclaration{\Class{tudmathposter}} und 
\InlineDeclaration{\Class{tudposter}} können ebenfalls auf die 
\TUDScript-Klasse \Class{tudscrposter} umgestellt werden. In allen Fällen kann 
dabei das Paket \Package{tudscrcomp} für einen einfacheren Umstieg zum Einsatz 
kommen. 

Für alle weiteren Klassen des besagten Vorlagenbundles von Klaus~Bergmann
\InlineDeclaration{\Class{tudbeamer}}, \InlineDeclaration{\Class{tudletter}},
\InlineDeclaration{\Class{tudfax}}, \InlineDeclaration{\Class{tudhaus}} sowie
\InlineDeclaration{\Class{tudform}} werden \emph{momentan} durch \TUDScript
keine äquivalenten Klassen angeboten.

Eine Umsetzung des \CDs für die \Class{beamer}"~Klasse sowie für Briefe und 
Geschäftsschreiben auf Basis von \KOMAScript ist bis jetzt leider noch nicht 
mit \TUDScript realisiert worden, soll jedoch langfristig erfolgen. Für 
Präsentationen im \TUDCD existieren für die \Class{beamer}"~Klasse allerdings 
bereits mehrere Stile, die im \GitHubRepo(tud-cd/tud-cd) zu finden sind. 
Für das Erstellen von Briefen mit den \TUDScript-Klassen ließe sich das Paket 
\Package{scrletter} nutzen.
