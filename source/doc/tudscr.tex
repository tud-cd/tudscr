\RequirePackage[ngerman=ngerman-x-latest]{hyphsubst}
\documentclass[english,ngerman,final]{tudscrman}
\usepackage{selinput}\SelectInputMappings{adieresis={ä},germandbls={ß}}
\usepackage[T1]{fontenc}
\lstset{%
  inputencoding=utf8,extendedchars=true,
  literate=%
    {ä}{{\"a}}1 {ö}{{\"o}}1 {ü}{{\"u}}1
    {Ä}{{\"A}}1 {Ö}{{\"O}}1 {Ü}{{\"U}}1
    {~}{{\textasciitilde}}1 {ß}{{\ss}}1
}
\begin{document}
\addtokomafont{subtitle}{\univbn}
\subject{\TUDScript{} \vTUDScript{} basierend auf \KOMAScript{} \vKOMAScript}
\title{%
  Ein \NoCaseChange{\hologo{LaTeXe}}-Bundle für Dokumente im~neuen \CD der \TnUD
}
\ifdef{\tudprintflag}{%
  \subtitle{Benutzerhandbuch\thanks{\href{tudscr}{Online-Version}}}%
}{%
  \subtitle{Benutzerhandbuch\thanks{\href{tudscr_print}{Druckversion}}}%
}
\author{Falk Hanisch\thanks{\noexpand\href{mailto:\tudscrmail}{\tudscrmail}}}
\faculty{http://tu-dresden.de/cd}
\date{03.09.2014}
\maketitle
\addchap{\prefacename}
Die im Folgenden beschriebenen Klassen und Pakete wurden für das Erstellen von 
\hologo{LaTeX}"=Dokumenten im \CD der \TnUD entwickelt.%
\footnote{%
  \url{http://tu-dresden.de/cd}\hfill
  \url{http://tu-dresden.de/service/publizieren/cd/6_handbuch/index.html}%
}
Sie basieren auf den gerade im deutschsprachigen Raum häufig verwendeten 
\KOMAScript"=Klassen, welche eine Vielzahl von Einstellmöglichkeiten bieten, 
die weit über die Möglichkeiten der \hologo{LaTeX}"=Standardklassen 
hinausgehen. Zusätzlich bietet das hier dokumentierten \TUDScript-Bundle 
weitere, insbesondere das Dokumentlayout betreffende Auswahlmöglichkeiten.

Es sei angemerkt, dass die hier beschriebenen Klassen eine Abweichung vom \CD 
der \TnUD zulassen, da dieses gerade unter typographischen Gesichtspunkten 
durchaus als diskussionswürdig zu erachten ist. Mit den entsprechenden 
Einstellungen kann bis auf das Standardlayout der \KOMAScript"=Klassen 
zurückgestellt werden. Inwieweit der Nutzer der \TUDScript"=Klassen von diesen 
Möglichkeiten Gebrauch macht, bleibt ihm selbst überlassen. Ohne die gezielte 
Verwendung der entsprechenden Optionen werden standardmäßig alle Vorgaben des 
\CDs umgesetzt.

Dieses Handbuch soll dazu dienen, eine schnelle Einführung in die neuen Klassen
und Pakete zu ermöglichen. Es werden Hinweise für eine einfache Installation 
und einen Überblick über die zusätzlich zu den \KOMAScript"=Klassen nutzbaren 
Optionen sowie die neu eingeführten Befehle gegeben. Dies bedeutet, dass 
Grundkenntnisse in der Verwendung von \hologo{LaTeX} vorausgesetzt werden. 
Sollten diese nicht vorhanden sein, wird dem Nutzer zumindest das Lesen der 
Kurzbeschreibung von \hologo{LaTeXe}
\hrfn{http://mirrors.ctan.org/info/lshort/german/l2kurz.pdf}{\File{l2kurz.pdf}}
dringend empfohlen. Des Weiteren sollte sowohl der Einsteiger als auch der 
erfahrene Nutzer mindestens einmal das \hologo{LaTeXe}"=Sündenregister
\hrfn{http://mirrors.ctan.org/info/l2tabu/german/l2tabu.pdf}{\File{l2tabu.pdf}}
überblickt haben, um sehr typische Fehler beim Umgang mit \hologo{LaTeX} zu 
vermeiden. Ein umfangreiches Tutorial für \hologo{LaTeX}-Einsteiger ist unter 
diesem \hrfn{http://www.fadi-semmo.de/latex/workshop/}{Link} zu finden. 
Antworten auf häufige Fragen liefert
\hrfn{http://projekte.dante.de/DanteFAQ/WebHome}{DANTE-FAQ}. Sollte der Nutzer 
unerfahren bei der Verwendung der \KOMAScript"=Klassen sein, so ist ein Blick 
in das dazugehörige Anwenderhandbuch \scrguide sehr zu empfehlen, wenn nicht 
sogar unumgänglich. Nichtsdestotrotz werden in \autoref{part:additional} 
Minimalbeispiele sowie etwas ausführlichere Tutorials für angeboten. 

Der aktuelle Stand der Klassen und Pakete aus dem \TUDScript-Bundle wurde nach 
bestem Wissen und Gewissen auf Herz und Nieren getestet. Dennoch kann nicht für 
das Ausbleiben von Fehlern garantiert werden. Beim Auftreten eines Problems 
sollte dieses bitte genauso wie Inkompatibilitäten mit anderen Paketen im Forum 
unter
\begin{quote}
\Forum*%
\end{quote}
gemeldet beziehungsweise geäußert werden. Für eine schnelle und erfolgreiche 
Fehlersuche sollte ein \hrfn{http://www.komascript.de/minimalbeispiel} 
{\textbf{lauffähiges~Minimalbeispiel}} bereitgestellt werden. Auf Anfragen ohne 
dieses werde ich gegebenenfalls verspätet oder gar nicht reagieren. Ebenso sind 
dort auch \emph{Fragen}, \emph{Kritik} und \emph{Verbesserungsvorschläge}~-- 
sowohl das Bundle selbst als auch die Dokumentation betreffend~-- gerne 
gesehen. Da dieses Bundle in meiner Freizeit entstanden ist und auch gepflegt 
wird, bitte ich um Nachsicht, falls ich nicht sofort antworte und/oder eine 
Fehlerkorrektur vornehmen kann.

\makeatletter
\bigskip
\noindent Falk Hanisch\newline
Dresden, \@date
\makeatother
\makeatletter
\renewcommand*\@pnumwidth{1.7em}
\makeatother
\tableofcontents
\include{tudscr_introduction}
\setpartpreamble{%
  \begin{abstract}
    \hypersetup{linkcolor=red}
    \noindent Dies ist der Hauptteil des \TUDScript-Bundles. Hier findet der 
    Anwender alle verfügbaren Optionen, Umgebungen und Befehle, die über 
    die Funktionalität von \KOMAScript{} hinausgehen.
  \end{abstract}
}
\part{Das \TUDScript-Bundle}\label{part:main}
\include{tudscr_mainclasses}
\include{tudscr_bundle}
\setpartpreamble{%
  \begin{abstract}
    \hypersetup{linkcolor=red}
    \noindent Für die Verwendung des \TUDScript-Bundles ist es nicht notwendig,
    diesen Teil zu lesen. In \autoref{sec:exmpl} sind insbesondere für 
    \hologo{LaTeX}"=Neulinge sowie neue Anwender des \TUDScript-Bundles mehrere 
    einfache Beispiele sowie umfangreichere Tutorials für dessen Verwendung zu 
    sehen. In \autoref{sec:packages} werden Einsteigern~-- und auch dem bereits 
    versierten \hologo{LaTeX}-Nutzer~-- meiner Meinung nach empfehlenswerte 
    Pakete kurz vorgestellt.
    
    Anwendungshinweise sowie der eine oder andere allgemeine Hinweis bei der 
    Verwendung von \hologo{LaTeXe} wird in \autoref{sec:tips} gegeben. Dabei 
    sind diese durchaus für die Verwendung sowohl des \TUDScript-Bundles als 
    auch anderer \hologo{LaTeX}-Klassen interessant. Für Anregungen, Hinweise, 
    Ratschläge oder Empfehlungen zu weiteren Pakete sowie Tipps bin ich 
    jederzeit empfänglich.
  \end{abstract}
}
\part{Ergänzungen und Hinweise}
\label{part:additional}
\include{tudscr_examples}
\include{tudscr_packages}
\include{tudscr_hints}
\appendix
\part{Anhang}
\include{tudscr_installation}
\chapter{Obsolete sowie vollständig entfernte Optionen und Befehle}
\label{sec:obsolete}%
%
Einige Optionen und Befehle waren während der Weiterentwicklung von \TUDScript
in ihrer ursprünglichen Form nicht mehr umsetzbar oder wurden schlichtweg 
verworfen. Dennoch wird hier \emph{teilweise} gezeigt, wie die Funktionalität 
mit \TUDScript in der Version \vTUDScript{} darstellbar ist.

\begin{Declaration}{\Option{cd}[alternative]}{entfällt}
\begin{Declaration}{\Option{cdtitle}[alternative]}{entfällt}
\begin{Declaration}{\Length{titlecolwidth}}{entfällt}
\begin{Declaration}{\Term{authortext}}{entfällt}
\printdeclarationlist*%
\index{Titel!alternativer}%
%
Die alternative Titelseite ist komplett aus dem \TUDScript-Bundle entfernt 
worden. Dementsprechend entfallen auch die dazugehörigen Optionen sowie Länge 
und Bezeichner.
\end{Declaration}
\end{Declaration}
\end{Declaration}
\end{Declaration}

\begin{Declaration}{\Option{color}[\PBoolean]}{siehe \Option*{cd}[color]}
\printdeclarationlist*%
%
Die Einstellungen der farbigen Ausprägung des Dokumentes erfolgt über die 
Option \Option*{cd}.
\end{Declaration}

\begin{Declaration}{\Option{tudfonts}[\PBoolean]}{siehe \Option*{cdfont}[\PSet]}
\printdeclarationlist*%
%
Die Option zur Schrifteinstellung ist wesentlich erweitert worden. Aus Gründen 
der Konsistenz wurde diese umbenannt.
\end{Declaration}

\begin{Declaration}{\Option{tudfoot}[\PBoolean]}{
  siehe \Option*{cdfoot}[\PBoolean]%
}
\printdeclarationlist*%
%
Ebenso wurde diese Option umbenannt, um dem Namensschema der restlichen 
Optionen zu entsprechen.
\end{Declaration}

\begin{Declaration}{\Option{headfoot}[\PSet]}{entfällt}
\printdeclarationlist*%
%
Diese Option war in der Version~v1.0 notwendig, um die parallele Verwendung von 
\Package*{typearea} und \Package*{geometry} zu ermöglichen. Dies wurde komplett 
überarbeitet, an das Paket \Package*{geometry} werden die Einstellungen für die 
\KOMAScript"=Optionen \Option*{headinclude} und \Option*{footinclude} jetzt 
direkt weitergereicht. Damit ist die Option \Option*{headfoot} nicht mehr 
notwendig und wurde entfernt.
\end{Declaration}

\begin{Declaration}{\Option{partclear}[\PBoolean]}{%
  entfällt, siehe \Option*{cleardoublespecialpage}%
}
\begin{Declaration}{\Option{chapterclear}[\PBoolean]}{%
  entfällt, siehe \Option*{cleardoublespecialpage}%
}
\printdeclarationlist*%
%
Beide Optionen sind in der neuen Option \Option*{cleardoublespecialpage} 
aufgegangen, womit ein konsistentes Layout erreicht wird. Die ursprünglichen 
Optionen entfallen. 
\end{Declaration}
\end{Declaration}

\begin{Declaration}{\Option{abstracttotoc}[\PBoolean]}{%
  entfällt, siehe \Option*{abstract}[\PSet]%
}
\begin{Declaration}{\Option{abstractdouble}[\PBoolean]}{%
  entfällt, siehe \Option*{abstract}[\PSet]%
}
\printdeclarationlist*%
%
Beide Optionen wurden in die Option \Option*{abstract} integriert und sind 
deshalb überflüssig.
\end{Declaration}
\end{Declaration}

\begin{Declaration}{\Macro{confirmationandrestriction}}{%
  entfällt, siehe \Macro*{declaration}%
}
\begin{Declaration}{\Macro{restrictionandconfirmation}}{%
  entfällt, siehe \Macro*{declaration}%
}
\begin{Declaration}{\Macro{location}\Parameter{Ort}}{%
  siehe \Macro*{place} sowie auch Parameter \Key*{\Macro{declaration}}{place}%
}
\printdeclarationlist*%
%
Die ersten beiden Befehle entfallen, \Macro*{declaration} kann alternativ dazu 
verwendet werden. In Anlehnung an andere \hologo{LaTeX}-Pakete und "~Klassen 
wurde \Macro*{location} in \Macro*{place} umbenannt.
\end{Declaration}
\end{Declaration}
\end{Declaration}

\begin{Declaration}{\Macro{logofile}\Parameter{Dateiname}}%
  {siehe \Macro*{headlogo}\Parameter{Dateiname}%
}
\begin{Declaration}{\Macro{logofilename}\Parameter{Dateiname}}%
  {siehe \Macro*{headlogo}\Parameter{Dateiname}%
}
\printdeclarationlist*%
%
Der Befehl \Macro*{logofile} wurde in \Macro*{headlogo} umbenannt.
\end{Declaration}
\end{Declaration}

\begin{Declaration}{\Length{chapterheadingvskip}}{%
  siehe \Length*{pageheadingsvskip} sowie \Length*{headingsvskip}
}
\begin{Declaration}{\Length{signatureheight}}{entfällt}
\printdeclarationlist*%
%
Die vertikale Positionierung von Überschriften wurde zweigeteilt. Die Höhe für 
die Zeile der Unterschriften wurde dehnbar gestaltet. Eine Anpassung durch den 
Anwender ist nicht vonnöten.
\end{Declaration}
\end{Declaration}

\begin{Declaration}{\Term{titlecoldelim}}{%
  entfällt, siehe \Macro*{titledelimiter}%
}
\printdeclarationlist*%
%
Das Trennzeichen für Bezeichnungen beziehungsweise beschreibende Texte und dem 
eigentlichen Feld auf der Titelseite ist nicht mehr sprachabhängig und wurde 
umbenannt.
\end{Declaration}

\begin{Declaration}{\Macro{submissiondate}\Parameter{Datum}}{%
  Alias für \Macro*{date}%
}
\begin{Declaration}{\Macro{birthday}\Parameter{Geburtsdatum}}{%
  Alias für \Macro*{dateofbirth}%
}
\begin{Declaration}{\Macro{birthplace}\Parameter{Geburtsort}}{%
  Alias für \Macro*{placeofbirth}
}
\begin{Declaration}{\Macro{studentid}\Parameter{Matrikelnummer}}{%
  Alias für \Macro*{matriculationnumber}
}
\begin{Declaration}{\Macro{enrolmentyear}\Parameter{Immatrikulationsjahr}}{%
  Alias für \Macro*{matriculationyear}%
}
\printdeclarationlist*%
%
Alle Befehle wurden umbenannt und sind jetzt für die Titelseite im \CD nutzbar.
\end{Declaration}
\end{Declaration}
\end{Declaration}
\end{Declaration}
\end{Declaration}

\begin{Declaration}{\Term{submissiontext}}{umbenannt, siehe \Term*{datetext}}
\begin{Declaration}{\Term{birthdaytext}}{%
  umbenannt, siehe \Term*{dateofbirthtext}%
}
\begin{Declaration}{\Term{birthplacetext}}{%
  umbenannt, siehe \Term*{placeofbirthtext}%
}
\begin{Declaration}{\Term{studentidname}}{%
  umbenannt, siehe \Term*{matriculationnumbername}%
}
\begin{Declaration}{\Term{enrolmentname}}{%
  umbenannt, siehe \Term*{matriculationyearname}%
}
\begin{Declaration}{\Term{supervisorIIname}}{%
  umbenannt, siehe \Term*{supervisorothername}%
}
\begin{Declaration}{\Term{defensetext}}{%
  umbenannt, siehe \Term*{defensedatetext}%
}
\printdeclarationlist*%
%
Die Bezeichner wurden in Anlehnung an die dazugehörigen Befehlsnamen umbenannt.
\end{Declaration}
\end{Declaration}
\end{Declaration}
\end{Declaration}
\end{Declaration}
\end{Declaration}
\end{Declaration}


\minisec{Aufgabenstellung}
Die Umgebung für die Erstellung einer Aufgabenstellung für eine 
wissenschaftliche Arbeit wurde in das Paket \Package*{tudscrsupervisor} 
ausgelagert. Dieses muss für die Verwendung der Umgebung \Environment*{task} 
und der daraus abgeleiteten standardisierten Form zwingend geladen werden.

\begin{Declaration}{\Option{cdtask}[\PSet]}{entfällt, siehe \Environment*{task}}
\begin{Declaration}{\Option{taskcompact}[\PBoolean]}{entfällt}
\begin{Declaration}{\Macro{tasks}\Parameter{Ziele}\Parameter{Schwerpunkte}}{%
  umbenannt, siehe \Macro*{taskform}%
}
\begin{Declaration}{\Length{taskcolwidth}}{entfällt}
\printdeclarationlist*%
%
Die Klassenoption \Option*{cdtask} ist komplett entfernt worden, alle 
Einstellungen, welche \Environment*{task} betreffen erfolgen direkt über das 
optionale Argument der Umgebung. Die Variante eines kompakten Kopfes mit der 
Option \Option*{taskcompact} wird nicht mehr bereitgestellt. Der Befehl 
\Macro*{tasks} wurde in \Macro*{taskform} umbenannt und in der Funktionalität 
erweitert. Die manuelle Einstellung der Spaltenbreite für den Kopf der 
Aufgabenstellung mit \Length*{taskcolwidth} wurde aufgrund der verbesserten 
automatischen Berechnung entfernt.
\end{Declaration}
\end{Declaration}
\end{Declaration}
\end{Declaration}

\begin{Declaration}{\Macro{startdate}\Parameter{Ausgabedatum}}{%
  Alias für \Macro*{issuedate}%
}
\begin{Declaration}{\Macro{enddate}\Parameter{Abgabetermin}}{%
  Alias für \Macro*{duedate}%
}
\begin{Declaration}{\Term{starttext}}{umbenannt, siehe \Term*{issuedatetext}}
\begin{Declaration}{\Term{duetext}}{umbenannt, siehe \Term*{duedatetext}}
\begin{Declaration}{\Term{focustext}}{umbenannt, siehe \Term*{focusname}}
\begin{Declaration}{\Term{objectivestext}}{%
  umbenannt, siehe \Term*{objectivesname}%
}
\printdeclarationlist*%
%
Alle genannten Befehle und Bezeichner wurden umbenannt.
\end{Declaration}
\end{Declaration}

\end{Declaration}
\end{Declaration}
\end{Declaration}
\end{Declaration}
\include{tudscr_index}
\ListOfToDo
\ToDo[man]{Layout und Umbrüche kontrollieren}
\ToDo[man]{Datum in tudscr-version.dtx, Handbuch und README aktualisieren}
\ToDo[man]{Hinweis auf falschen Link auf CTAN}
\ToDo[man]{Homepage überarbeiten}
\ToDo[doc]{Änderungsnotizen}[v2.02]
\end{document}