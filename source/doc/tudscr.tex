\documentclass[english,ngerman]{tudscrmanual}
\iftutex
  \usepackage{fontspec}
\else
  \usepackage[T1]{fontenc}
  \usepackage[ngerman=ngerman-x-latest]{hyphsubst}
\fi

\usepackage{babel}
\usepackage[babel]{microtype}
\usepackage{widows-and-orphans}

\KOMAoptions{headings=optiontoheadandtoc}

%\tracinglabels[all]
%\tracingmarkup
%\tracingbundle

\lstset{%
  inputencoding=utf8,extendedchars=true,
  literate=%
    {ä}{{\"a}}1 {ö}{{\"o}}1 {ü}{{\"u}}1
    {Ä}{{\"A}}1 {Ö}{{\"O}}1 {Ü}{{\"U}}1
    {ß}{{\ss}}1 {~}{{\textasciitilde}}1
    {»}{{\guillemetright}}1 {«}{{\guillemetleft}}1
}

\usepackage{bookmark}
\GitHubBase{\TUDScriptRepository}
\begin{document}
\newrobustcmd*\cdurl{%
  \begingroup
    \hypersetup{hidelinks}\urlstyle{same}%
    \url{https://tu-dresden.de/cd}%
  \endgroup
}
\faculty{\cdurl}
\subject{\TUDScript \vTUDScript{} basierend auf \KOMAScript}
\title{Ein \hologo{LaTeX}-Bundle für Dokumente im \TUDCD}
\ifdef{\tudprintflag}{%
  \subtitle{Benutzerhandbuch\thanks{\href{tudscr}{Online-Version}}}%
}{%
  \subtitle{Benutzerhandbuch\thanks{\href{tudscr_print}{Druckversion}}}%
}
\author{Falk Hanisch\TUDScriptContactTitle}
\publishers{\url{\TUDScriptRepository}}
\date{2022-08-09}

\makeatletter
\begingroup%
  \def\and{, }%
  \let\thanks\@gobble%
  \let\footnote\@gobble%
  \let\emailaddress\@gobble%
  \hypersetup{%
    pdfauthor   = {\@author},%
    pdftitle    = {\@title},%
    pdfsubject  = {\@subtitle},%
    pdfkeywords = {LaTeX, \TUDScript, Benutzerhandbuch},%
  }%
\endgroup%
\tud@newdim\tudscrdim
\makeatother


\maketitle


\addchap[tocentry={}]{\prefacename}
\currentpdfbookmark{\prefacename}{preface}
In diesem Handbuch werden die für das Erstellen von \hologo{LaTeX}"=Dokumenten 
im \hrfn{https://tu-dresden.de/cd}{\TUDCD} entwickelten Klassen und Pakete 
beschrieben. Diese basieren auf den gerade im deutschsprachigen Raum häufig 
verwendeten \KOMAScript-Klassen, die eine Vielzahl von Einstellmöglichkeiten 
bieten, welche weit über die Möglichkeiten der \hologo{LaTeX}"=Standardklassen 
hinausgehen. Zusätzlich bietet das \TUDScript-Bundle weitere, insbesondere das 
Dokumentlayout betreffende Auswahlmöglichkeiten.

Es sei angemerkt, dass die hier beschriebenen Klassen eine Abweichung vom 
\TUDCD zulassen, da dieses gerade unter typografischen Gesichtspunkten 
durchaus als diskussionswürdig zu erachten ist. Prinzipiell ist es mit den 
entsprechenden Einstellungen möglich, auf das Layout der \KOMAScript-Klassen 
zurückzuschalten. Ohne die gezielte Verwendung dieser Optionen durch den 
Anwender werden per Voreinstellung alle Vorgaben des \CDs umgesetzt.

Dieses Handbuch enthält eine Beschreibung der \emph{zusätzlich} zu den 
\KOMAScript-Klassen nutzbaren Optionen und Befehle. Dabei werden 
Grundkenntnisse in der Verwendung von \hologo{LaTeX} vorausgesetzt. Sollten 
diese nicht vorhanden sein, wird das Lesen der \hologo{LaTeXe}"=Kurzbeschreibung
\hrfn{http://mirrors.ctan.org/info/lshort/german/l2kurz.pdf}{\File{l2kurz.pdf}}
dringend empfohlen. Für den vertiefenden Einstieg in \hologo{LaTeX} stellt 
Nicola~L.~C.~Talbot sehr ausführliche Tutorials für 
\hrfn{http://www.dickimaw-books.com/latex/novices/}{\hologo{LaTeX}-Novizen} 
und \hrfn{http://www.dickimaw-books.com/latex/thesis/}{Dissertationen} zur 
freien Verfügung. Zusätzlich werden in \autoref{part:additional} dieses 
Handbuchs Minimalbeispiele sowie etwas ausführlichere Tutorials angeboten.

Des Weiteren sollte \emph{jeder} Anwender das \hologo{LaTeXe}"=Sündenregister 
\hrfn{http://mirrors.ctan.org/info/l2tabu/german/l2tabu.pdf}{\File{l2tabu.pdf}}
kennen, um typische Fehler zu vermeiden. Antworten auf häufig gestellte Fragen 
liefert \hrfn{http://projekte.dante.de/DanteFAQ/WebHome}{DanteFAQ}. Falls der 
Nutzer unerfahren bei der Verwendung von \KOMAScript{} sein sollte, so ist ein 
Blick in das \scrguide[dazugehörige Handbuch] sehr zu empfehlen, wenn nicht 
sogar unumgänglich.

Der aktuelle Stand der Klassen und Pakete aus dem \TUDScript-Bundle in der 
\vTUDScript{Version} wurde nach bestem Wissen und Gewissen auf Herz und Nieren 
getestet. Dennoch kann nicht für das Ausbleiben von Fehlern garantiert werden. 
Beim Auftreten eines Problems sollte dieses genauso wie Inkompatibilitäten mit 
anderen Paketen an der zentralen Anlaufstelle
\begin{quoting}
\renewcommand*\href[2]{\url{#1}\quad(#2)}
\GitHubRepo<issues>%
\end{quoting}
gemeldet werden. Für eine schnelle und erfolgreiche Fehlersuche ist ein 
\href{http://www.komascript.de/minimalbeispiel}{\textbf{Minimalbeispiel}} 
bereitzustellen. Auf Anfragen ohne dieses werde ich gegebenenfalls verspätet 
oder gar nicht reagieren. Ebenso sind auch \emph{Fragen}, \emph{Kritik} und 
\emph{Verbesserungsvorschläge}~-- sowohl das \TUDScript-Bundle selbst als auch 
dessen Dokumentation betreffend~-- im \GitHubRepo gerne gesehen. Da \TUDScript 
in meiner Freizeit entstanden ist und gepflegt wird, bitte ich um Nachsicht, 
falls ich nicht sofort antworte und/oder eine Fehlerkorrektur vornehmen kann.

\bigskip
\noindent Falk Hanisch\newline
Dresden, \getfield{date}

\tableofcontents
\chapter{Einleitung}
%
Zur fehlerfreien Verwendung der \TUDScript-Klassen in der Version~\vTUDScript{} 
werden sowohl die \KOMAScript-Klassen der Version~\vKOMAScript{} oder später 
als auch die Hausschrift des \CDs \OpenSans aus dem Paket \Package{opensans} 
zwingend benötigt. Zusätzlich müssen weitere Pakete verfügbar sein, welche 
unter \autoref{sec:packages:needed} aufgeführt sind. Beim Einsatz einer 
aktuellen Version von \TeXLive|?|, \MacTeX|?| oder \MiKTeX|?| sollte dies kein 
Problem darstellen. Wird jedoch eine \DistributionGeneral verwendet, die 
\TUDScript in der Version~\vTUDScript{} nicht zur Verfügung stellt und eine 
Aktualisierung dieser nicht möglich sein, so sollte \autoref{sec:install:ext} 
konsultiert werden. In diesem Fall ist der Anwender selbst dafür 
verantwortlich, alle benötigten Pakete in der jeweils notwendigen Version 
bereitzustellen, wobei sämtliche Paketabhängigkeiten zu beachten sind. Dieses 
Vorgehen ist jedoch äußerst fehleranfällig, weshalb dringlich dazu geraten 
wird, eine aktuelle \DistributionGeneral zu verwenden.



\section{Bestandteile und Nutzung des \TUDScript-Bundles}
%
\ChangedAt{v2.01:\TUDScript-Bundle über das \CTAN<pkg/tudscr> veröffentlicht}%
\index{Aktualisierung}%
Das \TUDScript-Bundle wird über das \CTAN<pkg/tudscr> bereitgestellt und kann 
durch eine \DistributionGeneral wie \TeXLive, \MacTeX oder auch \MiKTeX genutzt 
werden. Es ist hauptsächlich für das Erstellen wissenschaftlicher Texte sowie 
Abhandlungen gedacht und stellt hierfür zum einen die drei Hauptklassen 
\Class{tudscrbook}, \Class{tudscrreprt} sowie \Class{tudscrartcl} und zum 
anderen die Klasse \Class{tudscrposter} zur Verfügung, welche in 
\autoref{sec:mainclasses} beziehungsweise \autoref{sec:poster} vorgestellt 
werden. Das Paket \Package{tudscrsupervisor}~-- in \autoref{sec:supervisor} 
dokumentiert~-- lässt sich zusätzlich in Verbindung mit diesen Klassen für die 
Erstellung von Aufgabenstellungen, Aushängen oder Gutachten zu studentischen 
Arbeiten nutzen. Weiterhin existieren auch eigenständige Pakete, welche in 
\autoref{sec:bundle} beschrieben sind. 

Für die Verwendung des \TUDScript-Bundles ist~-- neben \KOMAScript mindestens 
in der Version~\vKOMAScript{} sowie den in \autoref{sec:packages:needed} 
aufgeführten \hologo{LaTeX}"~Paketen~-- seit der Version~v2.06 lediglich die 
Schrift \OpenSans vonnöten, welche durch das Paket \Package{opensans} zur 
Verfügung gestellt wird. Eine lokale Nutzerinstallation der Schriften~-- wie in 
vorherigen Versionen~-- ist nicht notwendig. Lediglich für den Fall, dass 
gezielt die alten Schriften \Univers und \DIN eingesetzt werden sollen, müssen 
diese auch installiert sein. Weitere Hinweise zu deren Installation sowie 
Aktivierung sind in \autoref{sec:install:fonts} zu finden.

\minisec{Anmerkung zu Windows}
%
Unter Windows kann \TeXLive respektive \MiKTeX als \DistributionGeneral genutzt 
werden. Die Vorteile von \TeXLive liegen zum einen in der Wartung durch mehrere 
Autoren sowie der etwas früheren Verfügbarkeit aller Updates über das \CTAN. 
Zum anderen werden zusätzlich zu \hologo{LaTeX} ein \emph{Perl"~Interpreter} 
sowie \emph{Ghostscript} mitgeliefert, wodurch die Ad"~hoc"=Verwendung einiger 
Pakete wie beispielsweise \Package{glossaries} vereinfacht wird. Für \MiKTeX 
müssen diese externen Programme gegebenenfalls manuell installiert werden. 
Demgegenüber entfällt die alljährliche Neuinstallation, welche bei \TeXLive 
notwendig ist. Weiterhin werden zuvor nicht installierte jedoch benötigte 
\hologo{LaTeX}"~Pakete im Bedarfsfall~-- eine aktive Internetverbindung 
vorausgesetzt~-- automatisch nachinstalliert.

\minisec{Anmerkung zu Linux und OS~X}
%
Die Installation der \DistributionGeneral \TeXLive oder \MacTeX sollte direkt 
über die angebotenen Pakete (\mbox{\url{https://tug.org/texlive/}} oder 
\mbox{\url{https://tug.org/mactex/}}) und nicht über \Path{apt-get install} 
erfolgen. Damit wird sichergestellt, dass die aktuelle Variante der jeweiligen 
\DistributionGeneral genutzt wird.



\section{Zur Verwendung dieses Handbuchs}
%
Sämtliche neu definierten Optionen, Umgebungen und Befehle der Klassen und 
Pakete des \TUDScript-Bundles werden im Handbuch aufgeführt und beschrieben. Am 
Ende des Dokumentes befinden sich mehrere Indexe, die das Nachschlagen oder 
Auffinden dieser erleichtern sollen. Darin werden auch ausgewählte Optionen, 
Umgebungen und Befehle aufgeführt, welche nicht zu \TUDScript gehören und 
dennoch innerhalb dieses Handbuchs Erwähnung finden.

Die im Folgenden beschriebenen Optionen können~-- wie ein Großteil aller 
Einstellungen für \KOMAScript~-- als Schlüssel"=Wert"=Paare bei der Wahl der 
Dokumentklasse angegeben werden:
\Macro{documentclass|\OPValue{\PName{Schlüssel}=\PName{Wert}}\MPName{Klasse}}

Des Weiteren eröffnen die \KOMAScript-Klassen die Möglichkeit der späten 
Optionenwahl. Damit können Optionen nicht nur direkt beim Laden als sogenannte 
Klassenoptionen angegeben werden, sondern lassen sich auch noch innerhalb des 
Dokumentes nach dem Laden der Klasse ändern. Die \KOMAScript-Klassen sehen 
hierfür zwei Befehle vor. Mit \Macro{KOMAoptions|\MPName{Optionenliste}}
lassen sich beliebig viele Schlüsseln jeweils genau einen Wert zuweisen, 
\Macro{KOMAoption|\MPName{Option}\MPName{Werteliste}} erlaubt das gleichzeitige 
Setzen mehrere Werte für genau einen Schlüssel. Für die von \TUDScript 
\emph{zusätzlich} zur Verfügung gestellten Optionen werden äquivalent dazu die 
Befehle \Macro{TUDoptions|\MPName{Optionenliste}} und 
\Macro{TUDoption|\MPName{Option}\MPName{Werteliste}} definiert. Damit kann das 
Verhalten von Optionen im Dokument~-- innerhalb einer Gruppe auch lokal~-- 
geändert werden.

Bei der Beschreibung aller Optionen sind direkt neben dieser deren jeweilige 
Standardwerte mit \mbox{\enquote*{Voreinstellung: \PName*{Wert}}} angeführt. 
Einige dieser sind nicht immer gleich sondern werden in Abhängigkeit der 
genutzten Benutzereinstellungen und Optionen gesetzt. Diese bedingten 
Standardwerte werden mit 
\mbox{\enquote*{%
  Voreinstellung: \PName*{Wert}\,\textbar\,Bedingung: \PName*{bedingter Wert}%
}}
angegeben. Wird ein Schlüssel durch den Benutzer \emph{ohne} eine Wertzuweisung 
genutzt, so wird~-- falls vorhanden~-- ein vordefinierter Säumniswert gesetzt, 
welcher in der Beschreibung aller Optionen durch die~\PValue{\emph{kursive}} 
Schreibweise innerhalb der Werteliste gekennzeichnet ist. In den meisten Fällen 
ist der Säumniswert eines Schlüssels \PValue{true}, er entspricht folglich der 
Angabe \PName{Schlüssel}\PValue{=true}. Mit der expliziten Wertzuweisung eines 
Schlüssels werden sowohl einfache als auch bedingte Voreinstellungen in jedem 
Fall überschrieben. Die neben den Optionen neu eingeführten Umgebungen und 
Befehle der Klassen werden~-- gegebenenfalls zusammen mit den dazugehörigen 
optionalen Parametern~-- im gleichen Stil erläutert.



\section{Schnelleinstieg}
%
Das Handbuch gliedert sich in drei Teile. In \autoref{part:main} ist die 
Dokumentation von \TUDScript zu finden. Hier werden alle neuen Optionen, 
Umgebungen und Befehle, welche über die Funktionalität von \KOMAScript 
hinausgehen, erläutert. \autoref{part:additional} enthält zum einen einfache 
Minimalbeispiele, um den prinzipiellen Umgang und die Funktionalitäten von 
\TUDScript zu demonstrieren. Zum anderen werden hier auch ausführliche und 
dokumentierte Tutorials vor allem für \hologo{LaTeX}"=Neulinge angeboten. 
Insbesondere das Tutorial \Tutorial{treatise} ist mehr als einen Blick wert, 
wenn eine wissenschaftliche Arbeit mit \hologo{LaTeX} verfasst werden soll.
Abschließend werden verschiedenste Pakete vorgestellt, die nicht speziell für 
das \TUDScript-Bundle selber sondern auch für andere \hologo{LaTeX}"~Klassen
verwendet werden können und demzufolge für jeden Anwender interessant sein 
könnten. Außerdem werden hier einige Tipps \& Tricks beim Umgang mit 
\hologo{LaTeX} beschrieben, um kleinere oder größere Probleme zu lösen.

Die Klassen \Class{tudscrbook}, \Class{tudscrreprt} und \Class{tudscrartcl} 
sind Wrapper-Klassen der bekannten und korrelierenden \KOMAScript-Klassen 
\Class{scrbook}, \Class{scrreprt} sowie \Class{scrartcl} und können einfach 
anstelle deren verwendet werden, wobei ein Blick in das \scrguide sehr zu 
empfehlen ist~-- insbesondere wenn Sie noch keine oder nur wenig Erfahrung im 
Umgang mit den genannten Klassen haben. Auf diesen basierende Dokumente können 
durch das Umstellen der Dokumentklasse einfach in das \TUDCD überführt werden. 
Bei Fragestellungen bezüglich Layout, Schriften oder ähnlichem ist in jedem 
Fall ein weiterer Blick in das hier vorliegende Handbuch empfehlenswert.



\section{Identifikation von \TUDScript}
%
\begin{Entity}{\Bundle{tudscr}}
Im \TUDScript-Bundle gibt es neben den Klassen selbst auch noch zusätzliche 
Pakete. Ein Teil dieser Pakete~-- genauer \Package{tudscrsupervisor} und 
\Package{tudscrcomp}~-- sind ausschließlich mit den \TUDScript-Klassen nutzbar.
Andere wiederum~-- die beiden Pakete \Package{tudscrfonts}~(Schriften) und 
\Package{tudscrcolor}~(Farben) für Belange des \CDs sowie die davon vollkommen 
unabhängigen Pakete \Package{mathswap} und \Package{twocolfix}~-- können mit 
allen existierenden \hologo{LaTeX}"~Klassen genutzt werden. Sämtliche Klassen 
und Pakete aus dem \TUDScript-Bundle enthalten die folgenden Befehle, welche 
diese als dessen Bestandteil identifizieren.

\begin{Declaration}
  {\Macro{TUDScript}}
  [v2.04]
\printdeclarationlist
%
Diese Anweisung setzt das Logo respektive die Wortmarke \enquote{\TUDScript{}} 
in serifenloser Schrift und mit leichter Sperrung des in Versalien gesetzten 
Teils. Dieser Befehl wird von allen Klassen und Paketen des \TUDScript-Bundles 
definiert.
\end{Declaration}

\begin{Declaration}
  {\Macro{TUDScriptClassName}}
  [v2.04]
\printdeclarationlist
%
Die Bezeichnung der jeweiligen, im Dokument verwendeten \TUDScript-Klasse ist 
im Makro \Macro{TUDScriptClassName} abgelegt. Soll also in Erfahrung gebracht 
werden, ob~-- und wenn ja, welche~-- \TUDScript-Klasse verwendet wird, so kann 
einfach auf diese Anweisung getestet werden. \KOMAScript stellt zusätzlich 
noch die beiden Anweisungen \Macro*{KOMAClassName} und \Macro*{ClassName} 
bereit, welche den Namen der zugrundeliegenden \KOMAScript-Klasse sowie die 
durch diese ersetzte Standardklasse enthalten.
\end{Declaration}

\begin{Declaration}
  {\Macro{TUDScriptVersion}}
  [v2.04]
\begin{Declaration}
  {\Macro{TUDScriptVersionNumber}}
  [v2.05]
\printdeclarationlist
%
In \Macro{TUDScriptVersion} ist die Hauptversion von \TUDScript in der Form
\begin{quoting}
\PName{Datum}~\PName{Version}~\PValue{TUD-Script}
\end{quoting}
abgelegt. Die Version ist für alle Klassen und Pakete des \TUDScript-Bundles
gleich und kann nach dem Laden einer Klasse oder eines Paketes abgefragt 
werden. Beispielsweise wurde diese Anleitung mit \enquote{\TUDScriptVersion{}} 
erstellt.

Eventuell will der Anwender auf die aktuell verwendete Version von \TUDScript 
prüfen, um gegebenenfalls eigene Anpassungen in Abhängigkeit der verwendeten 
Version vorzunehmen. Hierfür kann \Macro{TUDScriptVersionNumber} verwendet 
werden. Darin ist alleinig die Versionsnummer enthalten. Die für das Handbuch 
verwendete Version lautet \enquote{\TUDScriptVersionNumber{}}.
\end{Declaration}
\end{Declaration}
\end{Entity}



\section{Weitere Klassen und Pakete für das \CD}
%
Für das Erstellen von Dokumenten im \TUDCD mit \hologo{LaTeX} existiert eine 
\hrfn{%
  https://tud.de/hilfe/kommunizieren-und-publizieren/cd/vorlagen/druck/latex%
}{Vielzahl von Paketen}. Das \TUDScript-Bundle soll diese nicht ersetzen, 
jedoch längerfristig sämtliche Variationen vereinheitlichen und mit einer 
konsistenten Benutzerschnittstelle ausstatten. 

Aktuell können ältere Dokumente, welche zuvor mit der Klasse \Class*{tudbook} 
und gegebenenfalls dem Paket \Package*{tudthesis} von Klaus~Bergmann gesetzt 
wurden, auf eine der \TUDScript-Klassen \Class{tudscrbook} beziehungsweise 
\Class{tudscrreprt} oder \Class{tudscrartcl} migriert werden. Bereits 
existierende Poster basierend auf \Class*{tudmathposter} und \Class*{tudposter} 
können ebenfalls auf die \TUDScript-Klasse \Class{tudscrposter} umgestellt 
werden. In allen Fällen kann dabei das Paket \Package'fullref*'{tudscrcomp} für 
einen einfacheren Umstieg zum Einsatz kommen. 

Für alle weiteren Klassen des Vorlagenbundles von Klaus~Bergmann~-- namentlich 
\Class*{tudletter}, \Class*{tudfax}, \Class*{tudhaus} und \Class*{tudform} 
sowie \Class*{tudbeamer}~-- werden durch \TUDScript \emph{momentan keine} 
äquivalenten Klassen angeboten. Sollten Sie auf deren Verwendung angewiesen 
sein, können Sie zumindest das Paket \Package'fullref*'{fix-tudscrfonts} 
verwenden, um die aktuellen Hausschriften zu aktivieren.

Eine anwenderfreundlich und vollständige Umsetzung des \CDs für Briefe und 
Geschäftsschreiben auf Basis von \KOMAScript ist bis jetzt leider noch nicht 
mit \TUDScript realisiert worden, soll jedoch langfristig erfolgen. Für das 
Erstellen von Briefen mit den \TUDScript-Klassen lässt sich allerdings relativ 
einfach das Paket \Package{scrletter} nutzen. Für Präsentationen im \TUDCD wird 
von \TUDScript bisher ebenfalls nichts angeboten, allerdings existieren für die 
\Class{beamer}"~Klasse bereits mehrere Stile, die im \GitHubRepo(tud-cd/tud-cd) 
zu finden sind. 




\setpartpreamble{%
  \begin{abstract}
    \hypersetup{linkcolor=red}
    \noindent Dies ist der Hauptteil der Dokumentation des \TUDScript-Bundles. 
    Hier findet der Anwender alle verfügbaren Optionen, Umgebungen und 
    Befehle, die über die Funktionalität von \KOMAScript{} hinausgehen.
  \end{abstract}
}%
\part{Das \TUDScript-Bundle\label{part:main}}
\chapter[%
  Die Hauptklassen \Class*{tudscrbook}, \Class*{tudscrreprt} und 
  \Class*{tudscrartcl}%
]{%
  Die Hauptklassen%
  \label{sec:mainclasses}%
  \index{Klassen|!}%
}

\begin{DeclareEntity*}{\Class{tudscrbook}}
\begin{DeclareEntity*}{\Class{tudscrreprt}}
\begin{DeclareEntity*}{\Class{tudscrartcl}}
\ChangedAt*{%
  v2.02:Umbenennung einiger Befehle für Kompatibilität mit anderen Paketen;%
  v2.03:Anpassungen interner Befehle an \KOMAScript-Version~v3.15%
}%
Es werden die drei neuen Hauptklassen
%
\begin{description}
\item \Class{tudscrbook}
\item \Class{tudscrreprt}
\item \Class{tudscrartcl}
\end{description}
%
eingeführt, welche auf den \KOMAScript-Klassen basieren und grundsätzlich alle 
deren bereitgestellte Optionen~-- beispielsweise 
\Option{parskip=\PName{Methode}}(\Bundle{koma-script})'none' für die 
Absatzeinstellungen oder 
\Option{BCOR=\PName{Längenwert}}(\Package{typearea})'none' zur Festlegung der 
Bindekorrektur~-- sowie Umgebungen und Befehle unterstützen. Zusätzlich zu den 
\KOMAScript"=Klassen werden weitere Pakete zwingend benötigt, welche unter 
\autoref{sec:packages:needed} aufgeführt sind und durch \TUDScript geladen 
werden.

Es sei hier abermals auf die \scrguide[Anwenderdokumentation von \KOMAScript] 
hingewiesen, viele der folgend beschriebenen Befehle und Optionen beziehen sich 
auf die darin vorgestellten Einstellungsmöglichkeiten. Die Anpassungen und 
Erweiterungen der \KOMAScript"=Klassen an das \CD und die neu definierten 
beziehungsweise geänderten oder erweiterten Befehle und Optionen werden im 
Folgenden erläutert.

\begin{Declaration}
  {\Macro{TUDoptions}[\MPName{Optionenliste}]}
\begin{Declaration}
  {\Macro{TUDoption}[\MPName{Option}\MPName{Werteliste}]}
\printdeclarationlist[Optionen|!]

Mit diesen beiden Befehlen besteht für die meisten der neuen Klassen- respektive
Paketoptionen die Möglichkeit, den Wert der Optionen noch nach dem Laden der 
Klasse beziehungsweise des Paketes zu ändern. Sie sind das Äquivalent zu den 
beiden \KOMAScript-Befehlen \Macro{KOMAoptions}(\Bundle{koma-script})'none' 
beziehungsweise \Macro{KOMAoption}(\Bundle{koma-script})'none'.

Man kann wahlweise mit der Anweisung \Macro{TUDoptions} die Werte mehrerer 
Optionen gleichzeitig ändern, wobei diese durch Kommata zu trennen sind. Dabei 
muss innerhalb der Optionenliste jede zu ändernde Option in der Form 
\PName{Option}\PValue{=}\PName{Wert} angegeben werden. Die meisten der 
\TUDScript-Optionen besitzen auch einen Standard- respektive Säumniswert. 
Werden diese in der Form \PName{Option} ohne die Angabe eines Wertes genutzt, 
so wird der jeweiligen Option einfach der vorgesehene Säumniswert zugewiesen.

Einige Optionen können gleichzeitig mehrere, sich ergänzende Werte besitzen. 
Für diese besteht die Möglichkeit, mit \Macro{TUDoption} einer einzelnen Option 
nacheinander eine Reihe von Werten zuzuweisen, wobei diese in der Werteliste 
durch Komma voneinander zu trennen sind.

Diese beiden Befehlen erlauben es im Bedarfsfall das Verhalten von einer Option 
oder mehreren Optionen im Dokument zu ändern. Werden diese Befehle in einer 
Umgebung oder Gruppe genutzt, bleiben die vorgenommenen Einstellungen innerhalb 
dieser lokal begrenzt.
\end{Declaration}
\end{Declaration}



\section{%
  Die Schriften des \CDs%
  \label{sec:fonts}%
  \index{Schriftart|?(}%
}

\ChangedAt*{%
  v2.00:Schriften~-- insbesondere für den mathematischen Satz~-- verbessert;%
  v2.01:Unterschneidung (Kerning) der Ziffern verbessert;%
}%
Das \TUDCD gibt die Verwendung der Schriftfamilie \OpenSans für den Fließtext 
vor, was in der Standardkonfiguration durch \TUDScript auch so umgesetzt wird. 
Da jedoch in längeren Texten die Verwendung einer Serifenschriften aus 
typografischer Sicht zu empfehlen ist, gibt es allerdings die Möglichkeit, die 
ursprünglich vorgesehenen Schriften des \CDs nicht zu laden und stattdessen die 
\hologo{LaTeX}"=Standardschriften beziehungsweise ein anderes Schriftpaket zu 
verwenden. Die Erläuterungen dazu sind in \autoref{sec:text} zu finden. Zur 
Verwendung der Schriften mit \hologo{LaTeXe}"=Klassen, welche nicht zum 
\TUDScript-Bundle gehören, lässt sich das Paket \Package{tudscrfonts} laden.

\ChangedAt{%
  v2.02;%
  v2.04:Einfachere Verwendung des Paketes \Package{fontspec};%
  v2.06:\OpenSans als neue Schrift im \TUDCD;%
}%
Durch das \CD werden keine Schriften für den Mathematiksatz festgelegt. Das ist 
insbesondere für sowohl mathematische Abhandlungen als auch ingenieur- und 
naturwissenschaftliche Dokumente nicht tragbar. Im Mathematikmodus werden 
deshalb die lateinischen Glyphen mithilfe des Paketes \Package{mathastext}
sowie die griechischen Lettern der \OpenSans genutzt. Zu Ergänzung kann für 
zusätzliche mathematische Symbole das Paket \Package{mdsymbol} geladen werden.

Diese Einstellung lässt sich deaktivieren, wodurch sich die Standardschriften 
oder gegebenenfalls die eines zusätzlichen Paketes für den mathematischen Satz 
nutzen lassen. Die dafür relevanten Einstellungen werden in \autoref{sec:math} 
erläutert. Weiterhin sind ergänzende Hinweise zu einem typografisch sauberen
Mathematiksatz in \autoref{sec:tut} zu finden.


\minisec{Die Schriftformate Type1 und OpenType}

\ChangedAt{v2.02:OpenType-Schriften mit Paket \Package{fontspec} verwendbar}%
Das \TUDScript-Bundle unterstützt die Nutzung der Schriften des \CDs sowohl 
im Type1- als auch im OpenType-Format. Beide Formate werden durch das Paket 
\Package{opensans} bereitgestellt, wobei unabhängig vom genutzten 
Textsatzsystem die Schriften im Type1-Format geladen werden. Die Verwendung der 
Schriften im OpenType-Format ist auch beim Einsatz von \Format{LuaLaTeX} oder 
\Format{XeLaTeX} nicht zwingend erforderlich. Dennoch ist dies problemlos 
möglich, indem einfach das Paket \Package{fontspec} eingebunden wird.

Ein Verzicht auf die Type1-Schriften ist dennoch nicht ohne Weiteres möglich. 
Denn einerseits sind diese für die Kompilierung eines Dokumentes über den 
klassischen Prozesspfad via \Path{latex\,>\,dvips\,>\,ps2pdf}~-- wie es 
beispielsweise für die Erstellung von Grafiken mit \Package{pstricks} notwendig 
ist~-- erforderlich. Andererseits werden für den Mathematiksatz die Glyphen den 
Type1-Schriften entnommen. Eine Verwendung der OpenType-Schriften für den 
Mathematiksatz ist leider nicht ohne einen erheblichen Aufwand möglich, weshalb 
\TUDScript~-- insbesondere aufgrund des nicht vorhandenen Mehrwerts~-- darauf 
verzichtet.



\subsection{%
  Schriften für den Textsatz%
  \label{sec:text}%
  \index{Schriftstärke|?(}%
}

\begin{Declaration}
  {\Option{cdfont=\PMisc}}
  (true,normalbold,liningfigures)
[%
  v2.02!\Option{cdfont=head};
  v2.02!\Option{cdfont=heavyhead};
  v2.03!\Option{cdfont=din};
  v2.03!\Option{cdfont=nodin};
  v2.05!\Option{cdfont=normalbold};
  v2.05!\Option{cdfont=ultrabold};
  v2.06!\Option{cdfont=liningfigures};
  v2.06!\Option{cdfont=oldstylefigures};
  v2.06!\Option{cdfont=standardgreek};
]
\printdeclarationlist

\ToDo{Indexeinträge an spezifische Werte knüpfen?}[v2.06]%
Mit der Option \Option{cdfont} können durch den Anwender alle zentralen 
Schrifteinstellungen für die \TUDScript-Klassen vorgenommen werden. Dies 
betrifft die Schriften sowohl für den Fließtext als auch den Mathematiksatz.
Dabei lassen sich insbesondere unterschiedliche Kombinationen von normaler und 
fetter Schriftstärke für die \OpenSans einstellen. Zu Beginn des Dokumentes 
sind diese maßgeblich für die Definition der Mathematikschriften. Die 
Schriftstärke im charakteristischen Querbalken der Kopfzeile lässt sich mit 
dieser Option ebenfalls einstellen.%
\index{Layout!Kopfzeile}%
\index{Layout!Überschriften}%
%
\begin{values}{\Option{cdfont}}
\itemfalse
  Es werden die \hologo{LaTeX}"=Standardschriften und nicht die Hausschrift 
  des \CDs verwendet. Der Anwender kann beliebige Schriftpakete nutzen.%
  \footnote{%
    Für die Verwendung der klassischen \hologo{LaTeX}"=Schriften, ist das Paket 
    \Package{lmodern} sehr empfehlenswert.%
  }
\itemtrue*[light/lightfont/noheavyfont]
  Es wird die Hausschrift \OpenSans im Stil des \CDs genutzt. Die Überschriften 
  der obersten Gliederungsebenen bis einschließlich \Macro*{subsubsection} 
  werden entweder in fetter oder extra"~fetter Schriftstärke gesetzt 
  (\seeref{\Option{headings=\PMisc}'ppage'}), darunter liegende%
  \footnote{%
    \Macro{paragraph}(\Bundle{koma-script})'none' und 
    \Macro{subparagraph}(\Bundle{koma-script})'none'%
  } verwenden immer den fetten Schriftschnitt. Dieser nutzt standardmäßig 
  \textcdrn{Open~Sans~Regular}, kann allerdings mit \Option{cdfont=heavybold} 
  stärker eingestellt werden. Im Fließtext kommt \textcdln{Open~Sans~Light} zum 
  Einsatz.
\item[heavy/heavyfont]
  Diese Einstellung unterscheidet sich von \Option{cdfont=true} insoweit als 
  die Schriftstärke der Hausschrift erhöht wird. Der Fließtext des Dokumentes 
  wird in \textcdrn{Open~Sans~Regular} gesetzt. Der fette Schriftschnitt ist im 
  Normalfall auf \textcdsn{Open~Sans~Semi"~Bold} festgelegt.
\end{values}
%
Die Stärke der fetten Schriften lässt sich mit folgenden Einstellungen anpassen.
%
\begin{values}{\Option{cdfont}}
\item[normalbold]
  \ChangedAt{v2.05}%
  Für die fette Schriftstärke wird \textcdrn{Open~Sans~Regular} respektive bei 
  stärkerer Grundschrift (\Option{cdfont=heavy}) \textcdsn{Open~Sans~Semi"~Bold}
  verwendet. Dies ist die Voreinstellung.
\item[heavybold/ultrabold]
  \ChangedAt{v2.05}%
  Die fetten Schriften werden stärker hervorgehoben. Es kommt 
  \textcdsn{Open~Sans~Semi"~Bold} respektive \textcdbn{Open~Sans~Bold} bei 
  erhöhter Schriftstärke (\Option{cdfont=heavy}) zum Einsatz.
\end{values}
%
Die folgende Tabelle verdeutlicht die Möglichkeiten zur Kombination der 
Einstellungen:
\begin{center}
\footnotesize%
\setlength{\tabcolsep}{.5em}%
\begin{tabular}{ccccc}
  \toprule
  & \multicolumn{2}{c}{\Option{cdfont=normalbold}}
  & \multicolumn{2}{c}{\Option{cdfont=heavybold}}
  \tabularnewline
  & \Macro{mdseries} & \Macro{bfseries} & \Macro{mdseries} & \Macro{bfseries}
  \tabularnewline\midrule
  \Option{cdfont=true} 
    & \textcdln{Open~Sans~Light} & \textcdrn{Open~Sans~Regular}
    & \textcdln{Open~Sans~Light} & \textcdsn{Open~Sans~Semi"~Bold}
  \tabularnewline\midrule
  \Option{cdfont=heavy}
    & \textcdrn{Open~Sans~Regular} & \textcdsn{Open~Sans~Semi"~Bold}
    & \textcdrn{Open~Sans~Regular} & \textcdbn{Open~Sans~Bold}
  \tabularnewline\bottomrule
\end{tabular}
\end{center}
\par\medskip\noindent
%
\index{Ziffernform}%
Weiterhin lässt sich die Gestalt der Ziffern anpassen, wobei zwischen 
Versalziffern und Mediävalziffern gewählt werden.
%
\begin{values}{\Option{cdfont}}
\item[liningfigures/normalfigures]
  \ChangedAt{v2.06}%
  Mit dieser Einstellung kommen dicktengleiche/äquidistant/proportionale 
  Versalziffern oder auch Majuskelziffern (\textbf{1234567890}) zum Einsatz.
  Mit \Macro{oldstylenums}[\MPName{Ziffer(n)}] lassen sich 
  Mediävalziffern oder auch Minuskelziffern explizit setzen. Dieses Verhalten 
  ist standardmäßig aktiviert.
\item[oldstylefigures/osf/oldnumbers]
  \ChangedAt{v2.06}%
  Für den Fließtext werden nichtproportionale Mediävalziffern oder auch 
  Minuskelziffern (\textbf{\oldstylenums{0123456789}}) standardmäßig verwendet.
\end{values}
%
Außerdem dient die Option \Option{cdfont=\PMisc} als erweiterte Schnittstelle 
für den Anwender, um zusätzliche Einstellungen für die Schriftnutzung vornehmen 
zu können, für welche normalerweise separate Optionen vorgesehen sind. Diese 
Möglichkeiten werden folgend kurz erläutert und auf die tatsächlich zugrunde 
liegende Option verweisen.

Die genutzten Mathematikschriften können mit folgenden Werte beeinflusst werden:
%
\begin{values}{\Option{cdfont}}
\item[nomath/nocdmath] 
  Diese Einstellung deaktiviert die \OpenSans für den mathematischen Satz. Es 
  werden die \hologo{LaTeX}"=Standardschriften verwendet und es lassen sich  
  beliebige Schriftpakete für den Mathematikmodus nutzen, 
  \seeref{\Option{cdmath=false}}.
\item[math/cdmath]
  Die \OpenSans wird für die Mathematikschriften genutzt, 
  \seeref{\Option{cdmath=true}}.
\item[upgreek/uprightgreek]
  Griechische Groß- und Kleinbuchstaben sind aufrecht, 
  \seeref{\Option{slantedgreek=false}}.
\item[slgreek/slantedgreek]
  Die Ausgabe der griechischen Glyphen erfolgt kursiv, 
  \seeref{\Option{slantedgreek=true}}.
\item[texgreek/standardgreek]
  \ChangedAt{v2.06}%
  Es kommt der \LaTeX-Standard zum Einsatz, bei dem griechische Majuskeln 
  aufrecht und Minuskeln kursiv ausgegeben werden, 
  \seeref{\Option{slantedgreek=standard}}.
\end{values}
%
\ChangedAt{v2.02}%
Für die \TUDScript-Klassen gibt es bei den \PageStyle{tudheadings}-Seitenstilen 
folgende Einstellmöglichkeiten für die Schriftart im Querbalken:%
\index{Layout!Kopfzeile}%
%
\begin{values}{\Option{cdfont}}
\item[nohead/noheadfont]
  Bei deaktivierter Hausschrift für den Fließtext können diese ebenfalls im 
  Querbalken deaktiviert werden, \seeref{\Option{cdhead=false}}.
\item[head/lighthead/lightfonthead/noheavyfonthead]
  Für den Querbalken der Kopfzeile wird~-- unabhängig von der Verwendung der 
  Hausschrift im Fließtext~-- \OpenSans in normaler Schriftstärke verwendet, 
  \seeref{\Option{cdhead=true}}.
\item[heavyhead/heavyfonthead]
  Im Querbalken wird~-- unabhängig von der Dokumentschrift~-- \OpenSans in 
  erhöhter Stärke genutzt, \seeref{\Option{cdhead=heavy}}.
\end{values}
\end{Declaration}

\begin{Declaration}
  {\Option{ttfont=\PMisc}}
  (true)
  [v2.06]
\printdeclarationlist

\ChangedAt*{%
  v2.06:Verwendung der \texttt{Roboto~Mono} als Schreibmaschinenschrift möglich
}%
Bei aktivierter Option \Option{cdfont} lässt sich zusätzlich die verwendete
\texttt{Schreibmaschinenschrift} einstellen, wofür selbige aus einem der Pakete 
\Package{roboto-mono} oder \Package{lmodern} genutzt wird und insbesondere die 
gewählten Einstellungen für die Schriftstärke der Option \Option{cdfont} 
beachtet werden. Ein direktes Laden des jeweiligen Paketes durch den Benutzer 
ist nicht notwendig.
%
\begin{values}{\Option{ttfont}}
\itemfalse
  Es findet keine Anpassung Schreibmaschinenschrift (\Macro*{ttfamily}) statt, 
  mögliche Änderungen können durch den Anwender erfolgen.
\itemtrue*[roboto/roboto-mono]
  Sollte das Paket \Package{roboto-mono} installiert sein, wird die darin 
  definierte Schreibmaschinenschrift genutzt. Diese ist serifenlos und liegt in 
  einer Vielzahl unterschiedlicher Schriftschnitte vor, weshalb sich diese 
  ideal mit der \OpenSans kombinieren lässt.
\item[lmodern/lmtt/lm]
  Es wird explizit die Schreibmaschinenschrift aus \Package{lmodern} aktiviert. 
\end{values}
\end{Declaration}


\minisec{%
  Auszeichnungen im Text%
  \index{Schriftauszeichnung|(}%
}

Unabhängig davon, welche Schriftfamilie im Dokument verwendet wird, können die 
Schriften des \CDs jederzeit mit einem der hier aufgeführten Textschalter oder 
Textkommandos innerhalb des Dokumentes genutzt werden. Ein Textschalter wirkt 
sich~-- wenn er nicht lokal durch eine Gruppe oder Umgebung begrenzt wird~-- 
global auf das Dokument aus, wie etwa beispielsweise \Macro*{bfseries}. Bei 
einem Textkommando hingegen kommt die Schriftart nur für das nachfolgend 
angegebene Argument zum Einsatz, wie bei \Macro*{textbf}[\MPName{Text}]. 
%
\begin{Declaration}
  {\Macro{cdfont}[\MPName{Schriftart}]}
  [v2.04]
\begin{Declaration}
  {\Macro{textcd}[\MPName{Schriftart}\MPName{Text}]}
  [v2.04]
\printdeclarationlist

Diese beiden Befehle dienen zur gezielten Aktivierung einer Schriftart des \CDs 
in Stärke und Schnitt. Hierbei entspricht \Macro{cdfont} einem Textschalter und 
ändert die verwendete Schriftart unverzüglich im aktuellen Geltungsbereich auf 
\PName{Schriftart}, wohingegen \Macro{textcd} als Textkommando fungiert und den 
im zweiten Argument gegebenen \PName{Text} in \PName{Schriftart} setzt ohne 
dabei die Dokumentschriftart selbst zu ändern.

Für die Schriftauswahl muss im ersten Argument die Bezeichnung der gewünschten 
Schriftart angegeben werden. Mögliche Werte sind der nachfolgenden Tabelle zu 
entnehmen. Der Vorsatz \PValue{Open Sans} für die \PName{Schriftart} im 
Argument ist dabei optional. Ebenso sind weder Leerzeichen noch die passende 
Groß- und Kleinschreibung notwendig. Für die Wahl der Schriftstärke ist allein 
die Bezeichnung \PValue{Light/Regular/Semi-Bold/Bold/Extra-Bold} ausreichend. 
Anstelle des Suffix' \PValue{Italic} ist auch die Nutzung von \PValue{Oblique} 
oder \PValue{Slanted} als Alias für die kursiven Schriftlagen möglich.
%
\begin{center}%
  \newcommand*\listfonts[2]{%
    \csuse{textcd#2}{Open Sans #1} 
      & \InlineDeclaration{\Macro{cdfont#2}}
      & \InlineDeclaration{\Macro{textcd#2}[\MPName{Text}]}
    \tabularnewline
  }%
  \begin{tabular}{lll}%
    \toprule%
    \textbf{Schriftart/Bezeichnung} & \textbf{Schalter} & \textbf{Textkommando} 
    \tabularnewline
    \midrule
    \listfonts{Light}{ln}
    \listfonts{Regular}{rn}
    \listfonts{Semi-Bold}{sn}
    \listfonts{Bold}{bn}
    \listfonts{Extra-Bold}{xn}
    \listfonts{Light Italic}{li}
    \listfonts{Regular Italic}{ri}
    \listfonts{Semi-Bold Italic}{si}
    \listfonts{Bold Italic}{bi}
    \listfonts{Extra-Bold Italic}{xi}
    \bottomrule%
  \end{tabular}%
\end{center}%
%
Alternativ zu \Macro{cdfont} und \Macro{textcd} werden für jeden Schriftschnitt 
auch explizite Textschalter und "~kommandos bereitgestellt, bei denen das 
Argument für die Bezeichnung der Schriftart entfällt. Diese sind in der zweiten 
und dritten Spalte der Tabelle aufgeführt.
\end{Declaration}
\end{Declaration}
\index{Schriftauszeichnung|)}%
\index{Schriftstärke|?)}%
\index{Schriftart|?)}%



\subsection{%
  Schriften für den Mathematiksatz%
  \label{sec:math}%
  \index{Mathematiksatz|?(}%
}

\begin{Declaration}
  {\Option{cdmath=\PBoolean}}
  (true|\Option{cdfont=false}:false)
  [v2.03]
\printdeclarationlist

Diese Option dient zur Anpassung der Mathematikschriften. Wird diese aktiviert, 
so werden zur Hausschrift passende Glyphen im Mathematikmodus genutzt. Normale 
sowie fette Schriftstärke werden zu \emph{Beginn des Dokuments} abhängig von 
der zu diesem Zeitpunkt aktiven Einstellung für die Schriften des Fließtexts 
(\seeref{\Option{cdfont}}) definiert. Auf fette Mathematikschriften kann im 
Dokument mit \Macro{boldmath} umgeschaltet werden, eine Anpassung der 
Einstellungen ist durch \Macro{TUDoptions} möglich. Gültige Werte für die 
Option \Option{cdmath} sind:
%
\begin{values}{\Option{cdmath}}
\itemfalse
  Es werden die normalen \hologo{LaTeX}"=Serifenschriften beziehungsweise die 
  Schriften beliebig nutzbarer Pakete für den Mathematiksatz verwendet.
\itemtrue*
  Im Mathematikmodus kommt \OpenSans sowohl für lateinische als auch 
  griechische Buchstaben zum Einsatz.
\end{values}
\end{Declaration}

\begin{Declaration}
  {\Option{slantedgreek=\PBoolean}}
  (true)
  [v2.06!\Option{slantedgreek=standard}]
\printdeclarationlist[Griechische~Buchstaben]

Die Option ändert die standardmäßige Neigung der griechischen Großbuchstaben im 
Mathematikmodus bei der Verwendung der Standardbefehle für griechische 
Buchstaben.
%
\begin{values}{\Option{slantedgreek}}
\itemfalse
  Die griechischen Glyphen werden im Mathematiksatz allesamt aufrecht gesetzt.
\itemtrue*
  Alle griechischen Buchstaben werden im Mathematikmodus kursiv ausgegeben.
\item[standard/latex]
  Die Ausgabe entspricht mit aufrechten Majuskeln und kursiven Minuskeln der 
  griechischen Lettern dem \LaTeX-Standard beim mathematischen Satz.
\end{values}
\end{Declaration}

\begin{Declaration}{\Macro{Gamma}}
\begin{Declaration}{\Macro{Delta}}
\begin{Declaration}{\Macro{Theta}}
\begin{Declaration}{\Macro{Lambda}}
\begin{Declaration}{\Macro{Xi}}
\begin{Declaration}{\Macro{Pi}}
\begin{Declaration}{\Macro{Sigma}}
\begin{Declaration}{\Macro{Upsilon}}
\begin{Declaration}{\Macro{Phi}}
\begin{Declaration}{\Macro{Psi}}
\begin{Declaration}{\Macro{Omega}}
\begin{Declaration}{\Macro{alpha}}
\begin{Declaration}{\Macro{beta}}
\begin{Declaration}{\Macro{gamma}}
\begin{Declaration}{\Macro{delta}}
\begin{Declaration}{\Macro{epsilon}}
\begin{Declaration}{\Macro{varepsilon}}
\begin{Declaration}{\Macro{zeta}}
\begin{Declaration}{\Macro{eta}}
\begin{Declaration}{\Macro{theta}}
\begin{Declaration}{\Macro{vartheta}}
\begin{Declaration}{\Macro{iota}}
\begin{Declaration}{\Macro{kappa}}
\begin{Declaration}{\Macro{lambda}}
\begin{Declaration}{\Macro{mu}}
\begin{Declaration}{\Macro{nu}}
\begin{Declaration}{\Macro{xi}}
\begin{Declaration}{\Macro{pi}}
\begin{Declaration}{\Macro{varpi}}
\begin{Declaration}{\Macro{rho}}
\begin{Declaration}{\Macro{varrho}}
\begin{Declaration}{\Macro{sigma}}
\begin{Declaration}{\Macro{varsigma}}
\begin{Declaration}{\Macro{tau}}
\begin{Declaration}{\Macro{upsilon}}
\begin{Declaration}{\Macro{phi}}
\begin{Declaration}{\Macro{varphi}}
\begin{Declaration}{\Macro{chi}}
\begin{Declaration}{\Macro{psi}}
\begin{Declaration}{\Macro{omega}}
\settowidth\tempdim{\Macro{varepsilon}}%
\addtolength\tempdim{\dimexpr 2\tabcolsep+2\arrayrulewidth-\textwidth\relax}%
\printdeclarationlist[Griechische~Buchstaben](%
  \begin{minipage}{-\tempdim}%
    \small%
    \newcommand\tablecontent{}%
    \newcommand*\greekLetters{%
      Gamma,Delta,Theta,Lambda,Xi,Pi,Sigma,Upsilon,Phi,Psi,Omega,%
      alpha,beta,gamma,delta,epsilon,varepsilon,zeta,eta,theta,vartheta,%
      iota,kappa,lambda,mu,nu,xi,pi,varpi,rho,varrho,sigma,varsigma,tau,%
      upsilon,phi,varphi,chi,psi,omega%
    }%
    \def\do#1{%
      \appto\tablecontent{%
%        \InlineDeclaration{\Macro{up#1}}
        \Macro*{up#1}'none'%
        & $\csuse{up#1}$ & & 
%        \InlineDeclaration{\Macro{it#1}}
        \Macro*{it#1}'none'%
        & $\csuse{it#1}$\tabularnewline
      }%
    }%
    \expandafter\docsvlist\expandafter{\greekLetters}%
    \centering%
    \vspace{\intextsep}\noindent
    \begin{tabularm}{5}
      \toprule%
      \textbf{Befehl (aufrecht)} & \textbf{Symbol} & &
      \textbf{Befehl (kursiv)} & \textbf{Symbol}
      \tabularnewline\midrule\tablecontent\bottomrule%
    \end{tabularm}
  \end{minipage}%
)

Unabhängig von der Option \Option{slantedgreek} können sowohl kursive als auch 
aufrechte griechischen Glyphen im Mathematikmodus mit diesen Befehlen direkt 
verwendet werden. Dies ist nützlich, um zwischen \emph{kursiven Variablen} und 
\emph{aufrechten Konstanten} zu unterscheiden. Weiterhin kann anstelle von 
\Macro*{up\MPName{Glyphe}} oder \Macro*{it\MPName{Glyphe}} auch der 
Präfix \Macro*{other\MPName{Glyphe}} genutzt werden. Damit wird~-- abhängig 
von der Option \Option{slantedgreek}~-- die komplementäre Schriftlage der 
angegebenen Glyphe gesetzt.
\end{Declaration}
\end{Declaration}
\end{Declaration}
\end{Declaration}
\end{Declaration}
\end{Declaration}
\end{Declaration}
\end{Declaration}
\end{Declaration}
\end{Declaration}
\end{Declaration}
\end{Declaration}
\end{Declaration}
\end{Declaration}
\end{Declaration}
\end{Declaration}
\end{Declaration}
\end{Declaration}
\end{Declaration}
\end{Declaration}
\end{Declaration}
\end{Declaration}
\end{Declaration}
\end{Declaration}
\end{Declaration}
\end{Declaration}
\end{Declaration}
\end{Declaration}
\end{Declaration}
\end{Declaration}
\end{Declaration}
\end{Declaration}
\end{Declaration}
\end{Declaration}
\end{Declaration}
\end{Declaration}
\end{Declaration}
\end{Declaration}
\end{Declaration}
\end{Declaration}


\minisec{Zusätzliche Hinweise zum Mathematiksatz}

Unter Umständen werden zusätzliche Symbole für den Mathematiksatz benötigt. 
Sehr oft kommt hierfür das Paket \Package{amssymb} zum Einsatz. Dieses stellt 
zahlreiche zusätzliche für die \hologo{LaTeX}"=Standardschriften zur Verfügung. 
Diese sind in Verbindung mit der \OpenSans aus typografischer Sicht jedoch 
keine ideale Lösung. Die Symbole aus dem Paket \Package{mdsymbol} passen 
wesentlich besser zu besagter Schriftfamilie, weshalb diesem Paket der Vorzug 
gegeben werden sollte, falls die Schriften des \CDs für den mathematischen Satz 
genutzt werden. Weitere Hinweise zum typografisch guten Mathematiksatz sind 
außerdem in \autoref{sec:tut} zu finden.%
\index{Mathematiksatz|?)}%



\subsection{%
  Vertikaler Leerraum in Abhängigkeit der Schriftgröße%
  \index{Abstände}%
  \index{Längen}%
  \index{Schriftgröße}%
}

Bei den \TUDScript-Klassen sind im Normalfall mehrere Längen von der für das 
Dokument gewählten Schriftgröße abhängig~-- im Gegensatz zu \KOMAScript. Dies 
hat den großen Vorteil, dass bei einer Änderung der Schriftgröße die folgend 
genannten Befehle respektive Längen nicht separat durch den Anwender angepasst 
werden müssen, um weiterhin sinnvoll verwendet werden zu können.

Die Anpassung an die Schriftgröße erfolgt sowohl die dehnbaren Längen 
\Length{bigskipamount}, \Length{medskipamount} und \Length{smallskipamount}, 
welche von den Befehlen \Macro{bigskip}, \Macro{medskip} beziehungsweise 
\Macro{smallskip} für das Einfügen vertikaler Abstände genutzt werden, als auch 
die beiden Längen \Length{abovecaptionskip}(\Bundle{koma-script})'none' und 
\Length{belowcaptionskip}(\Bundle{koma-script})'none' für den Abstand zwischen 
einem Gleitobjekt~-- beispielsweise eine Abbildung oder eine Tabelle~-- und 
dessen mit \Macro{caption}(\Bundle{koma-script},\Package{caption})'none' 
gesetzten Beschreibung. Außerdem wird die Länge \Length{columnsep} als Maß für 
den horizontalen Abstand der einzelnen Textspalten im zwei- oder mehrspaltigen 
Layout, wie es beispielsweise mit dem Paket \Package{multicol} erzeugt werden 
kann, in Relation zur Schriftgröße sinnvoll festgelegt.

Die verwendete Schriftgröße kann durch den Anwender über die \KOMAScript-Option 
\Option{fontsize=\PName{Schriftgröße}}(\Bundle{koma-script})'none' festgelegt 
werden.
\Attention{%
  Dabei ist zu beachten, dass diese immer als Klassenoption angegeben werden 
  sollte.%
}
Weitere Hinweise zur Wahl der passenden Schriftgröße sind außerdem in 
\autoref{sec:fontsize} zu finden.

\begin{Declaration}
  {\Option{relspacing=\PBoolean}}
  (true)
  [v2.05]
\printdeclarationlist

Mit der Option \Option{relspacing=\PBoolean} lässt sich die zuvor beschriebene 
Schriftgrößenabhängigkeit sowohl für vertikalen Leerraum zwischen zwei Absätzen 
oder bei Beschriftungen als auch für den horizontalen Abstand zwischen den 
Textspalten im mehrspaltigen Layout anpassen.
%
\begin{values}{\Option{relspacing}}
\itemfalse[absolute/standard]
  Die besagten Längen werden nicht angepasst, passende Werte sollten bei einer 
  Änderung der Schriftgröße durch den Anwender gewählt werden.
\itemtrue*[relative/fontsize]
  In Abhängigkeit von der gewählten Schriftgröße werden die zuvor genannten 
  Längen automatisch festgelegt.
\end{values}
\end{Declaration}



\section{%
  Das Layout des \CDs%
  \index{Layout|(}%
}
\subsection{%
  Der Satzspiegel%
  \index{Satzspiegel|?(}%
}

\begin{Declaration}
  {\Option{cdgeometry=\PMisc}}
  (true,restricted|\Option{cd=false}:false)
[%
  v2.03;
  v2.05!\Option{cdgeometry=restricted};%
  v2.05!\Option{cdgeometry=adapted};%
  v2.05!\Option{cdgeometry=calculated};%
  v2.05!\Option{cdgeometry=custom};%
]
\printdeclarationlist[Satzspiegel!doppelseitig]

Diese Option ist für die Aufteilung beziehungsweise die Berechnung des 
Satzspiegels verantwortlich. Das Maß der Seitenränder ist im \CD fest 
vorgegeben und wird standardmäßig von den \TUDScript-Klassen eingehalten. 
Allerdings lassen sich die Seitenränder anpassen, um beispielsweise einen 
vernünftigen doppelseitigen Satz zu ermöglichen.%
\footnote{Hierbei sollte der innere Rand schmaler als der äußere sein}
Des Weiteren besteht die Möglichkeit, auf das \KOMAScript"=Standardverhalten 
zurückzufallen und die Satzspiegelberechnung durch das Paket \Package{typearea} 
vornehmen zu lassen. Hier hat insbesondere die Klassenoption 
\Option{DIV=\PMisc}(\Package{typearea})'none'|declare| maßgeblichen Einfluss auf
den Satzspiegel. Mehr dazu ist in der \scrguide[Dokumentation von \KOMAScript] 
zu finden.
\ChangedAt{v2.05}%
Zusätzlich lässt sich mit \Option{cdgeometry=custom} der Satzspiegel durch den 
Anwender (fast) beliebig festlegen.
%
\begin{values}{\Option{cdgeometry}}
\itemfalse
  Die Satzspiegelberechnung erfolgt via \Package{typearea}, die Vorgaben des 
  \CDs bezüglich der Seitenränder werden ignoriert.
\itemtrue*[asymmetric/cd]
  Die Seitenränder werden im asymmetrischen Stil des \CDs fest definiert und 
  auch für den doppelseitigen Satz
  (Klassenoption \Option{twoside=true}(\Package{typearea})'none') genutzt.%
  \footnote{links: 30\,mm, rechts: 20\,mm, oben: 20\,mm, unten: 30\,mm}
\item[symmetric/centred/centered]
  Der Satzspiegel wird im einseitigen sowie doppelseitigen Satz auf der Seite 
  zentriert.%
  \footnote{links: 25\,mm, rechts: 25\,mm, oben: 20\,mm, unten: 30\,mm}
\item[twoside/balanced]
  Diese Einstellung aktiviert den doppelseitigen Satz 
  (\Option{twoside=true}(\Package{typearea})'none') und ändert den Satzspiegel 
  derart, dass die Ränder der inneren Seiten schmaler sind als die der äußeren.%
  \footnote{innen: 20\,mm, außen: 30\,mm, oben: 20\,mm, unten: 30\,mm}
  \Attention{%
    Der so erzeugte Satzspiegel ist jedoch unvorteilhaft, da das Logo der \TnUD 
    sehr nah am inneren Seitenrand des Dokumentes gesetzt wird und folglich auf 
    rechten respektive ungeraden Seiten sehr weit an den Blattrand rückt.
  }
  Dieses Problem kann~-- bei \Class{tudscrbook} sowie \Class{tudscrreprt}~-- 
  prinzipiell gelöst werden, indem Titel, Teile und Kapitel über das Aktivieren 
  der \KOMAScript-Option \Option{open=left}(\Bundle{koma-script})'none' immer 
  auf einer linken Seite beginnen, was allerdings aus typografischer Sicht eher 
  unüblich ist.
\end{values}
%
\ChangedAt{v2.05:Neue Möglichkeiten bei der Satzspiegelberechnung im \CD}%
Mit den folgenden Werten lässt sich einstellen, in welcher Variante der 
Satzspiegel nach dem \TUDCD erstellt werden soll.
%
\begin{values}{\Option{cdgeometry}}
\item[restricted]
  Der Satzspiegel entspricht den expliziten Vorgaben des \CDs.
\item[adapted]
  Laut dem Handbuch zum \CD werden für Papierformate zwischen DIN~A6 und DIN~A4 
  \enquote{im Interesse größter Einheitlichkeit die Maßverhältnisse über einen 
  größeren Formatbereich hinweg \enquote{eingefroren}.} Dies kann jedoch zu 
  schlecht nutzbaren Satzspiegeln führen. Mit dieser Einstellung kann die 
  Fixierung deaktiviert und der äquivalente Satzspiegel beispielsweise für das 
  Format~DIN~A5 aktiviert werden. Bei Formaten außerhalb des fixierten Bereichs 
  hat diese Einstellung keinen Einfluss. 
\item[calculated]
  Der Satzspiegel wird anhand der Referenzmaße für das Format~DIN~A4 für das 
  eingestellte Papierformat \emph{skaliert}. Die eigentlich definierten 
  diskreten Maße bei unterschiedlichen Gestaltungshöhen werden ignoriert.
\end{values}
%
\ChangedAt{v2.05}%
Da es häufig sehr restriktive~-- wenn auch meistens völlig willkürliche~-- 
Vorgaben für die Seitenränder gibt, besteht außerdem die Möglichkeit, diese 
weitestgehend manuell einzustellen. 
\begin{values}{\Option{cdgeometry}}
\item[custom]
  Für die Festlegung der Seitenränder ist das Paket \Package{geometry} 
  maßgeblich verantwortlich. Der Anwender kann mit den beiden, durch das Paket 
  bereitgestellten Befehlen \Macro{geometry}(\Package{geometry})'none' und 
  \Macro{newgeometry}(\Package{geometry})'none' den Satzspiegel selbst 
  definieren. Für Hinweise zur Verwendung der Befehle sollte die Dokumentation 
  von \Package{geometry} genutzt werden.
\end{values}
\end{Declaration}

\begin{Declaration}
  {\Option{extrabottommargin=\PName{Höhe}}}
  (0pt)
  [v2.03]
\printdeclarationlist

Mit dieser Option kann die Größe des unteren Seitenrandes angepasst werden, 
wenn der Satzspiegel des \CDs verwendet wird (\seeref{\Option{cdgeometry}}).
Insbesondere für den Fall, dass bei Seiten im Stil \PageStyle{tudheadings} 
im Fußbereich entweder mit \Macro{footlogo} Drittlogos verwendet werden und 
diese über das optionale Argument oder via \Option{footlogoheight} über die 
Standardhöhe hinaus vergrößert wurden oder mit \Macro{footcontent} ein 
übergroßer Inhalt angegeben wurde, kann dieser unter Umständen etwas zu klein 
sein. Mit der Option \Option{extrabottommargin} wird der Fußbereich durch 
positive Werte vergrößert, negative Werte verkleinern diesen entsprechend. 

Alternativ zur Option \Option{extrabottommargin} kann auch die Einstellung 
\Option{cdfoot=\PSet{Höhe}} mit einem Längenwert verwendet werden. 
Dabei spielt es für beide Optionen keine Rolle, ob eine \hologo{LaTeX}"=Länge, 
ein \hologo{TeX}"=Abstand oder eine \hologo{TeX}"=Ausdehnung als Länge bei der 
Wertzuweisung verwendet wird.
\end{Declaration}


\minisec{%
  Bindekorrektur%
  \index{Satzspiegel!Bindekorrektur|?}%
}

Im Zusammenhang mit den Seitenrändern oder besser dem Satzspiegel ist die durch 
das Paket \Package{typearea} zur Verfügung gesellte \KOMAScript-Option 
\Option{BCOR=\PName{Längenwert}}(\Package{typearea})'none'|declare| zu 
erwähnen. Mit dieser kann bei der Satzspiegelberechnung ein Heftrand respektive 
eine Bindekorrektur berücksichtigt werden. Durch die \TUDScript-Klassen wird 
der mit dieser Option angegebene Wert auch an das Paket \Package{geometry} 
weitergereicht, sodass der Benutzer unabhängig von der Satzspiegelgestaltung 
(\Option{cdgeometry}) die Option \Option{BCOR}(\Package{typearea})'none' nutzen 
kann. So lässt sich eine Bindekorrektur von beispielsweise 5\,mm mit der 
\emph{Klassenoption} \Option*{BCOR=5mm} festlegen.

Eine Anpassung der Bindekorrektur hat natürlich \emph{immer} eine Änderung der 
verfügbaren Breite des Textbereichs zur Folge hat und führt somit zwingend zu 
einer Anpassung des Satzspiegels. Da die Bindekorrektur jedoch abhängig von der 
Höhe des Buchblocks gewählt werden sollte, welche letztendlich erst mit dem 
Druck des fertiggestellten Dokumentes bestimmt werden kann, muss diese zu 
Beginn abgeschätzt werden.
%
\begin{Example}
Als Faustregel gilt, dass die erforderliche Bindekorrektur in etwa der halben 
Höhe des Buchblocks entsprechen sollte. Dessen Höhe wiederum ist abhängig von 
der Anzahl der Seiten sowie der Dichte des verwendeten Papiers. Wird normales 
Papier mit einer Dichte von 80\,g/m² verwendet, so entsprechen 100~Blatt in 
etwa einer Höhe von 10\,mm, bei 100\,g/m² ca. 12\,mm. Dementsprechend wäre eine 
Bindekorrektur von \Option*{BCOR=5mm} beziehungsweise \Option*{BCOR=6mm} bei 
diesem Beispiel zu wählen.%
\end{Example}


\minisec{%
  Kopf"~ und Fußzeile im Zusammenspiel mit dem Satzspiegel%
  \index{Layout!Kopfzeile}%
  \index{Layout!Fußzeile}%
}

Da im \CD nicht festgelegt ist, wie die Gestaltung der Kopf"~ und Fußzeilen in 
einer wissenschaftlichen Arbeit auszuführen ist, bleibt dem Nutzer dafür eine 
gewisse Freiheit. Dafür sollte idealerweise das zu \KOMAScript{} gehörige Paket 
\Package{scrlayer-scrpage} genutzt werden. 

In der Dokumentation zu \Package{typearea} wird auch darauf eingegangen, wann 
Kopf"~ und Fußzeile bei der Satzspiegelkonstruktion entweder dem Rand oder dem 
Textkörper zugeschlagen werden sollten. Dies sollte bei der Erstellung eigener 
Kopf"~ und Fußzeilen beachtet werden. Die Einstellung dafür erfolgt mit den 
beiden \KOMAScript"=Optionen 
\Option{headinclude=\PBoolean}(\Package{typearea})'none'|declare| sowie 
\Option{footinclude=\PBoolean}(\Package{typearea})'none'|declare|.
Diese können~-- unabhängig von der Einstellung zur Satzspiegelgestaltung durch 
die Option \Option{cdgeometry}~-- verwendet werden.%
\index{Satzspiegel|?)}%



\subsection{%
  Die Gestalt von Titel, Umschlagseite und Gliederungsebenen%
  \index{Seitenstil|(}%
  \index{Layout!Farbgestaltung|?(}%
  \index{Layout!Überschriften|(}%
  \index{Layout!Kopfzeile|(}%
}

Falls die nachfolgend beschriebene Option \Option{cd=\PMisc} aktiviert ist, 
werden einige, spezielle Seiten im prägnanten Stil mit dem Logo der \TnUD und 
der dazugehörigen Kopfzeile mit Querbalken gesetzt. Dies betrifft insbesondere 
\hyperref[sec:title]{die Umschlagseite und den Titel aus \autoref{sec:title}}, 
die \hyperref[sec:part]{Teileseiten in \autoref{sec:part}} sowie die
\hyperref[sec:chapter]{Kapitelseiten in \autoref{sec:chapter}}. Mit den 
\PageStyle{tudheadings}"=Seitenstilen oder der \Environment{tudpage}"=Umgebung 
können weitere Seiten in diesem Stil erzeugt werden. Wird das Paket 
\Package{tudscrsupervisor} verwendet und mit den bereitgestellten Befehlen oder 
Umgebungen eine Aufgabenstellung, ein Gutachten oder ein Aushang erstellt, so 
erscheinen auch diese in besagtem Seitenstil.
%
\begin{Declaration}
  {\Option{cd=\PMisc}}
  (true)
[%
  v2.03!\Option{cd=bicolor}:%
    Farbeinsatz nur im Kopf mit farbig abgesetztem Querbalken;
  v2.03!\Option{cd=fullcolor}:%
    voller Farbeinsatz mit farbig abgesetztem Querbalken;%
  v2.04!\Option{cd=barcolor}:nur farbig abgesetzter Querbalken;%
]
\printdeclarationlist[Titel|?,Umschlagseite|?,Teil|?,Kapitel|?]

Mit dieser Option wird festgelegt, ob und wie das \TUDCD im gesamten Dokument 
verwendet wird. Sie hat Einfluss auf die Ausprägung von Titel"~, Teil"~, und 
Kapitelseiten sowie die Überschriften der weiteren Gliederungsebenen.
%
\begin{values}{\Option{cd}}
\itemfalse
  Diese Einstellung erzeugt das Standard"=Verhalten der \KOMAScript"=Klassen, 
  es wird kein \CD genutzt.
\itemtrue*[nocolor/monochrome]
  Das Layout für Titel"~, Teil"~ und Kapitelseiten ist im \CD, es wird 
  schwarze Schrift für Titel, Teil"~ und Kapitelüberschriften verwendet. Die 
  Ausprägung des Seitenkopfes ist abhängig von der Option \Option{cdhead}.
\item[lightcolor/pale]
  Die Einstellung entspricht weitestgehend der Option \Option{cd=true}, 
  allerdings wird die primäre Hausfarbe \Color{HKS41} für den Kopf des 
  \PageStyle{tudheadings}"=Seitenstils und Überschriften genutzt.
\item[barcolor]
  \ChangedAt{v2.04}%
  Zusätzlich zur vorherigen Einstellung wird außerdem der Querbalken farbig 
  abgesetzt.
\item[bicolor/bichrome]
  \ChangedAt{v2.03}%
  Der Kopf wird mit einem farbigen Hintergrund in der Hausfarbe gesetzt, auch 
  der Querbalken wird farbig hinterlegt. Für die Überschriften wird die 
  primären Hausfarbe verwendet.
\item[color]
  Der Titel sowie Teil"~ und Kapitelseiten werden allesamt farbig gestaltet, 
  der Seitenkopf wird in der primären Hausfarbe \Color{HKS41} gesetzt, der 
  Querbalken erhält Linien als Begrenzung.
\item[fullcolor/full]
  \ChangedAt{v2.03}%
  Entspricht der vorherigen Einstellung, allerdings wird der Querbalken nicht 
  durch Linien begrenzt sondern farbig hinterlegt.
\end{values}
\end{Declaration}
\index{Layout!Kopfzeile|)}%

\begin{Declaration}
  {\Option{headings=\PMisc}}
  (heavy)
  [v2.06]
\printdeclarationlist[Schriftstärke]

Die Option \Option{headings=\PMisc}(\Bundle{koma-script})'none' wird bereits 
von \KOMAScript{} definiert und dient unter anderem zum Festlegen der Größe von 
Überschriften. Diese wird von \TUDScript erweitert, um im Layout des \CDs die 
Schriftstärke der Überschriften anpassen zu können.
%
\begin{values}{\Option{headings}}
\item[light/normalbold]
  Die Überschriften werden in fetter Schrift gesetzt.
\item[heavy/ultrabold]
  Für Überschriften wird die extra"~fette Schriftstärke verwendet.
\end{values}
%
Die Einstellungen gelten für die Gliederungsebenen bis einschließlich 
\Macro*{subsubsection}.
\end{Declaration}



\subsubsection{Individuelle Einstellungen für einzelne Elemente des Layouts}

Das Verhalten aller für das Layout relevanten Elemente wird von der eben zuvor 
erläuterten Option \Option{cd=\PMisc} bestimmt. Dies betrifft zum einen sowohl 
den Titel~(\Macro{maketitle}) als auch die Umschlagseite~(\Macro{makecover}) 
und zum anderen alle Teileseiten
(%
  \Macro{part}(\Bundle{koma-script})'none', 
  \Macro{addpart}(\Bundle{koma-script})'none'%
)
und Kapitelseiten 
(%
  \Macro{chapter}(\Bundle{koma-script})'none', 
  \Macro{addchap}(\Bundle{koma-script})'none'%
)
sowie alle darunter liegenden Gliederungsebenen.

Soll ein bestimmtes Element des Layouts abweichend von der allgemeinen 
Einstellung für das gesamte Dokument erscheinen, so kann eine der folgenden 
Optionen genutzt werden, um dieses individuell anzupassen und die mit 
\Option{cd=\PMisc} eingestellten Vorgaben für das jeweilige Element zu 
überschreiben.%
\footnote{%
  \Option{cdtitle} für den Titel, \Option{cdcover} für die Umschlagseite,
  \Option{cdpart} für Teile, \Option{cdchapter} für Kapitel sowie
  \Option{cdsection} für alle darunter liegenden Gliederungsebenen%
}
Die gültigen Wertzuweisungen für die einzelnen Elemente entsprechend dabei den 
möglichen Werten für die Option \Option{cd=\PMisc}. 

Zu beachten ist dabei, dass die verwendete Schrift für die Elemente des Layouts 
abhängig von der Option \Option{cdfont=\PMisc} ist. Für Titel- und Umschlagseite
kann diese direkt über das optionale Argument von \Macro{maketitle} respektive 
\Macro{makecover} geändert werden.

\begin{Declaration}
  {\Option{cdtitle=\PMisc}}
[%
  v2.03!\Option{cdtitle=bicolor}:%
    Farbeinsatz nur im Kopf mit farbig abgesetztem Querbalken;
  v2.03!\Option{cdtitle=fullcolor}:%
    voller Farbeinsatz mit farbig abgesetztem Querbalken;
  v2.04!\Option{cdtitle=barcolor}:nur farbig abgesetzter Querbalken;%
]
\printdeclarationlist[Titel|?]

Mit der Option \Option{cdtitle} kann die allgemeine Einstellung für den Titel 
überschrieben werden. Es kann zwischen dem normalen (\Option*{cdtitle=false}) 
und dem im \CD umgeschaltet werden. Die neue Titelseite unterstützt alle durch 
\KOMAScript definierten Befehle für den Titel.%
\footnote{%
  \Macro{extratitle}[\MPName{Schmutztitel}](\Bundle{koma-script}),
  \Macro{frontispiece}[\MPName{Frontispiz}](\Bundle{koma-script}) für eine 
  Seite vor dem eigentlichen Haupttitel sowie
  \Macro{titlehead}[\MPName{Kopf}](\Bundle{koma-script}),
  \Macro{subject}[\MPName{Typisierung}], \Macro{title}[\MPName{Titel}],
  \Macro{subtitle}[\MPName{Untertitel}], \Macro{author}[\MPName{Autor}],
  \Macro{and}, \Macro{thanks}[\MPName{Fußnote}](\Bundle{koma-script}), 
  \Macro{publishers}[\MPName{Verlag}](\Bundle{koma-script}) und 
  \Macro{date}[\MPName{Datum}] für den Titel selbst sowie 
  \Macro{uppertitleback}[\MPName{Titelrückseitenkopf}](\Bundle{koma-script})
  und
  \Macro{lowertitleback}[\MPName{Titelrückseitenfuß}](\Bundle{koma-script}) 
  für die Rückseite des Titels sowie abschließend 
  \Macro{dedication}[\MPName{Widmung}](\Bundle{koma-script}) für eine 
  Danksagung.
}
Zusätzlich werden viele neue Felder definiert, welche vor allem für eine 
wissenschaftliche Arbeit von Relevanz sind. Genaueres dazu 
ist in \autoref{sec:title} nachzulesen. Unabhängig von der gewählten Variante 
der Titelseite wird diese immer mit \Macro{maketitle} erzeugt.
\end{Declaration}

\begin{Declaration}
  {\Option{cdcover=\PMisc}}
[%
  v2.02;%
  v2.03!\Option{cdcover=bicolor}:%
    Farbeinsatz nur im Kopf mit farbig abgesetztem Querbalken;
  v2.03!\Option{cdcover=fullcolor}:%
    voller Farbeinsatz mit farbig abgesetztem Querbalken;
  v2.04!\Option{cdcover=barcolor}:nur farbig abgesetzter Querbalken;%
]
\printdeclarationlist[Umschlagseite|?]

Die \TUDScript-Klassen führen zusätzlich den Befehl \Macro{makecover} ein, mit 
dem sich neben dem Titel eine separate Umschlagseite erzeugen lässt. Diese ist 
in ihrer Gestalt der Titelseite sehr ähnlich, wird normalerweise jedoch in 
einem anderen Satzspiegel als dem des Buchblocks gesetzt. Mit der Option 
\Option{cdcover} kann~-- unabhängig von \Option{cd=\PMisc}~-- das Aussehen der 
Umschlagseite geändert werden. Wird \Option*{cdcover=false} gewählt, entspricht 
die Umschlagseite dem originalen \KOMAScript-Titel. Die Verwendung des Befehls 
\Macro{makecover} sowie die dazugehörigen Parameter werden detailliert in 
\autoref{sec:title} erläutert.
\end{Declaration}

\begin{Declaration}
  {\Option{cdpart=\PMisc}}
[%
  v2.03!\Option{cdpart=bicolor}:%
    Farbeinsatz nur im Kopf mit farbig abgesetztem Querbalken;
  v2.03!\Option{cdpart=fullcolor}:%
    voller Farbeinsatz mit farbig abgesetztem Querbalken;
  v2.04!\Option{cdpart=barcolor}:nur farbig abgesetzter Querbalken;%
]
\printdeclarationlist[Teil|?]

Für die Teileseiten kann der Wert des Schlüssels \Option{cd=\PMisc} separat 
überschrieben und somit deren Layout respektive Erscheinungsbild beeinflusst 
werden, welches bei der Benutzung der Befehle 
\Macro{part}(\Bundle{koma-script})'none' und 
\Macro{addpart}(\Bundle{koma-script})'none' sowie deren Sternversionen 
verwendet wird. In \autoref{sec:part} sind weitere Hinweise zur Teileseite im 
\CD zu finden.
\end{Declaration}

\begin{Declaration}
  {\Option{cdchapter=\PMisc}}
[%
  v2.03!\Option{cdchapter=bicolor}:%
    Farbeinsatz nur im Kopf mit farbig abgesetztem Querbalken;
  v2.03!\Option{cdchapter=fullcolor}:%
    voller Farbeinsatz mit farbig abgesetztem Querbalken;
  v2.04!\Option{cdchapter=barcolor}:nur farbig abgesetzter Querbalken;%
]
\printdeclarationlist[Kapitel|?]

Für Kapitelseiten kann der Schlüsselwert \Option{cd=\PMisc} ebenfalls angepasst 
und damit das Layout respektive Erscheinungsbild geändert werden, das bei der 
Verwendung von \Macro{chapter}(\Bundle{koma-script})'none' beziehungsweise 
\Macro{addchap}(\Bundle{koma-script})'none' und den dazugehörigen 
Sternversionen genutzt wird. Weitere Hinweise zur Kapitelseite im \CD sind in 
\autoref{sec:chapter} zu finden.
\end{Declaration}

\begin{Declaration}
  {\Option{cdsection=\PMisc}}
  [v2.05]
\printdeclarationlist

Für Überschriften der weiteren Gliederungsebenen
\Macro{section}(\Bundle{koma-script})'none', 
\Macro{subsection}(\Bundle{koma-script})'none', 
\Macro{subsubsection}(\Bundle{koma-script})'none' sowie 
\Macro{paragraph}(\Bundle{koma-script})'none' und 
\Macro{subparagraph}(\Bundle{koma-script})'none'
werden in der primären Hausfarbe \Color{HKS41} ausgegeben, falls über die 
Option \Option{cd=\PMisc} eine farbige Ausprägung des Layouts eingestellt wurde.
Mit der Angabe von \Option*{cdsection=true} erscheinen die Überschriften der 
genannten Gliederungsebenen ohne Farbeinsatz.%
\end{Declaration}
%
\begin{Example}
Soll die Titelseite in Farbe, der Rest des Dokumentes allerdings in schwarzer 
Schrift gesetzt werden, so kann dies folgendermaßen erreicht werden:
\begin{Code}
\documentclass[cd=true,cdtitle=color]§\MPName{Dokumentklasse}§
\end{Code}
\end{Example}
\index{Layout!Farbgestaltung|?)}%



\subsubsection{Die vertikale Position der Überschriften}

\begin{Declaration}
  {\Option{pageheadingsvskip=\PName{Längenwert}}}
  [v2.05]
\begin{Declaration}
  {\Option{headingsvskip=\PName{Längenwert}}}
  [v2.05]
\printdeclarationlist

Mit diesen beiden Optionen kann die vertikale Position spezieller Überschriften 
verändert werden. Mit der Option \Option{pageheadingsvskip} lässt sich sowohl 
der Titel auf einer Titelseite 
(\KOMAScript-Option \Option{titlepage=true}(\Bundle{koma-script})'none') als 
auch die Überschriften von separaten Kapitelseiten (\Option{chapterpage=true}) 
und Teilen vertikal verschieben. Demgegenüber kann mit der zweiten Option 
\Option{headingsvskip}, sowohl den Titel des Titelkopfes 
(\KOMAScript-Option \Option{titlepage=false}(\Bundle{koma-script})'none') als 
auch die Kapitelüberschrift bei deaktivierter Kapitelseite 
(\Option{chapterpage=false}) in ihrer vertikalen Position anzupassen.

Die zuvor genannten Überschriften werden normalerweise im Layout relativ tief 
im Textbereich gesetzt. Mit negativen Werten werden die Überschriften nach oben 
verschoben, wobei darauf geachtet werden sollte, dass diese sich danach noch 
innerhalb des Satzspiegels befinden. Positive Werte setzen die Überschriften 
dementsprechend tiefer.%
\end{Declaration}
\end{Declaration}
\index{Layout!Überschriften|)}%



\subsection{%
  Seiten im Stil des \CDs%
  \label{sec:tudheadings}%
  \index{Seitenstil|?(}%
}

\begin{Declaration}
  {\Macro{faculty}[\MPName{Fakultät}]}
\begin{Declaration}
  {\Macro{department}[\MPName{Einrichtung}]}
\begin{Declaration}
  {\Macro{institute}[\MPName{Institut}]}
\begin{Declaration}
  {\Macro{chair}[\MPName{Lehrstuhl}]}
\begin{Declaration}
  {\Macro{extraheadline}[\MPName{Textzeile}]}
\printdeclarationlist[Layout!Kopfzeile]

Für den Seitenstil des \TUDCDs charakteristisch ist die Kopfzeile mit dem 
prägnanten Querbalken. In dieser wird~-- falls angegeben~-- in fetter Schrift 
die Fakultät ausgegeben, danach folgen durch Kommas getrennt die Einrichtung, 
das Institut und der Lehrstuhl beziehungsweise die Professur. Sollte der Platz 
in der ersten Zeile nicht ausreichen, erfolgt ein automatischer Zeilenumbruch.

In besonderen Ausnahmefällen erlaubt das \CD die Angabe einer zusätzlichen
zweiten beziehungsweise dritten Zeile unterhalb der Angaben des Bereichs an der 
\TnUD, welche weitere, frei wählbare Angaben enthält. Diese kann mit dem Befehl 
\Macro{extraheadline}[\MPName{Textzeile}] definiert werden.
\end{Declaration}
\end{Declaration}
\end{Declaration}
\end{Declaration}
\end{Declaration}
%
\begin{Declaration}
  {\PageStyle{tudheadings}}
  [v2.02]
\begin{Declaration}
  {\PageStyle{plain.tudheadings}}
  [v2.02]
\begin{Declaration}
  {\PageStyle{empty.tudheadings}}
  [v2.02]
\printdeclarationlist[Seitenstil,Layout!Kopfzeile,Layout!Fußzeile,Layout!Logo]

\ChangedAt*{%
  v2.03:Seitenstile um zweifarbigen Kopf und farbigen Fuß erweitert%
}%
Ein zentrales Element des \TUDCDs ist der prägnante Seitenkopf mit der Angabe 
von Fakultät~(\Macro{faculty}), Einrichtung~(\Macro{department}), 
Institut~(\Macro{institute}) und Lehrstuhl~(\Macro{chair}) im dazugehörigen 
Querbalken. Durch die Nutzung des Paketes \Package{scrlayer-scrpage} lassen 
sich entweder einzelne Seiten oder auch ganze Dokumente sehr einfach in diesem 
Stil setzen. Dazu muss lediglich mit 
\Macro{pagestyle}[\MPName{Seitenstil}](\Bundle{koma-script})'none' einer 
der Seitenstile geladen werden. 

Allen Seitenstilen gemein ist der typische Kopf mit dem charakteristischen 
Querbalken, dessen Gestalt für \emph{alle} Seitenstile gleichermaßen über die 
Option \Option{cdhead} angepasst werden kann. Mit dem Befehl \Macro{headlogo} 
lässt sich ein zusätzliches Zweitlogo im Kopfbereich ausgegeben.
\Attention{%
  Für die speziellen Layout-Elemente Titel und Umschlagseite sowie Teile- und 
  Kapitelseiten wird die Einstellung von \Option{cdhead} durch die Nutzung der 
  Option~\Option{cd=\PMisc} überschrieben.
}

Die Ausprägung des Fußes unterscheidet sich bei den einzelnen Seitenstilen. 
Dieser ist beim Seitenstil \PageStyle{empty.tudheadings} immer leer. Die beiden 
Stile~-- oder vielmehr das Seitenstil"~Paar~-- \PageStyle{tudheadings} und 
\PageStyle{plain.tudheadings} übernehmen die Einstellungen für die Fußzeile aus 
der Anwenderschnittstelle von \Package{scrlayer-scrpage}.%
\footnote{%
  Es können die Befehle
  \Macro{lefoot}(\Package{scrlayer-scrpage})'none', 
  \Macro{cefoot}(\Package{scrlayer-scrpage})'none' und 
  \Macro{refoot}(\Package{scrlayer-scrpage})'none' sowie 
  \Macro{lofoot}(\Package{scrlayer-scrpage})'none', 
  \Macro{cofoot}(\Package{scrlayer-scrpage})'none' und 
  \Macro{rofoot}(\Package{scrlayer-scrpage})'none' respektive 
  \Macro{ofoot}(\Package{scrlayer-scrpage})'none', 
  \Macro{cfoot}(\Package{scrlayer-scrpage})'none' und 
  \Macro{ifoot}(\Package{scrlayer-scrpage})'none'
  genutzt werden.%
}
Wie die einzelnen Befehle zur Individualisierung der Fußzeile zu verwenden 
sind, kann der \scrguide[\KOMAScript-Anleitung] entnommen werden. Alternativ 
zu einer eigenen Definition der Fußzeile lässt sich außerdem die Option 
\Option{cdfoot} verwenden. Zusätzlich kann über \Macro{footcontent} ein freier 
Inhalt für den Fußbereich definiert werden, mit \Macro{footlogo} ist die 
Ausgabe von einem oder mehreren Logos in diesem möglich. Die verwendete Schrift 
im Fußbereich wird durch das Schriftelement~\Font{tudheadings} festgelegt.

Sobald einer der definierten Stile mit 
\Macro{pagestyle}[\MPName{Seitentil}](\Bundle{koma-script})'none' aktiviert 
wurde, sind die beiden Seitenstile \PageStyle{tudheadings} sowie 
\PageStyle{plain.tudheadings} zusätzlich unter den Namen \PageStyle{headings} 
respektive \PageStyle{plain} verwendbar. Dies hat den Vorteil, dass bei 
Optionen oder Befehlen, welche automatisch zwischen den beiden Seitenstilen 
\PageStyle{headings} und \PageStyle{plain} umschalten, durch die einmalige 
Auswahl von einem der \PageStyle{tudheadings}"=Stilen nun zwischen diesen 
umgeschaltet wird.

Der Seitenstil \PageStyle{empty} erzeugt allerdings weiterhin eine komplett 
leere Seite. Soll eine Seite mit der prägnanten Kopfzeile der \TnUD und leerem 
Seitenfuß erschienen, so muss \Macro*{pagestyle}[\MPValue{empty.tudheadings}]
manuell aufgerufen werden. Um auf das normale Verhalten von \KOMAScript{} 
zurückzuschalten, muss mit \Macro*{pagestyle}[\MPName{Seitentil}] einer der 
beiden Seitenstile \PageStyle{scrheadings}(\Package{scrlayer-scrpage})'none'
oder \PageStyle{plain.scrheadings}(\Package{scrlayer-scrpage})'none' aktiviert 
werden.%
\end{Declaration}
\end{Declaration}
\end{Declaration}
\index{Seitenstil|?)}%

\begin{Declaration}
  {\Option{cdhead=\PMisc}}
  (true,nocolor,textwidth)
[%
  v2.03;%
  v2.04!\Option{cdhead=barcolor}:nur farbig abgesetzter Querbalken;%
  v2.05!\Option{cdhead=date}:Datum zwischen Kopf- und Textbereich;%
]
\printdeclarationlist[Layout!Kopfzeile,Layout!Farbgestaltung]

Mit dieser Option lassen sich für die \PageStyle{tudheadings}"=Seitenstile 
sowohl die Gestalt des Logos sowie des Querbalkens als auch die darin 
verwendete Schrift beeinflussen. Die folgenden Werte können für eine Anpassung 
der Schriftart im Balken verwendet werden:
%
\begin{values}{\Option{cdhead}}
\itemfalse
  Sollte mit \Option{cdfont=false} die Verwendung der Hausschrift im Stil des 
  \TUDCDs deaktiviert worden sein, wird die Kopfzeile im Querbalken in den 
  Serifenlosen der aktiven Schrift gesetzt. Ist die Hausschrift aktiviert, 
  hat diese Einstellung keinen Einfluss.
\itemtrue*[light/lightfont/noheavyfont]
  Im Querbalken wird für \Macro{faculty} \textcdsn{Open~Sans~Semi"~Bold} 
  verwendet, für \Macro{department}, \Macro{institute}, 
  \Macro{chair} und \Macro{extraheadline} kommt \textcdln{Open~Sans~Light} zum 
  Einsatz.
\item[heavy/heavyfont]
  Der Inhalt von \Macro{faculty} wird in \textcdbn{Open~Sans~Bold} gesetzt, für 
  die restlichen Felder wird \textcdrn{Open~Sans~Regular} genutzt.
\end{values}
%
Bei der Ausprägung des Kopfes und des Querbalkens gibt es mehrere Varianten. 
Einerseits kann der Querbalken mit zwei Außenlinien dargestellt werden:
%
\begin{values}{\Option{cdhead}}
\item[nocolor/monochrome]
  Der Kopf und die Linien des Querbalkens erscheinen in schwarzer Farbe.
\item[lightcolor/pale]
  Sowohl Kopf als auch Querbalken werden in der primären Hausfarbe gesetzt.
\end{values}
%
Andererseits ist auch eine Darstellung mit mehr Farbeinsatz möglich, bei 
welcher der Querbalken und gegebenenfalls der ganze Seitenkopf farbig 
abgesetzt wird. Dabei erstreckt sich der Kopfbereich immer über die komplette 
Seitenbreite, \seeref{\Option{cdhead=paperwidth}}.
%
\begin{values}{\Option{cdhead}}
\item[barcolor]
  \ChangedAt{v2.04}%
  Im Gegensatz zur vorherigen Einstellung wird der Querbalken mit farbigem 
  Hintergrund verwendet, der darüber liegende Teil des Kopfes wird ohne 
  farbigen Hintergrund gesetzt.
\item[bicolor/bichrome]
  \ChangedAt{v2.03}%
  Die Kopfzeile wird farbig abgesetzt, wobei der Hintergrund des Logos und der 
  Querbalken unterschiedlich ausfallen. Die Außenlinien der Querbalkens 
  entfallen.
\end{values}
%
Für den Fall, dass der Querbalken lediglich mit zwei Außenlinien dargestellt 
wird, kann zusätzlich dessen Laufweite festgelegt werden:
%
\begin{values}{\Option{cdhead}}
\item[textwidth/slim]
  Der Querbalken im Kopf erstreckt sich nur über den Textbereich. Diese 
  Einstellung ist insbesondere sinnvoll, wenn ein randloser Ausdruck technisch 
  nicht möglich ist. 
\item[paperwidth/wide]
  Die horizontale Ausdehnung des Querbalkens erstreckt sich über die komplette 
  Seitenbreite bis an den Blattrand. Dieses Verhalten lann im farbigen Layout 
  des Kopfes nicht deaktiviert werden.
\end{values}
%
\ChangedAt{v2.05}%
Neben den zuvor beschriebenen Möglichkeiten zur Gestaltung des Kopfbereiches 
kann auf allen Seiten mit aktiviertem \PageStyle{tudheadings}"=Seitenstil 
unterhalb des Querbalkens das mit \Macro{date} angegebene Datum rechtsbündig 
eingeblendet werden.%
\index{Datum}%
%
\begin{values}{\Option{cdhead}}
\item[date/showdate]
  Das eingestellte Datum wird rechts oberhalb vom Textbereich eingeblendet.
\item[nodate/hidedate]
  Es erscheint kein Datum zwischen Kopf- und Textbereich.%
\end{values}
\end{Declaration}

\begin{Declaration}
  {\Option{cdfoot=\PMisc}}
  (false)
[%
  v2.03!\Option{cdfoot=color}:farbiger Hintergrund der Fußzeile;%
  v2.03!\Option{cdfoot=\PSet{Höhe}};%
]
\printdeclarationlist[Layout!Fußzeile,Layout!Farbgestaltung]

Die \TUDScript-Klassen sind~-- insbesondere aufgrund der Möglichkeit zur 
Verwendung des Paketes \Package{scrlayer-scrpage}~-- bei der Gestaltung der 
Kopf"~ und Fußzeilen sehr flexibel und individuell anpassbar. Die Ausprägung 
und der Inhalt dieser ist nicht explizit durch das \CD vorgegeben und können 
durch den Anwender beliebig gewählt und geändert werden. 

Eine Möglichkeit für deren Gestaltung zeigt das Handbuch für das \TUDCD. Dieses 
wird ohne Kopf"~ und mit einer einfachen Fußzeile gesetzt, welche den aktuellen 
Kolumnentitel sowie die Paginierung enthält. Mit \Option{cdfoot} kann diese 
Ausprägung aktiviert werden, was auch für dieses Anwenderhandbuch geschehen ist.
%
\begin{values}{\Option{cdfoot}}
\itemfalse
  Die Kopf"~ und Fußzeilen zeigen Standardverhalten, zur manuellen Änderung 
  sollte unbedingt das \KOMAScript"=Paket \Package{scrlayer-scrpage} verwendet 
  werden.
\itemtrue*
  \index{Satzspiegel!doppelseitig}%
  Die Fußzeilen des Dokumentes werden äquivalent zum Handbuch des \TUDCDs mit 
  lebenden Kolumnentitel und Seitenzahl gesetzt, wobei im doppelseitigen Satz 
  (Klassenoption \Option{twoside=true}(\Package{typearea})'none') die 
  Paginierung außen platziert wird.
\end{values}
%
\index{Kolumnentitel}%
Wird beim Laden der Klasse respektive des Paketes \Package{scrlayer-scrpage} 
die Option \Option{manualmark}(\Package{scrlayer-scrpage})'none'|declare| nicht 
explizit angegeben, so werden mit \Option{cdfoot=true} über die 
\KOMAScript-Option \Option{automark}(\Package{scrlayer-scrpage})'none'|declare| 
auch gleichzeitig die automatischen Kolumnentitel aktiviert, welche als Marken 
die Titel der Gliederungsebenen verwendet. Genaueres dazu und der Möglichkeit, 
die Kolumnentitel manuell festzulegen, ist dem \scrguide zu entnehmen.

\ChangedAt{v2.03}%
Sollte einer der \PageStyle{tudheadings}"=Seitenstil aktiviert sein und es wird 
auf der erzeugten Seite ein farbiges Layout~--  beispielsweise der zweifarbige 
Kopf (\Option{cdhead=bicolor}) oder ein farbiger Seitenhintergrund~-- genutzt, 
so kann auch die Fußzeile einen farbigen Hintergrund erhalten.
%
\begin{values}{\Option{cdfoot}}
\item[nocolor/monochrome]
  Der Fuß wird immer ohne farbigen Hintergrund gesetzt.
\item[color/bicolor/bichrome]
  Die Fußzeile wird farbig abgesetzt, falls entweder der Kopf in einer farbigen
  Variante genutzt wird (\seeref{\Option{cdhead}}) oder eine Seite mit einem 
  farbigen Hintergrund in der Hausfarbe (Titel oder Kapitelseite) aktiv ist.
\end{values}
%
Mit den Befehlen \Macro{footlogo}'ppage' und \Macro{footcontent}'ppage' können 
für den Fußbereich zusätzliche Inhalte definiert werden. Sollte der zur 
Verfügung stehende Platz hierfür nicht ausreichen, lässt sich dieser vergrößern.
%
\begin{values}{\Option{cdfoot}}
\item[\PSet{Höhe}]
  Wird der Option ein Längenwert übergeben, entspricht dies exakt der 
  Verwendung von Option \Option{extrabottommargin=\PName{Höhe}}.%
\end{values}
\end{Declaration}

\begin{Declaration}
  {\Font{tudheadings}}
  [v2.04]
\printdeclarationlist
  [Seitenstil,Layout!Kopfzeile,Layout!Fußzeile,Kolumnentitel,Schriftelemente]

Im Fußbereich der Seiten im \PageStyle{tudheadings}"=Seitenstil wird das 
Schriftelement~\Font{tudheadings} verwendet. Dieses wirkt sich auf die 
Seitenzahlen, den Kolumnentitel und die mit \Macro{footcontent} angegebenen 
Inhalte aus. Hierüber wird die Wahl der richtigen Schriftfarbe in Abhängigkeit 
vom Seitenhintergrund und den Einstellungen für die Optionen \Option{cdhead} 
sowie \Option{cdfoot} realisiert. Wie \Font{tudheadings} angepasst werden 
kann, ist in \autoref{sec:fonts:elements} zu finden.
\end{Declaration}

\begin{Declaration}
  {\Macro{headlogo}[\OList\MPName{Dateiname}]}
\printdeclarationlist[Layout!Kopfzeile,Layout!Logo]

Neben dem Logo der \TnUD darf zusätzlich ein Zweitlogo im Kopf verwendet 
werden. Dieses lässt sich mit diesem Befehl einbinden. Normalerweise wird es 
auf die Höhe der Erstlogos skaliert. Über das optionale Argument können weitere 
Formatierungsbefehle an den im Hintergrund verwendeten Befehl 
\Macro{includegraphics}(\Package{graphicx})'none' durchgereicht werden, um 
beispielsweise die Größe des Zweitlogos anzupassen. Welche Parameter angepasst 
werden können, ist der Dokumentation des \Package{graphicx}"=Paketes zu 
entnehmen.

Sollte die Option \Option{ddc} aktiviert sein, wird das \DDC-Logo nicht im Kopf 
sondern automatisch im Fuß gesetzt. Die Option \Option{ddchead} setzt dieses 
auf jeden Fall im Kopf und überschreibt damit das mit \Macro{headlogo} 
angegebene Zweitlogo.
\end{Declaration}

\begin{Declaration}
  {\Macro{footlogo}[\OList\MPName{Dateinamenliste}]}
  [v2.03]
\begin{Declaration}
  {\Macro{footlogosep}}
  [v2.03]
\printdeclarationlist[Layout!Fußzeile,Layout!Logo]

Laut den Richtlinien des \CDs dürfen im Fußbereich weitere Logos erscheinen, 
beispielsweise von kooperierenden Unternehmen oder Sponsoren. Die Dateinamen 
der gewünschten Logos können als kommaseparierte Liste im obligatorischen 
Argument des Befehls \Macro{footlogo} angegeben werden. Sollte tatsächlich 
nicht nur ein Dateiname sondern eine Liste übergeben worden sein, so wird bei 
der Ausgabe der Logos zwischen diesen jeweils der in \Macro{footlogosep} 
gespeicherte Separator~-- standardmäßig \Macro*{hfill}~-- gesetzt. Dieser kann 
mit \Macro*{renewcommand*}[\MPValue{\Macro{footlogosep}}\MPValue{\dots}] 
beliebig durch den Anwender angepasst werden. Der Separator wird auch gesetzt, 
falls in \PName{Dateinamenliste} lediglich ein Komma verwendet wurde. Mit 
\Macro{footlogo}[\MPValue{,\PName{Dateiname},}] kann so beispielsweise ein 
Logo zentriert im Fuß gesetzt werden.
\Attention{%
  Dabei ist zu beachten, dass ein mit der Option \Option{ddc} beziehungsweise 
  \Option{ddcfoot} gesetztes \DDC-Logo im Fußbereich~-- im Gegensatz zur 
  Verwendung von \Macro{footcontent}~-- überlagert werden könnte. Hier muss der 
  Anwender im Zweifel durch das Einfügen von Separatoren~-- sprich Kommas~-- im 
  Argument von \Macro{footlogo} etwas Formatierungsarbeit leisten.
}

Das optionale Argument von \Macro{footlogo} wird an 
\Macro{includegraphics}(\Package{graphicx})'none' weitergereicht. Dies 
geschieht für alle angegeben Dateien aus der Liste gleichermaßen. Sollen für 
einzelne Logos individuelle Einstellungen vorgenommen werden, so sind die 
entsprechenden Parameter im obligatorischen Argument nach dem jeweiligen 
Dateinamen mit einem Doppelpunkt~\enquote{\PValue{:}} als Separator 
(\Macro{footlogo}[\MPValue{\PName{Dateiname}:\PName{Parameter}}]) zu 
übergeben, wobei diese \emph{nach} den allgemeinen Einstellungen für alle Logos 
angewendet werden. Die möglichen Parameter und Werte für die optionalen 
Argumente sind der Dokumentation des \Package{graphicx}"=Paketes zu entnehmen.
\end{Declaration}
\end{Declaration}

\begin{Declaration}
  {\Option{footlogoheight=\PName{Längenwert}}}
  [v2.05]
\printdeclarationlist[Layout!Fußzeile,Layout!Logo]

Ohne die Angabe eines optionalen Argumentes bei \Macro{footlogo} für die Größe 
werden alle Logos im Fuß auf die Höhe des Logos der \TnUD skaliert. Dies kann 
global für alle Logos geändert werden, indem die Option 
\Option{footlogoheight=\PName{Längenwert}} gesetzt wird. Sollte die Höhe des 
Fußbereiches nicht ausreichen, um alle Logos in der gewünschten Größe 
darstellen zu können, kann diese über \Option{extrabottommargin=\PName{Höhe}} 
beziehungsweise \Option{cdfoot=\PSet{Höhe}} angepasst werden.
\end{Declaration}

\begin{Declaration}
  {\Macro{footcontent}[\OPName{Anweisungen}\MPName{Inhalt}\OPName{Inhalt}]}
[%
  v2.04;
  v2.05:Änderung des Inhaltes nur einer Spalte möglich
]
\begin{Declaration}
  {\Macro{footcontent*}[\OPName{Anweisungen}\MPName{Inhalt}\OPName{Inhalt}]}
  [v2.04]
\printdeclarationlist[Layout!Fußzeile,Layout!Logo,Satzspiegel!zweispaltig]

Mit diesem Befehl kann beliebiger Inhalt entweder einspaltig oder zweispaltig 
im Fußbereich der \PageStyle{tudheadings}"=Seitenstile gesetzt werden. In der 
Form \Macro{footcontent}[\MPName{Inhalt}] wird der Inhalt über die komplette 
Textbreite im Fuß ausgegeben. Wird der Befehl jedoch in der zweiten Variante 
\Macro{footcontent}[\MPName{linker Inhalt}\OPName{rechter Inhalt}] mit 
einem optionalen \emph{nach} dem obligatorischen Argument verwendet, so 
erscheint der Fußbereich zweispaltig, wobei der Inhalt aus dem ersten, 
obligatorischen Argument in der linken und der Inhalt aus dem zweiten, 
optionalen Argument entsprechend in der rechten Fußspalte gesetzt wird. Dabei 
wird ein etwaiges \DDC-Logo, welches über die Option \Option{ddc} respektive
\Option{ddcfoot} gesetzt wurde, beachtet und der für den Text zur Verfügung 
stehende Platz im Fuß reduziert. Mögliche Einstellungen des Paketes  
\Package{scrlayer-scrpage} für den Fuß werden nicht berücksichtigt, hier kann 
es zu Überlagerungen der Inhalte kommen. Gleiches gilt für die Verwendung der 
Werte \PValue{true} und \PValue{false} für die Option \Option{cdfoot}.

\ChangedAt{v2.05}%
Wird an das Argument für die linke oder die rechte Spalte lediglich ein Stern 
\PValue{*} übergeben, so bleibt der bis dahin definierte Inhalt in dieser 
Spalte erhalten. Beispielsweise kann die linke Fußbereichsspalte mit 
\Macro{footcontent}[\MPName{Inhalt}\OPValue{*}] angepasst werden ohne 
dabei den Inhalt der rechten Spalte zu verändern oder es ließe sich lediglich 
die verwendete Schrift des Fußbereichs bei gleichbleibendem Inhalt mit 
\Macro{footcontent}[\OPName{Anweisungen}\MPValue{*}\OPValue{*}]%
anpassen.

\index{Schriftelemente}%
Im Fußbereich wird für die Schrift das Schriftelement \Font{tudheadings} 
verwendet. Dabei wird auch die Schriftgröße angepasst, wobei diese sich an der 
Kopfzeile orientiert. Zusätzlich können mit dem ersten optionalen Argument von 
\Macro{footcontent}~-- vor der eigentlichen Ausgabe des Inhaltes~-- zusätzliche 
Schrifteinstellungen respektive \PName{Anweisungen} ausgeführt werden. Soll die 
Definition des Inhalts im Fußbereich \emph{ohne} eine automatische Anpassung 
der Schriftgröße erfolgen, so ist die Sternversion \Macro{footcontent*} zu 
verwenden. Auch hier lässt sich gegebenenfalls das optionale Argument für die 
Schriftformatierung nutzen.
\ChangedAt{v2.06}%
Zu guter Letzt können bei der Änderung des Inhaltes im Fußbereich auch 
vorherige Anpassungen der Schrift unverändert bleiben, indem mit
\Macro{footcontent}[\OPValue{*}\MPName{Inhalt}\OPName{Inhalt}]
im optionalen Argument ein Stern verwendet wird. 
\end{Declaration}
\end{Declaration}

\begin{Declaration}
  {\Option{ddc=\PMisc}}
  (false)
[%
  v2.02:Logo von \DDC automatisch in Kopf/Fuß;%
  v2.02!\Option{ddc=colorblack};
  v2.02!\Option{ddc=gray};
  v2.02!\Option{ddc=black};
  v2.02!\Option{ddc=blue};
  v2.02!\Option{ddc=white};
]
\begin{Declaration}
  {\Option{ddchead=\PMisc}}
  [v2.02]
\begin{Declaration}
  {\Option{ddcfoot=\PMisc}}
  [v2.02]
\printdeclarationlist[Layout!Kopfzeile,Layout!Fußzeile,Layout!Logo]

Diese Option fügt das Logo von \DDC entweder im Kopf oder Fuß der Seiten mit 
dem Stil \PageStyle{tudheadings} ein. Diese wird automatisch entweder im Kopf 
oder~-- falls mit \Macro{headlogo} ein Zweitlogo angegeben wurde~-- im Fuß 
gesetzt.

Alternativ dazu können die Optionen \Option{ddchead} beziehungsweise 
\Option{ddcfoot} genutzt werden, welche das Logo explizit entweder im Kopf oder 
Fuß setzen. Ein mit \Macro{headlogo} angegebenes Zweitlogo wird durch 
\Option{ddchead=\PMisc} definitiv unterdrückt, \Option{ddcfoot=\PMisc} setzt das
\DDC-Logo in jedem Fall in den Seitenfuß. Die Verwendung einer der drei 
Optionen führt folglich zur Deaktivierung der anderen beiden.

Das \DDC-Logo wird standardmäßig sowohl im Kopf als auch im Fuß in der gleichen 
Höhe gesetzt, wie das Logo der \TnUD und kann zumindest für den Kopf nicht 
geändert. Wird es im Fuß verwendet, lässt sich die Größe über die Option 
\Option{footlogoheight} ändern. Sollte nach einer Vergrößerung der Darstellung 
die Höhe des Fußbereiches nicht ausreichen, so kann diese über 
\Option{extrabottommargin=\PName{Höhe}} respektive 
\Option{cdfoot=\PSet{Höhe}} angepasst werden.
Die möglichen Werte für die drei Schlüssel sind:
%
\begin{values}{\Option{ddc}}
\itemfalse
  Bei den \PageStyle{tudheadings}"=Seitenstile erscheint kein Logo von \DDC.
\itemtrue*
  Das Logo von \DDC wird im Kopf beziehungsweise im Fuß verwendet. Die Wahl der 
  Farbe des Logos geschieht passend zur farblichen Ausprägung der Seite selbst.
\end{values}
%
Die Farbe des \DDC-Logos wird in Abhängigkeit der farblichen Ausprägung des 
Layouts (Option \Option{cd=\PMisc}) automatisch gewählt. Dies lässt sich 
manuell ändern:
%
\begin{values}{\Option{ddc}}
\item[color]
  Im Kopf oder Fuß wird die achtfarbige 4C"~Variante des \DDC-Logos genutzt.
\item[colorblack]
  Es wird das achtfarbige Logo mit schwarzem \DDC-Schriftzug anstelle des 
  grauen verwendet. Für den Fuß wird der grüne Claim durch einen schwarzen 
  ersetzt. Dies ist insbesondere für kleine Darstellungen des Logos im Fuß 
  sinnvoll.
\item[gray/grey]
  Dies Ausgabe des \DDC-Logos erfolgt in Graustufen.
\item[black]
  Verwendung des Logos in Graustufen mit schwarzem Schriftzug.
\item[blue]
  Schriftzug und Logo werden in Abstufungen der Hausfarbe \Color{HKS41} gesetzt.
\item[white]
  Das \DDC-Logo sowie der dazugehörige Schriftzug sind vollständig weiß.%
\end{values}
\end{Declaration}
\end{Declaration}
\end{Declaration}

\begin{Declaration}
  {\Environment{tudpage}[\OPList{Sprache}]}
\begin{Declaration}
  {\Key{\Environment{tudpage}}{language=\PName{Sprache}}}
\begin{Declaration}
  {\Key{\Environment{tudpage}}{columns=\PName{Anzahl}}}
\begin{Declaration}
  {\Key{\Environment{tudpage}}{pagestyle=\PMisc}}
  [v2.02]
\begin{Declaration}
  {\Key{\Environment{tudpage}}{cdfont=\PMisc}}
  <\seeref{\Option{cdfont}'ppage'}>
\begin{Declaration}
  {\Key{\Environment{tudpage}}{cdhead=\PMisc}}
  <\seeref{\Option{cdhead}'ppage'}>
  [v2.03]
\begin{Declaration}
  {\Key{\Environment{tudpage}}{cdfoot=\PMisc}}
  <\seeref{\Option{cdfoot}'ppage'}>
  [v2.03]
\begin{Declaration}
  {\Key{\Environment{tudpage}}{headlogo=\PName{Dateiname}}}
  <\seeref{\Macro{headlogo}'ppage'}>
\begin{Declaration}
  {\Key{\Environment{tudpage}}{footlogo=\PName{Dateinamenliste}}}
  <\seeref{\Macro{footlogo}'ppage'}>
  [v2.03]
\begin{Declaration}
  {\Key{\Environment{tudpage}}{ddc=\PMisc}}
  <\seeref{\Option{ddc}'ppage'}>
  [v2.02]
\begin{Declaration}
  {\Key{\Environment{tudpage}}{ddchead=\PMisc}}
  <\seeref{\Option{ddchead}'ppage'}>
  [v2.02]
\begin{Declaration}
  {\Key{\Environment{tudpage}}{ddcfoot=\PMisc}}
  <\seeref{\Option{ddcfoot}'ppage'}>
  [v2.02]
\printdeclarationlist[Seitenstil,Layout!Kopfzeile,Layout!Fußzeile]

Die \Environment{tudpage}"=Umgebung hat ihren Ursprung in einer früheren 
Version, als die \PageStyle{tudheadings}"=Seitenstile noch nicht verfügbar 
waren, welche mittlerweile anstelle dieser Umgebung verwendet werden können.
Für die \Environment{tudpage}"=Umgebung lassen sich verschiedene Parameter als 
optionales Argument angegeben. Wird das Paket \Package{babel} genutzt, kann die 
genutzte Sprache mit \Key{\Environment{tudpage}}{language=\PName{Sprache}}
geändert werden, was zur Anpassung der sprachspezifischen Trennungsmuster und 
Bezeichner führt. Wurde das Paket \Package{multicol} geladen, wird mit dem 
Parameter \Key{\Environment{tudpage}}{columns=\PName{Anzahl}} der Inhalt der 
Umgebung mehrspaltig gesetzt. Mit \Key{\Environment{tudpage}}{pagestyle} kann 
der Seitenstil angepasst werden, wobei \PValue{headings}, \PValue{plain} und 
\PValue{empty} gültige Werte sind. 

Die weiteren Parameter entsprechen in ihrem Verhalten den gleichnamigen 
Klassenoptionen oder Befehlen, wirken sich jedoch nur innerhalb der 
\Environment{tudpage}"=Umgebung aus. Das Verhalten sowie gültige 
Wertzuweisungen ist auf den angegebenen Seiten dokumentiert.
\end{Declaration}
\end{Declaration}
\end{Declaration}
\end{Declaration}
\end{Declaration}
\end{Declaration}
\end{Declaration}
\end{Declaration}
\end{Declaration}
\end{Declaration}
\end{Declaration}
\end{Declaration}
\index{Seitenstil|)}%



\subsection{%
  Der Titel und die Umschlagseite%
  \label{sec:title}%
  \index{Titel|!(}%
  \index{Umschlagseite|!}%
}

\ChangedAt*{v2.03:Bugfix für Umschlagseite und Titel beim Satzspiegel}%
Für das Erstellen eines Titels mit dem Befehl \Macro{maketitle} wird mit der 
\KOMAScript-Option 
\Option{titlepage=\PBoolean}(\Bundle{koma-script})'none'|declare| 
festgelegt, ob dieser in Gestalt einer ganzen Titelseite oder nur als Titelkopf 
erscheinen soll. Für den Titel im \TUDCD werden alle Felder unterstützt, welche 
bereits durch \KOMAScript{} definiert sind. Darüber hinaus werden für die 
\TUDScript-Klassen weitere Felder bereitgestellt, welche Auswirkungen auf die 
Gestalt des Titels haben. Diese werden nachfolgend in diesem \autorefname 
erläutert. Der Titel~-- bestehend aus einem möglichen Schmutztitel 
(\Macro{extratitle}(\Bundle{koma-script})) inklusive dazugehöriger Rückseite 
(\Macro{frontispiece}[\MPName{Frontispiz}](\Bundle{koma-script})), der 
eigentlichen Titelseite respektive des Titelkopfes und der nachgelagerten 
Elementen~-- kann mit \Macro{maketitle} ausgegeben werden. Außerdem kann im 
zweispaltigen Satz \Macro{maketitleonecolumn} verwendet werden, womit eine 
einspaltige Ergänzung nach dem Titel selbst ermöglicht wird.

Zusätzlich zum Titel lässt sich mit \Macro{makecover} eine Umschlagseite 
erzeugen. Diese kann insbesondere für gebundene Arbeiten verwendet werden. Es 
wird~-- im Vergleich zum Titel~-- lediglich einer reduzierte Anzahl an Feldern 
auf dieser ausgegeben.

\begin{Declaration}
  {\Macro{maketitle}[\OPList{Seitenzahl}]}
[%
  v2.01:Bugfix für Schriftstärke auf Titelseite;%
  v2.02:Unterstützung der Schriftelemente \Font*{titlehead}, 
    \Font*{subject}, \Font*{title}, \Font*{subtitle}, \Font*{author}, 
    \Font*{date}, \Font*{publishers}, \Font*{dedication}, 
    \Font*{titlepage} und \Font*{thesis}%
]
\begin{Declaration}
  {\Key{\Macro{maketitle}}{pagenumber=\PName{Seitenzahl}}}
  [v2.02]
\begin{Declaration}
  {\Key{\Macro{maketitle}}{cdgeometry=\PMisc}}
  <\seeref{\Option{cdgeometry}'ppage'}>
  [v2.06]
\begin{Declaration}
  {\Key{\Macro{maketitle}}{cdfont=\PMisc}}
  <\seeref{\Option{cdfont}'ppage'}>
  [v2.02]
\begin{Declaration}
  {\Key{\Macro{maketitle}}{cdhead=\PMisc}}
  <\seeref{\Option{cdhead}'ppage'}>
  [v2.03]
\begin{Declaration}
  {\Key{\Macro{maketitle}}{cdfoot=\PMisc}}
  <\seeref{\Option{cdfoot}'ppage'}>
  [v2.03]
\begin{Declaration}
  {\Key{\Macro{maketitle}}{headlogo=\PName{Dateiname}}}
  <\seeref{\Macro{headlogo}'ppage'}>
  [v2.03]
\begin{Declaration}
  {\Key{\Macro{maketitle}}{footlogo=\PName{Dateinamenliste}}}
  <\seeref{\Macro{footlogo}'ppage'}>
  [v2.03]
\begin{Declaration}
  {\Key{\Macro{maketitle}}{ddc=\PMisc}}
  <\seeref{\Option{ddc}'ppage'}>
  [v2.03]
\begin{Declaration}
  {\Key{\Macro{maketitle}}{ddchead=\PMisc}}
  <\seeref{\Option{ddchead}'ppage'}>
  [v2.03]
\begin{Declaration}
  {\Key{\Macro{maketitle}}{ddcfoot=\PMisc}}
  <\seeref{\Option{ddcfoot}'ppage'}>
  [v2.03]
\printdeclarationlist[Satzspiegel!doppelseitig]

Der Befehl \Macro{maketitle} setzt für \Option{cdtitle=false} den normalen 
\KOMAScript"=Titel{}, ansonsten wird die Titelseite im \TUDCD erzeugt. Die 
letztere Variante ist im Vergleich zum Standardtitel um eine Vielzahl von 
Feldern erweitert worden und erlaubt insbesondere die Angabe von Daten für das 
Deckblatt einer akademischen Abschlussarbeit. Die einzelnen Felder werden 
später in diesem \autorefname erläutert. Wird das Dokument doppelseitig und mit 
rechts öffnenden Kapiteln gesetzt,%
\footnote{%
  \KOMAScript-Optionen \Option{twoside=true}(\Package{typearea})'none' und 
  \Option{open=right}(\Bundle{koma-script})'none', Standard für 
  \Class{tudscrbook}%
}
so wird zusätzlich die Option \Option{cleardoublespecialpage} einbezogen. Dies 
ist insbesondere bei den Befehlen \Macro{uppertitleback}(\Bundle{koma-script}) 
respektive \Macro{lowertitleback}(\Bundle{koma-script}) für die Titelrückseite 
zu beachten.

Das optionale Argument erlaubt~-- ebenso wie bei den \KOMAScript"=Klassen~-- 
die Änderung der Seitenzahl der Titelseite. Diese wird jedoch nicht ausgegeben, 
sondern beeinflusst lediglich die Zählung. Sie sollten hier unbedingt eine 
ungerade Zahl wählen, da sonst die gesamte Zählung durcheinander gerät. 
Wird eine Titelseite
(\KOMAScript-Option \Option{titlepage=true}(\Bundle{koma-script})'none') im 
\TUDCD gesetzt (\Option{cdtitle=true}), können auch die weiterhin aufgeführten 
Parameter im optionalen Argument verwendet werden. Diese entsprechen in ihrem 
Verhalten den gleichnamigen Optionen respektive Befehlen, wirken sich jedoch 
nur lokal und einzig auf die Titelseite aus. So kann beispielsweise die Nutzung 
eines \DDC-Logos auf den Titel beschränkt bleiben.
\end{Declaration}
\end{Declaration}
\end{Declaration}
\end{Declaration}
\end{Declaration}
\end{Declaration}
\end{Declaration}
\end{Declaration}
\end{Declaration}
\end{Declaration}
\end{Declaration}

\begin{Declaration}
  {\Macro{maketitleonecolumn}[\OPList{Seitenzahl}\MPName{Einspaltentext}]}
\printdeclarationlist[Satzspiegel!doppelseitig,Satzspiegel!zweispaltig]

Im zweispaltigen Satz 
(Klassenoption~\Option{twocolumn}(\Package{typearea})'none'|declare|) wird mit 
\Macro{maketitle} die Titelseite selbst immer einspaltig gesetzt. Direkt nach 
dem Titel folgt normalerweise der zweispaltige Fließtext. Mit dem Befehl 
\Macro{maketitleonecolumn} kann nach dem Titel zusätzlich weiterer Inhalt~-- 
zum Beispiel eine Zusammenfassung respektive eine Kurzfassung~-- einspaltig 
gesetzt werden.

Bei einer aktivierten Titelseite 
(\KOMAScript-Option~\Option{titlepage=true}(\Bundle{koma-script})'none') 
erfolgt die Ausgabe des Argumentes \MPName{Einspaltentext} direkt nach 
dieser auf einer oder gegebenenfalls mehreren neuen Seiten ebenfalls 
einspaltig. Wird der Befehl \Macro{maketitleonecolumn} statt mit einer 
Titelseite jedoch mit einem Titelkopf 
(\KOMAScript-Option~\Option{titlepage=false}(\Bundle{koma-script})'none') zum 
Einsatz, so folgt diesem die einspaltige Textpassage aus dem obligatorischen 
Argument direkt, wobei gegebenenfalls bei entsprechendem Inhalt ein 
automatischer Seitenumbruch erfolgt. Danach wird direkt und ohne zusätzlichen 
Umbruch auf das zweispaltige Layout umgeschaltet.

Der optionale Parameter von \Macro{maketitleonecolumn} kann äquivalent zu 
\Macro{maketitle} für die Änderung der Seitenzahl, der verwendeten Schrift 
sowie zur Anpassung von Kopf und Fuß verwendet werden. Dabei ist zu beachten, 
dass ein Großteil der Parameter nur Auswirkungen haben, falls eine Titelseite
(\KOMAScript-Option \Option{titlepage=true}(\Bundle{koma-script})'none') 
verwendet wird.
\end{Declaration}

\begin{Declaration}
  {\Macro{makecover}[\OPList{Seitenzahl}]}
[%
  v2.02:Umschlagseite für Layout ohne \noexpand\CD hinzugefügt;%
  v2.02:Unterstützung der Schriftelemente \Font*{titlehead}, 
    \Font*{subject}, \Font*{title}, \Font*{subtitle}, \Font*{author}, 
    \Font*{publishers}, \Font*{titlepage} und \Font*{thesis}%
]
\begin{Declaration}
  {\Key{\Macro{makecover}}{pagenumber=\PName{Seitenzahl}}}
  [v2.02]
\begin{Declaration}
  {\Key{\Macro{makecover}}{cdgeometry=\PBoolean}}
\begin{Declaration}
  {\Key{\Macro{makecover}}{cdfont=\PMisc}}
  <\seeref{\Option{cdfont}'ppage'}>
  [v2.02]
\begin{Declaration}
  {\Key{\Macro{makecover}}{cdhead=\PMisc}}
  <\seeref{\Option{cdhead}'ppage'}>
  [v2.03]
\begin{Declaration}
  {\Key{\Macro{makecover}}{cdfoot=\PMisc}}
  <\seeref{\Option{cdfoot}'ppage'}>
  [v2.03]
\begin{Declaration}
  {\Key{\Macro{makecover}}{headlogo=\PName{Dateiname}}}
  <\seeref{\Macro{headlogo}'ppage'}>
  [v2.03]
\begin{Declaration}
  {\Key{\Macro{makecover}}{footlogo=\PName{Dateinamenliste}}}
  <\seeref{\Macro{footlogo}'ppage'}>
  [v2.03]
\begin{Declaration}
  {\Key{\Macro{makecover}}{ddc=\PMisc}}
  <\seeref{\Option{ddc}'ppage'}>
  [v2.03]
\begin{Declaration}
  {\Key{\Macro{makecover}}{ddchead=\PMisc}}
  <\seeref{\Option{ddchead}'ppage'}>
  [v2.03]
\begin{Declaration}
  {\Key{\Macro{makecover}}{ddcfoot=\PMisc}}
  <\seeref{\Option{ddcfoot}'ppage'}>
  [v2.03]
\printdeclarationlist[Umschlagseite|!]

Eine Umschlagseite wird zumeist für gebundene Abschlussarbeiten verlangt, um 
diese beispielsweise für einen Prägedruck auf dem Buchdeckel zu verwenden. 
Deshalb ist die farbige Ausprägung der Umschlagseite auch deaktiviert, wenn 
diese für das restliche Dokument aktiv ist (\Option{cd=color}). Dies kann 
jedoch jederzeit mit \Option{cdcover=\PMisc} überschrieben werden.

Wird \Option{cdcover=true} gewählt, so wird die Umschlagseite im \TUDCD 
gesetzt. Auf dieser werden der Titel des Dokumentes, die Typisierung 
durch \Macro{thesis} und/oder \Macro{subject} sowie der Autor oder respektive 
die Autoren und gegebenenfalls der mit \Macro{publishers}(\Bundle{koma-script})
angegebene Verlag ausgegeben.
\ChangedAt{v2.02}%
Für die Einstellung \Option{cdcover=false} wird lediglich der normale 
\KOMAScript"=Titel als separate Umschlagseite ausgegeben. 

Die Titelseite selbst gehört immer zum Buchblock und sollte daher im gleichen 
Satzspiegel gesetzt werden. Dem entgegen steht die Umschlagseite, welche 
zumeist in einem anderen Layout erscheint. Normalerweise wird das Cover~-- 
unabhängig von der Option \Option{cdgeometry}~-- im asymmetrischen Satzspiegel 
des \CDs gesetzt. Mit \Key*{\Macro{makecover}}{cdgeometry=false} im optionalen 
Argument kann das Verhalten geändert werden. In diesem Fall erscheint auch die 
Umschlagseite im Buchblock des restlichen Dokumentes. Allerdings können für 
diese Einstellung die Seitenränder mit den Befehlen 
\Macro{coverpagetopmargin}(\Bundle{koma-script})'none', 
\Macro{coverpageleftmargin}(\Bundle{koma-script})'none', 
\Macro{coverpagerightmargin}(\Bundle{koma-script})'none' sowie 
\Macro{coverpagebottommargin}(\Bundle{koma-script})'none' 
durch den Nutzer frei angepasst werden. Mehr dazu ist im \scrguide zu finden.

Außerdem kann mit dem optionalen Argument die Seitenzahl der Umschlagseite 
geändert werden. Diese wird jedoch nicht ausgegeben, sondern beeinflusst 
lediglich die Zählung. Sie sollten hier unbedingt eine ungerade Zahl wählen, da 
sonst die gesamte Zählung durcheinander gerät. Die weiterhin aufgeführten 
Parameter entsprechen in ihrem Verhalten beziehungsweise ihrer Funktion den 
gleichnamigen Optionen respektive Befehlen, wirken sich jedoch nur lokal und 
einzig auf die Umschlagseite aus.%
\end{Declaration}
\end{Declaration}
\end{Declaration}
\end{Declaration}
\end{Declaration}
\end{Declaration}
\end{Declaration}
\end{Declaration}
\end{Declaration}
\end{Declaration}
\end{Declaration}

\index{Titel!Felder|(}%
\begin{Declaration}
  {\Macro{title}[\MPName{Titel}]}
\begin{Declaration}
  {\Macro{subtitle}[\MPName{Untertitel}]}
  [v2.01:Bugfix für Schriftstärke bei Verwendung des Untertitels]
\printdeclarationlist

Die Befehle \Macro{title} und \Macro{subtitle} sind selbsterklärend. Anzumerken 
ist, dass die Schriftstärke des Titels von der Option \Option{headings} abhängt
und der Untertitel immer im fetten Schnitt erscheint. 
\end{Declaration}
\end{Declaration}

\begin{Declaration}
  {\Font{titlepage}}
  [v2.02]
\begin{Declaration}
  {\Font{thesis}}
  [v2.02]
\begin{Declaration}
  {\Macro{raggedtitle}}
  [v2.06]
\printdeclarationlist[Schriftelemente]

Die \TUDScript-Klassen definieren diese neuen Schriftelemente. Dabei wird 
\Font{titlepage} auf der Titelseite für alle Felder verwendet, welche kein 
spezielles Schriftelement verwenden, welches ohnehin durch \KOMAScript{} 
bereitgestellt wird. Das mit \Macro{thesis} angegebene Feld, in welchem der Typ 
einer Abschlussarbeit angegeben wird, nutzt das Schriftelement~\Font{thesis}. 

\ChangedAt{v2.02}%
Für alle Felder des Titels und der Umschlagseite lassen sich die verwendeten
Schriften anpassen. In \autoref{sec:fonts:elements} lässt sich nachlesen, wie 
dies genau funktioniert. Dabei werden für Titel und Umschlagseite sowohl die 
bereits durch \KOMAScript{} bereitgestellten Schriftelemente
\Font{titlehead}(\Bundle{koma-script})'none', 
\Font{subject}(\Bundle{koma-script})'none', 
\Font{title}(\Bundle{koma-script})'none', 
\Font{subtitle}(\Bundle{koma-script})'none', 
\Font{author}(\Bundle{koma-script})'none', 
\Font{publishers}(\Bundle{koma-script})'none',
\Font{date}(\Bundle{koma-script})'none' und 
\Font{dedication}(\Bundle{koma-script})'none'
als auch die neuen \Font{titlepage} und \Font{thesis} unterstützt.
\ChangedAt{v2.06}%
Der Befehl \Macro{raggedtitle} definiert die Ausrichtung des Titels, welcher 
standardmäßig linksbündig gesetzt wird.
\begin{Example}
Um die Einträge auf der Titelseite zentriert auszugeben, genügt folgende 
Definition:
\begin{Code}
\let\raggedtitle\centering
\end{Code}
\end{Example}
\end{Declaration}
\end{Declaration}
\end{Declaration}

\begin{Declaration}
  {\Macro{author}[\MPName{Autor(en)}]}
\begin{Declaration}
  {\Macro{authormore}[\MPName{Autorenzusatz}]}
\begin{Declaration}
  {\Macro{emailaddress}[\OPName{Einstellungen}\MPName{E-Mail-Adresse}]}
[%
  v2.02;%
  v2.05:optionales Argument zur Formatierung mit \Macro*{hypersetup};%
  v2.06:wird für alle Klassen bereitgestellt, Verwendung im Titel möglich%
]
\begin{Declaration}
  {\Macro{emailaddress*}[\MPName{E-Mail-Adresse}]}
[%
  v2.05;%
  v2.06:wird für alle Klassen bereitgestellt, Verwendung im Titel möglich%
]
\begin{Declaration}
  {\Macro{dateofbirth}[\MPName{Geburtsdatum}]}
\begin{Declaration}
  {\Macro{placeofbirth}[\MPName{Geburtsort}]}
\begin{Declaration}
  {\Macro{matriculationnumber}[\MPName{Matrikelnummer}]}
\begin{Declaration}
  {\Macro{matriculationyear}[\MPName{Immatrikulationsjahr}]}
\begin{Declaration}
  {\Macro{course}[\MPName{Studiengang}]}
  [v2.05:Für Titel verwendbar]
\begin{Declaration}
  {\Macro{discipline}[\MPName{Studienrichtung}]}
[%
  v2.02;%
  v2.05:Für Titel verwendbar%
]
\printdeclarationlist
  [Autorenangaben|?,Datum!Geburtsdatum|?,Kollaboratives~Schreiben|?]

Mit dem Befehl \Macro{author} wird der Autor angegeben. Innerhalb des 
Argumentes können auch mehrere Autoren aufgeführt werden, wobei diese in diesem 
Fall jeweils mit \Macro{and} zu trennen sind. Alle weiteren hier vorgestellten 
Befehle können selbst im Argument von \Macro{author} verwendet werden, wodurch 
für jeden Autor individuelle Angaben möglich sind.

Mit \Macro{authormore} wird unter dem Autor eine Zeile ausgegeben, welche 
durch den Anwender frei belegt werden kann. Mit \Macro{emailaddress} kann für 
jeden Autor eine E"~Mail"=Adresse angegeben werden, welche als Hyperlink
definiert wird, falls das Paket \Package{hyperref} geladen wurde. Das optionale 
Argument wird an \Macro{hypersetup}(\Package{hyperref})'none' aus besagtem
Paket übergeben und kann somit für zusätzliche Einstellungen genutzt werden. 
Mit der Sternversion \Macro{emailaddress*} erfolgt keine Formatierung des 
Eintrags im Argument.

Sollte das Paket \Package{isodate} oder \Package{datetime2} geladen sein, wird 
die damit eingestellte Formatierung des Datums durch \Macro{dateofbirth}~-- wie 
übrigens bei jedem anderem Datumsfeld der \TUDScript-Klassen auch~-- für das 
Geburtsdatum auf dem Titel verwendet. Hierfür wird entweder  
\Macro{printdate}(\Package{isodate})'none' von \Package{isodate} oder 
\Macro{DTMDate}(\Package{datetime2})'none' aus \Package{datetime2} genutzt. Mit 
dem Befehl \Macro{placeofbirth} lässt sich zusätzlich ein Geburtsort angeben.

Die weiteren Befehle als zusätzliche Angabe erklären sich quasi von selbst. 
Anzumerken ist, dass die mit den Befehlen \Macro{matriculationnumber},  
\Macro{matriculationyear}, \Macro{course} sowie \Macro{discipline} gemachten 
Angaben ebenfalls vom Paket \Package{tudscrsupervisor} innerhalb der 
\Environment{task}(\Package{tudscrsupervisor})"=Umgebung genutzt werden, falls 
diese denn zum Einsatz kommt.
\end{Declaration}
\end{Declaration}
\end{Declaration}
\end{Declaration}
\end{Declaration}
\end{Declaration}
\end{Declaration}
\end{Declaration}
\end{Declaration}
\end{Declaration}

\begin{Declaration}
  {\Macro{and}}
\printdeclarationlist[Kollaboratives~Schreiben|?]

Dieser Befehl wird sowohl bei den \hologo{LaTeX}"=Standardklassen als auch bei 
den \KOMAScript"=Klassen lediglich auf der Titelseite dazu verwendet, mehrere 
Autoren im Argument von \Macro{author} voneinander zu trennen.

Bei den \TUDScript-Klassen hingegen ist dieser Befehl derart in seiner Funktion 
erweitert worden, dass damit die Angabe einer kollaborativen Autorenschaft für 
Abschlussarbeiten innerhalb des Befehls \Macro{author} möglich ist. Außerdem 
kann er noch im Argument von \Macro{supervisor}, \Macro{referee} sowie 
\Macro{advisor} verwendet werden, um mehrere Betreuer beziehungsweise Gutachter 
und Fachreferenten anzugeben. Er ist dabei nicht auf die Verwendung für den 
Titel allein beschränkt sondern kann auch bei der Angabe von Personen in den 
entsprechenden Feldern der Umgebungen
\begin{Entity}{\Package{tudscrsupervisor}}
\Environment{task}, \Environment{evaluation} und \Environment{notice}
\end{Entity}
aus dem Paket \Package{tudscrsupervisor} eingesetzt werden.
\end{Declaration}
%
\begin{Example}
Angenommen, es soll eine Abschlussarbeit von zwei unterschiedlichen Autoren in 
kollaborativer Gemeinschaft erstellt werden, so lässt sich die Autorenangaben 
folgendermaßen gestalten:
\begin{Code}
\author{%
  Mickey Mouse%
  \matriculationnumber{12345678}%
  \dateofbirth{2.1.1990}%
  \placeofbirth{Dresden}%
\and%
  Donald Duck%
  \matriculationnumber{87654321}%
  \dateofbirth{1.2.1990}%
  \placeofbirth{Berlin}%
}
\matriculationyear{2010}
\end{Code}
Alle zusätzlichen Angaben außerhalb des Argumentes von \Macro{author} werden 
für beide Autoren gleichermaßen übernommen. Angaben innerhalb des Argumentes 
von \Macro{author} werden den jeweiligen, mit \Macro{and} getrennten Autoren 
zugeordnet. Mehr dazu ist im Minimalbeispiel in \autoref{sec:exmpl:thesis}.
\end{Example}

\begin{Declaration}
  {\Macro{date}[\OPList{Suffix}\MPName{Datum}]}
  [v2.05:Angabe von Parametern für Prä- und Suffix bei Datumsausgabe möglich]
\begin{Declaration}
  {\Macro{date*}[\OPList{Suffix}\MPName{Datum}]}
  [v2.05]
\begin{Declaration}
  {\Key{\Macro{date}}{before=\PName{Präfix}}}
  [v2.05]
\begin{Declaration}
  {\Key{\Macro{date}}{after=\PName{Suffix}}}
  [v2.05]
\begin{Declaration}
  {\Key{\Macro{date}}{place=\PName{Ort}}}
  [v2.05]
\begin{Declaration}
  {\Macro{defensedate}[\MPName{Verteidigungsdatum}]}
\printdeclarationlist[Datum|!,Datum!Abgabedatum|?,Datum!Verteidigungsdatum|?]

Mit dem Befehl \Macro{date} lässt sich das Datum angegeben. 
\ChangedAt{v2.05}%
Über das optionale Argument können die beiden Parameter 
\Key{\Macro{date}}{before} und \Key{\Macro{date}}{after} genutzt werden, um 
ergänzende Angaben vor beziehungsweise nach dem eigentlichen Datum auszugeben. 
Die Sternversion \Macro{date*} setzt den mit \Macro{place} angegebenen Ort vor 
das Datum. Dies geschieht auch für die normale Version von \Macro{date}, wenn 
der Parameter \Key{\Macro{date}}{place} verwendet wird.

Das Datum wird bei normalen Dokumenten direkt nach dem Autor respektive den 
Autoren ausgegeben. Bei Abschlussarbeiten~-- aktiviert durch die Verwendung von 
\Macro{thesis} oder \Option{subjectthesis} in Verbindung mit \Macro{subject}~-- 
erscheint dieses am Ende der Titelseite als Abgabedatum. Außerdem kann in 
diesem Fall mit dem Befehl\Macro{defensedate} das Datum der Verteidigung 
angegeben werden, wie es beispielsweise bei dem Druck von Dissertationen üblich 
ist.

Sollte eines der Pakete \Package{isodate} oder \Package{datetime2} geladen 
sein, so wird mit \Macro{printdate}(\Package{isodate})'none' beziehungsweise 
\Macro{DTMDate}(\Package{datetime2})'none' die durch das jeweilige Paket 
eingestellte Ausgabeformatierung des Datums für alle Datumsfelder des 
Dokumentes und folglich auch für die beiden Felder \Macro{date} und 
\Macro{defensedate} verwendet.
\end{Declaration}
\end{Declaration}
\end{Declaration}
\end{Declaration}
\end{Declaration}
\end{Declaration}

\begin{Declaration}
  {\Macro{thesis}[\MPName{Typisierung}]}
\begin{Declaration}
  {\Macro{subject}[\MPName{Typisierung}]}
\printdeclarationlist[Abschlussarbeit|!,Typisierung]

Mit diesen beiden Befehlen kann der Typ der Dokumentes beziehungsweise der 
Abschlussarbeit angegeben werden. Während der Befehl \Macro{thesis} den Inhalt 
des Feldes unter dem Titel vertikal zentriert auf der Titelseite ausgibt, 
erscheint der Inhalt des Befehls \Macro{subject} oberhalb des Titels. Es können 
auch beide Befehle parallel mit unterschiedlichen Inhalten verwendet werden. 
Der Befehl \Macro{thesis} dient den \TUDScript"=Dokumentklassen außerdem zur 
Erkennung von Abschlussarbeiten gedacht, da für diese spezielle Felder 
bereitgehalten werden und auch die Titelseite leicht geändert gesetzt wird.

Des Weiteren ist es bei beiden Befehlen möglich, spezielle Werte als Argument 
zur Typisierung des Dokumentes zu verwenden. Diese werden entsprechend der 
gewählten Dokumentensprache~-- entweder Deutsch oder Englisch~-- entschlüsselt 
und gesetzt. Die möglichen Werte sind \autoref{tab:thesis} zu entnehmen. Dabei 
ist zu beachten, dass das Setzen eines speziellen Wertes für \emph{entweder} 
\Macro{thesis} \emph{oder} \Macro{subject} möglich ist. Die Verwendung eines 
der genannten Werte führt immer dazu, dass das Dokument als Abschlussarbeiten 
erkannt und die erweiterte Titelseite aktiviert wird. Gleichzeitig wird damit 
die Option \Option{subjectthesis} beeinflusst. Sollte vom Anwender kein 
explizites Verhalten für \Option{subjectthesis} definiert sein, so führt die 
Verwendung von \Macro{thesis}[\MPName{Wert}] zu \Option{subjectthesis=false} 
und \Macro{subject}[\MPName{Wert}] zu \Option{subjectthesis=true}.
%
\begin{table}
\caption{%
  Spezielle Werte zur Typisierung des Dokumentes für
  \Macro{thesis} und \Macro{subject}%
  \label{tab:thesis}%
  \index{Bezeichner}%
  \index{Typisierung}%
}%
\centering%
\newcommand*\typecast[2]{%
  \PValue{#1} & \Term{#2} & \csuse{#2} & \selectlanguage{english}\csuse{#2}
  \tabularnewline%
}%
\begin{tabular}{llll}
  \toprule
  \textbf{Wert} & \textbf{Bezeichner} & \textbf{Deutsch} & \textbf{Englisch}
  \tabularnewline
  \midrule
  \typecast{habil}{habilitationname}  
  \typecast{diss}{dissertationname}
  \typecast{phd}{dissertationname}
  \typecast{diploma}{diplomathesisname}
  \typecast{master}{masterthesisname}
  \typecast{bachelor}{bachelorthesisname}
  \typecast{student}{studentthesisname}
  \typecast{evidence}{studentresearchname}
  \typecast{project}{projectpapername}
  \typecast{seminar}{seminarpapername}
  \typecast{term}{termpapername}
  \typecast{research}{researchname}
  \typecast{log}{logname}
  \typecast{report}{reportname}
  \typecast{internship}{internshipname}
  \bottomrule
\end{tabular}
\end{table}
\end{Declaration}
\end{Declaration}

\begin{Declaration}
  {\Option{subjectthesis=\PBoolean}}
  (false|\Macro{subject}[\MPName{\autoref{tab:thesis}}]:true)
\printdeclarationlist

Der Befehl \Macro{thesis} dient den \TUDScript"=Hauptklassen zur Unterscheidung 
zwei unterschiedlicher Ausprägungen der Titelseite und ist speziell für 
Abschlussarbeiten gedacht. Außerdem kann bei der Nutzung spezieller Werte 
aus \autoref{tab:thesis} innerhalb des Argumentes von \Macro{subject} ebenfalls 
das Verhalten für Abschlussarbeiten aktiviert werden, wobei hierdurch die 
Einstellung \Option{subjectthesis=true} automatisch vorgenommen wird.

Für den Standardfall~-- bekanntlich \Option{subjectthesis=false}~-- wird der 
durch \Macro{thesis} gegebene Typ der Abschlussarbeit sowie der gegebenenfalls 
durch \Macro{graduation} gesetzte angestrebte Abschluss in großen Lettern und 
sehr zentral auf der Titelseite gesetzt. Die Verwendung von \Macro{subject} ist 
hierbei weiterhin möglich.
%
Wird die Option mit \Option{subjectthesis=true} aktiviert, so wird die mit 
\Macro{thesis} gesetzte Bezeichnung nicht unterhalb sondern oberhalb des Titels 
an der Stelle von \Macro{subject} ausgegeben. Der mit \Macro{graduation} 
angegebene Abschluss wird weiterhin unter dem Titel, allerdings in schlankerer 
Schrift gesetzt. Eine etwaige Verwendung des Befehls \Macro{subject} wird in 
diesem Fall ignoriert.
%
\begin{values}{\Option{subjectthesis}}
\itemfalse
  Die Ausgabe des Typs der Abschlussarbeit (\Macro{thesis}) selbst sowie des 
  angestrebten Abschlusses (\Macro{graduation}) erfolgt in großen Lettern 
  zentral auf der Titelseite.
\itemtrue*
  Der Typ der Abschlussarbeit (\Macro{thesis}) wird oberhalb des Titels in der 
  Betreffzeile gesetzt. Der angestrebte Abschluss (\Macro{graduation}) wird 
  zentral ausgegeben.
\end{values}
\end{Declaration}

\begin{Declaration}
  {\Macro{graduation}[\OPName{Kurzform}\MPName{Grad}]}
  [v2.02]
\printdeclarationlist

Mit diesem Befehl wird der angestrebte akademische Grad auf der Titelseite 
ausgegeben. Da dies nur mit einer Abschlussarbeit erreicht werden kann erfolgt 
die Ausgabe nur, wenn entweder \Macro{thesis} oder \Macro{subject} verwendet 
wurde, wobei bei letzterem Befehl im Argument zwingend ein Wert aus 
\autoref{tab:thesis} verwendet werden muss.

Bei der Ausgabe des akademischen Grades hat die Option \Option{subjectthesis} 
Einfluss auf die Ausgabe auf der Titelseite. Bei \Option{subjectthesis=false} 
wird der Abschluss~-- ähnlich wie der Typ der Abschlussarbeit~-- zentral und in 
relativ großen Lettern gesetzt. Für \Option{subjectthesis=true} erfolgt die 
Ausgabe kleiner und in weniger starken Buchstaben.
\end{Declaration}

\begin{Declaration}
  {\Macro{supervisor}[\MPName{Name(n)}]}
\begin{Declaration}
  {\Macro{referee}[\MPName{Name(n)}]}
\begin{Declaration}
  {\Macro{advisor}[\MPName{Name(n)}]}
\begin{Declaration}
  {\Macro{professor}[\MPName{Name}]}
\printdeclarationlist[Betreuer|?,Gutachter|?,Referent|?]

Mit \Macro{supervisor}, \Macro{referee} und \Macro{advisor} werden die Betreuer 
einer Abschlussarbeit beziehungsweise die Gutachter und Fachreferenten einer 
Dissertation angegeben. Zusätzlich kann mit \Macro{professor} der betreuende 
Hochschullehrer beziehungsweise die betreuenden Professoren für studentische 
Arbeiten angegeben werden. Die Angabe mehrerer Person erfolgt wie beim Befehl 
\Macro{author} durch die Trennung mittels \Macro{and}.
\end{Declaration}
\end{Declaration}
\end{Declaration}
\end{Declaration}

\begin{Declaration}
  {\Option{titlesignature=\PBoolean}}
  (false)
  [v2.06]
\printdeclarationlist

Einige Lehrstühle an der \TnUD verlangen eine Unterschrift des Autors einer 
Abschlussarbeit direkt auf der Titelseite. Mit dem Aktivieren dieser Option 
wird ein solches Feld am unteren Seitenrand des Titels erzeugt.
\end{Declaration}

\begin{Declaration}
  {\Macro{titledelimiter}[\MPName{Trennzeichen}]}
\printdeclarationlist[Titel!Trennzeichen]

Für den Titel und die Umschlagseite werden durch die \TUDScript-Klassen eine 
Reihe von zusätzlichen Feldern bereitgestellt. Einigen dieser Felder wird eine 
Beschreibung (\seeref{\autoref{sec:localization}}) vorangestellt. Dazwischen 
wird bei der Ausgabe ein Trennzeichen eingefügt. Ein Doppelpunkt gefolgt von 
einem Leerzeichen (:\Macro*{nobreakspace}) ist hierfür die Voreinstellung. Mit 
dem Befehl \Macro{titledelimiter} lässt sich dieses Trennzeichen beliebig an 
die individuellen Wünsche des Anwenders anpassen.%
\end{Declaration}
\index{Titel!Felder|)}%


\begin{Entity}{\Bundle{koma-script}}
\begin{Declaration}
  {\Macro{extratitle}[\MPName{Schmutztitel}]}
\begin{Declaration}
  {\Macro{frontispiece}[\MPName{Frontispiz}]}
\begin{Declaration}
  {\Macro{titlehead}[\MPName{Kopf}]}
\begin{Declaration}
  {\Macro{publishers}[\MPName{Verlag}]}
\begin{Declaration}
  {\Macro{thanks}[\MPName{Fußnote}]}
\begin{Declaration}
  {\Macro{uppertitleback}[\MPName{Titelrückseitenkopf}]}
\begin{Declaration}
  {\Macro{lowertitleback}[\MPName{Titelrückseitenfuß}]}
\begin{Declaration}
  {\Macro{dedication}[\MPName{Widmung}]}
\printdeclarationlist

Diese Befehle entsprechen den in ihrem Verhalten den originalen Pendants der 
\KOMAScript"=Klassen und sollen hier der Vollständigkeit halber erwähnt werden.

Die Ausgabe des mit \Macro{extratitle} definierten Schmutztitels~-- welcher 
beliebig gestaltet und formatiert werden kann~-- und der gegebenenfalls mit 
\ChangedAt[\Macro{frontispiece}](){%
  v2.06:Schmutztitelrückseite aus \Bundle{koma-script} wird unterstützt%
}%
\Macro{frontispiece} definierten Rückseite erfolgt als Bestandteil der Titelei 
mit \Macro{maketitle} vor der eigentlichen Titelseite. Mit dem Befehl 
\Macro{titlehead} kann ein zusätzlicher, beliebig formatierbarer Text oberhalb 
der Typisierung und des Titels ausgegeben werden. Da die vertikale Position des 
Dokumenttitels durch das \CD fest vorgegeben ist, kann es~-- im Gegensatz zu 
den \KOMAScript"=Klassen~-- passieren, dass der Kopf des Haupttitels selbst in 
die Kopfzeile ragt. Dies wird durch die \TUDScript-Klassen nicht geprüft und 
muss gegebenenfalls vom Anwender kontrolliert werden. Der mit dem Befehl 
\Macro{publishers} definierte Inhalt muss nicht zwingende einen Verlag 
bezeichnen sondern kann auch andere Informationen beinhalten, welche am Ende 
der Titelseite ausgegeben werden sollen.

Fußnoten werden auf dem Titel nicht mit 
\Macro{footnote}(\Bundle{koma-script})'none', sondern mit der Anweisung 
\Macro{thanks} erzeugt. Diese dienen in der Regel für Anmerkungen bei Titel 
oder den Autoren. Als Fußnotenzeichen werden dabei Symbole statt Zahlen 
verwendet. Der Befehl \Macro{thanks} kann nur innerhalb des Arguments einer der 
Anweisungen für die Titelseite wie beispielsweise \Macro{author} oder 
\Macro{title} verwendet werden.

\index{Satzspiegel!doppelseitig}%
Im doppelseitigen Druck lässt sich die Rückseite der Haupttitelseite für 
weitere Angaben nutzen. Sowohl den Titelrückseitenkopf als auch den
Titelrückseitenfuß kann der Anwender mit \Macro{uppertitleback} und 
\Macro{lowertitleback} frei gestalten.

Mit \Macro{dedication} lässt eine separate Widmungsseite zentriert und in etwas 
größerer Schrift setzen. Die Rückseite ist~-- wie auch die des Schmutztitels~-- 
grundsätzlich leer. Die Widmung wird mit der restlichen Titelei ausgegeben und 
muss daher vor der Nutzung von \Macro{maketitle} angegeben werden.%
\end{Declaration}
\end{Declaration}
\end{Declaration}
\end{Declaration}
\end{Declaration}
\end{Declaration}
\end{Declaration}
\end{Declaration}
\index{Titel|!)}%
\end{Entity}



\subsection{%
  Die Teileseite%
  \label{sec:part}%
  \index{Teil|?(}%
  \index{Layout!Überschriften|(}%
}

\ChangedAt[\Macro{partpagestyle}(\Bundle{koma-script})]{%
  v2.02:Seitenstil \PageStyle{plain.tudheadings} wird genutzt%
}%
Wird für die Teileseiten das Layout des \CDs verwendet, so wird der Seitenstil 
dieser (\Macro{partpagestyle}(\Bundle{koma-script})'none') auf 
\PageStyle{plain.tudheadings} gesetzt. Möchten Sie stattdessen einen anderen 
Seitenstil nutzen, so kann dieser mit 
\Macro*{renewcommand*}[%
  \MPValue{\Macro{partpagestyle}(\Bundle{koma-script})'none'}%
  \MPName{Seitenstil}%
]
angepasst werden.

\begin{Declaration}
  {\Macro{setpartsubtitle}[\MPName{Untertitel}]}
  [v2.06]
\begin{Declaration}
  {\Font{partsubtitle}}
  [v2.06]
\printdeclarationlist[Schriftelemente]

Mit \Macro{setpartsubtitle}[\MPName{Untertitel}] kann für Teile nach der 
Überschrift selbst ein Untertitel gesetzt werden. Der Befehl muss in gleicher 
Weise wie \Macro{setpartpreamble}(\Bundle{koma-script})'none' \emph{vor} der 
Verwendung von \Macro{part}(\Bundle{koma-script})'none' oder den davon 
abgeleiteten Varianten angegeben werden. Mit dem Schriftelement 
\Font{partsubtitle} lässt sich die Schrift für den gegebenen Untertitel 
verändern. In \autoref{sec:fonts:elements} ist zu finden, wie es angepasst 
werden kann.
\end{Declaration}
\end{Declaration}

\begin{Declaration}
  {\Option{parttitle=\PBoolean}}
  (false)
\printdeclarationlist

Diese Option ermöglicht es, den mit \Macro{title} gegebenen Titel des 
Dokumentes selbst in großer Schrift auf einer Teileseite auszugeben, die 
Bezeichnung des mit 
\Macro{part}[\MPName{Bezeichnung}](\Bundle{koma-script})'none' erzeugten 
Teils wird in diesem Fall als Untertitel direkt darunter gesetzt. Diese 
Layout"=Variante findet sich im Handbuch für das \TUDCD. \notudscrartcl
%
\begin{values}{\Option{parttitle}}
\itemfalse
  Die Bezeichnung des Teils erscheint in großer Schrift, der Titel des 
  Dokumentes wird nicht ausgegeben.
\itemtrue*
  Der angegebene Titel wird in großer Auszeichnung auf der Teileseite gesetzt,
  die Bezeichnung des Teils selber als Untertitel.
\end{values}
\end{Declaration}
\index{Layout!Überschriften|)}%
\index{Teil|?)}%



\subsection{%
  Die Kapitelseite%
  \label{sec:chapter}%
  \index{Kapitel|?(}%
  \index{Layout!Überschriften|(}%
}

\begin{Declaration}
  {\Option{chapterpage=\PBoolean}}
  (false|\Option{cd=color}:true)
\printdeclarationlist[Satzspiegel!doppelseitig]

Mit dieser Einstellung kann die Überschrift eines Kapitels separat auf einer 
Seite ausgegeben werden. Der nachfolgende Text wird auf der nächsten 
beziehungsweise bei doppelseitigem Satz und rechts öffnenden Kapiteln%
\footnote{%
  \KOMAScript-Optionen \Option{twoside=true}(\Package{typearea})'none' und 
  \Option{open=right}(\Bundle{koma-script})'none', Standard für 
  \Class{tudscrbook}%
}
auf der übernächsten Seite ausgegeben. Die in diesem Fall erzeugte Rückseite 
wird in ihrer Ausprägung~-- wie auch Teileseiten~-- durch die Einstellung von 
\Option{cleardoublespecialpage} bestimmt. Beim farbigen Layout ist diese Option 
standardmäßig aktiviert. \notudscrartcl
%
\begin{values}{\Option{chapterpage}}
\itemfalse
  Es gibt keine Sonderstellung von Kapiteln, der nachfolgende Text wird direkt 
  unter der Überschrift respektive nach der mit 
  \Macro{setchapterpreamble}(\Bundle{koma-script})'none' erzeugten 
  Kapitelpräambel auf der gleichen Seite ausgegeben.
\itemtrue*
  \index{Vakatseiten}%
  Die Kapitelüberschrift und gegebenenfalls die Kapitelpräambel werden auf 
  einer separaten Seite gesetzt. Der folgende Text erscheint auf der nächsten 
  respektive übernächsten Seite, \seeref*{\Option{cleardoublespecialpage}}.
\end{values}
%
\ChangedAt[\Macro{chapterpagestyle}(\Bundle{koma-script})]{%
  v2.02:nicht mehr abhängig von \Macro{partpagestyle}(\Bundle{koma-script})
}%
Mit \Macro*{renewcommand*}[%
  \MPValue{\Macro{chapterpagestyle}(\Bundle{koma-script})'none'}%
  \MPName{Seitenstil}%
]
lässt sich übrigens~-- unabhängig von der Option \Option{chapterpage}~-- der 
Seitenstil von Kapiteln anpassen. Bei der Verwendung von Kapitelseiten mit 
\Option{chapterpage=true} ist außerdem das Aktivieren der \KOMAScript-Option 
\Option{chapterprefix=\PBoolean}(\Bundle{koma-script})'none'|declare| 
empfehlenswert. Damit werden die Kapitelüberschriften mit einer Vorsatzzeile 
gesetzt. Wird ein nummeriertes Kapitel erzeugt, so wird zunächst in einer Zeile 
\enquote{Kapitel} gefolgt von der aktuellen Kapitelnummer ausgegeben, in der 
nächsten Zeile wird anschließend die eigentliche Überschrift in linksbündigem 
Flattersatz ausgegeben. Mehr dazu ist der \scrguide[\KOMAScript-Dokumentation] 
zu entnehmen.
\end{Declaration}

\begin{Declaration}
  {\Macro{setchaptersubtitle}[\MPName{Untertitel}]}
  [v2.06]
\begin{Declaration}
  {\Font{chaptersubtitle}}
  [v2.06]
\printdeclarationlist[Schriftelemente]

Mit \Macro{setchaptersubtitle}[\MPName{Untertitel}] kann für Kapitel nach 
der Überschrift selbst ein Untertitel gesetzt werden. Der Befehl muss in 
gleicher Weise wie \Macro{setchapterpreamble}(\Bundle{koma-script})'none' 
\emph{vor} der Verwendung von \Macro{chapter}(\Bundle{koma-script})'none' oder 
den davon abgeleiteten Varianten angegeben werden. Mit dem Schriftelement 
\Font{chaptersubtitle} lässt sich die Schrift für den gegebenen Untertitel 
verändern. In \autoref{sec:fonts:elements} ist zu finden, wie es angepasst 
werden kann.%
\end{Declaration}
\end{Declaration}
\index{Layout!Überschriften|)}%
\index{Kapitel|?)}%



\subsection{%
  Vakatseiten%
  \label{sec:vacat}%
  \index{Vakatseiten|(}%
}

Automatisch erzeugte Vakatseiten~-- auch absichtliche Leerseiten genannt~-- 
sind in Abhängigkeit der 
\KOMAScript-Optionen \Option{twoside=\PMisc}(\Package{typearea})'none'|declare| 
und \Option{open=\PName{Methode}}(\Bundle{koma-script})'none'|declare| am 
Beginn von Teilen und Kapiteln in Dokumenten zu finden.%
\footnote{%
  Voreinstellungen
  \Class{tudscrbook}: 
  \Option{twoside=true}(\Package{typearea})'none', 
  \Option{open=right}(\Bundle{koma-script})'none';
  \Class{tudscrartcl}, \Class{tudscrreprt}: 
  \Option{open=any}(\Bundle{koma-script})'none', 
  \Option{twoside=false}(\Package{typearea})'none'%
}
Für die Vakatseiten kann der Seitenstil mit der \KOMAScript"=Option 
\Option{cleardoublepage=\PMisc}(\Bundle{koma-script})'none'|declare| 
eingestellt werden.

\begin{Declaration}
  {\Option{cleardoublespecialpage=\PMisc}}
  (true)
  [v2.06:Einstellung für farbige Rückseiten]
\printdeclarationlist
  [Titel,Teil,Kapitel,Layout!Farbgestaltung,Satzspiegel!doppelseitig,Seitenstil]

Diese Option wirkt sich lediglich bei aktiviertem doppelseitigem Satz und 
ausschließlich rechts eröffnenden Seiten für Teile beziehungsweise Kapitel
aus.%
\footnote{%
  \KOMAScript"=Option \Option{twoside=true}(\Package{typearea})'none' und 
  \Option{open=right}(\Bundle{koma-script})'none'%
}
In diesem Fall kann der Stil der darauffolgenden, linken Seite~-- sprich der 
Rückseite~-- beeinflusst werden. Das Normalverhalten sieht vor, dass nach einem 
Teil die nachfolgende Rückseite unabhängig von der Einstellung für 
\Option{cleardoublepage}(\Bundle{koma-script})'none' immer als vollständig 
leere Seite ohne Kopf"~ oder Fußzeilen gesetzt wird.

Diese Option erlaubt es, das Normalverhalten zu deaktivieren und für die Seite 
nach der Teileseite~-- und abhängig von \Option{chapterpage} auch nach einem 
Kapitelanfang auf einer separaten Seite~-- den Seitenstil der Option 
\Option{cleardoublepage}(\Bundle{koma-script})'none' zu übernehmen. Des 
Weiteren kann auch ein anderer, bereits definierter Seitenstil gewählt werden. 
Außerdem kann im farbigen Layout die Rückseite in der gleichen Farbe wie die 
Vorderseite von Titel, Teil oder Kapitel gesetzt werden. 
\notudscrartcl
%
\begin{values}{\Option{cleardoublespecialpage}}
\itemfalse
  Die Rückseiten sind vollständig leere Seiten, unabhängig von Option
  \Option{cleardoublepage}(\Bundle{koma-script})'none'.
\itemtrue*
  Der Seitenstil der Rückseite von Teilen und gegebenenfalls Kapiteln entspricht
  der Einstellung von \Option{cleardoublepage}(\Bundle{koma-script})'none' für 
  Vakatseiten.
\item[current]
  Für die erzeugte Rückseite wird der aktuell definierte Seitenstil 
  (\Macro{pagestyle}(\Bundle{koma-script})'none') verwendet.
\item[\PSet{Seitenstil}]
  Mit der Angabe von \Option{cleardoublespecialpage=\PName{Seitenstil}} 
  kann ein beliebiger, bereits definierter Seitenstil für die Rückseite nach 
  Teilen und Kapiteln verwendet werden.
\item[color]
  Im farbigen Layout ist auch die Rückseite von Teilen und Kapiteln farbig,  
  \seeref{\Option{cd}}. Die Einstellung wirkt sich ebenfalls auf die Rückseite 
  des Titels aus.\footnote{%
    \seeref{\Macro{uppertitleback}(\Bundle{koma-script}) und 
    \Macro{lowertitleback}(\Bundle{koma-script})} im \scrguide*%
  }%
\item[nocolor]
  Es werden weiße Rückseiten bei Titel, Teilen und gegebenenfalls Kapiteln 
  erzeugt.%
\end{values}
\end{Declaration}
\index{Vakatseiten|)}%



\subsection{%
  Verwendung von Schriftelementen%
  \label{sec:fonts:elements}%
  \index{Schriftelemente|!}%
}

Vom \TUDScript-Bundle werden weitere Schriftelemente~-- in Ergänzung zu den 
bereits durch \KOMAScript{} bereitgestellten~-- definiert. Dies sind 
\Font{titlepage}, \Font{thesis}, \Font{tudheadings} sowie \Font{partsubtitle}
und \Font{chaptersubtitle}. Sowohl die durch \KOMAScript{} definierten als auch 
alle hier genannten und folgend erläuterten Schriftelemente lassen sich im 
Bedarfsfall über den Befehl \Macro{addtokomafont}[%
  \MPName{Schriftelement}\MPName{Einstellungen}%
](\Bundle{koma-script})'none' angepassen. Mehr dazu ist im \scrguide innerhalb 
des Abschnitts \emph{Textauszeichnungen} zu finden.



\subsection{%
  Verwendung von Feldinhalten%
  \index{Felder|!}%
}

\begin{Declaration}
  {\Macro{getfield}[\MPName{Feldname}]}
  [v2.06]
\printdeclarationlist

Mit diesem Makro kann auf den Inhalt eines zuvor angegebenen Feldes zugegriffen 
werden.
\begin{Example}
Das für dieses Dokument angegebene Datum lautet \enquote{\getfield{date}}. Es 
kann wie folgt ausgegeben werden:
\begin{Code}
\getfield{date}
\end{Code}
\end{Example}
Weitere mögliche Argumente sind beispielsweise \PValue{faculty}, 
\PValue{department}, \PValue{institute} und \PValue{chair} sowie \PValue{title},
\PValue{author}, \PValue{thesis}, \PValue{professor}, \PValue{supervisor} oder
\PValue{place}.
\end{Declaration}



\subsection{Die Farben des \CDs}

Zur Verwendung der Farben des \CDs wird das Paket \Package{tudscrcolor} 
genutzt. Falls dieses nicht in der Präambel geladen wird~-- um beispielsweise 
zusätzliche Optionen aufzurufen~-- binden die \TUDScript"=Klassen dieses 
automatisch ein. Detaillierte Informationen sind in der Dokumentation von 
\Package{tudscrcolor}'full' zu finden.%
\index{Layout|)}%



\section{Zusätzliche Optionen und Erweiterungen}

\ChangedAt*{%
  v2.03:Bugfix für \abstractname, \confirmationname{} und \blockingname{} 
    bei der Festlegung von Seitenstil und Kolumnentitel%
}%
Neben den Befehlen für die Anpassung des Layouts an das \TUDCD stellen die 
\TUDScript-Klassen weitere Befehle und Umgebungen zur Verfügung, um die 
Anwendung insbesondere für wissenschaftliche Arbeiten zu erleichtern.



\subsection{%
  Zusammenfassung/Kurzfassung%
  \index{Zusammenfassung|!(}%
}

\begin{Declaration}
  {\Option{abstract=\PMisc}}
[%
  v2.02!\Option{abstract=multiple}:ersetzt \Option{abstract=double};%
  v2.02!\Option{abstract=tocleveldown};%
  v2.02!\Option{abstract=markboth};%
  v2.04!\Option{abstract=tocmultiple}%
]
\printdeclarationlist

Diese Option wird bereits durch \KOMAScript{} für die Klassen \Class{scrartcl} 
und \Class{scrreprt} standardmäßig bereitgestellt. Für die Klasse 
\Class{scrbook} geschieht dies nicht. Dazu heißt es im Handbuch:
%
\begin{quoting}
Bei Büchern wird in der Regel eine andere Art der Zusammenfassung verwendet. 
Dort wird ein entsprechendes Kapitel an den Anfang oder Ende des Werks gesetzt. 
Oft wird diese Zusammenfassung entweder mit der Einleitung oder einem weiteren 
Ausblick verknüpft. Daher gibt es bei \Class{scrbook} generell keine 
\Environment{abstract}"=Umgebung. Bei Berichten im weiteren Sinne, etwa einer 
Studien- oder Diplomarbeit, ist ebenfalls eine Zusammenfassung in dieser Form 
zu empfehlen.
\end{quoting}
%
Durch die \TUDScript-Klassen wird die \Option{abstract}"=Option erweitert. 
Neben den Auswahlmöglichkeit, welche bereits \KOMAScript{} für die Klassen 
\Class{tudscrartcl} und \Class{tudscrreprt} anbietet, kann die Überschrift für 
die Zusammenfassung außerdem in Gestalt eines \sectionautorefname{}s oder für 
\Class{tudscrreprt} und \Class{tudscrbook} in der Form eines 
\chapterautorefname{}s ausgegeben werden.%
\index{Layout!Überschriften}%
%
\begin{values}{\Option{abstract}}
\itemfalse(nur \Class{tudscrartcl} und \Class{tudscrreprt})
  Es wird keine Überschrift für die \Environment{abstract}"=Umgebung ausgegeben.
\itemtrue*(nur \Class{tudscrartcl} und \Class{tudscrreprt})
  Wie bei den \KOMAScript"=Klassen wird eine zentrierte Überschrift mit dem 
  Bezeichner \Term{abstractname} vor der eigentlichen Zusammenfassung gesetzt.
\item[section/addsec]
  Die Überschrift (\Term{abstractname}) verwendet den Gliederungsbefehl 
  \Macro{section}(\Bundle{koma-script})'none'.
\item[chapter/addchap][\Class{tudscrbook}](%
    nur \Class{tudscrreprt} und \Class{tudscrbook}%
  )
  Es wird der Befehl \Macro{chapter}(\Bundle{koma-script})'none' für das 
  Setzen der Überschrift (\Term{abstractname}) genutzt. 
\item[heading]
  Es wird die höchstmögliche Gliederungsebene verwendet. Für 
  \Class{tudscrartcl} entspricht dies \Option{abstract=section}, bei 
  \Class{tudscrreprt} und \Class{tudscrbook} \Option{abstract=chapter}.
\end{values}
%
Abhängig von der gewählten Gliederungsebene der Überschrift wird das Verhalten 
für das Setzen eines Eintrages ins Inhaltsverzeichnis festgelegt. Ohne oder mit 
zentrierter Überschrift wird per Voreinstellung kein Eintrag erzeugt. Wird die 
Überschrift jedoch in Form einer Gliederungsebene gewählt, so erscheint die 
Zusammenfassung für gewöhnlich im Inhaltsverzeichnis auf der obersten Ebene. 
Das voreingestellte Verhalten für die Einträge ins Inhaltsverzeichnis kann 
jederzeit mit folgenden Werten durch den Anwender überschrieben werden.%
\index{Inhaltsverzeichnis}%
%
\begin{values}{\Option{abstract}}
\item[notoc/nottotoc]
  Die Zusammenfassung wird definitiv nicht ins Inhaltsverzeichnis eingetragen.
\item[toc/totoc]
  Es wird auf der obersten Gliederungsebene der aktuell verwendeten 
  Dokumentklasse (\Macro*{chapter} oder \Macro*{section}) ein nicht 
  nummerierten Eintrag im Inhaltsverzeichnis für die Zusammenfassung gesetzt.
\item[tocleveldown/leveldown/totocleveldown]
  \ChangedAt{v2.02}%
  Der Inhaltsverzeichniseintrag wird eine Stufe unterhalb der obersten 
  Gliederungsebene (\Macro*{section} oder \Macro*{subsection}) erzeugt.
\item[tocmultiple/totocmultiple/tocaggregate/totocaggregate]
  \ChangedAt{v2.04}%
  Es wird ein \emph{einziger} Inhaltsverzeichniseintrag für \emph{alle} 
  Zusammenfassungen erstellt.
\end{values}
%
\ChangedAt{v2.02}%
Außerdem kann das Verhalten für die Kolumnentitel durch den Nutzer beeinflusst 
werden. Diese werden normalerweise automatisch gesetzt, wenn diese über die 
\KOMAScript-Option \Option{automark}(\Package{scrlayer-scrpage})'none' 
aktiviert wurden und sind von der aktuellen Gliederungsebene der Überschrift 
abhängig. Werden jedoch mit Hilfe der \KOMAScript-Option 
\Option{manualmark}(\Package{scrlayer-scrpage})'none' manuelle Kolumnentitel 
genutzt, müssen diese normalerweise auch für die Zusammenfassung durch den 
Anwender gesetzt werden. Mit \Option{abstract=markboth} lässt sich allerdings 
das Setzen der Kolumnentitel unabhängig davon forcieren.%
\index{Kolumnentitel}%
%
\begin{values}{\Option{abstract}}
\item[markboth]
  Unabhängig von der Verwendung manueller oder automatischer Kolumnentitel 
  werden diese auf rechten sowie linken Seiten mit \Term{abstractname} gesetzt.
\item[nomarkboth]
  Die Einstellung für manuelle oder automatische Kolumnentitel werden beachtet 
  und abhängig von der verwendeten Gliederungsebene der Überschrift gesetzt.
\end{values}
%
Mit dem optionalen Parameter \Key{\Environment{abstract}}{markboth} der 
\Environment{abstract}"=Umgebung kann der Kolumnentitel mit einem beliebigen 
Inhalt gesetzt werden.

Häufig wird für Abschlussarbeiten verlangt, neben der deutschsprachigen auch 
noch eine englischsprachige Zusammenfassung zu verfassen. Mit der Einstellung 
\Option{abstract=multiple} lassen sich mehrere Zusammenfassungen auf einer 
Seite ausgeben~-- sofern genügend Platz vorhanden ist. Außerdem kann die 
standardmäßige vertikale Zentrierung der \Environment{abstract}"=Umgebung 
auf einer Seite unterdrückt werden. Diese Einstellungen zur Positionierung der 
Zusammenfassungen innerhalb der \Environment{abstract}"=Umgebung werden nur 
wirksam, wenn eine Titelseite
(\KOMAScript-Option \Option{titlepage=true}(\Bundle{koma-script})'none') und 
\emph{keine} Überschriften in Form von Kapiteln (\Option{abstract=chapter}) 
verwendet werden.
%
\begin{values}{\Option{abstract}}
\item[single/one/simple]
  Jede Zusammenfassung wird auf einer eigenen Seite
  beziehungsweise im zweispaltigen Satz in einer neuen Spalte ausgegeben.
\item[multiple/multi/all/aggregate]
  \ChangedAt{v2.02}%
  Zusammenfassungen, welche mit \Macro{nextabstract} getrennt wurden, werden 
  direkt nacheinander auf der gleichen Seite ausgegeben, wenn ausreichend Platz 
  auf dieser vorhanden sein sollte. Ist die Klassenoption 
  \Option{twocolumn}(\Package{typearea})'none' aktiviert, erfolgt die Ausgabe 
  aller Zusammenfassungen ohne Spaltenumbruch.
\item[fill/fil/vfil/vfill]
  Alle Zusammenfassungen werden bei der Ausgabe auf einer Seite vertikal 
  zentriert. Diese Einstellung steht für den zweispaltigen Satz
  (Klassenoption \Option{twocolumn}(\Package{typearea})'none') nicht zur 
  Verfügung.
\item[nofill/nofil/novfil/novfill]
  Die Ausgabe erfolgt wie im normalen Fließtext auch.
\end{values}
\end{Declaration}

\begin{Declaration}
  {\Environment{abstract}[\OPList{Sprache}]}
  [v2.02:Trennung einzelner Abschnitte mit \Macro{nextabstract}]
\begin{Declaration}
  {\Macro{nextabstract}[\OPList{Sprache}]}
\begin{Declaration}
  {\Key{\Environment{abstract}}{language=\PName{Sprache}}}
\begin{Declaration}
  {\Key{\Environment{abstract}}{markboth=\PMisc}}
  [v2.02]
\begin{Declaration}
  {\Key{\Environment{abstract}}{pagestyle=\PName{Seitenstil}}}
  [v2.02]
\begin{Declaration}
  {\Key{\Environment{abstract}}{columns=\PName{Anzahl}}}
\begin{Declaration}
  {\Key{\Environment{abstract}}{option=\PMisc}}
  <\seeref{\Option{abstract}'ppage'}>
\printdeclarationlist

Die \Environment{abstract}"=Umgebung dient speziell für die Ausgabe einer 
Zusammenfassung, entweder zu Beginn eines Dokumentes oder beispielsweise vor 
einem Teil oder Kapitel. Wird ein Titelkopf und keine Titelseite verwendet 
(\KOMAScript-Option \Option{titlepage=false}(\Bundle{koma-script})'none'), so 
wird eine Zusammenfassung~-- identisch zu den \KOMAScript"=Klassen~-- in einer 
\Environment{quotation}(\Bundle{koma-script})'none'"=Umgebung ausgegeben, 
wobei die Überschrift \emph{nicht} in der Form einer Gliederungsebene gesetzt 
wird. Diese Umgebung hat jedoch den Nachteil, dass die \KOMAScript-Option 
\Option{parskip=\PName{Methode}}(\Bundle{koma-script})'none' nicht beachtet 
wird. Um dies zu beheben, kann das Paket \Package{quoting} geladen werden, 
wodurch stattdessen die Umgebung \Environment{quoting}(\Package{quoting})'none' 
verwendet wird.

Mit der zuvor erläuterten Option \Option{abstract} kann eingestellt werden, in 
welcher Gestalt die Zusammenfassung ausgegeben werden soll. Des Weiteren lässt 
sich jede \Environment{abstract}"=Umgebung individuell über weitere Parameter 
als optionales Argument anpassen. Damit lassen sich gegebenenfalls für eine 
bestimmte \Environment{abstract}"=Umgebung die globalen Einstellungen 
der Option \Option{abstract} lokal ändern und gezielt anpassen. 

\index{Sprachunterstützung}%
Wird das Paket \Package{babel} durch den Anwender geladen, kann mit dem 
optionalen Parameter \Key{\Environment{abstract}}{language=\PName{Sprache}}die 
Sprache innerhalb der \Environment{abstract}"=Umgebung geändert werden. Dafür 
muss die gewünschte Sprache bereits mit dem Laden von \Package{babel} entweder 
als Paketoption oder besser noch als Klassenoption angegeben worden sein. 
Dadurch werden innerhalb der Umgebung die Bezeichnung \Term{abstractname} und 
die Trennungsmuster sprachspezifisch angepasst. Die gewünschte Sprache kann 
auch ohne die Verwendung des Parameters \Key{\Environment{abstract}}{language} 
direkt als optionales Argument übergeben werden.

\ChangedAt{v2.02}%
\index{Kolumnentitel}%
Mit \Key{\Environment{abstract}}{markboth} können die gesetzten Kolumnentitel 
beeinflusst werden. Wird \Key{\Environment{abstract}}{markboth=false} 
angegeben, werden automatische respektive manuelle Kolumnentitel verwendet. Die 
Einstellung \Key{\Environment{abstract}}{markboth=true} wiederum setzt diese 
für linke und rechte Seiten auf \Term{abstractname}. Zusätzlich lässt sich mit 
\Key{\Environment{abstract}}{markboth=\PSet{Kolumnentitel}} der 
Kolumnentitel direkt festlegen. So können die Kolumnen beispielsweise mit der 
Verwendung von \Key{\Environment{abstract}}{markboth=\MPValue{}} auch 
gelöscht werden. Sollte \Key{\Environment{abstract}}{markboth} aktiviert 
werden, so wird in der Umgebung automatisch der Seitenstil \PageStyle{headings} 
genutzt~-- falls eine Titelseite
(\KOMAScript-Option \Option{titlepage=true}(\Bundle{koma-script})'none') 
verwendet wird. Mit dem Parameter \Key{\Environment{abstract}}{pagestyle} kann 
dieser auch manuell angegeben werden, die \PageStyle{tudheadings}"=Seitenstile 
werden dabei ebenfalls unterstützt werden.

\index{Satzspiegel!mehrspaltig}%
Wurde das Paket \Package{multicol} geladen, kann mit dem Parameter 
\Key{\Environment{abstract}}{columns=\PName{Anzahl}} die Zusammenfassung 
mehrspaltig gesetzt werden. Dem Parameter \Key{\Environment{abstract}}{option} 
können alle gültigen, bereits erläuterten Werte der Option \Option{abstract} 
übergeben werden. Die damit gemachten Einstellungen wirken sich~-- im Gegensatz 
zur Variante der späten Optionenwahl%
\footnote{%
  \Macro{TUDoption}[\MPValue{abstract}\MPName{Einstellung}] oder
  \Macro{TUDoptions}[\MPValue{abstract=\PName{Einstellung}}]%
}
respektive als Angabe einer Klassenoption~-- lediglich lokal auf die verwendete 
\Environment{abstract}"=Umgebung aus.

\ChangedAt{v2.02}%
Sollen mehrere Zusammenfassungen erzeugt und dabei die Einstellungen 
\Option{abstract=single} beziehungsweise \Option{abstract=multiple} sowie 
\Option{abstract=fill} respektive \Option{abstract=nofill} beachtet werden, so 
ist die \Environment{abstract}"=Umgebung nur einmal zu verwenden. Innerhalb 
dieser müssen die einzelnen Zusammenfassungen jeweils mit \Macro{nextabstract} 
voneinander getrennt werden. Der Befehl akzeptiert dabei im optionalen Argument 
alle Parameter, welche auch von der \Environment{abstract}"=Umgebung selbst 
unterstützt werden. Das Minimalbeispiel in \fullref{sec:exmpl:dissertation} 
zeigt hierfür das notwendige Vorgehen.

Wird die \Environment{abstract}"=Umgebung innerhalb des Argumentes der Befehle 
\Macro{setpartpreamble}(\Bundle{koma-script})'none' beziehungsweise 
\Macro{setchapterpreamble}(\Bundle{koma-script})'none' verwendet, so wird die 
Überschrift~-- im Fall, dass nicht \Option{abstract=false} gewählt ist~-- 
\emph{immer} in Textgröße und zentriert gesetzt.
\end{Declaration}
\end{Declaration}
\end{Declaration}
\end{Declaration}
\end{Declaration}
\end{Declaration}
\end{Declaration}


\minisec{%
  Umbenennung der \abstractname%
  \index{Bezeichner}%
}

Mit dem \KOMAScript-Befehl \Macro{renewcaptionname}(\Bundle{koma-script})'none'
kann der Bezeichner~-- sprich der Wortlaut~-- der für die 
\Environment{abstract}"=Umgebung verwendeten Überschrift verändert werden. Mehr 
dazu ist in \autoref{sec:localization} zu finden.
%
\begin{Example}
Die Überschrift der \Environment{abstract}"=Umgebung soll für die Sprache 
\PValue{ngerman} von \enquote{\abstractname} in \enquote{Kurzfassung} umbenannt 
werden. Das Makro \Macro{renewcaptionname}(\Bundle{koma-script})'none' 
erwartet die drei obligatorischen Argumente 
\MPName{Sprache}\MPName{Makro}\MPName{Inhalt}:
\begin{Code}
\renewcaptionname{ngerman}{\abstractname}{Kurzfassung}
\end{Code}
\end{Example}
\index{Zusammenfassung|!)}%



\subsection{%
  Selbstständigkeitserklärung und Sperrvermerk%
  \index{Selbstständigkeitserklärung|!(}%
  \index{Sperrvermerk|!(}%
}

\begin{Declaration}
  {\Option{declaration=\PMisc}}
  (true)
[%
  v2.02!\Option{declaration=multiple}:ersetzt \Option{declaration=double};%
  v2.02!\Option{declaration=tocleveldown};%
  v2.02!\Option{declaration=markboth};%
  v2.04!\Option{declaration=tocmultiple}%
]
\printdeclarationlist

Mit \Option{declaration} kann äquivalent zur Option \Option{abstract} die 
Gestaltung von Selbstständigkeitserklärung und Sperrvermerk angepasst werden.
Zur Ausgabe der Erklärungen werden die Umgebung \Environment{declarations} 
sowie die Befehle \Macro{declaration} beziehungsweise \Macro{confirmation} und 
\Macro{blocking} bereitgestellt. 

Die beiden Optionen \Option{abstract} und \Option{declaration} ähneln sich sehr 
stark. Alle möglichen Wertzuweisungen für \Option{declaration} wurden bereits 
bei der Beschreibung von \Option{abstract} ausführlich erläutert. Deshalb 
geschieht dies hier in einer etwas kürzeren Ausführung. Sollte Ihnen eine 
Erläuterung etwas dürftig erscheinen, so hilft mit Sicherheit ein Blick zur 
Erklärung der Option \Option{abstract}'full'.

Die möglichen Werte für die Gestaltung der Überschrift werden nachfolgend 
genannt. Im Gegensatz zur Option \Option{abstract} stehen die beiden 
Einstellungen \Option{declaration=true} und \Option{declaration=false} auch für 
die Klasse \Class{tudscrbook} zur Verfügung.%
\index{Layout!Überschriften}%
%
\begin{values}{\Option{declaration}}
\itemfalse
  Es wird keine Überschrift über den Erklärungen selbst ausgegeben.
\itemtrue*
  Eine zentrierte Überschrift mit dem Bezeichner \Term{confirmationname} vor 
  der Selbstständigkeitserklärung beziehungsweise \Term{blockingname} vor dem 
  Sperrvermerk wird gesetzt. 
\item[section/addsec]
  Die Überschrift verwendet den Gliederungsbefehl 
  \Macro{section}(\Bundle{koma-script})'none'.
\item[chapter/addchap][\Class{tudscrbook}](%
    nur \Class{tudscrreprt} und \Class{tudscrbook}%
  )
  Es wird der Befehl \Macro{chapter}(\Bundle{koma-script})'none' für das 
  Setzen der Überschrift genutzt. 
\item[heading]
  Es wird die höchstmögliche Gliederungsebene verwendet. Für 
  \Class{tudscrartcl} entspricht dies \Option{declaration=section}, bei 
  \Class{tudscrreprt} und \Class{tudscrbook} \Option{declaration=chapter}.
\end{values}
%
Abhängig von der gewählten Gliederungsebene der Überschrift wird das Verhalten 
für das Setzen eines Eintrages ins Inhaltsverzeichnis festgelegt. Normalerweise 
wird nur für Überschriften in Form einer Gliederungsebene ein Eintrag der 
Erklärung ins Inhaltsverzeichnis erstellt, für \Option{declaration=true} und 
\Option{declaration=false} geschieht dies standardmäßig nicht. Mit folgenden 
Werten kann das voreingestellte Verhalten überschrieben werden.%
\index{Inhaltsverzeichnis}%
%
\begin{values}{\Option{declaration}}
\item[notoc/nottotoc]
  Die Erklärung wird definitiv nicht ins Inhaltsverzeichnis eingetragen.
\item[toc/totoc]
  Unabhängig von der Wahl der Überschrift erhält jede Erklärung einen nicht
  nummerierten Eintrag im Inhaltsverzeichnis auf der obersten Gliederungsebene 
  der verwendeten Dokumentklasse (\Macro*{chapter} oder \Macro*{section}). 
\item[tocleveldown/leveldown/totocleveldown]
  \ChangedAt{v2.02}%
  Der Inhaltsverzeichniseintrag wird eine Stufer unter der obersten 
  Gliederungsebene (\Macro*{section} oder \Macro*{subsection}) erzeugt.
\item[tocmultiple/totocmultiple/tocaggregate/totocaggregate]
  \ChangedAt{v2.04}%
  Es wird ein \emph{einziger} Inhaltsverzeichniseintrag für \emph{alle} 
  Erklärungen erstellt.
\end{values}
%
\ChangedAt{v2.02}%
Normalerweise werden automatische Kolumnentitel abhängig von der 
Gliederungsebene der Überschrift gesetzt, falls diese über die 
\KOMAScript-Option \Option{automark}(\Package{scrlayer-scrpage})'none' 
aktiviert sind. Werden manuelle Kolumnentitel genutzt, müssen diese auch für 
die Erklärungen manuell gesetzt werden. Mit \Option{declaration=markboth} lässt 
sich unabhängig davon das Setzen der Kolumnentitel auf linken und rechten 
Seiten forcieren, wobei hierfür der Titel der Überschrift genutzt wird.%
\index{Kolumnentitel}%
%
\begin{values}{\Option{declaration}}
\item[markboth]
  Unabhängig von der Verwendung manueller oder automatischer Kolumnentitel 
  werden diese auf rechten sowie linken Seiten mit den Bezeichnern 
  \Term{confirmationname} beziehungsweise \Term{blockingname} gesetzt.
\item[nomarkboth]
  Die Einstellung für manuelle oder automatische Kolumnentitel werden beachtet.
\end{values}
%
Für \Macro{declaration} respektive \Macro{confirmation} und \Macro{blocking} 
sowie die \Environment{declarations}"=Umgebung lässt sich mit dem Parameter 
\Key{\Environment{declarations}}{markboth} ein beliebiger Kolumnentitel setzen. 

Die folgenden Einstellungen zur Positionierung der Erklärungen haben lediglich 
Auswirkungen, wenn die Überschrift der Erklärung \emph{nicht} im Form eines 
Kapitels ausgegeben und mit der \KOMAScript-Option 
\Option{titlepage=true}(\Bundle{koma-script})'none' eine Titelseite verwendet 
wird.
%
\begin{values}{\Option{declaration}}
\item[single/one/simple]
  Jede Erklärung wird auf einer separaten Seite
  beziehungsweise im zweispaltigen Satz in einer neuen Spalte ausgegeben.
\item[multiple/multi/all/aggregate]
  \ChangedAt{v2.02}%
  Erklärungen, welche in der \Environment{declarations}"=Umgebung mit den 
  Befehlen \Macro{confirmation}, \Macro{blocking} und \Macro{declaration} oder 
  außerhalb dieser mit \Macro{declaration} gesetzt wurden, werden direkt 
  nacheinander auf der gleichen Seite ausgegeben, wenn ausreichend Platz auf 
  dieser vorhanden sein sollte. Ist die Klassenoption 
  \Option{twocolumn}(\Package{typearea})'none' aktiviert, erfolgt die Ausgabe 
  aller Erklärungen ohne Spaltenumbruch.
\item[fill/fil/vfil/vfill]
  Alle Erklärungen auf einer Ausgabeseite werden vertikal zentriert. Für 
  den zweispaltigen Satz 
  (Klassenoption \Option{twocolumn}(\Package{typearea})'none') steht diese 
  Einstellung nicht zur Verfügung.
\item[nofill/nofil/novfil/novfill]
  Die Ausgabe erfolgt wie im normalen Fließtext auch.
\end{values}
\end{Declaration}

\begin{Declaration}
  {\Environment{declarations}[\OPList{Sprache}]}
[%
  v2.02;%
  v2.04:Trennung einzelner Abschnitte mit \Macro{nextdeclaration};%
]
\begin{Declaration}
  {\Macro{nextdeclaration}[\OPList{Sprache}\MPName{Überschrift}\MPName{Text}]}
\begin{Declaration}
  {\Key{\Environment{declarations}}{language=\PName{Sprache}}}
\begin{Declaration}
  {\Key{\Environment{declarations}}{markboth=\PMisc}}
\begin{Declaration}
  {\Key{\Environment{declarations}}{pagestyle=\PName{Seitenstil}}}
\begin{Declaration}
  {\Key{\Environment{declarations}}{columns=\PName{Anzahl}}}
\begin{Declaration}
  {\Key{\Environment{declarations}}{option=\PMisc}}
\begin{Declaration}
  {\Key{\Environment{declarations}}{supporter=\PName{Unterstützer}}}
\begin{Declaration}
  {\Key{\Environment{declarations}}{place=\PName{Ort}}}
\begin{Declaration}
  {\Key{\Environment{declarations}}{closing=\PName{Ende}}}
\begin{Declaration}
  {\Key{\Environment{declarations}}{company=\PName{Firma}}}
\printdeclarationlist

Für Selbstständigkeitserklärung und Sperrvermerk sollte im einfachsten Fall 
\Macro{declaration} beziehungsweise \Macro{confirmation} und \Macro{blocking} 
verwendet werden. Sobald diese jedoch in anderer Reihenfolge, mehrfacher 
Ausführung, unterschiedlichen Sprachen oder um zusätzliche Erklärungen ergänzt 
werden, bietet die \Environment{declarations}"=Umgebung entsprechende 
Freiheiten.

Innerhalb dieser Umgebung können Selbstständigkeitserklärung und Sperrvermerk 
mit dem Befehl \Macro{declaration} direkt nacheinander folgend beziehungsweise 
mit \Macro{confirmation} und \Macro{blocking} auch separat ausgegeben werden. 
Dies kann in beliebiger Reihenfolge und auch mehrmals geschehen, um diese 
beispielsweise mehrsprachig zu setzen.
\ChangedAt{v2.04}%
Des Weiteren gibt es mit \Macro{nextdeclaration} die 
Möglichkeit, eine Erklärung völlig frei zu verfassen. Dieser Befehl kann 
\emph{ausschließlich} innerhalb der \Environment{declarations}"=Umgebung 
genutzt werden, wobei im ersten Argument die gewünschte Überschrift und im 
zweiten der Inhalt respektive Text der Erklärung selbst angegeben werden muss.

Die folgend beschriebenen Parameter können sowohl für die Umgebung 
\Environment{declarations} selbst als auch für die zuvor genannten Befehle als 
optionales Argument verwendet werden. Ähnlich wie die gleichnamigen Optionen 
sind auch die Umgebungen \Environment{abstract} und \Environment{declarations} 
sehr ähnlich zueinander. Deshalb werden die Erläuterungen relativ kurz 
gehalten. Ist ein Erklärung für einen Parameter etwas unverständlich, kann 
diese bei der Umgebung \Environment{abstract}'full' nachgelesen werden.

\index{Sprachunterstützung}%
Wurde das Paket \Package{babel} geladen, kann die Sprache~-- sofern diese als 
Paketoption oder besser noch als Klassenoption angegeben wurde~-- mit dem 
Parameter \Key{\Environment{declarations}}{language=\PName{Sprache}} für die 
\Environment{declarations}"=Umgebung geändert werden. Dadurch werden die 
Bezeichner~-- unter anderem \Term{confirmationname} und \Term{blockingname}~-- 
sowie die Trennungsmuster innerhalb der Umgebung sprachspezifisch angepasst. 

\ChangedAt{v2.02}%
\index{Kolumnentitel}%
Die Kolumnentitel können mit \Key{\Environment{declarations}}{markboth} 
beeinflusst werden. Mit \Key{\Environment{declarations}}{markboth=true} werden 
für diese auf linker und rechter Seite \Term{confirmationname} beziehungsweise
\Term{blockingname} verwendet.
Automatische (\Option{automark}(\Package{scrlayer-scrpage})'none') respektive 
manuelle (\Option{manualmark}(\Package{scrlayer-scrpage})'none') Kolumnentitel 
werden mit \Key{\Environment{declarations}}{markboth=false} genutzt. Mit 
\Key{\Environment{declarations}}{markboth=\PSet{Kolumnentitel}} können 
diese direkt festgelegt werden. Wird \Key{\Environment{declarations}}{markboth} 
in irgendeiner Form genutzt, wird der Seitenstil \PageStyle{headings} 
automatisch gesetzt. Dieser lässt sich für die Umgebung mit dem Parameter 
\Key{\Environment{declarations}}{pagestyle} auch manuell angegeben. Wurde das 
Paket \Package{multicol} geladen, wird der Inhalt der Umgebung mit 
\Key{\Environment{declarations}}{columns=\PName{Anzahl}} mehrspaltig gesetzt. 
Für \Key{\Environment{declarations}}{option} können alle gültigen Werte der 
Option \Option{declaration} angegeben werden. Die Verwendung der weiteren 
Parameter \Key{\Macro{confirmation}}{supporter} sowie 
\Key{\Macro{confirmation}}{place} und \Key{\Macro{confirmation}}{closing} ist 
in der Dokumentation des Befehls \Macro{confirmation} zu finden, der Parameter 
\Key{\Macro{blocking}}{company} ist für \Macro{blocking} erläutert.
\end{Declaration}
\end{Declaration}
\end{Declaration}
\end{Declaration}
\end{Declaration}
\end{Declaration}
\end{Declaration}
\end{Declaration}
\end{Declaration}
\end{Declaration}
\end{Declaration}

\begin{Declaration}
  {\Macro{confirmation}[\OPList{Unterstützer}]}
\begin{Declaration}
  {\Macro{confirmation*}[\OList]}
  [v2.05]
\begin{Declaration}
  {\Key{\Macro{confirmation}}{supporter=\PName{Unterstützer}}}
\begin{Declaration}
  {\Key{\Macro{confirmation}}{place=\PName{Ort}}}
\begin{Declaration}
  {\Key{\Macro{confirmation}}{closing=\PName{Ende}}}
\begin{Declaration}
  {\Key{\Macro{confirmation}}{language=\PName{Sprache}}}
\begin{Declaration}
  {\Key{\Macro{confirmation}}{markboth=\PMisc}}
  [v2.02]
\begin{Declaration}
  {\Key{\Macro{confirmation}}{pagestyle=\PName{Seitenstil}}}
  [v2.02]
\begin{Declaration}
  {\Key{\Macro{confirmation}}{columns=\PName{Anzahl}}}
  [v2.02]
\begin{Declaration}
  {\Key{\Macro{confirmation}}{option=\PMisc}}
\printdeclarationlist[Datum]

Mit diesem Befehl wird ein sprachspezifischer Standardtext für eine 
Selbstständigkeitserklärung ausgegeben, welcher in \Term{confirmationtext} 
gespeichert ist. Wie dieser angepasst beziehungsweise geändert werden kann, ist 
unter \autoref{sec:localization} zu finden. Er kann sowohl innerhalb der 
\Environment{declarations}"=Umgebung als auch außerhalb dieser direkt im 
Dokument verwendet werden. 

Wird \Term{confirmationtext} nicht geändert, kann dieser über das optionale 
Argument von \Macro{confirmation} und die deklarierten Parameter angepasst 
werden. Im Standardtext der Selbstständigkeitserklärung werden sowohl der Titel 
als auch der Typ der Abschlussarbeit~-- falls dieser mit \Macro{thesis}, 
\Macro{subject}[\MPName{\autoref{tab:thesis}}] beziehungsweise mit der 
Option \Option{subjectthesis} angegeben wurde~-- aufgeführt. Über den Parameter 
\Key{\Macro{confirmation}}{supporter} oder \emph{zuvor} mit dem Befehl 
\Macro{supporter} können weitere an der Arbeit beteiligte Personen angegeben 
werden. Mehrere zu nennende Personen sind auch hier durch \Macro{and} zu 
trennen. Das Feld der Unterstützer kann auch mit dem bloßen optionalen Argument 
ohne die Angabe eines Parameters angepasst werden. 
\ChangedAt{v2.05}%
Mit der Sternversion \Macro{confirmation*} werden als Unterstützer die mit 
\Macro{supervisor}[\MPName{Name(n)}] definierten Betreuer der Arbeit 
angegeben.

Nach dem eigentlichen Text der Selbstständigkeitserklärung wird der mit 
\Key{\Macro{confirmation}}{place} beziehungsweise \Macro{place} angegebene Ort 
sowie das mit \Macro{date} eingestellte Datum ausgegeben. Als Voreinstellung 
ist für den Ort \enquote{Dresden} gewählt. Danach folgen~-- mit etwas 
vertikalem Leerraum für die notwendige Unterschrift~-- der Autor oder die 
Autoren, angegeben durch den Befehl \Macro{author}. Soll anstelle dessen etwas 
anderes nach dem Text der Selbstständigkeitserklärung gesetzt werden, kann dies 
mit dem Parameter \Key{\Macro{confirmation}}{closing} oder zuvor mit dem 
Befehl \Macro{confirmationclosing} angepasst werden. Die Parameter 
\Key{\Environment{declarations}}{language}, 
\Key{\Environment{declarations}}{markboth}, 
\Key{\Environment{declarations}}{pagestyle}, 
\Key{\Environment{declarations}}{columns} und 
\Key{\Environment{declarations}}{option} entsprechen in ihrem Verhalten denen 
der \Environment{declarations}"=Umgebung.
\end{Declaration}
\end{Declaration}
\end{Declaration}
\end{Declaration}
\end{Declaration}
\end{Declaration}
\end{Declaration}
\end{Declaration}
\end{Declaration}
\end{Declaration}

\begin{Declaration}
  {\Macro{blocking}[\OPList{Firma}]}
  [v2.02]
\begin{Declaration}
  {\Key{\Macro{blocking}}{company=\PName{Firma}}}
\begin{Declaration}
  {\Key{\Macro{blocking}}{language=\PName{Sprache}}}
\begin{Declaration}
  {\Key{\Macro{blocking}}{markboth=\PMisc}}
\begin{Declaration}
  {\Key{\Macro{blocking}}{pagestyle=\PName{Seitenstil}}}
\begin{Declaration}
  {\Key{\Macro{blocking}}{columns=\PName{Anzahl}}}
\begin{Declaration}
  {\Key{\Macro{blocking}}{option=\PMisc}}
\printdeclarationlist

Beim Sperrvermerk verhält es sich äquivalent zur Selbstständigkeitserklärung.
Es wird der in \Term{blockingtext} hinterlegte Standardtext in der gewählten 
Sprache ausgegeben. Dieser kann~-- wie in \autoref{sec:localization} 
beschrieben~-- geändert werden. Der Befehl \Macro{blocking} kann sowohl 
innerhalb der Umgebung \Environment{declarations} als auch direkt im Dokument 
verwendet werden. 

In seiner ursprünglichen Definition, kann er im optionalen Argument über die 
deklarierten Parameter angepasst werden. Im Standardtext des Sperrvermerks 
werden sowohl der Titel als auch der Typ der Abschlussarbeit~-- falls dieser 
mit \Macro{thesis}, \Macro{subject}[\MPName{\autoref{tab:thesis}}]
respektive mit der Option \Option{subjectthesis} angegeben wurde~-- aufgeführt. 
Mit \Key{\Macro{blocking}}{company} oder \emph{vorher} mit \Macro{company} kann 
zusätzlich eine im Sperrvermerk zu nennende Firma oder ähnliches angegeben 
werden. Dieses Feld lässt sich auch direkt im optionalen Argument ohne die 
Verwendung eines Parameters definieren. Die weiteren Parameter 
\Key{\Environment{declarations}}{language}, 
\Key{\Environment{declarations}}{markboth}, 
\Key{\Environment{declarations}}{pagestyle}, 
\Key{\Environment{declarations}}{columns} und 
\Key{\Environment{declarations}}{option} entsprechen in ihrem Verhalten denen 
der \Environment{declarations}"=Umgebung.
\end{Declaration}
\end{Declaration}
\end{Declaration}
\end{Declaration}
\end{Declaration}
\end{Declaration}
\end{Declaration}

\begin{Declaration}
  {\Macro{declaration}[\OList]}
\begin{Declaration}
  {\Macro{declaration*}[\OList]}
  [v2.05]
\begin{Declaration}
  {\Key{\Macro{declaration}}{language=\PName{Sprache}}}
\begin{Declaration}
  {\Key{\Macro{declaration}}{markboth=\PMisc}}
  [v2.02]
\begin{Declaration}
  {\Key{\Macro{declaration}}{pagestyle=\PName{Seitenstil}}}
  [v2.02]
\begin{Declaration}
  {\Key{\Macro{declaration}}{columns=\PName{Anzahl}}}
  [v2.02]
\begin{Declaration}
  {\Key{\Macro{declaration}}{option=\PMisc}}
\begin{Declaration}
  {\Key{\Macro{declaration}}{supporter=\PName{Unterstützer}}}
\begin{Declaration}
  {\Key{\Macro{declaration}}{place=\PName{Ort}}}
\begin{Declaration}
  {\Key{\Macro{declaration}}{closing=\PName{Ende}}}
\begin{Declaration}
  {\Key{\Macro{declaration}}{company=\PName{Firma}}}
\printdeclarationlist

Dieser Befehl gibt die Selbstständigkeitserklärung und den Sperrvermerk direkt 
aufeinanderfolgend aus. Dabei werden die Einstellungen zur Positionierung der 
einzelnen Erklärungen, welche über die Zuweisungen \Option{declaration=single} 
beziehungsweise \Option{declaration=multiple} sowie \Option{declaration=fill} 
respektive \Option{declaration=nofill} erfolgen, beachtet. Er kann sowohl 
innerhalb der \Environment{declarations}"=Umgebung als auch außerhalb dieser 
direkt im Dokument verwendet werden und akzeptiert im optionalen Argument dabei 
alle für die \Environment{declarations}"=Umgebung beschriebenen Parameter. 
\ChangedAt{v2.05}%
Die Sternversion erzwingt für die Selbstständigkeitserklärung eine Angabe der 
mit \Macro{supervisor}[\MPName{Name(n)}] definierten Betreuer in dieser.
\end{Declaration}
\end{Declaration}
\end{Declaration}
\end{Declaration}
\end{Declaration}
\end{Declaration}
\end{Declaration}
\end{Declaration}
\end{Declaration}
\end{Declaration}
\end{Declaration}

\begin{Declaration}
  {\Macro{supporter}[\MPName{Unterstützer}]}
\begin{Declaration}
  {\Macro{place}[\MPName{Ort}]}
\begin{Declaration}
  {\Macro{confirmationclosing}[\MPName{Ende}]}
\begin{Declaration}
  {\Macro{company}[\MPName{Firma}]}
\printdeclarationlist

Diese Makros ändern~-- im Gegensatz zu den Parametern von 
\Macro{confirmation} und \Macro{blocking}~-- die entsprechenden Feldwerte 
global. Damit lässt sich die \emph{mehrfache} Angabe eines Parameters 
vermeiden, wenn beispielsweise eine Erklärung in unterschiedlichen Sprachen 
erzeugt wird.%
\end{Declaration}
\end{Declaration}
\end{Declaration}
\end{Declaration}
\index{Selbstständigkeitserklärung|!)}%
\index{Sperrvermerk|!)}%



\subsection{%
  Lesezeichen%
  \index{Lesezeichen}%
  \index{Inhaltsverzeichnis}%
}

\begin{Declaration}
  {\Option{tudbookmarks=\PBoolean}}
  (true)
\begin{Declaration}
  {\Macro{tudbookmark}[\OPName{Ebene}\MPName{Text}\MPName{Ankername}]}
\printdeclarationlist

Diese Option wird wirksam, wenn \Package{hyperref} geladen wurde. Es werden für 
die Umschlag- und Titelseite, das Inhaltsverzeichnis sowie~-- bei der 
Verwendung des Paketes \Package{tudscrsupervisor}~-- die Aufgabenstellung 
Lesezeichen oder auch Outline"=Einträge im PDF"~Dokument erzeugt.
%
\begin{values}{\Option{tudbookmarks}}
\itemfalse
  Es erfolgt kein Eintrag von ergänzenden Lesezeichen.
\itemtrue*
  Es werden automatisch zusätzliche Lesezeichen eingetragen.
\end{values}
%
Der Befehl \Macro{tudbookmark} arbeitet prinzipiell in der gleichen Weise wie 
\Macro{pdfbookmark}(\Package{hyperref})'none' aus \Package{hyperref}. Die 
Lesezeichen werden jedoch nur bei aktivierte Option \Option{tudbookmarks} 
generiert.
\end{Declaration}
\end{Declaration}



\section{%
  Sprachabhängige Bezeichner%
  \label{sec:localization}%
  \index{Bezeichner|!(}%
  \index{Titel!Felder|(}%
}

Durch \KOMAScript werden Befehle, mit denen sich sprachabhängige Bezeichner 
erzeugen oder ändern lassen, zur Verfügung gestellt. Diese werden durch das
\TUDScript-Bundle genutzt, um lokalisierte Begriffe für die Sprachen 
\emph{Englisch} und \emph{Deutsch} bereitzustellen. Ein Großteil davon betrifft 
Bezeichnungen für Felder auf der Titelseite (\autoref{sec:title}). Hierfür wird
\Macro{providecaptionname}[%
  \MPName{Sprache}\MPName{Makro}\MPName{Inhalt}%
](\Bundle{koma-script})'none'
verwendet, wobei \PName{Sprache} dem geladenen Sprachpaket~-- normalerweise das 
Paket \Package{babel}~-- bekannt sein muss.

Sollte der Anwender die im Folgenden erläuterten oder auch andere Bezeichner, 
welche von einem beliebigen (Sprach"~)Paket bereitgestellt werden, ändern 
wollen, ist hierfür der Befehl
\Macro{renewcaptionname}[%
  \MPName{Sprache}\MPName{Makro}\MPName{Inhalt}%
](\Bundle{koma-script})'none'
zu verwenden. Es sollte natürlich dabei eine \PName{Sprache} angegeben werden, 
welche im Dokument durch \Package{babel} oder ein anderes Sprachpaket verwendet 
wird, beispielsweise \PValue{ngerman} oder \PValue{english}. 

Die Makros der Bezeichner und deren Verwendung werden folgend kurz beschrieben 
und tabellarisch aufgeführt. Dabei wurde versucht, alle Befehle der Bezeichner 
für bestimmte Begriffe auf \Term*{\noexpand\dots{}name} und beschreibende 
Texte auf \Term*{\noexpand\dots{}text} enden zu lassen.

\begin{Declaration}
  {\Term{refereename}}
  [v2.02:Unterscheidung von einem und mehreren Gutachtern]
\begin{Declaration}
  {\Term{refereeothername}}
\begin{Declaration}
  {\Term{advisorname}}
  [v2.05:Unterscheidung von einem und mehreren Fachreferenten]
\begin{Declaration}
  {\Term{advisorothername}}
\begin{Declaration}
  {\Term{supervisorname}}
  [v2.05:Unterscheidung von einem und mehreren Betreuern]
\begin{Declaration}
  {\Term{supervisorothername}}
\begin{Declaration}
  {\Term{professorname}}
  [v2.02:Unterscheidung von einem und mehreren Professoren]
\begin{Declaration}
  {\Term{professorothername}}
  [v2.02]
\printdeclarationlist[Betreuer,Gutachter,Hochschullehrer,Referent]

Diese sprachabhängigen Begriffe sind die Bezeichner für die Titelseitenfelder 
von Betreuer (\Macro{supervisor}), Gutachter (\Macro{referee}) und Fachreferent 
(\Macro{advisor}). Soll innerhalb eines dieser Felder mehr als eine Person 
angegeben werden, so sind die Einzelpersonen jeweils mit dem Befehl \Macro{and} 
voneinander zu trennen. In diesem Fall werden alle nach der erstgenannten 
folgenden Personen durch den Bezeichner \Term*{\noexpand\dots{}othername} 
ergänzt.

\ChangedAt{v2.02;v2.05}%
Bei den Bezeichnung wird unterschieden, ob eine oder mehrere Personen angegeben 
wurden. Wird lediglich eine Person genannt, so ist eine Unterscheidung nicht 
notwendig und es wird der Singular genutzt. Werden jedoch zwei oder mehr 
Personen angegeben, so wird geprüft, ob der dazugehörige Bezeichner für die 
Zweitperson (\Term*{\noexpand\dots{}othername}) definiert ist. Falls dies so 
ist, wird die alternative Bezeichnung für die erstgenannte Person verwendet, 
andernfalls wird der Plural des Bezeichners verwendet. Dies betrifft alle 
Felder, die über \Macro{referee}, \Macro{advisor}, \Macro{supervisor} oder 
\Macro{professor} angegeben wurden.

\renewcaptionname{ngerman}{\refereename}{(Erst-)Gutachter}
\renewcaptionname{english}{\refereename}{(First) Referee(s)}
\renewcaptionname{ngerman}{\advisorname}{(Erster) Fachreferent(en)}
\renewcaptionname{english}{\advisorname}{(First) Advisor(s)}
\renewcaptionname{ngerman}{\supervisorname}{(Erst-)Betreuer}
\renewcaptionname{english}{\supervisorname}{(First) Supervisors(s)}
\renewcaptionname{ngerman}{\professorname}{Betreuende(r) Hochschullehrer}
\renewcaptionname{english}{\professorname}{Supervising professor(s)}
\TermTable{%
  supervisorname,supervisorothername,refereename,refereeothername,%
  advisorname,advisorothername,professorname,professorothername%
}
\end{Declaration}
\end{Declaration}
\end{Declaration}
\end{Declaration}
\end{Declaration}
\end{Declaration}
\end{Declaration}
\end{Declaration}

\begin{Declaration}
  {\Term{dissertationname}}
\begin{Declaration}
  {\Term{diplomathesisname}}
\begin{Declaration}
  {\Term{masterthesisname}}
\begin{Declaration}
  {\Term{bachelorthesisname}}
\begin{Declaration}
  {\Term{studentthesisname}}
\begin{Declaration}
  {\Term{studentresearchname}}
\begin{Declaration}
  {\Term{projectpapername}}
\begin{Declaration}
  {\Term{seminarpapername}}
\begin{Declaration}
  {\Term{termpapername}}
\begin{Declaration}
  {\Term{researchname}}
\begin{Declaration}
  {\Term{logname}}
\begin{Declaration}
  {\Term{internshipname}}
\begin{Declaration}
  {\Term{reportname}}
\printdeclarationlist[Abschlussarbeit,Typisierung]

Diese Bezeichner dienen zur Typisierung speziell für eine Abschlussarbeit. Wie 
diese genutzt werden können, ist bei der Erläuterung von \Option*{subjectthesis}
beziehungsweise \Macro*{thesis} und \Macro*{subject}'full' zu finden.
\TermTable{%
  dissertationname,diplomathesisname,masterthesisname,bachelorthesisname,%
  studentresearchname,projectpapername,seminarpapername,researchname,%
  logname,internshipname,reportname%
}
\end{Declaration}
\end{Declaration}
\end{Declaration}
\end{Declaration}
\end{Declaration}
\end{Declaration}
\end{Declaration}
\end{Declaration}
\end{Declaration}
\end{Declaration}
\end{Declaration}
\end{Declaration}
\end{Declaration}

\begin{Declaration}
  {\Term{graduationtext}}
  [v2.02]
\printdeclarationlist[Abschlussarbeit,Typisierung]

Wurde erkannt, dass das aktuelle Dokument eine Abschlussarbeit ist, so kann der 
zu erlangende akademische Grad mit dem Befehl \Macro{graduation} angegeben 
werden. Bei dessen Ausgabe auf dem Titel wird dabei der entsprechende Text dazu 
angegeben.
\TermTable*{graduationtext}{.78\textwidth}
\end{Declaration}

\begin{Declaration}
  {\Term{datetext}}
\begin{Declaration}
  {\Term{defensedatetext}}
\printdeclarationlist
  [Abschlussarbeit,Datum,Datum!Abgabedatum,Datum!Verteidigungsdatum]

Wird mit \Macro{date} das (Abgabe-)Datum und mit \Macro{defensedate} ein Datum 
der Verteidigung für eine Abschlussarbeit angegeben, so werden auch diese 
Felder durch einen Text beschrieben.%
\TermTable{datetext,defensedatetext}%
\end{Declaration}
\end{Declaration}
\index{Titel!Felder|)}%

\begin{Declaration}
  {\Term{dateofbirthtext}}
\begin{Declaration}
  {\Term{placeofbirthtext}}
\begin{Declaration}
  {\Term{matriculationnumbername}}
\begin{Declaration}
  {\Term{matriculationyearname}}
\begin{Declaration}
  {\Term{coursename}}
\begin{Declaration}
  {\Term{disciplinename}}
  [v2.02]
\printdeclarationlist[Autorenangaben,Datum!Geburtsdatum]

Werden für den Autor oder die Autoren mit dem entsprechenden Befehl das 
Geburtsdatum (\Macro{dateofbirth}), der Geburtsort (\Macro{placeofbirth}), der 
Studiengang (\Macro{course}), die Studienrichtung (\Macro{discipline}) oder 
auch die Matrikelnummer (\Macro{matriculationnumber}) und/oder das 
Immatrikulationsjahr (\Macro{matriculationyear}) angegeben, werden sowohl auf 
der Titelseite als auch auf der gegebenenfalls mit \Package{tudscrsupervisor} 
erstellten Aufgabenstellung die dazugehörigen Bezeichner vorangestellt. Auf 
dem Titel werden diese dabei mit dem durch \Macro{titledelimiter} gegebenen 
Trennzeichen vom eigentlichen Feld abgegrenzt.
\TermTable{%
  dateofbirthtext,placeofbirthtext,matriculationnumbername,%
  matriculationyearname,coursename,disciplinename%
}
\end{Declaration}
\end{Declaration}
\end{Declaration}
\end{Declaration}
\end{Declaration}
\end{Declaration}

\begin{Declaration}
  {\Term{coverpagename}}
\begin{Declaration}
  {\Term{titlepagename}}
\printdeclarationlist[Lesezeichen]

Diese beiden Bezeichner werden bei aktivierter \Option{tudbookmarks} für das 
Eintragen von Lesezeichen in ein PDF"=Dokument genutzt.
\TermTable{coverpagename,titlepagename}
\end{Declaration}
\end{Declaration}

\begin{Declaration}
  {\Term{abstractname}}
\printdeclarationlist[Zusammenfassung]

Dieser Bezeichner wird für die Klasse \Class{tudscrbook} definiert, da selbiger 
von \KOMAScript für die Buchklasse nicht vorgesehen wird.
\TermTable{abstractname}
\end{Declaration}

\begin{Declaration}
  {\Term{confirmationname}}
\begin{Declaration}
  {\Term{blockingname}}
  [v2.02]
\printdeclarationlist[Selbstständigkeitserklärung,Sperrvermerk]

Es werden die Bezeichnungen für Selbstständigkeitserklärung und Sperrvermerk 
für die dazugehörigen Überschriften definiert.
\TermTable{confirmationname,blockingname}
\end{Declaration}
\end{Declaration}

\begin{Declaration}
  {\Term{confirmationtext}}
\begin{Declaration}
  {\Term{blockingtext}}
  [v2.02]
\printdeclarationlist[Selbstständigkeitserklärung,Sperrvermerk]

Die Texte der Erklärungen selbst sind derart aufgebaut, dass sie in 
Abhängigkeit von den angegebenen Informationen unterschiedlich ausgeführt 
werden. Innerhalb der Selbstständigkeitserklärung (\Macro{confirmation}) werden 
gegebenenfalls die Felder für den Titel (\Macro{title}) und die Typisierung der 
Abschlussarbeit sowie die angegebenen Unterstützer%
\footnote{%
  \Macro{confirmation}[\OPValue{%
    \Key{\Macro{confirmation}}{supporter=\PName{Unterstützer}}%
  }]
  oder \Macro{supporter}[\MPName{Unterstützer}]%
}
beachtet. Für den Sperrvermerk (\Macro{blocking}) wird neben dem Titel 
(\Macro{title}) optional außerdem noch das Feld der externen Firma%
\footnote{%
  \Macro{blocking}[%
    \OPValue{\Key{\Macro{blocking}}{company=\PName{Firma}}}%
  ]
  oder \Macro{company}[\MPName{Firma}]%
}
verwendet. Der Vollständigkeit halber werden im Folgenden noch die Texte für 
die Selbstständigkeitserklärung und den Sperrvermerk aufgeführt~-- allerdings 
lediglich die deutschsprachige Version. Dabei werden alle möglichen Felder 
angezeigt.

\begingroup
  \makeatletter
  \def\@@title{\PName{Titel}}
  \def\@@thesis{\PName{Abschlussarbeit}}
  \def\@supporter{\PName{Vorname Nachname} \and \PName{Vorname Nachname}}
  \def\@company{\PName{Firma}}
  \makeatother
  \vskip\medskipamount\noindent
  \textbf{Bezeichner}\quad\Term{confirmationtext}%
  \begin{quoting}
  \confirmationtext
  \end{quoting}
  \textbf{Bezeichner}\quad\Term{blockingtext}%
  \begin{quoting}
  \blockingtext
  \end{quoting}

  \makeatletter
  \def\@@author{\PName{Vorname Nachname}}%
  \def\@supporter{\PName{Vorname Nachname}}%
  \makeatother
  \newcommand*\showfield[1]{%
    \Macro*{getfield}[\MPValue{#1}]'none'~\textrightarrow~\getfield{#1}%
  }%
  \noindent
  Soll eine der Erklärungen geändert und dabei der Inhalt eines Feldes genutzt 
  werden, lässt sich hierfür \Macro{getfield} verwenden.%
  \footnote{%
    Titel:~\showfield{title}, Art der Abschlussarbeit:~\showfield{thesis},
    Autor:~\showfield{author}, Firma:~\showfield{company} sowie ebenfalls
    Unterstützer:~\showfield{supporter}
  }
\endgroup
Gegebenenfalls ist die Definition von \Macro{and} anzupassen.%
\end{Declaration}
\end{Declaration}

\begin{Declaration}
  {\Term{listingname}}
\begin{Declaration}
  {\Term{listlistingname}}
\printdeclarationlist[Quelltextdokumentation]

Sollte ein Paket zur Einbindung von externem Quelltext~-- beispielsweise 
das Paket \Package{listings}~-- verwendet werden, so werden diese Bezeichnungen 
für Quelltextausschnitte und das Quelltextverzeichnis verwendet.
\TermTable{listingname,listlistingname}%
\end{Declaration}
\end{Declaration}
\index{Bezeichner|!)}%



\ToDo[doc]{Wieder rein, falls vernünftig implementiert}[v2.07]
%\subsection{%
%  Fußnoten in Überschriften%
%  \index{Layout!Überschriften}%
%  \index{Fußnoten}%
%}
%
%\begin{Declaration}
%  {\Option{footnotes=\PMisc}}
%  (nosymbolheadings)
%  [v2.02:Fußnoten mit Symbolen in Überschriften möglich]
%\begin{Declaration}
%  {\Counter{symbolheadings}}
%  [v2.02]
%\printdeclarationlist
%
%\ToDo[imp]{Fehler \Macro*{addchap} beheben, Paket \Package*{footmisc}}[v2.06]
%\ToDo[imp]{Zähler auch bei Sternversionen von Kapiteln zurücksetzen}[v2.06]
%\ToDo[imp]{Fußnoten nicht ins Inhaltsverzeichnis und in die Kopfzeile?!}[v2.06]
%\ToDo[imp]{Problem mit \Package*{hyperref} lösbar?}[v2.06]
%%
%Für die Überschriften wird die \KOMAScript-Option \Option{footnotes} erweitert.
%Normalerweise kann diese die Werte \PValue{multiple} und \PValue{nomultiple} 
%annehmen, wobei Letzteres der Standardfall ist. Die \TUDScript-Hauptklassen 
%erweitern die Option dahingehend, dass auf die Verwendung von Symbolen anstelle
%von Zahlen innerhalb der Überschriften umgeschaltet werden kann. Hierfür wird 
%der Zähler \Counter{symbolheadings} definiert, der mit dem Beginn eines neuen 
%Kapitels zurückgesetzt wird.
%
%\Attention{%
%  Die Option \Option{footnotes=symbolheadings} ist experimentell und kann unter
%  Umständen zu Fehlern respektive unerwünschten Ergebnissen führen.%
%}
%%
%\begin{values}{\Option{footnotes}}
%\item[nosymbolheadings/numberheadings]
%  Die Fußnoten der Überschriften werden fortlaufend mit denen des Fließtextes 
%  gesetzt.
%\item[symbolheadings]
%  Für die Überschriften werden symbolische Fußnoten mit einem eigenen Zähler 
%  verwendet.
%\end{values}
%\end{Declaration}
%\end{Declaration}
\end{DeclareEntity*}
\end{DeclareEntity*}
\end{DeclareEntity*}

\chapter[%
  Die Posterklasse \Class{tudscrposter}%
]{%
  Die Posterklasse%
  \entitylabel[sec:poster]{\Class{tudscrposter}}%
  \index{Posterklasse|!}%
}

\begin{DeclareEntity}{\Class{tudscrposter}}
[
  v2.05:Neue Klasse für Poster im \TUDCD;
]
Ergänzend zu den Hauptklassen, welche für das Setzen von Dokumenten im \TUDCD 
angeboten werden, wird die Klasse \Class{tudscrposter} bereitgestellt. Mit 
dieser wird das Erstellen von Postern im gleichen Layout mit \Lettering{LaTeX} 
ermöglicht. Die Basis hierfür ist \Class{tudscrartcl} und \emph{fast} alle 
durch diese Klasse angebotenen Befehle und Optionen können gleichermaßen mit 
\Class{tudscrposter} verwendet werden. Ein Minimalbeispiel zur Verwendung der 
Klasse ist in \fullref{sec:exmpl:poster} zu finden.

Der größte Unterschied zu den Hauptklassen ist insbesondere ein vereinfachter 
Titel. Eine Umschlagseite steht für \Class{tudscrposter} nicht zur Verfügung, 
die entsprechende Option \Option*{cdcover} sowie der dazugehörige Befehl 
\Macro*{makecover} sind nicht definiert. Der Titel selbst kann mit 
\Macro{maketitle} lediglich als Titelkopf gesetzt werden, eine separate 
Titelseite existiert nicht. Aus diesem Grund sind mit \Macro{title}, 
\Macro{subtitle}, \Macro(){subject} und \Macro{titlehead} auch nur eine 
reduzierte Anzahl an Befehlen für den Titel verfügbar, die wie gewohnt genutzt 
werden können. Prinzipiell lassen sich auch noch die beiden Befehle für den 
Schmutztitel \Macro{extratitle} sowie \Macro{frontispiece} nutzen, wobei deren 
Verwendung für ein Poster eher fraglich ist.

Alle weiteren in \autoref{sec:title} vorgestellten Befehle und Optionen sind 
für \Class{tudscrposter} nicht definiert. Dies betrifft zum einen sowohl das 
Schriftelement \Font*{thesis} als auch die Befehle \Macro*{titledelimiter}, 
\Macro*{thesis}, \Macro*{referee}, \Macro*{advisor}, \Macro*{graduation} 
und \Macro*{defensedate}. Zum anderen stehen die Makros für ergänzende 
Autorenangaben \Macro*{dateofbirth}, \Macro*{placeofbirth}, 
\Macro*{matriculationyear} und \Macro*{matriculationnumber} wie auch 
die Option \Option*{subjectthesis} \emph{nicht} zur Verfügung. Die von 
\KOMAScript für eine Titelseite bereitgestellten Befehle \Macro*{publishers} 
und \Macro*{dedication} sowie \Macro*{uppertitleback} und 
\Macro*{lowertitleback} haben für \Class{tudscrposter} keinerlei 
Funktionalität. Dahingegen lassen sich \Macro{author} und \Macro{authormore} 
weiterhin verwenden. Diese beiden Befehle werden allerdings nicht für den Titel 
sondern für den speziellen Fußbereich des Posters genutzt, welcher in 
\autoref{sec:poster:foot} weiterführend beschrieben wird.

Neben der signifikanten Vereinfachung des Titels entfallen für die Klasse 
\Class{tudscrposter} einige Elemente. Namentlich sind dies die Optionen 
\Option*{headingsvskip} und \Option*{pageheadingsvskip}, die Umgebung 
\Environment*{tudpage} sowie alle zu Selbstständigkeitserklärung und 
Sperrvermerk gehörigen Elemente wie die Option \Option*{declaration}, die 
Umgebung \Environment*{declarations} und die Befehle \Macro*{declaration}, 
\Macro*{confirmation} sowie \Macro*{blocking}. Für eine \abstractname{} 
kann die Umgebung \Environment{abstract} weiterhin genutzt werden, allerdings 
lässt sich mit der Option \Option{abstract} lediglich noch die Gliederungsebene 
der Überschrift anpassen.



\section{Layout und Formatierung eines Posters}

Die augenscheinlichsten Einstellungen für die Gestaltung eines Posters sind 
sicherlich das verwendete Papierformat und die farbliche Ausprägung sowie die 
Auswahl der Schriftart und deren Größe. Als Grundeinstellung für die Klasse
\Class{tudscrposter} sind die Schriften des \TUDCDs aktiviert. Diese lassen 
sich~-- wie auch bei den Hauptklassen~-- anpassen. Weitere Informationen hierzu 
sind der Erläuterung zur Option \Option'full'{cdfont} zu entnehmen.

Nachfolgend wird kurz erläutert, wie sich allgemeine Formatierung eines mit 
\Class{tudscrposter} erstellten Posters anpassen lässt. Der Inhalt kann völlig 
frei gestaltet werden, es gibt hierfür bisher keinerlei von \TUDScript 
vordefinierte Befehle und Optionen, welche die Inhalte in ein bestimmtes Layout 
übersetzen.



\subsection{%
  Die Wahl von Papierformat und Schriftgröße%
  \label{sec:fontsize}%
  \index{Papierformat|!}%
  \index{Schriftgröße|!}%
}

Die Festlegung von \emph{Papierformat} und \emph{Schriftgröße} ist essentiell 
für das Erstellen eines Posters und sollten \emph{immer} vorgenommen werden. 
\Attention{Beide Einstellungen müssen zwingend als Klassenoption erfolgen.}
Bei der Schriftgrößenauswahl ist darauf zu achten, ob der Satz des Posters 
ein- oder mehrspaltig erfolgen soll. Für letzteres Unterfangen ist die 
\Environment(\Package{multicol}){multicols}"~Umgebung aus dem Paket 
\Package{multicol} sehr empfehlenswert. 

Zur Festlegung des Papierformats ist die \KOMAScript-Option 
\Option(\Package{typearea}){paper=\PMisc} zu verwenden. Dabei lassen 
sich mit \Option(\Package{typearea}){paper=\PName{Format}} unter anderem 
die gängigen Klassen der ISO/DIN"~Reihen A~bis~D als auch Quer- oder 
Längsformat auswählen. Ein beliebiges Format kann mit der Einstellung
\Option(\Package{typearea}){paper=\PName{Höhe:Breite}} gewählt werden. Für 
zusätzliche Hinweise ist das \scrguide zu Rate zu ziehen.

Passend zum ausgewählten Papierformat sowie der gewünschten Anzahl an 
Textspalten des Posters sollte unbedingt die Schriftgröße mit 
\Option(\Bundle{koma-script}){fontsize=\PLength} angegeben werden. Für eine 
passend abgestimmte Auswahl von Papierformat und Schriftgröße ist 
\autoref{tab:font+paper} als Referenz zu nutzen. Sollten Sie aufgrund der 
Schriftgrößenänderung eine oder mehrere Warnungen vom Typ
%
\begin{quoting}
\begin{Code}
Font shape `T1/cmr/m/n' in size <...> not available
\end{Code}
\end{quoting}
%
erhalten, so beachten Sie bitte die Hinweise aus \autoref{sec:tips:fontsize}.


\begin{table}
  \newcommand*\mtm{\small min\dots{}max}%
  \newcommand*\rng[2]{\small #1\dots{}#2pt}%
  \ttabbox[\linewidth]{%
    \setlength\tabcolsep{5pt}%
    \centering%
    \begin{subfloatrow}%
      \ttabbox{%
\begin{tabular}{r*{7}c}
  \toprule
      &\multicolumn{7}{c}{Klasse}                      \tabularnewline\midrule
      & 6    & 5    & 4    & 3    & 2    & 1    & 0    \tabularnewline\midrule
Reihe & \mtm & \mtm & \mtm & \mtm & \mtm & \mtm & \mtm \tabularnewline\midrule
    D &\rng{05}{07} &\rng{06}{09} &\rng{10}{14} &\rng{14}{20}%
      &\rng{20}{29} &\rng{28}{40} &\rng{40}{60} \tabularnewline\midrule
    A &\rng{06}{08} &\rng{07}{10} &\rng{11}{16} &\rng{16}{23}%
      &\rng{23}{33} &\rng{32}{46} &\rng{45}{66} \tabularnewline\midrule
    C &\rng{07}{09} &\rng{08}{11} &\rng{12}{18} &\rng{18}{26}%
      &\rng{26}{37} &\rng{36}{52} &\rng{50}{72} \tabularnewline\midrule
    B &\rng{08}{10} &\rng{09}{12} &\rng{13}{20} &\rng{20}{29}%
      &\rng{29}{41} &\rng{40}{58} &\rng{55}{78} \tabularnewline\bottomrule
\end{tabular}
      }{\caption{Einspaltiges Layout}}%
    \end{subfloatrow}%
    \par\medskip
    \begin{subfloatrow}%
      \ttabbox{%
\begin{tabular}{r*{4}c}
  \toprule
        &\multicolumn{4}{c}{Klasse} \tabularnewline\midrule
        & 3    & 2    & 1    & 0    \tabularnewline\midrule
  Reihe & \mtm & \mtm & \mtm & \mtm \tabularnewline\midrule
D&\rng{07}{10}&\rng{10}{14}&\rng{14}{19}&\rng{19}{28}\tabularnewline\midrule
A&\rng{08}{11}&\rng{11}{16}&\rng{16}{22}&\rng{22}{32}\tabularnewline\midrule
C&\rng{09}{12}&\rng{12}{18}&\rng{18}{25}&\rng{25}{36}\tabularnewline\midrule
B&\rng{10}{13}&\rng{13}{20}&\rng{20}{28}&\rng{28}{40}\tabularnewline\bottomrule
\end{tabular}
      }{\caption{Zweispaltiges Layout}}%
      \ttabbox{%
\begin{tabular}{r*{2}c}
  \toprule
        &\multicolumn{2}{c}{Klasse} \tabularnewline\midrule
        & 1           & 0           \tabularnewline\midrule
  Reihe & \mtm        & \mtm        \tabularnewline\midrule
      D &\rng{07}{10} &\rng{10}{14} \tabularnewline\midrule
      A &\rng{08}{11} &\rng{11}{16} \tabularnewline\midrule
      C &\rng{09}{12} &\rng{12}{18} \tabularnewline\midrule
      B &\rng{10}{13} &\rng{13}{20} \tabularnewline\bottomrule
\end{tabular}
      }{\caption{Dreispaltiges Layout}}%
    \end{subfloatrow}%
  }{%
    \caption{%
      Empfohlene Kombinationen für die Wahl von Papierformat 
      (\Option(\Package{typearea}){paper}) und Schriftgröße 
      (\Option(\Bundle{koma-script}){fontsize})%
      \label{tab:font+paper}%
      \index{Papierformat|!}%
      \index{Schriftgröße|!}%
    }%
  }%
\end{table}



\subsection{Die Gestalt eines Posters}

Die Festlegung der Farbausprägung eines Posters erfolgt mit der Option 
\Option{cd}, welche nachfolgend beschrieben wird. Dabei kann aus einigen 
Varianten zur Farbgestaltung gewählt werden. Sollte keiner dieser 
vordefinierten Werte das gewünschte Layout zur Verfügung stellen, lässt sich 
dieses mit den Optionen \Option{cdhead} und \Option{cdfoot} sowie 
\Option{cdtitle}, \Option{cdpart} und \Option{cdsection} nachträglich 
noch genauer anpassen.


\begin{Declaration}
  {\Option{cd=\PMisc}}
  (bicolor)
\printdeclarationlist[Layout!Farbgestaltung]

Äquivalent zu den \TUDScript-Hauptklassen wird mit dieser Option die Verwendung 
des \TUDCDs für das Poster festgelegt. Sie hat Einfluss auf die Farbgestaltung 
der Gliederungsüberschriften sowie des Seitenstils, welcher standardmäßig auf 
\PageStyle{empty.tudheadings} gesetzt wird.
\begin{DeclareValues}{\Option{cd}}
\itemval{false}
  Hiermit wird das \CD komplett deaktiviert und es werden keine spezifischen 
  Einstellungen für ein Poster vorgenommen. Lediglich der Seitenstil wird auf 
  \PageStyle{empty} festgelegt.
\itemval*{true,nocolor,monochrome}
  Es wird schwarze Schrift für Überschriften und den Seitenkopf verwendet. Der 
  Fußbereich wird nicht farbig akzentuiert.
\itemval{lightcolor,pale}
  Die Einstellung entspricht weitestgehend der Option \Option{cd=true}, 
  allerdings wird die primäre Hausfarbe \Color{HKS41} für Kopf sowie Fuß und 
  die Überschriften genutzt.
\itemval{barcolor}
  Zusätzlich zur vorherigen Einstellung wird außerdem der Querbalken farbig 
  abgesetzt.
\itemval{bicolor,color,fullcolor}
  Der Kopf wird mit einem farbigen Hintergrund in der primären Hausfarbe 
  \Color{HKS41} gesetzt, der Querbalken wird farbig abgesetzt. Ebenso wird für 
  alle Überschriften die Hausfarbe verwendet, der Fußbereich erhält ebenfalls 
  einen farbigen Hintergrund.
\end{DeclareValues}
\end{Declaration}

\begin{Declaration}
  {\Option{backcolor=\PMisc}}
  (true)
\printdeclarationlist[Layout!Farbgestaltung]

Mit dieser Option kann die Hintergrundfarbe eines Posters definiert werden.
\begin{DeclareValues}{\Option{backcolor}}
\itemval{false,nocolor}
  Es wird keine Farbe festgelegt, der Hintergrund erscheint weiß.
\itemval*{true,color}
  Der Seitenhintergrund wird in der primären Hausfarbe \Color{HKS41} gewählt.
\itemval{\PSet{Farbe}}
  Die angegebene \PName{Farbe} wird im Hintergrund für das Poster genutzt. 
  Dabei kann es sich im Sinne des Paketes \Package{xcolor} sowohl um eine 
  benannte Farbe als auch einen Farbausdruck handeln, da der gegebene Wert mit 
  \Macro(\Package{xcolor}){color|\MPName{Farbe}} verwendet wird.
\end{DeclareValues}
\end{Declaration}

\begin{Declaration}
  {\Option{bleedmargin=\PLength}}
  (0.2in)
\printdeclarationlist[Beschnittzugabe|?,Schnittmarken|?,Layout!Farbgestaltung]

Soll das Poster in einem Papierformat gedruckt werden, welches anschließend 
noch auf das Zielformat zugeschnitten wird, weil beispielsweise ein randloses 
Drucken nicht möglich ist, kann diese Option genutzt werden, um die farbigen 
Elemente des Layouts in den Bereich der Beschnittzugabe respektive Überfüllung 
zu vergrößern. Damit ist ein \enquote{Zuschneiden in die Farbe} sehr einfach 
und ohne große Probleme realisierbar.

Die von der Einstellung \Option{bleedmargin=\PLength} abhängigen 
Elemente sind zum einen Kopf- und Fußbereich, beeinflusst durch die Optionen 
\Option{cdhead} und \Option{cdfoot}. Werden diese farbig gesetzt, so werden 
diese um den angegebenen Längenwert über das gewünschte Zielformat hinaus 
vergrößert. Zum anderen wird auch der mit \Option{backcolor} gegebenenfalls 
eingestellte, farbige Seitenhintergrund erweitert. Wie sich der Entwurf eines 
Posters in einem bestimmten Zielformat auf einem übergroßem Papierbogen 
tatsächlich realisieren lässt, wird in \fullref{sec:tips:crop} exemplarisch 
dargestellt.
\end{Declaration}



\subsection{Fixierte Abbildungen und Tabellen}

\begin{Declaration}
  {\Environment{figurehere}}
  [v2.06]
\begin{Declaration}
  {\Environment{tablehere}}
  [v2.06]
\printdeclarationlist[Gleitobjekte]

Bei Gestalten eines Posters sind Gleitobjekte für Abbildungen oder Tabellen 
eher ungeeignet. Stattdessen kann man diese ganz einfach in einer 
\Environment*{minipage}"~Umgebung setzen und gegebenenfalls für eine 
Beschriftung den Befehl
\Macro(\Bundle{koma-script}){captionof|\MPName{Typ}\MPName{Beschriftung}} 
nutzen. Alternativ dazu können diese beiden Umgebungen verwendet werden, um 
nicht"~gleitende Abbildungen oder Tabellen zu setzen.
\end{Declaration}
\end{Declaration}



\section{%
  Felder für den Fußbereich%
  \label{sec:poster:foot}%
  \index{Layout!Fußzeile|(}%
}

Der Fußbereich eines Posters kann mit \Macro{footcontent} eigens und frei 
definiert werden. Geschieht dies nicht, wird standardmäßig ein vordefinierte 
Fuß gesetzt, welcher Angaben von bestimmten Feldern ausgibt, die insbesondere 
als Kontaktinformationen gedacht sind. Welche das im Einzelnen sind, wird 
nachfolgend erläutert. Die farbliche Ausprägung des Fußes wird durch die Option
\Option{cdfoot} festgelegt.

\begin{Declaration}
  {\Macro{faculty|\OPName{Fußzeile}\MPName{Fakultät}}}
\begin{Declaration}
  {\Macro{department|\OPName{Fußzeile}\MPName{Einrichtung}}}
\begin{Declaration}
  {\Macro{institute|\OPName{Fußzeile}\MPName{Institut}}}
\begin{Declaration}
  {\Macro{chair|\OPName{Fußzeile}\MPName{Lehrstuhl}}}
\printdeclarationlist

Die mit diesen Befehlen gemachten Angaben werden nicht nur im Kopf sondern 
zusätzlich auch im linken Teil des Fußbereichs ausgegeben. Sollen diese für den 
Fußbereich angepasst werden, lässt das optionale Argument hierfür verwenden, 
wobei die Angabe eines leeren optionalen Argumentes das Feld für den Fuß 
komplett unterdrückt. Vor allen Angaben wird der Bezeichner \Term{contactname} 
in fetter Schrift ausgegeben.
\end{Declaration}
\end{Declaration}
\end{Declaration}
\end{Declaration}


\begin{Declaration}
  {\Macro{professor|\MPName{Name}}}
\printdeclarationlist

Zusätzlich zu den Angaben der Einrichtung kann mit \Macro{professor} der 
aktuelle Inhaber der genannten Professur im linken Fußbereich angegeben werden.
\end{Declaration}

\begin{Declaration}
  {\Macro{author|\MPName{Autor(en)}}}
\begin{Declaration}
  {\Macro{authormore|\MPName{Autorenzusatz}}}
\begin{Declaration}
  {\Macro{course|\MPName{Studiengang}}}
\begin{Declaration}
  {\Macro{discipline|\MPName{Studienrichtung}}}
\begin{Declaration}
  {\Macro{contactperson|\MPName{Name(n)}}}
\begin{Declaration}
  {\Macro{telephone|\MPName{Telefonnummer}}}
\begin{Declaration}
  {\Macro{telefax|\MPName{Telefaxnummer}}}
\begin{Declaration}
  {\Macro{office|\MPName{Adresse/Gebäude}}}
\printdeclarationlist

\ToDo{Hinweise/Links auf Standardklassen (in Deklaration)?}[v2.07]
%  \Macro||'full'(\Bundle{tudscr}){author} bzw wenn umgesetzt mit
%  begin{Declaration}{...}(\Macro(\Bundle{tudscr}){author}), 
Der oder die mit \Macro{author} angegebenen Autoren werden im rechten Teil des 
Fußbereichs (nacheinander) ausgegeben, mehrere Autoren sind mit \Macro{and} 
voneinander zu trennen. Die Befehle \Macro{authormore}, \Macro{emailaddress}, 
\Macro{course} und \Macro{discipline} sowie \Macro{office}, \Macro{telephone}, 
und \Macro{telefax} können für zusätzliche Angaben zu jedem Autor verwendet 
werden. Soll dies individuell für mehrere Autoren erfolgen, sind Hinweise zur 
Verwendung bei der Beschreibung von \Macro'full'{author} zu finden. Vor der 
Ausgabe aller Autoreninformationen wird der Bezeichner \Term{authorname} in 
fetter Schrift gesetzt. 

Danach folgen alle mit \Macro{contactperson} in gleicher Weise gemachten 
Angaben. Auch hier ist \Macro{and} für eine Trennung mehrerer Personen zu 
nutzen, wobei hier lediglich die Befehle \Macro{emailaddress}, \Macro{office}, 
\Macro{telephone} und \Macro{telefax} nicht jedoch \Macro{authormore} sowie 
\Macro{course} und \Macro{discipline} für zusätzliche Angaben zu verwenden 
sind. Bevor die Ansprechpartner ausgegeben werden, wird der Bezeichner 
\Term{contactpersonname} in fetter Schrift gesetzt. Es ist natürlich auch 
möglich nur Autor(en) oder Ansprechpartner anzugeben.
\end{Declaration}
\end{Declaration}
\end{Declaration}
\end{Declaration}
\end{Declaration}
\end{Declaration}
\end{Declaration}
\end{Declaration}

\begin{Declaration}
  {\Macro{webpage|\OPName{Einstellungen}\MPName{URL}}}
\begin{Declaration}
  {\Macro{webpage*|\MPName{URL}}}
\printdeclarationlist

Ganz zum Schluss kann für die rechte Spalte des Fußbereichs eine Homepage 
angegeben werden. Wurde das Paket \Package{hyperref} geladen, wird diese in 
einen Hyperlink gewandelt. Über das optionale Argument können beliebige 
Einstellungen an \Macro(\Package{hyperref}){hypersetup} aus besagtem Paket 
übergeben werden. Soll die Formatierung des Eintrags manuell erfolgen, so kann 
die Sternversion \Macro{webpage*} verwendet werden, wobei alle gewünschten 
Einstellungen innerhalb des Argumentes~-- gegebenenfalls in einer Gruppe~-- 
vorgenommen werden müssen.
\end{Declaration}
\end{Declaration}
\index{Layout!Fußzeile|)}%



\section{Sprachabhängige Bezeichner für den Fußbereich}

\begin{Declaration}
  {\Term{contactname}}
\begin{Declaration}
  {\Term{authorname}}
\begin{Declaration}
  {\Term{contactpersonname}}
\printdeclarationlist[Bezeichner|!]

Wie bereits zuvor erläutert, werden diese Bezeichner in der linken respektive 
rechten Spalte im Fuß vor der Ausgabe der eigentlichen Felder gesetzt.
\TermTable{contactname,authorname,contactpersonname}
\end{Declaration}
\end{Declaration}
\end{Declaration}

\ToDo[imp]{%
  Symbole für Telefon, Fax und E-Mail in \autoref{sec:poster:foot}? 
  Gibt's mittlerweile im Iconset \File*{TUD_Piktogramme.zip}%
  %\Email\fax\Faxmachine\FAX\Letter\Mobilefone\Telefon
}[v2.07]
\end{DeclareEntity}

\chapter[%
  Das Paket \Package*{tudscrsupervisor} -- Studentische Betreuung%
]{%
  Betreuung studentischer Arbeiten%
  \tudmarkuplabel[sec:supervisor]{\Package{tudscrsupervisor}}%
}
\begin{DeclareEntity}{\Package{tudscrsupervisor}}
%
Das Paket \Package{tudscrsupervisor} stellt für das Erstellen von 
Aufgabenstellungen und Gutachten wissenschaftlicher Arbeiten sowie offiziellen 
Aushängen im \CD passende Umgebungen und Befehle für den Anwender bereit. 
Deshalb richtet es sich vornehmlich an Mitarbeiter der \TnUD, kann jedoch 
natürlich auch von Studenten genutzt werden, um beispielsweise die offiziell 
erhaltene Aufgabenstellung für eine Abschlussarbeit im gleichen Stil wie das 
restliche Dokument mit \hologo{LaTeX} zu setzen.

An der \TUD gibt es unterschiedlichste Vorlagen für Aufgabenstellungen, 
Aushänge und ähnlichem. Falls die folgend beschriebenen Umgebungen und Befehle 
nicht ausreichend sind, um die Ihnen auferlegten Vorgaben umzusetzen, dürfen 
Sie gerne das \Forum oder \GitHubRepo<issues> besuchen, um mögliche 
Anpassungen sowie Erweiterungen zu diskutieren.



\section{%
  Aufgabenstellung für eine wissenschaftliche Arbeit%
  \index{Aufgabenstellung|!(}%
}
%
\begin{Declaration}
  {\Environment{task|\OPList{Überschrift}}}
  <\Environment{tudpage}>
\begin{Declaration}
  {\Environment{task/headline=\PSet{Überschrift}}}
\begin{Declaration}
  {\Environment{task/style=\PSet{Stil}}}
  [v2.05]
\printdeclarationlist
%
Mit der \Environment{task}"~Umgebung kann ein Aufgabenstellung für eine 
wissenschaftliche Arbeit ausgegeben werden. Für diese wird normalerweise eine 
Überschrift gesetzt, welche sich aus \Term{taskname} und~-- falls der Typ der 
Abschlussarbeit angegeben wurde~-- noch aus \Term{tasktext} und \Macro{thesis} 
zusammensetzt. Der Parameter \Environment{task/headline} kann genutzt werden, 
um diese automatisch generierte Überschrift anzupassen. Als Basis fungiert 
dabei die \Environment{tudpage}"~Umgebung, weshalb sich im optionalen Argument 
zusätzlich auch deren zuvor beschriebenen Parameter nutzen lassen.

Als Kopf der Aufgabenstellung erscheint eine Tabelle mit den angegebenen 
Informationen zum Autor respektive zu den Autoren der Abschlussarbeit. Zwingend 
anzugeben sind dafür lediglich ein oder mehrere Verfasser der Abschlussarbeit 
(\Macro(\Bundle{tudscr}){author}) sowie der dazugehörige Titel (\Macro{title}), 
welcher am Ende der Tabelle in fetter Schrift aufgeführt wird. Optional werden 
noch die Felder für den Studiengang (\Macro(\Bundle{tudscr}){course}), die 
Fachrichtung (\Macro(\Bundle{tudscr}){discipline}) sowie Matrikelnummer und 
Immatrikulationsjahr (\Macro{matriculationnumber}, \Macro{matriculationyear}) 
angefügt, wobei nicht befüllte Felder ignoriert werden. Der eigentliche Inhalt 
der Umgebung~-- sprich die Aufgabenstellung selbst~-- wird nach dem generierten 
Kopf ausgegeben.

Dem Inhalt der Aufgabenstellung folgt eine zeilenweise Auflistung der 
angegebenen Gutachter respektive Prüfer (\Macro{referee}) sowie Betreuer 
(\Macro{supervisor}). Dabei wird vor dem jeweiligen Namen der dazugehörige 
Bezeichner (\Term{refereename}, \Term{refereeothername} beziehungsweise 
\Term{supervisorname}, \Term{supervisorothername}) gesetzt. 
\ChangedAt{v2.05}%
Dies ist das voreingestellte Verhalten und kann über die Wahl des Parameters
\Environment{task/style=table} aktiviert werden. Wird hingegen der Parameter 
\Environment{task/style=inline} gesetzt, erfolgt die Ausgabe von mehreren 
Prüfern und Betreuern in einer Zeile. Danach erscheinen das Ausgabedatum 
(\Macro{issuedate}) sowie der verpflichtende Abgabetermin (\Macro{duedate}). Am 
Ende wird für den Prüfungsausschussvorsitzenden (\Macro{chairman}) und den 
betreuenden Hochschullehrer (\Macro(\Bundle{tudscr}){professor}) eine 
Unterschriftzeile gesetzt, wobei unter den Namen selbst die dazugehörigen 
Bezeichner (\Term{chairmanname} und \Term{professorname}) ausgegeben werden. 
Die Möglichkeiten zur individuellen Anpassung der genutzten Bezeichner ist in 
\autoref{sec:localization} beschrieben. 
\end{Declaration}
\end{Declaration}
\end{Declaration}

\begin{Declaration}
  {\Macro{taskform|\OList\MPName{Ziele}\MPName{Schwerpunkte}}}
\printdeclarationlist
%
Zusätzlich zur der frei gestaltbaren Umgebung \Environment{task} zur Erstellung
einer Aufgabenstellung wird ein separater Befehl für eine standardisierte 
Ausgabe zur Verfügung gestellt. Dieser strukturiert die Aufgabenstellung in die 
zwei Bereiche \emph{Ziele} und \emph{Schwerpunkte} der Arbeit mit dazugehörigen 
Überschriften (\Term{objectivesname}, \Term{focusname}).

Im optionalen Argument können alle Parameter der Umgebung \Environment{task} 
verwendet werden. Im ersten obligatorischen Argument sollte ein Text mit einer 
kurzen thematischen Einordnung und dem eigentlichen Ziel der Arbeit erscheinen, 
im zweiten Argument sollen die thematischen Schwerpunkte in Stichpunkten 
benannt werden. Der Inhalt des zweiten notwendigen Argumentes wird in einer 
\Environment{itemize}"~Umgebung gesetzt. Deshalb \emph{muss} jedem Stichpunkt 
\Macro{item} vorangestellt werden.
\end{Declaration}
%
\begin{Example}
Die empfohlene Verwendung des Befehls \Macro{taskform} ist wie folgt:
\begin{Code}[escapechar=§]
\taskform{%
  Motivation der Arbeit im ersten Absatz§\dots§
  
  Ziele der Arbeit im zweiten Absatz§\dots§
}{%
  \item Schwerpunkt 1
  \item Schwerpunkt 2
}
\end{Code}
Hierzu sei auch auf das Minimalbeispiel in \autoref{sec:exmpl:task} verwiesen.%
\index{Aufgabenstellung|!)}%
\end{Example}

\begin{Declaration}
  {\Macro{chairman|\MPName{Prüfungsausschussvorsitzender}}}
\printdeclarationlist
%
Mit diesem Befehl wird am Ende der Aufgabenstellung zusätzlich zum betreuenden 
Hochschullehrer (\Macro(\Bundle{tudscr}){professor}) auch der Vorsitzende des 
Prüfungsausschusses aufgeführt. Dies wird zumeist für Abschlussarbeiten wie 
beispielsweise \masterthesisname{} oder \diplomathesisname{} benötigt.
\end{Declaration}

\begin{Declaration}
  {\Macro{issuedate|\MPName{Ausgabedatum}}}
\begin{Declaration}
  {\Macro{duedate|\MPName{Abgabetermin}}}
\printdeclarationlist
%
Mit diesen beiden Befehlen sollte das Datum der Ausgabe der Aufgabenstellung 
sowie der spätest mögliche Abgabetermin angegeben werden. Wurde das Paket 
\Package{isodate} oder \Package{datetime2} geladen, wird die damit eingestellte 
Ausgabeformatierung des Datums mit \Macro{printdate} respektive \Macro{DTMDate} 
für \Macro{duedate} und \Macro{issuedate} verwendet.
\end{Declaration}
\end{Declaration}



\section{%
  Gutachten für wissenschaftliche Arbeiten%
  \index{Gutachten|!(}%
}
%
\begin{Declaration}
  {\Environment{evaluation|\OPList{Überschrift}}}
  <\Environment{tudpage}>
\begin{Declaration}
  {\Environment{evaluation/headline=\PSet{Überschrift}}}
\begin{Declaration}
  {\Environment{evaluation/grade=\PSet{Note}}}
\printdeclarationlist
%
Diese Umgebung wird für das Erstellen eines Gutachtens einer wissenschaftlichen 
Arbeit bereitgestellt. Auch diese unterstützt alle Parameter, welche für die 
Umgebung \Environment{tudpage} beschrieben wurden. Für ein Gutachten wird 
gewöhnlich eine Überschrift aus \Term{evaluationname} und~-- falls der 
Abschlussarbeitstyp angegeben wurde~-- \Term{evaluationtext} sowie 
\Macro{thesis} generiert. Diese automatisch generierte Überschrift kann mit dem 
Parameter \Environment{evaluation/headline} ersetzt werden. Am Ende des 
Gutachtens wird die mit \Environment{evaluation/grade} gegebene Note in fetter 
Schrift ausgezeichnet.

Am Anfang der \Environment{evaluation}"~Umgebung wird die gleiche Tabelle mit 
Autorenangaben ausgegeben, wie dies bei der \Environment{task}"~Umgebung der 
Fall ist. Nach dem Tabellenkopf folgt auch hier der eigentliche Inhalt, sprich 
das Gutachten der Abschlussarbeit. Abgeschlossen wird die Umgebung mit der 
gegebenen Note~-- welche innerhalb von \Term{gradetext} ausgegeben wird~-- 
sowie der Orts- und Datumsangabe (\Macro{place}, \Macro{date}) und der 
darauffolgenden Unterschriftzeile für den oder die Gutachter (\Macro{referee}), 
welche wiederum mit den entsprechenden sprachabhängigen Bezeichner 
(\Term{refereename}, \Term{refereeothername}) ergänzt werden.
\end{Declaration}
\end{Declaration}
\end{Declaration}

\begin{Declaration}
  {\Macro{evaluationform|%
    \OList\MPName{Aufgabe}\MPName{Inhalt}\MPName{Bewertung}\MPName{Note}}}
\printdeclarationlist
%
Neben der individuell nutzbaren Umgebung \Environment{evaluation} wird ein 
separater Befehl zur Erstellung eines standardisierten Gutachtens 
bereitgestellt. Dieser strukturiert die Ausgabe in die vier Bereiche 
\emph{Aufgabe}, \emph{Inhalt}, \emph{Bewertung} und \emph{Note} und versieht 
diese jeweils mit der dazugehörigen Überschrift beziehungsweise Textausgabe 
(\Term{taskname}, \Term{contentname}, \Term{assessmentname} und 
\Term{gradetext}). Das optionale Argument unterstützt alle Parameter der 
\Environment{evaluation}"~Umgebung.
\end{Declaration}
%
\begin{Example}
Die empfohlene Verwendung des Befehls \Macro{evaluationform} ist wie folgt:
\begin{Code}[escapechar=§]
\evaluationform{%
  Kurzbeschreibung der Aufgabenstellung§\dots§
}{%
  Zusammenfassung von Inhalt und Struktur§\dots§
}{%
  Bewertung der schriftlichen Abschlussarbeit§\dots§
}{%
  Zahl (Note)
}
\end{Code}
Hierzu sei auch auf das Minimalbeispiel in \autoref{sec:exmpl:evaluation} 
verwiesen.%
\index{Gutachten|!)}%
\end{Example}

\begin{Declaration}
  {\Macro{grade|\MPName{Note}}}
\printdeclarationlist
%
Neben der Angabe der Note für ein Gutachten über den Parameter 
\Environment{evaluation/grade} der Umgebung \Environment{evaluation} kann dafür 
auch dieser global wirkende Befehl verwendet werden.
\end{Declaration}



\section{%
  Aushang%
  \index{Aushang|!(}%
}
%
\begin{Declaration}
  {\Environment{notice|\OPList{Überschrift}}}
  <\Environment{tudpage}>
\begin{Declaration}
  {\Environment{notice/headline=\PSet{Überschrift}}}
\printdeclarationlist
%
Für das Anfertigen eines Aushangs kann diese Umgebung verwendet werden, welche
abermals auf der \Environment{tudpage}"=Umgebung basiert und deren Parameter 
unterstützt. Wurde mit \Macro{date} ein Datum angegeben, wird dieses 
standardmäßig rechtsbündig oberhalb des Textbereiches angezeigt 
(\Environment||{tudpage/cdhead=date}). Danach erfolgt die Ausgabe der 
Überschrift, welche normalerweise \Term{noticename} entspricht und mit dem 
Parameter \Environment{notice/headline} geändert werden kann. Nach der 
Überschrift folgt der Inhalt der Umgebung. Wurde mit \Macro{contactperson} ein 
oder mehrere Ansprechpartner angegeben, werden diese Informationen am Ende der 
Umgebung gesetzt.
\end{Declaration}
\end{Declaration}

\begin{Declaration}
  {\Macro{noticeform|\OList\MPName{Inhalt}\MPName{Schwerpunkte}}}
\printdeclarationlist%
%
Auch für diese Umgebung gibt es einen Befehl für eine normierte Form. Diese 
soll vor allem Verwendung für den Aushang von Angeboten für studentische 
Arbeiten finden. Für das optionale Argument lassen sich sämtliche Parameter 
verwenden, die auch von der \Environment{notice}"~Umgebung unterstützt werden.

Das erste obligatorische Argument wird für eine kurze Inhaltsbeschreibung 
verwendet. Neben dem Text sollte hier~-- wenn möglich~-- eine thematisch 
passende Abbildung (\Macro{includegraphics}) eingefügt werden. Mit dem zweiten 
Argument erfolgt~-- wie schon bei \Macro{taskform}~-- die Ausgabe einiger 
Schwerpunkte, wobei hier ebenso eine \Environment{itemize}"~Umgebung direkt 
nach der Überschrift (\Term{focusname}) zum Einsatz kommt und deshalb allen 
Stichpunkten ein \Macro{item} vorangestellt werden \emph{muss}.
\end{Declaration}
%
\begin{Example}
Die empfohlene Verwendung des Befehls \Macro{noticeform} ist wie folgt:
\begin{Code}[escapechar=§]
\noticeform{%
  Kurzbeschreibung des Inhaltes der studentischen Arbeit§\dots§
  
  Bild (optional), einzubinden mit:
    \includegraphics[§\PName{Einstellungen}§]{§\PName{Datei}§}
}{%
  \item Schwerpunkt 1
  \item Schwerpunkt 2
  \item Schwerpunkt 3
}
\end{Code}
Hierzu sei auch auf das Minimalbeispiel in \autoref{sec:exmpl:notice} 
verwiesen.%
\index{Aushang|!)}%
\end{Example}

\begin{Declaration}
  {\Macro{contactperson|\MPName{Kontaktperson(en)}}}
  [v2.02]
\begin{Declaration*}
  {\Macro{emailaddress|\OPName{Einstellungen}\MPName{E-Mail-Adresse}}}
\begin{Declaration*}
  {\Macro{emailaddress*|\MPName{E-Mail-Adresse}}}
\begin{Declaration}
  {\Macro{telephone|\MPName{Telefonnummer}}}
  [v2.02]
\begin{Declaration}
  {\Macro{telefax|\MPName{Telefaxnummer}}}
  [v2.05]
\begin{Declaration}
  {\Macro{office|\MPName{Dienstsitz}}}
\printdeclarationlist
%
\ToDo{Hinweise/Links auf Standardklassen (in Deklaration)?}[v2.07]
%  \Macro||'full'(\Bundle{tudscr}){author} bzw wenn umgesetzt mit
%  begin{Declaration}{...}(\Macro(\Bundle{tudscr}){author}),
Am Ende eines Aushangs lassen sich mit dem Befehl \Macro{contactperson} 
Kontaktinformationen für einen oder mehr Ansprechpartner angeben. Mehrere 
Kontaktpersonen müssen innerhalb dieses Befehls mit \Macro{and} voneinander 
getrennt werden. Für jede Person kann innerhalb des Argumentes der Dienstsitz 
(\Macro{office}), die dienstliche Telefon- (\Macro{telephone}) sowie "~faxnummer
(\Macro{telefax}) und die geschäftliche E"~Mail"=Adresse (%
  \Macro(\Bundle{tudscr}){emailaddress}%
  \footnote{Erläuterungen zu \Macro||'full'(\Bundle{tudscr}){emailaddress}}%
) angegeben werden. 
\end{Declaration}
\end{Declaration}
\end{Declaration}
\end{Declaration*}
\end{Declaration*}
\end{Declaration}

%\begin{Declaration}
%  {\Macro{webpage|\OPName{Einstellungen}\MPName{URL}}}
%\begin{Declaration}
%  {\Macro{webpage*|\MPName{URL}}}
%\printdeclarationlist%
%%
%Ganz zum Schluss kann für die rechte Spalte des Fußbereichs eine Homepage 
%angegeben werden. Wurde das Paket \Package{hyperref} geladen, wird diese in 
%einen Hyperlink gewandelt. Über das optionale Argument können beliebige 
%Einstellungen an \Macro{hypersetup} aus besagtem Paket übergeben werden. Soll 
%die Formatierung des Eintrags manuell erfolgen, so kann die Sternversion 
%\Macro{webpage*} verwendet werden, wobei alle gewünschten Einstellungen 
%innerhalb des Argumentes~-- gegebenenfalls in einer Gruppe~-- vorgenommen 
%werden müssen.
%\end{Declaration}
%\end{Declaration}



\section{%
  Zusätzliche sprachabhängige Bezeichner%
  \index{Bezeichner|!(}%
}
%
Für die zuvor erläuterten Befehle und Umgebungen das werden ergänzend zu den 
\TUDScript-Klassen weitere Bezeichner definiert. Für eine etwaige Anpassung 
dieser sei auf \autoref{sec:localization} verwiesen. Soll ein bestimmter 
Bezeichner lediglich einmalig für eine Umgebung oder einen Befehl geändert 
werden, sollten Sie weiterhin die Empfehlungen in \autoref{sec:tips:local} 
beachten.

\begin{Declaration}
  {\Term{taskname}}
\begin{Declaration}
  {\Term{tasktext}}
\printdeclarationlist
%
Die Bezeichnung der Aufgabenstellung selbst ist in \Term{taskname} enthalten. 
Für die Generierung einer Überschrift wird dieser verwendet. Wurde außerdem mit 
\Macro{thesis} oder \Macro{subject} der Typ der Abschlussarbeit%
\footnote{%
  spezieller Wert aus \autoref{tab:thesis} oder \Option||{subjectthesis=true}%
}
angegeben, wird die Überschrift zusammen mit dem Bezeichner \Term{tasktext}
um die Typisierung erweitert. Falls gewünscht, kann die automatisch generierte 
Überschrift mit dem Parameter \Environment{task/headline} der Umgebung 
\Environment{task} überschrieben werden.
\TermTable%
\end{Declaration}
\end{Declaration}

\begin{Declaration}
  {\Term{namesname}}
  [v2.04]
\begin{Declaration}
  {\Term{titlename}}
\printdeclarationlist
%
Diese beiden Bezeichner werden in der Tabelle mit den Autoreninformationen zu 
Beginn der Aufgabenstellung verwendet.
\TermTable%
\end{Declaration}
\end{Declaration}

\begin{Declaration}
  {\Term{issuedatetext}}
\begin{Declaration}
  {\Term{duedatetext}}
\printdeclarationlist
%
Am Ende der Aufgabenstellung wird nach dem oder der Betreuer das Ausgabedatum 
und der Abgabetermin (\Macro{issuedate}, \Macro{duedate}) der Abschlussarbeit 
mit folgenden Bezeichner erläutert.
\TermTable%
\end{Declaration}
\end{Declaration}

\begin{Declaration}
  {\Term{chairmanname}}
\printdeclarationlist
%
Wurde der Prüfungsausschussvorsitzende (\Macro{chairman}) angegeben, erfolgt 
unter dem Namen selbst die Ausgabe des Bezeichners.
\TermTable%
\end{Declaration}

\begin{Declaration}
  {\Term{focusname}}
\begin{Declaration}
  {\Term{objectivesname}}
\printdeclarationlist
%
Die Standardformen für Aufgabenstellung (\Macro{taskform}) respektive Aushang 
(\Macro{noticeform}) nutzen für die gesetzten Überschriften diese Bezeichner.
\TermTable%
\end{Declaration}
\end{Declaration}

\begin{Declaration}
  {\Term{evaluationname}}
\begin{Declaration}
  {\Term{evaluationtext}}
\printdeclarationlist
%
Die Bezeichnung des Gutachten selbst ist in \Term{evaluationname} enthalten. 
Für die Generierung der Überschrift wird der Bezeichner \Term{evaluationtext} 
sowie der mit \Macro{thesis} oder gegebenenfalls mit \Macro{subject} gegebenen 
Typ der Abschlussarbeit verwendet. Diese automatisch generierte Überschrift 
kann mit dem Parameter \Environment{evaluation/headline} der Umgebung 
\Environment{evaluation} durch den Anwender überschrieben werden.
\TermTable%
\end{Declaration}
\end{Declaration}

\begin{Declaration}
  {\Term{contentname}}
\begin{Declaration}
  {\Term{assessmentname}}
\printdeclarationlist
%
Bei der standardisierten Form des Gutachtens (\Macro{evaluationform}) werden 
die darin~-- für eine strukturierte Gliederung~-- erzeugten Überschriften mit 
den Bezeichnern \Term{taskname}, \Term{contentname} und \Term{assessmentname} 
gesetzt.
\TermTable%
\end{Declaration}
\end{Declaration}

\begin{Declaration}
  {\Term{gradetext}}
\printdeclarationlist
%
Wird für das Gutachten einer wissenschaftlichen Arbeit die erzielte Note 
entweder mit dem Befehl \Macro{grade|\MPName{Note}} oder alternativ dazu 
mit dem Parameter \Environment{evaluation/grade=\PSet{Note}} der Umgebung 
\Environment{evaluation} angegeben, so wird diese innerhalb von 
\Term{gradetext} verwendet.
\grade{\PName{Note}}
\TermTable[.8\textwidth]%
\end{Declaration}

\begin{Declaration}
  {\Term{noticename}}
\begin{Declaration}
  {\Term{contactpersonname}}
  [v2.02]
\printdeclarationlist
%
Die Bezeichnung des Aushangs selbst ist in \Term{noticename} enthalten. Für 
die Generierung einer Überschrift wird dieser verwendet. Falls gewünscht, kann 
diese mit dem Parameter \Environment{notice/headline} der Umgebung 
\Environment{notice} überschrieben werden. Wurde eine Kontaktperson mit dem 
Befehl \Macro{contactperson} angegeben, wird als Überschrift der Kontaktdaten 
der Bezeichner \Term{contactpersonname} verwendet.
\TermTable%
\end{Declaration}
\end{Declaration}
\index{Bezeichner|!)}%
\end{DeclareEntity}

\setchapterpreamble{%
  \begin{abstract}
    Zusätzlich zu den bisher im Anwenderhandbuch vorgestellten Klassen und 
    Paketen werden im \TUDScript-Bundle weitere Paket bereitgestellt. Diese 
    sind nicht zwingend an die Verwendung einer der \TUDScript-Klassen 
    angewiesen sondern können prinzipiell mit jeder \hologo{LaTeX}"~Klasse 
    genutzt werden.%
    \ToDo[doc]{tudscrcolor als separates Kapitel, der Rest in Anhang}[v2.07]
  \end{abstract}
}
\chapter{%
  Zusätzliche Pakete im \TUDScript-Bundle%
  \label{sec:bundle}%
}
\section[%
  Das Paket \Package{tudscrcolor} -- Farben im \CD%
]{%
  Farben im \CD%
  \tudmarkuplabel{\Package{tudscrcolor}}%
  \index{Farben|(}%
  \index{Layout!Farben|?(}%
}
%
\begin{DeclareEntity}{\Package{tudscrcolor}}
\ToDo[doc]{als separates Kapitel}[v2.07]
Zur Verwendung der Farben des \CDs wird das Paket \Package{tudscrcolor} 
genutzt. Falls dieses nicht in der Präambel geladen wird~-- um beispielsweise 
zusätzliche Optionen aufzurufen~-- binden die \TUDScript-Klassen dieses 
automatisch ein.

Für das \CD sind mehrere Farben vorgesehen. Die prägnanteste aller ist die 
Hausfarbe \Color{HKS41}, danach folgen die Farben für Auszeichnungen der ersten
(\Color{HKS44}) und der zweiten Kategorie (\Color{HKS36}, \Color{HKS33}, 
\Color{HKS57}, \Color{HKS65}) sowie eine Ausnahmefarbe (\Color{HKS07}). 
Diese Farben dürfen sowohl in ihrer Grundform als auch in helleren Tönen mit 
einer Abstufung in 10\,\%"~Schritten verwendet werden. Das ohnehin verwendete 
Paket \Package{xcolor} stellt genau diese Funktionalität zur Verfügung. Jede 
der Farben kann sowohl mit \Color*{HKS\PName{Zahl}} als auch über ein Pseudonym 
\Color*{cd\PName{Farbe}} genutzt werden.
%
\begin{Example}
Die Grundfarbe \Color{HKS44} soll in der auf 20\% reduzierten, helleren 
Abstufung genutzt werden. Innerhalb eines Befehls, der als Argument eine 
gültige Farbe erwartet, muss lediglich \PValue{HKS44!20} angegeben werden. 
Dies wird hier exemplarisch mit der folgenden \colorbox{HKS44!20}{%
  Box \Macro{colorbox|\MPValue{HKS44!20}\MPValue{Box}}%
}
demonstriert.
\end{Example}
%
Bei der farbigen Gestaltung des \CDs ist~-- je nach spezifisch gewählter 
Ausprägung (\Option'page'(\Bundle{tudscr}){cd=\PMisc})~-- der Hintergrund 
respektive Querbalken von Umschlagseite, Titel sowie Teilen in \Color{HKS41} 
und die Schrift auf selbigen in \Color{HKS41|\Color*{!30}} gehalten. Der 
Hintergrund von separaten Kapitelseiten erscheint gegebenenfalls in 
\Color{HKS41|\Color*{!10}}, die Schrift in \Color{HKS41}. Bei geringerem 
Farbeinsatz werden lediglich die Schriften der Gliederungsseiten auf 
\Color{HKS41} gesetzt.

Soll das Paket \Package{xcolor} mit speziellen Optionen genutzt werden, lassen 
sich diese \emph{vor} dem Laden der genutzten Dokumentklasse mit 
\Macro{PassOptionsToPackage|\MPName{Paketoptionen}\MPValue{xcolor}} an dieses 
übergeben. Alternativ kann das Paket \Package{tudscrcolor} mit den gewünschten 
Optionen geladen werden, da dieses Paket unbekannte Einstellungen an 
\Package{xcolor} durchreicht.



\subsection{Generelle Farbdefinitionen}
%
\newcommand*\cdcolorcalc{}
\newcommand*\cdcolorname{}
\newcommand*\cdcolorvalue{}
\newcommand*\cdcolortext{}
\newcommand*\cdcolor[2][0]{%
  \ifhmode%
    \vskip\medskipamount%
  \else%
    \addvspace{\medskipamount}%
  \fi%
  \noindent%
  \begin{tikzpicture}[%
    every node/.style={%
      rectangle, minimum height=.1\linewidth, minimum width=7em%
    }%
  ]%
  \def\cdcolorcalc##1##2{%
    \pgfmathparse{100-##1*10}%
    \xdef\cdcolorname{HKS##2!\pgfmathresult}%
    \xdef\cdcolorvalue{\pgfmathresult}%
    \pgfmathparse{10+##1*10}%
  }%
  \foreach \x in {0,1,...,9}{%
    \cdcolorcalc{\x}{#2}%
    \ifnum\x<#1%
      \def\cdcolortext{white}%
    \else%
      \def\cdcolortext{black}%
    \fi%
    \node [fill=\cdcolorname,rotate=90] at (.\x\linewidth,0)%
      {\textcolor{\cdcolortext}{HKS#2!\pgfmathprintnumber\cdcolorvalue}};%
  }%
  \end{tikzpicture}%
  \par%
}
\minisec{Primäre Hausfarbe}
\begin{Declaration}
  {\Color{HKS41}}
  (Alternativ:\Color*{cddarkblue|})
\printdeclarationlist
\cdcolor[6]{41}
\end{Declaration}

\minisec{Sekundäre Hausfarbe (Geschäftsausstattung)}
\begin{Declaration}
  {\Color{HKS92}}
  (Alternativ:\Color*{cdgray|})
\printdeclarationlist
\cdcolor[4]{92}
\end{Declaration}

\minisec{Auszeichnungsfarbe 1.\,Kategorie}
\begin{Declaration}
  {\Color{HKS44}}
  (Alternativ:\Color*{cdblue|})
\printdeclarationlist
\cdcolor[4]{44}
\end{Declaration}

\minisec{Auszeichnungsfarbe 2.\,Kategorie}
\begin{Declaration}
  {\Color{HKS36}}
  (Alternativ:\Color*{cdindigo|})
\begin{Declaration}
  {\Color{HKS33}}
  (Alternativ:\Color*{cdpurple|})
\begin{Declaration}
  {\Color{HKS57}}
  (Alternativ:\Color*{cddarkgreen|})
\begin{Declaration}
  {\Color{HKS65}}
  (Alternativ:\Color*{cdgreen|})
\printdeclarationlist
\cdcolor[4]{36}
\cdcolor[4]{33}
\cdcolor[2]{57}
\cdcolor{65}
\end{Declaration}
\end{Declaration}
\end{Declaration}
\end{Declaration}

\minisec{Ausnahmefarbe}
\begin{Declaration}
  {\Color{HKS07}}
  (Alternativ:\Color*{cdorange|})
\printdeclarationlist
\cdcolor{07}
\end{Declaration}



\subsection{Zusätzliche Farbdefinitionen}
%
\begin{Declaration}
  {\Option{reduced}}
  ()
\printdeclarationlist
%
Das Paket \Package{tudscrcolor} definiert lediglich die zuvor beschriebenen 
Grundfarben \Color{HKS41}, \Color{HKS92}, \Color{HKS44}, \Color{HKS36}, 
\Color{HKS33}, \Color{HKS57}, \Color{HKS65} sowie \Color{HKS07}. 
Alle anderen farblichen Abstufungen können mit den beschrieben Möglichkeiten 
des Paketes \Package{xcolor} generiert werden.
\end{Declaration}

\begin{Declaration}
  {\Option{extended}}
\printdeclarationlist
%
Neben dem \TUDScript-Bundle existieren viele verschiedene Klassen und Pakete 
für das \CD, welche teilweise abweichende Farbdefinitionen nutzen. Durch die 
Paketoption \Option{extended} werden hierfür Farben nach dem Schema 
\Color*{HKS41K\PName{Zahl}} und \Color*{HKS41-\PName{Zahl}} definiert, wobei 
der angestellte Zahlenwert aus der 10er"~Reihe kommen muss.
\end{Declaration}



\subsection{%
  Umstellung des Farbmodells%
  \index{Farben!Farbmodell}%
}
%
Normalerweise verwendet \Package{tudscrcolor} das CMYK"~Farbmodell. Außerdem 
wird weiterhin noch der RGB"~Farbraum unterstützt. Eine Umschaltung des 
Farbmodells ist beispielsweise für gewisse Funktionen des Paketes 
\Package{tikz} notwendig.

\begin{Declaration}
  {\Option{RGB}}
\printdeclarationlist
%
Diese Option wird an das Paket \Package{xcolor} durchgereicht, wodurch bereits 
beim Laden des Paketes die globale Farbdefinition nicht nach dem 
CMYK"~Farbmodell sondern im RGB"~Farbraum erfolgt.
\end{Declaration}

\begin{Declaration}
  {\Macro{setcdcolors|\MPName{Farbmodell}}}
\printdeclarationlist
%
Mit diesem Befehl kann innerhalb des Dokumentes das verwendete Farbmodell 
angepasst werden. Damit ist es möglich, lokal innerhalb einer Umgebung den 
Farbmodus zu ändern und so nur in bestimmten Situationen beispielsweise aus dem 
CMYK"~Farbmodell in den RGB"~Farbraum zu wechseln. Unterstützte Werte für 
\PName{Farbmodell} sind \PValue{CMYK} und \PValue{RGB} beziehungsweise 
\PValue{rgb}.
\end{Declaration}

\smallskip\noindent
\Attention{%
  Die Darstellung der Farben kann im jeweiligen Farbmodus 
  (\PValue{CMYK} oder \PValue{RGB}) je nach verwendeter Bildschirm"~, Drucker- 
  und Softwarekonfiguration verschieden ausfallen. Die Farbwerte entstammen dem 
  Handbuch zum \CD und sind lediglich Näherungswerte. Abweichungen vom 
  gedruckten HKS"~Farbregister und selbst ermittelten Werten sind technisch 
  nicht zu vermeiden.%
}%
\index{Farben|)}%
\index{Layout!Farben|?)}%
\end{DeclareEntity}



\section[%
  Das Paket \Package{tudscrfonts} -- Schriften im \CD%
]{%
  Schriften im \CD%
  \tudmarkuplabel{\Package{tudscrfonts}}%
}
%
\begin{DeclareEntity}{\Package{tudscrfonts}}[%
  v2.02:Neues Paket zur Nutzung Hausschriften ohne \TUDScript-Klassen;%
  v2.06:\OpenSans als neue Hausschrift standardmäßig aktiviert;%
]
%
Das Paket \Package{tudscrfonts} stellt die Hausschriften des \TUDCDs für 
\hologo{LaTeX}"~Klassen bereit, welche \emph{nicht} zum \TUDScript-Bundle 
gehören. Das Paket unterstützt einen Großteil der normalerweise für die 
\TUDScript-Klassen bereitgestellten Optionen und Befehle für die 
Schriftauswahl. Um Dopplungen in der Dokumentation zu vermeiden, wird an dieser 
Stelle auf eine abermalige Erläuterung der im Paket \Package{tudscrfonts} 
verfügbaren Optionen und Befehle verzichtet. Diese werden im Folgenden 
lediglich noch einmal genannt, die dazugehörigen Erläuterungen sind in 
\fullref{sec:fonts} zu finden.

Die nutzbaren Paketoptionen sind für den Fließtext \Option{cdfont}~-- ohne die 
Einstellungsmöglichkeiten für den Querbalken des \CDs (\Option{cdhead})~-- und 
für die mathematischen Schriften \Option{cdmath} sowie \Option{slantedgreek}. 
Für die Auswahl der Schreibmaschinenschriften ist \Option{ttfont} verfügbar. 
Weiterhin wird die Option \Option{relspacing} bereitgestellt. Alle genannten 
Optionen können entweder direkt als Klassenoption oder als Paketoptionen im 
optionalen Argument von 
\Macro*{usepackage|\OPName{Paketoptionen}\MPValue{tudscrfonts}} angegeben 
werden. Zusätzlich ist nach dem Laden des Paketes die Optionenwahl mit 
\Macro{TUDoption} respektive \Macro{TUDoptions} möglich.

Die in \autoref{sec:fonts} beschriebenen Textschalter und "~kommandos zur 
expliziten Auswahl einzelner Schnitte der Hausschrift des \CDs sowie die 
Befehle für griechische Lettern werden ebenso bereitgestellt.
\ChangedAt{v2.06}%
Soll nicht \OpenSans sondern die alten Schriftfamilien \Univers und \DIN 
genutzt werden, ist eine Installation dieser notwendig. Hinweise hierzu sind 
unter \autoref{sec:install:fonts} zu finden. Um diese zu aktivieren, sei auf 
\Option{tudscrver=2.05} und \Option{cdoldfont} verwiesen.

\ChangedAt{v2.05}%
Ursprünglich war das Paket \Package{tudscrfonts} vorrangig zur Verwendung in 
Verbindung mit Dokumentklassen im \TUDCD, welche nicht zu \TUDScript gehören, 
vorgesehen. Allerdings traten dabei einige Unzulänglichkeiten auf, weshalb seit 
der Version~v2.05 empfohlen wird, hierfür das Paket \Package{fix-tudscrfonts} 
zu verwenden. Die unterstützen Klassen sind bei dessen Beschreibung gelistet. 
\end{DeclareEntity}




\section[%
  Das Paket \Package{mathswap} -- Dezimal- und Tausendertrennzeichen%
]{%
  Formatierung von Dezimal- und Tausendertrennzeichen%
  \tudmarkuplabel{\Package{mathswap}}%
  \index{Mathematiksatz|(}%
  \index{Zifferngruppierung|(}%
}
%
\begin{DeclareEntity}{\Package{mathswap}}
Die Verwendung von Dezimal- und Tausendertrennzeichen im mathematischen Satz 
sind regional sehr unterschiedlich. In den meisten englischsprachigen Ländern 
wird der Punkt als Dezimaltrennzeichen und das Komma zur Zifferngruppierung 
verwendet, im restlichen Europa wird dies genau entgegengesetzt praktiziert.
Generell sollte beim mathematischen Satz~-- möglichst automatisiert~-- der 
Unterschied beachtet werden, ob es sich bei einem zu setzenden Komma um eine 
gewöhnlichen Interpunktion ($z=f(x,y)$)~-- wo ein Abstand nach dem Komma 
gewünscht ist~-- oder einem Dezimaltrennzeichen ($t=1,\!2$) handelt.
\ToDo{in Tipps \& Tricks, Paket mathswap sollte verschwinden}

Das Paket \Package{mathswap} stellt diese Funktionalität bereit, indem Punkt 
(\:\textbf{.}\:) respektive Komma (\:\textbf{,}\:) im Mathematikmodus als 
Trennzeichen direkt vor einer Ziffer erkannt und nach bestimmten Vorgaben 
ersetzt werden. Sollte sich jedoch zwischen Trennzeichen und Ziffer Leerraum 
befinden, wird dieser als solcher auch gesetzt. Für ein Beispiel zur Verwendung 
des Paketes sei auf das Tutorial \Tutorial{mathswap} in \autoref{sec:tut} 
hingewiesen.

Das Paket \Package{ionumbers} verfolgt dasselbe Ziel mit einem ähnlichen 
Ansatz, stellt allerdings wesentlich mehr Anpassungsmöglichkeiten bereit und 
sollte deshalb als bevorzugte Alternative angesehen werden. Auch die Pakete 
\Package*{ziffer} und \Package*{icomma} stellen ähnliche Funktionalitäten 
bereit. Ersteres ist leider etwas unflexibel, was den Umgang mit Trennzeichen 
anbelangt und bei letzterem wird standardmäßig jedes Komma ohne Leerraum 
gesetzt, sodass der Autor manuell Leerzeichen einfügen muss, falls diese 
gewünscht sind. 

\begin{Declaration}
  {\Macro{commaswap|\MPName{Trennzeichen}}}
\begin{Declaration}
  {\Macro{dotswap|\MPName{Trennzeichen}}}
\printdeclarationlist
%
Die beiden Befehle \Macro{commaswap} und \Macro{dotswap} sind die zentrale 
Benutzerschnittstelle des Paketes. Das Makro \Macro{commaswap} definiert das 
Trennzeichen oder den Inhalt, wodurch ein Komma ersetzt werden soll, auf 
welches direkt danach eine Ziffer folgt. Normalerweise setzt \hologo{LaTeX}
nach einem Komma im mathematischen Satz zusätzlich einen horizontalen Abstand.
Bei der Ersetzung durch \Macro{commaswap} entfällt dieser. Die Voreinstellung
für \Macro{commaswap} ist deshalb auf ein Komma (,) gesetzt. Mit dem Makro 
\Macro{dotswap} kann definiert werden, wodurch der Punkt im mathematischen 
Satz ersetzt werden soll, wenn auf diesen direkt anschließend eine Ziffer 
folgt. Da der Punkt im deutschsprachigem Raum zur Gruppierung von Ziffern 
genutzt wird, ist hierfür standardmäßig ein halbes geschütztes Leerzeichen 
definiert (\Macro*{,}).
\end{Declaration}
\end{Declaration}

\begin{Declaration}
  {\Macro{mathswapon}}
\begin{Declaration}
  {\Macro{mathswapoff}}
\printdeclarationlist
%
\ChangedAt{v2.02:Funktionalität im Dokument umschaltbar}%
Die Funktionalität von \Package{mathswap} kann innerhalb des Dokumentes mit 
diesen beiden Befehlen an- und abgeschaltet werden. Beim Laden des Paketes ist 
es standardmäßig aktiviert.%
\end{Declaration}
\end{Declaration}
\index{Mathematiksatz|)}%
\index{Zifferngruppierung|)}%
\end{DeclareEntity}



\section[%
  Das Paket \Package{twocolfix} -- Bugfix für den zweispaltigen Satz%
]{%
  Bugfix für den zweispaltigen Satz%
  \tudmarkuplabel{\Package{twocolfix}}%
  \index{Satzspiegel!zweispaltig|?}%
}
%
\begin{DeclareEntity}{\Package{twocolfix}}
Der \hologo{LaTeXe}"~Kernel enthält einen Fehler, der Kapitelüberschriften im
zweispaltigen Layout höher setzt, als im einspaltigen. Der 
\href{http://www.latex-project.org/cgi-bin/ltxbugs2html?pr=latex/3126}{Fehler}
ist zwar schon länger bekannt, allerdings bisher noch nicht in den 
\hologo{LaTeXe}"~Kernel übernommen worden. Das Paket \Package{twocolfix} behebt 
dieses Problem. Eine Integration des Bugfixes in \KOMAScript wurde bereits bei 
Markus Kohm angefragt, jedoch von ihm bis jetzt 
\href{http://www.komascript.de/node/1681}{nicht weiter verfolgt}.
\end{DeclareEntity}


\addsec*{Zukünftige Arbeiten}
Diese Dinge sollen langfristig in das \TUDScript-Bundle eingearbeitet werden:

%\chapter[%
%  Das Paket \Package{tudscrletter} -- Briefe im \CD%
%]{%
%  Briefe im \CD%
%  \tudmarkuplabel{\Package{tudscrletter}}%
%}
%\begin{DeclareEntity}{\Package{tudscrletter}}
\ToDo[imp]{%
  Paket \Package*{tudscrletter} für Briefe im \CD, 
  \url{https://komascript.de/node/2200}%
}[v2.07]
Es soll das Paket \Package*{tudscrletter} für Briefe im \TUDCD entstehen. 
Auch Klassen für Fax und Hausmitteilungen sollen dabei abfallen.
%\end{DeclareEntity}

%\chapter[%
%  Das Paket \Package{tudscrbeamer} -- Präsentationen im \CD%
%]{%
%  Präsentationen im \CD%
%  \tudmarkuplabel{\Package{tudscrbeamer}}%
%}
%\begin{DeclareEntity}{\Package{tudscrbeamer}}
\ToDo[imp]{Paket \Package*{tudscrbeamer} für Präsentationen im \CD}[v2.08]
Mit \Package*{tudscrbeamer} soll ein Paket entstehen, mit dem sich
\hologo{LaTeX}"~Beamer"=Präsentationen im Stil des \TUDCDs mit den 
Einstellungsmöglichkeiten von \TUDScript erstellen lassen. Dies ist jedoch als 
sehr langfristiges Projekt anzusehen und eine Umsetzung momentan nicht 
absehbar. Aktuell können hierfür die Pakete aus dem \GitHubRepo(tud-cd/tud-cd) 
genutzt werden.
%\end{DeclareEntity}

%\chapter[%
%  Das Paket \Package{tudscrlayout} -- Seitenstil und Satzspiegel im \CD%
%]{%
%  Seitenstil und Satzspiegel im \CD%
%  \tudmarkuplabel{\Package{tudscrlayout}}%
%}
%\begin{DeclareEntity}{\Package{tudscrlayout}}
\ToDo[imp]{%
  Paket \Package*{tudscrlayout} \GitHubRepo(tud-cd/tud-cd)<6> für Satzspiegel, 
  evtl. auch für die Klasse über \Option*{cdgeometry=forced}
}[v2.09]
Außerdem ist ein Paket \Package*{tudscrlayout} vorstellbar, welches den 
durch das \CD vorgegebene Satzspiegel aktiviert ohne den Seitenstil selbst zu 
verwenden, um beispielsweise bereits mit dem Kopf der \TnUD bedrucktes Papier 
nutzen zu können. Ebenfalls wäre es denkbar, für andere Klassen die 
\PageStyle{tudheadings}"=Seitenstile verfügbar zu machen ohne dabei den 
Satzspiegel des \CDs umzusetzen. Dies hat jedoch nur geringe Priorität.
%\end{DeclareEntity}




\setpartpreamble{%
  \begin{abstract}
    \hypersetup{linkcolor=red}
    \noindent Für die Verwendung des \TUDScript-Bundles ist es nicht notwendig,
    diesen Teil zu lesen. In \autoref{sec:exmpl} werden insbesondere für 
    \hologo{LaTeX}"=Neulinge sowie neue Anwender des \TUDScript-Bundles 
    mehrere einfache Beispiele gezeigt. Zusätzlich sind in \autoref{sec:tut} 
    umfangreichere Tutorials für das Erstellen von wissenschaftlichen Arbeiten 
    zu finden. Die darin gegebenen Empfehlungen sind nicht auf \TUDScript 
    beschränkt sondern lassen sich auch mit anderen \hologo{LaTeX}-Klassen 
    umsetzen. In \autoref{sec:packages} werden Einsteigern~-- und versierten 
    \hologo{LaTeX}-Nutzern~-- meiner Meinung nach empfehlenswerte Pakete kurz 
    vorgestellt.
    
    Anwendungshinweise sowie der eine oder andere allgemeine Hinweis bei der 
    Verwendung von \hologo{LaTeX} wird in \autoref{sec:tips} gegeben. Dabei 
    sind diese durchaus für die Verwendung sowohl des \TUDScript-Bundles als 
    auch anderer \hologo{LaTeX}-Klassen interessant. Für Anregungen, Hinweise, 
    Ratschläge oder Empfehlungen zu weiteren Pakete sowie Tipps bin ich 
    jederzeit empfänglich.
  \end{abstract}
}
\part{Ergänzungen und Hinweise\label{part:additional}}
\setchapterpreamble{%
  \begin{abstract}
    Dieses \autorefname soll sowohl den Einstieg als auch die ersten Schritte 
    mit \TUDScript erleichtern. Dafür werden einige Minimalbeispiele gegeben, 
    die einzelne Funktionalitäten darstellen. Diese sind vermeintlich stark
    reduziert ausgeführt, so dass sie sich dem Anwender direkt aus der Syntax 
    erschließen können.
  \end{abstract}
}
\chapter{%
  Minimalbeispiele%
  \label{sec:exmpl}%
  \index{Minimalbeispiel|!(}%
}
\section{%
  Dokument%
  \index{Minimalbeispiel!Dokument}%
}

Die Präambel eines minimalen \Lettering{LaTeX}"=Dokumentes könnte wie hier 
gezeigt gestaltet werden. Dieser Ausschnitt kann prinzipiell als Grundlage für 
ein neu zu erstellendes Dokument verwendet werden. Lediglich das Einbinden des 
Paketes \Package{blindtext} mit \Macro*{usepackage|\MPValue{blindtext}} und der 
daraus stammenden Befehl \Macro*{blinddocument} können weggelassen werden.
\IncludeExample[4]{document}



\section{%
  Dissertation%
  \label{sec:exmpl:dissertation}%
  \index{Minimalbeispiel!Dissertation}%
}

Eine Abschlussarbeit oder ähnliches könnte wie hier gezeigt begonnen werden.
\IncludeExample[2]{dissertation}



\section{%
  Abschlussarbeit (kollaborativ)%
  \label{sec:exmpl:thesis}%
  \index{Minimalbeispiel!Abschlussarbeit}%
  \index{Minimalbeispiel!Kollaboratives~Schreiben}%
}

Zusätzlichen Angaben für beide Autoren gleichermaßen, können außerhalb des 
Argumentes von \Macro(\Bundle{tudscr}){author} angegeben werden.%
\footnote{In diesem Beispiel \Macro{matriculationyear}}
Die Angaben innerhalb des Argumentes von \Macro(\Bundle{tudscr}){author} 
werden den jeweiligen, mit \Macro|?|{and} getrennten Autoren zugeordnet.%
\footnote{%
  In diesem Beispiel \Macro{matriculationnumber}, \Macro{dateofbirth} und 
  \Macro{placeofbirth}%
}
Ohne die Verwendung von \Macro|?|{and} kann natürlich auch nur ein Autor 
aufgeführt werden. Außerdem sei auf die Verwendung von \Macro|?|{subject} 
anstatt \Macro|?|{thesis} mit einem speziellen Wert aus \autoref{tab:thesis} 
hingewiesen.
\IncludeExample{thesis}



\section{%
  Aufgabenstellung (kollaborativ)%
  \label{sec:exmpl:task}%
  \index{Minimalbeispiel!Aufgabenstellung}%
  \index{Minimalbeispiel!Kollaboratives~Schreiben}%
}

Eine Aufgabenstellung für eine wissenschaftliche Arbeit ist mithilfe der 
Umgebung \Environment|?|{task} oder dem Befehl \Macro|?|{taskform} aus dem 
Paket \Package{tudscrsupervisor} folgendermaßen dargestellt werden.
\IncludeExample{task}



\section{%
  Gutachten%
  \label{sec:exmpl:evaluation}%
  \index{Minimalbeispiel!Gutachten}%
}

Nach dem Laden des Paketes \Package{tudscrsupervisor} kann ein Gutachten für 
eine wissenschaftliche Arbeit mit der \Environment|?|{evaluation}"~Umgebung 
oder dem Befehl \Macro|?|{evaluationform} erstellt werden.
\IncludeExample{evaluation}



\section{%
  Aushang%
  \label{sec:exmpl:notice}%
  \index{Minimalbeispiel!Aushang}%
}

Das Paket \Package{tudscrsupervisor} stellt die Umgebung \Environment|?|{notice}
für das Anfertigen allgemeiner Aushänge sowie den Befehl \Macro|?|{noticeform} 
für die Ausschreibung wissenschaftlicher Arbeiten bereit.
\IncludeExample{notice}



\section{%
  Poster%
  \label{sec:exmpl:poster}%
  \index{Minimalbeispiel!Poster}%
}

Mit der Klasse \Class|?|{tudscrposter} lässt sich ein Poster im \TUDCD 
erstellen. Dabei ist die Angabe des gewünschten Papierformates sowie der 
passenden Schriftgröße zu beachten.
\IncludeExample{poster}
\index{Minimalbeispiel|!)}%




\setchapterpreamble{%
  \begin{abstract}
    In diesem \autorefname werden weiterführende Anwendungsbeispiele 
    bereitgestellt. Diese Tutorials sind nicht unmittelbar im Handbuch 
    enthalten sondern werden als externe Dateien bereitgehalten, welche direkt 
    via Hyperlink geöffnet werden können.
  \end{abstract}
}
\chapter{%
  Tutorials%
  \label{sec:tut}%
  \index{Tutorials|!(}%
}
\section{%
  Leitfaden für eine wissenschaftlichen Arbeit%
  \entitylabel{\Tutorial{treatise}}%
}

Die meisten Anwender der \TUDScript-Klassen sind Studenten oder angehörige der 
\TnUD, die ihre ersten Schritte mit \Lettering{LaTeX} beim Verfassen einer 
wissenschaftlichen Arbeit oder ähnlichem machen. Während der Einstiegsphase in 
\Lettering{LaTeX} kann ein Anfänger sehr schnell aufgrund der großen Anzahl an 
empfohlenen Pakete sowie der teilweise diametral zueinander stehenden Hinweise 
überfordert sein. Mit dem Tutorial \TUDScriptTutorial{treatise} soll versucht 
werden, ein wenig Licht ins Dunkel zu bringen. Es erhebt jedoch keinerlei 
Anspruch, vollständig oder perfekt zu sein. Einige der darin vorgestellten 
Möglichkeiten lassen sich mit Sicherheit auch anders, einfacher und/oder 
besser lösen. Dennoch ist es gerade für Neulinge~-- vielleicht auch für den 
einen oder anderen \Lettering{LaTeX}"~Veteran~-- als Leitfaden bezüglich der 
Erstellung einer wissenschaftlichen Arbeit gedacht.



\section{%
  Ein Beitrag zum mathematischen Satz mit \Lettering{LaTeX}%
  \entitylabel{\Tutorial{mathtype}}%
  \index{Mathematiksatz}%
}

Das Tutorial \TUDScriptTutorial{mathtype} richtet sich an alle Anwender, die in 
ihrem \Lettering{LaTeX}-Dokument mathematische Formeln setzen wollen. In diesem 
wird ausführlich darauf eingegangen, wie mit wenigen Handgriffen ein 
typografisch sauberer Mathematiksatz zu bewerkstelligen ist.



\section{%
  Änderung der Trennzeichen im Mathematikmodus%
  \entitylabel{\Tutorial{mathswap}}%
  \index{Mathematiksatz}%
  \index{Zifferngruppierung}%
}

Sollen beim Verfassen eines \Lettering{LaTeX}"=Dokumentes Daten in einem 
Zahlenformat importiert werden, welches nicht den Gepflogenheiten der 
Dokumentsprache entspricht, kommt es meist zu unschönen Ergebnissen bei der 
Ausgabe. Einfachstes Beispiel sind Daten, in denen als Dezimaltrennzeichen ein 
Punkt verwendet wird, wie es im englischsprachigen Raum der Fall ist. Sollen 
diese in einem Dokument deutscher Sprache eingebunden werden, müssten diese 
normalerweise allesamt angepasst und das ursprüngliche Dezimaltrennzeichen 
durch ein Komma ersetzt werden. Dieser Schritt wird mit dem \TUDScript-Paket 
\Package{mathswap} automatisiert. Wie dies genau funktioniert, wird im Tutorial 
\TUDScriptTutorial{mathswap} erläutert.%
\index{Tutorials|!)}%

\chapter{%
  Benötigte, unterstützte und empfehlenswerte Pakete%
  \label{sec:packages}%
  \index{Pakete|?}%
}
\section{%
  Notwendige und ergänzende Pakete für \TUDScript%
  \label{sec:packages:needed}%
}
%
Die im Folgenden aufgezählten Pakete werden von den \TUDScript-Klassen zwingend 
benötigt und spätestens zum Ende der Präambel geladen. Möchten Sie eines der 
Pakete selber nutzen, kann dieses wie gewohnt mit
\Macro{usepackage|\OPName{Paketoptionen}\MPName{Paket}} eingebunden werden. 
Alternativ lassen sich gewünschte Optionen~-- wie in \autoref{sec:tips:options} 
beschrieben~-- einfach bereits \emph{vor} dem Laden der Dokumentklasse an das 
entsprechende Paket übergeben, falls bei dessen nachfolgender Beschreibung 
nichts anderweitiges angegeben ist.

\begin{DeclarePackages}
\itempkg{}(\Bundle{koma-script})<%
    \Class*{koma-class},\Class{scrbook},\Class{scrreprt},\Class{scrartcl},%
    \Package{typearea},\Package{scrlayer-scrpage},\Package{scrletter},%
    \Package{scrbase},\Package{scrextend}%
  >%
  Die zentrale Grundlage für \TUDScript bilden~-- zusammenfassend in diesem 
  Dokument gegebenenfalls als \Class||{koma-class} bezeichnet~-- die drei 
  Hauptklassen \Class{scrbook}, \Class{scrreprt} sowie \Class{scrartcl} aus dem 
  \scrguide[\KOMAScript-Bundle]. Das Paket \Package{scrbase} erlaubt die 
  Definition von Optionen oder Schlüsseln, die sich auch noch nach dem Laden 
  einer Klasse oder eines Paketes aus dem \TUDScript-Bundle mit den Befehlen 
  \Macro{TUDoption} und \Macro{TUDoptions} ändern lassen. Weiterhin werden die 
  \PageStyle{tudheadings}"=Seitenstile der \TUDScript-Klassen mithilfe des 
  Paketes \Package{scrlayer-scrpage} bereitgestellt. Wenn dieses nicht~-- mit 
  beliebigen Optionen~-- durch den Anwender geladen wird, erfolgt dies 
  automatisch am Ende der Präambel. Für die Festlegung des Satzspiegels wird
  das Paket \Package{typearea} genutzt.
\itempkg{opensans,mathastext,iwona}[v2.06]
  \index{Schriftart|!}%
  Das Paket \Package{opensans} stellt die Schriftfamilie \OpenSans sowohl für 
  den Fließtext als auch den mathematischen Satz zur Verfügung. Es enthält alle 
  nötigen Schriftschnitte sowohl im Type1- als auch im OpenType-Format. Da die 
  Schriftfamilie in der aktuellen Version keine mathematischen Glyphen 
  bereitstellt, werden die Pakete \Package{mathastext} und \Package{iwona} 
  zusätzlich genutzt, um zumindest einen halbwegs erträglichen mathematischen 
  Satz mit \OpenSans zu ermöglichen.
%  Werden dabei zusätzliche Symbole benötigt, wird empfohlen, auf das Paket 
%  \Package{amssymb} zu verzichten und anstelle dessen \Package{mdsymbol} zu 
%  laden.
  \ToDo{roboto-mono, hinweis zu mdsymbol raus}[v2.07]
  \ToDo{mweights raus, Text in Hinweise verschieben?!}[v2.07]
%\itempkg{mweights}
%  \index{Schriftstärke}%
%  In \hologo{LaTeX} existieren die Schriftfamilien für Serifenschriften 
%  (\Macro{rmfamily}), serifenlose Schriften (\Macro{sffamily}) sowie die 
%  Schreibmaschinenschriften (\Macro{ttfamily}). Deren Schriftstärke wird für 
%  gewöhnlich mit den Befehlen \Macro{mddefault} und \Macro{bfdefault} 
%  einheitlich festgelegt. Bei der Verwendung unterschiedlicher Schriftpakete 
%  kann es unter Umständen zu Problemen bei den Schriftstärken kommen. Diese 
%  Paket erlaubt die individuelle Definition der Schriftstärke für jede der 
%  drei Schriftfamilien.
\itempkg{geometry}
  \index{Satzspiegel}%
  Das Paket wird zum Festlegen der Seitenränder respektive des Satzspiegels 
  verwendet.
  \Attention{%
    Ein Weiterreichen zusätzlicher Optionen an das Paket wird dringlich nicht 
    empfohlen.
  }%
\itempkg{graphicx}
  \index{Grafiken}%
  Zum Einbinden des Logos der \TnUD im Kopf sowie aller weitere Abbildungen und 
  Logos wird \Macro{includegraphics} genutzt.
\itempkg{xcolor}
  \index{Farben}%
  Damit werden die Farben des \CDs zur Verwendung im Dokument definiert. 
  Genaueres ist bei der Beschreibung von \Package'ref'{tudscrcolor} zu finden. 
\itempkg{iftex,etoolbox,xpatch,letltxmacro}
  Diese Pakete stellen viele Funktionen zum Testen und zur Ablaufkontrolle 
  bereit. Weiterhin wird das Manipulieren vorhandener Makros ermöglicht.
  \ToDo{kvsetkeys, environ, nicht mehr verwenden, Doku raus!}[v2.07]
  \ToDo{text von environ to xparse?}[v2.07]
%\itempkg{kvsetkeys}
%  Das von \Package{scrbase} geladene Paket \Package{keyval} macht das 
%  Definieren von Klassen- und Paketoptionen sowie Parametern nach dem 
%  Schlüssel"=Wert"=Prinzip möglich. Mit diesem Paket kann das Verhalten für 
%  unbekannte Schlüssel festgelegt werden.
%\itempkg{environ}
%  \index{Befehlsdeklaration}%
%  Es wird eine verbesserte Deklaration von Umgebungen ermöglicht, bei der auch 
%  beim Abschluss der Umgebung auf die übergebenen Parameter zugegriffen werden 
%  kann. 
\itempkg{trimspaces}
  Bei mehreren Eingabefeldern für den Anwender werden die Argumente mithilfe 
  dieses Paketes um eventuell angegebene, unnötige Leerzeichen befreit.
\end{DeclarePackages}

\minisec{%
  Durch \TUDScript direkt unterstütze Pakete%
  \index{Kompatibilität!Pakete}%
}
Einige dernachfolgen beschriebenen Pakete werden durch \TUDScript direkt 
unterstützt und erweitern dessen Funktionalität. Dies sind namentlich 
\Package{hyperref}, \Package{multicol}, \Package{quoting}, \Package{ragged2e} 
und \Package{crop} sowie \Package{isodate} respektive \Package{datetime2}. 
Weitere Informationen dazu ist den nachfolgenden Beschreibung des jeweiligen 
Paketes zu entnehmen.



\section{%
  Empfehlenswerte Pakete%
  \label{sec:packages:recommended}%
}
%
In diesem \autorefname wird eine Vielzahl an Paketen~-- zumeist kurz~-- 
vorgestellt, welche sich für mich persönlich bei der Arbeit mit \hologo{LaTeX} 
bewährt haben. Einige davon werden außerdem im Tutorial \Tutorial{treatise} in 
ihrer Anwendung beschrieben. Für detaillierte Informationen sowie bei Fragen zu 
den einzelnen Paketen sollte die jeweilige Dokumentation%
\footnote{Kommandozeile/Terminal: \Path{texdoc\,\PName{Paketname}}}
zu Rate gezogen werden, das Lesen der hier gegebenen Kurzbeschreibung ersetzt 
dies in keinem Fall.



\subsection{Pakete zur Verwendung in jedem Dokument}
%
Die hier vorgestellten Pakete gehören meiner Meinung nach in die Präambel eines 
jeden Dokumentes. Die Dokumentsprache sollte in jedem Fall mit \Package{babel} 
oder \Package{polyglossia} definiert werden~-- auch wenn dies Englisch ist. Für 
deutschsprachige Dokumente ist für eine annehmbare Worttrennung beim Einsatz 
von \Format{pdfLaTeX} das Paket \Package{hyphsubst} unbedingt zu verwenden.

\begin{DeclarePackages}
\itempkg{fontenc,fontspec}[v2.02]
  \index{Zeichensatzkodierung}%
  \ChangedAt*{%
    v2.02:OpenType-Schriften mit Paket \Package{fontspec} verwendbar;%
    v2.04:Unterstützung des Paketes \Package{fontspec} verbessert;%
  }%
  Die Zeichensatzkodierung des Ausgabefonts sollte immer festgelegt werden. Für 
  \Format{pdfLaTeX} ist die Ausgabe als 7"~bit"~kodierte Schrift in der 
  Voreinstellung gewählt, was unter anderem dazu führt, dass keine echten
  Umlaute im erzeugten PDF"~Dokument verwendet werden. Um auf 8"~bit"~Schriften
  zu schalten, ist \Macro||{usepackage|\OPValue{T1}\MPValue{fontenc}} zu
  nutzen.
  
  Für die Unicode"=Textsatzsysteme \Format{LuaLaTeX} oder \Format{XeLaTeX} 
  sollte stattdessen das Paket \Package{fontspec} verwendet werden. Damit 
  können Systemschriften im OpenType-Format und einer beliebigen 
  Zeichensatzkodierung eingebunden werden, womit sich die Auswahl der 
  verwendbaren Schriften stark erweitert. Das Paket wird durch \TUDScript 
  unterstützt.
\itempkg{microtype}
  \index{Typografie}%
  Dieses Paket kümmert sich um den optischen Randausgleich%
  \footnote{englisch: protrusion, margin kerning} und das Nivellieren der 
  Wortzwischenräume\footnote{englisch: font expansion} im Dokument. Es 
  funktioniert mit sowohl \Format||{pdfTeX} als auch \Format||{LuaTeX} sowie 
  \Format||{XeTeX} jedoch nicht mit der klassischen \Format||{TeX}"~Engine.
  \ToDo{Hinweis zu Optionen (final, spacing, kerning, tracking?)}
\itempkg{babel,polyglossia}
  \index{Sprachunterstützung}%
  \index{Bezeichner}%
  Mit dem Paket \Package{babel} erfolgt die Einstellung der im Dokument 
  verwendeten Sprache(n). Bei mehreren angegebenen Sprachen ist die zuletzt 
  geladene die Hauptsprache des Dokumentes. Die gewünschten Sprachen sollten 
  nicht als Paketoption sondern als Klassenoption und gesetzt werden, damit 
  auch andere Pakete auf die Spracheinstellungen zugreifen können. Für 
  deutschsprachige Dokumente sind die beiden Optionen \Option*{ngerman} für die 
  neue respektive \Option*{german} für die alte deutsche Rechtschreibung zu 
  verwenden. 
  
  Mit dem Laden von \Package{babel} und der dazugehörigen Sprachen werden 
  sowohl die Trennungsmuster als auch die sprachabhängigen Bezeichner angepasst.
  Von einer Verwendung der obsoleten Pakete \Package*{german} beziehungsweise 
  \Package*{ngerman} anstelle von \Package{babel} wird abgeraten. Für 
  \Format{LuaLaTeX} und \Format{XeLaTeX} kann das Paket \Package{polyglossia} 
  genutzt werden.
\itempkg{hyphsubst,dehyph-exptl}
  \index{Worttrennung|!}%
  Die möglichen Trennstellen von Wörtern wird von \hologo{LaTeX} mithilfe 
  eines Algorithmus berechnet. Dieser wird für deutschsprachige Texte mit dem 
  Paket \Package{hyphsubst} entscheidend verbessert. \Format{LuaLaTeX} und 
  \Format{XeLaTeX} nutzen diese besseren Trennungsmuster automatisch, für 
  \Format{pdfLaTeX} müssen diese mit folgendermaßen eingebunden werden:
  \begin{Code}
    \usepackage[ngerman=ngerman-x-latest]{hyphsubst}
  \end{Code}\vspace{-\baselineskip}%
  In \autoref{sec:tips:hyphenation} wird genauer auf das Zusammenspiel von 
  \Package{hyphsubst} und \Package{babel} sowie \Package{fontenc} eingegangen, 
  ein Blick dahin wird dringend empfohlen. Zusätzlich werden dort weitere 
  Hinweise für eine verbesserte Worttrennung gegeben.
\end{DeclarePackages}



\subsection{Pakete zur situativen Verwendung}
%
Die nachfolgenden Pakete sollten nicht zwangsweise in jedem Dokument geladen 
werden sondern nur, falls dies auch tatsächlich notwendig ist. Zur besseren 
Übersicht wurde versucht, diese thematisch passend zu gruppieren. Daraus lässt 
sich keinerlei Wertung bezüglich ihrer Nützlichkeit oder meiner persönlichen 
Wertschätzung ableiten.



\subsubsection{%
  Typografie und Layout%
  \index{Typografie|(}%
}
%
Neben dem zuvor beschriebenem Paket \Package{microtype}, welches verantwortlich 
für mikrotypografische Feinheiten ist, existieren weitere Pakete, die vorrangig 
die Makrotypografie adressieren.
%
\begin{DeclarePackages}
\itempkg{setspace}
  \index{Zeilenabstand}%
  \index{Durchschuss}%
  Die Vergrößerung des Zeilenabstandes wird:
  \begin{enumerate}[itemindent=0pt,labelwidth=*,labelsep=1em,label=\Roman*.]
  \item viel zu häufig und völlig unnötig gefordert und
  \item schließlich auch noch zu groß gewählt.
  \end{enumerate}
  Die Forderung nach Erhöhung des Zeilenabstandes~-- in der Typografie als 
  Durchschuss bezeichnet~-- kommt aus den Zeiten der Textverarbeitung mit der 
  Schreibmaschine. Ein einzeiliger Zeilenabstand bedeutete hier, dass die 
  Unterlängen der oberen Zeile genau auf der Höhe der Oberlängen der folgenden 
  Zeile lagen. Ein anderthalbzeiliger Zeilenabstand erzielte hier somit einen 
  akzeptablen Durchschuss. Eine Erhöhung des Durchschusses bei der Verwendung 
  von \hologo{LaTeX} ist an und für sich nicht notwendig. Sinnvoll ist dies 
  nur, wenn im Fließtext serifenlose Schriften zum Einsatz kommen, um die damit 
  verbundene schlechte Lesbarkeit etwas zu verbessern.
  
  Ist die Erhöhung des Durchschusses wirklich notwendig, sollte das Paket 
  \Package{setspace} genutzt werden. Dieses stellt den Befehl 
  \Macro{setstretch|\MPName{Faktor}} zur Verfügung, mit dem der Durchschuss 
  respektive Zeilenabstand angepasst werden kann. Der Wert des Faktors ist 
  standardmäßig auf~1 gestellt und sollte maximal bis~1.25 vergrößert werden. 
  Der Befehl \Macro{onehalfspacing} aus diesem Paket setzt diesen Wert auf 
  eben genau~1.25. Allerdings ist hier anzumerken, dass die Vergrößerung des 
  Zeilenabstandes~-- so wie ich es mir angelesen habe~-- aus der Sicht eines 
  Typographen keine Spielerei ist sondern vielmehr allein der Lesbarkeit des 
  Textes dient und möglichst gering ausfallen sollte.
  
  Ziel ist es, beim Lesen nach dem Beenden einer Zeile das Auffinden der neuen 
  Zeile zu vereinfachen. Bei Serifen ist dies durch die Betonung der Grundlinie 
  sehr gut möglich. Bei serifenlosen Schriften~-- wie der im \TUDCD verwendeten 
  \OpenSans~-- ist dies schwieriger und ein erweiterter Abstand der Zeilen kann 
  hierbei hilfreich sein. Jedoch sollte nicht nach dem Motto 
  \enquote{viel hilft viel} verfahren werden. Für dieses Dokument wurde 
  \Macro{setstretch|\MPValue{\baselinestretch}} für den Zeilenabstand gewählt. 
  Weitere Tipps sind in \autoref{sec:tips:headings} und   
  \autoref{sec:tips:headline} zu finden.
\itempkg{multicol}
  \index{Satzspiegel!zweispaltig|?}%
  \index{Mehrspaltensatz}%
  \index{Satzspiegel!mehrspaltig}%
  Hiermit kann jeglicher beliebiger Inhalt in zwei oder mehr Spalten ausgegeben 
  werden, wobei~-- im Gegensatz zum normalen zweispaltigen Satz über die
  \KOMAScript-Option \Option{twocolumn}~-- für beliebig viele Spalten ein 
  automatischer Ausgleich dieser erfolgt. Es wird innerhalb der Umgebungen 
  \Environment{abstract} und \Environment{tudpage} unterstützt, sofern es 
  geladen wird.
\itempkg{isodate,datetime2}
  \index{Datum|?}%
  \ToDo{Vorrang für datetime2?!}
  Mit \Macro{printdate|\MPName{Datum}} formatiert das Paket \Package{isodate} 
  die Ausgabe eines Datums automatisch in ein spezifiziertes Format. Die 
  Datumsangabe kann dabei im deutschen, englischen oder ISO"~Format erfolgen. 
  Alternativ kann auch das Paket \Package{datetime2} genutzt werden, welches 
  für \Macro{DTMDate|\MPName{ISO-Datum}} die Angabe im ISO"~Format erfordert. 
  Wird eines der Pakete geladen, werden alle Datumsfelder, welche durch die 
  \TUDScript-Klassen definiert wurden,%
  \footnote{%
    \Macro{date}, \Macro{dateofbirth} und \Macro{defensedate} sowie aus 
    \Package{tudscrsupervisor} \Macro{duedate} und \Macro{issuedate}%
  }
  im durch das jeweilige Paket definierten Ausgabeformat ausgegeben.
\itempkg{quoting}
  \index{Zitate}%
  \hologo{LaTeX} bietet von Haus aus \emph{zwei} verschiedene Umgebungen~-- 
  \Environment{quote} und \Environment{quotation}~-- für Zitate und ähnliches 
  an. Allerdings werden durch beide Umgebungen die \KOMAScript-Option 
  \Option{parskip=\PMisc} ignoriert. Mit der Umgebung \Environment{quoting} aus 
  dem gleichnamigen Paket lässt sich dieses Problem umgehen. Wird das Paket 
  geladen, wird diese innerhalb der \Environment{abstract}"~Umgebung verwendet.
\itempkg{csquotes}
  \index{Zitate}%
  Das Paket stellt unter anderem den Befehl \Macro{enquote|\MPName{Zitat}} zur 
  Verfügung, welcher Anführungszeichen in Abhängigkeit der gewählten Sprache 
  setzt. Zusätzlich werden weitere Kommandos und Optionen für die spezifischen 
  Anforderungen des Zitierens bei wissenschaftlichen Arbeiten angeboten. 
  Außerdem wird es durch \Package{biblatex} unterstützt und sollte zumindest 
  bei dessen Verwendung geladen werden.
  \ToDo{request für AfterEndEnvironment? geht das mit xparse?}[v2.07]
\itempkg{noindentafter}[v2.02]
  Mit diesem Paket lassen sich automatische Absatzeinzüge für selbst zu 
  bestimmende Befehle und Umgebungen unterdrücken.
\itempkg{ragged2e}
  \index{Worttrennung}%
  Das Paket verbessert den Flattersatz, indem für diesen die Worttrennung 
  aktiviert wird.
\itempkg{fnpct}[v2.05]
  \index{Fußnoten|?}%
  Diese Paket sorgt zum einen für das Einhalten der richtigen Reihenfolge von 
  Satzzeichen und Fußnoten und zum anderen wird das typografisch korrekte 
  Setzen mehrerer, nacheinander folgender Fußnoten unterstützt. 
\itempkg{xspace,xpunctuate}
  \index{Befehlsdeklaration}%
  Mit \Package{xspace} kann bei der Definition eigener Makros der Befehl 
  \Macro{xspace} genutzt werden. Dieser setzt ein gegebenenfalls notwendiges 
  Leerzeichen automatisch. In \autoref{sec:tips:xspace} ist die Definition 
  eines solchen Befehls exemplarisch ausgeführt. Durch das Paket 
  \Package{xpunctuate} wird \Package{xspace} um die Beachtung von 
  Interpunktionen erweitert.
\itempkg{ellipsis}
  \index{Befehlsdeklaration}%
  In \hologo{LaTeX} folgen den Befehlen für Auslassungspunkte (\Macro{dots} und 
  \Macro{textellipsis}) \emph{immer} ein Leerzeichen. Dies kann unter Umständen 
  unerwünscht sein. Mit dem Paket \Package{ellipsis} wird das nachfolgende 
  Leerzeichen~-- im Gegensatz zum Standardverhalten~-- nur gesetzt, wenn ein 
  Satzzeichen und kein Buchstabe folgt. Zusätzliche Hinweise zur Verwendung 
  sind in \autoref{sec:tips:dots} zu finden.
\itempkg{selnolig}(\Application'CTAN:delig'{DeLig|})
  \index{Ligaturen}%
  Wird eine Schriftart mit Ligaturen verwendet, sollten diese für einen guten 
  Textsatz bei bestimmten Wörtern~-- insbesondere in deutschen Texten aufgrund 
  der vielen Komposita~-- aufgelöst werden.%
  \footnote{%
    Das sind \enquote{ff}, \enquote{fi}, \enquote{fl}, \enquote{ffi}, und 
    \enquote{ffl} bei den meisten \hologo{LaTeX}"~Schriften.%
  }
  Mit \Format{LuaLaTeX} als Textsatzsystem kann das Paket \Package{selnolig} 
  für dieses Unterfangen verwendet werden, welches sowohl Muster für das 
  Auflösen von Ligaturen als auch Trennstellen für Wörter bereitstellt. 
  Das Java"~Programm \Application{DeLig} trennt anhand eines Wörterbuches 
  problematische Ligaturen alternativ dazu durch direktes Einfügen von 
  \PValue{"\textbar} in die \Path{.tex}"~Datei selbst, wofür das Paket 
  \Package{babel} verwendet werden muss.
\end{DeclarePackages}
\index{Typografie|)}%



\subsubsection{%
  Rechtschreibung%
  \index{Rechtschreibung|(}%
}
%
Für die Rechtschreibkontrolle zeichnet im Normalfall der verwendete Editor 
verantwortlich. Dennoch gibt es einige wenige Pakete, welche sich diesem Thema 
widmen. Diese sind jedoch ausschließlich nutzbar, wenn als Textsatzsystem 
\Format{LuaLaTeX} genutzt wird.
\begin{DeclarePackages}
\itempkg{lua-check-hyphen}
  \index{Worttrennung}%
  Hiermit lassen sich mit \Format{LuaLaTeX} Trennstellen am Zeilenende zur 
  Prüfung markieren. Zum Thema der \textit{korrekten Worttrennung} sei außerdem 
  auf \autoref{sec:tips:hyphenation} verwiesen.
\itempkg{spelling}
  Wird \Format{LuaLaTeX} als Textsatzsystem verwendet, wird mit diesem Paket 
  der reine Textanteil aus dem \hologo{LaTeX}"=Dokument extrahiert~-- wobei 
  Makros und aktive Zeichen entfernt werden~-- und in eine separate Textdatei 
  geschrieben. Anschließend kann diese Datei mit einer externen Software zur  
  Rechtschreibprüfung wie \Application{GNU~Aspell}, \Application{Hunspell} oder 
  \Application{LanguageTool} analysiert und falsch geschriebene Wörter im 
  PDF"~Dokument hervorgehoben werden.
\end{DeclarePackages}
\index{Rechtschreibung|)}%



\subsubsection{Schriften und Sonderzeichen}
%
\begin{DeclarePackages}
\itempkg{lmodern,cfr-lm}
  \index{Schriftart}%
  Soll mit den klassischen \hologo{LaTeX}"=Standardschriften gearbeitet werden, 
  empfiehlt sich die Verwendung des Paketes \Package{lmodern}. Dieses 
  verbessert die Darstellung der Computer~Modern sowohl am Bildschirm als auch 
  beim finalen Druck. Das experimentelle Paket \Package{cfr-lm} liefert 
  zusätzliche Schriftschnitte für die Latin~Modern~Schriftfamilie.
\itempkg{newtxtext,newtxmath}
  \index{Schriftart}%
  \index{Mathematiksatz}%
  Es werden einige alternative Schriften sowohl für den Fließtext 
  (\textit{Times} und \textit{Helvetica}) als auch den Mathematikmodus 
  bereitgestellt.
\itempkg{libertine,libgreek}
  \index{Schriftart}%
  Das Paket stellt die Schriften Linux~Libertine und Linux~Biolinum zur 
  Verfügung. Um diese Schriftart auch für den Mathematikmodus verwenden zu 
  können, sollte zusätzlich das Paket \Package{newtxmath} mit 
  \Macro||{usepackage|\OPValue{libertine}\MPValue{newtxmath}} in der 
  Präambel eingebunden werden. Das Paket \Package{libgreek} enthält griechische 
  Lettern für Linux~Libertine.
\itempkg{relsize}
  \index{Schriftgröße}%
  Die Größe einer Textauszeichnung kann relativ zur aktuellen Schriftgröße 
  gesetzt werden.
\itempkg{textcomp}
  \index{Sonderzeichen}%
  Es werden zusätzliche Symbole und Sonderzeichen wie beispielsweise das 
  Promille- oder Eurozeichen sowie Pfeile für den Fließtext zur Verfügung 
  gestellt.
\end{DeclarePackages}
%
\index{Mathematiksatz|(}%
Wird für den mathematischen Satz \OpenSans genutzt (\Option{cdmath=true}), wird 
vordefiniert das Paket \Package{mathastext} geladen und es sollten keine 
zusätzlichen Einstellungen vorgenommen werden. Wird darauf verzichtet 
(\Option{cdmath=false}), können serifenlose Mathematikschriften mit folgenden 
Paketen aktiviert werden.
%
\begin{DeclarePackages}
\itempkg{sansmathfonts,sansmath}
  Sollten die normalen \hologo{LaTeX}"~Schriften Computer~Modern verwendet 
  werden, lässt sich dieses Paket zum serifenlosen Setzen mathematischer 
  Ausdrücke nutzen. Ein alternatives Paket mit der gleichen Zielstellung ist 
  \Package{sansmath}
\itempkg{sfmath}
  Diese Paket verfolgt ein ähnliches Ziel, kann jedoch im Gegensatz zu 
  \Package{sansmath} nicht nur für Computer~Modern sondern mit der 
  entsprechenden Option auch für Latin~Modern, Helvetica und 
  Computer~Modern~Bright verwendet werden.
\end{DeclarePackages}



\subsubsection{Mathematiksatz}
%
Dies sind Pakete, die Umgebungen und Befehle für den Mathematiksatz sowie das 
Setzen von Einheiten und Zahlen im Allgemeinen anbieten.

\begin{DeclarePackages}
\itempkg{mathtools,amsmath}
  Das De"~facto"~Standard"~Paket für einen erweiterten mathematischen Satz ist 
  \Package{amsmath}, welches verschiedene Befehle und Umgebungen für dieses 
  Unterfangen bereitstellt. Das Paket \Package{mathtools} für dessen Befehle 
  und Umgebungen kleinere Bugfixes.
\itempkg{bm}
  Das Paket bietet mit \Macro{bm} eine Alternative zu \Macro{boldsymbol} im 
  \href{http://tex.stackexchange.com/q/3238}{Mathematiksatz}. Sollten bei
  der Nutzung des Paketes Fehler auftreten, sei auf dessen Dokumentation und 
  insbesondere auf das Makro \Macro||{bmmax} verwiesen.
\itempkg{ionumbers}
  \index{Zifferngruppierung}%
  Die korrekte Formatierung von Zahlen ist manchmal ein Ärgernis bei der  
  Verwendung von \hologo{LaTeX}. Insbesondere, wenn in einem deutschsprachigen 
  Dokument Daten im englischsprachigen Format verwendet werden, kommt es zu 
  Problemen. Mit diesem Paket können einfach Eingabe- und Ausgabeformat von 
  Zahlen definiert werden, um beispielsweise externe Datensätze einzulesen und 
  korrekt formatiert auszugeben.
%  Dafür wird im \TUDScript-Bundle das Paket \Package{mathswap} bereitgestellt. 
%  Dennoch gibt es zu diesem auch Alternativen.
\end{DeclarePackages}
%
Für das typografisch korrekte Setzen von Einheiten~-- ein halbes Leerzeichen 
zwischen Zahl und \emph{aufrecht} gesetzter Einheit~-- gibt es zwei gut 
nutzbare Pakete.
%
\begin{DeclarePackages}
\index{Einheiten|(}%
\itempkg{units}
  Dies ist ein einfaches und sehr zweckdienliches Paket zum Setzen von 
  Einheiten und für die meisten Anforderungen völlig ausreichend.
\itempkg{siunitx}
  Dieses Paket ist in seinem Umfang im Vergleich deutlich erweitert. Neben 
  Einheiten können zusätzlich auch Zahlen typografisch korrekt gesetzt werden. 
  Die Ausgabe lässt sich in vielerlei Hinsicht an individuelle Bedürfnisse 
  anpassen. Für deutschsprachige Dokumenten sollte die Lokalisierung angegeben 
  werden. Mehr dazu in \autoref{sec:tips:siunitx}.
\index{Einheiten|)}%
\end{DeclarePackages}
%
Weitere Hinweise zur mathematischen Typografie werden in \autoref{sec:tut} 
gegeben.%
\index{Mathematiksatz|)}%



\subsubsection{%
  Listen%
  \index{Listen|?(}%
}
%
\begin{DeclarePackages}
\itempkg{enumitem}
  Das Paket \Package{enumitem} erweitert die standardmäßig eher rudimentären 
  Funktionalitäten der \hologo{LaTeX}"=Umgebungen für Aufzählungen 
  \Environment{itemize}, \Environment{enumerate} und \Environment{description}.
  Durch die Bereitstellung optionaler Parameter im Schlüssel"=Wert"=Stil wird 
  die individuelle Anpassung dieser sowohl allgemein für alle als auch für 
  einzelne Auflistungen ermöglicht. Eine von mir sehr häufig genutzte Funktion 
  ist beispielsweise die Entfernung des zusätzlichen Abstand zwischen den 
  einzelnen Einträgen einer Liste mit \Macro{setlist|\MPValue{noitemsep}}.
\end{DeclarePackages}
\index{Listen|?)}%



\subsubsection{%
  Verzeichnisse aller Art%
  \index{Verzeichnisse|?(}%
}
%
Neben dem Erstellen des eigentlichen Dokumentes sind für eine wissenschaftliche 
Arbeit meist auch allerhand Verzeichnisse gefordert. Fester Bestandteil ist 
dabei das Literaturverzeichnis, auch ein Abkürzungs- und Formelzeichen- 
beziehungsweise Symbolverzeichnis werden häufig gefordert. Gegebenenfalls wird 
auch noch ein Glossar benötigt. Hier werden die passenden Pakete vorgestellt. 
Sollen im Dokument komplette Quelltexte oder auch nur Auszüge daraus erscheinen 
und für diese auch gleich ein entsprechendes Verzeichnis generiert werden, so 
sei auf das Paket \Package'full'{listings} verwiesen.

\begin{DeclarePackages}
\itempkg{biblatex}
  \index{Literaturverzeichnis}%
  Das Paket kann als legitimer Nachfolger zu \hologo{BibTeX} gesehen werden. 
  Ähnlich dazu bietet \Package{biblatex} die Möglichkeit, Literaturdatenbanken 
  einzubinden und verschiedene Stile der Referenzierung und Darstellung des 
  Literaturverzeichnisses auszuwählen. 
  
  Mit \Package{biblatex} ist die Anpassung eines bestimmten Stiles wesentlich 
  besser umsetzbar als mit \hologo{BibTeX}. Wird \Application{biber} für die 
  Sortierung des Literaturverzeichnisses genutzt, ist die Verwendung einer 
  UTF"~8"~kodierten Literaturdatenbank problemlos möglich. In Verbindung mit 
  \Package{biblatex} wird die zusätzliche Nutzung des Paketes 
  \Package{csquotes} sehr empfohlen.
\itempkg{glossaries,nomencl}
  \index{Glossar}%
  \index{Abkürzungsverzeichnis}%
  \index{Formelzeichenverzeichnis}%
  \index{Symbolverzeichnis}%
  Dies ist ein sehr mächtiges Paket zum Erstellen eines Glossars sowie 
  Abkürzungs- und Symbolverzeichnisses. Die mannigfaltige Anzahl an Optionen 
  ist zu Beginn eventuell etwas abschreckend. Insbesondere wenn Verzeichnisse 
  für Abkürzungen \emph{und} Formelzeichen respektive Symbole notwendig sind, 
  sollte dieses Paket in Erwägung gezogen werden.
  
  Alternativ dazu kann für ein Symbolverzeichnis auch lediglich eine einfache 
  Auflistung mit dem Paket \Package{nomencl} erzeugt werden.
\itempkg{acro,acronym}
  \index{Abkürzungsverzeichnis}%
  Soll lediglich ein Abkürzungsverzeichnis erstellt werden, ist dieses Paket 
  die erste Wahl. Es stellt Befehle zur Definition von Abkürzungen sowie zu 
  deren Verwendung im Text und zur sortierten Ausgabe eines Verzeichnisses 
  bereit. Alternativ dazu kann das Paket \Package{acronym} verwendet werden. 
  Die Sortierung des Abkürzungsverzeichnisses muss hier allerdings manuell 
  durch den Anwender erfolgen.
\end{DeclarePackages}
\index{Verzeichnisse|?)}%



\subsubsection{%
  Gleitobjekte%
  \index{Gleitobjekte|?(}%
}
%
Es werden Pakete für die Beeinflussung von Aussehen, Beschriftung und 
Positionierung von Gleitobjekten vorgestellt. Unter \autoref{sec:tips:floats} 
sind außerdem Hinweise zur manuellen Manipulation der Gleitobjektplatzierung zu 
finden.

\ToDo[doc]{Pakete float und keyfloat erläutern, widows-and-orphans}[v2.07]
%\documentclass[%
%  captions=tableabove
%]
%{scrreprt}
%\usepackage{caption}
%\usepackage{keyfloat}
%%\captionsetup{tableposition=above}
%%\captionsetup[table]{position=above}
%\begin{document}
%\begin{keytable}{c=keytable}
%\centering
%\fbox{this would be some content for a table}
%\end{keytable}
%\end{document}
\begin{DeclarePackages}
\itempkg{caption}
  Mit der Option \InlineDeclaration{\Option{captions=\PMisc}} bieten die 
  \KOMAScript-Klassen bereits einige Möglichkeiten zum Formatieren der 
  Beschriftungen für Gleitobjekte. Sollten diese nicht ausreichen, lässt sich 
  dieses Paket nutzen, wobei in diesem Fall auf mögliche Überschneidungen der 
  Einstellungen geachtet werden sollte.
\itempkg{subcaption}
  Diese Paket kann zum einfachen Setzen von Unterabbildungen oder "~tabellen 
  mit den entsprechenden Beschriftungen genutzt werden. Das häufig angeführte 
  Paket \Package*{subfig} ist keine gute Alternative hierzu, da es nicht mehr 
  gepflegt wird und es im Zusammenspiel mit anderen Paketen des Öfteren zu 
  Problemen kommt. Sollte der Funktionsumfang von \Package{subcaption} nicht 
  ausreichen, kann anstelle dessen das Paket \Package{floatrow} verwendet 
  werden, welches ähnliche Funktionalitäten wie \Package*{subfig} bereitstellt.
\itempkg{floatrow}
  Mit diesem Paket können global wirksame Einstellungen und Formatierungen für 
  \emph{alle} Gleitobjekte eines Dokumentes über die Paketoptionen oder mit 
  \Macro{floatsetup|\MPValue{\dots}} vorgenommen werden. So lässt sich unter 
  anderem die verwendete Schrift innerhalb der Umgebungen \Environment{figure} 
  und \Environment{table} mit \Macro{floatsetup|\MPValue{font=\dots}} 
  einstellen. Mit 
  \begin{Code}
    \floatsetup[figure]{capposition=bottom}
    \floatsetup[table]{capposition=top}
  \end{Code}\vspace{-\baselineskip}%
  lässt sich automatisch das typografisch korrekte Setzen von 
  Abbildungs\emph{unterschriften} sowie Tabellen\emph{überschriften} 
  erzwingen~-- unabhängig von der Position des Befehls zur Beschriftung 
  \Macro{caption} innerhalb der jeweiligen Gleitobjektumgebung. Wird das 
  Verhalten wie empfohlen eingestellt, sollte für eine gute vertikale 
  Platzierung der Tabellenüberschriften zusätzlich die \KOMAScript-Option 
  \Option{captions=tableheading} genutzt werden.
\itempkg{placeins}
  Mit diesem Paket kann die Ausgabe von Gleitobjekten vor Kapiteln und wahlweise
  Unterkapiteln erzwungen werden.
\itempkg{flafter}
  Dieses Paket erlaubt die frühestmögliche Platzierung von Gleitobjekten im 
  ausgegeben Dokument erst an der Stelle ihres Auftretens im Quelltext. Diese 
  werden dementsprechend nie vor ihrer Definition am Anfang der Seite 
  erscheinen.
  \ToDo{Hinweis zu Paket float}
\end{DeclarePackages}
\index{Gleitobjekte|?)}%



\subsubsection{%
  Tabellen%
  \index{Tabellen|?(}%
}
%
Für den Tabellensatz werden standardmäßig die Umgebungen \Environment{tabbing} 
und \Environment{tabular} respektive \Environment{tabular*} bereitgestellt, 
welche in ihrer Funktionalität für einen qualitativ hochwertigen Tabellensatz 
meist nicht ausreichen. Deshalb werden hier zusätzliche Pakete vorgestellt. 
%
\begin{DeclarePackages}
\itempkg{array}
  Dieses Paket ermöglicht mit \Macro{newcolumntype} das Erstellen neuer 
  Spaltentypen und die erweiterte Definition von Tabellenspalten
  (\PValue{>\MPValue{\dots}}\PName{Spaltentyp}\PValue{<\MPValue{\dots}}), 
  wobei mithilfe sogenannter \enquote{Hooks} vor und nach Einträgen innerhalb 
  einer Spalte gezielt Anweisungen gesetzt werden können. Die Zeilenhöhe lässt 
  sich mit \Macro{extrarowheight} ändern. 
\itempkg{booktabs}
  Für einen guten Tabellensatz mit \hologo{LaTeX} gibt es bereits zahlreiche 
  \href{http://userpage.fu-berlin.de/latex/Materialien/tabsatz.pdf}{Tipps} im 
  Internet zu finden. Zwei Regeln sollten dabei definitiv beachtet werden:
  \begin{enumerate}[itemindent=0pt,labelwidth=*,labelsep=1em,label=\Roman*.]
  \item keine vertikalen Linien
  \item keine doppelten Linien
  \end{enumerate}
  Das Paket \Package{booktabs} ist für den Satz von hochwertigen Tabellen eine 
  große Hilfe und stellt mit den Befehlen \Macro{toprule}, \Macro{bottomrule} 
  sowie \Macro{midrule} und \Macro{cmidrule} und unterschiedliche horizontale 
  Linien bereit.
\itempkg{widetable}
  Mit der Umgebung \Environment{tabular*} kann eine Tabelle mit einer 
  definierten Breite gesetzt werden. Dieses Paket stellt die zusätzliche 
  Umgebung \Environment{widetable} zur Verfügung, die als Alternative genutzt 
  werden kann und eine symmetrische Tabelle erzeugt.
\itempkg{tabularx}
  Auch mit diesem Paket lässt sich die Gesamtbreite einer Tabelle festlegen. 
  Dafür wird der Spaltentyp \PValue{X} definiert, welcher beliebig innerhalb 
  des Argumentes \MPName{Spalten} bei der Umgebung 
  \InlineDeclaration{\Environment{tabularx|\MPName{Breite}\MPName{Spalten}}} 
  angegeben werden kann. Die \PValue{X}"~Spalten ähneln denen vom 
  Typ~\PValue{p}\MPName{Breite}, wobei die Breite dieser aus der gewünschten 
  Tabellengesamtbreite abzüglich des benötigten Platzes der gegebenenfalls 
  vorhandenen Standardspalten \PValue{lcr} etc. automatisch berechnet wird.
\itempkg{tabulary}
  Dies ist ein weiteres Paket zur automatischen Berechnung von Spaltenbreiten. 
  Der zur Verfügung stehende Platz~-- gewünschte Gesamtbreite abzüglich der 
  notwendigen Breite für Standardspalten~-- wird jedoch nicht wie von 
  \Package{tabularx} auf alle Spalten gleichmäßig verteilt sondern für die 
  Spaltentypen~\PValue{LCRJ} anhand ihres Zellinhaltes in der Umgebung 
  \InlineDeclaration{\Environment{tabulary|\MPName{Breite}\MPName{Spalten}}} 
  gewichtet vergeben.
\itempkg{longtable,xltabular,ltxtable,ltablex}[v2.07:\Package{xltabular}]
  Sollen mehrseitige Tabellen mit Seitenumbruch erstellt werden, ist das Paket
  \Package{longtable} das Mittel der ersten Wahl. Zum Setzen einer mehrseitigen 
  \Environment{tabularx}"~Tabelle ist das Paket \Package{xltabular} 
  empfehlenswert, außerdem verfolgen die Pakete \Package{ltablex} und 
  \Package{ltxtable} das gleiche Ziel. Alternativ dazu lässt sich auch 
  \Package{tabu} nutzen.
\itempkg{multirow}
  Es wird der Befehl \Macro{multirow} definiert, der das Zusammenfassen von 
  mehreren Zeilen in einer Spalte ermöglicht~-- ähnlich zum Makro 
  \Macro{multicolumn} für Spalten.
\itempkg{tabularborder}
  Bei Tabellen wird zwischen Spalten automatisch ein horizontaler Abstand 
  (\Length{tabcolsep}) gesetzt~-- besser gesagt jeweils vor und nach einer 
  Spalte. Dies geschieht auch \emph{vor} der ersten und \emph{nach} der letzten 
  Spalte. Dieser zusätzliche Platz an den äußeren Rändern kann störend wirken, 
  insbesondere wenn die Tabelle über die komplette Textbreite gesetzt wird. Mit 
  dem Paket \Package{tabularborder} kann dieser Platz automatisch entfernt 
  werden.
  
  Dies funktioniert allerdings nur mit der \Environment{tabular}"~Umgebung. 
  Die Umgebungen aus den Paketen \Package{tabularx}, \Package{tabulary} und 
  \Package{tabu} werden nicht unterstützt. Wie der Abstand manuell bei den 
  zugehörigen Tabellen entfernt werden kann, ist unter \autoref{sec:tips:table} 
  zu finden.
\itempkg{tabu}[%
    v2.02:\Package{tabu} nur bedingt empfehlenswert;
    v2.07:\Package{tabu} in Version~v2.10 nur sehr eingeschränkt nutzbar;
  ]%
  Dieses Paket versucht, viele der zuvor genannten Funktionalitäten zu 
  implementieren und weitere bereitzustellen. Dafür werden die Umgebungen 
  \Environment{tabu} und \Environment{longtabu} definiert. Es lässt sich 
  alternativ zum Paket \Package{tabularx} verwenden und ist insbesondere 
  anstelle des Paketes \Package{ltxtable} eine Möglichkeit zur Erstellung 
  umbrechbarer Tabellen.
  
  \Attention{%
    Durch Änderungen am \hologo{LaTeXe}"~Kernel wurde das Paket unbrauchbar, 
    weshalb durch das \GitHubRepo[\hologo{LaTeX3}-Projektteam](tabu-fixed/tabu) 
    die allernötigsten Anpassungen vorgenommen worden, damit dieses in der 
    aktuellen Version~v2.9 zumindest grundsätzlich lauffähig ist. Weiterhin 
    wären seit einigen Jahren mehrere Bugfixes notwendig. Der originäre Autor 
    hatte außerdem die Änderung der 
    \href{https://groups.google.com/d/topic/comp.text.tex/xRGJTC74uCI}{%
      Benutzerschnittstelle in einer zukünftigen Version%
    }
    angekündigt, anscheinend aber mittlerweile die Pflege komplett eingestellt. 
    Ein Großteil der ursprünglich sehr guten Funktionalitäten kann momentan 
    nicht mehr genutzt werden, weshalb es aktuell nicht empfehlenswert ist.
  }%
\end{DeclarePackages}
\index{Tabellen|?)}%



\subsubsection{%
  Grafiken und Abbildungen%
  \index{Grafiken|?(}%
}
%
Grafiken für wissenschaftliche Arbeiten sollten als Vektorgrafiken erstellt 
werden, um die Skalierbarkeit und hohe Druckqualität zu gewährleisten. 
Bestenfalls folgen diese auch dem Stil der dazugehörigen Arbeit.%
\footnote{%
  Für qualitativ hochwertige Dokumente sollten übernommene Grafiken nicht 
  direkt kopiert oder gescannt sondern im gewünschten Zielformat neu erstellt 
  und mit einer Referenz auf die Quelle eingebunden werden.%
}
Für das Erstellen eigener Vektorgrafiken, welche die \hologo{LaTeX}"~Schriften 
und das Layout des Hauptdokumentes nutzen, gibt es zwei mögliche Ansätze. 
Entweder die Grafiken werden ähnlich wie das Dokument \enquote{programmiert} 
oder Zeichenprogramme, welche wiederum die Ausgabe oder das Weiterreichen von 
Text an \hologo{LaTeX} unterstützen, werden genutzt. Für das Programmieren von 
Grafiken sollen hier die wichtigsten Pakete vorgestellt werden. Wie diese zu 
verwenden sind, ist den dazugehörigen Paketdokumentationen zu entnehmen. 

\begin{DeclarePackages}
\itempkg{tikz}
  Dies ist ein sehr mächtiges Paket für das Programmieren von Vektorgrafiken 
  und sehr häufig~-- insbesondere bei Einsteigern~-- die erste Wahl bei der 
  Verwendung eines der Formate \Format{pdfLaTeX}, \Format{LuaLaTeX} oder 
  \Format{XeLaTeX}.
\itempkg{pstricks}
  Diese Paket stellt die zweite Variante zum Programmieren von Grafiken dar. 
  Da hiermit auf PostScript (direkt) zugegriffen werden kann, existieren 
  \emph{noch} mehr Möglichkeiten bei der Erstellung eigener Grafiken, wovon
  die bereitgestellten Befehle rege Gebrauch machen.
  
  Daraus resultiert allerdings der Nachteil, dass die mit \Package{pstricks} 
  erstellten Grafiken nicht direkt in eine PDF"~Datei kompiliert werden können. 
  Vielmehr müssen die gewünschten Grafiken zunächst über den konsekutiven 
  Aufruf des Pfades \Path{latex\,>\,dvips\,>\,ps2pdf} in PDF"~Dateien gewandelt 
  werden. Diese lassen sich von \Format{pdfLaTeX}, \Format{LuaLaTeX} oder 
  \Format{XeLaTeX} anschließend als Abbildungen einbinden. Um dieses Vorgehen 
  zu vereinfachen, lassen sich folgende Pakete nutzen, welche die Inhalte aus 
  den \Environment{pspicture}"~Umgebungen extrahieren und separat über den 
  genannten Pfad kompilieren:
  \begin{DeclarePackages}
  \itempkg{pst-pdf}
    Dieses Paket stellt Methoden für den Export von PostSript"~Grafiken in 
    PDF"~Datien bereit. Die einzelnen Aufrufe zur Kompilierung von DVI über 
    PostScript zu PDF müssen durch den Anwender manuell beziehungsweise über 
    die Ausgaberoutinen des verwendeten Editors durchgeführt werden.
  \itempkg{auto-pst-pdf,pdftricks2}
    Das Paket automatisiert die Erzeugung der \Package{pstricks}"~Grafiken mit 
    dem Paket \Package{pst-pdf}. Hierfür muss der Aufruf von \Format{pdfLaTeX}, 
    \Format{LuaLaTeX} oder \Format{XeLaTeX} mit erweiterten Schreibrechten 
    durch den Parameter \Path{-{}-shell-escape} erfolgen. Bitte beachten Sie 
    dazu die Hinweise in \autoref{sec:tips:auto-pst-pdf}. Eine Alternative dazu 
    ist das Paket \Package{pdftricks2}.
  \end{DeclarePackages}
\end{DeclarePackages}
%
Im Tutorial \Tutorial{treatise} wird für \Package{pstricks} und \Package{tikz} 
jeweils ein Beispiel gegeben. Um bei der Erstellung von Grafiken mit einem der 
beiden Paketen nicht bei jeder Änderung das komplette Dokument kompilieren zu 
müssen, können diese in separate Dateien ausgelagert werden. Hierfür sind die 
beiden Pakete \Package{standalone} oder \Package{subfiles} sehr nützlich.

Für das Zeichnen einer Grafik mit einem Bildbearbeitungsprogramm, welches die 
Weiterverarbeitung durch \hologo{LaTeX} erlaubt, möchte ich auf die freien 
Programme \Application{LaTeXDraw} und \Application{Inkscape} verweisen. 
Insbesondere das zuletzt genannte Programm ist sehr empfehlenswert. 
%
\begin{DeclarePackages}
\itempkg{svg}
  Mit diesem Paket können \emph{automatisiert} alle notwendigen Schritte zum 
  Einfügen einer mit der Anwendung \Application{Inkscape} erstellten 
  SVG"~Grafik in ein \hologo{LaTeX}"=Dokument durchgeführt werden. Weitere 
  Hinweise hierzu sind in \autoref{sec:tips:svg} zu finden.
\end{DeclarePackages}
\index{Grafiken|?)}%



\subsubsection{%
  Querverweise und Lesezeichen%
  \index{Querverweise|?(}%
  \index{Lesezeichen|?(}%
}
%
Für das Erzeugen von Querverweisen auf bestimmte Gliederungsebenen, Tabellen, 
Abbildungen oder auch Gleichungen muss für diese besagten Elemente zunächst mit 
\Macro{label|\MPName{Label}} ein \emph{eindeutiges} Label erzeugt werden, auf 
welches im Dokument entweder mit \Macro{ref} oder nach dem Laden von 
\Package{hyperref} besser noch mit \Macro{autoref} referenziert werden kann. In 
\autoref{sec:tips:references} sind diesbezüglich weitere Informationen zu 
finden.
%
\begin{DeclarePackages}
\itempkg{hyperref,bookmark}
  Für das Erstellen insbesondere von Hyperlinks aber auch erweiterten 
  Querverweisen sowie Lesezeichen~-- auch Outline-Einträge~-- in einem 
  PDF"~Dokument kann das Paket \Package{hyperref} genutzt werden. Wird 
  es geladen, sind außerdem die Option \Option{tudbookmarks} sowie der Befehl 
  \Macro{tudbookmark} nutzbar, welche von den \TUDScript-Klassen bereitgestellt 
  werden. Das Paket \Package{bookmark} erweitert nochmals die Funktionalität 
  für Lesezeichen, beispielsweise um die Möglichkeit zur Festlegung der Ebene 
  eines Outline-Eintrags sowie der Anpassung von Schriftfarbe- und "~stil.
  Beide Pakete sollten~-- bis auf sehr wenige, \emph{explizit dokumentierte} 
  Ausnahmen wie exemplarisch das Paket \Package{glossaries}~-- als letztes in 
  der Präambel eingebunden werden.
\itempkg{varioref}
  Hiermit können für Seitenverweise~-- nicht ausschließlich aber insbesondere 
  auf die aktuelle, vorhergehende oder nachfolgende sowie im doppelseitigen 
  Satz gegenüberliegende Seite~-- anpassbare Textbausteine anstelle der bloßen 
  Seitenzahl verwendet werden.
\itempkg{cleveref}
  Dieses Paket vereint die Vorzüge der automatischen Benennung referenzierter 
  Objekte mit dem Befehl \Macro{autoref} aus dem Paket \Package{hyperref} und 
  der Verwendung von \Package{varioref}.
\end{DeclarePackages}
\index{Lesezeichen|?)}%
\index{Querverweise|?)}%



\subsubsection{Aufteilung des Hauptdokumentes in Unterdateien}
%
Um während des Entwurfes eines Dokumentes die Zeitdauer für das Kompilieren zu 
verkürzen, kann dieses in Unterdokumente gegliedert werden. Dadurch wird es 
möglich, nur den momentan bearbeiteten Dokumentteil~-- respektive die aktuelle 
\Package{tikz}- oder \Package{pstricks}"~Grafik~-- zu kompilieren. Die meiner 
Meinung nach besten Pakete für dieses Unterfangen werden folgend vorgestellt.
%
\begin{DeclarePackages}
\itempkg{standalone}<\Class{standalone}>
  \ChangedAt*{v2.02:Bugfix für Verwendung der Klasse \Class{standalone}}%
  Das Paket ist für das Erstellen eigenständiger (Unter)"~Dokumente gedacht, 
  welche später in ein Hauptdokument eingebunden werden können. Jedes dieser 
  Teildokumente benötigt eine eigene Präambel. Optional lassen sich die 
  Präambeln der Unterdokumente automatisch in ein Hauptdokument einbinden. 
  Zusätzlich wird eine Klasse bereitgestellt, mit der eingenständige Elemente 
  wie beispielsweise zugeschnittene Grafiken erstellt werden können.
\itempkg{subfiles}[v2.02]
  Dieses Paket wählt einen etwas anderen Ansatz als \Package{standalone}. Es 
  ist von Anfang an dafür gedacht, ein dediziertes Hauptdokument zu verwenden. 
  Die darin mit \Macro{subfiles} eingebundenen Unterdateien nutzen bei der 
  autarken Kompilierung dessen Präambel.
\end{DeclarePackages}
%
Unabhängig davon, ob Sie eines der beiden Pakete nutzen oder alles in einem 
Dokument belassen, ist es ratsam, eigens definierte Befehle, Umgebungen und 
ähnliches in ein separates Paket auszulagern. Dafür müssen Sie lediglich eine 
leere Textdatei erzeugen und diese unter \Path{mypreamble.sty}~-- oder einem 
beliebigen anderen Namen mit der Dateiendung \Path{.sty}~-- im gleichen Ordner 
wie das Hauptdokument speichern. Anschließend können Sie in dieser Datei ihre 
Deklarationen vornehmen und diese mit dem gewählten Namen ohne die Dateiendung
\Macro||{usepackage|\MPValue{mypreamble}} in das Dokument einbinden. Dies hat 
den Vorteil, dass das Hauptdokument zum einen übersichtlich bleibt und Sie zum 
anderen Ihre persönliche Präambel generisch wachsen lassen und für andere 
Dokumente wiederverwenden können, wobei es dann sicherlich ratsam wäre, das 
Paket zentral in Ihrem lokalen \Path{texmf}"~Pfad abzulegen. 



\subsubsection{Die kleinen und großen Helfer\dots}
%
Hier taucht alles auf, was nicht in die vorherigen Kategorien eingeordnet 
werden konnte.
%
\begin{DeclarePackages}
\itempkg{marginnote}
  \index{Randnotizen}%
  Mit dem Befehl \Macro{marginpar} lassen sich Randnotizen erzeugen. Diese sind 
  in \hologo{LaTeX} spezielle Gleitobjekte, wodurch selbige nicht immer direkt 
  an der ursprünglich intendierten Stelle am Blattrand gesetzt werden. Das 
  Paket \Package{marginnote} stellt den Befehl \Macro{marginnote} für 
  nicht"~gleitende Randnotizen zur Verfügung. Eine Alternative dazu ist Paket 
  \Package{mparhack}.
\itempkg{todonotes}
  \index{Randnotizen}%
  Mit \Package{todonotes} können noch offene Aufgaben in unterschiedlicher 
  Formatierung am Blattrand oder im direkt Fließtext ausgegeben werden. Aus 
  allen Anmerkungen lässt sich eine Liste aller offenen Punkte erzeugen.
  \ToDo{\Package{fixme} erwähnen}[v2.07]
\itempkg{xparse}
  \index{Befehlsdeklaration}%
  Dieses mächtige Paket entstammt dem \hologo{LaTeX3}"~Projekt und bietet für 
  die Erstellung eigener Befehle und Umgebungen einen alternativen Ansatz zu 
  den bekannten \hologo{LaTeX}"=Deklarationsbefehlen \Macro||{newcommand} und 
  \Macro||{newenvironment} sowie deren Derivaten. Mit \Package{xparse} wird es 
  möglich, obligatorische und optionale Argumente an beliebigen Stellen 
  innerhalb des Befehlskonstruktes zu definieren. Auch die Verwendung anderer 
  Zeichen als eckige Klammern für die Spezifizierung eines optionalen 
  Argumentes ist möglich.
  \ToDo{xparse im Kernel?!}[v2.07]
\itempkg{keyval,xkeyval,pgfkeys,expkv,l3keys}[v2.02]
  \index{Befehlsdeklaration}%
  Von \Package{scrbase} wird das Paket \Package{keyval} geladen, um Optionen 
  oder Funktionsargumente mit einer Schlüssel"=Wert"=Syntax deklarieren zu 
  können. Diese Funktionalität kann auch für zusätzlich vom Anwender definierte 
  Makros genutzt werden, um innerhalb eines (optionalen) Argumentes mehrere 
  Parameter nutzen zu können.
  
  Soll gänzlich auf die Funktionalitäten von \Package{scrbase} verzichtet 
  werden~-- beispielsweise zur Entwicklung eigener Pakete~-- und dennoch eine 
  Verarbeitung von Optionen im Schlüssel"=Wert"=Format gewünscht sein, kann das 
  Paket \Package{xkeyval} zum Einsatz kommen, welches viele Möglichkeiten zur 
  Deklaration unterschiedlicher Typen von Schlüsseln bereitstellt. Alternativ 
  dazu kann auch die \Package{expkv}"~Familie respektive \Package{pgfkeys} oder 
  für \hologo{LaTeX3} auch \Package{l3keys} genutzt werden.
\itempkg{scrlfile}[v2.05]
  \index{Pakete!Abhängigkeiten}%
  Dieses \KOMAScript-Paket erlaubt es, auf das Laden von Klassen oder Paketen 
  direkt davor oder danach zu reagieren, um beispielsweise Paketabhängigkeiten 
  aufzulösen oder nach dem Laden eines bestimmten Paketes gezielt Befehle 
  anzupassen. Mehr dazu ist im \scrguide zu finden.
\itempkg{calc}
  \index{Berechnungen}%
  Normalerweise lassen sich Berechnungen im Dokument lediglich mit 
  Low"~Level"~\hologo{TeX}"=Primitiven durchführen. Dieses Paket stellt eine 
  einfachere Syntax für Rechenoperationen der vier Grundrechenarten zur 
  Verfügung. Zusätzlich werden neue Befehle zur Bestimmung der Höhe und Breite 
  bestimmter Textauszüge definiert.
\itempkg{mwe,blindtext}[v2.02]
  \index{Minimalbeispiel|?}%
  Mit dem Paket \Package{mwe} lassen sich sehr einfach Minimalbeispiele 
  erzeugen, die sowohl Blindtexte als auch Abbildungen enthalten sollen. Werden 
  keine Grafiken sondern lediglich Textabschnitte etc. benötigt, ist das Laden 
  von \Package{blindtext} ausreichend. Weiterhin können mit der Umgebung 
  \Environment{filecontents} gegebenfalls zusätzlich benötigte Textdateien~-- 
  beispielsweise eine einfache \File*{.bib}"~Literaturdatei~-- erzeugt werden.
\itempkg{pdfpages}
  Das Paket ermöglicht die Einbindung von einzelnen oder mehreren PDF"~Dateien.
\itempkg{crop}
  \index{Beschnittzugabe}%
  \index{Schnittmarken}%
  Hiermit können eine Beschnittzugabe sowie Schnittmarken~-- beispielsweise für 
  Poster~-- erzeugt werden. Hierzu ist in \autoref{sec:tips:crop} ein Beispiel 
  zu finden.
\itempkg{pagecolor}
  \index{Farben}%
  Mit dem Paket lässt sich die Hintergrundfarbe der Seiten im Dokument ändern.
\itempkg{afterpage}
  Der Befehl \Macro{afterpage|\MPValue{\dots}} kann genutzt werden, um den 
  Inhalt aus dessen Argument direkt nach der Ausgabe der aktuellen Seite 
  auszuführen.
  \ToDo{raus, text in Hinweise verschieben (pdffilemoddate)}[v2.07]
%\itempkg{filemod}
%  Wird entweder \Format{pdfLaTeX} oder \Format{LuaLaTeX} als Textsatzsystem 
%  eingesetzt, können mit diesem Paket das Änderungsdatum zweier Dateien 
%  miteinander verglichen und in Abhängigkeit davon definierbare Aktionen 
%  ausgeführt werden.
\itempkg{listings}[v2.02]
  \index{Quelltextdokumentation}%
  Dieses Paket eignet sich hervorragend zur Quelltextdokumentation in 
  \hologo{LaTeX}. Es bietet die Möglichkeit, externe Quelldateien einzulesen 
  und darzustellen sowie die Syntax in Abhängigkeit der verwendeten 
  Programmiersprache hervorzuheben. Zusätzlich lässt sich ein Verzeichnis mit 
  allen eingebundenen sowie direkt im Dokument angegebenen Quelltextauszügen 
  erstellen. Wird \Package{listings} in Dokumenten mit UTF"~8"~Kodierung 
  verwendet, sollten direkt nach dem Laden des Paketes in der Präambel folgende 
  Einstellungen vorgenommen werden:
  \begin{Code}
    \lstset{%
      inputencoding=utf8,extendedchars=true,
      literate=%
        {ä}{{\"a}}1 {ö}{{\"o}}1 {ü}{{\"u}}1
        {Ä}{{\"A}}1 {Ö}{{\"O}}1 {Ü}{{\"U}}1
        {ß}{{\ss}}1 {~}{{\textasciitilde}}1
        {»}{{\guillemetright}}1 {«}{{\guillemetleft}}1
    }
  \end{Code}\vspace{-\baselineskip}%
\itempkg{selinput,inputenc}
  \index{Eingabekodierung}%
  \index{Minimalbeispiel|?}%
  Die verwendete Eingabekodierung ist standardmäßig auf \PValue{utf8}
  festgelegt, was von (fast) allen \hologo{LaTeX}"~Editoren unterstützt wird. 
  Dies sollte im Normalfall auch nicht geändert werden. Ältere Dokumente sind 
  aber vielleicht noch in einer anderen Eingabekodierung gespeichert. Kann 
  diese nicht geändert werden, so ist es für \Format{pdfLaTeX} möglich, eine 
  andere~-- im \hyperref[sec:tips:editor]{Editor (\autoref*{sec:tips:editor})} 
  vom Anwender eingestellte~-- Eingabekodierungen zu nutzen. Diese kann mit dem 
  Paket \Package{selinput} (automatisch) für \hologo{LaTeX} festgelegt werden:
  \begin{Code}
    \usepackage{selinput}
    \SelectInputMappings{adieresis={ä},germandbls={ß}}
  \end{Code}\vspace{-\baselineskip}%
  Alternativ dazu lässt sich mit dem Paket \Package{inputenc} mit
  \begin{Code}[escapechar=§]
    \usepackage§\OPName{Eingabekodierung}§{inputenc}
  \end{Code}\vspace{-\baselineskip}%
  die Eingabekodierung manuell einstellen. Das Paket \Package{fontenc} sollte 
  in jedem Fall \emph{zuvor} geladen werden.
\end{DeclarePackages}



\subsubsection{Bugfixes}
%
\begin{DeclarePackages}
\itempkg{scrhack}
  \index{Kompatibilität!Pakete}%
  Das Paket behebt Kompatibilitätsprobleme der \KOMAScript-Klassen mit den 
  Paketen \Package{hyperref}, \Package{float}, \Package{floatrow} und
  \Package{listings}. Es ist durchaus empfehlenswert, jedoch sollte unbedingt 
  die Dokumentation beachtet werden.
\itempkg{mparhack}
  Zur Behebung falsch gesetzter Randnotizen wird ein Bugfix für den Befehl 
  \Macro{marginpar} bereitgestellt. Alternativ dazu lässt sich auch das Paket 
  \Package{marginnote} verwenden.
\itempkg{morewrites,scrwfile}
  Falls der Fehler 
  \begin{Code}
    No room for a new \write
  \end{Code}\vspace{-\baselineskip}
  erscheint, kann dieser sehr wahrscheinlich mit dem Paket \Package{morewrites} 
  behoben werden. Alternativ dazu lässt sich auch das Paket 
  \Package{scrwfile} nutzen. Mehr dazu in \autoref{sec:tips:write}.
\itempkg{fix-cm}
  Sollte bei einer Schriftgrößenänderung eine oder mehrere Warnungen der Form
  \begin{Code}
    Font shape `T1/cmr/m/n' in size <...> not available
  \end{Code}\vspace{-\baselineskip}%
  erscheinen, so sollte das Paket \Package{fix-cm} \emph{vor} der Klasse 
  geladen werden. Siehe dazu auch die Hinweise in \autoref{sec:tips:fontsize}.
\end{DeclarePackages}

\chapter{%
  Praktische Tipps \& Tricks%
  \label{sec:tips}%
}
\newcommand*\TaT[1]{%
  {"|Tipps und Tricks #1}@{\hyperref[sec:tips]{Tipps \& Tricks}: #1}%
}
\section{%
  \hologo{LaTeX}-Editoren%
  \label{sec:tips:editor}%
}
%
\ToDo[doc]{%
  Hinweis zu \Macro*{input glyphtounicode} und \Macro*{pdfgentounicode=1}; 
  Suche in source Dateien nach fontenc\textbar hyphsubst
}[v2.06]%
Hier werden die gängigsten Editoren zum Bearbeiten von \hologo{LaTeX}"~Dateien 
genannt. Ich persönlich nutze seit langem \Application{\hologo{TeX}studio}, da 
dieser viele Unterstützungs- und Assistenzfunktionen bietet. Neben diesen gibt 
es noch weitere, gut nutzbare \hologo{LaTeX}-Editoren. Unabhängig von der 
Auswahl des Editors, sollte dieser auf jeden Fall eine Unterstützung von 
Unicode~(UTF"~8) enthalten:
%
\begin{itemize}
\item \Application{\hologo{TeX}maker}
\item \Application{Kile}
\item \Application{\hologo{TeX}works}
\item \Application{\hologo{TeX}lipse}~-- Plug"~in für \Application{Eclipse}
\item \Application{\hologo{TeX}nicCenter}
\item \Application{WinEdt}
\item \Application{LEd}~-- früher \hologo{LaTeX}~Editor
\item \Application{\hologo{LyX}}~-- grafisches Front"~End für \hologo{LaTeX}
\end{itemize}
%
Für \Application{\hologo{TeX}studio} wird im \GitHubRepo'<releases> das Archiv 
\GitHubDownload<TeXstudio>{tudscr4texstudio.zip} zur Erweiterung der 
automatischen Befehlsvervollständigung für das \TUDScript-Bundle 
bereitgestellt. Die darin enthaltenen Dateien müssen unter Windows in 
\Path{\%APPDATA\%\textbackslash{}texstudio} beziehungsweise unter unixoiden 
Betriebssystemen in \Path{.config/texstudio} eingefügt werden.

Möchten Sie das grafische \hologo{LaTeX}"~Frontend~\Application{\hologo{LyX}} 
für das Erstellen eines Dokumentes mit den \TUDScript-Klassen nutzen, so werden 
dafür spezielle Layout"~Dateien benötigt, um die Klassendateien verwenden zu 
können. Diese sind zusammen mit einem \Application{\hologo{LyX}}"~Dokument als 
Archiv \GitHubDownload<LyX>{tudscr4lyx.zip} im \GitHubRepo'<releases> 
verfügbar. Um die Layout"~Dateien zu aktivieren, müssen diese in den passenden 
Unterordner im \Application{\hologo{LyX}}"=Installationspfad kopiert werden. 
Dieser ist bei Windows
\Path{%
  \%PROGRAMFILES(X86)\%\textbackslash{}LyX~2.1\textbackslash{}%
  Resources\textbackslash{}layouts%
}
beziehungsweise bei unixoiden Betriebssystemen \Path{/usr/share/lyx/layouts}.
Anschließend muss LyX über den Menüpunkt \emph{Werkzeuge} neu konfiguriert 
werden. 



\section{Literaturverwaltung in \hologo{LaTeX}}
%
\ChangedAt{v2.02:\TaT{Literaturverwaltung}}%
Die simpelste Variante, eine \hologo{LaTeX}"=Literaturdatenbank zu verwalten, 
ist dies mit dem Editor manuell zu erledigen. Wesentlich komfortabler ist es 
jedoch, die Referenzverwaltung mit einer darauf spezialisierten Anwendung zu 
bewerkstelligen. Dafür gibt es zwei sehr gute Programme:
%
\begin{itemize}
\item \Application{Citavi}
\item \Application{JabRef}
\end{itemize}
%
Das Programm \Application{Citavi} ermöglicht den Import von bibliografischen 
Informationen aus dem Internet. Allerdings sind diese teilweise unvollständig 
oder mangelhaft. Mit \Application{JabRef} hingegen muss die Literaturdatenbank 
manuell erstellt werden, wobei einzelne oder mehrere Einträge aus verschiedenen
\Path{.bib}"~Dateien importiert werden können. Beide Anwendungen unterstützen 
den Export beziehungsweise die Erstellung von Datenbanken im Stil von 
\Package{biblatex}. Für \Application{JabRef} muss dies durch den Anwender 
aktiviert werden.\footnote{Optionen/Einstellungen/Erweitert/BibLaTeX"~Modus} 
Zur Verwendung der beiden Programme in Verbindung mit \Package{biblatex} und 
\Application{biber} gibt es ein gutes Tutorial unter diesem
\hrfn{http://www.suedraum.de/latex/stammtisch/degenkolb_latex_biblatex_folien-final.pdf}{Link}.



\section{%
  Worttrennungen in deutschsprachigen Texten%
  \label{sec:tips:hyphenation}%
}
%
\ChangedAt{v2.02:\TaT{Worttrennungen}}%
Die möglichen Trennstellen von Wörtern werden von \hologo{LaTeX} mithilfe 
eines Algorithmus berechnet. Dieser ist jedoch in seiner ursprünglichen Form 
für die englische Sprache konzipiert worden. Für deutschsprachige Texte wird 
die Worttrennung~-- insbesondere bei zusammengeschriebenen Wörtern~-- mit dem 
Paket \Package{hyphsubst} entscheidend verbessert, welches ein um vielerlei 
Trennungsmuster ergänztes Wörterbuch aus dem Paket \Package{dehyph-exptl} 
nutzt. Weiterhin muss eines der beiden Sprachpakete \Package{babel} oder
\Package{polyglossia} geladen werden. 

Für \Format{pdfLaTeX} ist zusätzlich das Paket \Package{fontenc} beispielsweise 
mit der \PValue{T1}~Schriftkodierung erforderlich, damit auch Wörter mit 
Umlauten richtig getrennt werden. Bei der Verwendung von \Format{LuaLaTeX} 
oder \Format{XeLaTeX} werden~-- in Verbindung mit einem der zwei genannten 
Sprachpakete~-- die besseren Trennungsmuster automatisch aktiviert. Der Beginn 
einer Dokumentpräambel könnte folgendermaßen aussehen:
%
\begin{quoting}[rightmargin=0pt]
\begin{Code}[escapechar=§]
\documentclass[ngerman,§\PName{Klassenoptionen}§]§\MPName{Dokumentklasse}§
\usepackage{iftex}
\iftutex
  \usepackage{fontspec}
\else
  \usepackage[T1]{fontenc}
  \usepackage[ngerman=ngerman-x-latest]{hyphsubst}
\fi
\usepackage{babel}
\end{Code}
\end{quoting}
%
Eine Anmerkung noch zur Trennung von Wörtern mit Bindestrichen. Normalerweise 
sind die beiden von \hologo{LaTeX} verwendeten Zeichen für Bindestrich und 
Trennstrich identisch. Leider wird der Trennungsalgorithmus von \hologo{LaTeX} 
bei Wörtern, welche bereits einen Bindestrich enthalten, außer Kraft gesetzt. 
In der Folge werden~-- in der deutschen Sprache durchaus öfter anzutreffende~-- 
Wortungetüme wie die \enquote{Donaudampfschifffahrts-Gesellschafterversammlung} 
normalerweise nur direkt nach dem angegebenen Bindestrich getrennt. 

Allerdings gibt es die Möglichkeit, das genutzte Zeichen für den Trennstrich 
zu ändern. Dafür ist das Laden der \PValue{T1}"~Schriftkodierung mit dem Paket 
\Package{fontenc} zwingend erforderlich. Wenn von der verwendeten Schrift 
nichts anderes eingestellt ist, liegen sowohl Binde- als auch Trennstrich auf 
Position~\PValue{45} der Zeichentabelle. In der \PValue{T1}"~Schriftkodierung 
befindet sich auf der Position~\PValue{127} glücklicherweise für gewöhnlich das 
gleiche Zeichen noch einmal. Dies ist jedoch von der verwendeten Schrift 
abhängig. Wird der Ausdruck \Macro*{defaulthyphenchar}[\PValue{=127}] vor dem 
Laden des Paketes \Package{fontenc} verwendet, kann dieses Zeichen für den 
Trennstrich genutzt werden. 

Sollte trotz aller Maßnahmen dennoch einmal ein bestimmtes Wort falsch getrennt 
werden, so kann die Worttrennung dieses Wortes manuell und global geändert 
werden. Dies wird mit \Macro{hyphenation}[\MPValue{Sil-ben-tren-nung}] gemacht. 
Es ist zu beachten, dass dies für alle Flexionsformen des Wortes erfolgen 
sollte. Für eine lokale/temporäre Worttrennung kann mit Befehlen aus dem Paket 
\Package{babel} gearbeitet werden. Diese sind: 

\vskip\medskipamount\noindent
\begingroup
\newcommand*\listhyphens[2]{#1&\PValue{#2}\tabularnewline}%
\begin{tabular}{@{}ll}
  \textbf{Beschreibung}&\textbf{Befehl}\tabularnewline
  \listhyphens{ausschließliche Trennstellen}{\textbackslash-}
  \listhyphens{zusätzliche Trennstellen}{"'-} 
  \listhyphens{Umbruch ohne Trennstrich}{"'"'}
  \listhyphens{Bindestrich ohne Umbruch}{"'\textasciitilde} 
  \listhyphens{Bindestrich, der weitere Trennstellen erlaubt}{"'=}
\end{tabular}
\endgroup



\section{%
  Lokale Änderungen von Befehlen und Einstellungen%
  \label{sec:tips:local}%
  \index{Befehlsdeklaration!Geltungsbereich}%
}
%
\ChangedAt{v2.02:\TaT{Lokale Änderungen}}%
Ein zentraler Bestandteil von \hologo{LaTeX} ist die Verwendung von Gruppen 
oder Gruppierungen. Innerhalb dieser bleiben alle vorgenommenen Änderungen an 
Befehlen, Umgebungen oder Einstellungen lokal. Dies kann sehr nützlich sein, 
wenn beispielsweise das Verhalten eines bestimmten Makros einmalig oder 
innerhalb von selbst definierten Befehlen oder Umgebungen geändert werden, im 
Normalfall jedoch die ursprüngliche Funktionalität behalten soll.
\begin{Example}
\index{Schriftauszeichnung}%
Der Befehl \Macro{emph} wird von \hologo{LaTeX} für Hervorhebungen im Text 
bereitgestellt und führt normalerweise zu einer kursiven oder~-- falls kein 
Schriftschnitt mit echten Kursiven vorhanden ist~-- kursivierten oder auch 
geneigten Auszeichnung. Soll nun in einem bestimmten Abschnitt die Auszeichnung 
mit fetter Schrift erfolgen, kann der Befehl \Macro{emph} innerhalb einer 
Gruppierung geändert und verändert werden. Wird diese beendet, verhält sich der 
Befehl wie gewohnt.
\begin{Code}
In diesem Text wird genau ein \emph{Wort} hervorgehoben.

\begingroup
  \renewcommand*{\emph}[1]{\textbf{#1}}%
  In diesem Text wird genau ein \emph{Wort} hervorgehoben.
\endgroup

In diesem Text wird genau ein \emph{Wort} hervorgehoben.
\end{Code}
\end{Example}
Eine Gruppierung kann entweder mit \Macro*{begingroup} und \Macro*{endgroup} 
oder einfach mit einem geschweiften Klammerpaar \MPValue{\dots} definiert 
werden.



\section{%
  Bezeichnung der Gliederungsebenen durch \PackageRaw{hyperref}{\BooleanFalse}%
  \label{sec:tips:references}%
  \index{Querverweise}%
}
%
\ChangedAt{v2.02:\TaT{Bezeichnungen der Gliederungsebenen}}%
Das Paket \Package{hyperref} stellt für Querverweise unter anderem den Befehl 
\Macro{autoref}[\MPName{label}](\Package{hyperref}) zur Verfügung. Mit diesem 
wird~-- im Gegensatz zur Verwendung von \Macro{ref}~-- bei einer Referenz nicht 
nur die Nummerierung selber sondern auch das entsprechende Element wie Kapitel 
oder Abbildung vorangestellt. Zur Benennung des referenzierten Elementes wird 
geprüft, ob das Makro \Macro*{\PName{Element}autorefname} oder alternativ 
\Macro*{\PName{Element}name} existiert. Zur Änderung der Bezeichnung eines 
Elementes, muss der entsprechende Bezeichner angepasst werden.
%
\begin{Example}
Bezeichnungen von Gliederungsebenen können folgendermaßen verändert werden.
\begin{Code}
\renewcaptionname{ngerman}{\sectionautorefname}{Unterkapitel}
\renewcaptionname{ngerman}{\subsectionautorefname}{Abschnitt}
\end{Code}
\end{Example}



\section{%
  URL-Umbrüche im Literaturverzeichnis mit \PackageRaw{biblatex}{\BooleanFalse}%
  \index{Literaturverzeichnis}%
}
%
\ChangedAt{v2.02:\TaT{URL-Umbrüche im Literaturverzeichnis}}%
Wird das Paket \Package{biblatex} verwendet, kann es unter Umständen dazu 
kommen, das eine URL nicht vernünftig umbrochen wird. Ist dies der Fall, 
können die Zähler \Counter{biburlnumpenalty}(\Package{biblatex}), 
\Counter{biburlucpenalty}(\Package{biblatex}) und 
\Counter{biburllcpenalty}(\Package{biblatex}) erhöht werden. Das Manipulieren 
eines Zähler kann mit \Macro*{setcounter}[\MPName{Zähler}] oder lokal mit 
\Macro*{defcounter}[\MPName{Zähler}] aus dem Paket \Package{etoolbox} erfolgen. 
Die möglichen Werte liegen zwischen 0~und~10\,000, wobei es bei höheren 
Zählerwerten zu mehr URL"~Umbrüchen an 
Ziffern~(\Counter{biburlnumpenalty}(\Package{biblatex})), 
Majuskeln~(\Counter{biburlucpenalty}(\Package{biblatex})) und 
Minuskeln~(\Counter{biburllcpenalty}(\Package{biblatex})) kommt. 
Genaueres in der Dokumentation zu \Package{biblatex}.



\section{%
  Zeilenabstände in Überschriften%
  \index{Zeilenabstand}%
  \index{Durchschuss}%
  \label{sec:tips:headings}%
}
%
Mit dem Paket \Package{setspace} kann der Zeilenabstand beziehungsweise der 
Durchschuss innerhalb des Dokumentes geändert werden. Sollte dieser erhöht 
worden sein, können die Abstände bei mehrzeiligen Überschriften als zu groß 
erscheinen. Um dies zu korrigieren kann mit dem Befehl 
\Macro{addtokomafont}[%
  \MPValue{disposition}\MPValue{%
    \Macro{setstretch}[\MPValue{1}](\Package{setspace})%
  }%
](\Bundle{koma-script})
der Zeilenabstand aller Überschriften auf einzeilig zurückgeschaltet werden. 
Soll dies nur für eine bestimmte Gliederungsebene erfolgen, so ist der 
Parameter \PValue{disposition} durch das dazugehörige Schriftelement zu 
ersetzen.



\section{%
  Warnung wegen zu geringer Höhe der Kopf-/Fußzeile%
  \label{sec:tips:headline}%
}
%
Wird das Paket \Package{setspace} verwendet, kann es passieren, dass nach der 
Änderung des Zeilenabstandes \emph{innerhalb} des Dokumentes eine oder beide 
der folgenden Warnungen erscheinen:
%
\begin{quoting}
\begin{Code}
scrlayer-scrpage Warning: \headheight to low.
scrlayer-scrpage Warning: \footheight to low.
\end{Code}
\end{quoting}
%
Dies liegt an dem durch den vergrößerten Zeilenabstand erhöhten Bedarf für die
Kopf- und Fußzeile, die Höhen können in diesem Fall direkt mit der Verwendung 
von \Macro{recalctypearea}(\Package{typearea}) angepasst werden. Allerdings 
ändert das den Satzspiegel im Dokument, was eine andere und durchaus 
berechtigte Warnung von \Package{typearea} zur Folge hat. Falls die Änderung 
des Durchschusses wirklich nötig ist, sollte dies in der Präambel des 
Dokumentes einmalig passieren. Dann entfallen auch die Warnungen.



\section{%
  Einrückung von Tabellenspalten verhindern%
  \label{sec:tips:table}%
  \index{Tabellen}%
}
%
Normalerweise wird in einer Tabelle vor \emph{und} nach jeder Spalte durch 
\hologo{LaTeX} etwas horizontaler Raum mit \Macro*{hskip}\Length{tabcolsep} 
eingefügt.%
\footnote{%
  Der Abstand zweier Spalten beträgt folglich \PValue{2}\Length{tabcolsep}.%
}
Dies geschieht auch \emph{vor} der ersten und \emph{nach} der letzten Spalte. 
Diese optische Einrückung an den äußeren Rändern kann unter Umständen stören, 
insbesondere bei Tabellen, die willentlich~-- beispielsweise mit den Paketen 
\Package{tabularx}, \Package{tabulary} oder auch \Package{tabu}~-- über die 
komplette Seitenbreite aufgespannt werden.

Das Paket \Package{tabularborder} versucht, dieses Problem automatisiert zu 
beheben, ist jedoch nicht zu allen \hologo{LaTeX}"~Paketen für den 
Tabellensatz kompatibel, unter anderem auch nicht zu den drei zuvor genannten. 
Allerdings lässt sich dieses Problem manuell lösen. 

Bei der Deklaration einer Tabelle kann mit~\PValue{@}\MPValue{\dots} vor und 
nach dem Spaltentyp angegeben werden, was anstelle von \Length{tabcolsep} vor 
beziehungsweise nach der eigentlichen Spalte eingeführt werden soll. Dies kann 
für das Entfernen der Einrückungen genutzt werden, indem an den entsprechenden 
Stellen~\PValue{@\MPValue{}} bei der Angabe der Spaltentypen vor der ersten und 
nach der letzten Tabellenspalte verwendet wird.
%
\begin{Example}
Eine Tabelle mit zwei Spalten, wobei bei einer die Breite automatisch berechnet 
wird, soll über die komplette Textbreite gesetzt werden. Dabei soll der Rand 
vor der ersten und nach der letzten entfernt werden.
\begin{Code}[escapechar=§]
\begin{tabularx}{\textwidth}{@{}lX@{}}
§\dots§ & §\dots§ \tabularnewline
§\dots§
\end{tabularx}
\end{Code}
\end{Example}



\section{%
  Unterdrückung des Einzuges eines Absatzes%
  \index{Absatzauszeichnung}%
}
\begin{Entity}{\Bundle{koma-script}}
Durch die \KOMAScript-Option \InlineDeclaration{\Option{parskip=\PMisc}} wird
die Absatzauszeichnung  festgelegt. Werden hierfür keine vertikalen Abstände 
(\Option{parskip=half}'none' oder \Option{parskip=true}'none' sondern~-- wie 
es aus typografischer Sicht zumeist sinnvoll ist~-- stattdessen Einzüge 
verwendet (\Option{parskip=false})'none', kann es vorkommen, dass ein ganz 
bestimmter Absatz~-- beispielsweise der nach einer zuvor genutzten Umgebung 
folgende~-- ungewollt eingerückt ist.
\end{Entity}
Dies kann sehr einfach manuell behoben werden, indem direkt zu Beginn des 
Absatzes das Makro \Macro{noindent}(\Bundle{latex-base}) aufgerufen wird. Soll 
das Einrücken von Absätzen nach ganz bestimmten Umgebungen oder Befehlen 
automatisiert unterbunden werden, ist das Paket \Package{noindentafter} zu 
empfehlen.



\section{%
  Unterbinden des Zurücksetzens von Fußnoten%
  \label{sec:tips:counter}%
  \index{Fußnoten}%
}
%
Mit \Macro{counterwithin} und \Macro{counterwithout} können 
\hologo{LaTeX}"~Zähler so umdefiniert werden, dass sie bei der Änderung eines 
anderen Zählers zurückgesetzt werden oder nicht. Mit der Angabe von 
\Macro*{counterwithout*}[\MPValue{footnote}\MPValue{chapter}] wird das 
Zurücksetzen des Fußnotenzählers durch neue Kapitel deaktiviert, womit sich 
über Kapitel fortlaufende Fußnoten realisieren lassen.



\section{%
  Warnung beim Erzeugen des Inhaltsverzeichnisses%
  \index{Inhaltsverzeichnis}%
}
%
\ChangedAt{v2.02:\TaT{Warnung beim Erzeugen des Inhaltsverzeichnisses}}%
Wird mit \Macro{tableofcontents}(\Bundle{koma-script}) das Inhaltsverzeichnis 
für ein Dokument mit einer dreistelligen Seitenanzahl erstellt, so erscheinen 
unter Umständen viele Warnungen mit der Meldung:
%
\begin{quoting}
\begin{Code}
overfull \hbox
\end{Code}
\end{quoting}
%
Die Seitenzahlen im Verzeichnis werden in einer Box mit einer festen Breite 
von~\PValue{1.55em} gesetzt, welche im Makro \Macro*{@pnumwidth} hinterlegt 
ist und im Zweifel vergrößert werden sollte. Dabei ist auch der rechte Rand für 
\ChangedAt{v2.05}%
mehrzeilige Einträge im Verzeichnis \Macro*{@tocrmarg} zu vergrößern, welcher 
mit~\PValue{2.55em} voreingestellt ist. Die Werte sollten nur minimal geändert 
werden:
%
\begin{quoting}
\begin{Code}
\makeatletter
\renewcommand*{\@pnumwidth}{1.9em}\renewcommand*{\@tocrmarg}{2.9em}
\makeatother
\end{Code}
\end{quoting}



\section{%
  Leer- und Satzzeichen nach \hologo{LaTeX}-Befehlen%
  \label{sec:tips:xspace}%
  \index{Typografie}%
}
%
Normalerweise \enquote{schluckt} \hologo{LaTeX} die Leerzeichen nach einem 
Makro ohne Argumente. Dies ist jedoch nicht immer~-- genau genommen in den 
seltensten Fällen~-- erwünscht. Für dieses Handbuch ist beispielsweise der 
Befehl \Macro*{TUD} definiert worden, um \enquote{\TUD{}} nicht ständig 
ausschreiben zu müssen. Um sich bei der Verwendung des Befehl innerhalb eines 
Satzes für den Erhalt eines folgenden Leerzeichens das Setzen der geschweiften 
Klammer nach dem Befehl zu sparen (\Macro*{TUD}[\MPValue{}]), lässt sich  
\Macro{xspace}(\Package{xspace}) aus dem Paket \Package{xspace} nutzen. 
Damit wird ein folgendes Leerzeichen erhalten. Der Befehl \Macro*{TUD} ist wie 
folgt definiert:
%
\begin{quoting}
\begin{Code}
\newcommand*{\TUD}{Technische Universit\"at Dresden\xspace}
\end{Code}
\end{quoting}
%
Das Paket \Package{xpunctuate} erweitert die Funktionalität nochmals. Damit 
können auch Abkürzungen so definiert werden, dass ein versehentlicher Punkt 
ignoriert wird:
%
\begin{quoting}
\begin{Code}
\newcommand*{\zB}{z.\,B\xperiod}
\end{Code}
\end{quoting}



\section{%
  Finden von unbekannten \hologo{LaTeX}-Symbolen%
  \index{Symbole}%
}
%
Für \hologo{LaTeX} stehen jede Menge Symbole zur Verfügung, die allerdings 
nicht immer einfach zu finden sind. In der Zusammenfassung
\hrfn{http://mirrors.ctan.org/info/symbols/comprehensive/symbols-a4.pdf}%
{\File{symbols-a4.pdf}}
werden viele Symbole aus mehreren Paketen aufgeführt. Alternativ kann 
\hrfn{http://detexify.kirelabs.org/classify.html}{Detexify} verwendet werden. 
Auf dieser Webseite wird das gesuchte Symbol einfach gezeichnet, die dazu 
ähnlichsten werden zurückgegeben.



\section{%
  Das Setzen von Auslassungspunkten%
  \label{sec:tips:dots}%
  \index{Typografie}%
}
%
\ChangedAt{v2.02:\TaT{Das Setzen von Auslassungspunkten}}%
Auslassungspunkte werden für \hologo{LaTeX} mit den Befehlen \Macro{dots} oder 
\Macro{textellipsis} gesetzt. Für gewöhnlich folgt diesen \emph{immer} ein 
Leerzeichen, was nicht in jedem Fall gewünscht ist. Abhilfe schafft das Paket 
\Package{ellipsis} mit der Option \Option*{xspace}(\Package{ellipsis}), wodurch 
bei der Verwendung dieser Befehle nur ein Leerzeichen gesetzt wird, wenn nicht 
direkt danach ein Interpunktionszeichen folgt.
%
\begin{quoting}
\begin{Code}
\usepackage[xspace]{ellipsis}
\end{Code}
\end{quoting}
%
Im Ursprung ist es für das Setzen englischsprachiger Texte gedacht, wo zwischen 
Auslassungspunkten und Satzzeichen ein Leerzeichen gesetzt wird. Im Deutschen 
ist dies anders:
%
\begin{quoting}
\enquote{%
  Um eine Auslassung in einem Text zu kennzeichnen, werden drei Punkte gesetzt. 
  Vor und nach den Auslassungspunkten wird jeweils ein Wortzwischenraum 
  gesetzt, wenn sie für ein selbständiges Wort oder mehrere Wörter stehen. Bei 
  Auslassung eines Wortteils werden sie unmittelbar an den Rest des Wortes 
  angeschlossen. Am Satzende wird kein zusätzlicher Schlusspunkt gesetzt. 
  Satzzeichen werden ohne Zwischenraum angeschlossen.%
}~[Duden, 23. Aufl.]
\end{quoting} 
%
Um dieses Verhalten zu erreichen, sollte noch Folgendes in der Präambel 
eingefügt werden:
%
\begin{quoting}
\begin{Code}
\let\ellipsispunctuation\relax
\newcommand*{\qdots}{[\dots{}]\xspace}
\end{Code}
\end{quoting}
%
Der Befehl \Macro*{qdots} wird definiert, um Auslassungspunkte in eckigen 
Klammern (\OPValue{\dots}) setzen zu können, wie sie für das Kürzen von 
wörtlichen Zitaten häufig verwendet werden.



\section{%
  Lokalisierung für das Setzen von Einheiten mit 
  \PackageRaw{siunitx}{\BooleanFalse}%
  \label{sec:tips:siunitx}%
  \index{Einheiten}%
}
%
Wenn \Package{siunitx} in einem deutschsprachigen Dokument genutzt wird, muss 
zumindest die richtige Lokalisierung mit 
\Macro{sisetup}[\MPValue{locale = DE}](\Package{siunitx}) angegeben werden. 
Sollen auch die Zahlen richtig formatiert sein, müssen weitere Einstellungen 
vorgenommen werden. Die meiner Meinung nach besten sind die folgenden.
%
\begin{quoting}
\begin{Code}
\sisetup{%
  locale = DE,%
  input-decimal-markers={,},input-ignore={.},%
  group-separator={\,},group-minimum-digits=3%
}
\end{Code}
\end{quoting}
%
Das Komma kommt als Dezimaltrennzeichen zum Einsatz. Des Weiteren werden Punkte 
innerhalb der Zahlen ignoriert und eine Gruppierung von jeweils drei Ziffern 
vorgenommen. Alternativ zu diesem Paket kann übrigens auch \Package{units} 
verwendet werden.



\section{%
  Fehlermeldung beim Laden eines Paketes mit Optionen%
  \label{sec:tips:options}%
}
\ToDo{label für \Macro*{usepackage} etc. in Überschrift?}%
\ToDo{sind die Label überhaupt sinnvoll? wahrscheinich nur für PassOpt}%
\ChangedAt{v2.05:\TaT{Fehler beim Laden eines Paketes mit Optionen}}%
Es kann unter Umständen passieren, dass beim Laden eines Paketes mit 
zusätzlichen Optionen via \InlineDeclaration{%
  \Macro{usepackage}[\OPName{Paketoptionen}\MPName{Paket}](\Bundle{latex-base})%
} folgender Fehler ausgegeben wird:
%
\begin{quoting}
\begin{Code}[escapechar=§]
! LaTeX Error: Option clash for package <...>.
\end{Code}
\end{quoting}
%
Wahrscheinlich wird das angeforderte Paket bereits durch die verwendete 
Dokumentklasse oder ein anderes Paket geladen. Normalerweise genügt es, die 
gewünschten Optionen mit
\begin{quoting}
\begin{Entity}{\Bundle{latex-base}}
\InlineDeclaration{%
  \Macro{PassOptionsToPackage}[\MPName{Paketoptionen}\MPName{Paket}]%
}\newline%
\Macro*{documentclass}[\OPName{Klassenoptionen}\MPName{Klasse}]%
\end{Entity}
\end{quoting}
bereits vor dem Laden der Dokumentklasse an das betreffende Paket 
weiterzureichen.



\section{%
  Probleme bei der Verwendung von \PackageRaw{auto-pst-pdf}{\BooleanFalse}%
  \label{sec:tips:auto-pst-pdf}%
}
%
\ChangedAt{v2.02:\TaT{Hinweise zum Paket \Package{auto-pst-pdf}}}%
Bei der Verwendung von \Format{pdfLaTeX} liest das Paket \Package{auto-pst-pdf} 
die Präambel ein und erstellt anschließend über den PostScript-Pfad 
\Path{latex\,>\,dvips\,>\,ps2pdf} eine PDF"~Datei, welche lediglich alle in den 
vorhandenen \Environment{pspicture}(\Package{pstricks})"~Umgebungen erstellten 
Grafiken enthält. Mit dem Befehl \Macro{ifpdf}(\Package{iftex}) aus dem Paket 
\Package{iftex} lässt sich unterscheiden, ob die Ausgabe in eine PDF"~ oder 
DVI"~Datei erfolgt. Folgend wird abhängig vom Ausgabeformat unterschiedlicher 
Quelltext ausgeführt.
%
\begin{quoting}
\begin{Code}
\usepackage{iftex}
\end{Code}
\end{quoting}

\minisec{Die gleichzeitige Verwendung von \Package{floatrow}}
%
Das Paket \Package{floatrow} stellt Befehle bereit, mit denen die Beschriftung 
von Gleitobjekten sehr bequem gesetzt werden können. Diese Setzen ihren Inhalt 
vor der eigentlichen Ausgabe erst in einer Box, um deren Breite zu ermitteln. 
In Kombination mit \Package{auto-pst-pdf} führt das zu einer doppelten 
Erstellung der gewünschten Abbildung. Um dies zu vermeiden, müssen die durch 
\Package{floatrow} bereitgestellten Makros \enquote{unschädlich} gemacht 
werden. Die fraglichen Befehle akzeptieren allerdings bis zu drei optionale 
Argumente \emph{vor} den beiden obligatorischen, was bei \hologo{LaTeX} für die 
(Re"~)Definition von Befehlen normalerweise nicht vorgesehen ist. Deshalb wird 
das Paket \Package{xparse} geladen, mit welchem dies möglich wird. Genaueres 
dazu ist der dazugehörigen Paketdokumentation zu entnehmen. Mit folgendem 
Quelltextauszug lassen sich die Befehle des Paketes \Package{floatrow} zusammen 
mit der \Environment{pspicture}(\Package{pstricks})"~Umgebung wie gewohnt 
verwenden.
%
\begin{quoting}
\begin{Code}
\usepackage{floatrow}
\usepackage{xparse}
\ifpdf\else
  \RenewDocumentCommand{\fcapside}{ooo+m+m}{#4#5}
  \RenewDocumentCommand{\ttabbox}{ooo+m+m}{#4#5}
  \RenewDocumentCommand{\ffigbox}{ooo+m+m}{#4#5}
\fi
\end{Code}
\end{quoting}

\minisec{Die parallele Nutzung von \Package{tikz} und \Package{todonotes}}
%
Mit dem Paket \Package{tikz}~-- und auch allen anderen Paketen die 
selbiges nutzen wie beispielsweise \Package{todonotes}~-- gibt es in Verbindung 
mit \Package{auto-pst-pdf} ebenfalls Probleme. Lösen lässt sich dieses Dilemma, 
indem die fraglichen Pakete lediglich geladen werden, wenn \Format{pdfLaTeX} 
aktiv ist.
%
\begin{quoting}[rightmargin=0pt]
\begin{Code}
\ifpdf
  \usepackage{tikz}%\dots gegebenenfalls weitere auf tikz basierende Pakete
\fi
\end{Code}
\end{quoting}



\section{%
  Automatisiertes Einbinden von 
  \ApplicationRaw{Inkscape}{\BooleanFalse}-Grafiken%
  \label{sec:tips:svg}%
  \index{Grafiken}%
}
%
\ChangedAt{v2.05:\TaT{Einbinden von \Application{Inkscape}-Grafiken}}%
Das Einbinden von \Application{Inkscape}"=Grafiken in \hologo{LaTeX}"~Dokumente
ist Dank der entsprechenden Exportfunktion prinzipiell möglich. Allerdings muss 
hierfür relativ viel manueller Aufwand betrieben werden, die notwendigen 
Schritte sind im \CTAN[pkg/svg-inkscape] dokumentiert. Ein daraus 
abgeleiteter und verbesserter Ansatz wird durch das Paket \Package{svg} 
bereitgestellt. Mit diesem Paket ist ein \textbf{automatisierter} Export und 
anschließendes Einbinden von \Application{Inkscape}"=Grafiken in das Dokument 
bei der Nutzung eines \Format*{LaTeX}"~Formats möglich. Für diesen Zweck wird 
der Befehl \Macro*{includesvg}[\OPName{Parameter}\MPName{SVG-Datei}] als 
zentrale Benutzerschnittstelle durch das Paket definiert.

Dabei erfolgt der externe Aufruf von \Application{Inkscape} über Kommandozeile 
respektive Terminal mit \Path{inkscape.exe}. Damit dieser auch tatsächlich 
erfolgen kann, ist die Ausführung einer \Format*{LaTeX}"~Engine mit dem
Parameter \Path{-{}-shell-escape} zwingend notwendig. Außerdem muss der 
Pfad zur Datei \Path{inkscape.exe} dem System bekannt sein.%
\footnote{%
  Der Pfad zu \Path{inkscape.exe} in der Umgebungsvariable \Path{PATH} des 
  Betriebssystems enthalten sein.%
}
Für weiterführende Informationen sei auf die Dokumentation des Pakets 
\Package{svg} verwiesen.



\section{%
  Änderung des Papierformates%
  \index{Papierformat}%
}
%
Es kann vorkommen, dass innerhalb eines Dokumentes kurzzeitig das Papierformat 
geändert werden soll, um beispielsweise eine Konstruktionsskizze in der 
digitalen PDF"~Datei einzubinden. Dabei ist es mit der \KOMAScript-Option 
\Option{paper=\PMisc}(\Package{typearea}) sowohl möglich, lediglich die 
Ausrichtung in ein Querformat zu ändern, als auch die Größe des Papierformates 
selber.
%
\begin{Example}
Ein Dokument im A4"~Format soll kurzzeitig auf ein A3"~Querformat geändert 
werden. Das folgende Minimalbeispiel zeigt, wie sich dies mit den Möglichkeiten 
von \KOMAScript über die Optionen \Option*{paper=landscape}(\Package{typearea}) 
und \Option*{paper=A3}(\Package{typearea}) umsetzen lässt.
\begin{Code}
\documentclass[paper=A4,pagesize]{tudscrreprt}
\usepackage[T1]{fontenc}
\usepackage[ngerman]{babel}
\usepackage{blindtext}

\begin{document}
\chapter{Überschrift Eins}
\Blindtext

\cleardoublepage
\storeareas\PotraitArea% speichert den aktuellen Satzspiegel
\KOMAoptions{paper=A3,paper=landscape,DIV=current}
\chapter{Überschrift Zwei}
\Blindtext

\cleardoublepage
\PotraitArea% lädt den gespeicherten Satzspiegel
\chapter{Überschrift Drei}
\Blindtext
\end{document}
\end{Code}
\end{Example}



\section{%
  Beschnittzugabe und Schnittmarken%
  \label{sec:tips:crop}%
  \index{Beschnittzugabe|!}%
  \index{Schnittmarken|!}%
}
%
\ChangedAt{v2.05:\TaT{Beschnittzugabe und Schnittmarken}}%
Beim Plotten von Postern oder anderen farbigen Druckerzeugnissen besteht 
oftmals das Problem, dass ein randloses Drucken nur schwer realisierbar ist. 
Deshalb wird zu oftmals damit beholfen, dass der Druck des fertigen Dokumentes 
auf einem größeren Papierbogen erfolgt und anschließend auf das gewünschte 
Zielformat zugeschnitten wird, womit das Problem des nicht bedruckbaren Randes 
entfällt. Dies kann über zwei verschiedene Wege realisiert werden.

Der einfachste Weg ist die Verwendung des Paketes \Package{crop}. Mit diesem 
kann das Dokument ganz normal im gewünschten Zielformat erstellt werden. Vor 
dem Druck wird dieses Paket geladen und einfach das gewünschte Format des 
Papierbogens angegeben. 
%
\begin{quoting}[rightmargin=0pt]
\begin{Code}[escapechar=§]
\RequirePackage{fix-cm}
\documentclass[%
  paper=A1,
  fontsize=36pt
]{tudscrposter}
\usepackage[T1]{fontenc}
§\dots§
\usepackage{graphicx}
\usepackage[b1,center,cam]{crop}
\begin{document}
§\dots§
\end{document}
\end{Code}
\end{quoting}
%
\ToDo{check index, remove default, maybe use InlineDeclaration}[v2.07]%
\begin{Entity}{\Package{geometry}}
Alternativ dazu kann für die \TUDScript-Klassen auf die Funktionalität des 
Paketes \Package{geometry} zurückgegriffen werden. Dieses Paket stellt den 
Befehl \Macro{geometry} bereit, in dessen Argument mit
\Key{\Macro{geometry}}{paper=\PSet{Papierformat}} das Papierformat festgelegt 
werden kann. Wird zusätzlich noch der Parameter 
\Key{\Macro{geometry}}{layout=\PSet{Zielformat}} angegeben, so wird damit das 
gewünschte Zielformat definiert. Dabei sollte mit 
\Key{\Macro{geometry}}{layoutoffset=\PLength} dieser Bereich gegebenenfalls 
etwas eingerückt werden. Die Angabe von 
\Key{\Macro{geometry}}{showcrop=\PBoolean} generiert außerdem noch visuelle 
Schnittmarken.
\end{Entity}
%
\begin{quoting}[rightmargin=0pt]
\begin{Code}[escapechar=§]
\RequirePackage{fix-cm}
\documentclass[%
  paper=A1,
  fontsize=36pt
]{tudscrposter}
\usepackage[T1]{fontenc}
§\dots§
\geometry{paper=b1,layout=a1,layoutoffset=1in,showcrop}
\begin{document}
§\dots§
\end{document}
\end{Code}
\end{quoting}
%
Für genauere Erläuterungen sowie weitere Einstellmöglichkeiten sei auf die 
Dokumentation von \Package{crop} beziehungsweise \Package{geometry} verwiesen.
Mit der \TUDScript-Option \Option{bleedmargin}(\Class{tudscrposter}) können
zusätzlich die farbigen Bereiche der \PageStyle{tudheadings}"~Seitenstile 
erweitert werden, um ein \enquote{Zuschneiden in die Farbe} zu ermöglichen.



\section{%
  Platzierung von Gleitobjekten%
  \label{sec:tips:floats}%
  \index{Gleitobjekte!Platzierung|?}%
  \index{Satzspiegel!zweispaltig}%
}
Mit den beiden Paketen \Package{flafter} sowie \Package{placeins} gibt es die 
Möglichkeit, den für \hologo{LaTeX} zur Verfügung stehenden Raum für die 
Platzierung von Gleitobjekten einzuschränken. Darüber hinaus kann diese auch 
durch die im Folgenden aufgezählten Befehle beeinflusst werden, wobei zwischen 
einspaltigem und zweispaltigen Layout (\Option{twocolumn}(\Package{typearea})) 
unterschieden wird. Die nachfolgend erläuterten Makros lassen sich sehr 
einfach mit \Macro*{renewcommand*}[\MPName{Makro}\MPName{Wert}] ändern. 

\begin{Declaration}
  {\Macro*{floatpagefraction}}
  [0\floatpagefraction]
\begin{Declaration}
  {\Macro*{dblfloatpagefraction}}
  [0\dblfloatpagefraction]
\printdeclarationlist%
%
Der Wert gibt die relative Größe eines Gleitobjektes bezogen auf die Texthöhe 
(\Length*{textheight}) an, die mindestens erreicht sein muss, damit für dieses 
gegebenenfalls vor dem Beginn eines neuen Kapitels eine separate Seite erzeugt 
wird. Dabei wird einspaltiges (\Macro*{floatpagefraction}) und zweispaltiges 
(\Macro*{dblfloatpagefraction}) Layout unterschieden. Der Wert für beide 
Befehle sollte im Bereich von~\PValue{0.5\dots0.8} liegen.
\end{Declaration}
\end{Declaration}

\begin{Declaration}
  {\Macro*{topfraction}}
  [0\topfraction]
\begin{Declaration}
  {\Macro*{dbltopfraction}}
  [0\dbltopfraction]
\printdeclarationlist%
%
Diese Werte geben den maximalen Seitenanteil für Gleitobjekte an, die am oberen 
Seitenrand platziert werden. Er sollte größer als \Macro*{floatpagefraction} 
respektive \Macro*{dblfloatpagefraction} (zweispaltiges Layout) sein und im 
Bereich von \PValue{0.5\dots0.8} liegen. 
\end{Declaration}
\end{Declaration}

\begin{Declaration}
  {\Macro*{bottomfraction}}
  [0\bottomfraction]
\printdeclarationlist%
%
Dies ist der maximale Seitenanteil für Gleitobjekte, die am unteren Seitenrand 
platziert werden. Er sollte zwischen~\PValue{0.2} und~\PValue{0.5} betragen.
\end{Declaration}

\begin{Declaration}
  {\Macro*{textfraction}}
  [0\textfraction]
\printdeclarationlist%
%
Dies ist der notwendige Mindestanteil an Fließtext auf einer Seite mit 
Gleitobjekten, damit diese nicht auf einer separaten Seite ausgegeben werden. 
Er sollte im Bereich von~\PValue{0.1}\dots\PValue{0.3} liegen.
\end{Declaration}

\begin{Declaration}
  {\Length*{textfloatsep}}
  [\the\textfloatsep]
\begin{Declaration}
  {\Length*{dbltextfloatsep}}
  [\the\dbltextfloatsep]
\begin{Declaration}
  {\Length*{intextsep}}
  [\the\intextsep]
\begin{Declaration}
  {\Length*{floatsep}}
  [\the\floatsep]
\begin{Declaration}
  {\Length*{dblfloatsep}}
  [\the\dblfloatsep]
\printdeclarationlist%
%
Die Längen \Length*{textfloatsep} und \Length*{dbltextfloatsep} (zweispaltig) 
werden zwischen Fließtext und dem ersten Gleitobjekt am Ende beziehungsweise 
dem letzten am Anfang einer Seite eingefügt. Die Länge \Length*{intextsep} 
bestimmt den Abstand oberhalb und unterhalb von Gleitobjekten, wenn die Ausgabe 
innerhalb des Fließtextes forciert wird. Mehrere Gleitobjekte nacheinander 
werden mit \Length*{textfloatsep} respektive \Length*{dbltextfloatsep} 
(zweispaltig) voneinander separiert.
\end{Declaration}
\end{Declaration}
\end{Declaration}
\end{Declaration}
\end{Declaration}

\begin{Declaration}
  {\Counter*{totalnumber}}
  [\arabic{totalnumber}]
\begin{Declaration}
  {\Counter*{topnumber}}
  [\arabic{topnumber}]
\begin{Declaration}
  {\Counter*{dbltopnumber}}
  [\arabic{dbltopnumber}]
\begin{Declaration}
  {\Counter*{bottomnumber}}
  [\arabic{bottomnumber}]
\printdeclarationlist
%
Außerdem gibt es Zähler, welche die maximale Anzahl an Gleitobjekten pro Seite 
insgesamt (\Counter*{totalnumber}) sowie am oberen (\Counter*{topnumber}) 
und am unteren Seitenrand (\Counter*{bottomnumber}) sowie im zweispaltigen 
Satz beide Spalten überspannend (\Counter*{dbltopnumber}) festlegen. Die Werte 
können mit \Macro*{setcounter}[\MPName{Zähler}\MPName{Wert}] geändert werden.
\end{Declaration}
\end{Declaration}
\end{Declaration}
\end{Declaration}

\begin{Declaration}
  {\Length*{@fptop}}
\begin{Declaration}
  {\Length*{@fpsep}}
\begin{Declaration}
  {\Length*{@fpbot}}
\begin{Declaration}
  {\Length*{@dblfptop}}
\begin{Declaration}
  {\Length*{@dblfpsep}}
\begin{Declaration}
  {\Length*{@dblfpbot}}
\printdeclarationlist[%
  \index{Längen}%
]
Sind vor Beginn eines Kapitels noch Gleitobjekte verblieben, so werden diese 
von \hologo{LaTeX} normalerweise auf einer separaten Seite vertikal zentriert 
ausgegeben. Dabei bestimmen diese Längen den Abstand vor dem ersten Gleitobjekt 
zum oberen Seitenrand (\Length*{@fptop}, \Length*{@dblfptop}), zwischen den 
einzelnen Objekten (\Length*{@fpsep}, \Length*{@dblfpsep}) sowie zum unteren 
Seitenrand (\Length*{@fpbot}, \Length*{@dblfpbot}). Soll dies nicht 
geschehen, können die Längen durch den Anwender geändert werden.
\end{Declaration}
\end{Declaration}
\end{Declaration}
\end{Declaration}
\end{Declaration}
\end{Declaration}
%
\begin{Example}
Alle Gleitobjekte auf einer dafür speziell gesetzten Seite sollen direkt zu 
Beginn dieser ausgegeben werden. In der Dokumentpräambel lässt sich für dieses 
Unterfangen Folgendes nutzen:
\begin{Code}
\makeatletter
\setlength{\@fptop}{0pt}
\setlength{\@dblfptop}{0pt}% twocolumn
\makeatother
\end{Code}
\end{Example}




\section{%
  Fehlermeldung: ! No room for a new \textbackslash write%
  \label{sec:tips:write}%
}
%
\ChangedAt{v2.02:\TaT{Fehler beim Schreiben von Hilfsdateien}}%
Für das Erstellen und Schreiben externer Hilfsdateien steht \hologo{LaTeX} nur 
eine begrenzte Anzahl sogenannter Ausgabe-Streams zur Verfügung. Allein für 
jedes zu erstellende Verzeichnis reserviert \hologo{LaTeX} selbst jeweils einen 
neuen Stream. Auch einige bereits zuvor in diesem Handbuch vorgestellte, sehr 
hilfreiche Pakete~-- wie beispielsweise \Package{hyperref}, \Package{biblatex}, 
\Package{glossaries}, \Package{todonotes} oder auch \Package{filecontents}~-- 
benötigen eigene Hilfsdateien und öffnen für das Erstellen dieser einen 
Ausgabe-Stream oder mehr. Lädt der Anwender mehrere, in eine Hilfsdatei 
schreibende Pakete, kann es zur folgender Fehlermeldung kommen:
%
\begin{quoting}
\begin{Code}
! No room for a new \write .
\end{Code}
\end{quoting}
%
Abhilfe schafft das Paket \Package{morewrites}, welches die Ausgabe der Streams 
in eine Hilfsdatei umleitet. Es sollte möglichst frühzeitig innerhalb der 
Präambel geladen werden. Nur sehr selten ist die Verwendung des Paketes nicht 
von Erfolg gekrönt. 

Für diesen Fall lässt sich das Paket \Package{scrwfile}(\Bundle{koma-script}) 
als Alternative nutzen, welches einige Änderungen am \hologo{LaTeXe}"~Kernel 
vornimmt. Das Ziel dieses Paketes ist es, zumindest die Anzahl der benötigten 
Hilfsdateien für das Schreiben \emph{aller Verzeichnisse} zu reduzieren und 
somit nur einen einzigen Stream hierfür zu verwenden. Auch dieses sollte gleich 
zu Beginn der Präambel eingebunden werden. Der Entwicklungsstand des Paketes 
ist als experimentell eingestuft, auch wenn es meiner Erfahrung nach fehlerfrei 
arbeitet. Sollten dennoch Probleme auftreten, ist die Anleitung des Paketes im 
\scrguide zu finden. 



\section{%
  Warnung bei der Schriftgrößenwahl%
  \label{sec:tips:fontsize}%
}
%
\ChangedAt{v2.04:\TaT{Warnung bei der Schriftgrößenwahl}}%
Die im Dokument verwendete Schriftgröße lässt sich bei den \KOMAScript-Klassen 
einfach über die Option~\Option{fontsize}(\Bundle{koma-script}) einstellen, 
wobei diese \emph{immer} als Klassenoption anzugeben ist. Werden mit der Option 
relativ großen oder kleinen Schriftgrößen angegeben, kann dabei die folgende 
Warnung auftreten:
%
\begin{quoting}[rightmargin=0pt]
\begin{Code}[escapechar=§]
LaTeX Font Warning: Font shape `...' in size <xx> not available
\end{Code}
\end{quoting}
%
Dies liegt darin begründet, dass zum Zeitpunkt des Ladens einer Klasse zunächst 
immer nach den Standardschriften der Computer~Modern gesucht wird, unabhängig 
davon, ob im Nachhinein ein anderes Schriftpaket Anwendung findet. Die 
ursprünglichen Schriften der Computer~Modern sind de"~facto nicht auf alle 
Größen skalierbar. Um die genannten Warnungen zu beseitigen, sollte \emph{vor} 
der Dokumentklasse das Paket \Package{fix-cm} geladen werden:
%
\begin{quoting}[rightmargin=0pt]
\begin{Code}[escapechar=§]
\RequirePackage{fix-cm}
\documentclass§\OPName{Klassenoptionen}\MPName{Klasse}§
\usepackage[T1]{fontenc}
§\dots§
\begin{document}
§\dots§
\end{document}
\end{Code}
\end{quoting}



\section{Vermeiden des Skalierens einer PDF-Datei beim Druck}
%
\ChangedAt{v2.04:\TaT{Vermeiden des Skalierens einer PDF"~Datei beim Druck}}%
Beim Erzeugen eines Druckauftrages einer PDF"~Datei kann es unter Umständen 
dazu führen, dass diese durch den verwendeten PDF"~Betrachter unnötigerweise 
vorher skaliert wird und dabei die Seitenränder vergrößert werden. Um dieses 
Verhalten für Dokumente, die mit \Format{pdfLaTeX} erzeugt werden, zu 
unterdrücken, gibt es zwei Möglichkeiten:
%
\begin{enumerate}
\item Wenn im Dokument ohnehin das Paket \Package{hyperref} verwendet wird, 
  ist der simple Aufruf von 
  \Macro{hypersetup}[\MPValue{pdfprintscaling=None}](\Package{hyperref})
  ausreichend.
\item Der Low"~Level"~Befehl
  \Macro*{pdfcatalog}[\MPValue{/ViewerPreferences<{}</PrintScaling/None>{}>}]
  besitzt das gleiche Verhalten und lässt sich auch ohne \Package{hyperref} 
  nutzen.
\end{enumerate}
%
Weitere Informationen sind unter \url{http://www.komascript.de/node/1897} 
zu finden.





\setpartpreamble{\ToDo[doc]{installation und obsolete tauschen}[v2.07]}
\part{Anhang\label{part:appendix}}
\appendix
\chapter{%
  Weiterführende Installationshinweise%
  \label{sec:install:ext}%
  \index{Installation|(}%
}
\noindent\Attention{%
  Hier werden unterschiedliche Varianten erläutert, wie \TUDScript in der 
  Version~\vTUDScript{} genutzt werden kann, falls eine frühere Variante als 
  \textbf{lokale Nutzerinstallation} verwendet wurde.
}

\bigskip\noindent
Bis zur Version~v2.01 wurde \TUDScript ausschließlich über das \Forum zur 
lokalen Nutzerinstallation angeboten. In erster Linie hat das historische 
Hintergründe und hängt mit der Entstehungsgeschichte von \TUDScript zusammen. 
Eine lokale Nutzerinstallation bietet einen~-- eher zu vernachlässigenden~-- 
Vorteil. Treten bei der Verwendung von \TUDScript Probleme auf, können diese im 
Forum gemeldet und diskutiert werden. Ist für ein solches Problem tatsächlich 
eine Fehlerkorrektur respektive Aktualisierung von \TUDScript nötig, kann diese 
schnell und unkompliziert über das \GitHubRepo<releases/latest> bereitgestellt 
und durch den Anwender sofort genutzt werden.

Dies hat allerdings für alle Anwender, welche das Forum relativ wenig oder gar 
nicht besuchen, den großen Nachteil, dass Sie nicht von Aktualisierungen, 
Verbesserungen und Fehlerkorrekturen neuer Versionen profitieren können. Auch 
alle nachfolgenden Bugfixes und Aktualisierungen des \TUDScript-Bundles müssen 
durch den Anwender manuell durchgeführt werden. Daher wird die Verbreitung via 
\CTAN<pkg/tudscr> präferiert, sodass \TUDScript stets in der gerade aktuellen 
Version verfügbar ist~-- eine durch den Anwender aktuell gehaltene 
\DistributionGeneral vorausgesetzt. Der einzige Nachteil bei diesem Ansatz ist, 
dass die Verbreitung eines Bugfixes und die anschließende Bereitstellung durch 
die verwendete \DistributionGeneral für gewöhnlich bis zu zwei Tagen dauert.

Eine gängige \DistributionGeneral durchsucht im Regelfall zuerst das lokale 
\Path{texmf}"=Nutzerverzeichnis nach Klassen sowie Paketen und erst daran 
anschließend den \Path{texmf}"~Pfad der \DistributionGeneral selbst. Dabei 
spielt es keine Rolle, in welchem Pfad die neuere Version einer Klasse oder 
eines Paketes liegt. Sobald im Nutzerverzeichnis die gesuchte Datei gefunden 
wurde, wird die Suche beendet.
\Attention{%
  In der Konsequenz bedeutet dies, dass sämtliche Aktualisierungen über die 
  genutzte \DistributionGeneral \textbf{nicht} zum Tragen kommen, falls 
  \TUDScript als lokale Nutzerversion installiert wurde.
}

Deshalb wird Anwendern empfohlen, eine gegebenenfalls vorhandene lokale 
Nutzerinstallation von \TUDScript zu deinstallieren, falls diese nicht 
\emph{bewusst} installiert wurde. Das Vorgehen für eine Deinstallation wird in 
\autoref{sec:local:uninstall} erläutert. Nach dieser können Updates des 
\TUDScript-Bundles durch die Aktualisierungsfunktion der jeweils eingesetzten 
\DistributionGeneral erfolgen. 

Wie das \TUDScript-Bundle trotzdem als lokale Nutzerversion installiert oder 
aktualisiert werden kann, ist in \autoref{sec:local:install} beziehungsweise 
\autoref{sec:local:update} zu finden. Der Anwender sollte in diesem Fall 
allerdings genau wissen, was er damit bezweckt, da er in diesem Fall für die 
Aktualisierung von \TUDScript selbst verantwortlich ist.

\minisec{Nutzung der veralteten Schriftfamilien}
%
\ChangedAt{v2.06}%
Soll ein Dokument noch mit den veralteten Schriftfamilien \Univers und \DIN 
gesetzt werden, so ist eine lokale Installation der Type1-Schriften notwendig, 
welche in \autoref{sec:install:fonts} beschrieben wird. Zusätzlich sei auf die 
beiden Optionen \Option||{tudscrver=2.05} und \Option||{cdoldfont} hingewiesen. 
Für die Nutzung der \OpenSans wird lediglich das Paket \Package||{opensans} 
benötigt, welches über das \CTAN<pkg/opensans> bereitgestellt wird und über die 
verwendete \DistributionGeneral installiert werden kann.



\section{%
  Lokale Deinstallation des \TUDScript-Bundles%
  \label{sec:local:uninstall}%
}
%
Über die Kommandozeile beziehungsweise das Terminal kann mit
%
\begin{quoting}
\Path{kpsewhich --all tudscrbase.sty}
\end{quoting}
%
überprüft werden, ob eine lokale Nutzerinstallation von \TUDScript vorhanden 
ist. Es werden alle Pfade ausgegeben, in denen die gesuchte Datei gefunden 
wird. Erscheint nur der Pfad der \DistributionGeneral, ist die 
\TUDScript-Version selbiger aktiv und der Anwender kann mit dem 
\TUDScript-Bundle arbeiten.

Wird \emph{nur} das lokale Nutzerverzeichnis oder gar kein Verzeichnis 
gefunden, so wird wahrscheinlich eine veraltete \DistributionGeneral verwendet. 
In diesem Fall wird eine Aktualisierung dieser \emph{unbedingt} empfohlen. 
Sollte dies nicht möglich sein, \emph{muss} \TUDScript als lokale Nutzerversion 
installiert (\autoref{sec:local:install}) beziehungsweise~-- falls ein Pfad 
ausgegeben wurde~-- aktualisiert (\autoref{sec:local:update}) werden.

Sollte neben dem Pfad der \DistributionGeneral noch mindestens ein weiterer 
Pfad angezeigt werden, so ist eine lokale Nutzerversion installiert. In diesem 
Fall hat der Anwender drei Möglichkeiten:
%
\begin{enumerate}
\item Entfernen der lokalen Nutzerinstallation (skriptbasiert)
\item Entfernen der lokalen Nutzerinstallation (manuell)
\item Aktualisierung der lokalen Nutzerversion (\autoref{sec:local:update})
\end{enumerate}
%
Um die lokale Nutzerinstallation zu entfernen, kann für Windows
\GitHubDownload*<uninstall>{tudscr_uninstall.bat} sowie für unixartige 
Betriebssysteme \GitHubDownload*<uninstall>{tudscr_uninstall.sh} verwendet 
werden. Nach der Ausführung des jeweiligen Skriptes kann mit dem zu Beginn 
gezeigten Aufruf in der Kommandozeile respektive Terminal geprüft werden, ob 
die Deinstallation erfolgreich war. Wird immer noch mindestens ein lokaler Pfad 
ausgegeben, sollte \TUDScript manuelle deinstalliert werden, was nachfolgend 
beschrieben wird.

Nur die Deinstallation aller lokalen Nutzerinstallationen von \TUDScript 
ermöglicht die Verwendung der jeweils aktuellen Version über die 
\DistributionGeneral. Hierfür ist~-- unter der Annahme, dass das automatisierte 
Deinstallieren mithilfe der zuvor genannten Skripte zur Deinstallation nicht 
erfolgreich war~-- etwas Handarbeit durch den Anwender vonnöten. Der in der 
Kommandozeile respektive im Terminal mit
%
\begin{quoting}
\Path{kpsewhich --all tudscrbase.sty}
\end{quoting}
%
gefundene~-- zum Ordner der \DistributionGeneral \emph{zusätzliche}~-- 
Pfad hat die folgende Struktur:
%
\begin{quoting}
\Path{\PName{Installationspfad}/tex/latex/tudscr/tudscrbase.sty}
\end{quoting}
%
Um die Nutzerinstallation vollständig zu entfernen, muss als erstes zu 
\Path{\PName{Installationspfad}} navigiert werden. Anschließend ist in diesem 
Pfad Folgendes durchzuführen:
%
\settowidth\tudscrdim{\Path{tex/latex/tudscr/}~}%
\begin{description}[labelwidth=\tudscrdim,labelsep=.5em]
\item[\Path{tex/latex/tudscr/}]\Path{.cls}- und \Path{.sty}"~Dateien löschen
\item[\Path{tex/latex/tudscr/}]Ordner \Path{logo} vollständig löschen
\item[\Path{doc/latex/}] Ordner \Path{tudscr} vollständig löschen
\item[\Path{source/latex/}] Ordner \Path{tudscr} vollständig löschen
\end{description}
%
Zum Abschluss ist in der Kommandozeile beziehungsweise im Terminal der Befehl 
\Path{texhash} aufzurufen. Damit wurde die lokale Nutzerversion entfernt und es 
wird von nun an die Version von \TUDScript genutzt, welche durch die verwendete 
\DistributionGeneral bereitgestellt wird.



\section{%
  Lokale Installation des \TUDScript-Bundles%
  \label{sec:local:install}%
  \index{Installation!Nutzerinstallation|(}%
}
%
\Attention{%
  Eine lokale Nutzerinstallation des \TUDScript-Bundles sollte ausschließlich 
  durch Anwender ausgeführt werden, die genau wissen, aus welchen Gründen dies 
  geschehen soll. Zuvor sollte die \DistributionGeneral aktualisiert werden, um 
  Probleme beim Installationsprozess zu vermeiden.
}

Dazu werden die Archive \GitHubDownload*{TUD-Script_\vTUDScript_Windows.zip} 
für Windows beziehungsweise \GitHubDownload*{TUD-Script_\vTUDScript_Unix.zip} 
für unixoide Betriebssysteme bereitgestellt. Bei der Ausführung des jeweiligen 
Installationsskriptes aus dem Archiv werden alle Dateien in das lokale 
Nutzerverzeichnis der jeweiligen \DistributionGeneral installiert. 

Alternativ zur Nutzung der bereitgestellten Installationsskripte kann auch der 
Inhalt des TDS"~Archivs \GitHubDownload*{tudscr_\vTUDScript.zip} in das lokale 
\Path{texmf}"=Nutzerverzeichnis beziehungsweise in ein anderes der 
\DistributionGeneral bekanntes TDS"~Verzeichnis kopiert und abschließend in der 
Kommandozeile respektive im Terminal \Path{texhash} aufgerufen werden. 



\section{%
  Lokales Update des \TUDScript-Bundles%
  \label{sec:local:update}%
}
\subsection{Update des \TUDScript-Bundles ab Version~v2.02}
%
Ein Update und die lokale Installation unterscheiden sich ab der Version~v2.06 
nicht voneinander, das Vorgehen ist absolut identisch zu der Beschreibung in 
\autoref{sec:local:install}.
\Attention{%
  Die lokale Aktualisierung auf Version~\vTUDScript{} funktioniert allerdings 
  nur, wenn \TUDScript bereits mindestens in der Version~v2.02 entweder als 
  lokale Nutzerversion oder über die \DistributionGeneral installiert ist.%
}



\subsection{Update des \TUDScript-Bundles ab Version~v2.00}
%
Mit der Version~v2.02 gab es einige tiefgreifende Änderungen. Deshalb wird für 
vorausgehende Versionen~-- sprich v2.00 und v2.01~-- kein dediziertes Update 
angeboten. Die Aktualisierung kann durch den Anwender entweder~-- wie in 
\autoref{sec:local:install} erläutert~-- mit einer skriptbasierten oder mit 
einer manuellen Neuinstallation erfolgen.%
\index{Installation!Nutzerinstallation|)}%



\subsection{Update des \TUDScript-Bundles von Version v1.0}
%
Ist \TUDScript in der veralteten \emph{Version~v1.0} installiert, so wird vor 
der Aktualisierung dringlichst zu einem vollständigen Entfernen dieser Version 
geraten. Andernfalls werden nach einem Update bei der Verwendung massive 
Probleme und Fehler auftreten. Zur Deinstallation werden die Skripte 
\GitHubDownload*<uninstall>{tudscr_uninstall.bat} respektive
\GitHubDownload*<uninstall>{tudscr_uninstall.sh} bereitgestellt. Die aktuelle 
Version~\vTUDScript{} kann nach der vollständigen Deinstallation aller 
veralteten Versionen wie in \autoref{sec:local:install} beschrieben installiert 
werden.

Im Vergleich zur \emph{Version~v1.0} hat sich an der Benutzerschnittstelle 
nicht sehr viel verändert, ein Umstieg auf die Version~\vTUDScript{} dürfte 
keine Schwierigkeiten bereiten. Treten danach dennoch Probleme auf, sollte der 
Anwender als erstes die Beschreibung des Paketes \Package||'full'{tudscrcomp} 
lesen, welches eine Schnittstelle zur Nutzung alter und ursprünglich nicht mehr 
vorgesehener Befehle sowie Optionen bereitstellt. Einige dieser sind jedoch 
obsolet und werden nicht mehr bereitgestellt. Aufgeführt sind diese in 
\autoref{sec:cessations}. Sollten trotz aller Hinweise dennoch Fehler oder 
Probleme beim Umstieg auf die neue \TUDScript-Version auftreten, ist eine 
Meldung im \Forum die beste Möglichkeit, um Hilfe zu erhalten.



\section{%
  Installation veralteter Schriftfamilien%
  \label{sec:install:fonts}%
  \index{Installation!Schriftinstallation|(}%
}
%
\ChangedAt{%
  v2.02:Installationsroutine der Type1-Schriften angepasst;%
  v2.04:Installationsskripte verbessert und robuster gestaltet sowie für 
        portable Distributionen \TeXLive' und \MiKTeX' erweitert;%
  v2.06:\OpenSans erfordert \emph{keine} Schriftinstallation;%
}%
\ToDo{komplett raus?}[v2.07]
Bis Anfang des Jahres~2018 nutzte das \TUDCD als Hausschrift nicht \OpenSans 
sondern die Schriftfamilien \Univers und \DIN. Diese lassen sich weiterhin mit 
\TUDScript verwenden, um alte Dokumente kompilieren zu können. Hierfür sei auf 
die Optionen \Option||{tudscrver=2.05} und \Option||{cdoldfont} hingewiesen. 
Da es sich bei diesen um lizenzierte Schriften handelt, müssen diese beim 
Universitätsmarketing auf \href{https://tu-dresden.de/cd}{Anfrage} mit dem 
Hinweis auf die Verwendung von \hologo{LaTeX} bestellt und nach Erhalt der 
notwendigen Archive \File*{Univers_PS.zip} und \File*{DIN_Bd_PS.zip} für 
Windows (\autoref{sec:install:win}) beziehungsweise unixoide Betriebssysteme 
(\autoref{sec:install:unix}) installiert werden. Ohne einen triftigen Grund 
sollte jedoch in jedem Fall die \OpenSans genutzt werden, insbesondere für neu 
erstellte Dokumente.

Das \TUDScript-Bundle unterstützt besagte Schriften auch im OpenType-Format, 
welche ebenfalls über das Universitätsmarketing auf 
\href{https://tu-dresden.de/cd}{Anfrage} bestellt werden müssen. Die in den 
Archiven \File*{Univers_8_TTF.zip} und \File*{DIN_TTF.zip} enthaltenen 
Schriften lassen sich~-- sobald diese für das Betriebssystem installiert 
wurden~-- mit dem Paket \Package||{fontspec} verwenden. In \fullref{sec:fonts} 
sind weitere Hinweise zur Verwendung dieses Paketes zu finden.

Im \GitHubRepo<releases> sind die zur Schriftinstallation nötigen 
\GitHubRepo[Skripte für \TUDScript]<releases/fonts> ebenso zu finden wie die
\GitHubRepo[Skripte für das Bundle von Klaus~Bergmann]<releases/oldfonts>. Die 
unterschiedlichen Installationsskripte begründen sich insbesondere dadurch, 
dass bei der Installation für das \TUDScript-Bundle sowohl die Metriken als 
auch das Kerning der Schriften für Fließtext und den Mathematikmodus angepasst 
werden. Sollen die so verbesserten Schriften für die Klassen von Klaus~Bergmann 
verwendet werden, kann dies mit dem Paket \Package||{fix-tudscrfonts} erfolgen, 
was allerdings das Ergebnis der erzeugten Ausgabe beeinflusst, weshalb die 
Installationsskripte in unterschiedlichen Varianten weiterhin vorgehalten 
werden.

\minisec{Notwendige Pakete und Skripte für die Schriftinstallation}
%
Zur Schriftinstallation sind zum einen die Pakete \Package*{fontinst} sowie
\Package*{cmbright}, \Package*{hfbright}, \Package*{cm-super} und
\Package*{iwona} von \emph{essentieller} Bedeutung und daher \emph{zwingend} 
notwendig. Zum anderen werden die Skripte \Path{tftopl}, \Path{pltotf} und 
\Path{vptovf} benötigt, welche bei \TeXLive oder \MacTeX über 
\Bundle*{fontware} respektive bei \MiKTeX über \Bundle*{miktex-texware} sowie 
\Bundle*{miktex-fonts} bereitgestellt werden. 

\minisec{Anmerkung zu \hologo{MiKTeX}}
%
Vor der Installation der Schriften für \TUDScript sollte unbedingt ein Update 
von \MiKTeX durchgeführt werden. Außerdem ist es sehr ratsam, die Installation 
in der Mehrbenutzervariante mit Administratorrechten durchzuführen, da die 
Einzelbenutzervariante relativ unregelmäßig und nicht immer nachvollziehbar zu 
Problemen führen kann. Weiterhin sollten die zuvor genannten, zusätzlich 
notwendigen Pakete und Skripte~-- falls nicht vorhanden~-- über den 
\MiKTeX"=Paketmanager hinzugefügt werden.

Das Installationsskript scheitert außerdem bei einigen Anwendern~-- aufgrund 
eingeschränkter Nutzerzugriffsrechte~-- beim Eintragen der Schriften in die 
Map"~Datei. Dies muss gegebenenfalls durch den Anwender über die Kommandozeile 
%
\begin{quoting}
\Path{initexmf -{}-edit-config-file updmap}
\end{quoting}
%
erfolgen. In der sich öffnenden Datei sollte sich der Eintrag 
%
\begin{quoting}
\Path{Map~tudscr.map}
\end{quoting}
%
befinden. Ist dies nicht der Fall, muss diese Zeile manuell eingetragen und die 
Datei anschließend gespeichert werden. Danach ist in der Kommandozeile noch 
folgenden Aufruf auszuführen:
%
\begin{quoting}
\Path{initexmf~-{}-mkmaps}
\end{quoting}


\minisec{Anmerkung zu \hologo{TeX}~Live und Mac\hologo{TeX}}
%
Sollte keine Vollinstallation von \TeXLive durchgeführt worden sein, müssen die 
zuvor genannten, erforderlichen Pakete und Skripte zur Schriftinstallation über 
den \TeXLive"=Paketmanager manuell hinzugefügt werden.

Sind nach einem fehlerfreien Durchlauf des Installationsskriptes die Schriften 
dennoch nicht verfügbar, so muss die Synchronisierung aller Schriftdateien 
angestoßen werden. Daran anschließend müssen die Map"~Datei und die 
dazugehörigen Schriftdateien registriert werden. Die hierfür notwendigen 
Aufrufe lauten:
%
\begin{quoting}
\Path{updmap-sys -{}-syncwithtrees}\newline
\Path{updmap-sys -{}-enable Map=tudscr.map}\newline
\Path{updmap-sys -{}-force}
\end{quoting}
%
Sind die Schriften danach immer noch nicht verfügbar, so wurden bestimmt schon 
weitere Schriften auf dem System \emph{lokal} installiert. In diesem Fall 
sollte der Vorgang nochmals für eine lokale Schriftinstallation mit 
%
\begin{quoting}
\Path{updmap -{}-syncwithtrees}\newline
\Path{updmap -{}-enable Map=tudscr.map}\newline
\Path{updmap -{}-force}
\end{quoting}
%
ausgeführt werden. 

\Attention{%
  Dadurch wird allerdings der Befehl \Path{updmap-sys} von nun an wirkungslos.
}
Nach einer systemweiten Installation neuer Schriften~-- beispielsweise bei der 
Aktualisierung der \DistributionGeneral~-- müssen diese über den manuellen 
Aufruf von \Path{updmap} zukünftig durch den Anwender lokal bei 
\TeXLive respektive \MacTeX registriert werden.



\subsection{%
  Installation der Type1-Schriften unter Windows%
  \label{sec:install:win}%
}
%
Zur Installation der Schriften des \CDs für das \TUDScript-Bundle ist das 
Archiv \GitHubDownload*<fonts>{TUD-Script_fonts_Windows.zip} vorgesehen. Dieses 
ist sowohl für \TeXLive als auch \MiKTeX nutzbar und enthält~-- bis auf die 
jeweiligen Schriftarchive selbst~-- alle benötigten Dateien. Diese sollten nach 
dem Entpacken des Archivs in das gleiche Verzeichnis kopiert werden. Vor der 
Verwendung des Skriptes \File*{tudscr_fonts_install.bat} ist sicherzustellen, 
dass sich \emph{alle} der folgenden Dateien im selben Verzeichnis befinden:
%
\settowidth\tudscrdim{\File*{tudscr_fonts_install.zip}~}%
\begin{description}[labelwidth=\tudscrdim,labelsep=.5em]
  \item[\File*{tudscr_fonts_install.bat}]Installationsskript
  \item[\File*{Univers_PS.zip}]Archiv mit Schriftdateien für \Univers
  \item[\File*{DIN_Bd_PS.zip}]Archiv mit Schriftdateien für \DIN
  \item[\File*{tudscr_fonts_install.zip}]Archiv mit Metriken für die
    Schriftinstallation via \Package*{fontinst}
\end{description}
%
Beim Ausführen des Installationsskriptes werden alle Schriften standardmäßig in 
ein lokales Nutzerverzeichnis installiert. Wird das Skript über das Kontextmenü 
mit Administratorrechten ausgeführt, erfolgt die Installation in einem Pfad, 
der \emph{für alle Nutzer} gültig und lesbar ist.



\subsection{%
  Installation der Type1-Schriften unter Linux und OS~X%
  \label{sec:install:unix}%
}
%
Zur Installation der Schriften des \CDs für das \TUDScript-Bundle ist das 
Archiv \GitHubDownload*<fonts>{TUD-Script_fonts_Unix.zip} vorgesehen. Dieses 
ist sowohl für \TeXLive als auch \MacTeX nutzbar und enthält~-- bis auf die 
jeweiligen Schriftdateien selbst~-- alle benötigten Dateien. Diese sollten nach 
dem Entpacken des Archivs in das gleiche Verzeichnis kopiert werden. Vor der 
Verwendung des Skriptes \File*{tudscr_fonts_install.sh} ist sicherzustellen, 
dass sich \emph{alle} der folgenden Dateien im selben Verzeichnis befinden:
%
\settowidth\tudscrdim{\File*{tudscr_fonts_install.zip}~}%
\begin{description}[labelwidth=\tudscrdim,labelsep=.5em]
  \item[\File*{tudscr_fonts_install.sh}]Installationsskript
    (Terminal: \Path{bash\:tudscr_fonts_install.sh})
  \item[\File*{Univers_PS.zip}]Archiv mit Schriftdateien für \Univers
  \item[\File*{DIN_Bd_PS.zip}]Archiv mit Schriftdateien für \DIN
  \item[\File*{tudscr_fonts_install.zip}]Archiv mit Metriken für die
    Schriftinstallation via \Package*{fontinst}
\end{description}
%
\minisec{Anmerkung zu Linux und OS~X}
%
\Attention{%
  Nach dem Entpacken eines Release-Archivs im passenden Pfad\footnote{%
    beispielsweise \Path{cd~"\$HOME/Downloads/\PName{Unterordner}"}%
  } \textbf{muss das Skript zwingend} mit \Path{bash~\PName{Skript}.sh} im 
  Terminal in diesem Pfad mit den benötigten Dateien aufgerufen werden.
}
Dabei werden alle Schriften standardmäßig in das lokale Nutzerverzeichnis 
(\Path{\$TEXMFHOME}) installiert. Wird das Skript mit \Path{sudo} verwendet, 
erfolgt die Installation \emph{für alle Nutzer} in den lokalen Systempfad 
(\Path{\$TEXMFLOCAL}).

Es ist unbedingt darauf zu Achten, das beim Ausführen des Skriptes das Terminal 
im richtigen Verzeichnis aktiv ist. Bei den meisten unixoiden Betriebssystemen 
ist es problemlos möglich, das Terminal aus der Benutzeroberfläche heraus über 
das Kontextmenü im gewünschten Pfad zu öffnen. Geht dies nicht, so muss nach 
dem Öffnen des Terminals mit dem Befehl \Path{cd} erst zum entsprechenden 
Pfad~-- exemplarisch \Path{cd~"\$HOME/Downloads/\PName{Unterordner}"}~-- 
navigiert werden. Ein beispielhafter Aufruf im Terminal könnte also lauten:
%
\begin{quoting}
\Path{cd~"\$HOME/Downloads/TUD-Script_fonts_Unix"{}\,\OPValue{ENTER}}\newline
\Path{bash tudscr_fonts_install.sh\,\OPValue{ENTER}}
\end{quoting}



\subsection{Probleme bei der Installation der Type1-Schriften}
%
Wird Windows verwendet, kann es unter Umständen vorkommen, dass notwendige 
Befehlsaufrufe für das Installationsskript nicht ausgeführt werden können. In 
diesem Fall ist der Pfad zu den benötigten Dateien, welche normalerweise unter 
\Path{\%SystemRoot\%\textbackslash System32} zu finden sind, nicht in der 
Umgebungsvariable \Path{PATH} enthalten. Einen Hinweis zur Problemlösung ist 
in diesem \Forum[Beitrag im \TUDForum]<359> zu finden.

Treten bei der Installation wider Erwarten Probleme auf, so ist zur Lösung eine 
Logdatei zu erstellen. Hierfür sollte unter \textbf{Windows} das Skript, 
welches Probleme verursacht, \emph{nicht} aus der Kommandozeile oder dem 
Explorer heraus sondern über \emph{Windows PowerShell} ausgeführt werden. 
Hierfür ist die Eingabe von \enquote{PowerShell} im Startmenü von Windows mit 
einem nachfolgenden Öffnen mittels \OPValue{ENTER}"~Taste ausreichend. 
Danach muss mit \Path{cd} zum Ordner des Skriptes navigiert und dieses mit 
\Path{.\textbackslash\PName{Skript}.bat|Tee-Object -file \PName{Skript}.log} 
ausgeführt werden. Ein Aufruf aus der PowerShell"~Konsole könnte lauten:
%
\begin{quoting}[rightmargin=0pt]
\Path{%
  cd~"\$env:USERPROFILE\textbackslash{}Downloads\textbackslash{}%
  TUD-Script_fonts_Windows"{}\,\OPValue{ENTER}%
}\newline%
\Path{%
  .\textbackslash{}tudscr_fonts_install.bat%
  |Tee-Object -file fonts_install.log\,\OPValue{ENTER}%
}%
\end{quoting}
%
Für \textbf{unixartige Systeme} ist der Aufruf 
\Path{bash \PName{Skript}.sh > \PName{Skript}.log} aus dem Terminal heraus zu 
verwenden. Ein exemplarische Verwendung könnte lauten:
%
\begin{quoting}
\Path{cd~"\$HOME/Downloads/TUD-Script_fonts_Unix"{}\,\OPValue{ENTER}}\newline
\Path{bash tudscr_fonts_install.sh > fonts_install.log\,\OPValue{ENTER}}%
\end{quoting}
%
Die so erstellte Logdatei kann \emph{mit einer kurzen Fehlerbeschreibung} 
entweder im \Forum gepostet oder per E"~Mail an \mailto{\TUDScriptContact}
gesendet werden.


\subsection{%
  Installationshinweise für portable Installationen%
  \label{sec:install:portable}%
}
%
Prinzipiell ist die Installation der Type1-Schriften des \CDs bei der Nutzung 
von \TeXLive' respektive \MiKTeX' äquivalent zur nicht"~portablen Variante, 
welche in \autoref{sec:install:fonts} beschrieben wird. Alle dort gegebenen 
Hinweise sollten sorgfältig berücksichtigt werden. Der durch das jeweilige 
Installationsskript voreingestellte Installationspfad sollte für gewöhnlich 
nicht geändert werden. Geschieht dies dennoch, so sollte dieser sich 
logischerweise auf dem externen Speichermedium 
\Path{\PName{Laufwerksbuchstabe}:\textbackslash} befinden.

\minisec{\TeXLive'}
%
Das folgende Vorgehen wurde mit Windows getestet. Empfehlungen für die portable 
Installation für unixoide Betriebssysteme können an \mailto{\TUDScriptContact} 
gesendet werden.
\begin{enumerate}
\item Installation von \TeXLive' in 
  \Path{\PName{Laufwerksbuchstabe}:\textbackslash LaTeX\textbackslash texlive}
\item Die Datei \Path{tl-tray-menu.exe} im Installationspfad öffnen
\item Das Kontextmenü von \TeXLive' mit einem Rechtsklick auf das entsprechende 
  Symbol im Infobereich der Taskleiste öffnen und entweder über die grafische 
  Oberfläche (\emph{Package~Manager}) oder die Kommandozeile 
  (\emph{Command~Prompt}) ein Update durchführen
\item Über das Kontextmenü die Kommandozeile ausführen und in dieser das Skript 
  für die Installation der Schriften \File*{tudscr_fonts_install.bat} 
  starten. Dabei gegebenenfalls zuvor in den Pfad des Skriptes 
  wechseln~-- exemplarisch:
  \begin{quoting}[leftmargin=1.5em,rightmargin=0pt]
  \Path{%
    cd~/d~\%USERPROFILE\%\textbackslash{}Downloads%
    \textbackslash{}TUD-Script_fonts_Windows\,\OPValue{ENTER}
  }\newline%
  \Path{tudscr_fonts_install.bat}\,\OPValue{ENTER}
  \end{quoting}
  Unter Umständen meldet das Skript fehlende Pakete. Dieses müssen über durch 
  den Anwender über den \emph{\hologo{TeX}~Live~Manager} installiert werden.
  \Attention{%
    Ein Ausführen ohne die über \TeXLive' geöffnete Kommandozeile führt zu 
    Fehlern.
  }%
\end{enumerate}

\minisec{\MiKTeX'}
%
\begin{enumerate}
\item Installation von \MiKTeX' in 
  \Path{\PName{Laufwerksbuchstabe}:\textbackslash LaTeX\textbackslash MiKTeX}%
  \footnote{%
    Der Pfad darf \emph{nicht} auf der obersten Verzeichnisebene 
    \Path{\PName{Laufwerksbuchstabe}:\textbackslash} liegen.
  }%
\item Die Datei \Path{miktex-portable.cmd} im Installationspfad öffnen
\item Das Kontextmenü von \MiKTeX' mit einem Rechtsklick auf das entsprechende 
  Symbol im Infobereich der Taskleiste öffnen und ein Update durchführen
\item Über das Kontextmenü die Kommandozeile ausführen und in dieser das Skript 
  für die Installation der Schriften \File*{tudscr_fonts_install.bat} starten.
  Dabei gegebenenfalls zuvor in den Pfad des Skriptes wechseln~-- exemplarisch:
  \begin{quoting}[leftmargin=1.5em,rightmargin=0pt]
  \Path{%
    cd~/d~\%USERPROFILE\%\textbackslash{}Downloads%
    \textbackslash{}TUD-Script_fonts_Windows\,\OPValue{ENTER}
  }\newline
  \Path{tudscr_fonts_install.bat}\,\OPValue{ENTER}
  \end{quoting}
  Bei diesem Schritt werden möglicherweise die Pakete \Package*{fontinst}, 
  \Package*{cmbright} und \Package*{iwona} automatisch nachinstalliert.
  \Attention{%
    Ein Ausführen ohne die über \MiKTeX' geöffnete Kommandozeile führt zu 
    Fehlern.
  }%
\item Bei der erstmaligen Verwendung von \TUDScript werden alle benötigten 
  Pakete von \MiKTeX' installiert, falls die automatische Nachinstallation 
  aktiviert ist und diese noch nicht sind. Dies betrifft die Bundle 
  \Bundle{tudscr} und \Bundle{koma-script} als auch die Pakete 
  \Package||{etoolbox}, \Package||{xpatch}, \Package||{trimspaces} und
  \Package||{xcolor} sowie \Package*{mptopdf}.
\end{enumerate}
\index{Installation!Schriftinstallation|)}%
\index{Installation|)}%

\chapter{%
  Obsolete sowie vollständig entfernte Optionen und Befehle%
  \label{sec:obsolete}%
}
\section{%
  Veraltete Optionen und Befehle in \TUDScript%
  \index{Änderungen|!}%
  \index{Kompatibilität}%
}
Einige Optionen und Befehle waren während der Weiterentwicklung von \TUDScript
in ihrer ursprünglichen Form nicht mehr umsetzbar oder wurden~-- unter anderem 
aus Gründen der Kompatibilität zu anderen Paketen~-- schlichtweg verworfen. 
Dennoch besteht für die meisten entfallenen Direktiven eine Möglichkeit, deren 
Funktionalität ohne größere Aufwände mit \TUDScript in der aktuellen Version 
\vTUDScript{} darzustellen. Ist dies der Fall, wird hier entsprechend kurz 
darauf hingewiesen.

\NewDocumentCommand\ChangesTo{o m}{%
  \subsection{%
    Änderungen für \TUDScript~#2%
    \label{sec:obsolete:#2}%
    \index[changelog]{%
      #2!Allgemein@Allgemein: \hyperref[{sec:obsolete:#2}]{%
        Änderungen gegenüber der vorhergehenden Version%
      }%
    }%
    \IfValueT{#1}{\ChangedAt*{#2:#1}}%
  }%
}


\ChangesTo{v2.00}
%
\begin{Obsolete}{v2.00}[\Option{cd=\PMisc}]{\Option{cd=alternative}}
\begin{Obsolete}{v2.00}[\Option{cdtitle=\PMisc}]{\Option{cdtitle=alternative}}
\begin{Obsolete}{v2.00}{\Length{titlecolwidth}}
\begin{Obsolete}{v2.00}{\Term{authortext}}
\printobsoletelist%
%
\ToDo[doc]{keine Links im Index}[v2.07]%
Die alternative Titelseite ist komplett aus dem \TUDScript-Bundle entfernt 
worden. Dementsprechend entfallen auch die dazugehörigen Optionen sowie Länge 
und Bezeichner.
\end{Obsolete}
\end{Obsolete}
\end{Obsolete}
\end{Obsolete}

\begin{Obsolete}{v2.00:\Option{cd}}{\Option{color=\PBoolean}}
\printobsoletelist%
%
Die Einstellungen der farbigen Ausprägung des Dokumentes erfolgt über die 
Option \Option*{cd}.
\end{Obsolete}

\begin{Obsolete}{v2.00:\Option{cdfont}}{\Option{tudfonts=\PBoolean}}
\printobsoletelist%
%
Die Option zur Schrifteinstellung ist wesentlich erweitert worden. Aus Gründen 
der Konsistenz wurde diese umbenannt.
\end{Obsolete}

\begin{Obsolete}{v2.00:\Option{cdfoot}}{\Option{tudfoot=\PBoolean}}
\printobsoletelist%
%
Ebenso wurde die Option für den Seitenfuß umbenannt, um dem Namensschema der 
restlichen Optionen von \TUDScript zu entsprechen.
\end{Obsolete}

\begin{Obsolete}{v2.00}{\Option{headfoot=\PMisc}}{%
  \seeref{\KOMAScript-Optionen \Option*{headinclude} und \Option*{footinclude}}%
}
\printobsoletelist%
%
Diese Option war für \TUDScript in der \emph{Version~v1.0} notwendig, um die 
parallele Verwendung der Pakete \Package*{typearea} und \Package*{geometry} zu 
ermöglichen. Die Erstellung des Satzspiegels wurde komplett überarbeitet. 
Mittlerweile werden an das Paket \Package*{geometry} die Einstellungen der 
\KOMAScript-Optionen \Option*{headinclude}(\Package{typearea}) und 
\Option*{footinclude}(\Package{typearea}) direkt weitergereicht, sodass diese 
Option nicht mehr notwendig ist und deshalb entfernt wurde.
\end{Obsolete}

\begin{Obsolete}{v2.00:\Option{cleardoublespecialpage}}{%
  \Option{partclear=\PBoolean}%
}
\begin{Obsolete}{v2.00:\Option{cleardoublespecialpage}}{%
  \Option{chapterclear=\PBoolean}%
}
\printobsoletelist%
%
Beide Optionen sind in der neuen Option \Option*{cleardoublespecialpage} 
aufgegangen, womit ein konsistentes Layout erreicht wird. Die ursprünglichen 
Optionen entfallen. 
\end{Obsolete}
\end{Obsolete}

\begin{Obsolete}{v2.00:\Option{abstract}}{\Option{abstracttotoc=\PBoolean}}
\begin{Obsolete}{v2.00:\Option{abstract}}{\Option{abstractdouble=\PBoolean}}
\printobsoletelist%
%
Beide Optionen wurden in die Option \Option*{abstract} integriert und sind 
deshalb überflüssig.
\end{Obsolete}
\end{Obsolete}

\begin{Obsolete}{v2.00:\Macro{headlogo}}{%
  \Macro{logofile}[\MPName{Dateiname}]%
}
\printobsoletelist%
%
Dieser Befehl wurde in \Macro*{headlogo} umbenannt, wobei die Funktionalität 
weiterhin bestehen bleibt.
\end{Obsolete}

\begin{Obsolete}{v2.00:\Option{tudbookmarks}}{\Option{bookmarks=\PBoolean}}
\printobsoletelist%
%
Die Option wurde umbenannt, um Überschneidungen mit \Package*{hyperref} zu 
vermeiden.
\end{Obsolete}

\begin{Obsolete}{v2.00}{\Length{signatureheight}}
\printobsoletelist%
%
Die Höhe für die Zeile der Unterschriften wurde dehnbar gestaltet, eine etwaige 
Anpassung durch den Anwender ist nicht vonnöten.
\end{Obsolete}

\begin{Obsolete}{v2.00:\Macro{titledelimiter}}{\Term{titlecoldelim}}%
\printobsoletelist%
%
Das Trennzeichen für Bezeichnungen beziehungsweise beschreibende Texte und dem 
eigentlichen Feld auf der Titelseite ist nicht mehr sprachabhängig und wurde 
umbenannt.
\end{Obsolete}

\begin{Obsolete}{v2.00:\Macro{declaration}}{\Macro{confirmationandrestriction}}
\begin{Obsolete}{v2.00:\Macro{declaration}}{\Macro{restrictionandconfirmation}}
\printobsoletelist%
%
Die beiden Befehle entfallen, stattdessen sollte entweder der Befehl 
\Macro*{declaration} oder die Umgebung \Environment*{declarations} zusammen mit 
den Befehlen \Macro*{confirmation} und \Macro*{blocking} verwendet werden, 
wobei sich diese in der Umgebung in beliebiger Reihenfolge anordnen lassen.
\end{Obsolete}
\end{Obsolete}

\begin{Obsolete}{v2.00:\Macro{place}}{\Macro{location}[\MPName{Ort}]}
\printobsoletelist%
%
In Anlehnung an andere \hologo{LaTeX}"~Pakete und "~Klassen wurde 
dieser Befehl in \Macro*{place} umbenannt.
\end{Obsolete}

\minisec{\taskname}
\begin{Bundle}{\Package{tudscrsupervisor}}
Die Umgebung für die Erstellung einer Aufgabenstellung für eine 
wissenschaftliche Arbeit wurde in das Paket \Package{tudscrsupervisor} 
ausgelagert. Dieses muss für die Verwendung der Umgebung \Environment*{task} 
und der daraus abgeleiteten standardisierten Form zwingend geladen werden.

\begin{Obsolete}{v2.00:\Environment{task}}{\Option{cdtask=\PMisc}}
\begin{Obsolete}{v2.00}{\Option{taskcompact=\PBoolean}}
\begin{Obsolete}{v2.00}{\Length{taskcolwidth}}
\printobsoletelist%
%
Die Klassenoption \Option*{cdtask}'none' ist komplett entfernt worden, alle 
Einstellungen, erfolgen direkt über das optionale Argument der Umgebung 
\Environment*{task}. Die Variante eines kompakten Kopfes mit der Option 
\Option*{taskcompact}'none' wird nicht mehr bereitgestellt. Die Möglichkeit zur 
manuellen Festlegung der Spaltenbreite für den Kopf der Aufgabenstellung mit 
\Length*{taskcolwidth}'none' wurde aufgrund der verbesserten automatischen 
Berechnung entfernt.
\end{Obsolete}
\end{Obsolete}
\end{Obsolete}

\begin{Obsolete}{v2.00:\Macro{taskform}}{%
  \Macro{tasks}[\MPName{Ziele}\MPName{Schwerpunkte}]%
}
\begin{Obsolete}{v2.00:\Term{focusname}}{\Term{focustext}}
\begin{Obsolete}{v2.00:\Term{objectivesname}}{\Term{objectivestext}}
\printobsoletelist%
%
Dieser Befehl wurde in \Macro*{taskform} umbenannt und in der Funktionalität 
erweitert. Die Namen der darin verwendeten Bezeichner wurden ebenfalls leicht 
abgewandelt.
\end{Obsolete}
\end{Obsolete}
\end{Obsolete}

\begin{Obsolete}{v2.00:\Macro{matriculationnumber}}{%
  \Macro{studentid}[\MPName{Matrikelnummer}]%
}
\begin{Obsolete}{v2.00:\Macro{matriculationyear}}{%
  \Macro{enrolmentyear}[\MPName{Immatrikulationsjahr}]%
}
\begin{Obsolete}{v2.00:\Macro{date}}{\Macro{submissiondate}[\MPName{Datum}]}
\begin{Obsolete}{v2.00:\Macro{dateofbirth}}{%
  \Macro{birthday}[\MPName{Geburtsdatum}]%
}
\begin{Obsolete}{v2.00:\Macro{placeofbirth}}{%
  \Macro{birthplace}[\MPName{Geburtsort}]%
}
\begin{Obsolete}{v2.00:\Macro{issuedate}}{%
  \Macro{startdate}[\MPName{Ausgabedatum}]%
}
\printobsoletelist%
%
Alle Befehle wurden umbenannt und sind jetzt neben der \taskname{} auch für die 
Titelseite im \CD nutzbar.
\end{Obsolete}
\end{Obsolete}
\end{Obsolete}
\end{Obsolete}
\end{Obsolete}
\end{Obsolete}

\begin{Obsolete}{v2.00:\Term{matriculationnumbername}}{\Term{studentidname}}
\begin{Obsolete}{v2.00:\Term{matriculationyearname}}{\Term{enrolmentname}}
\begin{Obsolete}{v2.00:\Term{datetext}}{\Term{submissiontext}}
\begin{Obsolete}{v2.00:\Term{dateofbirthtext}}{\Term{birthdaytext}}
\begin{Obsolete}{v2.00:\Term{placeofbirthtext}}{\Term{birthplacetext}}
\begin{Obsolete}{v2.00:\Term{supervisorothername}}{\Term{supervisorIIname}}
\begin{Obsolete}{v2.00:\Term{defensedatetext}}{\Term{defensetext}}
\begin{Obsolete}{v2.00:\Term{issuedatetext}}{\Term{starttext}}
\begin{Obsolete}{v2.00:\Term{duedatetext}}{\Term{duetext}}
\printobsoletelist%
%
Die Bezeichner wurden in Anlehnung an die dazugehörigen Befehlsnamen umbenannt.
\end{Obsolete}
\end{Obsolete}
\end{Obsolete}
\end{Obsolete}
\end{Obsolete}
\end{Obsolete}
\end{Obsolete}
\end{Obsolete}
\end{Obsolete}
\end{Bundle}


\ChangesTo[%
  Umbenennung einiger Befehle für Kompatibilität mit anderen Paketen%
]{v2.02}
%
\begin{Obsolete}{v2.02:\Option{pageheadingsvskip}}{\Length{chapterheadingvskip}}
\printobsoletelist%
%
Die vertikale Positionierung von Überschriften wurde aufgeteilt. Zum einen kann 
diese für Titel"~, Teile- und Kapitelseiten (\Option*{chapterpage=true}) über 
die Option \Option*{pageheadingsvskip} geändert werden. Für den Titelkopf
(\Option*{titlepage=false}(\Bundle{koma-script})) sowie Kapitelüberschriften 
(\Option*{chapterpage=false}) kann dies mit \Option*{headingsvskip} unabhängig 
davon erfolgen.
\end{Obsolete}

\begin{Obsolete}{v2.02:\Macro{graduation}}{%
  \Macro{degree}[\OPName{Abk.}\MPName{Grad}]%
}
\begin{Obsolete}{v2.02:\Term{graduationtext}}{\Term{degreetext}}
\begin{Obsolete}{v2.02:\Macro{blocking}}{%
  \Macro{restriction}[\OPList{Firma}]%
}
\begin{Obsolete}{v2.02:\Term{blockingname}}{\Term{restrictionname}}
\begin{Obsolete}{v2.02:\Term{blockingtext}}{\Term{restrictiontext}}
\printobsoletelist%
%
Die beiden Befehle wurden zur Erhöhung der Kompatibilität aufgrund möglicher 
Konflikte mit anderen Paketen umbenannt, die jeweils dazugehörigen Bezeichner 
dahingehend angepasst.
\end{Obsolete}
\end{Obsolete}
\end{Obsolete}
\end{Obsolete}
\end{Obsolete}

\begin{Obsolete}{}{\Environment{tudpage}[\OPList{Sprache}]}
\begin{Obsolete}{v2.02:\Key{\Environment{tudpage}}{pagestyle}}{%
  \Key{\Environment{tudpage}}{head=\PMisc}
}
\begin{Obsolete}{v2.02:\Key{\Environment{tudpage}}{pagestyle}}{%
  \Key{\Environment{tudpage}}{foot=\PMisc}
}
\printobsoletelist%
%
Die Funktionalität beider Parameter wurde durch den Parameter 
\Key*{\Environment{tudpage}}{pagestyle} ersetzt.
\end{Obsolete}
\end{Obsolete}
\end{Obsolete}



\minisec{Änderungen im Paket \Package{tudscrsupervisor}}
%
Im Paket \Package{tudscrsupervisor} gab es ein paar kleinere Anpassungen.
\begin{Bundle}{\Package{tudscrsupervisor}}
\begin{Obsolete}{v2.02:\Macro{discipline}}{%
  \Macro{branch}[\MPName{Studienrichtung}]%
}
\begin{Obsolete}{v2.02:\Term{disciplinename}}{\Term{branchname}}
\printobsoletelist%
%
Für die \taskname{} wurden der Befehl sowie der dazugehörige Bezeichner 
umbenannt.
\end{Obsolete}
\end{Obsolete}

\begin{Obsolete}{v2.02:\Macro{contactperson}}{%
  \Macro{contact}[\MPName{Kontaktperson(en)}]%
}
\begin{Obsolete}{v2.02:\Term{contactpersonname}}{\Term{contactname}}
\begin{Obsolete}{v2.02:\Macro{telephone}}{%
  \Macro{phone}[\MPName{Telefonnummer}]%
}
\begin{Obsolete}{v2.02:\Macro{emailaddress}}{%
  \Macro{email}[\MPName{E-Mail-Adresse}]%
}
\printobsoletelist%
%
Alle genannten Befehle und Bezeichner wurden für den \noticename{} umbenannt.
\end{Obsolete}
\end{Obsolete}
\end{Obsolete}
\end{Obsolete}
\end{Bundle}


\ChangesTo{v2.03}
%
\begin{Obsolete}{v2.03:\Option{cdgeometry}}{\Option{geometry=\PBoolean}}
\printobsoletelist%
%
Diese Option wurde zur Konsistenz sowie dem Vermeiden eines möglichen 
Konfliktes mit einer späteren \KOMAScript-Version umbenannt. Die Funktionalität 
bleibt bestehen.
\end{Obsolete}

\begin{Obsolete}{v2.03:\Option{cdfont}}{\Option{cdfonts=\PBoolean}}
\begin{Obsolete}{v2.03:\Option{cdfont}}{\Option{din=\PBoolean}}
\printobsoletelist%
%
Die Option \Option*{cdfont} wurde erweitert und fungiert als zentrale 
Schnittstelle zur Schrifteinstellung. 
\end{Obsolete}
\end{Obsolete}

\begin{Obsolete}{v2.03:\Option{cdmath}}{\Option{sansmath=\PBoolean}}
\printobsoletelist%
%
Diese Option wurde aus Gründen der Konsistenz umbenannt. Zusätzlich wurde die 
Funktionalität erweitert.
\end{Obsolete}

\begin{Obsolete}{v2.03:\Option{cdhead}}{\Option{barfont=\PMisc}}
\begin{Obsolete}{v2.03:\Option{cdhead}}{\Option{widehead=\PBoolean}}
\printobsoletelist%
%
Die Optionen \Option*{barfont}'none' und \Option*{widehead}'none' wurden in der 
Option \Option*{cdhead} zusammengefasst.
\end{Obsolete}
\end{Obsolete}

\begin{Obsolete}{}{\Environment{tudpage}[\OPList{Sprache}]}
\begin{Obsolete}{v2.03}{\Key{\Environment{tudpage}}{color=\PName{Farbe}}}
\printobsoletelist%
%
Der Parameter \Key*{\Environment{tudpage}}{color=\PSet{Farbe}}'none' der 
\Environment*{tudpage}"~Umgebung wurde ersatzlos entfernt.
\end{Obsolete}
\end{Obsolete}


\ChangesTo{v2.04}
%
\begin{Obsolete}{v2.04}{\Option{fontspec=\PBoolean}}%
\printobsoletelist%
%
Anstatt diese Option zu aktivieren, kann einfach das Paket \Package{fontspec} 
in der Dokumentpräambel geladen werden. Dadurch lassen sich anschließend 
zusätzliche Pakete nutzen, welche auf die Verwendung von \Package{fontspec} 
angewiesen sind. Sollte diese Option dennoch genutzt werden, müssen alle auf 
das Paket \Package{fontspec} aufbauenden Einstellungen durch den Anwender mit 
\Macro*{AfterPackage}[\MPValue{fontspec}\MPValue{\dots}](\Bundle{koma-script}) 
verzögert werden. In \fullref{sec:fonts} sind weitere Hinweise zur Verwendung 
des Paketes \Package{fontspec} zu finden.
\end{Obsolete}


\ChangesTo{v2.05}
%
\begin{Obsolete}{v2.05:\Option{pageheadingsvskip}}{\Length{pageheadingsvskip}}
\begin{Obsolete}{v2.05:\Option{headingsvskip}}{\Length{headingsvskip}}
\printobsoletelist%
%
Die vertikale Positionierung von speziellen Überschriften erfolgt nicht mehr 
über diese beiden Längen sondern über die Optionen \Option*{headingsvskip} 
sowie \Option*{pageheadingsvskip}.
\end{Obsolete}
\end{Obsolete}


\begin{Obsolete}{v2.05:\Option{footlogoheight}}{\Length{footlogoheight}}%
\printobsoletelist%
%
Auch die Höhe der Logos im Fußbereich der \PageStyle*{tudheadings}"=Seitenstile 
wird von nun an mit der Option \Option*{footlogoheight} und nicht mehr mit 
dieser Länge festgelegt.
\end{Obsolete}


\ChangesTo{v2.06}
%
\begin{Declaration}[v2.06]{\Option{cdoldfont}}[false]
\printdeclarationlist%
%
Mit der Version~v2.06 wird standardmäßig \OpenSans als Hausschrift verwendet. 
Um jedoch weiterhin ältere Dokumente mit den Schriften \Univers und \DIN 
erzeugen zu können, wird diese Option bereitgestellt.
\Attention{%
  Diese kann ausschließlich als Klassenoption~-- oder für die Pakete 
  \Package*{tudscrfonts} und \Package*{fix-tudscrfonts} als Paketoption~-- 
  genutzt werden.
} Eine späte Optionenwahl mit \Macro*{TUDoption} oder \Macro*{TUDoptions} ist 
nicht möglich. Wurden mit \Option{cdoldfont=true} die alten Schriftfamilien 
aktiviert, kann jedoch weiterhin die Option \Option*{cdfont} wie gewohnt 
genutzt werden.
%
\begin{values}{\Option{cdoldfont}}
\item[false]
  Das Verhalten ist äquivalent zu \Option*{cdfont=false}, die Hausschrift ist 
  nicht aktiv.
\item[true]
  Es werden die alten Hausschriften \Univers für den Fließtext sowie \DIN für 
  Überschriften der obersten Gliederungsebenen bis einschließlich 
  \Macro*{subsubsection} genutzt. Die Schriftstärke lässt sich mit 
  \Option*{cdfont=true} respektive \Option*{cdfont=heavy} anpassen.
\end{values}
%
Für die \TUDScript-Klassen sowie die vom Paket \Package*{fix-tudscrfonts} 
unterstützten Dokumentklassen kann die für die Gliederungsebenen verwendete 
Schriftart angepasst werden.
%
\begin{values}{\Option{cdoldfont}}
\item[nodin]
  Für Überschriften wird \Univers anstatt \DIN verwendet.
\item[din]
  Mit dieser Einstellung wird \DIN in Überschriften genutzt. 
\item[onlydin]
  Überschriften werden in \DIN gesetzt, für den Fließtext wird nicht \Univers 
  sondern die \hologo{LaTeX}"=Standardschriften respektive die eines geladenen 
  Schriftpaketes verwendet.
\end{values}
\end{Declaration}

\begin{Obsolete}{v2.06}[\Option{cdfont=\PMisc}]{\Option{cdfont=din}}{%
  entfällt, \seeref{\Option{cdoldfont=din}}%
}
\begin{Obsolete}{v2.06}[\Option{cdfont=\PMisc}]{\Option{cdfont=nodin}}{%
  entfällt, \seeref{\Option{cdoldfont=nodin}}%
}
\printobsoletelist%
%
Die Einstellungen für Überschriften sind mit der Umstellung auf \OpenSans nicht 
mehr notwendig. Zur Verwendung der alten Schriftfamilien muss die Option 
\Option{cdoldfont} aktiviert werden.
\end{Obsolete}
\end{Obsolete}

\begin{Obsolete}{v2.06}{%
  \Macro{ifdin}[\MPName{Dann-Teil}\MPName{Sonst-Teil}]
}%
\printobsoletelist%
%
Dieser Befehl prüft, ob aktuell die Schriftfamilie \DIN aktiv ist und führt in 
diesem Fall \MPName{Dann-Teil} aus, andernfalls \MPName{Sonst-Teil}. 
\end{Obsolete}



\minisec{Auszeichnungen in Überschriften}
%
Die Überschriften aller Gliederungsebenen von \Macro*{part} bis einschließlich 
\Macro*{subsubsection} werden in Majuskeln der Schrift \DIN gesetzt, wenn dies 
mit \Option*{cdoldfont=true} respektive \Option*{cdoldfont=onlydin} aktiviert 
wurde. Hierfür wird intern \Macro*{MakeTextUppercase}(\Package{textcase}) aus 
dem Paket \Package{textcase} genutzt, welches zusammen mit den alten 
Schriftfamilien geladen wird. Sollen bestimmte Minuskeln erhalten bleiben, ist 
\Macro*{NoCaseChange}(\Package{textcase}) zu verwenden.
%
\begin{Example}
In einem Kapitel wird ein einzelnes Wort in Minuskeln geschrieben:
\begin{Code}[escapechar=§]
\chapter{§Ü§berschrift mit \NoCaseChange{kleinem} Wort}
\end{Code}
\end{Example}
%
Die Schrift \DIN durfte laut \CD nur mit Majuskeln verwendet werden, weshalb 
das beschriebene Vorgehen lediglich im \emph{Ausnahmefall} anzuwenden ist. 
Die manuellen Nutzung sollte mit 
\Macro*{MakeTextUppercase}[%
  \MPValue{\Macro{textdbn}[\MPName{Text}]}%
](\Package{textcase})'none' geschehen.

\begin{Obsolete}{v2.06}{\Macro{univln}}
\begin{Obsolete}{v2.06}{\Macro{textuln}[\MPName{Text}]}
\begin{Obsolete}{v2.06}{\Macro{univrn}}
\begin{Obsolete}{v2.06}{\Macro{texturn}[\MPName{Text}]}
\begin{Obsolete}{v2.06}{\Macro{univbn}}
\begin{Obsolete}{v2.06}{\Macro{textubn}[\MPName{Text}]}
\begin{Obsolete}{v2.06}{\Macro{univxn}}
\begin{Obsolete}{v2.06}{\Macro{textuxn}[\MPName{Text}]}
\begin{Obsolete}{v2.06}{\Macro{univls}}
\begin{Obsolete}{v2.06}{\Macro{textuls}[\MPName{Text}]}
\begin{Obsolete}{v2.06}{\Macro{univrs}}
\begin{Obsolete}{v2.06}{\Macro{texturs}[\MPName{Text}]}
\begin{Obsolete}{v2.06}{\Macro{univbs}}
\begin{Obsolete}{v2.06}{\Macro{textubs}[\MPName{Text}]}
\begin{Obsolete}{v2.06}{\Macro{univxs}}
\begin{Obsolete}{v2.06}{\Macro{textuxs}[\MPName{Text}]}
\begin{Obsolete}{v2.06}{\Macro{dinbn}}
\begin{Obsolete}{v2.06}{\Macro{textdbn}[\MPName{Text}]}
\printobsoletelist%
%
Wird die Option \Option{cdoldfont} nicht aktiviert, werden auch die Befehle zur 
expliziten Auswahl eines Schriftschnittes nicht mehr bereitgestellt. 
Stattdessen können \Macro*{cdfont} oder \Macro*{textcd}[\MPName{Text}] 
genutzt werden, welche in \fullref{sec:text} zu finden sind.
\end{Obsolete}
\end{Obsolete}
\end{Obsolete}
\end{Obsolete}
\end{Obsolete}
\end{Obsolete}
\end{Obsolete}
\end{Obsolete}
\end{Obsolete}
\end{Obsolete}
\end{Obsolete}
\end{Obsolete}
\end{Obsolete}
\end{Obsolete}
\end{Obsolete}
\end{Obsolete}
\end{Obsolete}
\end{Obsolete}

\begin{Obsolete}{v2.06:\Option{cleardoublespecialpage}}{%
  \Option{clearcolor=\PBoolean}%
}
\printobsoletelist%
%
Diese Option wurde zur Vereinfachung und Vereinheitlichung der 
Benutzerschnittstelle in \Option*{cleardoublespecialpage=\PMisc} integriert.
\end{Obsolete}



\section[%
  Das Paket \PackageRaw{tudscrcomp}{\BooleanFalse}
  -- Migration von anderen TUD-Klassen%
]{%
  Migration von anderen TUD-Klassen%
  \tudmarkuplabel{\Package{tudscrcomp}}%
  \index{Kompatibilität!\Class{tudbook}|(}%
  \index{Kompatibilität!\Class{tudmathposter}|(}%
}
\begin{DeclareEntity}{\Package{tudscrcomp}}
\noindent\Attention{%
  Sollten Sie eine der bereits in \autoref{sec:tudclasses} genannten Klassen
  \Class{tudbook}|?| (inklusive des Paketes \Package{tudthesis}) sowie 
  \Class{tudletter}|?|, \Class{tudfax}|?|, \Class{tudhaus}|?| und 
  \Class{tudform}|?| oder auch \Class{tudmathposter}|?| beziehungsweise eine 
  der \TUDScript-Klassen in der \emph{Version~v1.0} nie genutzt haben, können 
  Sie dieses \autorefname ohne Weiteres überspringen. Sämtliche der hier 
  vorgestellten Optionen und Befehle sind in der aktuellen Version von 
  \TUDScript obsolet.
}

\bigskip\noindent
Zu Beginn der Entwicklung von \TUDScript diente die Klasse \Class{tudbook} als 
grundlegende Basis zur Orientierung. Ziel war es, sämtliche Funktionalitäten 
dieser Klasse beizubehalten und zusätzlich den vollen Funktionsumfang der 
\KOMAScript-Klassen nutzbar zu machen. Bei der kompletten Neuimplementierung 
der \TUDScript-Klassen wurde sehr viel verändert und verbessert. Ein Teil der 
implementierten Optionen und Befehle waren jedoch bereits in der 
\emph{Version~v1.0} von \TUDScript unerwünschte Relikte, mit denen lediglich 
die Kompatibilität zur \Class{tudbook}"~Klasse und ihren Derivaten 
gewährleistet werden sollte. Mit der Version~v2.00 wurden einige der unnötigen 
Befehle und Optionen aus Gründen der Konsistenz nur umbenannt, andere wiederum 
wurden vollständig entfernt oder über neue Befehle und Optionen in ihrer 
Funktionalität ersetzt und teilweise erweitert. 

Das Paket \Package{tudscrcomp} dient der Überführung von Dokumenten, welche
entweder mit der \Class{tudbook}"~Klasse, ihren Derivaten, 
der Klasse \Class{tudmathposter} oder mit \emph{\TUDScript~v1.0} 
erstellt wurden, auf die aktuelle Version \TUDScriptVersion. Es werden einige 
Optionen und Befehle bereitgestellt, welche von den zuvor genannten Klassen 
definiert werden, um das entsprechende Verhalten nachzuahmen. Damit soll vor 
allem die Kompatibilität bei einer Änderung der Dokumentklasse sichergestellt 
werden. Unter anderem wird dafür an das Paket \Package{tudscrcolor} die Option 
\Option{extended}(\Package{tudscrcolor}) übergeben. Ist dies nicht erwünscht, 
sollte das Paket mit der Option \Option{reduced}(\Package{tudscrcolor}) geladen 
werden.
%
\begin{quoting}
\Attention{%
  Die Intention des Paketes ist, Dokumente möglichst schnell und einfach auf 
  die \TUDScript-Klassen portieren zu können. Das Ausgabeergebnis kann jedoch 
  stark von dem ursprünglichen Dokument abweichen. Falls Sie das Paket 
  verwenden wollen, sollte es \textbf{direkt} nach der Dokumentklasse geladen 
  werden. Andernfalls kann es im Zusammenhang mit anderen Paketen zu Problemen 
  kommen. Für den Satz neuer Dokumente wird empfohlen, auf den Einsatz dieses 
  Paketes zu verzichten und stattdessen direkt die von \TUDScript 
  bereitgestellten Optionen und Befehle zu nutzen.
}
\end{quoting}
%
Es ist zu beachten, dass nicht alle Funktionalitäten der genannten Klassen 
portiert werden konnten. Die betroffenen Optionen und Befehle erzeugen in 
diesem Fall eine Warnung mit einem Hinweis, wie ein gleiches oder zumindest 
ähnliches Verhalten mit \TUDScript erreicht werden kann. Weiterhin wird im 
Folgenden skizziert, wie sich die einzelnen Funktionalität ohne eine Verwendung 
des Paketes \Package{tudscrcomp} mit den Mitteln von \TUDScript umsetzen 
lassen. 

\begin{Declaration}{\Macro{einrichtung}[\MPName{Fakultät}]}(%
  identisch zu \Macro*{faculty}[\MPName{Fakultät}]()%
)
\begin{Declaration}{\Macro{fachrichtung}[\MPName{Einrichtung}]}(%
  identisch zu \Macro*{department}[\MPName{Einrichtung}]()%
)
\begin{Declaration}{\Macro{institut}[\MPName{Institut}]}(%
  identisch zu \Macro*{institute}[\MPName{Institut}]()%
)
\begin{Declaration}{\Macro{professur}[\MPName{Lehrstuhl}]}(%
  identisch zu \Macro*{chair}[\MPName{Lehrstuhl}]()%
)
\printdeclarationlist%
%
Dies sind die deutschsprachigen Befehle für den Kopf im \CD.
\end{Declaration}
\end{Declaration}
\end{Declaration}
\end{Declaration}

\begin{Declaration}{\Option{serifmath}}{%
  identisch zu \Option*{cdmath=false}%
}
\printdeclarationlist%
%
Die Funktionalität wird durch die Option \Option*{cdmath} bereitgestellt.
\end{Declaration}

\begin{Declaration}{\Option{colortitle}}{%
  identisch zu \Option*{cdtitle=color}%
}
\begin{Declaration}{\Option{nocolortitle}}{%
  identisch zu \Option*{cdtitle=true}%
}
\printdeclarationlist%
%
Die Funktionalität wird durch die Option \Option*{cdtitle} bereitgestellt.
\end{Declaration}
\end{Declaration}

\begin{Declaration}{\Macro{moreauthor}[\MPName{Autorenzusatz}]}{%
  identisch zu \Macro*{authormore}[\MPName{Autorenzusatz}]%
}
\printdeclarationlist%
%
Ursprünglich war der Befehl für das Unterbringen aller möglichen, zusätzlichen 
Autoreninformationen gedacht. Auch der Befehl \Macro*{authormore}() ist ein 
Relikt davon, welcher jedoch weiterhin für allgemeine oder nicht vordefinierte 
Angaben genutzt werden kann. Für spezielle Informationen auf dem Titel sind die 
Befehle \Macro*{emailaddress}, \Macro*{dateofbirth}, \Macro*{placeofbirth}, 
\Macro*{matriculationnumber} und \Macro*{matriculationyear} sowie 
\Macro*{course} und \Macro*{discipline} vorgesehen und sollten hierfür 
dementsprechend genutzt werden.
\end{Declaration}

\begin{Declaration}{\Option{ddcfooter}}{%
  identisch zu \Option*{ddcfoot=true}%
}
\printdeclarationlist%
%
Die Funktionalität wird durch die Option \Option*{ddcfoot} bereitgestellt.
\end{Declaration}

\begin{Declaration}{\Macro{tudfont}[\MPName{Scriftart}]}{%
  identisch zu \Macro*{cdfont}[\MPName{Scriftart}]%
}
\printdeclarationlist%
%
Die direkte Auswahl der Schriftart sollte mit \Macro*{cdfont} erfolgen. 
Zusätzlich gibt es den Befehl \Macro*{textcd}, mit dem die Auszeichnung 
eines bestimmten Textes in einer anderen Schriftart erfolgen kann, ohne die 
Dokumentschrift umzuschalten.
\end{Declaration}
\index{Kompatibilität!\Class{tudmathposter}|)}%


\subsection{Optionen und Befehle aus \ClassRaw{tudbook}{\BooleanFalse} \& Co.}
%
Die nachfolgenden Optionen, Umgebungen sowie Befehle werden~-- zumindest 
teilweise~-- von den Klassen \Class{tudbook}, \Class{tudletter}, 
\Class{tudfax}, \Class{tudhaus}, \Class{tudform} sowie dem Paket 
\Package{tudthesis} und \TUDScript in der \emph{Version~v1.0} definiert. Diese 
werden durch das Paket \Package{tudscrcomp} für \TUDScript~\vTUDScript{} zur 
Verfügung gestellt.

\begin{Declaration}{\Macro{submitdate}[\MPName{Datum}]}{%
  identisch zu \Macro*{date}[\MPName{Datum}]%
}
\printdeclarationlist%
%
Die Funktionalität wird durch den erweiterten Standardbefehl \Macro*{date} 
abgedeckt.
\end{Declaration}

\begin{Declaration}{\Macro{supervisorII}[\MPName{Name}]}{%
  identisch zur Verwendung von \Macro*{and} innerhalb von \Macro*{supervisor}%
}
\printdeclarationlist%
%
Es ist \Macro*{supervisor}[\MPValue{\PName{Name} \Macro*{and} \PName{Name}}]
statt \Macro*{supervisorII}[\MPName{Name}] zu verwenden.
\end{Declaration}

\begin{Declaration}{\Macro{submittedon}[\MPName{Bezeichnung}]}(%
  siehe \Term*{datetext}%
)
\begin{Declaration}{\Macro{supervisedby}[\MPName{Bezeichnung}]}(%
  siehe \Term*{supervisorname}%
)
\begin{Declaration}{\Macro{supervisedIIby}[\MPName{Bezeichnung}]}(%
  siehe \Term*{supervisorothername}%
)
\printdeclarationlist%
%
Zur Änderung der Bezeichnung der Betreuer sollten die sprachabhängigen 
Bezeichner wie in \autoref{sec:localization} beschrieben angepasst werden. Eine 
Verwendung der hier beschriebenen Befehle entfernt die Abhängigkeit der 
Bezeichner von der verwendeten Sprache.
\end{Declaration}
\end{Declaration}
\end{Declaration}

\begin{Declaration}{\Macro{dissertation}}
\printdeclarationlist%
%
Die Funktionalität kann durch die Befehle \Macro*{thesis}[\MPValue{diss}] und 
\Macro*{referee} sowie die Bezeichner \Term*{refereename} und 
\Term*{refereeothername} dargestellt werden.
\end{Declaration}

\begin{Declaration}{\Macro{chapterpage}}
\printdeclarationlist%
%
Durch diesen Befehl können Kapitelseiten konträr zur eigentlichen Einstellung 
aktiviert oder deaktiviert werden. Prinzipiell ist dies auch durch eine 
Änderung der Option \Option*{chapterpage} möglich. Allerdings wird davon 
abgeraten, da dies zu einem inkonsistenten Layout innerhalb des Dokumentes 
führt.
\end{Declaration}

\begin{Declaration}{\Environment{theglossary}[\OPName{Präambel}]}
\begin{Declaration}{\Macro{glossitem}[\MPName{Begriff}]}
\printdeclarationlist%
%
Die \Class{tudbook}"~Klasse stellt eine rudimentäre Umgebung für ein Glossar 
bereit. Allerdings gibt es dafür bereits zahlreiche und besser implementierte 
Pakete. Daher wird für diese Umgebung keine Portierung vorgenommen, sondern 
lediglich die ursprüngliche Definition übernommen. Allerdings sein an dieser 
Stelle auf wesentlich bessere Lösungen wie beispielsweise das Paket 
\Package{glossaries} oder~-- mit Abstrichen~-- das nicht ganz so umfangreiche 
Paket \Package{nomencl} verwiesen.
\end{Declaration}
\end{Declaration}
\index{Kompatibilität!\Class{tudbook}|)}%



\subsection{%
  Optionen und Befehle aus \ClassRaw{tudmathposter}{\BooleanFalse}%
  \index{Kompatibilität!\Class{tudmathposter}|(}%
}
\ChangedAt{v2.05:Unterstützung der Klasse \Class{tudmathposter}}%
Die Klasse~\Class{tudmathposter} wird~-- im Gegensatz zu den zuvor genannten 
Klassen von Klaus~Bergmann~-- weiterhin gepflegt und kann bedenkenlos zum 
Setzen von Postern im A0"~Format verwendet werden. Dennoch gab es vermehrt 
Anfragen bezüglich einer Posterklasse auf Basis der \TUDScript-Klassen, um 
beispielsweise die Schriftgröße oder auch das Papierformat einfach anpassen zu 
können. Um von \Class{tudmathposter} einen möglichst einfachen Übergang auf 
\Class{tudscrposter} zu gewährleisten, kann zusätzlich zu letzterer Klasse das 
Paket \Package{tudscrcomp} geladen werden, welches die nachfolgend erläuterten 
Anwenderbefehle bereitstellt.

Mit der Kombination von \Class{tudscrposter} und dem Paket \Package{tudscrcomp}
soll lediglich die Möglichkeit geschaffen werden, auf \Class{tudmathposter} 
basierende Dokumente zu überführen. Es ist nicht beabsichtigt, dass bei einem 
Umstieg das Ausgabeergebnis identisch ist. In jedem Fall sollte beim Einsatz 
der Klasse \Class{tudscrposter} beachtet werden, dass für diese eine explizite 
Wahl der Schriftgröße über die Option~\Option*{fontsize}(\Bundle{koma-script}) 
notwendig ist. Um kongruent zur Klasse \Class{tudmathposter} zu bleiben, ist 
eine Schriftgröße von \PValue{34\dots36pt} sinnvoll. Weitere Informationen dazu 
sind in \autoref{sec:fontsize} vorhanden. Weiterhin sollte für ein ähnliches 
Ausgabeergebnis die Absatzformatierung über die \KOMAScript-Option 
\Option*{parskip=half}(\Bundle{koma-script}) eingestellt werden. Ein blaues 
\DDC-Logo im Fußbereich lässt sich über \Option*{ddcfoot=blue} aktivieren.

\begin{Declaration}{\Option{loadpackages=\PBoolean}}
\printdeclarationlist%
%
Durch \Class{tudmathposter} werden normalerweise die Pakete \Package{calc}, 
\Package{textcomp} sowie \Package{tabularx} eingebunden, welche allerdings für 
die Funktionalität der Klasse selbst nicht (zwingend) benötigt werden. Deshalb 
wird bei der Nutzung von \Package{tudscrcomp} standardmäßig darauf verzichtet. 
Diese können bei Bedarf einfach in der Präambel geladen werden. Alternativ 
dazu 
lässt sich die Option \Option{loadpackages} nutzen, welche die Pakete am Ende 
der Präambel automatisch lädt.
\end{Declaration}

\begin{Declaration}{\Option{tudmathfoot=\PBoolean}}%
\printdeclarationlist%
%
Durch die Klasse \Class{tudmathposter} wird der Fußbereich zweispaltig jedoch 
asymmetrisch und ohne Überschriften innerhalb der beiden Spalten gesetzt. 
Dieses Verhalten lässt sich mit \Option{tudmathfoot=true} auswählen, wohingegen 
\Option{tudmathfoot=false} genutzt werden kann, um auf das Standardverhalten 
von \Class{tudscrposter} zu schalten.
\end{Declaration}

\begin{Declaration}{\Option{bluebg}}{%
  identisch zu \Option*{backcolor=true}(\Class{tudscrposter})%
}
\printdeclarationlist%
%
Mit der Klasse \Class{tudscrposter} lässt sich das Verhalten mit der Option 
\Option*{backcolor}(\Class{tudscrposter}) umsetzen.
\end{Declaration}

\begin{Declaration}{\Macro{email}[\MPName{E-Mail-Adresse}]}(identisch zu 
  \Macro*{emailaddress*}[\MPName{E-Mail-Adresse}]%
)
\begin{Declaration}{\Macro{telefon}[\MPName{Telefonnummer}]}(identisch zu 
  \Macro*{telephone}[\MPName{Telefonnummer}](\Class{tudscrposter})%
)
\begin{Declaration}{\Macro{fax}[\MPName{Telefaxnummer}]}(identisch zu 
  \Macro*{telefax}[\MPName{Telefaxnummer}](\Class{tudscrposter})%
)
\begin{Declaration}{\Macro{homepage}[\MPName{URL}]}(identisch zu 
  \Macro*{webpage*}[\MPName{URL}](\Class{tudscrposter})%
)
\printdeclarationlist%
%
Dies sind die von \Class{tudmathposter} definierten Befehle für die Felder im 
vordefinierten Fußbereich des Posters. Es ist dabei insbesondere zu beachten, 
dass eine angegebene E"~Mail"=Adresse sowie URL nicht automatisch formatiert 
werden.
\end{Declaration}
\end{Declaration}
\end{Declaration}
\end{Declaration}

\begin{Declaration}{%
  \Macro{topsection}[\OPName{Kurzform}\MPName{Überschrift}]%
}
\begin{Declaration}{%
  \Macro{topsubsection}[\OPName{Kurzform}\MPName{Überschrift}]%
}
\printdeclarationlist%
%
Der Grund für die Existenz dieser beiden Befehle bei \Class{tudmathposter} ist 
nicht ohne Weiteres nachvollziehbar. Beide entsprechen in ihrem Verhalten den 
Standardbefehlen \Macro*{section} und \Macro*{subsection}, setzen allerdings 
keinen vertikalen Abstand vor der erzeugten Überschrift. Auch wenn das aus 
typographischer Sicht unvorteilhaft ist, werden diese beiden Befehle 
bereitgestellt.
\end{Declaration}
\end{Declaration}

\begin{Declaration}{%
  \Macro{centersection}[\OPName{Kurzform}\MPName{Überschrift}]%
}
\begin{Declaration}{%
  \Macro{centersubsection}[\OPName{Kurzform}\MPName{Überschrift}]%
}
\begin{Declaration}{%
  \Macro{topcentersection}[\OPName{Kurzform}\MPName{Überschrift}]%
}
\begin{Declaration}{%
  \Macro{topcentersubsection}[\OPName{Kurzform}\MPName{Überschrift}]%
}
\printdeclarationlist%
%
Weiterhin werden auch noch eigene Makros zum Setzen zentrierter Überschriften 
definiert~-- ein simples Umdefinieren von \Macro*{raggedsection} wäre dafür im 
Normalfall absolut ausreichend. Und um die Sache vollständig zu machen, gibt es 
die zentrierten Überschriften auch noch ohne vorgelagerten, vertikalen Abstand.
\end{Declaration}
\end{Declaration}
\end{Declaration}
\end{Declaration}

\begin{Declaration}{\Macro{zweitlogofile}[\MPName{Dateiname}]}(%
  identisch zu \Macro*{headlogo}[\MPName{Dateiname}]
)
\begin{Declaration}{\Macro{institutslogofile}[\MPName{Dateiname}]}(%
  \seeref{\Macro*{footlogo}}%
)
\begin{Declaration}{\Macro{drittlogofile}[\MPName{Dateiname}]}(%
  \seeref{\Option*{ddc} und \Option*{ddcfoot}}%
)
\printdeclarationlist%
%
Für die Angabe von Logos für den Kopf- und Fußbereich existieren diese Befehle. 
Bei der Verwendung von \Macro{institutslogofile}[\MPName{Dateiname}] ist zu 
beachten, dass die angegebene Datei sehr weit rechts im Fußbereich des Posters 
gesetzt wird. Dabei kommt bei der Verwendung im Hintergrund der von \TUDScript 
für das Setzen von Logos im Fußbereich tatsächlich vorgesehene Befehl in der 
Form \Macro*{footlogo}[\MPValue{{{,}{,}{,}{,}{,}{,}{,}\PName{Dateiname}{,}}}]
zum Einsatz. Das Makro \Macro{drittlogofile} wird von \Class{tudmathposter} für 
die Angabe eines \DDC-Logos im rechten Seitenfuß bereitgestellt. Für die 
\TUDScript-Klassen gibt es hierfür die Optionen \Option*{ddc} beziehungsweise 
\Option*{ddcfoot}.
\end{Declaration}
\end{Declaration}
\end{Declaration}

\begin{Declaration}{\Macro{zweitlogo}[\MPName{Definition}]}(%
  keine Funktionalität, \seeref{\Macro*{headlogo}}
)
\begin{Declaration}{\Macro{institutslogo}[\MPName{Definition}]}(%
  keine Funktionalität, \seeref{\Macro*{footlogo}}
)
\begin{Declaration}{\Macro{drittlogo}[\MPName{Definition}]}(%
  keine Funktionalität, \seeref{\Option*{ddc} und \Option*{ddcfoot}}%
)
\printdeclarationlist%
%
Mit \Class{tudmathposter} kann der Anwender die Definition für das Einbinden 
diverser Logos selber vornehmen. Dies ist für \TUDScript nicht vorgesehen, 
die Makros geben lediglich eine Warnung aus. Im Bedarfsfall lassen sich die 
optionalen Parameter der korrelierenden Befehle nutzen. 
\end{Declaration}
\end{Declaration}
\end{Declaration}

\begin{Declaration}{\Macro{fusszeile}[\MPName{Inhalt}]}(%
  identisch zu \Macro*{footcontent}[\MPName{Inhalt}]
)
\begin{Declaration}{\Macro{footcolumn0}[\MPName{Inhalt}]}(%
  identisch zu \Macro*{footcontent}[\MPName{Inhalt}]
)
\begin{Declaration}{\Macro{footcolumn1}[\MPName{Inhalt}]}(%
  identisch zu \Macro*{footcontent}[\MPName{Inhalt}\OPValue{*}]
)
\begin{Declaration}{\Macro{footcolumn2}[\MPName{Inhalt}]}(%
  identisch zu \Macro*{footcontent}[\MPValue{*}\OPName{Inhalt}]
)
\printdeclarationlist%
%
Mit diesen Befehlen kann die Gestalt des Fußes angepasst werden, wobei entweder 
der Bereich über die gesamte Breite (\Macro{fusszeile}, \Macro{footcolumn0}) 
oder lediglich die linke (\Macro{footcolumn1}) respektive die rechte Spalte 
(\Macro{footcolumn2}) angepasst wird. Für zusätzliche Hinweise zur Anpassung 
des Fußbereichs~-- insbesondere für die Schriftformatierung~-- sollte die 
Beschreibung von \Macro*{footcontent}'full' zu Rate gezogen werden.
\end{Declaration}
\end{Declaration}
\end{Declaration}
\end{Declaration}

\begin{Declaration}{\Environment{farbtabellen}}
\begin{Declaration}{\Macro{blautabelle}}
\begin{Declaration}{\Macro{grautabelle}}
\printdeclarationlist%
%
Wird innerhalb der \Environment{farbtabellen}"~Umgebung eine Tabelle gesetzt, 
so werden die Zeilen alternierend farbig hervorgehoben. Standardmäßig sind 
hierfür leichte Blautöne eingestellt, was auch jederzeit mit dem Aufruf von 
\Macro{blautabelle} wiederhergestellt werden kann. Alternativ zu dieser 
Darstellung kann mit \Macro{grautabelle} auf leichte Grautöne umgestellt werden.
\end{Declaration}
\end{Declaration}
\end{Declaration}

\begin{Declaration}{\Macro{schnittrand}}
\printdeclarationlist%
%
Wird der Befehl \Macro{schnittrand} innerhalb der Präambel definiert, so wird 
dessen Inhalt als Längenwert interpretiert. Dieser wird verwendet, um den zuvor 
festgelegten Satzspiegel über die drei Parameter
\Key*{\Macro{geometry}}{paper=\PSet{Papierformat}}(\Package{geometry}),
\Key*{\Macro{geometry}}{layout=\PSet{Zielformat}}(\Package{geometry}) und 
\Key*{\Macro{geometry}}{layoutoffset=\PLength}(\Package{geometry}) des
Befehls \Macro*{geometry}(\Package{geometry}) aus dem Paket \Package*{geometry} 
zu setzen und das erzeugte Papierformat um den gegebenen Längenwert an allen 
Rändern zu vergrößern. Somit wird eine Beschnittzugabe hinzugefügt, \emph{ohne 
dabei die Seitenränder des Entwurfslayouts anzupassen}. Weitere Informationen 
hierzu sind in \fullref{sec:tips:crop} sowie im \GitHubRepo(tud-cd/tud-cd)<6>  
zu finden.
\end{Declaration}
\index{Kompatibilität!\Class{tudmathposter}|)}%
\end{DeclareEntity}

\chapter{Identifikation von \TUDScript}
%
Im \TUDScript-Bundle gibt es neben den Klassen selbst auch noch zusätzliche 
Pakete. Ein Teil dieser Pakete~-- genauer \Package{tudscrsupervisor} und 
\Package{tudscrcomp}~-- sind ausschließlich mit den \TUDScript-Klassen nutzbar.
Andere wiederum~-- die beiden Pakete \Package{tudscrfonts}~(Schriften) und 
\Package{tudscrcolor}~(Farben) für Belange des \CDs sowie die davon vollkommen 
unabhängigen Pakete \Package{mathswap} und \Package{twocolfix}~-- können mit 
allen existierenden \hologo{LaTeX}-Klassen genutzt werden. Sämtliche 
Klassen und Pakete aus dem \TUDScript-Bundle enthalten die folgenden Befehle, 
welche diese als dessen Bestandteil identifizieren.

\begin{Declaration}
  [v2.04]
  {\Macro{TUDScript}}
\printdeclarationlist
%
Diese Anweisung setzt das Logo respektive die Wortmarke \enquote{\TUDScript{}} 
in serifenloser Schrift und mit leichter Sperrung des in Versalien gesetzten 
Teils. Dieser Befehl wird von allen Klassen und Paketen des \TUDScript-Bundles 
mit \Macro*{DeclareRobustCommand}.
\end{Declaration}

\begin{Declaration}
  [v2.04]
  {\Macro{TUDScriptClassName}}
\printdeclarationlist
%
Die Bezeichnung der jeweiligen, im Dokument verwendeten \TUDScript-Klasse ist 
im Makro \Macro{TUDScriptClassName} abgelegt. Soll also in Erfahrung gebracht 
werden, ob~-- und wenn ja, welche~-- \TUDScript-Klasse verwendet wird, so kann 
einfach auf diese Anweisung getestet werden. \KOMAScript stellt zusätzlich 
noch die beiden Anweisungen \Macro*{KOMAClassName}(\Bundle{koma-script}) und 
\Macro*{ClassName}(\Bundle{koma-script}) bereit, welche den Namen der 
zugrundeliegenden \KOMAScript-Klasse sowie die durch diese ersetzte 
Standardklasse enthalten.
\end{Declaration}

\begin{Declaration}
  [v2.04]
  {\Macro{TUDScriptVersion}}
\begin{Declaration}
  [v2.05]
  {\Macro{TUDScriptVersionNumber}}
\printdeclarationlist
%
In \Macro{TUDScriptVersion} ist die Hauptversion von \TUDScript in der Form
\begin{quoting}
\PName{Datum}~\PName{Version}~\PValue{TUD-Script}
\end{quoting}
abgelegt. Die Version ist für alle Klassen und Pakete des \TUDScript-Bundles
gleich und kann nach dem Laden einer Klasse oder eines Paketes abgefragt 
werden. Beispielsweise wurde diese Anleitung mit \enquote{\TUDScriptVersion{}} 
erstellt.

Eventuell will der Anwender auf die aktuell verwendete Version von \TUDScript 
prüfen, um gegebenenfalls eigene Anpassungen in Abhängigkeit der verwendeten 
Version vorzunehmen. Hierfür kann \Macro{TUDScriptVersionNumber} verwendet 
werden. Darin ist alleinig die Versionsnummer enthalten. Die für das Handbuch 
verwendete Version lautet \enquote{\TUDScriptVersionNumber{}}.
\end{Declaration}
\end{Declaration}

\CrossIndex{Optionen}{options}%
\CrossIndex*{Befehle}{macros}%
\CrossIndex{Bezeichner}{terms}%
\CrossIndex{Seitenstile,Schriftelemente,Farben}{elements}%
\CrossIndex*{Längen}{misc}%
\CrossIndex{Zähler}{misc}%
\CrossIndex*{Klassen,Pakete,Dateien}[Index der Dateien etc.]{files}%
\CrossIndex*{Änderungen,Changelog,Version}[Änderungsliste]{changelog}%
%
\SeeRef*{Distribution}{\Distribution{Mac\hologo{TeX}}}
\SeeRef*{Distribution}{\Distribution{\hologo{TeX}~Live}}
\SeeRef*{Distribution}{\Distribution{\hologo{MiKTeX}}}
%
\index{Abbildungen|see{Grafiken}}%
\index{Abschlussarbeit|see{Typisierung}}%
\index{Aufzählungen|see{Listen}}%
\index{Autorenangaben|see{Titel}}%
\index{Bindekorrektur|see{Satzspiegel}}%
\index{Cover|see{Umschlagseite}}%
\index{Dezimaltrennzeichen|see{Zifferngruppierung}}%
\index{doppelseitiger Satz|see{Satzspiegel}}%
\index{Dresden-concept-Logo@\DDC-Logo|see{Layout}}%
\index{Drittlogo|see{Layout}}%
\index{Fachreferent|see{Referent}}%
\index{Farben|see{Layout}}%
\index{Fußzeile|see{Layout}}%
\index{Gliederung|see{Layout!Überschriften}}%
\index{Grafiken|see{Gleitobjekte}}%
\index{Kapitel|see{Layout}}%
\index{Klassenoptionen|see{Optionen}}%
\index{Kolumnentitel|see{Layout}}%
\index{Kopfzeile|see{Layout}}%
\index{Kurzfassung|see{Zusammenfassung}}%
\index{Layout!Seitenränder|see{Satzspiegel}}%
\index{Layout!Titel|see{Titel}}%
\index{Layout!Umschlagseite|see{Umschlagseite}}%
\index{Leerseiten|see{Vakatseiten}}%
\index{Lokalisierung|see{Bezeichner}}%
\index{Makros|see{Befehle}}%
\index{Mathematiksatz|see{Einheiten}}%
\index{Mathematiksatz|see{Griechische Buchstaben}}%
\index{Mathematiksatz|see{Zifferngruppierung}}%
\index{Nutzerinstallation|see{Installation}}%
\index{Outline-Eintrag|see{Lesezeichen}}%
\index{Parameter|see{Befehle}}%
\index{Professor|see{Hochschullehrer}}%
\index{Querbalken|see{Layout}}%
\index{Seitenränder|see{Satzspiegel}}%
\index{Seitenstil|see{Layout}}%
\index{Silbentrennung|see{Worttrennung}}%
\index{Sprachunterstützung|see{Bezeichner}}%
\index{Sprachunterstützung|see{Worttrennung}}%
\index{Sprungmarken|see{Lesezeichen}}%
\index{Tabellen|see{Gleitobjekte}}%
\index{Tausendertrennzeichen|see{Zifferngruppierung}}%
\index{Teil|see{Layout}}%
\index{Titel!Umschlagseite|see{Umschlagseite}}%
\index{Überfüllung|see{Beschnittzugabe}}%
\index{Umgebungen|see{Befehle}}%
\index{Trennungsmuster|see{Worttrennung}}%
\index{Vektorgrafiken|see{Grafiken}}%
\index{Versalien|see{Schriftauszeichnung}}%
\index{zweiseitiger Satz|see{Satzspiegel}}%
\index{zweispaltiger Satz|see{Satzspiegel}}%
\index{Zweitlogo|see{Layout}}%


\setchapterpreamble{%
  \addparttocentry{}{\indexname}%
  \begin{abstract}
    \noindent Die Formatierung der Einträge in allen aufgeführten Indexen ist 
    folgendermaßen aufzufassen: \textbf{Zahlen in fetter Schrift} verweisen auf 
    die \textbf{Erklärung} zu einem Stichwort, wobei in der digitalen Fassung 
    dieses Handbuchs dieser Eintrag selbst ein Hyperlink zu seiner Erläuterung 
    ist. Seitenzahlen in normaler Schriftstärke hingegen deuten auf zusätzliche 
    Informationen, wobei diese für \textit{kursiv hervorgehobene Zahlen} als 
    besonders \textit{wichtig} erachtet werden.
    
    Bei Einträgen für \hyperref[idx:options]{Klassen- und Paketoptionen}, 
    \hyperref[idx:macros]{Umgebungen und Befehlen} sowie 
    \hyperref[idx:elements]{Seitenstilen, Schriftelementen und Farben} 
    respektive \hyperref[idx:misc]{Längen und Zählern}, zu denen keine direkte 
    \textbf{Erklärung} existiert~-- und diese dementsprechend auch nicht als 
    Hyperlink dargestellt sind~-- handelt es sich um zusätzliche Hinweise für 
    Elemente, die nicht durch \TUDScript sondern von anderen Klassen oder 
    Paketen zur Verfügung gestellt werden, welche im Suffix des entsprechenden 
    Eintrages angegeben werden. 
  \end{abstract}
}
\chapter*{\indexname}
\PrintIndex
\PrintChangelog




\clearpage

\ToDo[doc]{http(?!s) ersetzen, wo möglich}[v2.07]
\ToDo[doc]{nach Markupbefehlen mit Stern suchen überprüfen}[v2.07]
\ToDo[imp,doc]{\Macro*{if...} auf \Macro*{If...} umstellen?}[v2.07]
% \ifxblank \ifstrbool \ifdin \ifdisposition \ifisgreeksymbol
% \iflanguageloaded (noch sinnvoll?) 
\ToDo[imp,doc]{Verwendung des Paketes \Package{filecontents} obsolet}[v2.07]
\ToDo[doc]{Link für \Package{flafter} zu \hologo{LaTeX}-Bundle}[v2.07]
\ToDo[imp,doc]{%
  Benutzerdefinierte Anpassungen von \TUDScript, Kapitelstile, Farbeinsatz etc.%
}[v2.07]
\ToDo[imp]{%
  Spezialseite zur freien Gestaltung mit Hintergrundebene für Bilder und Texte 
  (CD-Handbuch S. 80 ff.); Seitenstil: \PageStyle*{special.tudheadings}
}[v2.07]
\ToDo[doc,imp]{%
  Rückseite bei Kapitelseiten (auch bei Teilen?) im doppelseitigen Satz per 
  Option nicht als Vakatseite setzen. Implementierung von farbigen Seiten 
  davor (\Option*{open=right}) sowie danach (\Option*{open=left}). Setzen von 
  speziellen Inhalten auf diesen Seiten äquivalent zu \Macro*{setpartpreamble} 
  bzw. \Macro*{setchapterpreamble}; ggf. temporär umschalten bzw. Warnung bei 
  Konflikt.
}[v2.08]

\ToDo[imp]{Alle Datumsangaben auf YYYY-MM-DD umstellen}[v2.07]

\ToDo[imp]{LyX: layout-Dateien in das Projekt integrieren lassen?}[v2.07]
\ToDo[imp]{%
  \Package*{glossaries}: Paket für Stile aus Tutorial, siehe \GitHubRepo<90>
}[v2.08]

\ToDo[rls]{GitHub auf offene Issues prüfen}
\ToDo[rls]{\emph{alle} dtx-Dateien der vorherigen Version mit WinMerge sichten}
\ToDo[rls]{Datumsangaben \emph{aller} dtx-Dateien prüfen}%: ^((?!%).)*\d{4}(/|-)
\ToDo[rls]{Layout und Umbrüche kontrollieren}
\ToDo[rls]{Datum in tudscr-version.dtx, tudscr-comp.dtx, Handbuch \& README}
\ToDo[rls]{Release auf GitHub und CTAN, Tag für Homepage ändern}

\ListOfToDo
\end{document}
