\RequirePackage[ngerman=ngerman-x-latest]{hyphsubst}
\documentclass[%
  english,ngerman,%
  headings=optiontoheadandtoc,captions=tableheading,numbers=noenddot,%
  chapterpage,cdfoot,%
]{tudscrman}
\usepackage{selinput}
\SelectInputMappings{adieresis={ä},germandbls={ß}}
\usepackage[T1]{fontenc}
\usepackage{fixltx2e}
\usepackage{setspace}
\setstretch{1.1}\recalctypearea
\usepackage{babel}
\usepackage{csquotes}
\usepackage{quoting}
\usepackage{booktabs}
\usepackage{isodate}
\usepackage{metalogo}
\usepackage{hologo}
\usepackage{tikz}
\usepackage{amsmath}
\usepackage{scrhack}

\usepackage{tudscrsupervisor}

%\TUDoptions{ToDo=on}
\newlength{\tempdim}
\newcommand*\vKOMA{v3.11b}
\lstset{%
  inputencoding=utf8,
  extendedchars=true,
  literate=%
    {ä}{{\"a}}1 {ö}{{\"o}}1 {ü}{{\"u}}1
    {Ä}{{\"A}}1 {Ö}{{\"O}}1 {Ü}{{\"U}}1
    {~}{{\textasciitilde}}1 {ß}{{\ss}}1
}
% übler Hack
\makeatletter
\newcommand*\umlautshack{\vphantom{\"A\"O\"U}\ignorespaces}
\apptocmd{\tud@makeuppercase}{\umlautshack}{}{}
\makeatother
\begin{document}
\pagestyle{headings}
\subject{\TUDScript{} \vTUDScript}
\title{%
  Ein \NoCaseChange{\hologo{LaTeX}}"=Bundle für Dokumente\newline
  im neuen \CD der \TnUD%
}
\addtokomafont{subtitle}{\univbn}
\subtitle{Basierend auf \KOMAScript{} \vKOMA}
\author{Falk Hanisch\thanks{\noexpand\href{mailto:\filemail}{\filemail}}}
\faculty{http://tu-dresden.de/cd}
\date{22.04.2014}
\maketitle

\ToDo[v2.1]{%
  Schriftelemente aus \KOMAScript{}~v3.12 auf Titel dokumentieren und ergänzen, 
  Indizes erzeugen (titlepage, thesis \& neue)
%  \begin{Declaration}{\Font{thesis}}
%  \begin{Declaration}{\Font{titlepage}}
%  \printdeclarationlist%
%  \end{Declaration}
%  \end{Declaration}
}


\addchap{\prefacename}
Das im Folgenden beschriebenen Klassen und Pakete wurden für das Erstellen von 
\hologo{LaTeX}"=Dokumenten im \CD der \TnUD entwickelt.%
\footnote{%
  \url{http://tu-dresden.de/cd}\quad
  \url{http://tu-dresden.de/service/publizieren/cd/6_handbuch/index.html}%
}
Sie basieren auf den gerade im deutschsprachigen Raum häufig verwendeten 
\KOMAScript"=Klassen, welche eine Vielzahl von Einstellmöglichkeiten bieten, die 
weit über die Möglichkeiten der \hologo{LaTeX}"=Standardklassen hinausgehen. 
Zusätzlich bietet das hier dokumentierten \TUDScript-Bundle weitere, 
insbesondere das Layout des Dokumentes betreffende Auswahlmöglichkeiten.

Es sei angemerkt, dass die hier beschriebenen Klassen~-- im Gegensatz zur 
\Class{tudbook}"=Klasse von Klaus Bergmann~-- eine Abweichung vom \CD der \TnUD 
zulassen, da dieses gerade unter typographischen Gesichtspunkten durchaus als 
diskussionswürdig zu erachten ist. Mit den entsprechenden Einstellungen kann 
bis auf das Standardlayout der \KOMAScript"=Klassen zurückgestellt werden. 
Inwieweit der Nutzer der \TUDScript"=Klassen von diesen Möglichkeiten Gebrauch 
macht, bleibt ihm selbst überlassen. Ohne die gezielte Verwendung der 
entsprechenden Optionen werden standardmäßig alle Vorgaben des \CDs umgesetzt.

Diese Anleitung soll dazu dienen, eine schnelle Einführung in die neuen Klassen
und Pakete zu ermöglichen. Sie soll Hinweise für eine einfache Installation und 
einen Überblick über die zusätzlich zu den \KOMAScript"=Klassen nutzbaren 
Optionen sowie die neu eingeführten Befehle geben. Dies bedeutet, dass 
Grundkenntnisse in der Verwendung von \hologo{LaTeX} vorausgesetzt werden. 
Sollten 
diese nicht vorhanden sein, wird dem Nutzer zumindest das Lesen der 
Kurzbeschreibung von \hologo{LaTeX} (\File{l2kurz.pdf}) dringend empfohlen. Des 
Weiteren sollte sowohl der Einsteiger als auch der erfahrene Nutzer mindestens 
einmal das \hologo{LaTeX}"=Sündenregister (\File{l2tabu.pdf}) überblickt haben, 
um sehr typische Fehler beim Umgang mit \hologo{LaTeX} zu vermeiden. Ein 
umfangreiches Einführungsskript ist unter \url{http://www.mieser.name/~sbecker/} 
abrufbar, eine ganze Vortragsreihe ist mit dem Link
\url{http://www.fadi-semmo.de/latex/workshop/} zu finden. Antworten auf häufig 
gestellte Fragen werden bei DANTE%
\footnote{\url{http://projekte.dante.de/DanteFAQ/WebHome}} gegeben.

Sollte der Nutzer unsicher oder unerfahren bei der Verwendung der 
\KOMAScript"=Klassen sein, so ist ein Blick in das dazugehörige Anwenderhandbuch
(\File{scrguide.pdf}) sehr zu empfehlen, wenn nicht sogar unumgänglich. Die 
Dokumentation eines speziellen \hologo{LaTeX}"=Paketes lässt übrigens sich sehr 
leicht 
auf der Kommandozeile mit dem Befehl \Path{texdoc\,\emph{<Paketname>}} aufrufen.

Der aktuelle Stand der Klassen und Pakete aus dem \TUDScript-Bundle wurde nach 
bestem Wissen und Gewissen auf Herz und Nieren getestet. Dennoch kann nicht für 
das Ausbleiben von Fehlern garantiert werden. Beim Auftreten eines Problems 
sollte dieses bitte genauso wie Inkompatibilitäten mit anderen Paketen im Forum 
unter
\begin{quote}
\forum%
\end{quote}
gemeldet beziehungsweise geäußert werden. Grundlage für eine schnelle und 
erfolgreiche Fehlersuche ist ein \textbf{lauffähiges Minimalbeispiel}.%
\footnote{\label{fn:mwe}\url{http://www.komascript.de/minimalbeispiel}}
Auf Anfragen ohne dieses werde ich gegebenenfalls verspätet oder gar nicht 
reagieren. Ebenso sind dort auch \emph{Fragen}, \emph{Kritik} und 
\emph{Verbesserungsvorschläge}~-- sowohl das Bundle selbst als auch die 
Dokumentation betreffend~-- gerne gesehen. Da dieses Bundle in meiner Freizeit 
entstanden ist und auch gepflegt wird, bitte ich um Nachsicht, falls ich nicht 
sofort antworte und/oder eine Fehlerkorrektur vornehmen kann.

\makeatletter
\bigskip
\noindent Falk Hanisch\newline
Dresden, \@@date
\makeatother

\tableofcontents



\chapter{Einleitung}
Für die Verwendung der \TUDScript-Klassen in der Version~\vTUDScript{} werden 
zwingend die \KOMAScript"=Klassen der Version~\vKOMA{} sowie die Schriften des 
\CDs \Univers und \DIN benötigt. Die alten Vorlagen von Klaus Bergmann sind für 
die Verwendung nicht notwendig. Allerdings beinhaltet das letztgenannte 
Vorlagen"=Paket weitere Klassen zum Erstellen von Folien und Briefen. Die hier 
vorgestellten Klassen und Pakete sind hauptsächlich für die Erstellung 
wissenschaftlicher Texte und Arbeiten gedacht und sollen die bekannten Vorlagen 
\emph{momentan} nicht ersetzen sondern vielmehr ergänzen. 

Eine Umsetzung des \CDs für die \Class{beamer}"=Klasse sowie für Briefe und 
Geschäftsschreiben auf Basis der \KOMAScript"=Brief"=Klasse \Class{scrlttr2} 
ist bis jetzt leider noch nicht entstanden, soll jedoch langfristig 
bereitgestellt werden. Allerdings existieren bereits im Bundle 
\Class{tudmathposter} für die \Class{beamer}"=Klasse mehrere Stile. Dieses 
Bundle ist sowohl auf \mbox{GitHub} unter \url{https://github.com/tud-cd/} als 
auch auf der \hologo{LaTeX}"=Seite der \TnUD unter
\url{http://tu-dresden.de/service/publizieren/cd/4_latex} zu finden.


\section{Zur Verwendung dieses Handbuchs}
Sämtliche neu definierten Optionen, Umgebungen und Befehle der 
\TUDScript-Klassen und \TUDScript-Pakete werden im Handbuch aufgeführt und 
beschrieben. Am Ende des Dokumentes befinden sich mehrere Indizes, die das 
Nachschlagen oder Auffinden von bisher unbekannten Befehlen oder Optionen 
erleichtern sollen.

Die folgend beschriebenen Optionen können~-- wie ein Großteil der Einstellungen 
der \KOMAScript"=Klassen~-- in der Syntax des \Package{keyval}"=Paketes als 
Schlüssel"=Wert"=Paare bei der Wahl der Dokumentklasse angegeben werden:
\Macro*{documentclass}\POParameter{\PName{Schlüssel}\PValue{=}\PName{Wert}}%
\Parameter{Klasse}.

Des Weiteren eröffnen die \KOMAScript"=Klassen die Möglichkeit der späten 
Optionenwahl. Dadurch können Optionen nicht nur direkt beim Laden als sogenannte 
Klassenoptionen angegeben werden, sondern lassen sich auch noch innerhalb des 
Dokumentes nach dem Laden der Klasse ändern. Die \KOMAScript"=Klassen sehen 
hierfür zwei Befehle vor. Mit \Macro{KOMAoptions}\Parameter{Optionenliste}
kann man beliebig vielen Schlüsseln jeweils genau einen Wert zuweisen, 
\Macro{KOMAoption}\Parameter{Option}\Parameter{Werteliste} erlaubt das 
gleichzeitige Setzen mehrere Werte für genau einen Schlüssel. Äquivalent 
dazu werden für die \emph{zusätzlichen} Optionen der \TUDScript-Klassen mit 
die Befehle \Macro{TUDoptions}\Parameter{Optionenliste} und 
\Macro{TUDoption}\Parameter{Option}\Parameter{Werteliste} definiert.

Die Voreinstellung einer jeden Option ist durch \PValue{preset:\,}\PName{Wert}
bei deren Beschreibung angegeben. Einige dieser Standardwerte sind nicht 
immer gleich sondern werden zusätzlich in Abhängigkeit der genutzten Optionen und
Benutzereinstellungen gesetzt. Diese bedingten Voreinstellungen werden durch
\PValue{preset:\,}\PName{Wert}\PValue{\,|\,}\PName{Bedingung}\PValue{:\,}%
\PName{bedingter~Wert} angegeben.

Jedem Schlüssel wird normalerweise durch den Benutzer ein gewünschter, gültiger 
Wert zugewiesen. Wird ein Schlüssel jedoch ohne Wertzuweisung genutzt, so 
wird~-- falls vorhanden~-- ein vordefinierter Säumniswert gesetzt, welcher in 
der Beschreibung der einzelnen Optionen durch die \PValue{\emph{kursive}} 
Schreibweise gekennzeichnet ist. In den allermeisten Fällen ist der Säumniswert
eines Schlüssels \PValue{true}, er entspricht also normalerweise der Angabe 
\PName{Schlüssel}\PValue{=true}. Mit der expliziten Wertzuweisung 
eines Schlüssels durch den Benutzer werden immer sowohl normale als auch 
bedingte Voreinstellungen überschrieben. Die neben den Optionen neu 
eingeführten Befehle und Umgebungen der Klassen werden im gleichen Stil 
erläutert.

Für den schnellen Einstieg sind in \autoref{sec:mwe} mehrere Minimalbeispiele zu 
finden, in welchen beispielhaft einige Optionen, Befehle und Umgebungen in ihrer 
Anwendung erläutert sind.


\section{Installation des \TUDScript-Bundles}
\label{sec:installation}
Die Installation der \TUDScript-Klassen erfolgt mit einem Skript für das 
jeweilige Betriebssystem. Die Klassen und Pakete sowie deren Dokumentation sind 
zusammen mit den Logos der \TnUD im Archiv \File{tudscr\_\vTUDScript.zip} zu 
finden. Für die Installation der Schriften werden neben den Archiven 
\File{Univers\_PS.zip} und \File{DIN\_Bd\_PS.zip} mit den 
Schriftdateien~-- welche auf Anfrage über das \CD%
\footnote{\url{http://tu-dresden.de/service/publizieren/cd/4_latex}} bestellt 
werden können~-- außerdem die Pakete \Package{fontinst}, \Package{cmbright}, 
\Package{iwona} sowie \Package{eulervm} zwingend benötigt und müssen durch die 
verwendete \hologo{LaTeX}"=Distribution bereitgestellt werden.%
\footnote{%
  Die Möglichkeit, Open Type Schriften aus dem System sehr einfach mit dem Paket
  \Package{fontspec} für \hologo{LuaLaTeX} beziehungsweise \hologo{XeLaTeX} 
  einzubinden, wird momentan genutzt. Eine Schriftinstallation via Skript wäre 
  damit obsolet. Allerdings sind die PostScript"=Schriften für die Kompilierung 
  via \Path{latex \textrightarrow{} dvips \textrightarrow{} ps2pdf}~-- wie es 
  beispielsweise für die Erstellung von Grafiken mit \Package{pstricks} 
  notwendig ist~-- auch weiterhin nötig.
}

\subsection{Installation unter Windows}
%\index{Distribution|!}%
%\index{Distribution!\hologo{TeX}~Live}%
%\index{Distribution!\hologo{MiKTeX}}%
Sollte noch keine \hologo{LaTeX}"=Distribution auf ihrem System installiert 
sein, so rate ich persönlich zur Verwendung von \Distribution{\hologo{TeX}~Live} 
anstelle von \Distribution{\hologo{MiKTeX}}.%
\footnote{%
  Der Vorteil ist, dass diese Distribution von mehreren Autoren gewartet wird, 
  Updates von Paketen und Klassen auf CTAN meist schneller verfügbar sind und 
  zusätzlich ein \textsc{Perl}"=Interpreter sowie \textsc{Ghostscript} 
  mitgeliefert werden, welche die Ad"=hoc"=Verwendung einiger 
  \hologo{LaTeX}"=Pakete vereinfacht beziehungsweise verbessert.
}
Für eine Installation sowohl der \TUDScript-Klassen als auch der dazugehörigen 
Schriften für die Distributionen \Distribution{\hologo{TeX}~Live} oder 
\Distribution{\hologo{MiKTeX}} sollte das mitgelieferte Batch"~Skript 
\File{tudscr\_\vTUDScript\_install.bat} in Verbindung mit der Standalone-Version 
von 7"~zip~(\File{7za.exe}) verwendet werden. Es werden alle notwendigen Dateien 
in das lokale Nutzerverzeichnis der jeweiligen Distribution installiert, falls 
kein anderes Verzeichnis explizit angegeben wird. Vor der Verwendung des 
Skriptes sollte sichergestellt werden, dass sich für eine vollständige 
Installation unter Windows \emph{alle} der folgenden Dateien im selben 
Verzeichnis befinden:
%
\settowidth{\tempdim}{\File{tudscr\_\vTUDScript\_install.bat}}%
\begin{description}[labelwidth=\tempdim,labelsep=1em]
  \item[\File{tudscr\_\vTUDScript.zip}]Archiv mit allen Klassen- und Paketdateien
  \item[\File{tudscr\_\vTUDScript\_install.bat}]Installationsskript
  \item[\File{Univers\_PS.zip}]Archiv mit Schriftdateien für \Univers
  \item[\File{DIN\_Bd\_PS.zip}]Archiv mit Schriftdateien für \DIN
  \item[\File{tudscrfont.zip}]Archiv mit Metriken für die Schriftinstallation   
    via \Package{fontinst}
  \item[\File{7za.exe}]Standalone-Version von 7-zip zum Entpacken der Archive
\end{description}
%
Durch die Verwendung der 7"~zip"=Standalone"=Version kann es möglicherweise zu
einer Warnung durch die Firewall kommen. Für den Fall, dass Sie 
\Distribution{\hologo{MiKTeX}} verwenden ist es eventuell ratsam, die 
\TUDScript-Klassen nicht in das Standardnutzerverzeichnis sondern in ein 
separates, neu angelegtes zu installieren.%
\footnote{%
  Öffnen Sie die Anwendung \Distribution*{\hologo{MiKTeX}~Options}
  (Start \textrightarrow{} Programme \textrightarrow{} MiKTeX~2.x 
  \textrightarrow{} Maintenance \textrightarrow{} Settings)
  und klicken Sie auf den Kartenreiter \emph{Roots}.
}
Sonst muss bei einem Versionsupdate von \Distribution{MiKTeX} das 
\TUDScript-Bundle möglicherweise neu installiert werden, was allerdings kein 
großes Problem darstellen sollte.


Treten bei der Installation wider Erwarten Probleme auf, so sollte das Skript 
von der Kommandozeile aus mit 
\Path{tudscr\_\vTUDScript\_install.bat > tudscr\_\vTUDScript\_install.log} 
aufgerufen werden. Die erstellte Logdatei kann anschließend entweder direkt an 
\Email{\filemail} gesendet oder im Forum unter \forum gepostet werden.

\subsection{Installation unter Linux und OS~X}
Für die Erstellung des Installationsskriptes für Linux und OS~X geht mein Dank 
an Jons-Tobias Wamhoff (\Email{jons@inf.tu-dresden.de}), welcher sich um die 
Portierung des Skriptes von Windows freiwillig zur Verfügung stellte.  Für eine 
Installation sowohl der \TUDScript-Klassen als auch der dazugehörigen Schriften 
sollte das mitgelieferte Skript \File{tudscr\_\vTUDScript\_install.sh} verwendet 
werden. Es werden alle notwendigen Dateien in das lokale Nutzerverzeichnis der 
jeweiligen Distribution installiert. Vor der Verwendung des Skriptes sollte 
sichergestellt werden, dass sich für eine vollständige Installation \emph{alle} 
der folgenden Dateien im selben Verzeichnis befinden:
%
\settowidth{\tempdim}{\File{tudscr\_\vTUDScript\_install.sh}}%
\begin{description}[labelwidth=\tempdim,labelsep=1em]
  \item[\File{tudscr\_\vTUDScript.zip}]Archiv mit allen Klassen- und Paketdateien
  \item[\File{tudscr\_\vTUDScript\_install.sh}]Installationsskript
  \item[\File{Univers\_PS.zip}]Archiv mit Schriftdateien für \Univers
  \item[\File{DIN\_Bd\_PS.zip}]Archiv mit Schriftdateien für \DIN
  \item[\File{tudscrfont.zip}]Archiv mit Metriken für die Schriftinstallation   
    via \Package{fontinst}
\end{description}


\section{Update des \TUDScript-Bundles von \vTUD}
\DeclareClass{tudscrbookold}\DeclareClass{tudscrreprtold}%
\DeclareClass{tudscrartclold}%
\index{Update|!}%
\index{Hauptklassen}
\index{Version v1.0}%
Ist bereits die \TUDScript-\vTUD installiert, so wird dringend zu einer 
Deinstallation dieser geraten. Geschieht dies nicht, könnte es zu Problemen 
kommen. Sollen die obsoleten \TUDScript-Klassen in der \vTUD nach einer 
Aktualisierung weiterhin genutzt werden, so müssen diese erst de"~~und 
anschließend neu installiert werden.

Über den Link \url{http://wwwpub.zih.tu-dresden.de/~fahan/tudscr/} können sowohl 
das Skript zur Deinstallation \File{tudscr\_v1.0\_uninstall.bat} als auch das 
neue Archiv mit den obsoleten Klassen \File{TUD-KOMA-Script\_v1.0\_old.zip}
heruntergeladen werden. Nach der Installation aus diesem sind die alten Klassen 
der \vTUD mit \Class*{tudscrbookold}, \Class*{tudscrreprtold} und 
\Class*{tudscrartclold} weiterhin und parallel zur Version~\vTUDScript{} 
verwendbar. Die aktuelle Version~\vTUDScript{} kann nach der Deinstallation der 
\vTUD wie unter \autoref{sec:installation} beschrieben installiert werden.

Im Vergleich zur \vTUD hat sich für den Benutzer nicht sehr viel verändert. 
Sollten nach dem Umstieg von der \vTUD auf die Version~\vTUDScript{} dennoch 
Probleme auftreten, sollte der Anwender als erstes in \autoref{sec:comp} sehen. 
Hier werden die gemachten Änderungen erläutert und im alten Dokument notwendige 
Anpassungen beschrieben. Sollten dennoch Fehler oder Probleme beim Umstieg auf 
die neue \TUDScript-Version auftreten, ist eine Meldung im Forum unter \forum 
die beste Möglichkeit, um Hilfe zu erhalten.

\ToDo[v2.1]{Beschreibung der Installation des Updates aktivieren}
%\section{Update des \TUDScript-Bundles von Version~\NoCaseChange{v}2.x}
%\index{Update}
%\index{Version!v2.0}
%Sollte bereits mindestens die \TUDScript-Version~v2.0 inklusive der Schriften 
%und der benötigten Logos der \TnUD installiert sein, kann sehr einfach auf 
%eine neuere Version gewechselt werden.
%
%\subsection{Update unter Windows}
%Für eine Aktualisierung der Version unter Windows auf \vTUDScript{} sollten 
%sich folgende Dateien im selben Verzeichnis befinden:
%%
%\settowidth{\tempdim}{\File{tudscr\_\vTUDScript\_update.bat}}%
%\begin{description}[labelwidth=\tempdim,labelsep=1em]
%  \item[\File{tudscr\_\vTUDScript\_update.zip}]Archiv mit allen Klassen- und 
%    Paketdateien
%  \item[\File{tudscr\_\vTUDScript\_update.bat}]Installationsskript
%  \item[\File{7za.exe}]Standalone-Version von 7-zip zum Entpacken der Archive
%\end{description}
%%
%Ist dies der Fall, muss lediglich \File{tudscr\_\vTUDScript\_update.bat}
% ausgeführt werden.
%
%\subsection{Update unter Linux und OS~X}
%Für eine Aktualisierung der Version unter Linux oder OS~X auf \vTUDScript{} 
% sollten sich folgende Dateien im selben Verzeichnis befinden:
%%
%\settowidth{\tempdim}{\File{tudscr\_\vTUDScript\_update.zip}}%
%\begin{description}[labelwidth=\tempdim,labelsep=1em]
%  \item[\File{tudscr\_\vTUDScript\_update.zip}]Archiv mit allen Klassen- und 
%    Paketdateien
%  \item[\File{tudscr\_\vTUDScript\_update.sh}]Installationsskript
%\end{description}
%%
%Ist dies der Fall, muss lediglich \File{tudscr\_\vTUDScript\_update.sh} 
%ausgeführt werden.



\chapter[Die Hauptklassen tudscrbook, tudscrreprt, tudscrartcl]{Die Hauptklassen}
\DeclareClass{tudscrbook}\DeclareClass{tudscrreprt}\DeclareClass{tudscrartcl}
\index{Hauptklassen|!}
Es werden die drei neuen Hauptklassen
%
\begin{description}
\item \Class*{tudscrbook}
\item \Class*{tudscrreprt}
\item \Class*{tudscrartcl}
\end{description}
%
eingeführt, welche auf den \KOMAScript"=Klassen basieren und grundsätzlich alle
deren bekannten Optionen, Umgebungen und Befehle~-- beispielsweise
\Option{BCOR} zur Festlegung der Bindekorrektur oder aber \Option{parskip}
zur Festlegung der Absatzeinstellungen~-- unterstützen. Zusätzlich zu den 
\KOMAScript"=Klassen werden weitere Pakete zwingend benötigt, welche unter 
\autoref{sec:packages:needed} aufgeführt sind und durch den Anwender nicht noch 
zusätzlich geladen werden müssen. 

Es sei hier abermals auf die Anwenderdokumentation (\File{scrguide.pdf}) von 
\KOMAScript{} hingewiesen, viele der folgend beschriebenen Befehle und Optionen 
beziehen sich auf die darin vorgestellten Einstellungsmöglichkeiten. Die 
Anpassungen und Erweiterungen der \KOMAScript"=Klassen an das \CD und die neu 
definierten beziehungsweise geänderten Befehle und Optionen werden im Folgenden 
erläutert.

\begin{Declaration}{\Macro{TUDoptions}\Parameter{Optionenliste}}
\begin{Declaration}{\Macro{TUDoption}\Parameter{Option}\Parameter{Werteliste}}
\printdeclarationlist%
\index{Optionenwahl|!}%
%
Mit diesen Befehlen hat man bei den meisten der neuen Klassenoptionen die 
Möglichkeit, den Wert der Optionen noch nach dem Laden der Klasse zu ändern.
Man kann wahlweise mit der Anweisung \Macro*{TUDoptions} die Werte einer Reihe 
von Optionen ändern. Jede Option der Optionenliste hat dabei die Form
\PName{Option}\PValue{=}\PName{Wert}. Die meisten Optionen besitzen auch einen 
Säumniswert\footnote{engl.: default value}. Versäumt man die Angabe eines 
Wertes~-- verwendet also einfach die Form \PName{Option}~-- so wird automatisch 
dieser Säumniswert angenommen.

Manche Optionen können gleichzeitig mehrere Werte besitzen. Für diese besteht 
die Möglichkeit, mit \Macro*{TUDoption} der einen Option nacheinander eine 
Reihe von Werten zuzuweisen. Die einzelnen Werte sind dabei in der Werteliste 
durch Komma voneinander getrennt.
\end{Declaration}
\end{Declaration}


\section{Die Schriften des \CDs}
\index{Schrift|?}
Das \CD der \TnUD gibt die Verwendung der Schriften \Univers für den Fließtext 
sowie \DIN für das Setzen von Überschriften vor. Im Standardfall wird dies so 
unterstützt. Da jedoch in längeren Texten die Verwendung von Serifenschriften zu 
empfehlen ist, gibt es die Möglichkeit, die eigentlich vorgesehenen Schriften 
nicht zu laden und die Standardschriften beziehungsweise ein anderes 
Schriftpaket zu verwenden. Die Einstellungen und Befehle für den Fließtext sind 
in \autoref{sec:text} zu finden.

Durch das \CD werden keine Schriften für den Mathematiksatz bereitgestellt. Dies 
ist insbesondere für sowohl mathematische als auch natur"~ und 
ingenieurwissenschaftliche Dokumente nicht tragbar. Dieser Mangel wird behoben, 
indem im Mathematikmodus die lateinischen Buchstaben der Hausschriften mit 
griechischen Lettern und mathematischen Symbolen aus anderen Paketen ergänzt 
werden.%
\footnote{%
  \Package{cmbright} sowie \Package{eulervm} für die Schriftfamilie \Univers und 
  \Package{iwona} für die Schrift \DIN%
}
Diese Einstellungen kann natürlich ebenfalls mit der entsprechenden Option 
deaktiviert werden. Dann werden die Standardschriften oder gegebenenfalls die 
eines zusätzlichen Paketes für den mathematischen Satz genutzt. Alle Befehle und 
Optionen für den Mathematiksatz sind in \autoref{sec:math} erläutert. Weitere 
Hinweise zum typographisch guten Mathematiksatz sind außerdem in 
\autoref{sec:mwe:swap} sowie \autoref{sec:mathtype} zu finden.


\subsection{Schriften für den Textsatz}\label{sec:text}
\index{Schrift}\index{Schrift!Fließtext}%
\begin{Declaration}{\Option{cdfont}[\PSet]}[true]%
\printdeclarationlist%
\index{Schrift!Corporate Design}\index{Schrift!Stärke}%
%
Mit dieser Option können durch den Benutzer alle zentralen Schrifteinstellungen 
für die \TUDScript-Klassen vorgenommen werden. Dies betrifft sowohl die 
Schriften für den Fließtext, als auch die Mathematikschriften.
%
\begin{values}
\itemfalse
  Es werden keine Hausschriften sondern die \hologo{LaTeX}"=Standardschriften 
  verwendet und der Benutzer kann beliebige Schriftpakete nutzen.%
  \footnote{%
    Für die Verwendung der klassischen \hologo{LaTeX}"=Schriften, ist das Paket 
    \Package{lmodern} sehr empfehlenswert.%
  }
  Sollte das Layout des \CDs aktiviert sein (siehe \Option{cd}), werden die 
  Überschriften in serifenlosen Majuskeln\footnote{Großbuchstaben} gesetzt.
\itemtrue*[light/lightfont/noheavyfont]
   Es werden die Hausschriften im Stil des \CDs der \TnUD genutzt. 
   Überschriften der obersten Gliederungsebenen bis einschließlich 
   \Macro*{subsubsection} verwenden \DIN, darunter liegende%
   \footnote{\Macro*{paragraph} und \Macro*{subparagraph}} 
   \textubn{Univers~65~Bold}. Für den Fließtext im Dokument kommt 
   \textuln{Univers~45~Light} zum Einsatz. Aus \Package{cmbright} wird die
   \texttt{Schreibmaschinenschrift} verwendet.
\item[heavy/heavyfont]
  Die Schriftstärke der Hausschriften wird erhöht. Die beiden untersten 
  Gliederungsebenen werden in \textuxn{Univers~75~Black} gesetzt, der Fließtext 
  in \texturn{Univers~55~Regular}. Ansonsten entspricht alles der Option 
  \Option*{cdfont}[true]. Die Mathematikschriften werden durch diese Einstellung 
  nicht beeinflusst. Gegebenenfalls sollte mit \Macro{boldmath} auf den fetten 
  Schnitt umgeschaltet werden.
\item[din]
  Mit dieser Einstellung wird die Schrift \DIN in den Überschriften verwendet. 
  Sie ist standardmäßig aktiviert.
\item[nodin]
  Für die Überschriften wird nicht \DIN verwendet. Ist \Option*{cdfont}[true] 
  gewählt, wird \Univers genutzt. Die Schriftstärke ist dabei abhängig von der 
  Einstellung \Option*{cdfont}[light/heavy]. Ist die Verwendung der Schriften 
  des \CDs deaktiviert (\Option*{cdfont}[false]), kommt die fette Schriftstärke 
  der eigenstellten serifenlosen Schriftfamilie zum Einsatz.
\item[serifmath/serif/nosansmath/nosans]  
  Diese Einstellung deaktiviert die Verwendung von serifenlosen Schriften für 
  den mathematischen Satz. Es werden die \hologo{LaTeX}"=Standardschriften 
  verwendet und der Benutzer kann beliebige Schriftpakete für den 
  Mathematikmodus nutzen, siehe \Option{sansmath}[false].
\item[sansmath/sans]
  Es werden serifenlose Mathematikschriften für lateinische und griechische 
  Lettern genutzt, siehe \Option{sansmath}[true].
\item[upgreek/uprightgreek/uprightGreek]
  Die großen griechischen Buchstaben werden im Mathematikmodus aufrecht gesetzt,
  siehe \Option{slantedgreek}[false].
\item[slgreek/slantedgreek/slantedGreek]
  In mathematischen Umgebungen erfolgt die Ausgabe der griechischen Majuskeln 
  kursiv, siehe \Option{slantedgreek}[true].
\end{values}
\end{Declaration}

\subsubsection{Auszeichnungen in Überschriften}
\index{Schrift!Überschriften}%
\index{Überschriften}\index{Schriftauszeichnung}%
\begin{Declaration}{\Macro{MakeTextUppercase}\Parameter{Text}}%
\begin{Declaration}{\Macro{NoCaseChange}\Parameter{Text}}%
\printdeclarationlist%
%
Diese beiden Befehle stammen aus dem Paket \Package{textcase}. Der Befehl 
\Macro*{MakeTextUppercase} setzt den Text seines Argumentes in Majuskeln. Die 
Überschriften der Gliederungsebenen bis einschließlich \Macro*{subsubsection} 
werden mit diesem Befehl in Großbuchstaben der Schrift \DIN gesetzt. Es kann 
jedoch unter Umständen sein, dass ein oder mehrere Kleinbuchstaben erhalten 
bleiben sollen. Für diesen Fall ist der Befehl \Macro{NoCaseChange} zu nutzen.
%
\begin{Example}
In einer Kapitelüberschrift wird ein einzelnes Wort in Kleinbuchstaben 
geschrieben:
\begin{code}[escapechar=§]
\chapter{§Ü§berschrift mit \NoCaseChange{kleinem} Wort}
\end{code}
\end{Example}
%
\end{Declaration}
\end{Declaration}

\begin{Declaration}{\Macro{ifdin}\Parameter{Dann-Teil}\Parameter{Sonst-Teil}}%
\printdeclarationlist%
Des Weiteren wird der Befehl \Macro*{ifdin} definiert. Dieser prüft, ob die 
Schriftfamilie \DIN aktiv ist und führt in diesem Fall \Parameter{Dann-Teil} 
aus, andernfalls \Parameter{Sonst-Teil}. Dies ist beispielsweise bei 
Überschriften sinnvoll, wenn zwischen der eigentlichen Ausgabe im Fließtext und 
dem Eintrag für Inhaltsverzeichnis und/oder Kolumnentitel unterschieden werden 
soll.
\end{Declaration}

\subsubsection{Auszeichnungen im Text}
\begin{Declaration}{\Macro{univln}}
\begin{Declaration}{\Macro{textuln}\Parameter{Text}}
\begin{Declaration}{\Macro{univrn}}
\begin{Declaration}{\Macro{texturn}\Parameter{Text}}
\begin{Declaration}{\Macro{univbn}}
\begin{Declaration}{\Macro{textubn}\Parameter{Text}}
\begin{Declaration}{\Macro{univxn}}
\begin{Declaration}{\Macro{textuxn}\Parameter{Text}}
\begin{Declaration}{\Macro{univls}}
\begin{Declaration}{\Macro{textuls}\Parameter{Text}}
\begin{Declaration}{\Macro{univrs}}
\begin{Declaration}{\Macro{texturs}\Parameter{Text}}
\begin{Declaration}{\Macro{univbs}}
\begin{Declaration}{\Macro{textubs}\Parameter{Text}}
\begin{Declaration}{\Macro{univxs}}
\begin{Declaration}{\Macro{textuxs}\Parameter{Text}}
\begin{Declaration}{\Macro{dinbn}}
\begin{Declaration}{\Macro{textdbn}\Parameter{Text}}
\settowidth{\tempdim}{\Macro{textuln}\Parameter{Text}}%
\addtolength{\tempdim}{\dimexpr 2\tabcolsep+2\arrayrulewidth-\textwidth}%
\printdeclarationlist(%
  \begin{minipage}{-\tempdim}%
  \centering%
  \begin{tabularm}{3}%
    \toprule%
    \textbf{Schriftart}                  & \textbf{Schalter}
      & \textbf{Textkommando}\tabularnewline
    \midrule
    \textuln{Univers 45 Light}           & \Macro*{univln}{}
      & \Macro*{textuln}\Parameter{Text}\tabularnewline
    \texturn{Univers 55 Regular}         & \Macro*{univrn}{}
      & \Macro*{texturn}\Parameter{Text}\tabularnewline
    \textubn{Univers 65 Bold}            & \Macro*{univbn}{}
      & \Macro*{textubn}\Parameter{Text}\tabularnewline
    \textuxn{Univers 75 Black}           & \Macro*{univxn}{}
      & \Macro*{textuxn}\Parameter{Text}\tabularnewline
    \textuls{Univers 45 Light Oblique}   & \Macro*{univls}{}
      & \Macro*{textuls}\Parameter{Text}\tabularnewline
    \texturs{Univers 55 Regular Oblique} & \Macro*{univrs}{}
      & \Macro*{texturs}\Parameter{Text}\tabularnewline
    \textubs{Univers 65 Bold Oblique}    & \Macro*{univbs}{}
      & \Macro*{textubs}\Parameter{Text}\tabularnewline
    \textuxs{Univers 75 Black Oblique}   & \Macro*{univxs}{}
      & \Macro*{textuxs}\Parameter{Text}\tabularnewline
    \DIN & \Macro*{dinbn}{}
      & \Macro*{textdbn}\Parameter{Text}\tabularnewline
    \bottomrule%
    \allcolumnpar{\footnotesize\vskip0pt%
       Die Schrift \DIN darf laut \CD nur mit Majuskeln (Großbuchstaben) 
       verwendet werden. Wird diese Schrift manuell verwendet, sollte dies mit 
       \Macro{MakeTextUppercase}\PParameter{\Macro{textdbn}\Parameter{Text}}  
       geschehen. Sollen dabei im Argument einzelne Teile zwingend klein 
       geschrieben werden, wird der Befehl \Macro{NoCaseChange} benötigt.
    }
  \end{tabularm}%
  \end{minipage}%
)%
\index{Schrift!Befehle}\index{Schrift!Schalter}%
%
Unabhängig davon, welche Schriftfamilie verwendet wird, können die Schriften 
des \CDs jederzeit entweder mit einem Textschalter oder mit einem Textkommando
innerhalb des Dokumentes genutzt werden.

Ein Textschalter wirkt sich ohne Maßnahmen global auf das Dokument aus. Wird er 
jedoch innerhalb einer Gruppe verwendet, so werden die Auswirkungen durch diese 
lokal begrenzt.\footnote{\Macro*{begingroup} und \Macro*{endgroup}} Bei einem 
Textkommando hingegen erfolgt die Änderung der Schriftart nur für das angegebene 
Argument. Deshalb ist die Verwendung der letzteren Variante vorzuziehen.  
\end{Declaration}
\end{Declaration}
\end{Declaration}
\end{Declaration}
\end{Declaration}
\end{Declaration}
\end{Declaration}
\end{Declaration}
\end{Declaration}
\end{Declaration}
\end{Declaration}
\end{Declaration}
\end{Declaration}
\end{Declaration}
\end{Declaration}
\end{Declaration}
\end{Declaration}
\end{Declaration}

\subsection{Schriften für den Mathematiksatz}\label{sec:math}
\index{Schrift!Mathematiksatz}\index{Mathematiksatz|!}%
\index{Schrift!Griechische Buchstaben}\index{Griechische Buchstaben}%
%Im Vergleich zur \vTUD wurden die Schriften für den Mathematiksatz stark 
%erweitert. Um die Schriftfamilie \Univers zu ergänzen, werden unter anderem die 
%griechischen Buchstaben und einige Symbole aus dem Paket \Package{cmbright} 
%verwendet. Weitere Symbole werden entweder generisch erzeugt oder aus dem Paket 
%\Package{eulervm} entnommen. Die Schrift \DIN wird um Symbole aus dem Paket 
%\Package{iwona} ergänzt. Damit ist es nun erstmals möglich, die Schriften des 
%\CDs vernünftig für den mathematischen Satz zu nutzen.
%
%Weitere Hinweise zum typographisch guten Mathematiksatz sind außerdem in 
%\autoref{sec:mwe:swap} sowie \autoref{sec:mathtype} zu finden.
\begin{Declaration}{\Option{sansmath}[\PBoolean]}%
  [true][\Option{cdfont}[false]:false]
\printdeclarationlist%
%
Diese Option dient zur Verwendung serifenloser Mathematikschriften. Dafür werden 
zum einen die griechischen Buchstaben aus \Package{cmbright} und zum anderen die 
Symbole aus dem \Package{eulervm} verwendet. Für die lateinischen Buchstaben 
wird \Univers genutzt. Ein Umschalten auf Serifenlose und zurück innerhalb des 
Dokumentes ist~-- beispielsweise in einer Abbildung oder in einer Tabelle~-- 
durch \Macro{TUDoptions}\PParameter{\Option*{sansmath}[true]} und 
\Macro{TUDoptions}\PParameter{\Option*{sansmath}[false]} möglich. Mit
\Macro{boldmath} kann auf fette Mathematikschriften umgeschaltet werden.

Mit der Einstellung \Option*{sansmath}[false] wird auf die Standardschriften
für den Mathematikmodus zurückgeschaltet. Sollen stattdessen andere serifenlose 
Mathematikschriften genutzt werden, so sei auf \Package{sansmath}, 
\Package{sansmathfonts}, \Package{mathastext}, \Package{sfmath} sowie 
\Package{sansmathaccent} verwiesen.
%
\begin{values}
\itemfalse
  Es werden die normalen \hologo{LaTeX}"=Serifenschriften beziehungsweise die 
  Schriften beliebig nutzbarer Pakete für den Mathematiksatz verwendet.
\itemtrue*
  Die serifenlose Mathematikschriften werden aktiviert.
\end{values}
\end{Declaration}

\subsubsection{Griechischen Buchstaben}\label{sec:greek}
\index{Griechische Buchstaben}%\index{Griechische Buchstaben!Neigung}%
\begin{Declaration}{\Macro{varDelta}}
\begin{Declaration}{\Macro{varTheta}}
\begin{Declaration}{\Macro{varLambda}}
\begin{Declaration}{\Macro{varXi}}
\begin{Declaration}{\Macro{varPi}}
\begin{Declaration}{\Macro{varSigma}}
\begin{Declaration}{\Macro{varUpsilon}}
\begin{Declaration}{\Macro{varPhi}}
\begin{Declaration}{\Macro{varPsi}}
\begin{Declaration}{\Macro{varOmega}}
\begin{Declaration}{\Macro{upDelta}}
\begin{Declaration}{\Macro{upTheta}}
\begin{Declaration}{\Macro{upLambda}}
\begin{Declaration}{\Macro{upXi}}
\begin{Declaration}{\Macro{upPi}}
\begin{Declaration}{\Macro{upSigma}}
\begin{Declaration}{\Macro{upUpsilon}}
\begin{Declaration}{\Macro{upPhi}}
\begin{Declaration}{\Macro{upPsi}}
\begin{Declaration}{\Macro{upOmega}}
\index{Schrift!Griechische Buchstaben}\index{Griechische Buchstaben}%
\settowidth{\tempdim}{\Macro{varUpsilon}}%
\addtolength{\tempdim}{\dimexpr 2\tabcolsep+2\arrayrulewidth-\textwidth}%
\printdeclarationlist(%
  \begin{minipage}{-\tempdim}%
    \newcommand\tablecontent{}   
    \newcommand*\greekLetters{%
      Delta,Theta,Lambda,Xi,Pi,Sigma,Upsilon,Phi,Psi,Omega%
    }%
    \def\do#1{\appto\tablecontent{%
      \Macro*{var#1} & $\csuse{var#1}$ & & 
      \Macro*{up#1} & $\csuse{up#1}$\tabularnewline
    }}%
    \expandafter\docsvlist\expandafter{\greekLetters}%
    \centering%
    \vspace{\intextsep}\noindent
    \begin{tabularm}{5}
      \toprule%
      \textbf{Befehl (kursiv)} & \textbf{Symbol} & &
      \textbf{Befehl (aufrecht)} & \textbf{Symbol}
      \tabularnewline\midrule\tablecontent\bottomrule%
      \allcolumnpar{\footnotesize\vskip0pt%
        Die Befehle \Macro*{up}\PName{Name} und \Macro*{var}\PName{Name}
        werden normalerweise durch einige Pakete, unter anderem auch von 
        \Package{cmbright} oder \Package{amsmath}, bereitgestellt.
      }
    \end{tabularm}
  \end{minipage}%
)%
Für die \TUDScript-Klassen werden griechische Majuskeln sowohl in aufrechter als 
auch in geneigter Form bereitgestellt. Unabhängig von den Einstellungen für die 
Optionen \Option*{sansmath} und \Option*{slantedgreek} können sowohl kursive als 
auch aufrechte griechischen Großbuchstaben im Mathematikmodus direkt verwendet 
werden. Dies ist nützlich, um zwischen kursiven Variablen und aufrechten 
Konstanten zu unterscheiden. Die griechischen Minuskeln sind leider nur in der 
kursiven Variante verfügbar.
\end{Declaration}
\end{Declaration}
\end{Declaration}
\end{Declaration}
\end{Declaration}
\end{Declaration}
\end{Declaration}
\end{Declaration}
\end{Declaration}
\end{Declaration}
\end{Declaration}
\end{Declaration}
\end{Declaration}
\end{Declaration}
\end{Declaration}
\end{Declaration}
\end{Declaration}
\end{Declaration}
\end{Declaration}
\end{Declaration}

\begin{Declaration}{\Option{slantedgreek}[\PBoolean]}%
  [true][\Option{cdfont}[false]:false]
\printdeclarationlist%
\index{Schrift!Griechische Buchstaben}\index{Griechische Buchstaben}%
\index{Griechische Buchstaben!Neigung}%
%
Die Option ändert die standardmäßige Neigung der griechischen Großbuchstaben im 
Mathematikmodus bei der Verwendung der Befehle \Macro*{Delta}, \Macro*{Theta}, 
\Macro*{Lambda}, \Macro*{Xi}, \Macro*{Pi}, \Macro*{Sigma}, \Macro*{Upsilon}, 
\Macro*{Phi}, \Macro*{Psi} und \Macro*{Omega}. Wie unabhängig von der Option 
\Option*{slantedgreek} gezielt kursive und aufrechte Buchstaben gesetzt werden 
können, ist \vpageref{sec:greek} beschrieben.
%
\begin{values}
\itemfalse
  Die griechischen Majuskeln werden wie bei den Standardklassen aufrecht gesetzt.
\itemtrue*
  Die Ausgabe der griechischen Großbuchstaben erfolgt kursiv.
\end{values}
\end{Declaration}

\section{Das Layout des \CDs}
Das Hauptaugenmerk der neuen Klassen liegt auf der Umsetzung des \CDs der
\TnUD für \hologo{LaTeX}. Ein großer Teil der definierten Optionen und Befehle
dient genau dazu und wird folgend beschrieben.

Einige spezielle Seiten werden im prägnanten Stil mit dem Logo der \TnUD und der 
dazugehörigen Kopfzeile mit Querbalken gesetzt. Dies betrifft insbesondere die 
\hyperref[sec:title]{Titelseite} und die \hyperref[sec:chapter]{Kapitelseiten}. 
Außerdem können mit der \Environment{tudpage}-Umgebung weitere Seiten im 
gleichen Stil erzeugt werden. Sollte das Paket \Package{tudscrsupervisor} 
verwendet werden und mit den entsprechenden Befehlen oder Umgebungen eine 
Aufgabenstellung, ein Gutachten oder ein Aushang erstellt werden, so erscheinen 
auch diese in besagtem Seitenstil.


\subsection{Das Erscheinungsbild von Titel, Teilen und Kapiteln}
\begin{Declaration}{\Option{cd}[\PSet]}[true]
\printdeclarationlist%
\index{Layout}%
%
Diese Option bestimmt, ob und wie das \CD der \TnUD verwendet wird. Sie hat
Einfluss auf die Ausprägung für Titel"~, Teil"~, und Kapitelseiten.
%
\begin{values}
\itemfalse
  Diese Einstellung erzeugt das Standard"=Verhalten der \KOMAScript"=Klassen, 
  es wird kein \CD genutzt.
\itemtrue*[standard/simple/monochrom]
  Das Layout für Titel"~, Teil"~ und Kapitelseiten ist im \CD, es wird 
  schwarze Schrift für Titel, Teil"~ und Kapitelüberschriften sowie im 
  Seitenkopf verwendet.
\item[lite/light/pale]
  Die Einstellung entspricht weitestgehend der Option \Option*{cd}[true], 
  allerdings wird die primäre Hausfarbe \Color{HKS41} anstelle schwarzer 
  Schrift genutzt.
\item[color/colour/full]
  Der Titel sowie Teil"~ und Kapitelseiten werden allesamt farbig und im \CD 
  gestaltet, der Seitenkopf wird in der primären Hausfarbe \Color{HKS41} gesetzt.
\end{values}
\end{Declaration}

\begin{Declaration}{\Length{chapterheadingvskip}}
\printdeclarationlist%
\index{Kapitelseiten}\index{Layout!Kapitelseiten}%
\index{Überschriften!Position}%
Mit dieser Länge kann die vertikale Position der Kapitelüberschriften bei 
deaktivierter Kapitelseite (\Option{chapterpage}[false]) angepasst werden. 
Normalerweise werden diese im Layout relativ tief im Textbereich gesetzt. Mit 
negativen Werten wird die Kapitelüberschrift nach oben verschoben, positive 
Werte setzen diese dementsprechend tiefer. Beim Verschieben nach oben, sollte 
darauf geachtet werden, dass diese sich danach noch innerhalb des Satzspiegels 
befinden.
\end{Declaration}

\subsubsection{Vakatseiten/Leerseiten}
\index{Leerseiten}%
Automatisch erzeugte Vakatseiten~-- auch absichtliche Leerseiten genannt~-- 
findet man in Dokumenten mit den aktivierten Optionen \Option{twoside} und 
\Option{open}[right]\footnote{Standard bei \Class{tudscrbook}} beziehungsweise 
\Option{open}[left] beim Beginn von Teilen und Kapiteln. Für diese kann der 
Seitenstil mit der \KOMAScript"=Option \Option{cleardoublepage} eingestellt 
werden.

\begin{Declaration}{\Option{cleardoublespecialpage}[\PSet]}[true]%
\printdeclarationlist%
\index{Teileseiten}\index{Layout!Teileseiten}%
\index{Kapitelseiten}\index{Layout!Kapitelseiten}%
\index{Satzspiegel!doppelseitig}\index{Layout!Rückseiten}%
  Diese Option hat lediglich Auswirkungen bei aktiviertem doppelseitigem Satz
  und ausschließlich rechts eröffnenden Seiten für Teile beziehungsweise 
  Kapitel.\footnote{\Option{twoside} und \Option{open}[right]}
  Dann kann das Aussehen der darauffolgenden, linken Seite~-- sprich der 
  Rückseite~-- beeinflusst werden. Das Normalverhalten der \KOMAScript"=Klassen 
  sieht vor, dass nach einem Teil die Rückseite unabhängig von der Einstellung 
  für \Option{cleardoublepage} immer als vollständig leere Seite ohne Kopf~ oder 
  Fußzeilen gesetzt wird.
  
  Diese Einstellung erlaubt es, dieses Normalverhalten zu deaktivieren und für 
  die Seite nach der Teileseite~-- und abhängig von \Option{chapterpage} 
  auch nach einem Kapitelanfang auf einer separaten Seite~-- den Seitenstil der 
  Option \Option{cleardoublepage} zu übernehmen. Des Weiteren kann auch ein 
  anderer, beliebiger, bereits definierter Seitenstil gewählt werden. Außerdem
  kann im farbigen Layout die Rückseite in der gleichen Farbe wie die 
  Vorderseite von Teil oder Kapitel gesetzt werden. \notudscrartcl
  %
  \begin{values}
  \itemfalse
    Die Rückseiten sind vollständig leere Seiten, unabhängig von Option
    \Option{cleardoublepage}.
  \itemtrue*
    Der Seitenstil der Rückseite von Teilen und gegebenenfalls Kapiteln 
    entspricht der Einstellung von \Option{cleardoublepage} für Vakatseiten.
  \item[current]
    Es wird der aktuell definierte Seitenstil (\Macro{pagestyle}) für die 
    erzeugte Rückseite verwendet.
  \item[color/colour]
    Im farbigen Layout ist auch die Rückseite von Teilen und Kapiteln farbig, 
    siehe \Option{clearcolor}.
  \makeatletter\item@values[\PName{Seitenstil}\textit{:}]\makeatother
    Mit der Angabe von \Option*{cleardoublespecialpage}[\PName{Seitenstil}] kann 
    ein beliebiger, bereits definierter Seitenstil für die Rückseite nach 
    Teilen und Kapiteln verwendet werden.
  \end{values}
\end{Declaration}

\begin{Declaration}{\Option{clearcolor}[\PBoolean]}[false]%
\printdeclarationlist%
\index{Titel}\index{Layout!Titel}%
\index{Teileseiten}\index{Layout!Teileseiten}%
\index{Kapitelseiten}\index{Layout!Kapitelseiten}%
\index{Satzspiegel!doppelseitig}\index{Leerseiten}%
  Sollte beim farbigen Layout die Optionen \Option{twoside} und 
  \Option{open}[right] gesetzt sein,\footnote{Standard bei \Class{tudscrbook}} 
  so werden beim Aktivieren dieser Option die Rückseiten von Teilen~-- und je 
  nach Einstellung von \Option{chapterpage} gegebenenfalls auch von Kapiteln~-- 
  farbig gesetzt.%
  \footnote{%
    Dies führt beim kolorierten Druck zu farbigen Blättern (Vorder"~ und 
    Rückseite) der entsprechenden Layoutelemente.
  }
  Die Option wirkt sich ebenfalls auf die Rückseite des Titels aus.%
  \footnote{%
    siehe \Macro{uppertitleback} und \Macro{lowertitleback} der 
    \KOMAScript"=Dokumentation (\File{scrguide.pdf})
  }
  Der Stil dieser zusätzlich eingefügten Rückseiten ist abhängig von 
  \Option{cleardoublespecialpage}.
  \begin{values}
  \itemfalse
    Es werden weiße Rückseiten bei Titel, Teilen und gegebenenfalls Kapiteln 
    erzeugt.
  \itemtrue*
    Die rückwärtigen Seiten der genannten Layoutelemente sind farbig.
  \end{values}
\end{Declaration}

\subsubsection{Unterschiedliche Einstellungen Titel, Teile und Kapitel}
Das Verhalten aller Elemente%
\footnote{%
  Titelseite (\Macro{maketitle}), Teileseite (\Macro{part}, \Macro{addpart}),
  Kapitelseite (\Macro{chapter}, \Macro{addchap})%
}
wird normalerweise von der Option \Option*{cd}[\PSet] bestimmt. Bedarfsweise 
können einzelne Elemente aber auch individuell mit abweichenden Wertzuweisungen 
angepasst werden. Soll ein bestimmtes Layoutelement anders erscheinen als der 
Rest des Dokumentes, so kann der entsprechende Wert mit Hilfe der folgenden 
drei Optionen überschrieben werden.

\begin{Declaration}{\Option{cdtitle}[\PSet]}
\printdeclarationlist%
\index{Titel}\index{Layout!Titel}%
%
Der Wert des Schlüssels \Option*{cd} kann für die Titelseite separat 
überschrieben werden. Damit kann zwischen dem Standardtitel~-- welcher durch
\KOMAScript{} bereitgestellt wird~-- und dem Titel im \CD umgeschaltet 
werden. Die neue Titelseite unterstützt alle Befehle für den Titel, welche durch
\KOMAScript{} definiert werden.%
\footnote{\raggedright%
  \Macro{extratitle}\Parameter{Schmutztitel},\Macro{titlehead}\Parameter{Kopf},
  \Macro{subject}\Parameter{Typisierung},\Macro{title}\Parameter{Titel},
  \Macro{subtitle}\Parameter{Untertitel},\Macro{author}\Parameter{Autor},
  \Macro{date}\Parameter{Datum},\Macro{publishers}\Parameter{Verlag},
  \Macro{and} und \Macro{thanks}\Parameter{Fußnote} sowie
  \Macro{uppertitleback}\Parameter{Titelrückseitenkopf},
  \Macro{lowertitleback}\Parameter{Titelrückseitenfuß}
  und \Macro{dedication}\Parameter{Widmung}
}
Sie wird ebenfalls mit \Macro{maketitle} erzeugt. Hierzu sei ergänzend auf
\autoref{sec:title} verwiesen.
\end{Declaration}

\begin{Declaration}{\Option{cdpart}[\PSet]}
\printdeclarationlist%
\index{Teileseiten}\index{Layout!Teileseiten}%
%
Für die Teileseiten kann der Wert des Schlüssels \Option*{cd} separat 
überschrieben und somit deren Layout%
\footnote{\label{fn:layout}%
  \KOMAScript"=Layout beziehungsweise monochromes oder farbiges Erscheinungsbild 
  im \CD%
}
beeinflusst werden, welches bei der Benutzung der Befehle \Macro{part} 
beziehungsweise \Macro{addpart} und deren Sternversionen genutzt wird.
\end{Declaration}

\begin{Declaration}{\Option{cdchapter}[\PSet]}
\printdeclarationlist%
\index{Kapitelseiten}\index{Layout!Kapitelseiten}%
%
Für Kapitelseiten kann der Schlüsselwert \Option*{cd} ebenfalls angepasst und 
damit das Erscheinungsbild\footref{fn:layout} geändert werden, das bei der 
Verwendung von \Macro{chapter} beziehungsweise \Macro{addchap} und den 
dazugehörigen Sternversionen genutzt wird.
\end{Declaration}

\begin{Example}
Soll die Titelseite in Farbe, der Rest des Dokumentes allerdings in schwarzer 
Schrift gesetzt werden, so kann dies folgendermaßen erreicht werden:
\begin{code}[escapechar=§]
\documentclass[cd=true,cdtitle=color]{§\PName{Dokumentklasse}§}
\end{code}
\end{Example}

\subsection{Die Kopfzeile}
\index{Kopfzeile}\index{Layout!Kopfzeile}%
\begin{Declaration}{\Macro{faculty}\Parameter{Fakultät}}
\begin{Declaration}{\Macro{department}\Parameter{Einrichtung}}
\begin{Declaration}{\Macro{institute}\Parameter{Institut}}
\begin{Declaration}{\Macro{chair}\Parameter{Lehrstuhl}}
\begin{Declaration}{\Macro{extraheadline}\Parameter{Textzeile}}
\printdeclarationlist%
\index{Kopfzeile!Felder}%
\index{Querbalken}\index{Layout!Querbalken}
%
Für den Seitenstil des \CDs der \TnUD typisch ist die Kopfzeile mit dem 
charakteristischen Querbalken. In dieser wird~-- falls angegeben~-- in fetter 
Schrift die Fakultät ausgegeben, danach folgen durch Kommas getrennt die 
Einrichtung, das Institut und der Lehrstuhl beziehungsweise die Professur. 
Sollte der Platz in der ersten Zeile nicht ausreichen, erfolgt ein automatischer 
Zeilenumbruch.

In besonderen Ausnahmefällen erlaubt das \CD die Angabe einer zusätzlichen
zweiten beziehungsweise dritten Zeile, welche weitere, frei wählbare Angaben 
enthält. Diese kann mit dem Befehl \Macro*{extraheadline}\Parameter{Textzeile} 
definiert werden.
\end{Declaration}
\end{Declaration}
\end{Declaration}
\end{Declaration}
\end{Declaration}

\begin{Declaration}{%
  \Macro{headlogo}\LParameter\Parameter{Dateiname}%
}
\printdeclarationlist%
\index{Zweitlogo|?}\index{Layout!Zweitlogo}%
%
Neben dem Logo der \TnUD darf zusätzlich ein Zweitlogo im Kopf verwendet 
werden. Dieses lässt sich mit diesem Befehl einbinden. Normalerweise wird es
ohne weitere Angaben auf die Höhe der Erstlogos skaliert. Über das optionale 
Argument können weitere Formatierungsbefehle an den verwendeten Befehl 
\Macro{includegraphics} aus dem \Package{graphicx}-Paket durchgereicht werden.
Sollte über \Option{ddc} das Logo von \DDC eingebunden worden sein, so wird dies 
mit diesem Befehl überschrieben.
\end{Declaration}

\begin{Declaration}{\Option{widehead}[\PBoolean]}%
  [false][\Option{cd}[color]:true]%
\printdeclarationlist%
\index{Querbalken}\index{Layout!Querbalken}%
%
Für die \TUDScript-Klassen ist ein Seitenlayout entstanden, welche den Kopf des
\CDs umsetzt. Dieser besteht aus dem Logo der \TnUD sowie einem darunter 
befindlichen Querbalken, in welchem Fakultät, Einrichtung, Institut und 
Lehrstuhl%
\footnote{%
  \Macro{faculty}, \Macro{department}, \Macro{institute} sowie \Macro{chair}%
}
aufgeführt werden können. Bei der Ausprägung dieses Balkens gibt es zwei 
Varianten. Die Außenlinien laufen entweder bis zum Text"~ oder bis zum 
Blattrand. Die letztere Variante kann für den Fall, dass ein randloser Ausdruck 
technisch nicht möglich ist, Probleme bereiten. Deshalb kann mit der Option 
\Option*{widehead} die Breite des Querbalkens angepasst werden. Normalerweise 
ist der Balken auf die Textbreite begrenzt, lediglich im farbigen Layout wird 
dieser standardmäßig bis zum Blattrand verlängert.
%
\begin{values}
\itemfalse
  Der Querbalken im Kopf erstreckt sich nur über den Textbereich.
\itemtrue*
  Die horizontale Ausdehnung des Querbalkens erstreckt sich bis an den 
  Blattrand.\footnote{Voreinstellung bei \Option{cd}[color]} 
\end{values}
\end{Declaration}

\subsection{Der Titel}\label{sec:title}
\index{Titel|!(}
\begin{Declaration}{\Macro{makecover}\OLParameter{cdlayout}}
\begin{Declaration}{\Key{\Macro{makecover}}{cdlayout}[\PBoolean]}
\begin{Declaration}{\Key{\Macro{makecover}}{cd}[\PSet]}
\begin{Declaration}{\Key{\Macro{makecover}}{cdfont}[\PSet]}
\begin{Declaration}{\Key{\Macro{makecover}}{widehead}[\PBoolean]}
\printdeclarationlist%
\index{Umschlagseite|!}%
\index{Titel!Umschlagseite}\index{Layout!Umschlagseite}%
Zusätzlich zum Titel selbst~-- bestehend aus möglichem Schmutztitel, der 
eigentlichen Titelseite und der nachgelagerten Elementen, welche alle mit dem 
Befehl \Macro{maketitle} ausgegeben werden~-- kann eine Umschlagseite erzeugt 
werden. Der Titel selbst gehört immer zum Buchblock und wird daher im gleichen 
Satzspiegel gesetzt. Dem entgegen steht das Cover, welches zumeist in einem 
separaten Layout erscheint. Auf diesem werden lediglich der Titel des 
Dokumentes, die Typisierung durch \Macro{thesis} und/oder \Macro{subject} sowie 
der Autor oder respektive die Autoren ausgegeben.

Der Satzspiegel der Umschlagseite kann mit dem Parameter
\Key*{\Macro{makecover}}{cdlayout}[\PBoolean] im optionalen Argument geändert 
werden. Standardmäßig ist \Key*{\Macro{makecover}}{cdlayout}[true] 
gesetzt, was dazu führt, dass das Cover~-- unabhängig von der Option 
\Option{geometry}~-- im asymmetrischen Satzspiegel des \CDs gesetzt wird. Ist 
\Key*{\Macro{makecover}}{cdlayout}[false] gewählt, so wird die Umschlagseite im
gleichen Satzspiegel gesetzt, wie das restliche Dokument. Der gewünschte Wert 
des Parameters kann auch direkt als optionales Argument ohne den dazugehörigen 
Schlüssel angegeben werden. Die anderen Parameter entsprechen in ihrem Verhalten 
prinzipiell den gleichnamigen Klassenoptionen.%
\footnote{%
  Dies betrifft \Option'{cd}, \Option'{cdfont} und \Option'{widehead}.
  Das Verhalten sowie die jeweils gültigen Wertzuweisungen können bei den
  Beschreibungen der jeweiligen Option in den entsprechenden Abschnitten des 
  Handbuchs nachgelesen werden.
}
Die Einstellungen dieser Parameter wirken sich jedoch nur lokal und einzig auf 
die Umschlagseite aus.
\end{Declaration}
\end{Declaration}
\end{Declaration}
\end{Declaration}
\end{Declaration}

\begin{Declaration}{\Macro{maketitle}\OParameter{Seitenzahl}}
\begin{Declaration}{%
  \Macro{maketitleonecolumn}\OParameter{Seitenzahl}\OParameter{Einspaltentext}%
}
\printdeclarationlist%
\index{Layout!Titel}%
\index{Satzspiegel!doppelseitig}%
\index{Zweispaltensatz}%
%
Der Befehl \Macro*{maketitle} setzt für \Option{cdtitle}[false] den normalen 
\KOMAScript"=Titel{}, ansonsten wird die Titelseite im \CD der \TnUD 
erzeugt. Diese ist im Vergleich zum Standardtitel um eine Vielzahl von 
Feldern erweitert und erlaubt insbesondere die Angabe von Daten für das 
Deckblatt einer akademischen Abschlussarbeit. Die einzelnen Felder werden 
nachfolgend erläutert. Das optionale Argument erlaubt, wie bei den 
\KOMAScript"=Klassen, die Änderung der Seitenzahl der Titelseite. Wird das 
Dokument doppelseitig und mit rechts öffnenden Kapiteln gesetzt,%
\footnote{%
  \Option{twoside} und \Option{open}[right], Standard für \Class{tudscrbook}
}
so wird zusätzlich die Option \Option{clearcolor} beachtet.

Für die \TUDScript-Klassen gibt es mit \Macro*{maketitleonecolumn} einen Befehl, 
welcher es ermöglicht, im zweispaltigen Satz (\Option{twocolumn}) die 
Titelseite selbst einspaltig zu setzen. Auch hier kann die Seitenzahl optional 
geändert werden. Soll nach dem Titel zusätzlich auch noch eine weitere 
Textpassage~-- beispielsweise eine Kurzfassung~-- einspaltig gesetzt werden, 
so kann man dafür ebenfalls das optionale Argument nutzen. Für die 
gleichzeitige Verwendung beider Möglichkeiten muss erst die gewünschte 
Seitenzahl und danach der Text jeweils in eckigen Klammern angegeben werden.
\ToDo[v2.1]{Fehler mit \Package*{scrlayer} beheben und Warnung entfernen}
\Attention In der Version~\vTUDScript{} ist die Verwendung eines Titelkopfes im 
\CD mit der typischen Kopfzeile noch nicht möglich. Dieses Problem wird in der 
nächsten Version behoben.
\end{Declaration}
\end{Declaration}

\begin{Declaration}{\Macro{title}\Parameter{Titel}}
\begin{Declaration}{\Macro{subtitle}\Parameter{Untertitel}}
\printdeclarationlist%
\index{Titel!Felder}%
%
Sowohl Titel als auch Untertitel werden normalerweise in Majuskeln und 
\DIN gesetzt. Während sowohl Größe als auch Schriftart des Titels durch das \CD 
vorgegeben sind und denen von Teile"~ und Kapitelüberschriften entsprechen, 
können diese beim Untertitel angepasst werden.
%
\begin{Example}
In diesem Dokument wurde der Untertitel derart geändert, dass dieser nicht 
standardmäßig in \DIN sondern in \textubn{Univers~65~Bold} ausgegeben wird.
\begin{code}[escapechar=§]
\addtokomafont{subtitle}{\univbn}
\subtitle{§\PName{Untertitel}§}
\end{code}
\end{Example}
\end{Declaration}
\end{Declaration}

\begin{Declaration}{\Macro{author}\Parameter{Autor(en)}}
\begin{Declaration}{\Macro{authormore}\Parameter{Autorenzusatz}}
\begin{Declaration}{\Macro{dateofbirth}\Parameter{Geburtsdatum}}
\begin{Declaration}{\Macro{placeofbirth}\Parameter{Geburtsort}}
\begin{Declaration}{\Macro{matriculationnumber}\Parameter{Matrikelnummer}}
\begin{Declaration}{\Macro{matriculationyear}\Parameter{Immatrikulationsjahr}}
\printdeclarationlist%
\index{Titel!Felder}\index{Autorenangaben|?}%
\index{Datum!Geburtsdatum|?}%
%
Mit dem Befehl \Macro*{author} wird der Autor angegeben. Innerhalb des 
Argumentes können auch mehrere Autoren aufgeführt werden, wobei diese dann 
jeweils mit \Macro{and} zu trennen sind. Zu erwähnen ist, dass alle weiteren 
hier vorgestellten Befehle selbst im Argument von \Macro*{author} stehen 
können. Damit wird es möglich, jedem Autor unterschiedliche Angaben mitzugeben.

Mit \Macro*{authormore} wird unter dem Autor eine Zeile ausgegeben, welche 
durch den Anwender frei belegt werden kann. Sollte das Paket \Package{isodate} 
geladen sein, so wird die damit eingestellte Formatierung des Datums durch 
\Macro*{dateofbirth}~-- wie übrigens bei jedem anderem Datumsfeld der 
\TUDScript-Klassen auch~--  verwendet. Dafür der Befehl \Macro{printdate} aus 
diesem Paket verwendet. Die weiteren Befehle als zusätzliche Angabe erklären 
sich von selbst.
\end{Declaration}
\end{Declaration}
\end{Declaration}
\end{Declaration}
\end{Declaration}
\end{Declaration}

\begin{Declaration}{\Macro{and}}
\printdeclarationlist%
\index{Kollaboratives Schreiben|?}\index{Titel!Kollaboratives Schreiben}%
%
Dieser Befehl wird sowohl bei den \hologo{LaTeX}"=Standardklassen als auch bei 
den \KOMAScript"=Klassen lediglich auf der Titelseite dazu verwendet, mehrere 
Autoren im Argument von \Macro{author} voneinander zu trennen.

Bei den \TUDScript-Klassen hingegen ist dieser Befehl derart in seiner Funktion 
erweitert worden, dass damit die Angabe einer kollaborativen Autorenschaft für 
Abschlussarbeiten innerhalb des Befehls \Macro{author} möglich ist. Außerdem 
kann er noch im Argument von \Macro{supervisor}, \Macro{referee} sowie 
\Macro{advisor} verwendet werden, um mehrere Betreuer beziehungsweise Gutachter 
und Fachreferenten anzugeben. Er ist dabei nicht auf die Verwendung für den 
Titel allein beschränkt. Auch bei den Umgebungen \Environment{task}, 
\Environment{evaluation} und \Environment{notice} kann er eingesetzt werden.
%
\begin{Example}
Angenommen, es soll eine Abschussarbeit von zwei unterschiedlichen Autoren in 
kollaborativer Gemeinschaft erstellt werden, so könnte man die Autorenangaben 
folgendermaßen gestalten:
\begin{code}
\author{%
  Mickey Mouse
  \matriculationnumber{12345678}
  \dateofbirth{2.1.1990}
  \placeofbirth{Dresden}
\and%
  Donald Duck
  \matriculationnumber{87654321}
  \dateofbirth{1.2.1990}
  \placeofbirth{Berlin}
}
\matriculationyear{2010}
\end{code}
Alle zusätzlichen Angaben außerhalb des Argumentes von \Macro{author} werden 
für beide Autoren gleichermaßen übernommen. Die Angaben innerhalb des Argumentes 
von \Macro{author} werden den jeweiligen, mit \Macro{and} getrennten Autoren 
zugeordnet. Siehe dazu auch das Minimalbeispiel in \autoref{sec:mwe:collab}.
\end{Example}
\end{Declaration}

\begin{Declaration}{\Macro{thesis}\Parameter{Typisierung}}
\begin{Declaration}{\Macro{subject}\Parameter{Typisierung}}
\printdeclarationlist%
\index{Titel!Felder}%
\index{Abschlussarbeit|!}\index{Typisierung}%
%
Mit diesen beiden Befehlen kann der Typ der Dokumentes beziehungsweise der 
Abschlussarbeit angegeben werden. Während der Befehl \Macro*{thesis} den Inhalt 
des Feldes unter dem Titel vertikal zentriert und in \DIN auf der Titelseite 
ausgibt, erscheint der Inhalt des Befehls \Macro*{subject} in \Univers oberhalb 
des Titels. Es können auch beide Befehle parallel mit unterschiedlichen 
Inhalten verwendet werden. Der Befehl \Macro*{thesis} dient den 
\TUDScript"=Dokumentklassen außerdem zur Erkennung von Abschlussarbeiten 
gedacht, da für diese spezielle Felder bereitgehalten werden und auch die 
Titelseite leicht geändert gesetzt wird.

Des Weiteren ist es bei beiden Befehlen möglich, spezielle Werte als Argument 
zur Typisierung des Dokumentes zu verwenden. Diese werden entsprechend der 
gewählten Dokumentensprache~-- entweder Deutsch oder Englisch~-- entschlüsselt 
und gesetzt. Die möglichen Werte sind \autoref{tab:thesis} zu entnehmen. Dabei 
ist zu beachten, dass das Setzen eines speziellen Wertes für \emph{entweder} 
\Macro*{thesis} \emph{oder} \Macro*{subject} möglich ist. Die Verwendung eines 
der genannten Werte führt immer dazu, dass das Dokument als Abschlussarbeiten 
erkannt und die erweiterte Titelseite aktiviert wird. Gleichzeitig wird damit 
die Option \Option{subjectthesis} beeinflusst. Sollte vom Anwender kein 
explizites Verhalten für \Option{subjectthesis} definiert sein, so führt die 
Verwendung von \Macro*{thesis}\Parameter{Wert} zu \Option{subjectthesis}[false] 
und \Macro*{subject}\Parameter{Wert} zu \Option{subjectthesis}[true].
%
\begin{table}
\index{Bezeichner}\index{Bezeichner!Typisierung}%
\caption{%
  Spezielle Werte zur Typisierung des Dokumentes für
  \Macro*{thesis} und \Macro*{subject}%
}
\label{tab:thesis}%
\centering%
\makeatletter%
\def\@tempa#1{%
  \Term{#1} & \@nameuse{#1} & \selectlanguage{english}\@nameuse{#1}%
  \tabularnewline%
}%
\begin{tabular}{llll}
  \toprule
  \textbf{Wert} & \textbf{Bezeichner}
    & \textbf{Deutsch} & \textbf{Englisch} \tabularnewline
  \midrule
  diss & \@tempa{dissertationname}
  doctoral & \@tempa{dissertationname}
  phd & \@tempa{dissertationname}
  diploma & \@tempa{diplomathesisname}
  master & \@tempa{masterthesisname}
  bachelor & \@tempa{bachelorthesisname}
  student & \@tempa{studentresearchname}
  project & \@tempa{projectpapername}
  seminar & \@tempa{seminarpapername}
  research & \@tempa{researchname}
  log & \@tempa{logname}
  report & \@tempa{reportname}
  internship & \@tempa{internshipname}
  \bottomrule
\end{tabular}
\makeatother%
\end{table}
\end{Declaration}
\end{Declaration}

\begin{Declaration}{\Option{subjectthesis}[\PBoolean]}%
  [false][\Macro{subject}\Parameter{\autoref{tab:thesis}}:true]
\printdeclarationlist%
%
Der Befehl \Macro{thesis} dient den \TUDScript"=Hauptklassen zur Unterscheidung 
zweier unterschiedlichen Ausprägungen der Titelseite und ist im speziellen für 
Abschlussarbeiten gedacht. Außerdem kann bei der Verwendung spezieller Werte 
aus \autoref{tab:thesis} innerhalb des Argumentes von \Macro{subject} ebenfalls 
das Verhalten für Abschlussarbeiten aktiviert werden, wobei hierdurch die 
Einstellung \Option*{subjectthesis}[true] automatisch vorgenommen wird.

Für den Standardfall~-- also \Option*{subjectthesis}[false]~-- wird der durch 
\Macro{thesis} gegebene Typ der Abschlussarbeit sowie der gegebenenfalls durch 
\Macro{degree} gesetzte angestrebte Abschluss in großen Lettern und sehr 
zentral auf der Titelseite gesetzt. Die Verwendung von \Macro{subject} ist 
hierbei weiterhin möglich.
%
Wird die Option mit \Option*{subjectthesis}[true] aktiviert, so wird die mit 
\Macro{thesis} gesetzte Bezeichnung nicht unterhalb sondern oberhalb des Titels 
an der Stelle von \Macro{subject} ausgegeben. Der mit \Macro{degree} angegebene 
Abschluss wird weiterhin unter dem Titel, allerdings in schlankerer Schrift 
gesetzt. Eine etwaige Verwendung des Befehls \Macro{subject} wird in diesem Fall 
ignoriert.
%
\begin{values}
\itemfalse
  Die Ausgabe von Abschlussarbeitstyp (\Macro{thesis}) und angestrebtem 
  Abschluss (\Macro{degree}) erfolgt in großen Lettern in \DIN zentral auf der 
  Titelseite.
\itemtrue*
  Der Typ der Abschlussarbeit (\Macro{thesis}) wird oberhalb des Titels in der 
  Betreffzeile gesetzt. Der angestrebte Abschluss (\Macro{degree}) wird zentral 
  in der schlankeren \Univers ausgegeben.
\end{values}
\end{Declaration}

\begin{Declaration}{\Macro{degree}\OParameter{Kurzform}\Parameter{Grad}}
\printdeclarationlist%
\index{Titel!Felder}%
%
Mit diesem Befehl wird der angestrebte akademische Grad auf der Titelseite 
ausgegeben. Da dies nur mit einer Abschlussarbeit erreicht werden kann erfolgt 
die Ausgabe nur, wenn entweder \Macro{thesis} oder \Macro{subject} verwendet 
wurde, wobei bei letzterem Befehl im Argument zwingend ein Wert aus 
\autoref{tab:thesis} verwendet werden muss.

Die Option \Option{subjectthesis} hat Einfluss auf die Ausgabe auf der 
Titelseite. Für die Einstellung \Option{subjectthesis}[false] wird der 
Abschuss~-- ähnlich wie 
der Typ der Abschlussarbeit~-- zentral und in realtiv großen Lettern gesetzt. 
Für \Option{subjectthesis}[true] erfolgt die Ausgabe kleiner und in weniger 
starken Buchstaben.
\end{Declaration}

\begin{Declaration}{\Macro{supervisor}\Parameter{Name(n)}}
\begin{Declaration}{\Macro{referee}\Parameter{Name(n)}}
\begin{Declaration}{\Macro{advisor}\Parameter{Name(n)}}
\begin{Declaration}{\Macro{professor}\Parameter{Name}}
\printdeclarationlist%
\index{Titel!Felder}%
\index{Betreuer|?}\index{Gutachter|?}\index{Referent|?}%
%
Mit \Macro*{supervisor}, \Macro*{referee} und \Macro*{advisor} werden die 
Betreuer einer Abschlussarbeit beziehungsweise die Gutachter und Fachreferenten 
einer Dissertation angegeben. Die Angabe mehrerer Person erfolgt wie beim Befehl 
\Macro{author} durch die Trennung mittels \Macro{and}.

Neben der Angabe eines oder mehrerer Betreuer kann mit \Macro*{professor}
der betreuende Hochschullehrer für studentische Arbeiten angegeben werden.
\end{Declaration}
\end{Declaration}
\end{Declaration}
\end{Declaration}

\begin{Declaration}{\Macro{date}\OParameter{Ergänzung}\Parameter{Datum}}
\begin{Declaration}{\Macro{defensedate}\Parameter{Verteidigungsdatum}}
\printdeclarationlist%
\index{Titel!Felder}
\index{Datum|?}\index{Datum!Verteidigungsdatum|?}%
%
Mit \Macro*{date} kann das Datum angegeben werden. Das optionale Argument 
erlaubt eine zusätzliche Anmerkung, welche nach dem Datum ausgegeben wird. Das 
Datum wird bei normalen Dokumenten direkt nach dem Autor beziehungsweise den 
Autoren ausgegeben. Bei Abschlussarbeiten~-- aktiviert durch die Verwendung von 
\Macro{thesis}~-- erscheint es am Ende der Titelseite als Abgabedatum. Außerdem 
kann für in diesem Fall mit \Macro*{defensedate} das Datum der Verteidigung 
angegeben werden, wie es beispielsweise bei dem Druck von Dissertationen üblich 
ist.

Sollte das Paket \Package{isodate} geladen sein, so wird die damit eingestellte 
Formatierung des Datums durch den Befehl \Macro{printdate} aus diesem Paket für 
\Macro*{date} und \Macro*{defensedate} verwendet.
\end{Declaration}
\end{Declaration}

\begin{Declaration}{\Option{ddc}[\PSet]}[false]
\begin{Declaration}{\Option{ddcfoot}[\PSet]}[false]
\printdeclarationlist%
\index{Zweitlogo}\index{Layout!Zweitlogo}%
%
Diese Optionen fügen das Logo von \DDC entweder im Kopf oder im Fuß der 
Titelseite ein. Dies geschieht passend zur gewählten farblichen Ausprägung, 
welche über die Option \Option{cdtitle} gewählt wird.
%
\begin{values}
\itemfalse
  Es erscheint kein Logo von \DDC auf der Titelseite.
\itemtrue*
  Das Logo von \DDC wird auf dem Titel im Kopf (\Option*{ddc}[true]) 
  beziehungsweise im Fuß (\Option*{ddcfoot}[true]) verwendet.
\item[color/colour]
  Im Kopf (\Option*{ddc}[color]) respektive im Fuß (\Option*{ddcfoot}[color]) 
  auf der Titelseite wird, unabhängig von den gewählten Farbeinstellung, das 
  farbige Logo von \DDC genutzt.
\end{values}
%
Der Befehl \Macro{headlogo}\Parameter{Dateiname} überschreibt die Einstellungen 
für \Option*{ddc}.
\end{Declaration}
\end{Declaration}

\begin{Declaration}{\Macro{titledelimiter}\Parameter{Trennzeichen}}
\printdeclarationlist%
\index{Titel!Felder}\index{Titel!Trennzeichen}%%
%
Mit diesem Befehl lässt sich das Trennzeichen, welches auf der Titelseite 
jeweils zwischen der Beschreibung eines Feldes und dem Feld selber steht, 
beliebig anpassen. Voreingestellt ist ein Doppelpunkt gefolgt von einem 
Leerzeichen (:\Macro*{nobreakspace}).
\end{Declaration}

\begin{Declaration}{\Macro{extratitle}\Parameter{Schmutztitel}}
\begin{Declaration}{\Macro{titlehead}\Parameter{Kopf}}
\begin{Declaration}{\Macro{publishers}\Parameter{Verlag}}
\begin{Declaration}{\Macro{thanks}\Parameter{Fußnote}}
\begin{Declaration}{\Macro{uppertitleback}\Parameter{Titelrückseitenkopf}}
\begin{Declaration}{\Macro{lowertitleback}\Parameter{Titelrückseitenfuß}}
\begin{Declaration}{\Macro{dedication}\Parameter{Widmung}}
\printdeclarationlist%
%
Diese Befehle entsprechen den in ihrem Verhalten den originalen Pendants der 
\KOMAScript"=Klassen{} und sollen hier der Vollständigkeit halber erwähnt 
werden.

Die Ausgabe des mit \Macro*{extratitle} definierten Schmutztitels~-- welcher 
beliebig gestaltet und formatiert werden kann~-- erfolgt als Bestandteil der 
Titelei mit \Macro{maketitle} vor der eigentlichen Titelseite.

Mit dem Befehl \Macro*{titlehead} kann ein zusätzlicher, beliebig 
formatierbarer Text oberhalb der Typisierung und des Titels ausgegeben werden. 
Da die Position des Dokumententitels allerdings durch das \CD fest 
vorgegeben ist, kann es~-- im Gegensatz zu den \KOMAScript"=Klassen~-- 
passieren, dass der Kopf des Haupttitels selbst in die Kopfzeile ragt. Dies wird 
durch die \TUDScript-Klassen nicht geprüft und muss gegebenenfalls vom Anwender 
kontrolliert werden.

Der mit dem Befehl \Macro*{publishers} definierte Inhalt muss nicht zwingende 
einen Verlag bezeichnen sondern kann auch andere Informationen beinhalten, 
welche am Ende der Titelseite ausgegeben werden sollen.

Fußnoten werden auf dem Titel nicht mit \Macro*{footnote}, sondern mit der 
Anweisung \Macro*{thanks} erzeugt. Sie dienen in der Regel für Anmerkungen bei 
Titel oder den Autoren. Als Fußnotenzeichen werden dabei Symbole statt Zahlen 
verwendet. Es ist zu beachten, dass \Macro*{thanks} innerhalb des Arguments 
einer der Anweisungen für die Titelseite wie beispielsweise \Macro{author} oder 
\Macro{title} zu verwenden qist.

\index{Satzspiegel!doppelseitig}%
Im doppelseitigen Druck lässt sich die Rückseite der Haupttitelseite für 
weitere Angaben nutzen. Sowohl den Titelrückseitenkopf als auch den
Titelrückseitenfuß kann der Anwender mit \Macro*{uppertitleback} und 
\Macro*{lowertitleback} frei gestalten.

Mit \Macro*{dedication} kann eine eigene Widmungsseite zentriert und in etwas 
größerer Schrift gesetzt werden. Die Rückseite ist wie die des Schmutztitels 
grundsätzlich leer. Die Widmungsseite wird zusammen mit der restlichen Titelei 
durch \Macro{maketitle} ausgegeben und muss daher vor dieser Anweisung 
definiert sein.
\end{Declaration}
\end{Declaration}
\end{Declaration}
\end{Declaration}
\end{Declaration}
\end{Declaration}
\end{Declaration}
\index{Titel|!)}

\clearpage
\subsection{Die Teileseite}\label{sec:part}
\begin{Declaration}{\Option{parttitle}[\PBoolean]}[false]%
\printdeclarationlist%
\index{Teileseiten|?}\index{Layout!Teileseiten}%
%
Diese Option ermöglicht es, den mit \Macro{title} gegebenen Titel des 
Dokumentes selbst in großer Schrift auf einer Teileseite auszugeben, die 
Bezeichnung des mit \Macro{part}\Parameter{Bezeichnung} erzeugten Teils wird 
in diesem Fall in kleiner Schrift direkt darunter gesetzt. Diese 
Layout"=Variante findet sich im  Handbuch für das \CD der \TnUD. \notudscrartcl
\begin{values}
\itemfalse
  Die Bezeichnung des Teils erscheint in großer Schrift auf der Seite, der 
  Titel des Dokumentes gar nicht.
\itemtrue*
  Der Titel wird in großer Auszeichnung auf der Teileseite gesetzt, die 
  Bezeichnung des Teils selber in kleinerer.
\end{values}
\end{Declaration}

\subsection{Die Kapitelseite}\label{sec:chapter}
\begin{Declaration}{\Option{chapterpage}[\PBoolean]}%
  [false][\Option{cd}[color]:true]%
\printdeclarationlist%
\index{Kapitelseiten|?}\index{Layout!Kapitelseiten|?}%
\index{Satzspiegel!doppelseitig}\index{Leerseiten}%
%
Mit dieser Einstellung kann die Überschrift eines Kapitels separat auf einer 
Seite ausgegeben werden. Wird diese Option aktiviert, so werden Kapitelseiten 
genauso wie Teileseiten behandelt. Der Seitenstil beider wird gleichgesetzt.%
\footnote{%
  \Macro*{renewcommand*}\PParameter{\Macro{chapterpagestyle}}%
  \PParameter{\Macro{partpagestyle}}
}
Der nachfolgende Text wird auf der nächsten beziehungsweise bei doppelseitigem 
Satz und rechts öffnenden Kapiteln%
\footnote{%
  \Option{twoside} und \Option{open}[right], Standard für \Class{tudscrbook}
}
auf der übernächsten Seite ausgegeben. Die in diesem Fall erzeugte Rückseite 
wird in ihrer Ausprägung~-- wie auch Teileseiten~-- durch die Einstellung von 
\Option{cleardoublespecialpage} bestimmt. Beim farbigen Layout ist diese Option 
standardmäßig aktiviert. \notudscrartcl
%
\begin{values}
\itemfalse
  Es gibt keine Sonderstellung von Kapiteln, der nachfolgende Text wird direkt 
  unter der Überschrift auf der gleichen Seite ausgegeben.
\itemtrue*
  Die Kapitelüberschrift wird auf einer separaten Seite gesetzt, der folgende
  Text wird erst auf der nächsten beziehungsweise übernächsten Seite ausgegeben. 
  Siehe dazu auch die Option \Option{cleardoublespecialpage}.
\end{values}
\end{Declaration}

\subsection{Satzspiegel und Kolumnentitel}
\index{Seitenstil}\index{Layout!Seitenstil}%
\begin{Declaration}{\Option{geometry}[\PSet]}[true]%
\printdeclarationlist%
\index{Satzspiegel}\index{Satzspiegel!doppelseitig}\index{Layout!Satzspiegel}%
\index{Layout!Seitenränder}%
%
Diese Option ist für die Aufteilung beziehungsweise die Berechnung des 
Satzspiegels verantwortlich. Das Maß der Seitenränder ist im \CD fest vorgegeben 
und wird standardmäßig von den \TUDScript-Klassen eingehalten. Allerdings lassen 
sich die Seitenränder anpassen, um beispielsweise einen vernünftigen 
doppelseitigen Satz zu ermöglichen.%
\footnote{Hierbei sollte der innere Rand schmaler als der äußere sein}
Des Weiteren besteht die Möglichkeit, auf das Standardverhalten von 
\KOMAScript{} zurückzufallen und die Satzspiegelberechnung durch das Paket
\Package{typearea} vornehmen zu lassen. Hier hat insbesondere die Klassenoption 
\Option{DIV}[\PSet] maßgeblichen Einfluss auf den Satzspiegel. Siehe dazu die 
Dokumentation von \KOMAScript{}.
%
\begin{values}
\itemfalse
  Die Satzspiegelberechnung erfolgt via \Package{typearea}, die Vorgaben des 
  \CDs bezüglich der Seitenränder werden ignoriert.
\itemtrue*[tud/cd/asymmetric]
  Die Seitenränder werden im asymmetrischen Stil des \CDs fest definiert und 
  auch für den doppelseitigen Satz (\Option{twoside}[true]) genutzt.%
  \footnote{links: 30\,mm, rechts: 20\,mm, oben: 25\,mm, unten: 30\,mm}
\item[symmetric/normal/standard/std]
  Der Satzspiegel wird im einseitigen sowie doppelseitigen Satz auf der Seite 
  zentriert.%
  \footnote{links: 25\,mm, rechts: 25\,mm, oben: 25\,mm, unten: 30\,mm}
\item[balanced/twoside]
  Im einseitigen Layout entspricht das Verhalten der Einstellung 
  \Option*{geometry}[symmetric]. Beim doppelseitigen Satz wird der Satzspiegel 
  derart verändert, dass die Ränder der inneren Seiten schmaler sind als die der 
  äußeren.%
  \footnote{innen: 20\,mm, außen: 30\,mm, oben: 25\,mm, unten: 30\,mm}
\end{values}
%
Für die Festlegung der Seitenränder selbst wird das Paket \Package{geometry} 
verwendet. Für den Fall, dass die Option \Option*{geometry}[false] gewählt ist, 
erfolgt die Berechnung des Satzspiegels durch \Package{typearea}. Die damit 
berechneten Werte werden anschließend an \Package{geometry} weitergereicht und 
durch dieses umgesetzt.
\end{Declaration}

\subsubsection{Binderandkorrektur}
\index{Binderandkorrektur|!}\index{Layout!Binderandkorrektur}%
Zu erwähnen im Zusammenhang mit Seitenrändern und Satzspiegel ist die durch 
\Package{typearea} angebotene Option \Option{BCOR}[\PName{Länge}], durch 
die bei der Satzspiegelberechnung ein Heftrand beziehungsweise eine 
Binderandkorrektur berücksichtigt wird. Die \TUDScript-Klassen reichen diesen 
Wert auch an \Package{geometry} weiter, so dass der Benutzer unabhängig von der 
Auswahl zur Satzspiegelgestaltung diese Option nutzen kann. So kann 
beispielsweise eine Binderandkorrektur von 5\,mm  mit der Klassenoption 
\Option{BCOR}[5mm] gesetzt werden.

\subsubsection{Kopf"~ und Fußzeile im Zusammenspiel mit dem Satzspiegel}
\index{Kopfzeile|!}\index{Layout!Kopfzeile}%
\index{Fußzeile|!}\index{Layout!Fußzeile}%
Da im \CD nicht festgelegt ist, wie die Gestaltung der Kopf"~ und Fußzeilen in 
einer wissenschaftlichen Arbeit auszuführen ist, bleibt dem Nutzer dafür eine 
gewisse Freiheit. Dafür können beispielsweise die Pakete \Package{scrpage2} oder 
ab der \KOMAScript"=Version~v3.12 besser noch \Package{scrlayer-scrpage} genutzt 
werden. 

In der Dokumentation zu \Package{typearea} wird auch darauf eingegangen, wann 
Kopf"~ und Fußzeile bei der Satzspiegelkonstruktion dem Rand oder aber dem 
Textkörper zugeschlagen werden sollten. Dies sollte bei der Erstellung eigener 
Kopf"~ und Fußzeilen beachtet werden. Die Einstellung dafür erfolgt mit den 
beiden \KOMAScript"=Optionen \Option{headinclude}[\PBoolean] sowie 
\Option{footinclude}[\PBoolean]. Diese können~-- unabhängig von der gewählten 
Einstellung zur Satzspiegelgestaltung über \Option{geometry}~-- verwendet werden.

\begin{Declaration}{\Option{cdfoot}[\PBoolean]}[false]%
\printdeclarationlist%
\index{Kolumnentitel}\index{Layout!Kolumnentitel}
\index{Satzspiegel!doppelseitig}%

Eine Möglichkeit zur Gestaltung der Kolumnentitel zeigt das Handbuch für das \CD 
der \TnUD. Dieses wird ohne Kopf"~ und mit einer einfachen Fußzeile gesetzt. 
Diese enthält dabei den aktuellen Kolumnentitel sowie die Paginierung. Eine 
derartige Ausprägung ist nicht explizit durch das \CD vorgegeben, wurde jedoch 
innerhalb der alten \Class{tudbook}"=Klasse exakt so umgesetzt.

Die neuen \TUDScript-Klassen sind~-- insbesondere aufgrund der Möglichkeit zur 
Verwendung der Pakete \Package{scrlayer-scrpage} oder \Package{scrpage2}, wobei 
das letztere seit der \KOMAScript"=Version~v3.12 veraltet ist~-- bei den 
Gestaltungsmöglichkeiten der Kopf"~ und Fußzeilen flexibler. Dennoch kann mit 
dieser Option das beschriebene Verhalten aktiviert werden. Hierbei wird beim 
doppelseitigen Satz (\Option{twoside}[true]) die Seitenzahl außen gesetzt.
%
\begin{values}
\itemfalse
  Die Kopf"~ und Fußzeilen zeigen Standardverhalten, zur Änderung sollte das
  \KOMAScript"=Paket \Package{scrlayer-scrpage} beziehungsweise das ab 
  \KOMAScript"=Version~v3.12 veraltete \Package{scrpage2} verwendet werden.
\itemtrue*
  Die Kopf"~ und Fußzeilen des Dokumentes werden wie im Handbuch des \CDs der 
  \TnUD beziehungsweise der \Class{tudbook}"=Klasse gesetzt.
\end{values}
\end{Declaration}

\subsection{Seiten im Stil des \CDs}
\begin{Declaration}{\Environment{tudpage}[\OLParameter{Sprache}]}
\begin{Declaration}{\Key{\Environment{tudpage}}{language}[\PName{Sprache}]}
\begin{Declaration}{\Key{\Environment{tudpage}}{cdfont}[\PSet]}
\begin{Declaration}{\Key{\Environment{tudpage}}{color}[\PName{Farbe}]}
\begin{Declaration}{\Key{\Environment{tudpage}}{columns}[\PName{Anzahl}]}
\begin{Declaration}{\Key{\Environment{tudpage}}{widehead}[\PBoolean]}
\begin{Declaration}{\Key{\Environment{tudpage}}{head}[\PSet]}
\begin{Declaration}{\Key{\Environment{tudpage}}{foot}[\PSet]}
\begin{Declaration}{\Key{\Environment{tudpage}}{logo}[\PName{Dateiname}]}
\printdeclarationlist%
\index{Layout}\index{Layout!Seitenstil}%
\index{Kopfzeile}\index{Layout!Kopfzeile}%
\index{Fußzeile}\index{Layout!Fußzeile}%
\index{Schrift}\index{Kopfzeile!Schrift}
%
Ein zentrales Element des \CDs der \TnUD ist der eingeführte prägnante 
Seitenkopf mit der Angabe von Fakultät (\Macro{faculty}), Einrichtung 
(\Macro{department}), Institut (\Macro{institute}) und Lehrstuhl 
(\Macro{chair}). Die beschriebene Umgebung erlaubt es, diesen Kopf für eine oder 
mehrere beliebige Seiten innerhalb des Dokumentes zu verwenden. Beispielsweise 
könnte eine Kurzfassung einer Abschlussarbeit oder eines Papers nicht in der 
\Environment{abstract}"=Umgebung sondern mit dem Kopf des \CDs gesetzt werden.

Zusätzlich können weitere Parameter als optionales Argument angegeben werden. 
Wird das Paket \Package{babel} verwendet, kann mit dem Parameter 
\Key*{\Environment{tudpage}}{language}[\PName{Sprache}] die Sprache innerhalb 
der \Environment*{tudpage}"=Umgebung geändert werden. Dafür muss die gewünschte 
Sprache bereits mit dem Laden von \Package{babel} entweder als Paketoption oder 
besser noch als Klassenoption angegeben worden sein. Dadurch werden lokal 
innerhalb der Umgebung alle Bezeichner und die Trennungsmuster sprachspezifisch 
angepasst. Die gewünschte Sprache kann auch direkt und ohne den Schlüssel 
\Key*{\Environment{tudpage}}{language} als optionales Argument übergeben werden.

Mit dem Parameter \Key*{\Environment{tudpage}}{color} kann die Farbe des Kopfes 
auf jede beliebige, bereits definierte geändert werden. Diese ist für den Fall 
eines farbigen Layouts (\Option{cd}[pale|color]) auf die primäre Hausfarbe 
\Color{HKS41} gesetzt, sonst ist der Kopf schwarz. Des Weiteren wird das Paket 
\Package{multicol} unterstützt. Wird dieses geladen, kann mit dem Paramet 
\Key*{\Environment{tudpage}}{columns}[\PName{Anzahl}] der Inhalt der Umgebung 
mehrspaltig gesetzt werden. Soll für die \Environment*{tudpage}"=Umgebung lokal 
ein anderes Zweitlogo als das mit \Macro{headlogo} gegebene erscheinen, so 
kann man \Key*{\Environment{tudpage}}{logo}[\PName{Dateiname}] verwenden.

Der Parameter \Key*{\Environment{tudpage}}{head} dient zur Steuerung des 
Erscheinungsbildes des Kopfes. In erster Linie kann man damit einstellen, ob und 
welches Sekundärlogo für den Kopf verwendet werden soll. Normalerweise wird dies 
ausgegeben, wenn mit \Macro{headlogo} eines angegeben wurde. Folgende 
Einstellungen für den Parameter \Key*{\Environment{tudpage}}{head} sind möglich:
%
\begin{values}
\itemfalse Es wird lediglich das typische Logo der \TnUD im Kopf auf der linken 
  Seite verwendet. Ein mit \Macro{headlogo} gegebenes Sekundärlogo wird 
  unterdrückt.
\item[\noexpand\emph{logo}]Das durch \Macro{headlogo} gegebene Logo 
  erscheint rechts im Kopf. Dies entspricht der Voreinstellung. Diese kann lokal 
  mit dem Parameter \Key*{\Environment{tudpage}}{logo}[\PName{Dateiname}] 
  überschrieben werden.
\item[ddc] Es wird das einfarbige Logo von \DDC ausgegeben. Im farbigen Layout
  (\Option{cd}[pale|color]) erscheint dies in der primären Hausfarbe 
  \Color{HKS41}, ansonsten in schwarz.
\item[ddcolor] Auf der rechten Kopfseite erscheint das mehrfarbige Logo von \DDC.
\item[cdfont] Wird der Fließtext innerhalb der \Environment*{tudpage}"=Umgebung
  nicht in der Schrift des \CDs gesetzt (Klassenoption \Option{cdfont}[false] 
  bzw. Parameter \Key*{\Environment{tudpage}}{cdfont}[false]), so werden auch 
  die Felder im Querbalken des Kopfes normalerweise in Serifenlosen der 
  verwendeten Schriftfamilie ausgegeben. Mit der Wahl des Parameters 
  \Key*{\Environment{tudpage}}{head}[cdfont] wird hierfür dennoch die Schrift 
  \Univers verwendet.
\end{values}
%
Mit dem Parameter \Key*{\Environment{tudpage}}{foot} lässt sich die Fußzeile 
innerhalb der \Environment*{tudpage}"=Umgebung anpassen, folgende Einstellungen 
sind für diesen möglich:
%
\begin{values}
\item[empty] Die Fußzeile bleibt komplett leer, auch die Seitenzahl entfällt.
\item[plain] Es wird lediglich die Seitenzahl in der Fußzeile ausgegeben.
\item[ddc] Es wird das einfarbige kleine Logo von \DDC im Fuß gesetzt. Im 
  kolorierten Layout (\Option{cd}[pale|color]) in der primären Hausfarbe 
  \Color{HKS41}, ansonsten in schwarz.
\item[ddcolor] Auf der rechten Fußseite erscheint das mehrfarbige Logo von \DDC.
\end{values}
%
Die anderen Parameter entsprechen in ihrem Verhalten prinzipiell den 
gleichnamigen Klassenoptionen, wirken sich jedoch nur lokal innerhalb der 
\Environment*{tudpage}"=Umgebung aus. Namentlich sind dies die Optionen 
\Option'{cdfont} und \Option'{widehead}. Das Verhalten sowie die jeweils 
gültigen Wertzuweisungen können in den entsprechenden Abschnitten der 
Dokumentation nachgelesen werden.
\ToDo[v2.1]{Fehler mit \Package*{scrlayer} beheben und Warnung entfernen}%
\Attention Innerhalb der \Environment{tudpage}"=Umgebung können in der 
aktuellen Version~\vTUDScript{} die Befehle \Macro{part} und \Macro{chapter} 
nicht fehlerfrei verwendet werden.
\end{Declaration}
\end{Declaration}
\end{Declaration}
\end{Declaration}
\end{Declaration}
\end{Declaration}
\end{Declaration}
\end{Declaration}
\end{Declaration}

\section{Die Farben des \CDs}
Zur Verwendung der Farben des \CDs wird das Paket \Package{tudscrcolor} 
genutzt. Falls dieses nicht in der Präambel geladen wird~-- um beispielsweise 
zusätzliche Optionen aufzurufen~-- binden die \TUDScript"=Klassen dieses 
automatisch ein. Weitere Informationen zur Verwendung sind in der Dokumentation 
des Paketes \Package*'{tudscrcolor} zu finden.


\section{Zusätzliche Optionen und Erweiterungen}
Neben den Befehlen für die Anpassung des Layouts an das \CD der \TnUD stellen 
die \TUDScript-Klassen weitere Befehle und Umgebungen zur Verfügung, um die 
Anwendung insbesondere für wissenschaftliche Arbeiten zu erleichtern.

\subsection{Zusammenfassung beziehungsweise Kurzfassung}
\begin{Declaration}{\Option{abstract}[\PSet]}%
\printdeclarationlist%
\index{Zusammenfassung|!(}%
\index{Zweispaltensatz}%

Diese Einstellungsmöglichkeit wird bereits durch \KOMAScript{} für die Klassen
\Class{scrartcl} und \Class{scrreprt} standardmäßig bereitgestellt. Für die
Klasse \Class{scrbook} geschieht dies nicht. Dazu heißt es in der Anleitung zu 
\KOMAScript{}:
%
\begin{quoting}
Bei Büchern wird in der Regel eine andere Art der Zusammenfassung verwendet. Dort
setzt man ein entsprechendes Kapitel an den Anfang oder Ende des Werks. Oft wird 
diese Zusammenfassung entweder mit der Einleitung oder einem weiteren Ausblick 
verknüpft. Daher gibt es bei \Class{scrbook} überhaupt keine 
\Environment{abstract}"=Umgebung. Bei Berichten im weiteren Sinne, etwa einer 
Studien- oder Diplomarbeit, ist ebenfalls eine Zusammenfassung in dieser Form zu 
empfehlen.
\end{quoting}
%
Durch die \TUDScript-Klassen wird diese Option erweitert. Neben der 
standardmäßigen Wahl innerhalb der Klassen \Class{tudscrartcl} und 
\Class{tudscrreprt}, ob keine oder eine kleine und zentrierte Überschrift 
innerhalb der \Environment{abstract}"=Umgebung gesetzt werden soll, kann die 
Überschrift für die Zusammenfassung außerdem in Gestalt eines Abschnitts
oder für die Klassen \Class{tudscrreprt} und \Class{tudscrbook} in der Form 
eines Kapitels ausgegeben werden.

Abhängig von der gewählten Gliederungsebene der Überschrift erfolgt ein 
Voreinstellung für das Setzen eines Eintrages ins Inhaltsverzeichnis. Dies kann 
allerdings jederzeit mit der Option \Option*{abstract}[toc/notoc] durch den 
Anwender explizit eingestellt werden.
%
\begin{values}
\itemfalse[][nur für \Class*{tudscrartcl} und \Class*{tudscrreprt} verfügbar]
  Es wird keine Überschrift für die \Environment{abstract}"=Umgebung ausgegeben.
  Für gewöhnlich erfolgt kein Eintrag ins Inhaltsverzeichnis.
\itemtrue*[][nur für \Class*{tudscrartcl} und \Class*{tudscrreprt} verfügbar]
  Wie bei den \KOMAScript"=Klassen wird eine zentrierte Überschrift mit dem 
  Bezeichner \Term{abstractname} vor der eigentlichen Zusammenfassung gesetzt.
  Normalerweise wird für das Inhaltsverzeichnis kein Eintrag erzeugt.
\item[section] Die Überschrift verwendet den Befehl \Macro{addsec}. Im 
  Standardfall wird die Zusammenfassung im Inhaltsverzeichnis eingetragen.
\item[chapter][(Säumniswert für \Class*{tudscrbook})
  nur für \Class*{tudscrreprt} und \Class*{tudscrbook} verfügbar]
  Es wird der Befehl \Macro{addchap} für das Setzen der Überschrift genutzt. 
  Falls nicht anderweitig angegeben, wird die Zusammenfassung im 
  Inhaltsverzeichnis aufgeführt.
\item[heading] Es wird die jeweils höchstmögliche Gliederungsebene verwendet.
  Für \Class{tudscrartcl} entspricht dies der Einstellung 
  \Option*{abstract}[section], bei \Class{tudscrreprt} und \Class{tudscrbook}   
  der Wahl \Option*{abstract}[chapter].
\item[toc/totoc]
  Unabhängig von der Wahl der Überschrift erhält die Zusammenfassung einen nicht
  nummerierten Eintrag im Inhaltsverzeichnis auf der obersten Gliederungsebene. 
\item[notoc/nottotoc]
  Die Zusammenfassung wird nicht ins Inhaltsverzeichnis eingetragen.
\end{values}
%
Die folgenden Einstellungen haben lediglich Auswirkungen, wenn die Überschrift 
der Zusammenfassung \emph{nicht} mit \Option*{abstract}[chapter] auf das Format 
eines Kapitels gesetzt wurde und die \Option{titlepage}"=Option aktiviert ist.

Häufig wird für Abschlussarbeiten verlangt, neben der deutschsprachigen 
Kurzfassung auch noch einen englischen Abstract zu verfassen. Es kann 
eingestellt werden, beide zusammen auf einer Seite auszugeben~-- sofern genügend 
Platz vorhanden ist.
  
Außerdem kann die standardmäßige vertikale Zentrierung der 
\Environment{abstract}"=Umgebung auf einer Seite unterdrückt werden. Damit kann 
der Anwender gegebenenfalls die Positionierung selbstständig vornehmen. 
%
\begin{values}
\item[one/simple/single]Jede Zusammenfassung wird auf einer eigenen Seite
  beziehungsweise im zweispaltigen Satz in einer neuen Spalte ausgegeben.
\item[two/both/double]
  Sollte ausreichend Platz auf einer Seite vorhanden sein, werden zwei direkt 
  aufeinanderfolgende Zusammenfassungen auf dieser ausgegeben. Ist die Option 
  \Option{twocolumn} aktiviert, erfolgt die Ausgabe der zweiten Zusammenfassung 
  nach der ersten in der gleichen Spalte.
\item[nofil/nofill/novfil/novfill]
  Die Ausgabe erfolgt wie im normalen Textsatz auch.
\item[fil/fill/vfil/vfill]
  Die Zusammenfassung(en) werden auf der Ausgabeseite vertikal zentriert. Für 
  den zweispaltigen Satz steht diese Option nicht zur Verfügung.
\end{values}
\end{Declaration}

\begin{Declaration}{\Environment{abstract}[\OLParameter{Sprache}]}
\begin{Declaration}{\Key{\Environment{abstract}}{language}[\PName{Sprache}]}
\begin{Declaration}{\Key{\Environment{abstract}}{columns}[\PName{Anzahl}]}
\begin{Declaration}{\Key{\Environment{abstract}}{option}[\PSet]}
\printdeclarationlist%
\index{Zweispaltensatz}%

Diese Umgebung dient speziell für die Ausgabe einer Zusammenfassung. Mit der 
Option \Option{abstract} kann eingestellt werden, in welcher Gestalt diese 
ausgegeben werden soll. Wird keine Titelseite sondern ein Titelkopf verwendet 
(\Option{titlepage}[false]), so wird für den Fall, dass die Zusammenfassung 
\emph{nicht} mit einer Überschrift einer Gliederungsebene gesetzt wird, diese 
wie bei den \KOMAScript"=Klassen in einer \Environment{quotation}"=Umgebung 
gesetzt, um die Zusammenfassung abzuheben. Diese hat jedoch den Nachteil, dass 
in dieser die Option \Option{parskip} nicht beachtet wird. Um dieses Problem zu 
beheben, kann das Paket \Package{quoting} geladen werden. Dann wird 
stattdessen 
die \Environment{quoting}"=Umgebung verwendet und kann gegebenenfalls angepasst 
werden.

Zusätzlich können weitere Parameter als optionales Argument angegeben werden. 
Wird das Paket \Package{babel} verwendet, kann mit dem Parameter 
\Key*{\Environment{abstract}}{language}[\PName{Sprache}] die Sprache innerhalb 
der \Environment{abstract}"=Umgebung geändert werden. Dafür muss die gewünschte 
Sprache bereits mit dem Laden von \Package{babel} entweder als Paketoption oder 
besser noch als Klassenoption angegeben worden sein. Dadurch werden lokal 
innerhalb der Umgebung die Bezeichnung \Term{abstractname} und die 
Trennungsmuster sprachspezifisch angepasst. Die gewünschte Sprache kann auch 
direkt und ohne den Schlüssel \Key*{\Environment{abstract}}{language} als 
optionales Argument übergeben werden.

Des Weiteren wird das Paket \Package{multicol} unterstützt. Wird dieses geladen, 
kann mit dem Paramet \Key*{\Environment{abstract}}{columns}[\PName{Anzahl}] die 
Zusammenfassung mehrspaltig gesetzt werden.

Dem Parameter \Key*{\Environment{abstract}}{option} können alle gültigen, 
bereits erläuterten Werte der Option \Option{abstract} übergeben werden. 
Die damit gemachten Einstellungen wirken sich~-- im Gegensatz zur Angabe als 
Klassenoption oder über die Variante der späten Optionenwahl%
\footnote{%
  \Macro{TUDoption}\PParameter{abstract}\Parameter{Einstellung} oder
  \Macro{TUDoptions}\PParameter{abstract=\PName{Einstellung}}
}~-- lediglich lokal 
auf die gerade verwendete \Environment{abstract}"=Umgebung aus.

Sollte die \Environment{abstract}"=Umgebung innerhalb des Argumentes der Befehle 
\Macro{partpreamble} beziehungsweise \Macro{chapterpreamble} verwendet werden, 
wird die Überschrift~-- für den Fall, dass nicht \Option{abstract}[false] 
gewählt ist~-- \emph{immer} in Textgröße und zentriert gesetzt.
\end{Declaration}
\end{Declaration}
\end{Declaration}
\end{Declaration}
\index{Zusammenfassung|!)}%

\subsection{Selbstständigkeitserklärung und Sperrvermerk}
\begin{Declaration}{\Option{declaration}[\PSet]}[true]%
\printdeclarationlist%
\index{Selbstständigkeitserklärung|!}\index{Sperrvermerk|!}%
%
Mit dieser Einstellung kann äquivalent zur Option \Option{abstract} die 
Gestaltung von Selbstständigkeitserklärung und Sperrvermerk angepasst werden.
Zur Ausgabe der Erklärungen werden die Befehle \Macro{confirmation} und 
\Macro{restriction} beziehungsweise \Macro{declaration} bereitgestellt. Ob ein 
Eintrag ins Inhaltsverzeichnis erfoglt, wird abhängig von der gewählten 
Überschrift gegebenenfalls voreingestellt, kann jedoch jederzeit mit der Option 
\Option*{declaration}[toc/notoc] fest vorgegeben werden.
%
\begin{values}
\itemfalse
  Es wird keine Überschrift über den Erklärungen selbst ausgegeben. Für 
  gewöhnlich erfolgt auch kein Eintrag ins Inhaltsverzeichnis.
\itemtrue*
  Es wird eine zentrierte Überschrift mit dem Bezeichner \Term{confirmationname} 
  vor dem Text der Selbstständigkeitserklärung beziehungsweise 
  \Term{restrictionname} vor  der Sperrvermerk gesetzt. Normalerweise
  wird im Inhaltsverzeichnis kein Eintrag erzeugt.
\item[section] Die Überschrift verwendet den Befehl \Macro{addsec}. Im 
  Standardfall wird die Erklärung im Inhaltsverzeichnis eingetragen.
\item[chapter]
  Es wird der Befehl \Macro{addchap} für das Setzen der Überschrift genutzt. 
  Falls nicht anderweitig angegeben, wird die ausgegebene Erklärung im 
  Inhaltsverzeichnis aufgeführt.
\item[heading] Es wird die jeweils höchstmögliche Gliederungsebene verwendet.
  Für \Class{tudscrartcl} entspricht dies der Einstellung 
  \Option*{declaration}[section], bei \Class{tudscrreprt} und \Class{tudscrbook} 
  der Wahl \Option*{declaration}[chapter].
\item[toc/totoc]
  Unabhängig von der Wahl der Überschrift erhält die Erklärung einen nicht
  nummerierten Eintrag im Inhaltsverzeichnis auf der obersten Gliederungsebene. 
\item[notoc/nottotoc]
  Die Erklärung wird nicht ins Inhaltsverzeichnis eingetragen.
\end{values}
%
Die folgenden Einstellungen haben lediglich Auswirkungen, wenn die Überschrift 
der Zusammenfassung \emph{nicht} mit \Option*{declaration}[chapter] auf das 
Format eines Kapitels gesetzt wurde und die \Option{titlepage}"=Option aktiviert 
ist.
%
\begin{values}
\item[one/simple/single]Jede Erklärung wird auf einer separaten Seite
  beziehungsweise im zweispaltigen Satz in einer neuen Spalte ausgegeben.
\item[two/both/double]
  Ist ausreichend Platz auf einer Seite vorhanden, werden durch den Befehl 
  \Macro{declaration} Selbstständigkeitserklärung und Sperrvermerk auf 
  dieser ausgegeben. Mit \Macro{confirmation} und \Macro{restriction} können 
  beispielsweise auch Selbstständigkeitserklärung beziehungsweise Sperrvermerk 
  in verschiedenen Sprachen auf einer Seite gesetzt werden. Ist die Option 
  \Option{twocolumn} aktiviert, erfolgt die Ausgabe aller Erklärungen ohne den 
  Beginn einer neuen Spalte.
\item[nofil/nofill/novfil/novfill]
  Die Ausgabe erfolgt wie im normalen Textsatz auch.
\item[fil/fill/vfil/vfill]
  Die Erklärung(en) werden auf der Ausgabeseite vertikal zentriert. Im 
  zweispaltigen Satz steht diese Option nicht zur Verfügung.
\end{values}
\end{Declaration}

\begin{Declaration}{\Macro{confirmation}\OLParameter{supporter}}
\begin{Declaration}{\Key{\Macro{confirmation}}{supporter}[\PName{Unterstützer}]}
\begin{Declaration}{\Key{\Macro{confirmation}}{place}[\PName{Ort}]}
\begin{Declaration}{\Key{\Macro{confirmation}}{closing}[\PName{Abschluss}]}
\begin{Declaration}{\Key{\Macro{confirmation}}{language}[\PName{Sprache}]}
\begin{Declaration}{\Key{\Macro{confirmation}}{option}[\PSet]}
\printdeclarationlist%
\index{Selbstständigkeitserklärung}\index{Datum}%

Mit diesem Befehl wird ein sprachspezifischer Standardtext für eine 
Selbstständigkeitserklärung ausgegeben, welcher in \Term{confirmationtext} 
gespeichert ist. Innerhalb der Erklärung werden alle Personen aufgeführt, welche 
entweder durch den Schlüssel \Key*{\Macro{confirmation}}{supporter} angegeben 
wurden. Auch hier können mehrere Personen durch \Macro{and} getrennt werden. Für 
den Fall, dass mit \Macro{supervisor} ein oder mehrere Betreuer angegeben 
wurden, werden diese standardmäßig auch als Unterstützer genannt. Diese Angabe 
kann jederzeit mit \Key*{\Macro{confirmation}}{supporter}[\PParameter{}] durch 
den Anwender gelöscht werden. Soll kein weiterer der möglichen Parameter 
geändert werden, kann das Feld der Unterstützer auch mit dem bloßen optionalen 
Argument ohne die Angabe eines Schlüssels angepasst werden. Wie der Standardtext 
geändert werden kann, ist unter \autoref{sec:localization} zu finden. 

Nach dem eigentlichen Text der Selbstständigkeitserklärung wird der mit 
\Key*{\Macro{confirmation}}{place} angegebene Ort sowie das mit \Macro{date} 
eingestellte Datum ausgegeben. Als Voreinstellung ist für den Ort 
\enquote{Dresden} gewählt. Danach folgen~--  mit etwas vertikalem Freiraum für 
die notwendige Unterschrift~-- der Autor oder die Autoren, angegeben durch den 
Befehl \Macro{author}. Soll anstelle dessen etwas anderes nach der eigentlichen 
Selbstständigkeitserklärung gesetzt werden, kann die Ausgabe mit
\Key*{\Macro{confirmation}}{closing}[\PName{Abschluss}] angepasst werden.

Mit dem Parameter \Key*{\Macro{confirmation}}{language} kann die Sprache für den 
ausgegebenen Text der Selbstständigkeitserklärung lokal geändert werden. Für den 
Schlüssel \Key*{\Macro{confirmation}}{option} können alle gültigen Werte der 
Option \Option{declaration} angebeben werden. Diese wirken sich ebenfalls nur 
lokal aus.
\end{Declaration}
\end{Declaration}
\end{Declaration}
\end{Declaration}
\end{Declaration}
\end{Declaration}

\begin{Declaration}{\Macro{restriction}\OLParameter{company}}
\begin{Declaration}{\Key{\Macro{restriction}}{company}[\PName{Firma}]}
\begin{Declaration}{\Key{\Macro{restriction}}{language}[\PName{Sprache}]}
\begin{Declaration}{\Key{\Macro{restriction}}{option}[\PSet]}
\printdeclarationlist%
\index{Sperrvermerk}%

Beim Sperrvermerk verhält es sich äquivalent zur Selbstständigkeitserklärung 
(\Macro{confirmation}). Als optionales Argument kann hier der Name der Firma 
angegeben werden. Der Standardtext lässt sich über \Term{restrictiontext} 
ändern, nachzulesen unter \autoref{sec:localization}. Auch hier kann über den 
Parameter \Key*{\Macro{restriction}}{language} die Sprache für den ausgegebenen 
Text sowie mit dem Parameter \Key*{\Macro{restriction}}{option} die Werte der 
Option \Option{declaration} lokal geändert werden.
\end{Declaration}
\end{Declaration}
\end{Declaration}
\end{Declaration}

\begin{Declaration}{\Macro{declaration}\LParameter}
\begin{Declaration}{\Key{\Macro{declaration}}{supporter}[\PName{Unterstützer}]}
\begin{Declaration}{\Key{\Macro{declaration}}{place}[\PName{Ort}]}
\begin{Declaration}{\Key{\Macro{declaration}}{closing}[\PName{Abschluss}]}
\begin{Declaration}{\Key{\Macro{declaration}}{company}[\PName{Firma}]}
\begin{Declaration}{\Key{\Macro{declaration}}{language}[\PName{Sprache}]}
\begin{Declaration}{\Key{\Macro{declaration}}{option}[\PSet]}
\printdeclarationlist%
\index{Selbstständigkeitserklärung}\index{Sperrvermerk}%

Dieser Befehl gibt Selbstständigkeitserklärung und Sperrvermerk direkt 
nacheinander aus. Quasi, als ob im Dokument die Befehle \Macro{confirmation} und 
\Macro{restriction} direkt nacheinander gesetzt wurden. Dementsprechend 
akzeptiert dieser auch alle dafür beschriebenen Parameter.
\end{Declaration}
\end{Declaration}
\end{Declaration}
\end{Declaration}
\end{Declaration}
\end{Declaration}
\end{Declaration}

\begin{Declaration}{\Macro{supporter}\Parameter{Unterstützer}}
\begin{Declaration}{\Macro{place}\Parameter{Ort}}
\begin{Declaration}{\Macro{confirmationclosing}\Parameter{Abschluss}}
\begin{Declaration}{\Macro{company}\Parameter{Firma}}
\printdeclarationlist%
\index{Selbstständigkeitserklärung}\index{Sperrvermerk}%

Diese Makros ändern~-- im Gegensatz zu den Parametern der bereits vorgestellten 
Befehle \Macro{confirmation} und \Macro{restriction}~-- die entsprechenden 
Feldwerte global für das gesamte Dokument. Genutzt werden kann dies 
beispielsweise wenn ein Erklärungstyp in unterschiedlichen Sprachen ausgegeben 
wird. Dann kann man sich mit diesen Makros die mehrfache Angabe eines Parameters 
sparen.
\end{Declaration}
\end{Declaration}
\end{Declaration}
\end{Declaration}

\subsection{Lesezeichen}
\begin{Declaration}{\Option{tudbookmarks}[\PBoolean]}[true]%
\begin{Declaration}{%
  \Macro{tudbookmark}\OParameter{Ebene}\Parameter{Text}\Parameter{Ankername}%
}%
\printdeclarationlist%
\index{Lesezeichen}%
\index{Umschlagseite}\index{Titel}\index{Inhaltsverzeichnis}%
\index{Aufgabenstellung}\index{Gutachten}\index{Aushang}%
%
Diese Option wird wirksam, wenn \Package{hyperref} geladen wurde. Es werden für 
die Umschlag- und Titelseite, das Inhaltsverzeichnis sowie~-- bei der Verwendung 
des Paketes \Package{tudscrsupervisor}~-- die Aufgabenstellung jeweils 
Lesezeichen oder auch Outline"=Einträge im PDF-Dokument erzeugt.
%
\begin{values}
\itemfalse
  Es erfolgt kein Eintrag von ergänzenden Lesezeichen.
\itemtrue*
  Es werden automatisch zusätzliche Lesezeichen eingetragen.
\end{values}
%
Der Befehl \Macro*{tudbookmark} arbeitet wie \Macro{pdfbookmark} aus 
\Package{hyperref} mit dem Unterschied, dass die Lesezeichen nur generiert 
werden, wenn die Option \Option*{tudbookmarks} aktiviert ist.
\end{Declaration}
\end{Declaration}


\section{Sprachabhängige Bezeichner}
\label{sec:localization}
\index{Bezeichner|!(}%
%
Durch \KOMAScript{} werden Befehle, mit denen sprachabhängige Bezeichner erzeugt 
oder geändert werden können, zur Verfügung gestellt. Diese werden durch das 
\TUDScript-Bundle genutzt, um lokalisierte Begriffe für die Sprachen Englisch 
und Deutsch bereitzustellen. Ein Großteil davon betrifft Bezeichnungen für 
Felder auf der Titelseite (\autoref{sec:title}). Hierfür wird
\Macro{providecaptionname}\Parameter{Sprache}\Parameter{Makro}\Parameter{Inhalt} 
verwendet, wobei \PName{Sprache} dem geladenen Sprachpaket~-- normalerweise das 
Paket \Package{babel}~-- bekannt sein muss.

Sollte der Anwender die im Folgenden erläuterten oder auch andere Bezeichner, 
welche von einem beliebigen (Sprach"~)Paket bereitgestellt werden, ändern 
wollen, ist hierfür der Befehl
\Macro{renewcaptionname}\Parameter{Sprache}\Parameter{Makro}\Parameter{Inhalt} 
zu verwenden. Dabei sollte darauf geachtet werden, dass dies entweder 
\textbf{nach} \Macro*{begin}\PParameter{document} geschieht oder aber in der 
Dokumentpräambel mit \Macro*{AtBeginDocument}\PParameter{\dots} verzögert wird.%
\footnote{%
  Ab der \KOMAScript"=Version~v3.12 passiert dies automatisch, der Anwender muss 
  darauf nicht mehr achten.%
}
Dies stellt sicher, dass die Bezeichner nicht durch ein später geladenes Paket 
abermals geändert werden können. Es sollte natürlich dabei eine \PName{Sprache} 
angegeben werden, welche im Dokument durch \Package{babel} oder ein anderes 
Sprachpaket verwendet wird, beispielsweise \PValue{ngerman} oder 
\PValue{english}. 

Die Makros der Bezeichner und deren Verwendung werden folgend kurz beschrieben 
und tabellarisch aufgeführt. Dabei wurde versucht, alle Bezeichnerbefehle für 
bestimmte Begriffe auf \PValue{\dots{}name} und beschreibende Texte auf 
\PValue{\dots{}text} enden zu lassen.
%
\begin{Declaration}{\Term{dateofbirthtext}}
\begin{Declaration}{\Term{placeofbirthtext}}
\begin{Declaration}{\Term{matriculationnumbername}}
\begin{Declaration}{\Term{matriculationyearname}}
\printdeclarationlist%
\index{Titel}\index{Autorenangaben}\index{Datum!Geburtsdatum}%
%
Werden für den Autor oder die Autoren das Geburtsdatum (\Macro{dateofbirth}), 
der Geburtsort (\Macro{placeofbirth}), die Matrikelnummer 
(\Macro{matriculationnumber}) und/oder das Immatrikulationsjahr 
(\Macro{matriculationyear}) für die Titelseite angegeben, werden bei der Ausgabe 
die dazugehörigen Bezeichner vorangestellt und durch \Macro{titledelimiter} vom 
eigentlichen Feld getrennt.
\TermTable{%
  dateofbirthtext,placeofbirthtext,matriculationnumbername,matriculationyearname%
}
\end{Declaration}
\end{Declaration}
\end{Declaration}
\end{Declaration}

\begin{Declaration}{\Term{degreetext}}
\printdeclarationlist%
\index{Titel}\index{Abschlussarbeit}\index{Typisierung}%
%
Wurde erkannt, dass das Dokument eine Abschlussarbeit ist,%
\footnote{%
  Entweder wurde der Befehl \Macro{thesis} oder \Macro{subject} mit einem 
  speziellen Wert aus \autoref{tab:thesis} beziehungsweise mit der Option 
  \Option{subjectthesis} verwendet.
}
dann kann dr zu erlangende akademische Grad mit dem Befehl \Macro{degree} 
angegeben werden. Bei dessen Ausgabe auf dem Titel wird dabei der entsprechende 
Text dazu angegeben.
\TermTable{degreetext}
\end{Declaration}

\begin{Declaration}{\Term{dissertationname}}
\begin{Declaration}{\Term{diplomathesisname}}
\begin{Declaration}{\Term{masterthesisname}}
\begin{Declaration}{\Term{bachelorthesisname}}
\begin{Declaration}{\Term{studentresearchname}}
\begin{Declaration}{\Term{projectpapername}}
\begin{Declaration}{\Term{seminarpapername}}
\begin{Declaration}{\Term{researchname}}
\begin{Declaration}{\Term{logname}}
\begin{Declaration}{\Term{internshipname}}
\begin{Declaration}{\Term{reportname}}
\printdeclarationlist%
\index{Titel}\index{Abschlussarbeit}\index{Typisierung}%
%
Diese Bezeichner dienen zur Typisierung speziell für eine Abschlussarbeit. Wie 
diese genutzt werden können, ist bei der Erläuterung von \Macro{thesis} und 
\Macro'{subject} beziehungsweise in \autoref{tab:thesis} zu finden.
\end{Declaration}
\end{Declaration}
\end{Declaration}
\end{Declaration}
\end{Declaration}
\end{Declaration}
\end{Declaration}
\end{Declaration}
\end{Declaration}
\end{Declaration}
\end{Declaration}

\begin{Declaration}{\Term{supervisorname}}
\begin{Declaration}{\Term{supervisorothername}}
\begin{Declaration}{\Term{refereename}}
\begin{Declaration}{\Term{refereeothername}}
\begin{Declaration}{\Term{advisorname}}
\begin{Declaration}{\Term{advisorothername}}
\begin{Declaration}{\Term{professorname}}
\printdeclarationlist%
\index{Titel}%
\index{Betreuer}\index{Gutachter}\index{Hochschullehrer}%
\index{Referent}%
%
Diese sprachabhängigen Begriffe sind die Bezeichner für die Titelseitenfelder 
von Betreuer (\Macro{supervisor}), Gutachter (\Macro{referee}), Fachreferent 
(\Macro{advisor}) und den betreuenden Hochschullehrer (\Macro{professor}). Wird 
für eines der ersten drei Felder mehr als eine Person angegeben, so werden 
diese~-- wenn die Einzelpersonen jeweils mit \Macro{and} voneinander getrennt 
wurden~-- durch den Bezeichner \PValue{\bsc\dots{}othername} ergänzt.
\TermTable{%
  supervisorname,supervisorothername,refereename,refereeothername,%
  advisorname,advisorothername,professorname%
}
\end{Declaration}
\end{Declaration}
\end{Declaration}
\end{Declaration}
\end{Declaration}
\end{Declaration}
\end{Declaration}

\begin{Declaration}{\Term{datetext}}
\begin{Declaration}{\Term{defensedatetext}}
\printdeclarationlist%
\index{Titel}\index{Abschlussarbeit}%
\index{Datum}\index{Datum!Verteidigungsdatum}%
%
Wird mit \Macro{date} das Datum und mit \Macro{defensedate} ein Datum der 
Verteidigung für eine Abschlussarbeit angegeben, so werden auch diese Felder 
durch einen einleitenden Text beschrieben.
\TermTable{datetext,defensedatetext}
\end{Declaration}
\end{Declaration}


\begin{Declaration}{\Term{coverpagename}}
\begin{Declaration}{\Term{titlepagename}}
\printdeclarationlist%
\index{Lesezeichen}\index{Umschlagseite}\index{Titel!Umschlagseite}\index{Titel}%
%
Diese beiden Bezeichner werden bei aktivierter \Option{tudbookmarks} für das 
Eintragen von Lesezeichen in ein PDF"=Dokument genutzt.
\TermTable{coverpagename,titlepagename}
\end{Declaration}
\end{Declaration}

\begin{Declaration}{\Term{abstractname}}
\printdeclarationlist%
%
Dieser Bezeichner wird lediglich für \Class{tudscrbook} definiert, da dieser von 
\KOMAScript{} für die Buchklasse nicht vorgesehen wird.
\TermTable{abstractname}
\end{Declaration}

\begin{Declaration}{\Term{listingname}}
\begin{Declaration}{\Term{listlistingname}}
\printdeclarationlist%
%
Sollte ein Paket zur Einbindung von externem Quelltext~-- beispielsweise 
\Package{listings}~-- verwendet werden, so werden diese Bezeichnungen für 
Quelltextausschnitte und das Quelltextverzeichnis verwendet.
\TermTable{listingname,listlistingname}
\end{Declaration}
\end{Declaration}

\begin{Declaration}{\Term{confirmationname}}
\begin{Declaration}{\Term{restrictionname}}
\printdeclarationlist%
\index{Selbstständigkeitserklärung}\index{Sperrvermerk}%
%
Es werden die Bezeichnungen für Selbstständigkeitserklärung und Sperrvermerk für 
die dazugehörigen Überschriften definiert.
\TermTable{confirmationname,restrictionname}
\end{Declaration}
\end{Declaration}

\begin{Declaration}{\Term{confirmationtext}}
\begin{Declaration}{\Term{restrictiontext}}
\printdeclarationlist%
Die Texte der Erklärungen selbst sind derart aufgebaut, dass sie in Abhängigkeit 
von den angegebenen Informationen unterschiedlich ausgeführt werden. Innerhalb 
der Selbstständigkeitserklärung (\Macro{confirmation}) werden gegebenenfalls die 
Felder für den Titel (\Macro{title}) und die Typisierung der Abschlussarbeit%
\footnote{%
  entweder \Macro{thesis} oder \Macro{subject}\Parameter{\autoref{tab:thesis}}
  beziehungsweise Option \Option{subjectthesis}[true]
}
sowie die angegebenen Unterstützer%
\footnote{%
  \Macro{confirmation}\POParameter{\Key{\Macro{confirmation}}{supporter}=\dots}
  oder \Macro{supporter}\PParameter{\dots}%
}
beachtet. Für den Sperrvermerk (\Macro{restriction}) wird neben dem Titel 
(\Macro{title}) optional außerdem noch das Feld der externen Firma%
\footnote{%
  \Macro{restriction}\POParameter{\Key{\Macro{restriction}}{company}=\dots}
  oder \Macro{company}\PParameter{\dots}%
}
verwendet. Der Vollständigkeit halber werden im Folgenden noch die Texte für die 
Selbstständigkeitserklärung und den Sperrvermerk aufgeführt~-- alerdings 
lediglich die deutschsprachige Version. Dabei werden alle möglichen Felder 
angezeigt.

\begingroup
  \makeatletter
  \def\@@title{\PName{Titel}}
  \def\@@thesis{\PName{Abschlussarbeit}}
  \def\@supporter{\PName{Vorname Nachname} \and \PName{Vorname Nachname}}
  \def\@company{\PName{Firma}}
  \makeatother

  \bigskip\noindent
  \textbf{Bezeichner}\quad\Term*{confirmationtext}\par
  \begin{center}\begin{minipage}{.8\textwidth}
  \confirmationtext
  \end{minipage}\end{center}
  
  \bigskip\noindent
  \textbf{Bezeichner}\quad\Term*{restrictiontext}\par
  \begin{center}\begin{minipage}{.8\textwidth}
  \restrictiontext
  \end{minipage}\end{center}
\endgroup
\end{Declaration}
\end{Declaration}
\index{Bezeichner|!)}


\section{Kompatibilität zur \vTUD}\label{sec:comp}
\index{Version v1.0}\index{Kompatibilität!Version v1.0}%

Bei der kompletten Neuimplementierung der \TUDScript-Klassen wurde sehr viel 
verändert und verbessert. Teilweise wurden einige Befehle und Optionen aus 
Konsistenzgründen lediglich umbenannt, andere wurden vollständig entfernt oder 
über neue Befehle und Optionen in ihrer Funktionalität ersetzt und erweitert. 
Falls es möglich war, wurden alte Einstellungsmöglichkeiten und Kommandos als 
Alias für die neueren bereitgestellt.

Einige dieser waren jedoch bereits in der \vTUD Relikte, um die Kompatibilität 
zur \Class{tudbook}-Klasse zu gewährleisten. Diese sind mittlerweile komplett 
entfernt worden, ihre Funktionalität wird allerdings weiterhin teilweise 
bereitgestellt. Sollten diese dennoch benötigt werden, kann das Paket 
\Package{tudscrcomp} genutzt werden.

\begin{Declaration}{\Option{cd}[alternative]}{entfällt}
\begin{Declaration}{\Option{cdtitle}[alternative]}{entfällt}
\begin{Declaration}{\Length{titlecolwidth}}{entfällt}
\begin{Declaration}{\Term{authortext}}{entfällt}
\printdeclarationlist*%
\index{Titel!alternativer}%
%
Die alternative Titelseite ist komplett aus dem \TUDScript-Bundle entfernt 
worden. Dementsprechend entfallen auch die dazugehörigen Optionen sowie Länge 
und Bezeichner.
\end{Declaration}
\end{Declaration}
\end{Declaration}
\end{Declaration}

\begin{Declaration}{\Length{headingsvskip}}{%
  Alias für \Length*{chapterheadingvskip}%
}
\begin{Declaration}{\Length{signatureheight}}{entfällt}
\printdeclarationlist*%
%
Die Länge \Length*{headingsvskip} wurde zum besseren Verständnis in 
\Length*{chapterheadingvskip} umbenannt. Sie hat jetzt~-- im Gegensatz zur 
\vTUD~-- nur noch Einfluss auf die vertikale Position der Kapitelüberschriften, 
wenn keine separaten Kapitelseiten verwendet werden. Die Platzierung des Titels 
auf der Titelseite sowie der Überschriften von Teilen~-- und Kapiteln bei 
aktivierter \Option*{chapterpage}"=Option~-- ist fixiert und nicht veränderbar.
\end{Declaration}
\end{Declaration}

\begin{Declaration}{\Option{tudfonts}[\PBoolean]}{%
  Alias für \Option*{cdfont}[\PSet]%
}
\printdeclarationlist*%
%
Die Option zur Schrifteinstellung ist wesentlich erweitert worden. Aus Gründen 
der Konsistenz wurde diese umbenannt.
\end{Declaration}

\begin{Declaration}{\Option{tudfoot}[\PBoolean]}{%
  Alias für \Option*{cdfoot}[\PBoolean]%
}
\printdeclarationlist*%
%
Ebenso wurde diese Option umbenannt, um dem Namensschema der restlichen Optionen 
zu entsprechen.
\end{Declaration}

\begin{Declaration}{\Option{headfoot}[\PSet]}{entfällt}
\printdeclarationlist*%
%
Diese Option war in der \vTUD notwendig, um die parallele Verwendung von 
\Package*{typearea} und \Package*{geometry} zu ermöglichen. Dies wurde komplett 
überarbeitet, an das Paket \Package*{geometry} werden die Einstellungen für die 
\KOMAScript"=Optionen \Option*{headinclude} und \Option*{footinclude} jetzt 
direkt weitergereicht. Damit wird die Option \Option*{headfoot} obsolet.
\end{Declaration}

\begin{Declaration}{\Option{partclear}[\PBoolean]}{%
  entfällt, siehe \Option*{cleardoublespecialpage}%
}
\begin{Declaration}{\Option{chapterclear}[\PBoolean]}{%
  entfällt, siehe \Option*{cleardoublespecialpage}%
}
\printdeclarationlist*%
%
Beide Optionen sind in der neuen Option \Option*{cleardoublespecialpage} 
aufgegangen, womit ein konsistentes Layout erreicht wird. Die ursprünglichen 
Optionen entfallen. 
\end{Declaration}
\end{Declaration}

\begin{Declaration}{\Option{abstracttotoc}[\PBoolean]}{%
  entfällt, siehe \Option*{abstract}%
}
\begin{Declaration}{\Option{abstractdouble}[\PBoolean]}{%
  entfällt, siehe \Option*{abstract}%
}
\printdeclarationlist*%
%
Beide Optionen wurden in die Option \Option*{abstract} integriert und sind 
deshalb überflüssig.
\end{Declaration}
\end{Declaration}

\begin{Declaration}{\Macro{logofile}\Parameter{Dateiname}}%
  {Alias für \Macro*{headlogo}%
}
\printdeclarationlist*%
%
Der Befehl \Macro*{logofile} wurde in \Macro*{headlogo} umbenannt.
\end{Declaration}

\begin{Declaration}{\Macro{confirmationandrestriction}}{%
  entfällt, siehe \Macro*{declaration}%
}
\begin{Declaration}{\Macro{restrictionandconfirmation}}{%
  entfällt, siehe \Macro*{declaration}%
}
\begin{Declaration}{\Macro{location}\Parameter{Ort}}{%
  Alias für \Macro*{place}, siehe auch Parameter \hyperref[idx:keys]{place}%
}
\printdeclarationlist*%
%
Die ersten beiden Befehle entfallen, \Macro*{declaration} kann alternativ dazu 
verwendet werden. In Anlehnung an andere \hologo{LaTeX}-Pakete und "~Klassen 
wurde \Macro*{location} in \Macro*{place} umbenannt.
\end{Declaration}
\end{Declaration}
\end{Declaration}

\begin{Declaration}{\Term{titlecoldelim}}{%
  entfällt, siehe \Macro*{titledelimiter}%
}
\printdeclarationlist*%
%
Das Trennzeichen für Bezeichnungen beziehungsweise beschreibende Texte und dem 
eigentlichen Feld auf der Titelseite ist nicht mehr sprachabhängig und wurde 
umbenannt.
\end{Declaration}

\begin{Declaration}{\Macro{submissiondate}\Parameter{Datum}}{%
  Alias für \Macro*{date}%
}
\begin{Declaration}{\Macro{birthday}\Parameter{Geburtsdatum}}{%
  Alias für \Macro*{dateofbirth}%
}
\begin{Declaration}{\Macro{birthplace}\Parameter{Geburtsort}}{%
  Alias für \Macro*{placeofbirth}
}
\begin{Declaration}{\Macro{studentid}\Parameter{Matrikelnummer}}{%
  Alias für \Macro*{matriculationnumber}
}
\begin{Declaration}{\Macro{enrolmentyear}\Parameter{Immatrikulationsjahr}}{%
  Alias für \Macro*{matriculationyear}%
}
\printdeclarationlist*%
%
Alle Befehle wurden umbenannt und sind jetzt für die Titelseite im \CD nutzbar.
\end{Declaration}
\end{Declaration}
\end{Declaration}
\end{Declaration}
\end{Declaration}

\begin{Declaration}{\Macro{moreauthor}\Parameter{Autorenzusatz}}{%
  umbenannt, siehe \Macro*{authormore}%
}
\begin{Declaration}{\Macro{dissertation}}{%
  entfällt, siehe \Macro*{thesis} und \Macro*{subject}%
}
\begin{Declaration}{\Macro{supervisorII}\Parameter{Name}}{%
  entfällt, siehe \Macro*{supervisor} und \Macro*{and}%
}
\printdeclarationlist*%
%
Die Funktionalität aller Befehle wird durch andere ersetzt. Sollten diese 
dennoch benötigt werden, kann das Paket \Package*{tudscrcomp} geladen werden.
\end{Declaration}
\end{Declaration}
\end{Declaration}

\begin{Declaration}{\Term{submissiontext}}{umbenannt, siehe \Term*{datetext}}
\begin{Declaration}{\Term{birthdaytext}}{%
  umbenannt, siehe \Term*{dateofbirthtext}%
}
\begin{Declaration}{\Term{birthplacetext}}{%
  umbenannt, siehe \Term*{placeofbirthtext}%
}
\begin{Declaration}{\Term{studentidname}}{%
  umbenannt, siehe \Term*{matriculationnumbername}%
}
\begin{Declaration}{\Term{enrolmentname}}{%
  umbenannt, siehe \Term*{matriculationyearname}%
}
\begin{Declaration}{\Term{supervisorIIname}}{%
  umbenannt, siehe \Term*{supervisorothername}%
}
\begin{Declaration}{\Term{defensetext}}{umbenannt, siehe \Term*{defensedatetext}}
\printdeclarationlist*%
%
Die Bezeichner wurden in Anlehnung an die dazugehörigen Befehlsnamen umbenannt.
\end{Declaration}
\end{Declaration}
\end{Declaration}
\end{Declaration}
\end{Declaration}
\end{Declaration}
\end{Declaration}

\subsubsection{Aufgabenstellung}
Die Umgebung für die Erstellung einer Aufgabenstellung für eine 
wissenschaftliche Arbeit wurde in das Paket \Package*{tudscrsupervisor} 
ausgelagert. Dieses muss für die Verwendung der Umgebung \Environment*{task} und 
der daraus abgeleiteten standardisierten Form zwingend geladen werden.

\begin{Declaration}{\Option{cdtask}[\PSet]}{entfällt, siehe \Environment*{task}}
\begin{Declaration}{\Option{taskcompact}[\PBoolean]}{entfällt}
\begin{Declaration}{\Macro{tasks}\Parameter{Ziele}\Parameter{Schwerpunkte}}{%
  umbenannt, siehe \Macro*{taskform}%
}
\begin{Declaration}{\Length{taskcolwidth}}{entfällt}
\printdeclarationlist*%
%
Die Klassenoption \Option*{cdtask} ist komplett entfernt worden, alle 
Einstellungen, welche \Environment*{task} betreffen erfolgen direkt über das 
optionale Argument der Umgebung. Die Variante eines kompakten Kopfes mit der 
Option \Option*{taskcompact} wird nicht mehr bereitgestellt. Der Befehl 
\Macro*{tasks} wurde in \Macro*{taskform} umbenannt und in der Funktionalität 
erweitert. Die manuelle Einstellung der Spaltenbreite für den Kopf der 
Aufgabenstellung mit \Length*{taskcolwidth} wurde aufgrund der verbesserten 
automatischen Berechnung entfernt.
\end{Declaration}
\end{Declaration}
\end{Declaration}
\end{Declaration}

\begin{Declaration}{\Macro{startdate}\Parameter{Ausgabedatum}}{%
  Alias für \Macro*{issuedate}%
}
\begin{Declaration}{\Macro{enddate}\Parameter{Abgabetermin}}{%
  Alias für \Macro*{duedate}%
}
\printdeclarationlist*%
%
Die Befehle wurden lediglich umbenannt.
\end{Declaration}
\end{Declaration}

\begin{Declaration}{\Term{starttext}}{umbenannt, siehe \Term*{issuedatetext}}
\begin{Declaration}{\Term{duetext}}{umbenannt, siehe \Term*{duedatetext}}
\begin{Declaration}{\Term{focustext}}{umbenannt, siehe \Term*{focusname}}
\begin{Declaration}{\Term{objectivestext}}{%
  umbenannt, siehe \Term*{objectivesname}%
}
\printdeclarationlist*%
%
Alle genannten Bezeichner wurden umbenannt.
\end{Declaration}
\end{Declaration}
\end{Declaration}
\end{Declaration}



\newcommand*\cdcolorcalc{}
\newcommand*\cdcolorname{}
\newcommand*\cdcolorvalue{}
\newcommand*\cdcolortext{}
\newcommand*\cdcolor[2][0]{%
  \noindent%
  \begin{tikzpicture}[every node/.style={%
    rectangle,minimum height=.1\linewidth,minimum width=25mm%
  }]%
  \def\cdcolorcalc##1##2{%
    \pgfmathparse{100-##1*10}%
    \xdef\cdcolorname{HKS##2!\pgfmathresult}%
    \xdef\cdcolorvalue{\pgfmathresult}%
    \pgfmathparse{10+##1*10}%
  }%
  \foreach \x in {0,1,...,9}{%
    \cdcolorcalc{\x}{#2}%
    \ifnum\x<#1%
      \def\cdcolortext{white}%
    \else%
      \def\cdcolortext{black}%
    \fi%
    \node [fill=\cdcolorname,rotate=90] at (.\x\linewidth,0)%
      {\textcolor{\cdcolortext}{HKS#2!\pgfmathprintnumber\cdcolorvalue}};%
  };%
  \end{tikzpicture}%\endgraf\noindent%
}
\chapter{Zusätzliche Pakete im \TUDScript-Bundle}
\section{Farben im \CD mit \Package*{tudscrcolor}}
\DeclarePackage{tudscrcolor}
\index{Farben|!}
%
Für das \CD sind mehrere Farben vorgesehen. Die prägnanteste aller Farben ist 
die Hausfarbe \Color*{HKS41}, danach folgen die Auszeichnungsfarben der ersten
(\Color*{HKS44}) und der zweiten Kategorie (\Color*{HKS36}, \Color*{HKS33}, 
\Color*{HKS57}, \Color*{HKS65}) sowie eine Ausnahmefarbe (\Color*{HKS07}). Diese 
Farben dürfen sowohl in ihrer Grundform als auch in jeweils helleren Tönen 
verwendet werden, wobei eine Abstufung in 10\,\%"~Schritten erfolgen soll. Dafür 
wird das Paket \Package{xcolor} bereits durch die \TUDScript-Klassen geladen, 
mit welchem genau diese Funktion umgesetzt werden kann. Soll beispielsweise die 
Farbe \Color*{HKS44} heller genutzt und ein gewisser Weißanteil beigemischt 
werden, kann dies beispielsweise so wie diese Box 
\colorbox{HKS44!20}{\Macro*{colorbox}\PParameter{HKS44!20}} aussehen.

Für die farbige Gestaltung im \CD (\Option{cd}[color]) ist der Hintergrund der 
Titelseite \Color*{HKS41}, die Schrift auf selbiger in \Color*{HKS41!30} und 
der Kapitelseitenhintergrund in \Color*{HKS41!10} gehalten. Jede der Farben aus 
dem \CD kann sowohl über ihren HKS"=Namen als auch über ein Pseudonym 
angesprochen werden. Diese werden folgend genannt und dargestellt.

Sollen bestimmte Optionen an das Paket \Package{xcolor} weitergereicht werden, 
gibt es dafür zwei Möglichkeiten. Diese kann entweder vor dem Laden der Klasse 
direkt an \Package{xcolor} übergeben werden%
\footnote{%
  \Macro*{PassOptionsToPackage}\Parameter{Option}\PParameter{xcolor} gefolgt von
  \Macro*{documentclass}\OParameter{Klassenoptionen}\PParameter{tudscr\dots}
} oder es wird \Package*{tudscrcolor} mit der entsprechenden Option geladen.%
\footnote{
  \Macro*{usepackage}\OParameter{Option}\PParameter{\Package*{tudscrcolor}};
  \Package*{tudscrcolor} reicht \PName{Option} an \Package{xcolor} weiter
}

\subsection{Primäre Hausfarbe}
\begin{Declaration}{\Color{HKS41}[cddarkblue]}
\printdeclarationlist%
\cdcolor[6]{41}
\end{Declaration}

\subsection{Sekundäre Hausfarbe (Geschäftsausstattung)}
\begin{Declaration}{\Color{HKS92}[cdgray]}
\printdeclarationlist%
\cdcolor[4]{92}
\end{Declaration}

\subsection{Auszeichnungsfarbe 1.Kategorie}
\begin{Declaration}{\Color{HKS44}[cdblue]}
\printdeclarationlist%
\cdcolor[4]{44}
\end{Declaration}

\subsection{Auszeichnungsfarbe 2.Kategorie}
\begin{Declaration}{\Color{HKS36}[cdindigo]}
\begin{Declaration}{\Color{HKS33}[cdpurple]}
\begin{Declaration}{\Color{HKS57}[cddarkgreen]}
\begin{Declaration}{\Color{HKS65}[cdgreen]}
\printdeclarationlist%
\cdcolor[4]{36}\vskip\baselineskipglue
\cdcolor[4]{33}\vskip\baselineskipglue
\cdcolor[2]{57}\vskip\baselineskipglue
\cdcolor{65}
\end{Declaration}
\end{Declaration}
\end{Declaration}
\end{Declaration}

\subsection{Ausnahmefarbe}
\begin{Declaration}{\Color{HKS07}[cdorange]}
\printdeclarationlist%
\cdcolor{07}
\end{Declaration}

\subsection{Zusätzliche Farbdefinitionen}
Das Paket \Package*{tudscrcolor} definiert im Normalfall lediglich die zuvor 
beschriebenen Grundfarben \Color*{HKS41}, \Color*{HKS92}, \Color*{HKS44}, 
\Color*{HKS36}, \Color*{HKS33}, \Color*{HKS57}, \Color*{HKS65} sowie 
\Color*{HKS07}. Alle anderen farblichen Abstufungen können mit den beschrieben 
Möglichkeiten des Paketes \Package{xcolor} generiert werden.

\begin{Declaration}{\Option{full}}
\printdeclarationlist%
In den letzten Jahren sind viele verschiedene Klassen und Pakete für das \CD der 
\TnUD entstanden. Innerhalb dieser existieren abweichende Farbdefinitionen. Um 
eine Migration von anderen Klassen und Paketen auf das \TUDScript-Bundle zu 
ermöglichen, existiert die Paketoption \Option*{full}. Wird diese aktiviert, so 
werden zusätzlich weitere Farben nach dem Schema \Color*{HKS41K}\PName{Zahl} und 
\Color*{HKS41-}\PName{Zahl} bereitgestellt, wobei der hinten angestellte 
Zahlenwert aus der 10er-Reihe kommen muss.
\end{Declaration}


\subsection{Umstellung des Farbmodells}
\index{Farben!Farbmodell}%
Normalerweise verwendet \Package*{tudscrcolor} das CMYK"=Farbmodell. Außerdem 
wird weiterhin noch der RGB"=Farbraum unterstützt. Eine Umschaltung des 
Farbmodells ist beispielsweise für gewisse Funktionen des Paketes \Package{tikz} 
notwendig.

\begin{Declaration}{\Option{RGB}}
\printdeclarationlist%
Mit dieser Option wird bereits beim Laden des Paketes \Package*{tudscrcolor} die 
Farben nicht nach dem CMYK"=Farbmodell sondern im RGB"=Farbraum global definiert.
\end{Declaration}


\begin{Declaration}{\Macro{setcdcolors}\Parameter{Farbmodell}}
\printdeclarationlist%
%
Mit diesem Befehl kann innerhalb des Dokumentes das verwendete Farbmodell 
angepasst werden. Damit ist es möglich, lokal innerhalb einer Umgebung den 
Farbraum zu ändern und so nur in bestimmten Situationen beispielsweise aus dem 
CMYK"=Farbmodell in den RGB"=Farbraum zu wechseln. Unterstützte Werte für 
\PName{Farbmodell} sind \PValue{CMYK} und \PValue{RGB} beziehungsweise 
\PValue{rgb}.
\end{Declaration}


\section{Das Paket \Package*{tudscrsupervisor}}
\DeclarePackage{tudscrsupervisor}
Dieses Paket richtet sich an Betreuer wissenschaftlicher Arbeiten an der \TnUD. 
Es stellt Umgebungen und Befehle für das Erstellen von Aufgabenstellungen, 
Gutachten und Aushängen bereit.

\subsection{Aufgabenstellung für eine wissenschaftliche Arbeit}
\begin{Declaration}{\Environment{task}[\OLParameter{Überschrift}]}{%
  \Environment{tudpage} in \autoref{macros:tudpage}
}
\begin{Declaration}{\Key{\Environment{task}}{headline}[\PName{Überschrift}]}
\printdeclarationlist%
\index{Aufgabenstellung|!(}%
Diese Umgebung basiert auf der \Environment{tudpage}"=Umgebung und es werden 
alle optionalen Parameter dieser unterstützt, welche bei der Beschreibung von 
\Environment'{tudpage} erläutert wurden. Für die Aufgabenstellung wird  
normalerweise eine Überschrift gesetzt, welche sich aus \Term{taskname} und 
gegebenenfalls noch aus \Term{tasktext} und \Macro{thesis} zusammensetzt. Mit 
dem Parameter \Key*{\Environment{task}}{headline} kann diese automatisch 
generierte Überschrift überschrieben werden.

Zu Beginn der Aufgabenstellung erscheint eine Tabelle mit den angegebenen 
Informationen zum Autor bzw. zu den Autoren der Abschlussarbeit. Zwingend 
anzugeben sind dafür lediglich der Name des oder der Verfasser (\Macro{author}) 
sowie der Titel der Arbeit (\Macro{title}), welcher am Ende der Tabelle in 
fetter Schrift aufgeführt wird. Optional werden noch die Felder für den 
Studiengang (\Macro{course}), die Fachrichtung (\Macro{branch}) sowie die 
Matrikelnummer (\Macro{matriculationnumber}) und das Immatrikulationsjahr 
(\Macro{matriculationyear}) ausgegeben, wobei nicht angegebene Felder beim Satz 
ignoriert werden. Der eigentliche Inhalt der Umgebung~-- die Aufgabenstellung 
selbst~-- wird nach dem generierten Kopf ausgegeben

Nach der Ausgabe des Inhaltes der Aufgabenstellung werden der oder die mit 
\Macro{supervisor} definierten Betreuer aufgelistet. Dabei wird unter dem 
jeweiligen Namen selbst der sprachabhängige Bezeichner (\Term{supervisorname}, 
\Term{supervisorothername}) gesetzt. Darauf folgend erscheint das Ausgabedatum 
(\Macro{issuedate}) und der verpflichtende Abgabetermin (\Macro{duedate}). Zum 
Schluss wird die Unterschriftzeile für den Prüfungsausschussvorsitzenden 
(\Macro{chairman}) und den betreuenden Hochschullehrer (\Macro{professor}) 
gesetzt. Für genannte Personen werden unter dem Namen selbst die Bezeichner 
ausgegeben (\Term{chairmanname} und \Term{professorname}).
\end{Declaration}
\end{Declaration}

\begin{Declaration}{\Macro{taskform}\LParameter%
  \Parameter{Ziele}\Parameter{Schwerpunkte}%
}
\printdeclarationlist%
Zusätzlich zur der frei gestaltbaren Umgebung \Environment*{task} zur Erstellung
einer Aufgabenstellung wird ein separater Befehl für eine standardisierte 
Ausgabe zur Verfügung gestellt. Dieser strukturiert die Aufgabenstellung in die 
zwei Bereiche \emph{Ziele} und \emph{Schwerpunkte} der Arbeit mit dazugehörigen 
Überschriften (\Term{objectivesname}, \Term{focusname}).

Im optionalen Argument können alle Parameter der Umgebung \Environment*{task} 
verwendet werden. Im ersten obligatorischen Argument sollte ein Text mit einer 
kurzen thematischen Einordnung und dem eigentlichen Ziel der Arbeit erscheinen. 
im zweiten Argument sollen die thematischen Schwerpunkte in Stichpunkten benannt 
werden. Der Inhalt des zweiten notwendigen Argumentes wird in einer 
\Environment{itemize}"=Umgebung gesetzt. Deshalb \emph{muss} jedem Stichpunkt 
\Macro*{item} vorangestellt werden.
\index{Aufgabenstellung|!)}%
\end{Declaration}
%
\begin{Example}
Die empfohlene Verwendung des Befehls \Macro*{taskform} ist wie folgt:
\begin{code}[escapechar=§]
\taskform{%
  Motivation der Arbeit im ersten Absatz§\dots§
  
  Ziele der Arbeit im zweiten Absatz§\dots§
}{%
  \item Schwerpunkt 1
  \item Schwerpunkt 2
}
\end{code}
Hierzu sei auch auf das Minimalbeispiel in \autoref{sec:mwe:task} verwiesen.
\end{Example}

\begin{Declaration}{\Macro{course}\Parameter{Studiengang}}
\begin{Declaration}{\Macro{branch}\Parameter{Studienrichtung}}
\printdeclarationlist%
\index{Kollaboratives Schreiben}%
Mit diesen beiden Befehlen kann der Studiengang sowie die Studienrichtung für 
den Autor oder die Autoren angegeben werden. Diese Informationen werden zu 
Beginn der \Environment{task}"=Umgebung gesetzten Tabelle ausgegeben. Werden 
diese Befehle innerhalb des Makros \Macro{author} verwendet, können auch 
unterschiedliche Angaben für mehrere Autoren gemacht werden. Dabei sind die 
Autoren mit \Macro{and} voneinander zu trennen.
\end{Declaration}
\end{Declaration}

\begin{Declaration}{\Macro{chairman}\Parameter{Prüfungsausschussvorsitzender}}
\printdeclarationlist%
Wird dieses Feld genutzt, wird neben dem betreuenden Hochschullehrer 
(\Macro{professor}) auch der Vorsitzende des Prüfungsausschusses am Ende der 
Aufgabenstellung aufgeführt
\end{Declaration}

\begin{Declaration}{\Macro{issuedate}\Parameter{Ausgabedatum}}
\begin{Declaration}{\Macro{duedate}\Parameter{Abgabetermin}}
\printdeclarationlist%
Mit diesen beiden Befehlen sollte das Datum der Ausgabe der Aufgabenstellung 
sowie der spätest mögliche  Abgabetermin angegeben werden. Ist das Paket 
\Package{isodate} geladen, wird die damit eingestellte Formatierung des Datums 
durch den Befehl \Macro{printdate} aus diesem Paket für \Macro*{issuedate} und 
\Macro*{duedate} verwendet.
\end{Declaration}
\end{Declaration}

\subsection{Gutachten für wissenschaftliche Arbeiten}
\begin{Declaration}{\Environment{evaluation}[\OLParameter{Überschrift}]}{%
  \Environment{tudpage} in \autoref{macros:tudpage}
}
\begin{Declaration}{\Key{\Environment{evaluation}}{headline}[\PName{Überschrift}]}
\begin{Declaration}{\Key{\Environment{evaluation}}{grade}[\PName{Note}]}
\printdeclarationlist%
\index{Gutachten|!(}%
Diese Umgebung wird für die Erstellung eines Gutachtens bereitgestellt. Auch 
diese unterstützt alle Parameter, welche der Umgebung \Environment'{tudpage} 
beschrieben wurden. Für ein Gutachten wird gewöhnlich eine Überschrift aus 
\Term{evaluationname} und~-- falls der Typ der Abschlussarbeit angegeben 
wurde~-- \Term{evaluationtext} sowie \Macro{thesis} generiert. Diese kann mit 
dem Parameter \Key*{\Environment{evaluation}}{headline} ersetzt werden. Außerdem 
wird am Ende des Gutachtens die mit \Key*{\Environment{evaluation}}{grade} 
gegebene Note in fetter Schrift ausgezeichnet.

Am Anfang der \Environment*{evaluation}"=Umgebung wird die gleiche Tabelle mit 
Autorenangaben ausgegeben, wie dies bei der \Environment{task}"=Umgebung der 
Fall ist. Nach dem Tabellenkopf folgt auch hier der eigentliche Inhalt, sprich 
das Gutachten der Abschlussarbeit. Abgeschlossen wird die Umgebung mit der 
gegebenen Note~-- welche innerhalb von \Term{gradetext} ausgegeben wird~-- sowie 
der Orts- und Datumsangabe (\Macro{place}, \Macro{date}) und der darauffolgenden 
Unterschriftzeile für den oder die Gutachter (\Macro{referee}), welche wiederum 
mit den entsprechenden sprachabhängigen Bezeichner (\Term{refereename}, 
\Term{refereeothername}) ergänzt werden.
\end{Declaration}
\end{Declaration}
\end{Declaration}

\begin{Declaration}{\Macro{evaluationform}\LParameter%
  \Parameter{Aufgabe}\Parameter{Inhalt}\Parameter{Bewertung}\Parameter{Note}%
}
\printdeclarationlist%
Neben der individuell nutzbaren Umgebung \Environment*{evaluation} wird ein 
separater Befehl zur Erstellung eines standardisierten Gutachtens 
bereitgestellt. Dieser strukturiert die Ausgabe in die vier Bereiche 
\emph{Aufgabe}, \emph{Inhalt}, \emph{Bewertung} und \emph{Note} und versieht 
die jeweiligen Bereiche mit der dazugehörigen Überschrift beziehungsweise 
Textausgabe (\Term{taskname}, \Term{contentname}, \Term{assessmentname} und 
\Term{gradetext}). Das optionale Argument unterstützt alle Parameter der 
\Environment{evaluation}"=Umgebung.
\index{Gutachten|!)}%
\end{Declaration}
%
\begin{Example}
Die empfohlene Verwendung des Befehls \Macro*{evaluationform} ist wie folgt:
\begin{code}[escapechar=§]
\evaluationform{%
  Kurzbeschreibung der Aufgabenstellung§\dots§
}{%
  Zusammenfassung von Inhalt und Struktur§\dots§
}{%
  Bewertung der schriftlichen Abschlussarbeit§\dots§
}{%
  Zahl (Note)
}
\end{code}
Hierzu sei auch auf das Minimalbeispiel in \autoref{sec:mwe:eval} verwiesen.
\end{Example}

\begin{Declaration}{\Macro{grade}\Parameter{Note}}
\printdeclarationlist%
Neben der Angabe der Note für ein Gutachten über den Parameter 
\Key*{\Environment{evaluation}}{grade} kann dafür auch dieser global wirkende 
Befehl verwendet werden.
\end{Declaration}

\subsection{Aushang}
\begin{Declaration}{\Environment{notice}[\OLParameter{Überschrift}]}{%
  \Environment{tudpage} in \autoref{macros:tudpage}
}
\begin{Declaration}{\Key{\Environment{notice}}{headline}[\PName{Überschrift}]}
\printdeclarationlist%
\index{Aushang|!(}%
Für das Anfertigen eines Aushangs kann diese Umgebung verwendet werden. Sie 
basiert abermals auf der Umgebung \Environment'{tudpage} und unterstützt alle 
deren Parameter.

Wurde ein Datum angegeben, wird dieses in der oberen rechten Ecke gesetzt. 
Anschließend wird die Überschrift gesetzt, welche für gewöhnlich dem Inhalt von 
\Term{noticename} entspricht und mit dem optionalen Parameter 
\Key*{\Environment{notice}}{headline} geändert werden kann. Nach der Überschrift 
wird bereits der Inhalt der Umgebung ausgegeben. Wurde mit \Macro{contact} ein 
oder mehrere Ansprechpartner angegeben, werden diese Informationen am Ende der 
Umgebung ausgegeben.
\end{Declaration}
\end{Declaration}

\begin{Declaration}{\Macro{noticeform}\LParameter%
  \Parameter{Inhalt}\Parameter{Schwerpunkte}%
}
\printdeclarationlist%
Auch für diese Umgebung gibt es einen Befehl für eine normierte Form. Diese soll 
vor allem Verwendung für den Aushang studentischer Arbeitsthemen finden. Für das 
optionale Argument werden alle Parameter der \Environment*{notice}"=Umgebung 
unterstützt.

Das erste obligatorische Argument sollte für eine kurze Inhaltsbeschreibung 
verwendet werden. Neben dem textuellen Teil sollte hier wenn möglich eine 
thematisch passende Abbildung eingebunden werden (\Macro{includegraphics}). Das 
zweite Argument wird~--wie schon bei \Macro{taskform}~-- dazu verwendet, einige 
Schwerpunkte aufzuzählen. Auch hier kommt nach der gliedernden Überschrift 
(\Term{focusname}) eine \Environment{itemize}"=Umgebung zum Einsatz, allen 
Schwerpunkten muss ein \Macro*{item} vorangestellt werden.
\index{Aushang|!)}%
\end{Declaration}
%
\begin{Example}
Die empfohlene Verwendung des Befehls \Macro*{noticeform} ist wie folgt:
\begin{code}[escapechar=§]
\noticeform{%
  Kurzbeschreibung des Inhaltes der studentischen Arbeit§\dots§
  
  Bild (optional), einzubinden mit:
    \includegraphics[§\PName{Optionen}§]{§\PName{Datei}§}
}{%
  \item Schwerpunkt 1
  \item Schwerpunkt 2
}
\end{code}
Hierzu sei auch auf das Minimalbeispiel in \autoref{sec:mwe:note} verwiesen.
\end{Example}

\begin{Declaration}{\Macro{contact}\Parameter{Kontaktperson(en)}}
\begin{Declaration}{\Macro{office}\Parameter{Dienstsitz}}
\begin{Declaration}{\Macro{phone}\Parameter{Telefonnummer}}
\begin{Declaration}{\Macro{email}\Parameter{E-Mail-Adresse}}
\printdeclarationlist%
Am Ende eines Aushangs können mit \Macro*{contact} Kontaktinformationen für eine 
oder mehrere Personen angegeben werden. Soll mehr als eine Kontaktperson genannt 
werden, so müssen diese innerhalb von \Macro*{contact} mit dem Befehl 
\Macro{and} getrennt werden. Für jede Person kann innerhalb von \Macro*{contact} 
der Dienstsitz (\Macro*{office}), die dienstliche Telefonnummer (\Macro*{phone}) 
sowie die geschäftliche E-Mail"=Adresse (\Macro*{email}) angegeben werden. 
Sollte das Paket \Package*{hyperref} geladen werden, wird die gegebene 
E-Mail"=Adresse direkt in einen entsprechenden Link gewandelt.
\end{Declaration}
\end{Declaration}
\end{Declaration}
\end{Declaration}

\subsection{Zusätzliche sprachabhängige Bezeichner}
\index{Bezeichner|!(}
Für das Paket \Package*{tudscrsupervisor} werden für die zusätzlichen Umgebungen 
weiter Bezeichner definiert. Für eine etwaige Anpassung dieser sei auf 
\autoref{sec:localization} verwiesen.
\begin{Declaration}{\Term{taskname}}
\begin{Declaration}{\Term{tasktext}}
\printdeclarationlist%
Die Bezeichnung der Aufgabenstellung selbst ist in \Term*{taskname} enthalten. 
Für die Generierung einer Überschrift wird dieser verwendet. Wurde außerdem mit 
\Macro{thesis} oder \Macro{subject} der Typ der Abschlussarbeit%
\footnote{%
  \Option{subjectthesis} oder spezieller Wert aus \autoref{tab:thesis}
}
angegeben, wird die Überschrift zusammen mit dem Bezeichner \Term*{tasktext}
um die Typisierung erweitert. Falls gewünscht, kann die generierte Überschrift 
mit dem Parameter \Key{\Environment{task}}{headline} der Umgebung 
\Environment{task} überschrieben werden.
\TermTable{taskname,tasktext}
\end{Declaration}
\end{Declaration}

\begin{Declaration}{\Term{authorname}}
\begin{Declaration}{\Term{titlename}}
\begin{Declaration}{\Term{coursename}}
\begin{Declaration}{\Term{branchname}}
\printdeclarationlist%
Diese Bezeichner werden in der Tabelle mit den Autoreninformationen zu Beginn 
der Aufgabenstellung verwendet. Dabei werden \Term*{coursename} und 
\Term*{branchname} nur genutzt, wenn für mindestens einen Autor die Befehle 
\Macro{course} und/oder \Macro{branch} verwendet wurden.
\TermTable{authorname,titlename,coursename,branchname}
\end{Declaration}
\end{Declaration}
\end{Declaration}
\end{Declaration}

\begin{Declaration}{\Term{issuedatetext}}
\begin{Declaration}{\Term{duedatetext}}
\printdeclarationlist%
Am Ende der Aufgabenstellung wird nach dem oder der Betreuer das Ausgabe- und 
Abgabedatum (\Macro{issuedate}, \Macro{duedate}) der Abschlussarbeit mit 
folgenden Bezeichner erläutert.
\TermTable{issuedatetext,duedatetext}
\end{Declaration}
\end{Declaration}

\begin{Declaration}{\Term{chairmanname}}
\printdeclarationlist%
Wurde der Prüfungsausschussvorsitzende (\Macro{chairman}) angegeben, erfolgt 
unter dem Namen selbst die Ausgabe des Bezeichners.
\TermTable{chairmanname}
\end{Declaration}

\begin{Declaration}{\Term{focusname}}
\begin{Declaration}{\Term{objectivesname}}
\printdeclarationlist%
Wird die standardisierte Form der Aufgabenstellung genutzt (\Macro{taskform}), 
so werden die zwei Überschriften mit diesen Bezeichnern gesetzt.
\TermTable{focusname,objectivesname}
\end{Declaration}
\end{Declaration}

\begin{Declaration}{\Term{evaluationname}}
\begin{Declaration}{\Term{evaluationtext}}
\printdeclarationlist%
Die Bezeichnung des Gutachten selbst ist in \Term*{evaluationname} enthalten. 
Für die Generierung einer Überschrift wird der Bezeichner \Term*{evaluationtext} 
sowie der mit \Macro{thesis} oder gegebenenfalls mit \Macro{subject} gegebenen 
Typ der Abschlussarbeit verwendet. Falls gewünscht, kann diese generierte 
Überschrift mit dem Parameter \Key{\Environment{evaluation}}{headline} der 
Umgebung \Environment{evaluation} überschrieben werden.
\TermTable{evaluationname,evaluationtext}
\end{Declaration}
\end{Declaration}

\begin{Declaration}{\Term{contentname}}
\begin{Declaration}{\Term{assessmentname}}
\printdeclarationlist%
Bei der standardisierten Form des Gutachten (\Macro{evaluationform}) werden die 
gliedernden Überschriften mit den Bezeichnern \Term*{taskname}, 
\Term*{contentname} und \Term*{assessmentname} gesetzt.
\TermTable{contentname,assessmentname}
\end{Declaration}
\end{Declaration}

\begin{Declaration}{\Term{gradetext}}
\printdeclarationlist%
Wird für ein Gutachten mit dem Befehl \Macro{grade}\Parameter{Note} oder mit dem 
Parameter \Key{\Environment{evaluation}}{grade}[\PName{Note}] die Note 
angegeben, so wird diese innerhalb von \Term*{gradetext} verwendet.
\grade{\PName{Note}}
\TermTable*{gradetext}{.7\textwidth}
\end{Declaration}

\begin{Declaration}{\Term{noticename}}
\begin{Declaration}{\Term{contactname}}
\printdeclarationlist%
Die Bezeichnung des Aushangs selbst ist in \Term*{noticename} enthalten. Für die 
Generierung einer Überschrift wird dieser verwendet. Falls gewünscht, kann die 
generierte Überschrift mit dem Parameter \Key{\Environment{notice}}{headline} 
der Umgebung \Environment{notice} überschrieben werden. Der Bezeichner 
\Term*{contactname} wird für die Überschrift der Kontaktdaten verwendet, falls 
\Macro{contact} zum Einsatz kam.
\TermTable{noticename,contactname}
\end{Declaration}
\end{Declaration}
\index{Bezeichner|!)}


\section{Das Paket \Package*{mathswap}}
\DeclarePackage{mathswap}
\index{Trennzeichen}\index{Mathematiksatz}
\index{Trennzeichen!Dezimaltrennzeichen}
\index{Trennzeichen!Tausendertrennzeichen}
Die Verwendung von Dezimal- und Tausendertrennzeichen im mathematischen Satz 
sind regional sehr unterschiedlich. In den meisten englischsprachigen Ländern 
wird der Punkt als Dezimaltrennzeichen und das Komma zur Zifferngruppierung 
verwendet, im restlichen Europa wird dies genau entgegengesetzt praktiziert.
Dieses Paket soll dazu dienen, beliebige formatierte Zahlen in ihrer Ausgabe 
anzupassen. Dafür werden die Zeichen Punkt (\ .\ ) und Komma (\ ,\ ) als 
aktive Zeichen im Mathematikmodus definiert.

Ähnliche Funktionalitäten werden bereits durch die Pakete \Package{icomma} und 
\Package{ziffer} bereitgestellt. Bei \Package{icomma} muss jedoch beim
Verfassen des Dokumentes durch den Autor beachtet werden, ob das verwendete
Komma einem Dezimaltrennzeichen entspricht ($t=1,\!2$) oder aber einem
normalen Komma im Mathematiksatz ($z=f(x,y)$), wo ein gewisser Abstand nach
dem Komma durchaus gewünscht ist. Das Paket \Package{ziffer} liefert dafür die
gewünschte Funktionalität,%
\footnote{kein Leerraum nach Komma, wenn direkt danach eine Ziffer folgt}
ist allerdings etwas unflexibel, was den Umgang mit den Trennzeichen anbelangt.
Als Alternative zu diesem Paket kann außerdem \Package{ionumbers} verwendet 
werden.

Das Paket \Package*{mathswap} sorgt dafür, dass Trennzeichen direkt vor einer 
Ziffer erkannt und nach bestimmten Vorgaben ersetzt werden. Sollte sich jedoch 
zwischen Trennzeichen und Ziffer Leerraum befinden, wird dieser als solcher
auch gesetzt. Zur Verwendung des Paketes sei auf das Minimalbeispiel in 
\autoref{sec:mwe:swap} verwiesen. Außerdem wird in \autoref{sec:mathtype} kurz 
auf einen typographisch sauberen Mathematiksatz eingegangen.

\begin{Declaration}{\Macro{commaswap}\Parameter{Trennzeichen}}
\begin{Declaration}{\Macro{dotswap}\Parameter{Trennzeichen}}
\printdeclarationlist%
Die beiden Befehle \Macro*{commaswap} und \Macro*{dotswap} sind die zentrale 
Benutzerschnittstelle des Paketes. Das Makro \Macro*{commaswap} definiert das 
Trennzeichen oder den Inhalt, wodurch ein Komma ersetzt werden soll, auf 
welches direkt danach eine Ziffer folgt. Normalerweise setzt \hologo{LaTeX}
nach einem Komma im mathematischen Satz zusätzlich einen horizontalen Abstand.
Bei der Ersetzung durch \Macro*{commaswap} entfällt dieser. Die Voreinstellung
für \Macro*{commaswap} ist deshalb auf ein Komma (,) gesetzt.

Mit dem Makro \Macro*{dotswap} kann definiert werden, wodurch der Punkt im 
mathematischen Satz ersetzt werden soll, wenn auf diesen direkt anschließend 
eine Ziffer folgt. Da der Punkt im deutschsprachigem Raum zur Gruppierung von
Ziffern genutzt wird, ist hierfür standardmäßig ein halbes geschütztes
Leerzeichen definiert (\Macro*{,}).
\end{Declaration}
\end{Declaration}


\section{Das Paket \Package*{tudscrcomp}}
\DeclarePackage{tudscrcomp}
\index{Kompatibilität!\Class*{tudbook}|(}
Dieses Paket dient zum Umstieg von der alten Klasse \Class{tudbook} auf die 
neuen Klassen aus dem \TUDScript-Bundle. Nach dem Laden dieses Paketes stehen 
Befehle bereit, welche für \Class{tudbook} definiert wurden und das 
entsprechende Verhalten nachahmen. Die Intention ist, alte Dokumente möglichst 
schnell auf die neuen Klassen portieren zu können. Für den Satz neuer Dokumente 
wird jedoch empfohlen, auf den Einsatz dieses Paketes zu verzichten und 
stattdessen die neuen Befehle zu nutzen.

\begin{Declaration}{\Option{serifmath}}{%
  identisch zu \Option{sansmath}[false]%
}
\printdeclarationlist%
Die Funktionalität wird durch die Option \Option{sansmath} bereitgestellt.
\end{Declaration}

\begin{Declaration}{\Option{colortitle}}{%
  identisch zu \Option{cdtitle}[color]%
}
\begin{Declaration}{\Option{nocolortitle}}{%
  identisch zu \Option{cdtitle}[true]%
}
\printdeclarationlist%
Die Funktionalität wird durch die Option \Option{cdtitle} bereitgestellt.
\end{Declaration}
\end{Declaration}

\begin{Declaration}{\Macro{einrichtung}\Parameter{Fakultät}}{%
  identisch zu \Macro{faculty}
}
\begin{Declaration}{\Macro{fachrichtung}\Parameter{Einrichtung}}{%
  identisch zu \Macro{department}
}
\begin{Declaration}{\Macro{institut}\Parameter{Institut}}{%
  identisch zu \Macro{institute}
}
\begin{Declaration}{\Macro{professur}\Parameter{Lehrstuhl}}{%
  identisch zu \Macro{chair}
}
\printdeclarationlist%
Dies sind die deutschsprachigen Befehle für den Kopf im \CD.
\end{Declaration}
\end{Declaration}
\end{Declaration}
\end{Declaration}

\begin{Declaration}{\Macro{moreauthor}\Parameter{Autorenzusatz}}{%
  identisch zu \Macro{authormore}%
}
\printdeclarationlist%
Ursprünglich war diese Befehl für das Unterbringen aller möglichen, zusätzlichen 
Autoreninformationen gedacht. Auch der Befehl \Macro{authormore} ist ein 
Rudiment davon. Empfohlen wird die Verwendung der Befehle \Macro{dateofbirth}, 
\Macro{placeofbirth}, \Macro{matriculationnumber} und \Macro{matriculationyear} 
sowie für die Aufgabenstellung einer wissenschaftlichen Arbeit \Macro{course} 
und \Macro{branch} aus dem Paket \Package{tudscrsupervisor}.
\end{Declaration}

\begin{Declaration}{\Macro{submitdate}\Parameter{Datum}}{%
  identisch zu \Macro{date}%
}
\printdeclarationlist%
Die Funktionalität wird durch den erweiterten Standardbefehl \Macro{date} 
abgedeckt.
\end{Declaration}

\begin{Declaration}{\Macro{supervisorII}\Parameter{Name}}{%
  identisch zur Verwendung von \Macro{and} innerhalb von \Macro{supervisor}%
}
\printdeclarationlist%
Es sollte \Macro{supervisor}\PParameter{\PName{Name} \Macro{and} \PName{Name}}
anstelle des Befehls \Macro*{supervisorII}\Parameter{Name} verwendet werden.
\end{Declaration}

\begin{Declaration}{\Macro{supervisedby}\Parameter{Bezeichnung}}{%
  siehe \Term{supervisorname}%
}
\begin{Declaration}{\Macro{supervisedIIby}\Parameter{Bezeichnung}}{%
  siehe \Term{supervisorothername}%
}
\begin{Declaration}{\Macro{submittedon}\Parameter{Bezeichnung}}{%
  siehe \Term{datetext}%
}
\printdeclarationlist%
Zur Änderung der Bezeichnung der Betreuer sollten die sprachabhängigen 
Bezeichner wie in \autoref{sec:localization} beschrieben angepasst werden. Eine 
Verwendung der alten Befehle entfernt die Abhängigkeit der Bezeichner von der 
verwendeten Sprache.
\end{Declaration}
\end{Declaration}
\end{Declaration}

\begin{Declaration}{\Macro{dissertation}}{%
  identisch zu \Option{ddcfoot}[true]%
}
\printdeclarationlist%
Die Funktionalität kann durch die Befehle \Macro{thesis}\PParameter{diss} und 
\Macro{referee} sowie die Bezeichner \Term{refereename} und 
\Term{refereeothername} dargestellt werden.
\end{Declaration}

\begin{Declaration}{\Option{ddcfooter}}{%
  identisch zu \Option{ddcfoot}[true]%
}
\printdeclarationlist%
Die Funktionalität wird durch die Option \Option{ddcfoot} bereitgestellt.
\end{Declaration}

\begin{Declaration}{\Macro{chapterpage}}
\printdeclarationlist%
Durch diesen Befehl können Kapitelseiten konträr zur eigentlichen Einstellung 
aktiviert oder deaktiviert werden. Prinzipiell ist dies auch durch Änderung der 
Option \Option{chapterpage} möglich. Allerdings wird davon abgeraten, da dies zu 
einem inkonsistenten Layout innerhalb des Dokumentes führt.
\end{Declaration}

\begin{Declaration}{\Environment{theglossary}[\OParameter{Präambel}]}
\begin{Declaration}{\Macro{glossitem}\Parameter{Begriff}}
\printdeclarationlist%
Die \Class{tudbook}-Klasse stellt eine rudimentäre Umgebung für ein Glossar 
bereit. Allerdings gibt es dafür bereits zahlreiche und besser implementierte 
Pakete. Daher wird für diese Umgebung keine Portierung vorgenommen, sondern 
lediglich die ursprüngliche Definition übernommen. Allerdings sein an dieser 
Stelle auf wesentlich bessere Lösungen wie beispielsweise das Paket 
\Package{glossaries} oder~-- mit Abstrichen~-- das nicht ganz so umfangreiche 
Paket \Package{nomencl} verwiesen. 
\end{Declaration}
\end{Declaration}
\index{Kompatibilität!\Class*{tudbook}|)}


\section{Das Paket \Package*{twocolfix}}
\DeclarePackage{twocolfix}
\index{Zweispaltensatz|?}%
Der \hologo{LaTeXe}-Kernel enthält einen Fehler, der Kapitelüberschriften im
zweispaltigen Layout höher setzt, als im einspaltigen. Der Fehler%
\footnote{\url{http://www.latex-project.org/cgi-bin/ltxbugs2html?pr=latex/3126}}
ist zwar schon länger bekannt, allerdings noch nicht in \Package{ltxfix2e} 
übernommen worden. Das Paket \Package{twocolfix} behebt das Problem. Eine 
Integration dieses Bugfixes in \KOMAScript{} wurde bereits bei Markus Kohm 
angefragt, jedoch von ihm bis jetzt nicht weiter verfolgt.%
\footnote{\url{http://www.komascript.de/node/1681}}


\section{Zukünftige Arbeiten}
\subsection*{Das Paket \Package*{tudscrposter}}
\DeclarePackage{tudscrposter}
\ToDo[v2.x]{Paket \Package*{tudscrposter} als Ersatz für \Class{tudmathposter}}
Die Funktionalität der Klasse \Class{tudmathposter} soll in ein eigenständiges 
Paket mit dem Namen \Package*{tudscrposter} o.\,ä. überführt werden.

\subsection*{\Class*{tudscrlttr} auf Basis von \Class*{scrlttr2}}
\DeclareClass{tudscrlttr}
\ToDo[v2.x]{Klasse \Class*{tudscrlttr}}
Es soll die Klasse
\begin{description}
  \item \Class*{tudscrlttr}
\end{description}
für Briefe im \CD der \TnUD entstehen. Auch Vorlagen für Fax und 
Hausmitteilungen sollen dabei abfallen.



%\chapter[Die Briefklasse \Class*{tudscrlttr}]{Die Briefklasse}



\setchapterpreamble{%
  \begin{abstract}
  \noindent Für die Verwendung des \TUDScript-Bundles ist es nicht notwendig, 
  dieses Kapitel zu lesen. Allerdings soll hier insbesondere dem Einsteiger~-- 
  und auch dem bereits versierten \hologo{LaTeX}-Nutzer~-- der eine oder andere 
  Tipp gegeben und meiner Meinung nach empfehlenswerte Pakete kurz vorgestellt  
  werden. Über Anregungen, Hinweise, Ratschläge oder Empfehlungen für weitere 
  Pakete würde ich mich freuen.
  \end{abstract}
}
\chapter{Ergänzungen und Hinweise}
\index{Kompatibilität!Pakete}
\section{Durch die neuen Hauptklassen genutzte Pakete}
\label{sec:packages:needed}
In diesem Abschnitt werden alle Pakete genannt, die von den neuen Klassen 
zwingend benötigt und geladen werden, um den Anwender das mehrmalige Laden 
dieser Pakete oder mögliche Konflikte mit anderen Paketen zu ersparen.
%
\begin{packages}
\item[scrbase]
  Dieses Paket gehört zum \KOMAScript-Bundle und erlaubt das Definieren von 
  Klassenoptionen im Stil von \KOMAScript, welche auch noch nach dem Laden der 
  Klasse mit den Befehlen \Macro*{TUDoption} und \Macro*{TUDoptions} geändert 
  werden können.
\item[keyval]
  Es wird von \KOMAScript{} selbst benötigt und geladen und erlaubt das  
  Definieren von Klassen"~ und Paketoptionen sowie Parametern nach dem   
  Schlüssel"=Wert"=Prinzip.
\item[kvsetkeys]
  Hiermit wird das Paket \Package*{keyval} verbessert. Unter anderem kann das 
  Verhalten für den Fall, dass ein unbekannter Schlüssel übergeben wird,  
  festgelegt werden.
\item[etoolbox]
  Es werden viele Funktionen zum Testen und zur Ablaufkontrolle bereitgestellt 
  und das einfache Manipulieren vorhandener Makros ermöglicht.
\item[graphicx]\index{Grafiken}
  Dies ist das De-facto-Standard-Paket zum Einbinden von Grafiken. Es wird für 
  die Einbindung des Logos der \TnUD im Kopf benötigt. Dafür wird der Befehl
  \Macro{includegraphics}\OParameter{Parameter}\Parameter{Dateiname} genutzt.
\item[xcolor]\index{Farben}
  Damit werden die Farben des \CDs zur Verwendung im Dokument definiert.  
  Genaueres ist bei der Beschreibung von \Package{tudscrcolor} in   
  \autoref{files:tudscrcolor} zu finden.
\item[geometry]\index{Satzspiegel}
  Das Paket ist essentiell für die \TUDScript-Klassen. Es wird zum Festlegen 
  der Seitenränder respektive des Satzspiegels verwendet. Genaueres ist der 
  Beschreibung der Option \Option{geometry} zu entnehmen.
\item[environ]\index{Befehle!Deklaration}
  Es wird eine verbesserte Deklaration von Umgebungen ermöglicht, bei der auch 
  beim Abschluss der Umgebung auf die übergebenen Parameter zugegriffen werden 
  kann. Dies wird die Neugestaltung der \Environment{abstract}"=Umgebung   
  benötigt.
\item[textcase]\index{Schriftauszeichnung}
  Es werden die Befehle \Macro{MakeTextUppercase} für die erzwungene   
  Großschreibung der Überschriften in \DIN genutzt. Um dies im Ausnahmefall
  zu unterbinden, kann der Befehl \Macro{NoCaseChange} genutzt werden.
\item[mweights]\index{Schrift!Stärke}
  Es kann die Schriftstärke jeder einzelnen Schriftfamilie individuell 
  festgelegt werden.
\item[trimspaces]
  Entfernt überflüssigen Leerraum zu Beginn und am Ende eines Strings.    
\end{packages}
%
Soll eines der hier aufgezählten Pakete mit bestimmten Optionen geladen werden, 
so müssen diese bereits \emph{vor} der Definition der Dokumentklasse an das 
entsprechende Paket werden.
\begin{Example}
Das Weiterreichen von Optionen an Pakete muss folgendermaßen erfolgen:
\begin{code}[escapechar=§]
\PassOptionsToPackage§\Parameter{Optionenliste}\Parameter{Paket}§
\documentclass§\OParameter{Klassenoptionen}\PParameter{tudscr\dots}§
\end{code}
\end{Example}


\section{Durch \TUDScript direkt unterstütze Pakete}
\begin{packages}
\item[hyperref]\index{Lesezeichen}
  Mit diesem Paket können in einem PDF-Dokument Lesezeichen, Querverweise und 
  Hyperlinks erstellt werden. Wird \Package*{hyperref} geladen, können die 
  Option \Option{tudbookmarks} sowie der Befehl \Macro{tudbookmark} genutzt 
  werden. Das Paket sollte als letztes im Vorspann eingebunden werden.%
  \footnote{%
    \Package{glossaries} ist eine von wenigen Ausnahmen und muss \textbf{nach} 
    \Package*{hyperref} geladen werden.
  }
  Das Paket \Package{bookmark} erweitert die Unterstützung nochmals.
\item[isodate]\index{Datum|?}
  Dieses Paket formatiert mit \Macro{printdate}\Parameter{Datum} die Ausgabe 
  eines Datums automatisch in ein spezifiziertes Format. Wird es geladen, werden 
  alle Datumsfelder, welche durch die \TUDScript-Klassen definiert wurden,%
  \footnote{%
    \Macro{date}, \Macro{dateofbirth}, \Macro{defensedate}, \Macro{duedate}, 
    \Macro{issuedate}
  }
  in diesem Format ausgegeben.
\item[multicol]\index{Zweispaltensatz|?}
  Hiermit kann jeglicher beliebiger Inhalt in zwei oder mehr Spalten ausgegeben 
  werden, wobei~-- im Gegensatz zur \hologo{LaTeX}-Option \Option{twocolumn}~-- 
  für einen Spaltenausgleich gesorgt wird. Unterstützt wird das Paket innerhalb 
  der Umgebungen \Environment{abstract} und \Environment{tudpage}.
\item[ragged2e]\index{Worttrennung}
  Das Paket verbessert den Flattersatz, indem für diesen die Worttrennung 
  aktiviert wird.
\item[quoting]\index{Zitate}
  \hologo{LaTeX} bietet von Haus aus \emph{zwei} verschiedene Umgebungen für 
  Zitate und ähnliches. Beide sind in ihrer Ausprägung starr und ignorieren 
  beispielsweise die Einstellungen von \Option{parskip}. Dies wird durch die 
  Umgebung \Environment{quoting} verbessert. Wird das Paket geladen, kommt diese 
  gegebenenfalls innerhalb der \Environment{abstract}="Umgebung zum Einsatz.
\item[pagecolor]\index{Farben}
  Mit dem Paket kann die Hintergrundfarbe der Seiten im Dokument geändert 
  werden. Nach der Ausgabe einer farbigen Titel"~, Teile"~, oder Kapitelseite 
  wird auf diese zurückgeschaltet.
\end{packages}


\section{Empfehlenswerte Pakete}
\label{sec:packages:recommended}
In diesem Abschnitt wird eine Vielzahl an Paketen genannt und~-- zumeist kurz~-- 
charakterisiert, welche sich bei meiner Arbeit mit \hologo{LaTeX} bewährt haben.
Für detaillierte Informationen sowie bei Fragen zu den einzelnen Paketen sollte 
die jeweilige Dokumentation zu Rate gezogen werden,%
\footnote{Konsolenaufruf: \Path{texdoc\,\emph{<Paketname>}}}
das Lesen der hier gegebenen Kurzbeschreibung ersetzt dies in keinem Fall.
\subsection{Pakete zur Verwendung in jedem Dokument}
Die hier vorgestellten Pakete gehören meiner Meinung nach in die Präambel eines 
jeden Dokumentes. Egal, in welcher Sprache das Dokument verfasst wird, sollte 
diese mit dem Paket \Package*{babel} definiert werden~-- auch wenn dies Englisch 
ist. Für deutschsprachige Dokumente ist für eine annehmbare Worttrennung das 
Paket \Package*{hyphsubst} unbedingt zu verwenden.
\begin{packages}
\item[fontenc]\index{Zeichensatzkodierung}
  Das Paket erlaubt Festlegung der Zeichensatzkodierung des Ausgabefonts. Als 
  Voreinstellung ist die Ausgabe als 7"~bit kodierte Schrift gewählt, was unter 
  anderem dazu führt, dass keine echten Umlaute im erzeugten PDF-Dokument 
  verwendet werden. Um auf 8"~bit"~Schriften zu schalten, sollte man
  \Macro*{usepackage}\POParameter{T1}\PParameter{fontenc} nutzen.
\item[selinput]\index{Eingabekodierung}
  Hiermit erfolgt die (automatische) Festlegung der Eingabekodierung. Diese ist 
  von der gewählten Einstellung des \hyperref[sec:tat:editor]{Editors} abhängig. 
  Zu verwenden ist es wie folgt:
  \begin{code}
  \usepackage{selinput}
  \SelectInputMappings{adieresis={ä},germandbls={ß}}
  \end{code}\vskip-\baselineskip%
  Dies macht den Quelltext portabel. Außerdem kann so beispielsweise ganz 
  einfach via Copy~\&~Paste ein \hyperref[fn:mwe]{Minimalbeispiel} bei 
  Problemstellungen in einem Forum gepostet werden. Alternativ dazu kann mit dem 
  Paket \Package{inputenc} die Eingabekodierung manuell eingestellt werden 
  (\Macro*{usepackage}\OParameter{Eingabekodierung})\PParameter{inputenc}).
\item[microtype]\index{Typographie}
  Dieser Paket kümmert sich um den optischen Randausgleich%
  \footnote{englisch: protrusion, margin kerning}
  und das Nivellieren der Wortzwischenräume%
  \footnote{englisch: font expansion}
  im Dokument. Es funktioniert nicht mit der klassischen \hologo{TeX}-Engine, 
  wohl aber mit \hologo{pdfTeX} als auch \hologo{LuaTeX} sowie \hologo{XeTeX}.
\item[babel]\index{Sprachunterstützung}\index{Bezeichner}
  Mit diesem Paket erfolgt die Einstellung der im Dokument verwendeten 
  Sprache(n). Bei mehreren angegebenen Sprachen ist die zuletzt geladene die 
  Hauptsprache des Dokumentes. Die gewünschten Sprachen sollten als nicht als 
  Paketoption sondern als Klassenoption und gesetzt werden, damit auch andere 
  Pakete auf die Spracheinstellungen zugreifen können. Für deutschsprachige 
  Dokumente ist die Option \Option*{ngerman} für die neue oder \Option*{german} 
  für die alte deutsche Rechtschreibung zu verwenden. 
  
  Mit dem Laden von \Package*{babel} und der dazugehörigen Sprachen werden 
  sowohl die Trennmuster als auch die sprachabhängigen Bezeichner angepasst.
  Von einer Verwendung der obsoleten Pakete \Package*{german} beziehungsweise 
  \Package*{ngerman} anstelle von \Package*{babel} wird abgeraten. Für 
  \hologo{LuaLaTeX} und \hologo{XeLaTeX} kann das Paket \Package{polyglossia} 
  genutzt werden.
\item[hyphsubst]\index{Worttrennung|!}
  Die möglichen Trennstellen von Wörtern wird von \hologo{LaTeX} mit Hilfe eines 
  Algorithmus berechnet. Dieser ist jedoch in seiner ursprünglichen Form für die 
  englische Sprache konzipiert worden. Für deutschsprachige Texte wird die 
  Worttrennung~-- insbesondere für zusammengeschriebenen Wörtern~-- mit dem 
  Paket \Package*{hyphsubst} entscheidend verbessert. Dafür wird ein um 
  Trennstellen ergänztes Wörterbuch aus dem Paket \Package{dehyph-exptl} 
  genutzt. \Package*{hyphsubst} muss bereits \emph{vor} den Dokumentklassen 
  selbst wie folgt geladen werden:
  \begin{code}[escapechar=§]
  \RequirePackage§\POParameter{ngerman=ngerman-x-latest}\PParameter{hyphsubst}§
  \documentclass§\OParameter{Klassenoptionen}\PParameter{tudscr\dots}§
  \end{code}\vskip-\baselineskip%
  Sollte trotzdem einmal ein bestimmtes Wort falsch getrennt werden, so kann die 
  Worttrennung dieses Wortes manuell und global geändert werden. Dies wird mit 
  dem Befehl \Macro{hyphenation}\PParameter{Sil-ben-tren-nung} gemacht. Es 
  ist zu beachten, dass dies für alle Flexionsformen des Wortes erfolgen sollte. 
  Für eine lokale/temporäre Worttrennung kann mit Befehlen aus dem Paket 
  \Package{babel} gearbeitet werden. Diese sind:
  %
  \vskip\topsep\noindent
  \begin{tabular}{@{}ll}
  \textbf{Beschreibung} & \textbf{Befehl}\tabularnewline
  ausschließliche Trennstellen & \textbackslash-\tabularnewline
  zusätzliche Trennstellen & "'-\tabularnewline
  Umbruch ohne Trennstrich & "'"'\tabularnewline
  Bindestrich, welcher weitere Trennstellen erlaubt & "'=\tabularnewline
  geschützter Bindestrich ohne Umbruch & "'\textasciitilde\tabularnewline
  \end{tabular}
  \vskip\topsep
\end{packages}

\subsection{Pakete zur situativen Verwendung}
\subsubsection{Verzeichnisse aller Art}
\index{Verzeichnisse|?}
Neben dem Erstellen des eigentlichen Dokumentes sind für eine wissenschaftliche 
Arbeit meist auch allerhand Verzeichnisse gefordert. Fester Bestandteil ist 
dabei das Literaturverzeichnis, aber auch ein Abkürzungs- und Formelzeichen- 
beziehungsweise Symbolverzeichnis werden häufig gefordert. Gegebenenfalls wird 
auch noch ein Glossar benötigt. Hier werden die passenden Pakete vorgestellt. 
Für das Erstellen eines Quelltextverzeichnisses sei auf \Package'{listings} 
verwiesen.
\begin{packages}
\item[biblatex]
  Das Paket gibt es seit geraumer Zeit und es kann als legitimer Nachfolger zu 
  \Package*{bibtex} gesehen werden. Ähnlich zu diesem bietet \Package*{biblatex} 
  die Möglichkeit, Literaturdatenbanken einzubinden und verschiedene Stile der 
  Referenzierung und Darstellung des Literaturverzeichnisses auszuwählen. 
  Allerdings ist mit \Package*{biblatex} die Anpassung eines bestimmten Stiles 
  wesentlich besser umsetzbar als mit \Package*{bibtex}. Wird zum Sortieren des 
  Verzeichnisses außerdem \Application{biber} genutzt, ist die Verwendung 
  einer Unicode kodierte Literaturdatenbank problemlos möglich. In Verbindung 
  mit \Package*{biblatex} sei die Verwendung von \Package*{csquotes} empfohlen.
\item[acro]
  Soll lediglich ein Abkürzungsverzeichnis erstellt werden, ist dieses Paket die 
  erste Wahl. Es stellt Befehle zur Definition von Abkürzungen sowie zu deren 
  Verwendung im Text und zur sortierten Ausgabe eines Verzeichnisses bereit. 
  Alternativ dazu kann das Paket \Package{acronym} verwendet werden, wo die 
  Sortierung allerdings manuell durch den Anwender erfolgen muss.
\item[glossaries]
  Dies ist ein sehr mächtiges Paket zum Erstellen eines Glossars sowie 
  Abkürzungs- und Symbolverzeichnisses. Die mannigfaltige Anzahl an Optionen ist 
  für den Einstieg eventuell etwas abschreckend. Insbesondere wenn jedoch 
  Verzeichnisse für Abkürzungen \emph{und} Formelzeichen beziehungsweise Symbole 
  benötigt werden, sollte dieses Paket in Erwägung gezogen werden. Alternativ 
  kann für ein Symbolverzeichnis auch lediglich eine händisch gesetzte Tabelle 
  genutzt werden. Das häufig empfohlene Paket \Package{nomencl} bietet meiner 
  Meinung nach demgegenüber keinerlei Vorteile.
\end{packages}

\subsubsection{Grafiken und Abbildungen}
\index{Grafiken|?}
Grafiken für wissenschaftliche Arbeiten sollten als Vektorgrafiken erstellt 
werden, um Skalierbarkeit und hohe Druckqualität zu gewährleisten. Bestenfalls 
folgen diese auch dem Stil der dazugehörigen Arbeit.%
\footnote{%
  Aus anderen Arbeiten übernommene Grafiken sollten meiner Meinung nach für 
  qualitativ hochwertige Dokumente nicht direkt kopiert sondern nach der Vorlage 
  im entsprechenden Format neu erstellt und mit der Referenz auf die Quelle ins 
  Dokument eingebunden werden.
}
Für das Erstellen eigener Vektorgrafiken in \hologo{LaTeX}, die unter anderem 
die \hologo{LaTeX}"=Schriften und das Layout des Hauptdokumentes nutzen, gibt
es zwei mögliche Wege. Entweder, man \enquote{programmiert} die Grafiken, 
ähnlich wie auch das Dokument selber oder man nutzt Zeichenprogramme, die 
wiederum die Ausgabe oder das Weiterreichen von Text an \hologo{LaTeX} 
unterstützen. Für das Programmieren von Grafiken sollen hier die wichtigsten 
Pakete vorgestellt werden. Wie diese zu verwenden sind, ist den dazugehörigen 
Paketdokumentationen zu entnehmen.
\begin{packages}
\item[tikz]
  Dies ist ein sehr mächtiges Paket für das Programmieren von Vektorgrafiken und 
  höchstwahrscheinlich die erste Wahl bei der Verwendung von \hologo{pdfLaTeX}.
\item[pstricks]
  Das Paket \Package*{pstricks} stellt die zweite Variante zum Programmieren von 
  Grafiken dar. Mit diesem Paket hat man \emph{noch} mehr Möglichkeiten bei der 
  Erstellung eigener Grafiken, da man mit \Package*{pstricks} auf PostScript 
  zugreifen kann und einige der bereitgestellten Befehle davon rege Gebrauch 
  machen. Der daraus resultierende Nachteil ist, dass mit \Package*{pstricks} 
  die direkte Verwendung von \hologo{pdfLaTeX} nicht möglich ist.
  
  Die Grafiken aus den \Environment{pspicture}"=Umgebungen müssen deshalb erst 
  über den Pfad \Path{latex \textrightarrow{} dvips \textrightarrow{} ps2pdf}
  in PDF"~Dateien gewandelt werden. Diese lassen sich von   \hologo{pdfLaTeX} 
  anschließend als Abbildungen einbinden. Um dieses Vorgehen zu ermöglichen, 
  können folgende Pakete genutzt werden:
  \begin{packages}
  \item[pst-pdf]
    Dieses Paket stellt die prinzipiellen Methoden für den Export bereit. Die 
    einzelnen Kompilierungsschritte müssen manuell durchgeführt werden.
  \item[auto-pst-pdf]
    Das Paket automatisiert die Erzeugung der \Package*{pstricks}"=Grafiken. 
    Dafür muss \hologo{pdfLaTeX} mit der Option \Path{-\/-shell-escape}    
    aufgerufen werden.
  \item[pdftricks2]
    Ein weiteres Paket mit der gleichen Intention wie \Package*{auto-pst-pdf}, 
    allerdings anders implementiert. Auch hier ist für \hologo{pdfLaTeX} die 
    Option \Path{-\/-shell-escape} notwendig.
  \end{packages}
\item[standalone]
  Sollte \Package*{tikz} und/oder \Package*{pstricks} eingesetzt werden, kann 
  das Paket \Package*{standalone} genutzt werden, um die Grafiken einerseits als 
  eigenständiges Dokument übersetzen zu können und andererseits diese Grafiken 
  mit dem Hauptdokument zu kompilieren. Damit muss beim Erstellen oder Ändern 
  einer Grafik nicht immer das vollständige Hauptdokument mit kompiliert werden.
\end{packages}
%
Für die Varianten des Zeichnens einer Grafik mit einem Bildbearbeitungsprogramm, 
welches die Weiterverarbeitung durch \hologo{LaTeX} erlaubt, möchte ich auf die 
freien Programme \Application{LaTeXDraw} und \Application{Inkscape} verweisen. 
Insbesondere das zuletzt genannte Programm ist sehr empfehlenswert. Für die 
erstellten Grafiken kann man den Export für die Einbindung in \hologo{LaTeX} 
manuell durchführen. In \Package{svg-inkscape}%
\footnote{\url{http://www.ctan.org/pkg/svg-inkscape}}
ist beschrieben, wie sich dies automatisieren lässt. In \autoref{sec:tat:svg} 
wird eine daraus abgeleitete und verbesserte Variante vorgestellt.

\subsubsection{Gleitobjekte}
\index{Gleitobjekte|?}
\index{Tabellen}\index{Grafiken}
Es werden Pakete zur Beeinflussung von Aussehen, Beschriftung und Positionierung 
von Gleitobjekten vorgestellt. Unter \autoref{sec:tat:floats} sind außerdem 
Hinweise zur manuellen Manipulation der Gleitobjektplatzierung zu finden.
\begin{packages}
\item[caption]\index{Gleitobjekte!Beschriftung}
  Die \KOMAScript-Klassen bietet bereits einige Möglichkeiten zum Setzen der  
  Beschriftungen für Gleitobjekte. Dieses Paket ist daher meist nur in gewissen
  Ausnahmefällen für spezielle Anweisungen notwendig, allerdings auch bei der  
  Verwendung unbedenklich.
\item[subcaption]\index{Gleitobjekte!Beschriftung}
  Diese Paket kann zum einfachen Setzen von Unterabbildungen oder -tabellen mit 
  den entsprechenden Beschriftungen genutzt werden. Das dazu alternative Paket 
  \Package{subfig} sollte vermieden werden, da es nicht mehr gepflegt wird und 
  es mit diesem im Zusammenspiel mit anderen Paketen des Öfteren zu Problemen 
  kommt. Sollte der Funktionsumfang von \Package*{subcaption} nicht ausreichen, 
  kann anstelle dessen das Paket \Package*{floatrow} verwendet werden, welches 
  ähnliche Funktionalitäten wie \Package{subfig} bereitstellt.
\item[floatrow]\index{Gleitobjekte!Beschriftung}
  Mit diesem Paket können global wirksame Einstellungen und Formatierungen für 
  \emph{alle} Gleitobjekte eines Dokumentes vorgenommen werden. So kann unter 
  anderem die verwendete Schrift (\Macro*{floatsetup}\PParameter{font=\dots}) 
  innerhalb der Umgebungen \Environment*{float} und \Environment*{table} 
  eingestellt werden. Das typographisch richtige Setzen der Beschriftungen von 
  Abbildungen als Unterschriften 
  (\Macro*{floatsetup}\POParameter{figore}\PParameter{capposition=bottom})
  sowie Tabellen als Überschriften 
  (\Macro*{floatsetup}\POParameter{table}\PParameter{capposition=top})
  kann automatisch erzwungen werden~-- unabhängig von der Position des 
  Beschriftungsbefehls \Macro{caption} innerhalb der Gleitobjektumgebung.
  
  Wird das Verhalten so wie empfohlen mit dem \Package*{floatrow}-Paket 
  eingestellt, sollte außerdem die \KOMAScript-Option 
  \Option{captions}[tableheading] genutzt werden.
\item[placeins]\index{Gleitobjekte!Platzierung}
  Mit diesem Paket kann die Ausgabe von Gleitobjekten vor Kapiteln und wahlweise
  Unterkapiteln erzwungen werden.
\item[flafter]\index{Gleitobjekte!Platzierung}
  Dieses Paket erlaubt die frühestmögliche Platzierung von Gleitobjekten im 
  ausgegeben Dokument erst an der Stelle ihres Auftretens im Quelltext. Sie 
  werden dementsprechend nie vor ihrer Definition am Anfang der Seite erscheinen.
\end{packages}

\subsubsection{Listen und Tabellen}
\index{Listen|?}\index{Tabellen|?}
Für den Tabellensatz in \hologo{LaTeX} werden von Haus aus die Umgebungen 
\Environment*{tabbing} und \Environment*{tabular} beziehungsweise 
\Environment*{tabular*} bereitgestellt, welche in ihrer Funktionalität meist für 
einen qualitativ hochwertigen Tabellensatz nicht ausreichen. Es werden deshalb 
Pakete vorgestellt, die zusätzlich verwendet werden können. Ebenfalls können die 
Umgebungen für Auflistungen in \hologo{LaTeX} verbessert werden.
\begin{packages}
\item[enumitem]
  Das Paket \Package*{enumitem} erweitert die rudimentären Funktionalitäten der 
  \hologo{LaTeX}"=Standardlisten \Environment{itemize}, \Environment{enumerate}
  sowie \Environment{description} und ermöglicht die individuelle Anpassung 
  dieser durch die Bereitstellung vieler optionale Parameter nach dem
  Schlüssel"=Wert"=Prinzip. 
  
  Eine von mir sehr häufig genutzte Funktion ist beispielsweise die Entfernung 
  des zusätzlichen Abstand zwischen den einzelnen Einträgen einer Liste mit 
  \Macro*{setlist}\PParameter{noitemsep}.
\item[array]
  Dieses Paket ermöglicht die erweiterte Definition von Tabellenspalten sowie 
  das Erstellen neuer Spaltentypen mit \Macro*{newcolumntype}. Außerdem kann mit 
  \Macro*{extrarowheight} die Höhe der Zeilen einer Tabelle angepasst werden.
\item[multirow]
  Es wird der Befehl \Macro*{multirow} definiert, der~-- ähnlich zum Makro 
  \Macro*{multicolumn}~-- das Zusammenfassen von mehreren Zeilen in einer Spalte 
  ermöglicht.
\item[booktabs]
  Für einen guten Tabellensatz mit \hologo{LaTeX} gibt es bereits zahlreiche   
  Empfehlungen im Internet zu finden.%
  \footnote{\url{http://userpage.fu-berlin.de/latex/Materialien/tabsatz.pdf}}
  Zwei Regeln sollten dabei definitiv beachtet werden:
  %
  \begin{enumerate}[itemindent=0pt,labelwidth=*,labelsep=1em,label=\Roman*.]
  \makeatletter
  \item@packages keine vertikalen Linien
  \item@packages keine doppelten Linien
  \makeatother
  \end{enumerate}
  %
  Das Paket \Package*{booktabs} ist für den Satz von hochwertigen Tabellen~-- 
  zusammen mit der deutschsprachigen Dokumentation \Package*{booktabs-de}~-- 
  eine große Hilfe und stellt neue Befehle für horizontale Linien bereit.
\item[widetable]
  Mit der Standard"=\hologo{LaTeX}"=Umgebung \Environment*{tabular*} kann eine 
  Tabelle mit einer definierten Breite gesetzt werden. Dieses Paket stellt die 
  Umgebung \Environment*{widetable} zur Verfügung, die als Alternative genutzt 
  werden kann und eine symmetrische Tabelle erzeugt.
\item[tabularx]
  Auch mit diesem Paket kann die Breite eine Tabelle spezifiziert werden. Dafür 
  wird der neue Spaltentyp \PValue{X} definiert, welcher als Argument der 
  \Environment*{tabularx}"=Umgebung beliebig häufig angegeben werden kann
  (\Macro*{begin}\PParameter{tabularx}\Parameter{Breite}\Parameter{Spalten}). 
  \PValue{X}"~Spalten entsprechen denen vom Typ~\PValue{p}\OParameter{Breite}
  (äquivalent zu \Macro*{parbox}\Parameter{Breite}), die Breite wird allerdings 
  aus der gegebenen Tabellenbreite und dem benötigten Platz der Standardspalten 
  automatisch berechnet.
\item[longtable]
  Sollen mehrseitige Tabellen mit Seitenumbruch erstellt werden, ist dieses 
  Paket das Mittel der ersten Wahl. Für die Kombination mehrseitiger Tabellen mit
  einer \Environment*{tabularx}"=Umgebung können die Pakete \Package{ltablex} 
  oder besser noch \Package*{ltxtable} verwendet werden.
\item[ltxtable]
  Wie bereits erwähnt sollte dieses Paket für mehrseitige Tabellen, die mit der 
  Umgebung \Environment*{tabularx} erstellt wurden, verwendet werden. Alternativ 
  dazu kann auch \Package*{tabu} verwendet werden.
\item[tabu]
  Dies ist ein relativ neues Paket, welches versucht, viele der zuvor genannten 
  Funktionalitäten zu implementieren und weitere bereitzustellen. Dafür werden 
  die Umgebungen \Environment*{tabu} und \Environment*{longtabu} definiert. Es 
  kann als Alternative für \Package*{tabularx} verwendet werden und insbesondere 
  als Ersatz für das Paket \Package*{ltxtable} empfehlenswert.
\item[tabularborder]
  \hologo{LaTeX} setzt bei Tabellen zwischen Spalten automatisch einen kleinen 
  horizontalen Abstand (\Length{tabcolsep}), besser gesagt jeweils vor und nach 
  einer Spalte. Dies geschieht auch \emph{vor} der ersten und \emph{nach} der 
  letzten Spalte. Dieser zusätzliche Platz an den äußeren Rändern kann störend 
  wirken, insbesondere wenn die Tabelle über die komplette Textbreite gesetzt 
  wird. Mit dem Paket \Package*{tabularborder} kann dieser Platz automatisch 
  entfernt werden.
  
  Dies funktioniert allerdings nur mit der \Environment*{tabular}"=Umgebung. Die 
  Tabellen-Umgebungen aus \Package*{tabu} und \Package*{tabularx} werden nicht 
  unterstützt. Bei diesen muss der Abstand vor der ersten und nach der letzten 
  Spalte weiterhin manuell mit \PValue{@}\PParameter{} entfernt werden. Ein 
  kurzes Beispiel dazu ist in \autoref{sec:tat:table} zu finden.
\end{packages}


\subsubsection{Mathematiksatz}
\index{Mathematiksatz}
Dies sind Pakete, die Umgebungen und Befehle für den Mathematiksatz sowie das 
Setzen von Einheiten und Zahlen im Allgemeinen anbieten. Außerdem sei auf 
\autoref{sec:mathtype} verwiesen.
\begin{packages}
  \item[mathtools]
    Dieses Paket stellt für das De-facto-Standard-Paket \Package{amsmath} für 
    Mathematikumgebungen Bugfixes zur Verfügung und erweitert dieses.
  \item[sansmath]
    Sollten die normalen \hologo{LaTeX}-Schriften Computer~Modern verwendet 
    werden, kann man dieses Paket zum serifenlosen Setzen mathematischer 
    Ausdrücke nutzen. Für die \TUDScript-Hauptklassen sei hierzu auf die Option
    \Option{sansmath} verwiesen.
  \item[sfmath]
    Diese Paket verfolgt ein ähnliches Ziel, kann jedoch im Gegensatz zu 
    \Package*{sansmath} nicht nur für Computer~Modern sondern mit der     
    entsprechenden Option auch für Latin~Modern, Helvetica und     
    Computer~Modern~Bright verwendet werden.
  \item[mathastext]
     Mit dem Paket wird das Ziel verfolgt, aus der genutzten Schrift für den 
     Fließtext alle notwendigen Zeichen für den Mathematiksatz zu extrahieren.
  \item[bm]
    Das Paket bietet mit \Macro*{bm} eine Alternative zu \Macro*{boldsymbol} im 
    Mathematiksatz.\footnote{\url{http://tex.stackexchange.com/questions/3238}}
\end{packages}
Die korrekte Formatierung von Zahlen ist immer wieder ein Problem bei der 
Verwendung von \hologo{LaTeX}. Insbesondere, wenn in einem deutschsprachigen 
Dokument Daten im typischen englischsprachigen Format verwendet werden, kommt es 
zu Problem. Dafür wird im \KOMAScript{}-Bundle das Paket \Package{mathswap} 
bereitgestellt. Dennoch gibt es zu diesem auch Alternativen.
\begin{packages}\index{Trennzeichen}
  \item[icomma]
    Wird im Mathematikmodus nach dem Komma ein Leerzeichen gesetzt, wird dies 
    bei der Ausgabe beachtet. Der Verfasser muss sich demzufolge jederzeit 
    selbst um die typographisch korrekte Ausgabe kümmern.
  \item[ziffer]
    Für deutschsprachige Dokumente wird das Komma als Dezimaltrennzeichen 
    zwischen zwei Ziffern definiert. Folgt dem Komma keine Ziffer, wird 
    jederzeit der obligatorische Freiraum gesetzt, was meiner Meinung nach 
    besser als das Verhalten von \Package*{icomma} ist.
  \item[ionumbers]
    Dieses Paket ist mir tatsächlich erst bei der Arbeit an \Package*{mathswap} 
    bekannt geworden. Es bietet mehr Funktionalitäten und kann als Alternative 
    dazu betrachtet werden.
\end{packages}
Für das typographisch korrekte Setzen von Einheiten~-- ein halbes Leerzeichen 
zwischen Zahl und \emph{aufrecht} gesetzter Einheit~-- gibt es zwei gut nutzbare 
Pakete.
\begin{packages}\index{Einheiten}
\item[units]
  Dies ist ein einfaches und sehr zweckdienliches Paket zum Setzen von Einheiten 
  und für die meisten Anforderungen völlig ausreichend.
\item[siunitx]
  Dieses Paket ist in seinem Umfang im Vergleich deutlich erweitert. Mir hat 
  sich persönlich noch nicht erschlossen, was genau die daraus resultierenden 
  Vorteile sind. Damit das Paket in deutschsprachigen Dokumenten alle Ziffern 
  richtig setzt, muss zumindest die Lokalisierung angegeben werden. Mehr dazu in 
  \autoref{sec:tat:siunitx}.
\end{packages}


\subsubsection{Typographie und Layout}
\index{Typographie}
\begin{packages}
\item[setspace]\index{Zeilenabstand}
  Die Vergrößerung des Zeilenabstandes wird:
  \begin{enumerate}[itemindent=0pt,labelwidth=*,labelsep=1em,label=\Roman*.]
  \makeatletter
  \item@packages viel zu häufig und völlig unnötig gefordert sowie
  \item@packages dann auch noch zu groß gewählt.
  \makeatother
  \end{enumerate}
  Die Forderung nach Erhöhung des Zeilenabstandes~-- in der Typographie als 
  Durchschuss bezeichnet~-- kommt noch aus den Zeiten der Textverarbeitung mit 
  der Schreibmaschine. Ein einzeiliger Zeilenabstand bedeutete hier, dass die 
  Unterlängen der oberen Zeile genau auf der Höhe der Oberlängen der folgenden 
  Zeile lagen. Ein anderthalbzeiliger Zeilenabstand erzielte hier also einen 
  akzeptablen Durchschuss. Eine Erhöhung des Durchschusses bei der Verwendung 
  von \hologo{LaTeX} ist eigentlich nicht notwendig. Sinnvoll ist dies nur, wenn 
  im Fließtext serifenlose Schriften zum Einsatz kommen, um die damit verbundene 
  schlechte Lesbarkeit etwas zu verbessern.
  
  Ist die Erhöhung des Durchschusses wirklich notwendig, sollte das Paket 
  \Package*{setspace} verwendet werden. Dieses stellt den Befehl 
  \Macro*{setstretch}\Parameter{Faktor} zur Verfügung, mit dem der Durchschuss 
  beziehungsweise Zeilenabstand angepasst werden kann. Der Wert des Faktors 
  ist standardmäßig auf~1 gestellt und sollte maximal bis~1.25 vergrößert 
  werden. Der Befehl \Macro*{onehalfspacing} aus diesem Paket setzt diesen Wert 
  auf eben genau~1.25. Allerdings ist hier anzumerken, dass die Vergrößerung des 
  Zeilenabstandes~-- so wie ich es mir angelesen habe~-- aus der Sicht eines 
  Typographen keine Spielerei ist sondern vielmehr allein der Lesbarkeit des 
  Textes dient und möglichst gering ausfallen sollte.
  
  Ziel ist es, beim Lesen nach dem Beenden der aktuellen Zeile das Auffinden der 
  neuen Zeile zu vereinfachen. Bei Serifen ist dies durch die Betonung der 
  Grundlinie sehr gut möglich. Bei serifenlosen Schriften~-- wie der hier 
  verwendeten \Univers~-- ist dies schwieriger und ein erweiterter Abstand der 
  Zeilen kann dabei durchaus hilfreich sein, jedoch sollte nicht nach dem Motto 
  \enquote{viel hilft viel} verfahren werden. Bei diesem Dokument wurde als 
  Faktor für den Zeilenabstand \Macro*{setstretch}\PParameter{1.1} gewählt. Nach 
  einer Einstellung des Zeilenabstandes sollte der Satzspiegel unbedingt mit 
  \Macro{recalctypearea} neu berechnet werden. Siehe dazu auch 
  \autoref{sec:tat:problemA} und \ref{sec:tat:problemB}.
\item[csquotes]\index{Zitate}
  Das Paket stellt unter anderem den Befehl \Macro{enquote}\Parameter{Zitat} 
  zur Verfügung, welcher Anführungszeichen in Abhängigkeit der gewählten   
  Sprache setzt. Außerdem werden weitere Kommandos und Optionen für die   
  spezifischen Anforderungen des Zitierens bei wissenschaftlichen Arbeiten   
  angeboten. Außerdem wird es durch \Package{biblatex} unterstützt und sollte 
  zumindest bei dessen Verwendung geladen werden.
\item[xspace]\index{Befehle!Deklaration}
  Mit \Package*{xspace} kann bei der Definition eigener Makros der Befehl 
  \Macro*{xspace} genutzt werden. Dieser setzt ein gegebenenfalls notwendiges 
  Leerzeichen automatisch. In \autoref{sec:tat:xspace} ist die Definition eines 
  solchen Befehls exemplarisch ausgeführt.
\item[xpunctuate]\index{Befehle!Deklaration}
  Diese Paket erweitert die Funktionalität von \Package*{xspace} nochmals.
\item[ellipsis]\index{Befehle!Deklaration}
  In \hologo{LaTeX} folgten den Befehlen für Auslassungspunkte (\Macro*{dots} 
  und \Macro*{textellipsis}) \emph{immer} ein Leerzeichen. Dies kann unter 
  Umständen unerwünscht sein. Mit dem Paket \Package*{ellipsis} wird das 
  folgende Leerzeichen~-- im Gegensatz zum Standardverhalten~-- nur gesetzt, 
  wenn ein Satzzeichen und kein Buchstabe folgt.
\makeatletter\item@packages[\Application{DeLig}]\makeatother
  \index{Typographie}\index{Ligaturen}
  Hierbei handelt es sich um ein Java-Script, welches anhand eines Wörterbuches 
  versucht, falsche Ligaturen innerhalb eines Dokumentes automatisiert zu 
  entfernen. Wird \Univers verwendet ist dies jedoch nicht notwendig. Diese 
  liefert keinerlei Ligaturen, die insbesondere in deutschen Texten für einen 
  guten Satz manuell aufgelöst werden müssten.%
  \footnote{%
    Das sind ff, fi, fl, ffi, und ffl bei den \hologo{LaTeX}"=Standardschriften.
  }
\item[selnolig]\index{Typographie}\index{Ligaturen}
  Kommt \hologo{LuaLaTeX} zum Einsatz kann alternativ zu \Application*{DeLig} 
  auch \Package*{selnolig} verwendet werden, um falsche Ligaturen zu vermeiden.
\item[balance]\index{Zweispaltensatz}
  Dieses Paket stellt eine Methode zur Verfügung, um im Zweispaltensatz auf der 
  letzten Seite eines Dokumentes die Höhe der Spalten auszugleichen. Alternativ 
  dazu kann auch \Package{multicol} verwendet werden.
\end{packages}

\subsubsection{Schriften, Sonderzeichen und Rechtschreibung}
\begin{packages}
\item[lmodern]\index{Schriftart}
  Soll mit den klassischen \hologo{LaTeX}"=Standardschriften anstelle der 
  Schriften des \CDs gearbeitet werden, empfiehlt sich die Verwendung des 
  Paketes \Package*{lmodern}. Dieses verbessert unter anderem die Darstellung 
  der Computer~Modern am Bildschirm. Hierfür muss die Verwendung der \Univers 
  mit der Option \Option{cdfont}[false] deaktiviert werden.
\item[cfr-lm]\index{Schriftart}
  Dieses experimentelle Paket liefert weitere Schriftschnitte für das Paket 
  \Package{lmodern}.
\item[libertine]\index{Schriftart}
  Das Paket stellt die Schriften Linux~Libertine und Linux~Biolinum für die 
  Verwendung mit \hologo{LaTeX} zur Verfügung.
  \begin{packages}
    \item[libgreek]
      Es werden griechische Buchstaben für Linux~Libertine bereitgestellt.
    \item[newtxmath]
      Das Paket aus dem \Package*{newtx}-Bundle erlaubt die Verwendung der 
      Linux~Libertine im Mathematikmodus. Dafüru muss es mit
      \Macro*{usepackage}\POParameter{libertine}\PParameter{newtxmath} geladen 
      werden.
  \end{packages}
\item[relsize]\index{Schrift!Größe}
  Mit diesem Paket wird es möglich, die Größe eines Textes relativ zur aktuell 
  gewählten Schriftgröße zu setzen.
\item[textcomp]\index{Sonderzeichen}
  Es werden verschiedene zusätzliche Symbole wie beispielsweise das Promille- 
  oder Eurozeichen sowie Pfeile im Text zur Verfügung gestellt.
\item[spelling]
  Wird \hologo{LuaLaTeX} als Prozessor verwendet, können mit diesem Paket 
  mögliche Rechtschreibfehler im ausgegebenen PDF"~Dokument hervorgehoben werden.
\item[lua-check-hyphen]
  Mit diesem Paket lassen sich bei der Verwendung \hologo{LuaLaTeX} Trennstellen 
  am  Zeilenende zur Prüfung markieren.
\end{packages}

\subsubsection{Die kleinen und großen Helferlein\dots}
Hier taucht alles auf, was sich nicht eignete, in die vorherigen Kategorien 
eingeordnet zu werden.
\begin{packages}
\item[calc]\index{Berechnungen}
  Normalerweise können Berechnungen nur mit Low-Level-\hologo{TeX}-Primitiven im 
  Dokument durchgeführt werden. Diese Paket stellt für Rechenoperationen eine 
  einfachere Syntax bereit. Des Weiteren werden Befehle zur Berechnung der Höhe
  und Breite bestimmter Textauszüge definiert.
\item[bookmark]
  Dieses Paket verbessert und erweitert die von \Package{hyperref} angebotenen 
  Möglichkeiten zur Erstellung von Lesezeichen~-- auch Outline"=Einträge~-- im 
  PDF-Dokument.
\item[varioref]
  Mit diesem Paket lassen sich sehr gute Verweise auf Seiten erzeugen. 
  Insbesondere, wenn der Verweis auf die aktuelle, die vorhergehende oder 
  nachfolgende sowie im zweiseitigen Satz auf die gegenüberliegende Seite 
  erfolgt, werden passende Textbausteine für diesen verwendet.
\item[listings]\index{Quelltexte einbinden}
  Dieses Paket eignet sich hervorragend zur Quelltextdokumentation in 
  \hologo{LaTeX}. Es bietet die Möglichkeit, externe Quelldateien einzulesen und 
  darzustellen. Auch eine Syntaxhervorhebung in Abhängigkeit der verwendeten 
  Programmiersprache ist für den Quelltext darstellbar. Zusätzlich lässt sich 
  ein Verzeichnis mit allen eingebundenen und im Dokument direkt angegebenen 
  Quelltextauszügen kann erstellt werden.
\item[chngcntr]\index{Zählermanipulation}
  Das Paket erlaubt die Manipulation aller möglichen, bereits definierten 
  \hologo{LaTeX}-Zähler. Es können Zähler so umdefiniert werden, dass sie bei 
  der Änderung eines anderen Zählers automatisch zurückgesetzt werden oder eben 
  nicht. Ein kleines Beispiel dazu ist in \autoref{sec:tat:counter} zu finden.
\item[coseoul]
  Mit diesem Paket kann man die Struktur der Gliederung relativ angeben. Es wird 
  keine absolute Gliederungsebene (\Macro*{chapter}, \Macro*{section}) angegeben 
  sondern die Relation zwischen vorheriger und aktueller Ebene 
  (\Macro*{levelup}, \Macro*{levelstay}, \Macro*{leveldown}).
\item[dprogress]\index{Debugging}
  Das Paket schreibt die Gliederung des Dokumentes in die Logdatei. Dies kann
  im Fehlerfall beim Auffinden des Problems im Dokument helfen. Allerdings 
  werden dafür die Gliederungsebenen so umdefiniert, dass diese keine optionalen 
  Argumente mehr unterstützen,was jedoch für die \TUDScript-Klassen von 
  essentieller Bedeutung ist. Zu Debugging-Zwecken kann es aber trotzdem 
  sporadisch eingesetzt werden.
\item[xparse]\index{Befehle!Deklaration}
  Diese Paket bietet für die Deklaration eigener Befehlen und Umgebungen einen 
  alternativen Ansatz zu \Macro*{newcommand} und \Macro*{newenvironment} sowie 
  deren Derivaten. Mit \Package*{xparse} wird es möglich, (fast) beliebig viele 
  optionale Argumente an willkürlichen Stellen innerhalb des Befehlskonstruktes 
  zu definieren. Auch die Verwendung anderer Zeichen als eckige Klammern für die 
  Spezifizierung eines optionalen Argumentes ist möglich.
\item[filemod]
  Wird entweder \hologo{pdfLaTeX} oder \hologo{LuaLaTeX} als Prozessor 
  eingesetzt, können mit diesem Paket das Änderungsdatum zweier Dateien 
  miteinander verglichen und in Abhängigkeit davon definierbare Aktionen 
  ausgeführt werden. 
\item[marginnote]
  Randnotizen, welche mit \Macro*{marginpar} erzeugt werden, sind spezielle 
  Gleitobjekte in \hologo{LaTeX}. Dies kann dazu führen, dass eine Notiz am 
  Blattrand nicht direkt da gesetzt wird, wo diese intendiert war. Dieses Paket 
  stellt den Befehl \Macro*{marginnote} für nicht"~gleitende Randnotizen zur 
  Verfügung. Alternativ dazu kann man auch \Package*{mparhack} verwenden.
\item[todonotes]
  Mit \Package*{todonotes} können noch offene Aufgaben in unterschiedlicher 
  Formatierung am Blattrand ausgegeben werden. Es lässt sich daraus auch eine  
  Liste aller offenen Punkte erzeugen.
\item[etex]
  Das Paket kann genutzt werden, falls die standardmäßig maximale Anzahl der 
  \hologo{LaTeX}-Register für Längen, Zähler etc. überschritten wurde.
\end{packages}

\subsubsection{Bugfixes}
\begin{packages}
\item[scrhack]
  Das Paket behebt Kompatibilitätsprobleme der \KOMAScript-Klassen mit den 
  Paketen \Package{hyperref}, \Package{float}, \Package{floatrow} und
  \Package{listings}. Es ist durchaus empfehlenswert, jedoch sollte man 
  unbedingt die Dokumentation beachten.
\item[fixltx2e]
  Dieses Paket enthält kleinere Bugfixes für \hologo{LaTeXe}. Da diese 
  Änderungen zu Inkompatibilitäten mit früheren Versionen führen würde, wurden 
  diese nicht direkt in den \hologo{LaTeXe}-Kernel eingepflegt.
\item[mparhack]
  Zur Behebung falsch gesetzter Randnotizen wird ein Bugfix für 
  \Macro*{marginpar} bereitgestellt. Alternativ dazu kann man auch 
  \Package*{marginnote} verwenden.  
\end{packages}


\section{Praktische Tipps \& Tricks}\label{sec:tat}
\subsection{Leer- und Satzzeichen am Ende benutzerdefinierter Befehle}%
\label{sec:tat:xspace}
\index{Typographie}
Normalerweise \enquote{schluckt} \hologo{LaTeX} die Leerzeichen nach einem Makro 
ohne Argumente. Dies ist jedoch nicht immer~-- eigentlich in den seltensten 
Fällen~-- erwünscht. Für dieses Handbuch ist beispielsweise der Befehl 
\Macro*{TUD} definiert worden, um \enquote{\TUD{}} nicht ständig ausschreiben zu 
müssen. Um sich bei der Verwendung des Befehl innerhalb eines Satzes sich für 
den Erhalt eines folgenden Leerzeichens das Setzen der geschweiften Klammer nach 
dem Befehl zu sparen (\Macro*{TUD}\PParameter{}), kann \Macro*{xspace} aus dem 
Paket \Package{xspace} genutzt werden. Damit wird ein folgendes Leerzeichen 
erhalten. Der Befehl \Macro*{TUD} ist wie folgt definiert:
\begin{code}
\newcommand*\TUD{Technische Universit\"at Dresden\xspace}
\end{code}
Das Paket \Package{xpunctuate} erweitert die Funktionalität nochmals. Damit 
können auch Abkürzungen so definiert werden, dass ein versehentlicher Punkt 
ignoriert wird:
\begin{code}
\newcommand*\zB{z.\,B\xperiod}
\end{code}

\subsection{Platzierung von Gleitobjekten}\label{sec:tat:floats}
\index{Gleitobjekte|?}
Die standardmäßige Platzierung von Gleitobjekten wird durch die im Folgenden 
aufgezählten Befehle beeinflusst. Diese können mit 
\Macro*{renewcommand*}\Parameter{Befehl}\Parameter{Wert} geändert werden.
%
\begin{Declaration}{\Macro{floatpagefraction}}[0\floatpagefraction]
\begin{Declaration}{\Macro{dblfloatpagefraction}}[0\dblfloatpagefraction]
\printdeclarationlist*
%
Der Wert gibt die relative Größe eines Gleitobjektes bezogen auf die Texthöhe 
(\Macro*{textheight}) an, die mindestens erreicht sein muss, damit für dieses 
gegebenenfalls vor dem Beginn eines neuen Kapitels eine separate Seite erzeugt 
wird. Dabei wird einspaltiges (\Macro*{floatpagefraction}) und zweispaltiges 
(\Macro*{dblfloatpagefraction}) Layout unterschieden. Der Wert für beide Befehle 
sollte im Bereich von \PValue{0.5\dots 0.8} liegen.
\end{Declaration}
\end{Declaration}

\begin{Declaration}{\Macro{topfraction}}[0\topfraction]
\begin{Declaration}{\Macro{dbltopfraction}}[0\dbltopfraction]
\printdeclarationlist*
%
Diese Werte geben den maximalen Seitenanteil für Gleitobjekte, die am oberen 
Seitenrand platziert werden, für einspaltiges und zweispaltiges Layout an. Er 
sollte im Bereich von \PValue{0.5\dots 0.8} liegen und größer als 
\Macro*{floatpagefraction} beziehungsweise \Macro*{dblfloatpagefraction} sein.
\end{Declaration}
\end{Declaration}

\begin{Declaration}{\Macro{bottomfraction}}[0\bottomfraction]
\printdeclarationlist*
%
Dies ist der maximale Seitenanteil für Gleitobjekte, die am unteren Seitenrand 
platziert werden. Er sollte zwischen \PValue{0.2} und \PValue{0.5} betragen.
\end{Declaration}

\begin{Declaration}{\Macro{textfraction}}[0\textfraction]
\printdeclarationlist*
%
Dies ist der Mindestanteil an Text, der auf einer Seite mit Gleitobjekten 
vorhanden sein muss, wenn diese nicht auf einer eigenen Seite ausgegeben werden. 
Er sollte im Bereich von \PValue{0.1\dots 0.3} liegen.
\end{Declaration}

\begin{Declaration}{\Counter{totalnumber}}[\arabic{totalnumber}]
\begin{Declaration}{\Counter{topnumber}}[\arabic{topnumber}]
\begin{Declaration}{\Counter{dbltopnumber}}[\arabic{dbltopnumber}]
\begin{Declaration}{\Counter{bottomnumber}}[\arabic{bottomnumber}]
\printdeclarationlist*
%
Außerdem gibt es noch Zähler, welche die maximale Anzahl an Gleitobjekten pro 
Seite insgesamt (\Counter*{totalnumber}), am oberen (\Counter*{topnumber}) und 
am unteren Rand der Seite (\Counter*{bottomnumber}) sowie im Zweispaltensatz 
beide Spalten überspannend (\Counter*{dbltopnumber}) festlegen. Die Werte können 
mit \Macro*{setcounter}\Parameter{Zähler}\Parameter{Wert} geändert werden.
\end{Declaration}
\end{Declaration}
\end{Declaration}
\end{Declaration}

\begin{Declaration}{\Length{@fptop}}
\begin{Declaration}{\Length{@fpsep}}
\begin{Declaration}{\Length{@fpbot}}
\begin{Declaration}{\Length{@dblfptop}}
\begin{Declaration}{\Length{@dblfpsep}}
\begin{Declaration}{\Length{@dblfpbot}}
\printdeclarationlist*
%
Sind vor Beginn eines Kapitels noch Gleitobjekte verblieben, so werden diese 
durch \hologo{LaTeX} normalerweise auf einer separaten vertikal zentriert Seite 
ausgegeben. Dabei bestimmen diese Längen jeweils den Abstand vor dem ersten 
Gleitobjekt zum oberen Seitenrand (\Length*{@fptop}, \Length*{@dblfptop}), 
zwischen den einzelnen Objekten (\Length*{@fpsep}, \Length*{@dblfpsep}) sowie 
zum unteren Seitenrand (\Length*{@fpbot}, \Length*{@dblfpbot}). Soll dies nicht 
geschehen, können Sie die Längen ändern.
\end{Declaration}
\end{Declaration}
\end{Declaration}
\end{Declaration}
\end{Declaration}
\end{Declaration}

\begin{Example}
Alle Gleitobjekte auf einer dafür speziell gesetzten Seite sollen direkt zu 
Beginn dieser ausgegeben werden. In der Dokumentpräambel kann man dafür 
schreiben:
\begin{code}
\makeatletter
\setlength{\@fptop}{0pt}
\setlength{\@dblfptop}{0pt} % twocolumn
\makeatother
\end{code}
\end{Example}


\subsection{Automatisiertes Einbinden von \Application{Inkscape}-Grafiken }
\label{sec:tat:svg}
\index{Grafiken}
In \Package{svg-inkscape}\footnote{\url{http://www.ctan.org/pkg/svg-inkscape}}
wir das automatisierte Einbinden von \Application{Inkscape}-Grafiken in ein 
\hologo{LaTeX}"=Dokument erläutert. Hier wird ein darauf aufbauender Ansatz 
vorgestellt. Die mit \Application{Inkscape} erstellte Grafik soll automatisch 
kompiliert und eingebunden werden. Dies soll nicht bei jeder Kompilierung des 
Hauptdokumentes erfolgen, sondern lediglich, wenn die originale Bilddatei 
aktualisiert wurde. Diese Funktionalität stellt das Paket \Package{filemod} 
bereit. Die automatisierte Übersetzung einer Grafik im SVG"~Format in eine   
PDF"~Datei und die anschließende Einbindung dieser in das Dokument ist mit der 
Definition des Befehls \Macro{includesvg}\OParameter{Breite}\Parameter{Datei} 
in der Präambel des Dokumentes wie folgt möglich:
\begin{code}[escapechar=§]
\usepackage{filemod}
\newcommand*{\includesvg}[2][\textwidth]{%
  \def\svgwidth{#1}
  \filemodCmp{#2.pdf}{#2.svg}{}{%
    \immediate\write18{%
      inkscape -z -D --file=#2.svg --export-pdf=#2.pdf --export-latex
    }%
  }%
  \input{#2.pdf_tex}%
}
\end{code}
%
Dabei wird mit \Macro*{immediate}\Macro*{write18}\Parameter{externer Aufruf} 
das zwischenzeitliche Ausführen eines externen Programms beim Durchlauf von 
\hologo{pdfLaTeX}~-- in diesem Fall von \File{inkscape.exe}~-- möglich. Damit 
der externe Aufruf auch tatsächlich durchgeführt wird, muss \hologo{pdfLaTeX} 
mit der Option \Path{-\/-shell-escape} ausgeführt werden. Außerdem muss der Pfad 
zur Datei \File{inkscape.exe} dem System bekannt sein.%
\footnote{%
  Genauer gesagt, muss der Pfad zu \File{inkscape.exe} in der 
  \texttt{PATH}-Variable von Windows enthalten sein.
}
Bei der Verwendung des Befehls \Macro{includesvg} \emph{muss} der Dateiname 
ohne Endung angegeben werden. Die einzubindende SVG"~Datei sollte sich hierbei 
im gleichen Pfad wie das Hauptdokument befinden. Ist die SVG"~Datei in einem 
Unterordner relativ zum Pfad des Hauptdokumentes, kann dieser einfach mit 
\Macro{includesvg}\PParameter{\PName{Ordner}/\PName{Datei}} im Argument 
angegeben werden.

\subsection{Setzen von Einheiten mit \Package{siunitx}}\label{sec:tat:siunitx}
\index{Einheiten}
Wenn \Package*{siunitx} in einem deutschsprachigen Dokument genutzt soll
werden, muss zumindest mit \Macro*{sisetup}\PParameter{locale = DE} die richtige 
Lokalisierung angegeben werden. Sollen auch die Zahlen richtig formatiert sein, 
müssen weitere Einstellungen vorgenommen werden. Die meiner Meinung nach besten 
sind die folgenden.
\begin{code}
\sisetup{%
  locale = DE,%
  input-decimal-markers={,},%
  input-ignore={.},%
  group-separator={\,},%
  group-minimum-digits=3%
}
\end{code}
Das Komma kommt als Dezimaltrennzeichen zum Einsatz. Des Weiteren werden Punkte 
innerhalb der Zahlen ignoriert und eine Gruppierung von jeweils drei Ziffern 
vorgenommen. Alternativ zu diesem Paket kann übrigens auch \Package{units} 
verwendet werden.


\subsection{Einrückung der ersten und letzten Tabellenspalte verhindern}%
\label{sec:tat:table}
\index{Tabellen}
Bei Tabellen wird vor und nach Spalte durch \hologo{LaTeX} ein horizontaler 
Abstand von \Length{tabcolsep} gesetzt. Dies geschieht auch \emph{vor} der 
ersten und \emph{nach} der letzten Spalte. Diese Einrückung an den äußeren 
Rändern kann insbesondere bei Tabellen, welche die komplette Seitenbreite 
überspannen, stören.

Bei der Deklaration einer Tabelle kann mit \PValue{@}\PParameter{\dots} vor und 
nach dem Spaltentyp angegeben werden, was anstelle von \Length{tabcolsep} vor 
beziehungsweise nach der eigentlichen Spalte eingeführt werden soll. Dies kann 
man sich wür das Entfernen der Einrückungen zu Nutze machen.

\begin{Example}
Eine Tabelle mit zwei Spalten, wobei bei einer die Breite automatisch berechnet 
wird, soll über die komplette Textbreite gesetzt werden. Dabei soll der Rand vor 
der ersten und nach der letzten entfernt werden.
\begin{code}[escapechar=§]
\begin{tabularx}{\textwidth}{@{}lX@{}}
§\dots§ & §\dots§ \tabularnewline
§\dots§
\end{tabularx}
\end{code}
\end{Example}

\subsection{Unterbinden des Zurücksetzens von Fußnoten nach Kapiteln}%
\label{sec:tat:counter}
\index{Fußnoten}
Oft taucht die Frage auf, wie man auch über Kapitel fortlaufende Fußnoten 
erhalten kann. Dies ist sehr einfach mit dem Paket \Package{chngcntr} möglich. 
Nach dem Laden des Paketes, kann das Rücksetzen des Zählers nach einem Kapitel 
mit \Macro*{counterwithout*}\PParameter{footnote}\PParameter{chapter} 
deaktiviert werden.

\subsection{Unterdrückung des Einzuges eines Absatzes}
\index{Absatzauszeichnung}
Verwendet man~-- wie es aus typographischer Sicht zumeist sinnvoll ist~-- 
Einzüge und keine vertikalen Abstände zur Auszeichnung von Absätzen im Dokument
(\Option{parskip}[false]), kann es vorkommen, dass ein bestimmter Absatz~-- 
beispielsweise der nach einer gewissen Umgebung folgende~-- ungewollt eingerückt 
ist. Dies kann sehr einfach behoben werden, indem direkt zu Beginn des Absatzes 
das Makro \Macro{noindent} verwendet wird.

\subsection{Warnung über eine zu geringe Höhe der Kopfzeile}
\label{sec:tat:problemA}
Sollten Sie das Paket \Package{scrpage2} zusammen mit \Package{setspace} 
verwenden, kann es passieren, dass Sie die folgende Warnung erhalten:
\begin{quote}
\begin{verbatim}
scrpage2 Warning: seems you are using a very small headheight.
\end{verbatim}
\end{quote}
Dies liegt an dem durch den vergrößerten Zeilenabstand erhöhten Bedarf für die
Kopfzeile. In diesem Fall muss die Höhe der Kopfzeile angepasst werden um die 
Warnung zu beseitigen. Dafür können Sie \Option{headlines}[\Parameter{Faktor}] 
beziehungsweise \Option{headheight}[\Parameter{Länge}] als Klassenoption 
angeben~-- wobei standardmäßig \Option{headlines}[1.25] gesetzt ist~-- oder Sie 
rufen nach der Änderung des Zeilenabstandes einfach \Macro{recalctypearea} auf. 
Ab der \KOMAScript{}-Verion~v3.12 wäre es noch besser, auf \Package{scrpage2} 
komplett zu verzichten und stattdessen \Package{scrlayer-scrpage} zu verwenden.

\subsection{Zeilenabstände in Überschriften}
\label{sec:tat:problemB}
Mit dem Paket \Package{setspace} kann der Zeilenabstand beziehungsweise der 
Durchschuss innerhalb des Dokumentes geändert werden. Sollte dieser erhöht 
worden sein, können die Abstände bei mehrzeiligen Überschriften als zu groß 
erscheinen. Um dies zu korrigieren kann mit dem Befehl \Macro*{addtokomafont}%
\PParameter{disposition}\PParameter{\Macro*{setstretch}\PParameter{1}} der 
Zeilenabstand aller Überschriften auf einzeilig zurückgeschaltet werden. Soll 
dies nur für eine bestimmte Gliederungsebene erfolgen, so ist 
\PParameter{disposition} durch das entsprechende Schriftelement zu ersetzen.

\subsection{Änderung des Papierformates}
\index{Papierformat}
Es kann vorkommen, dass man innerhalb eines Dokumentes kurzzeitig das 
Papierformat ändern möchte, um beispielsweise eine Konstruktionsskizze in der 
digitalen PDF"~Datei einzubinden. Dabei ist es sowohl möglich, lediglich die 
Ausrichtung mit \Option*{paper}[landscape] in ein Querformat zu ändern, als auch 
die Größe des Papierformates selber.
\begin{Example}
Ein Dokument im A4"~Format soll kurzzeitig auf ein A3"=Querformat geändert 
werden. Das folgende Minimalbeispiel zeigt, wie das Papierformat mit den Mitteln 
von \KOMAScript{} geändert werden kann.
\begin{code}
\documentclass[paper=a4,pagesize]{tudscrreprt}
\usepackage{selinput}
\SelectInputMappings{adieresis={ä},germandbls={ß}}
\usepackage[T1]{fontenc}
\usepackage[ngerman]{babel}
\usepackage{blindtext}

\begin{document}
\chapter{Überschrift Eins}
\Blindtext

\cleardoublepage
\storeareas\PotraitArea% speichert den aktuellen Satzspiegel
\KOMAoptions{paper=A3,paper=landscape,DIV=current}
\chapter{Überschrift Zwei}
\Blindtext

\cleardoublepage
\PotraitArea% lädt den gespeicherten Satzspiegel
\chapter{Überschrift Drei}
\Blindtext
\end{document}
\end{code}
\end{Example}

\subsection{Finden von unbekannten Symbolen}
\index{Symbole}
Für \hologo{LaTeX} stehen jede Menge Symbole zur Verfügung, die allerdings nicht 
immer einfach zu finden sind. Mit der Datei \File{symbols-a4.pdf} werden viele 
Symbole aus mehreren Paketen aufgeführt. Allerdings ist das Auffinden eines 
speziellen Symbols nicht sehr komfortabel. Alternativ dazu kann der Link 
\url{http://detexify.kirelabs.org/classify.html} verwendet werden. Auf dieser 
Seite wird das gesuchte Symbol einfach gezeichnet, die dazu ähnlichsten werden 
zurückgegeben.


\section{Editoren}
\label{sec:tat:editor}
Hier werden die gängigsten Editoren zum Erzeugen von \hologo{LaTeX}"=Dateien 
genannt. Ich persönlich bin mittlerweile sehr überzeugter Nutzer von 
\Application{\hologo{TeX}studio}, da dieser viele Unterstützungs- und 
Assistenzfunktionen bietet. Neben diesen gibt es noch weitere, gut nutzbare 
\hologo{LaTeX}-Editoren. Egal, für welchen Editor man sich letztendlich 
entscheidet, sollte dieser auf jeden Fall eine Unicode"=Unterstützung enthalten:
%
\begin{itemize}
\item \Application{\hologo{TeX}maker}
\item \Application{\hologo{TeX}works}
\item \Application{\hologo{TeX}lipse} -- Plug-in für \Application{Eclipse}
\item \Application{\hologo{TeX}nicCenter}
\item \Application{WinEdt}
\item \Application{LEd} -- früher \hologo{LaTeX}~Editor
\item \Application{\hologo{LyX}} -- grafisches Front"~End für \hologo{LaTeX}
\end{itemize}
%
Unter \url{http://wwwpub.zih.tu-dresden.de/~fahan/tudscr/} werden für den Editor
\Application{\hologo{TeX}studio} zwei CWL"~Dateien zur Erweiterung 
der automatischen Befehlsvervollständigung für das \TUDScript-Bundle 
bereitgestellt. Außerdem findet man dort die notwendigen Layout-Dateien und ein 
Minimalbeispiel für die Verwendung des \TUDScript-Bundles zusammen mit 
\hologo{LyX}. Der Inhalt des Archives \File{tudscr4lyx.zip} muss zur Verwendung 
dafür in den \hologo{LyX}"=Installationspfad unter
\Path{\dots\textbackslash LyX~2.0\textbackslash Resources\textbackslash layouts} 
kopiert werden.


\appendix
\chapter{Minimalbeispiele}\label{sec:mwe}
\index{Minimalbeispiel}

\section{Dokument}\label{sec:mwe:doc}
\index{Minimalbeispiel!Dokument}
\includeexample{document}

\section{Abschlussarbeit}\label{sec:mwe:thesis}
\index{Minimalbeispiel!Abschlussarbeit}
\includeexample{thesis}

\section{Kollaborative Abschlussarbeit}\label{sec:mwe:collab}
\index{Minimalbeispiel!Kollaboratives Schreiben}
Alle zusätzlichen Angaben außerhalb des Argumentes von \Macro{author} werden 
für beide Autoren gleichermaßen übernommen.%
\footnote{In diesem Beispiel \Macro{matriculationyear}.}
Die Angaben innerhalb des Argumentes von \Macro{author} werden den jeweiligen, 
mit \Macro{and} getrennten Autoren zugeordnet.%
\footnote{%
  In diesem Beispiel \Macro{matriculationnumber}, \Macro{dateofbirth} und 
  \Macro{placeofbirth}.
}
\includeexample{collaborative}

\section{Aufgabenstellung (kollaborativ)}\label{sec:mwe:task}
\index{Minimalbeispiel!Aufgabenstellung}
\index{Minimalbeispiel!Kollaboratives Schreiben}
\includeexample{task}

\section{Gutachten}\label{sec:mwe:eval}
\index{Minimalbeispiel!Gutachten}
\includeexample{evaluation}

\section{Aushang}\label{sec:mwe:note}
\index{Minimalbeispiel!Aushang}
\includeexample{notice}

\section{Änderung der Trennzeichen im Mathematikmodus}\label{sec:mwe:swap}
\index{Minimalbeispiel!Trennzeichen Mathematikmodus}
Hierfür wird das Paket \Package{mathswap} genutzt. Eine Alternative dazu wäre 
das Paket \Package{ionumbers}.
\includeexample{mathswap}



\chapter{Ein Beitrag zum Mathematiksatz in \NoCaseChange{\hologo{LaTeX}}}
\label{sec:mathtype}\index{Mathematiksatz}
Ein guter Mathematiksatz ist in \hologo{LaTeX} durchaus Sisyphusarbeit. Wenn man
microtypografisch alles richtig machen möchte, muss man ziemlich auf der Hut
sein. Generell gilt, dass Variablen kursiv, Bezeichnungen und Konstanten
aufrecht gesetzt werden. Um beschreibende Indizes von Formelzeichen richtig zu
setzen, ist die Nutzung der Befehle \Macro*{mathrm}\PParameter{\dots} und
\Macro*{mathit}\PParameter{\dots} sehr zu empfehlen. Dadurch wird aus:
%
\begin{code}
\begin{equation*}
\begin{gathered}
M_{EM} = \frac{M_{Rad}}{i_g \cdot i_A} - M_{VM} \\
\textrm{für }
\begin{aligned}
0\leq M_{VM}\leq M_{VMmax} \\ 
M_{EMmin}\leq M_{EM}\leq M_{EMmax}
\end{aligned}
\end{gathered}
\end{equation*}
\end{code}
\begin{equation*}
\begin{gathered}
M_{EM} = \frac{M_{Rad}}{i_g \cdot i_A} - M_{VM} \\
\textrm{für }
\begin{aligned}
0\leq M_{VM}\leq M_{VMmax} \\ 
M_{EMmin}\leq M_{EM}\leq M_{EMmax}
\end{aligned}
\end{gathered}
\end{equation*}
%
mit ziemlich viel Anpassungsarbeit:
%
\begin{code}
\begin{equation*}
\begin{gathered}
M_\mathrm{EM} = \frac{M_\mathrm{Rad}}{i_g \cdot i_A} - M_\mathrm{VM} \\
\textrm{für }
\begin{aligned}
0\leq M_\mathrm{VM}\leq M_\mathrm{VM_{max}} \\ 
M_\mathrm{EM_{min}}\leq M_\mathrm{EM}\leq M_\mathrm{EM_{max}}
\end{aligned}
\end{gathered}
\end{equation*}
\end{code}
\begin{equation*}
\begin{gathered}
M_\mathrm{EM} = \frac{M_\mathrm{Rad}}{i_g \cdot i_A} - M_\mathrm{VM} \\
\textrm{für }
\begin{aligned}
0\leq M_\mathrm{VM}\leq M_\mathrm{VM_{max}} \\ 
M_\mathrm{EM_{min}}\leq M_\mathrm{EM}\leq M_\mathrm{EM_{max}}
\end{aligned}
\end{gathered}
\end{equation*}
%
Ziemlich viel Arbeit, das sollte sich auf alle Fälle vereinfachen lassen. Zu 
diesem Zwecke wird mit \Macro*{ind}\PParameter{\dots} ein Befehl für den 
Formelzeichenindex selber definiert. Und danach kann man sich noch beliebige
Befehle für häufig verwendete Ausdrücke schnitzen. Als Beispiel für ein 
Drehmoment könnte man folgendes definieren:
%
\newcommand*{\ind}[1]{\ensuremath{_\mathrm{#1}}}
\newcommand*{\M}[1]{\ensuremath{M\ind{#1}}}
\begin{code}
\newcommand*{\ind}[1]{\ensuremath{_\mathrm{#1}}}
\newcommand*{\M}[1]{\ensuremath{M\ind{#1}}}
\end{code}
%
und damit diese Ausgabe erzeugen:
%
\begin{code}
\begin{equation*}
\begin{gathered}
\M{EM} = \frac{\M{Rad}}{i_g \cdot i_A} - \M{VM} \\
\textrm{für }
\begin{aligned}
0\leq \M{VM}\leq \M{VM_{max}} \\ 
\M{EM_{min}}\leq \M{EM}\leq \M{EM_{max}}
\end{aligned}
\end{gathered}
\end{equation*}
\end{code}
\begin{equation*}
\begin{gathered}
\M{EM} = \frac{\M{Rad}}{i_g \cdot i_A} - \M{VM} \\
\textrm{für }
\begin{aligned}
0\leq \M{VM}\leq \M{VM_{max}} \\ 
\M{EM_{min}}\leq \M{EM}\leq \M{EM_{max}}
\end{aligned}
\end{gathered}
\end{equation*}
%
Netter Nebeneffekt ist, dass man aufgrund der Verwendung von 
\Macro*{ensuremath}\PParameter{\dots} nun diesen Befehl auch im Fließtext 
verwenden kann, beispielsweise wie hier \M{VM_{ind}} 
(\Macro*{M}\PParameter{VM\_{ind}}) für das induzierte Moment einer 
Verbrennungskraftmaschine.

Möchte man es sich noch bequemer machen, strikt man sich noch eine Lösung, in
der man -- im Gegensatz zum \hologo{LaTeX}"=Standardfall -- \textbf{nach} dem
obligatorischen Argument noch ein optionales für einen weiteren Index angeben
kann, um damit der üblichen Schreibweise zu entsprechen. Das ist aber ehrlich 
gesagt nur noch ein wenig Spielerei und soll zeigen, wie so etwas prinzipiell 
funktioniert. Es wird der Befehl \Macro*{M} so definiert, das dieser entweder in 
der Form \Macro*{M}\Parameter{Index} oder aber in der Variante 
\Macro*{M}\Parameter{Index}\OParameter{Indexindex} genutzt werden kann.
\makeatletter
\let\ind\undefined % Ausgangszustand herstellen, nicht übernehmen
\let\M\undefinied   % Ausgangszustand herstellen, nicht übernehmen
\newcommand*{\ind}[1]{\ensuremath{_\mathrm{#1}}}
\newcommand*{\M}[1]{\@ifnextchar[%]
  {\o@M{#1}}{\n@M{#1}}%
}
\newcommand*{\n@M}{}
\newcommand*{\o@M}{}
\def\n@M#1{\ensuremath{M\ind{#1}}}
\def\o@M#1[#2]{\ensuremath{M\ind{#1_{#2}}}}
\makeatother
\begin{code}
\makeatletter
\newcommand*{\ind}[1]{\ensuremath{_\mathrm{#1}}}
\newcommand*{\M}[1]{\@ifnextchar[%]
  {\o@M{#1}}{\n@M{#1}}%
}
\newcommand*{\n@M}{}
\newcommand*{\o@M}{}
\def\n@M#1{\ensuremath{M\ind{#1}}}
\def\o@M#1[#2]{\ensuremath{M\ind{#1_{#2}}}}
\makeatother
\end{code}
%
Zum Schluss noch einmal die entwickelte Variante in Quelltext und Ausgabe:
%
\noindent
\begin{code}
\begin{equation*}
\begin{gathered}
\M{EM} = \frac{\M{Rad}}{i\ind{G} \cdot i\ind{A}} - \M{VM} \\
\textrm{für }
\begin{aligned}
0\leq \M{VM}\leq \M{VM}[max] \\ 
\M{EM}[min]\leq \M{EM}\leq \M{EM}[max]
\end{aligned}
\end{gathered}
\end{equation*}
\end{code}
\begin{equation*}
\begin{gathered}
\M{EM} = \frac{\M{Rad}}{i\ind{G} \cdot i\ind{A}} - \M{VM} \\
\textrm{für }
\begin{aligned}
0\leq \M{VM}\leq \M{VM}[max] \\ 
\M{EM}[min]\leq \M{EM}\leq \M{EM}[max]
\end{aligned}
\end{gathered}
\end{equation*}


\Index{Optionen}{options}%
\Index{Befehle}{macros}%
\Index{Umgebungen}{macros}%
\Index{Parameter}{keys}%
\Index[Bezeichner!Übersicht]{Bezeichner}{terms}%
%\Index{Schriftelemente}{fonts}%
\Index[Farben!Übersicht]{Farben}{colors}%
\Index[Dateien]{Dateien etc.}{files}%
\Index[Pakete]{Dateien etc.}{files}%
\Index[Klassen]{Dateien etc.}{files}%
\index{Abbildungen|see{Grafiken}}%
\index{Abkürzungsverzeichnis|see{Verzeichnisse}}%
\index{Aktualisierung|see{Update}}%
\index{Aufzählungen|see{Listen}}%
\index{Befehle!Befehlsparameter|see{Parameter}}%
\index{Cover|see{Umschlagseite}}%
\index{Dezimaltrennzeichen|see{Trennzeichen}}%
\index{Distribution!\hologo{TeX}~Live|see{\hologo{TeX}~Live~{\protect\idxfont(Distribution)}}}
\index{Distribution!\hologo{MiKTeX}|see{\hologo{MiKTeX}~{\protect\idxfont(Distribution)}}}
\index{Fachreferent|see{Referent}}%
\index{Farbraum|see{Farben!Farbmodell}}%
\index{Formelzeichenverz.|see{Verzeichnisse}}%
\index{Glossar|see{Verzeichnisse}}%
\index{Grafiken!Beschriftung|see{Gleitobjekte}}
\index{Großbuchstaben|see{Schriftauszeichnung}}%
\index{Klassenoptionen|see{Optionen}}%
\index{Kleinbuchstaben|see{Schriftauszeichnung}}%
\index{Kurzfassung|see{Zusammenfassung}}%
\index{Literaturverzeichnis|see{Verzeichnisse}}%
\index{Lokalisierung|see{Bezeichner}}%
\index{Majuskeln|see{Schriftauszeichnung}}%
\index{Mathematiksatz!Einheiten|see{Einheiten}}
\index{Mathematiksatz!Trennzeichen|see{Trennzeichen}}
\index{Minuskeln|see{Schriftauszeichnung}}%
\index{Outline-Eintrag|see{Lesezeichen}}%
\index{Professor|see{Hochschullehrer}}%
\index{Quelltextverzeichnis|see{Verzeichnisse}}%
\index{Seitenränder|see{Satzspiegel}}
\index{Silbentrennung|see{Worttrennung}}%
\index{Sprachunterstützung!Lokalisierung|see{Bezeichner}}%
\index{Sprachunterstützung!Worttrennung|see{Worttrennung}}%
\index{Sprungmarken|see{Lesezeichen}}
\index{Symbolverzeichnis|see{Verzeichnisse}}%
\index{Tabellen!Beschriftung|see{Gleitobjekte}}
\index{Tausendertrennzeichen|see{Trennzeichen}}%
\index{Trennmuster|see{Worttrennung}}%
\index{Umgebungen!Umgebungsparameter|see{Parameter}}%
\index{Vakatseiten|see{Leerseiten}}%
\index{Vektorgrafiken|see{Grafiken}}%

\setchapterpreamble{%
  \begin{abstract}
  \noindent Die Formatierung der Einträge in allen aufgeführten Indizes ist 
  folgendermaßen aufzufassen: \textbf{Zahlen in fetter Schrift} verweisen auf 
  die \textbf{Erklärung} zu einem Stichwort, wobei in der digitalen Fassung 
  dieses Handbuchs dieser Eintrag selbst ein Hyperlink zu seiner Erläuterung 
  ist. Seitenzahlen in normaler Schriftstärke hingegen deuten auf zusätzliche 
  Informationen, wobei diese für \textit{kursiv hervorgehobene Zahlen} als 
  besonders \textit{wichtig} erachtet werden.
  
  Bei Einträgen für \hyperref[idx:options]{Klassen- und Paketoptionen} 
  beziehungsweise für \hyperref[idx:macros]{Umgebungen und Befehle}, zu denen 
  keine direkte \textbf{Erklärung} gegeben ist sondern lediglich zusätzliche 
  Hinweise vorhanden sind, handelt es sich um \KOMAScript"=Optionen. Diese sind 
  gegebenenfalls im dazugehörigen Handbuch nachzulesen (\File{scrguide.pdf}).
  \end{abstract}
}
\addchap{\indexname}
\let\umlautshack\relax
\pagestyle{myheadings}
\markboth{\indexname}{\indexname}
\PrintIndex
%\PrintChangelog
\end{document}