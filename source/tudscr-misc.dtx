% \iffalse meta-comment
%/GitFileInfo=tudscr-misc.dtx
%
%  TUD-Script -- Corporate Design of Technische Universität Dresden
% ----------------------------------------------------------------------------
%
%  Copyright (C) Falk Hanisch <hanisch.latex@outlook.com>, 2012-2020
%
% ----------------------------------------------------------------------------
%
%  This work may be distributed and/or modified under the conditions of the
%  LaTeX Project Public License, version 1.3c of the license. The latest
%  version of this license is in http://www.latex-project.org/lppl.txt and
%  version 1.3c or later is part of all distributions of LaTeX 2005/12/01
%  or later and of this work. This work has the LPPL maintenance status
%  "author-maintained". The current maintainer and author of this work
%  is Falk Hanisch.
%
% ----------------------------------------------------------------------------
%
%  Dieses Werk darf nach den Bedingungen der LaTeX Project Public Lizenz
%  in der Version 1.3c, verteilt und/oder verändert werden. Die aktuelle
%  Version dieser Lizenz ist http://www.latex-project.org/lppl.txt und
%  Version 1.3c oder später ist Teil aller Verteilungen von LaTeX 2005/12/01
%  oder später und dieses Werks. Dieses Werk hat den LPPL-Verwaltungs-Status
%  "author-maintained", wird somit allein durch den Autor verwaltet. Der
%  aktuelle Verwalter und Autor dieses Werkes ist Falk Hanisch.
%
% ----------------------------------------------------------------------------
%
% \fi
%
% \iffalse ins:batch + dtx:driver
%<*ins>
\ifx\documentclass\undefined
  \input docstrip.tex
  \ifToplevel{\batchinput{tudscr.ins}}
\else
  \let\endbatchfile\relax
\fi
\endbatchfile
%</ins>
%<*dtx>
\ProvidesFile{tudscr-misc.dtx}[2020/07/27]
\RequirePackage{tudscr-gitinfo}
\documentclass[english,ngerman,xindy]{tudscrdoc}
\iftutex
  \usepackage{fontspec}
\else
  \usepackage[T1]{fontenc}
  \usepackage[ngerman=ngerman-x-latest]{hyphsubst}
\fi
\usepackage{babel}
\usepackage{tudscrfonts}
\usepackage{bookmark}
\usepackage[babel]{microtype}

\begin{document}
  \maketitle
  \tableofcontents
  \DocInput{\filename}
\end{document}
%</dtx>
% \fi
%
% \selectlanguage{ngerman}
%
% \changes{v2.02}{2014/06/23}{Paket \pkg{titlepage} nicht weiter unterstützt}^^A
% \changes{v2.02}{2014/07/08}{\cs{FamilyKeyState} wird von Optionen genutzt}^^A
%
% \section{Verschiedenes für das \TUDScript-Bundle}
%
% Alles, wofür sich eine separate Datei nicht lohnt, landet hier.
% \ToDo{\cs{PreventPackageFromLoading} Pakete scrpage[2] und scrlettr}[v2.07]
% \ToDo{\opt{mathsubscript=true/false/upright/italic}, siehe vlp}[v2.07]
%^^A  % TODO option
%^^A  % TODO catcodes in marcos für true/false
%^^A  \AtBeginDocument{\catcode`\_=12 \mathcode`\_="8000}
%^^A  \tud@newif\if@mathsub@upright
%^^A    \newcommand*\mathsub@set[1]{%
%^^A      \if@mathsub@upright%
%^^A        \sb{\mathrm{#1}}%
%^^A      \else%
%^^A        \sb{\mathit{#1}}%
%^^A      \fi%
%^^A    }
%^^A    \newcommand*\mathsub@swap{}
%^^A    \def\mathsub@swap|#1|{%
%^^A      \begingroup%
%^^A        \swapuprightsubscripts%
%^^A        \mathsub@set{#1}%
%^^A      \endgroup%
%^^A    }
%^^A  \begingroup
%^^A    \catcode`\_=13
%^^A    \gdef_{\@ifnextchar|\mathsub@swap\mathsub@set}%
%^^A  \endgroup
%^^A  \newcommand*\enableuprightsubscripts{\@mathsub@uprighttrue}
%^^A  \newcommand*\disableuprightsubscripts{\@mathsub@uprightfalse}
%^^A  \newcommand*\swapuprightsubscripts{%
%^^A    \if@mathsub@upright%
%^^A      \@mathsub@uprightfalse%
%^^A    \else%
%^^A      \@mathsub@uprighttrue%
%^^A    \fi%
%^^A  }
%
% \StopEventually{\PrintIndex\PrintChanges\PrintToDos}
%
% \iffalse
%<*class&option>
% \fi
%
% \subsection{Sprungmarken}
%
% \begin{option}{tudbookmarks}
% \begin{macro}{\if@tud@bookmarks}
% Wird das Paket \pkg{hyperref} geladen, so kann die Option \opt{tudbookmarks}
% genutzt werden, um zu steuern, ob für Titel und Inhaltsverzeichnis
% automatisch ein Eintrag für die Sprungmarken erzeugt werden soll.
%    \begin{macrocode}
\tud@newif\if@tud@bookmarks
\TUD@key{tudbookmarks}[true]{%
  \TUD@set@ifkey{tudbookmarks}{@tud@bookmarks}{#1}%
}
%    \end{macrocode}
% \end{macro}^^A \if@tud@bookmarks
% \end{option}^^A tudbookmarks
%
% \iffalse
%</class&option>
%<*class&body>
% \fi
%
% \begin{macro}{\tudbookmark}
% \begin{macro}{\tud@x@hyperref@realfootnotes}
% Die Befehle zum Eintragen der Sprungmarken. Im Dokument kann \cs{tudbookmark}
% auch vom Anwender genutzt werden, um weitere Lesezeichen abhängig von der
% Option \opt{tudbookmarks} manuell zu erzeugen.
%
% Für die Titelseite werden~-- wie auch durch das Paket \pkg{hyperref}~-- die 
% Fußnoten mit dem Befehl \cs{tud@x@hyperref@realfootnotes} auf den originalen 
% Zustand zurückgesetzt.
% \ToDo{optionales Argument (default=nil) auf integer>0 prüfen}[v2.07]
%    \begin{macrocode}
\newcommand*\tudbookmark[1][]{\@gobbletwo}
\newcommand*\tud@x@hyperref@realfootnotes{}
\AfterPackage{hyperref}{%
  \renewcommand*\tudbookmark[3][]{%
    \relax%
    \if@tud@bookmarks%
      \phantomsection%
      \ifblank{#1}{\pdfbookmark{#2}{#3}}{\pdfbookmark[{#1}]{#2}{#3}}%
    \fi%
  }%
  \renewcommand*\tud@x@hyperref@realfootnotes{%
    \let\Hy@saved@footnotemark\@footnotemark%
    \let\Hy@saved@footnotetext\@footnotetext%
    \let\@footnotemark\H@@footnotemark%
    \let\@footnotetext\H@@footnotetext%
  }%
  \pdfstringdefDisableCommands{\let\NoCaseChange\@firstofone}%
}
%    \end{macrocode}
% \end{macro}^^A \tud@x@hyperref@realfootnotes
% \end{macro}^^A \tudbookmark
% \begin{macro}{\tud@x@bookmark@startatroot}
% \changes{v2.05}{2015/08/05}{neu}^^A
% Wird das Paket \pkg{bookmark} geladen, können die erzeugten Outlines gezielt 
% auf der obersten Ebene erzeugt werden.
%    \begin{macrocode}
\newcommand*\tud@x@bookmark@startatroot{\relax}
\AfterPackage{bookmark}{%
  \renewcommand*\tud@x@bookmark@startatroot{%
    \if@tud@bookmarks\bookmarksetup{startatroot}\fi%
  }%
}
%    \end{macrocode}
% \end{macro}^^A \tud@x@bookmark@startatroot
% \begin{environment}{titlepage}
% \changes{v2.05}{2015/08/05}{neu}^^A
% \begin{macro}{\@maketitle}
% \changes{v2.05}{2015/08/05}{neu}^^A
% \begin{macro}{\tud@maketitle@inpage}
% \changes{v2.05}{2015/08/05}{neu}^^A
% Ist die Option \opt{tudbookmarks} aktiviert, werden für Umschlag- und 
% Titelseite PDF"=Lesezeichen bzw. Outline-Einträge erzeugt.
%    \begin{macrocode}
%<*book|report|article>
\apptocmd\titlepage{%
  \tud@x@bookmark@startatroot%
  \@tud@res@swafalse%
  \if@tud@coverpage\@tud@res@swatrue\fi%
  \if@titlepageiscoverpage\@tud@res@swatrue\fi%
  \if@tud@res@swa%
    \tudbookmark[%
%<book|report>    0%
%<article>    1%
    ]{\coverpagename}{cover}%
  \else%
    \tudbookmark[%
%<book|report>    0%
%<article>    1%
    ]{\titlepagename}{title}%
  \fi%
}{}{\tud@patch@wrn{titlepage}}
%</book|report|article>
%    \end{macrocode}
% Dies geschieht außerdem auch für die Titelköpfe.
%    \begin{macrocode}
\pretocmd\tud@maketitle@inpage{%
  \tud@x@bookmark@startatroot%
  \tudbookmark[%
%<book|report>  0%
%<article|poster>  1%
  ]{\titlename}{title}%
}{}{\tud@patch@wrn{tud@maketitle@inpage}}
\pretocmd\@maketitle{%
  \tud@x@bookmark@startatroot%
  \tudbookmark[%
%<book|report>  0%
%<article|poster>  1%
  ]{\titlename}{title}%
}{}{\tud@patch@wrn{@maketitle}}
%    \end{macrocode}
% \end{macro}^^A \tud@maketitle@inpage
% \end{macro}^^A \@maketitle
% \end{environment}^^A titlepage
% \begin{macro}{\tableofcontents}
% \begin{macro}{\listoffigures}
% \begin{macro}{\listoftables}
% Für die Verzeichnisse geschieht dies auch. Für das Inhaltsverzeichnis ist 
% hierfür etwas mehr Aufwand notwendig. Für dieses wird im Normalfall keine 
% PDF-Outline erzeugt. Deshalb ist hier ein wenig Trickserei notwendig, um zwar 
% einen Outline-Eintrag für das Inhaltsverzeichnis zu erzeugen, dieses dabei
% aber nicht in sich selbst einzutragen.
%    \begin{macrocode}
\BeforeTOCHead[toc]{%
  \tud@x@bookmark@startatroot%
  \let\@tempc\relax%
  \Iftocfeature{toc}{totoc}{}{%
    \if@tud@bookmarks%
      \setuptoc{toc}{totoc}%
      \tud@cs@store{addtocontents}%
      \def\@tempc{%
        \unsettoc{toc}{totoc}%
        \tud@cs@restore{addtocontents}%
      }%
      \let\addtocontents\@gobbletwo%
    \fi%
  }%
}
\AfterTOCHead[toc]{\@tempc}
\BeforeTOCHead[lof]{\tud@x@bookmark@startatroot}
\BeforeTOCHead[lot]{\tud@x@bookmark@startatroot}
%    \end{macrocode}
% \end{macro}^^A \listoftables
% \end{macro}^^A \listoffigures
% \end{macro}^^A \tableofcontents
%
% \iffalse
%</class&body>
%<*class&option>
% \fi
%
% \subsection{Fußnoten in Überschriften}
%
% Eigentlich sollte die Option \opt{footnotes}|=|\val{symbolheadings} es 
% ermöglichen, Fußnoten mit Symbolen in Überschriften zu verwenden. Umgesetzt 
% werden sollte dies mit dem Paket \pkg{footmisc}. Da dieses aber mit 
% Inkompatibilitäten zu \pkg{hyperref} und \KOMAScript{} aufwartet, wird darauf 
% leider verzichtet.
%^^A% \begin{option}{footnotes}
%^^A% \changes{v2.02}{2014/06/27}{neu}^^A
%^^A% \begin{macro}{\if@tud@symbolheadings}
%^^A% \changes{v2.02}{2014/06/27}{neu}^^A
%^^A% \begin{macro}{\tud@sec@note}
%^^A% \changes{v2.02}{2014/06/27}{neu}^^A
%^^A% \begin{counter}{symbolheadings}
%^^A% \changes{v2.02}{2014/06/27}{neu}^^A
%^^A% Die \KOMAScript-Option \opt{footnotes} wird um den Schlüssel 
%^^A% \val{symbolheadings} erweitert, mit welchem eingestellt werden kann, ob 
%^^A% in den Überschriften Symbole für die Fußnoten anstelle von Zahlen 
%^^A% verwendet werden sollen.
%^^A%    \begin{macrocode}
%^^A\tud@newif\if@tud@symbolheadings
%^^A\TUD@key{footnotes}{%
%^^A  \TUD@set@numkey{footnotes}{@tempa}{%
%^^A    {nosymbolheadings}{0},{numberheadings}{0},%
%^^A    {symbolheadings}{1}%
%^^A  }{#1}%
%^^A  \ifx\FamilyKeyState\FamilyKeyStateProcessed%
%^^A    \ifcase\@tempa\relax% nosymbolheadings
%^^A      \@tud@symbolheadingsfalse%
%^^A    \or% symbolheadings
%^^A      \@tud@symbolheadingstrue%
%^^A    \fi%
%^^A  \fi%
%^^A}
%^^A%    \end{macrocode}
%^^A% Mit \cs{tud@sec@note} wird die Option \opt{footnotes=symbolheadings} 
%^^A% umgesetzt. Hierfür wird das Paket \pkg{footmisc} benötigt.
%^^A%    \begin{macrocode}
%^^A\newcommand*\tud@sec@note[1]{%
%^^A  \ClassWarning{\TUD@Class@Name}{%
%^^A    You have to load package `footmisc' with\MessageBreak%
%^^A    option `stable' in order to use\MessageBreak%
%^^A    footnotes with symbols in headings%
%^^A  }%
%^^A}
%^^A\AfterPackage{footmisc}{%
%^^A  \@ifpackagewith{footmisc}{stable}{%
%^^A    \renewcommand*\tud@sec@note[1]{%
%^^A      \begingroup%
%^^A        \if@tud@symbolheadings%
%^^A          \let\c@footnote\c@symbolheadings%
%^^A          \def\thefootnote{\fnsymbol{symbolheadings}}%
%^^A        \fi%
%^^A        \footnote{#1}%
%^^A      \endgroup%
%^^A    }%
%^^A  }{}%
%^^A}
%^^A%    \end{macrocode}
%^^A% Bei aktivierter Option \opt{footnotes}|=|\val{symbolheadings} wird  
%^^A% dieser Zähler für die Symboleauswahl von Fußnoten in Überschirften 
%^^A% inkrementiert.
%^^A%    \begin{macrocode}
%^^A\AtEndOfClass{%
%^^A%<book|report>  \newcounter{symbolheadings}[chapter]
%^^A%<article|poster>  \newcounter{symbolheadings}
%^^A}
%^^A%    \end{macrocode}
%^^A% \end{counter}^^A symbolheadings
%^^A% \end{macro}^^A \tud@sec@note
%^^A% \end{macro}^^A \if@tud@symbolheadings
%^^A% \end{option}^^A footnotes
%
% \subsection{Papierformat und Schriftgröße}
%
% Insbesondere für Poster aber auch für alle anderen \TUDScript-Klassen wird
% überprüft, ob nach einer Änderung des Standardpapierformates auch die
% Schriftgröße durch den Anwender angepasst wurde. Diese wird jedoch nicht auf
% Plausibilität geprüft, da der Aufwand hierfür relativ hoch wäre. Vielmehr 
% wird davon ausgegangen, dass die explizite Angabe der Schriftgröße bewusst 
% und dem Papierformat sowie den verwendeten Textspalten entsprechend erfolgt.
%
% \begin{macro}{\tud@x@scr@checkoptions}
% \changes{v2.06}{2018/07/31}{neu}^^A
% \begin{macro}{\tud@x@scr@pt}
% \changes{v2.06}{2018/07/31}{neu}^^A
% Das Makro \cs{tud@x@scr@checkoptions} wird mit \cs{DeclareOption*} für alle 
% Optionen aufgerufen, welche kein Mitglied der Familie~\val{TUD} sind. Damit 
% kann auf nicht definierte Schlüssel reagiert werden. Anschließend werden diese
% an die zugrundeliegende \KOMAScript-Klasse weitergereicht.
%    \begin{macrocode}
\newcommand*\tud@x@scr@checkoptions{%
  \expandafter\tud@x@scr@pt\CurrentOption pt\@nil%
  \PassOptionsToClass{\CurrentOption}{\TUD@Class@KOMA}%
}
%    \end{macrocode}
% Wird die Schriftgröße im veralteten Format \val{\dots pt} als Option
% übergeben, so wird der Schalter \cs{if@tud@x@scr@fontsize@set} auf \val{true} 
% gesetzt.
%    \begin{macrocode}
\newcommand*\tud@x@scr@pt{}
\def\tud@x@scr@pt #1pt#2\@nil{%
  \Ifstr{#2}{pt}{\Ifnumber{#1}{\@tud@x@scr@fontsize@settrue}{}}{}%
}
%    \end{macrocode}
% \end{macro}^^A \tud@x@scr@pt
% \end{macro}^^A \tud@x@scr@checkoptions
%
% \iffalse
%</class&option>
%<*class&body>
% \fi
%
% Wurde das Papierformat jedoch nicht die Schriftgröße geändert, wird eine
% Warnung erzeugt.
%    \begin{macrocode}
\AtBeginDocument{%
  \if@tud@x@scr@fontsize@set\else%
    \@tud@res@swafalse%
    \ifdim\tud@dim@layoutwidth<208mm\relax\@tud@res@swatrue\fi%
    \ifdim\tud@dim@layoutwidth>212mm\relax\@tud@res@swatrue\fi%
    \ifdim\tud@dim@layoutheight<295mm\relax\@tud@res@swatrue\fi%
    \ifdim\tud@dim@layoutheight>299mm\relax\@tud@res@swatrue\fi%
    \if@tud@res@swa%
      \ClassWarningNoLine{\TUD@Class@Name}{%
        You've set a layout size, which is different from\MessageBreak%
        default (paper=a4, paper=portrait). Therefore you\MessageBreak%
        should additionally specify an explicit font size. \MessageBreak%
        See the manual for further information%
      }%
    \fi%
  \fi%
}
%    \end{macrocode}
%
% \subsection{Fixierte Gleitobjekte für Poster}
%
% \begin{environment}{figurehere}
% \changes{v2.05}{2016/04/17}{neu}^^A
% \changes{v2.07}{2019/09/10}{für \cls{tudscrposter} übernommen}^^A
% \begin{environment}{tablehere}
% \changes{v2.05}{2016/04/17}{neu}^^A
% \changes{v2.07}{2019/09/10}{für \cls{tudscrposter} übernommen}^^A
% \begin{macro}{\tud@here@begin}
% \changes{v2.07}{2019/09/10}{neu}^^A
% \begin{macro}{\tud@here@end}
% \changes{v2.07}{2019/09/10}{neu}^^A
% Für die Klasse \cls{tudscrposter} werden zwei Umgebungen bereitgestellt, mit 
% denen sich Tabellen und Bilder an einer bestimmten Stelle exakt platzieren
% lassen. Prinzipiell könnte hierfür auch das Paket \pkg{float} zum Einsatz
% kommen.
%    \begin{macrocode}
%<*poster>
\newcommand*\tud@here@begin[1]{%
  \def\@captype{#1}%
  \vskip\intextsep%
  \color@vbox%
    \normalcolor%
    \vbox\bgroup%
      \hsize=\columnwidth%
      \@parboxrestore%
      \@floatboxreset%
      \@setnobreak%
      \ignorespaces%
}
\newcommand*\tud@here@end{%
      \vskip\intextsep%
      \@minipagefalse%
      \outer@nobreak%
    \egroup%
  \color@endbox%
}
\newenvironment{figurehere}{\tud@here@begin{figure}}{\tud@here@end}
\newenvironment{tablehere}{\tud@here@begin{table}}{\tud@here@end}
%</poster>
%    \end{macrocode}
% \end{macro}^^A \tud@here@end
% \end{macro}^^A \tud@here@begin
% \end{environment}^^A tablehere
% \end{environment}^^A figurehere
%
% \iffalse
%</class&body>
%<*body>
% \fi
%
% \subsection{Anpassungen für das Paket \pkg{textcase}}
%
% \begin{macro}{\@uclcnotmath}
% \begin{macro}{\tud@x@textcase@uclcnotmath}
% \changes{v2.02}{2014/06/27}{\pkg{textcase}: Ignorieren von Fußnoten im 
%   Argument des Befehls \cs{MakeTextUppercase}}^^A
% \changes{v2.03}{2015/01/21}{\pkg{textcase}: Ignorieren des Befehls 
%   \cs{@mkboth} im Argument des Befehls \cs{MakeTextUppercase}}^^A
% Damit Fußnoten innerhalb von \cs{MakeTextUppercase} nicht in Majuskeln
% gesetzt werden, wird der interne Befehl \cs{@uclcnotmath} aus dem Paket
% \pkg{textcase} angepasst.
%    \begin{macrocode}
\if@tud@cdoldfont@active
  \AtEndPreamble{\RequirePackage{textcase}[2019/09/14]}%
\fi
\AfterPackage*{textcase}{%
  \newcommand*\tud@x@textcase@uclcnotmath{%
    \@nonchangecase\thanks%
    \def\footnote##1##{\toks@{\noexpand\footnote##1}\@footnote}%
    \def\@footnote##1{\NoCaseChange{\the\toks@{##1}}}%
    \def\@mkboth##1##2{\NoCaseChange{\@mkboth{##1}{##2}}}%
  }%
  \robustify\NoCaseChange%
  \patchcmd\@uclcnotmath{%
    \@nonchangecase\ensuremath%
  }{%
    \@nonchangecase\ensuremath%
    \tud@x@textcase@uclcnotmath%
  }{}{\tud@patch@wrn{@uclcnotmath}}%
}
\TUD@deprecated@cs{NoCaseChange}{@firstofone}
%    \end{macrocode}
% \end{macro}^^A \tud@x@textcase@uclcnotmath
% \end{macro}^^A \@uclcnotmath
%
% \iffalse
%</body>
%<*option>
% \fi
%
% \subsection{Sukkzessives Abarbeiten von Optionen im Dokument}
%
% Werden mit \cs{TUDoptions} nach \cs{begin\{document\}} mehrere Layoutoptionen 
% gleichzeitig angegeben, muss dafür Sorge getragen werden, dass diese in der 
% korrekten Reihenfolge abgearbeitet werden. 
%
% Das Makro \cs{TUD@SpecialOptionAtDocument} kann innerhalb von \cs{TUD@key} 
% verwendet werden, um einen im obligatorischen Argument angegebenen Schalter 
% zu aktivieren. Dabei wird über \cs{tud@atdocument@hook} garantiert, dass der 
% Befehl \cs{tud@atdocument@process}, welcher für die Abarbeitung der einzelnen 
% Makros in der richtigen Reihenfolge verantwortlich ist, lediglich einmal über 
% \cs{AtEndOfFamilyOptions*} ausgeführt wird.
%
% \begin{macro}{\TUD@SpecialOptionAtDocument}
% \changes{v2.05}{2016/06/14}{neu}^^A
% \begin{macro}{\tud@atdocument@hook}
% \changes{v2.05}{2016/06/14}{neu}^^A
%    \begin{macrocode}
\newcommand*\tud@atdocument@hook{}
\let\tud@atdocument@hook\relax
\newcommand*\TUD@SpecialOptionAtDocument[1]{}
\AfterEndPreamble{%
  \renewcommand*\TUD@SpecialOptionAtDocument[1]{%
    \ifx\tud@atdocument@hook\relax%
      \gdef\tud@atdocument@hook{\global\let\tud@atdocument@hook\relax}%
      \AtEndOfFamilyOptions*{%
        \tud@atdocument@hook%
        \tud@atdocument@process%
      }%
    \fi%
    \gappto\tud@atdocument@hook{\global\booltrue{@#1}}%
  }%
}
%    \end{macrocode}
% \end{macro}^^A \tud@atdocument@hook
% \end{macro}^^A \TUD@SpecialOptionAtDocument
%
% \iffalse
%</option>
%<*body>
% \fi
%
% \begin{macro}{\tud@atdocument@process}
% \changes{v2.05}{2016/06/14}{neu}^^A
% \begin{macro}{\if@tud@font@set}
% \changes{v2.05}{2016/06/14}{neu}^^A
% \begin{macro}{\if@tud@math@set}
% \changes{v2.05}{2016/06/14}{neu}^^A
% \begin{macro}{\if@tud@relspacing@set}
% \changes{v2.05}{2016/06/14}{neu}^^A
% \begin{macro}{\if@tud@x@scr@headings@reset}
% \changes{v2.05}{2016/06/14}{neu}^^A
% \begin{macro}{\if@tud@cd@process}
% \changes{v2.05}{2016/06/14}{neu}^^A
% \begin{macro}{\if@tud@cdgeometry@process}
% \changes{v2.05}{2016/06/14}{neu}^^A
% Das Makro \cs{tud@atdocument@process} wird im Dokument ausgeführt, wenn 
% mindestens eine Option gesetzt wurde, welche \cs{TUD@SpecialOptionAtDocument} 
% verwendet. Je nachdem, welche Schalter insgesamt aktiviert wurden, werden die 
% dazugehörigen Befehle für die Einstellungen von Schriften, Layout und/oder 
% Satzspiegel in der richtigen Reihenfolge ausgeführt.
%    \begin{macrocode}
\tud@newif\if@tud@font@set
\tud@newif\if@tud@math@set
\tud@newif\if@tud@relspacing@set
%<*class>
\tud@newif\if@tud@x@scr@headings@reset
\tud@newif\if@tud@cd@process
\tud@newif\if@tud@cdgeometry@process
%</class>
\newcommand*\tud@atdocument@process{%
  \if@tud@font@set%
    \tud@font@set%
    \global\@tud@font@setfalse%
    \global\@tud@relspacing@setfalse%
%<class>    \global\@tud@cd@processfalse%
  \fi%
  \if@tud@relspacing@set%
    \tud@relspacing@set%
    \global\@tud@relspacing@setfalse%
  \fi%
  \if@tud@math@set%
    \tud@math@set%
    \global\@tud@math@setfalse%
  \fi%
%<*class>
  \if@tud@x@scr@headings@reset%
    \tud@x@scr@headings@reset%
    \global\@tud@x@scr@headings@resetfalse%
    \global\@tud@cd@processfalse%
  \fi%
  \if@tud@cd@process%
    \tud@cd@process%
    \global\@tud@cd@processfalse%
  \fi%
  \if@tud@cdgeometry@process%
    \tud@cdgeometry@process%
    \global\@tud@cdgeometry@processfalse%
  \fi%
%</class>
}
%    \end{macrocode}
% \end{macro}^^A \if@tud@cdgeometry@process
% \end{macro}^^A \if@tud@cd@process
% \end{macro}^^A \if@tud@x@scr@headings@reset
% \end{macro}^^A \if@tud@relspacing@set
% \end{macro}^^A \if@tud@math@set
% \end{macro}^^A \if@tud@font@set
% \end{macro}^^A \tud@atdocument@process
%
% \iffalse
%</body>
%<*execute>
%<*class&!inherit|package&fonts>
% \fi
%
% \subsection{Versionsabhängige Voreinstellungen}
%
% Voreinstellungen von neu eingeführten Optionen, welche die resultierende 
% Ausgabe entscheidend verändern könnten, werden hier versionsabhängig gesetzt.
% Hierfür wird der Schlüssel \opt{tudscrver} des Mitglieds \val{.comp} vor 
% allen anderen Schlüsseln ausgewertet. Danach kann mit \cs{tud@if@v@lower} auf 
% die gewählte Kompatibilitäsversion getestet werden.
%    \begin{macrocode}
\DeclareOption*{}
\TUDProcessOptions[.comp]\relax
%    \end{macrocode}
% Ab der \TUDScript-Version~v2.05 werden einige vertikale Abstände abhängig von 
% der Schriftgröße definiert. Sollte eine frühere Kompatibilitätseinstellung 
% angegeben worden sein, wird dieses verhalten deaktiviert.
%    \begin{macrocode}
\tud@if@v@lower{2.05}{%
  \TUD@key@preset{relspacing}{false}%
}{%
  \TUD@key@preset{relspacing}{true}%
}
%    \end{macrocode}
% Ab \TUDScript-Version~v2.06 werden die verwendeten Schriften auf \OpenSans 
% gesetzt, für die Komptibilität zu vorherigen Versionen auf \Univers und \DIN.
%    \begin{macrocode}
\tud@if@v@lower{2.06}{%
%<!(class&poster)>  \TUD@key@preset{cdoldfont}{true}%
%<class&poster>  \TUD@key@preset{cdoldfont}{ultrabold}%
  \TUD@key@preset{ttfont}{lmodern}%
%<class>  \TUD@key@preset{headings}{light}%
}{%
%<!(class&poster)>  \TUD@key@preset{cdfont}{true}%
%<class&poster>  \TUD@key@preset{cdfont}{ultrabold}%
  \TUD@key@preset{ttfont}{roboto}%
%<class>  \TUD@key@preset{headings}{heavy}%
}
%    \end{macrocode}
%
% \iffalse
%</class&!inherit|package&fonts>
% \fi
%
% Es werden die Standardoptionen ausgeführt. Für die Klasse \cls{tudscrposter} 
% werden die Farben aktiviert.
%    \begin{macrocode}
%<*package&tutorial>
\KOMAoptions{headings=small}
\TUDoptions{cdfoot=true}
%</package&tutorial>
%<*class&!inherit|package&fixfonts>
\TUDExecuteOptions{%
%<*package&fixfonts>
  headings=heavy%
%</package&fixfonts>
%<*class&!inherit>
%<*book|report|article>
  cd=true,tudbookmarks=true%
%</book|report|article>
%<*poster>
  cd=bicolor,tudbookmarks=false%
%</poster>
%</class&!inherit>
}
%</class&!inherit|package&fixfonts>
%    \end{macrocode}
%
% \subsection{Durchreichen von Optionen und Standardoptionen}
%
% Durchreichen aller Klassenoptionen an die \KOMAScript-Klasse bzw. an die
% genutzte \TUDScript-Elternklasse. Für Klassen wird vor dem Durchreichen noch 
% geprüft, ob ein spezielles Papierformat angegeben wurde.
%    \begin{macrocode}
%<*package>
%<*fonts>
\DeclareOption*{\TUD@KOMAoptions{\CurrentOption}}
%</fonts>
%<*fixfonts>
\DeclareOption*{\PassOptionsToPackage{\CurrentOption}{tudscrfonts}}
%</fixfonts>
%<*comp>
\DeclareOption*{\TUDoptions{\CurrentOption}}
%</comp>
%</package>
%<*class>
%<*!inherit>
\DeclareOption*{\tud@x@scr@checkoptions}
%</!inherit>
%<*inherit>
\DeclareOption*{\PassOptionsToClass{\CurrentOption}{\TUD@Class@Parent}}
%</inherit>
%</class>
%    \end{macrocode}
% Nach der Auswerung der Optionen wird die korrespondierende \KOMAScript-Klasse 
% bzw. \TUDScript-Elternklasse geladen.
%    \begin{macrocode}
\TUDProcessOptions\relax
%<*class&!inherit>
\LoadClass{\TUD@Class@KOMA}[%
%!TUD@KOMAVersion
]
%</class&!inherit>
%<*class&inherit>
%<manual>\PassOptionsToClass{ttfont=roboto}{\TUD@Class@Parent}
\LoadClass{\TUD@Class@Parent}[%
%!TUD@Version
]
%</class&inherit>
%    \end{macrocode}
%
% \iffalse
%<*class&!inherit|package&fonts>
% \fi
%
% \subsection{Optionen ohne späte Wahlmöglichkeit}
%
% Einige Optionen können nur zum Zeitpunkt des Ladens einer Klasse oder eines 
% Paketes respektive in der Präambel gesetzt werden. Diese müssen \emph{nach}
% \cs{TUDProcessOptions} mit \cs{TUD@key@toolate} redefiniert werden.
%
% \begin{macro}{\TUD@key@toolate}
% \changes{v2.06c}{2019/08/20}{neu}^^A
% Dieser Befehl definiert eine Fehlermeldung für Schlüssel respektive Optionen, 
% die zu spät verwendet wurden, wobei er innerhalb der Makros \cs{AtEndOfClass},
% \cs{AtEndOfClass}, \cs{AtEndPreamble} etc. genutzt werden sollte. Wird dieser 
% noch im Kontext von Klasse oder Paket (\verb=\@currname.\@currext-h@@k=)
% verwendet, wird gemeldet, dass der Schlüssel allein beim Laden der Klasse 
% respektive des Paketes genutzt werden darf. Andernfalls wird die zwingende
% Verwendung in der Präambel angemerkt.
%    \begin{macrocode}
\newcommand*\TUD@key@toolate[1]{%
  \csedef{tud@key@toolate@@#1}{%
%<*class>
    \noexpand\ClassError{\noexpand\TUD@Class@Name}%
%</class>
%<*package&fonts>
    \noexpand\PackageError{tudscrfonts}%
%</package&fonts>
    {Option `#1' too late}{%
      Option `#1' can only be set\space%
      \ifcsdef{\@currname.\@currext-h@@k}{%
        as optional \MessageBreak argument while loading\space%
%<*class>
        class `\noexpand\TUD@Class@Name'%
%</class>
%<*package&fonts>
        package `tudscrfonts'%
%</package&fonts>
      }{%
        in the preamble%
      }%
      \MessageBreak but you have tried to set it up later.%
    }%
  }%
  \TUD@key[]{#1}[]%
  {%
    \csuse{tud@key@toolate@@#1}%
    \FamilyKeyStateProcessed%
  }%
}
%    \end{macrocode}
% \end{macro}^^A \TUD@key@toolate
%
%
% \iffalse
%</class&!inherit|package&fonts>
%</execute>
% \fi
%
% \Finale
%
\endinput
