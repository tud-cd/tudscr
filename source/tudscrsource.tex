\ifx\documentclass\undefined
  \input docstrip.tex
  \ifToplevel{\batchinput{tudscr.ins}}
\else
  \let\endbatchfile\relax
\fi
\endbatchfile

\RequirePackage{tudscr-gitinfo}
\documentclass[english,ngerman,xindy]{tudscrdoc}
\iftutex
  \usepackage{fontspec}
\else
  \usepackage[T1]{fontenc}
  \usepackage[ngerman=ngerman-x-latest]{hyphsubst}
\fi
\usepackage{tudscrfonts}
\usepackage{babel}
\usepackage[babel]{microtype}

\GitHubBase{\TUDScriptRepository}
\begin{document}
\addtokomafont{subject}{\sffamily}
\subject{\TUDScript basierend auf \KOMAScript}
\title{Ein \LaTeX-Bundle für Dokumente \mbox{im \CD der} \mbox{\TnUD}}
\subtitle{Quelltextdokumentation}
\author{Falk Hanisch\TUDScriptContactTitle}
\date{\TUDScriptVersion}

\makeatletter
\begingroup%
  \def\and{, }%
  \let\thanks\@gobble%
  \let\footnote\@gobble%
  \let\mailto\@gobble%
  \let\qquad\relax%
  \hypersetup{%
    pdfauthor   = {\@author},%
    pdftitle    = {\@title},%
    pdfsubject  = {\@subtitle},%
    pdfkeywords = {LaTeX, \TUDScript, Quelltext},%
  }%
\endgroup%
\let\@maketitle\scr@maketitle%
\makeatother


\maketitle


Das \TUDScript-Bundle setzt das \TUDCD für \LaTeX{} um. Die enthaltenen 
Klassen und Pakete basieren auf dem \KOMAScript-Bundle und sind sehr eng mit 
diesen verwoben. Momentan ergänzen sie das Vorlagenbundle von Klaus~Bergmann, 
das auf den \LaTeX"=Standardklassen basiert und als veraltet betrachtet werden 
kann. Die dazugehörigen Klassen sollen mittel- bis langfristig ersetzt werden.%
\footnote{%
  aktuell ist dies \cls{tudbook}, geplant \cls{tudfax}, \cls{tudletter}, 
  \cls{tudform}, \cls{tudhaus} und evtl. auch \cls{tudbeamer}%
}
Es handelt sich bei diesem Dokument \emph{nicht} um das Anwenderhandbuch
sondern um den dokumentierten Quelltext der Implementierung von \TUDScript.
Das Anwenderhandbuch kann via Kommandozeile/Terminal mit \texttt{texdoc tudscr} 
geöffnet werden.

\tableofcontents
\ToDo{\cs{TeX}, \cs{LaTeX}, \cs{app} etc. mit \cs{Logo} ersetzen}[v2.07]
% TODO der ganze Kram mit [%]^^A kann einfach raus
% TODO Hinweis in README.md auf 
%(<texmf>/source/latex/tudscr/doc/examples/)%
%[https://ctan.org/tex-archive/macros/latex/contrib/tudscr/source/doc/examples]
\ToDo{alle \cs{PassOptionsTo...} \emph{vor} \cs{ProcessOptions}}[v2.07]
\ToDo{%
  \cs{babelprovide}\val{[hyphenrules=ngerman-x-latest]{ngerman}} 
  anstelle von \pkg{hyphsubst}
}[v2.07]
\ToDo{test auf \cs{@empty} mit \cs{ifblank} besser?}[v2.07]% \\ifx.*\\@empty
\ToDo{Verwendung aller \cs{tud@if@nil} prüfen}[v2.07]
\ToDo{Paket \pkg{letltxmacro} durch NewCommandCopy aus Kernel ersetzen}[v2.07]
\ToDo{Paket \pkg{environ} durch NewDocumentEnvironment Kernel ersetzen}[v2.07]
\ToDo{Paket \pkg{datetime2} gegenüber \pkg{isodate} bevorzugen?}[v2.07]
\ToDo{Paket \pkg{mweights} raus, DeclareFontSeriesDefault nutzen}[v2.07]
\ToDo{\cs{IfPackageAtLeastTF} anstelle von \cs{@ifpackagelater}}[v2.07]
\ToDo{Paket \pkg{kvsetkeys} durch \pkg{scrbase} ersetzen}[v2.07]
\ToDo{Paket \pkg{trimspaces} mit Kernel-Befehlen ersetzen}[v2.07]
\ToDo{\cs{Ifnumber} mit \cs{Ifisinteger} ersetzen?!}[v2.07]
\ToDo{restricted write ermöglichen, wofür ist shell-escape nötig}[v2.07]
\ToDo{ähnlich zu tudscrmanual mdfive test}[v2.07]
\ToDo{Release-Prozess mit \app{make} und \file{Makefile}}[v2.07]
\ToDo{^^A
  \cs{kernel@ifnextchar}, \cs{kernel@ifstar} mit \cs{NewDocumentCommand}^^A
}[v2.07]
\ToDo{\cs{IfArgIsEmpty} überprüfen/ersetzen}[v2.07]
% renewcommand[^\{]*?\\[^\{\[]*?\{
\ToDo{\cs{renewcommand} mit \cs{def} ersetzen, wo sinnvoll}[v2.07]
\clearpage

\DocInclude{tudscr-version}
\DocInclude{tudscr-base}
\DocInclude{tudscr-fonts}
\DocInclude{tudscr-area}
\DocInclude{tudscr-pagestyle}
\DocInclude{tudscr-layout}
\DocInclude{tudscr-title}
\DocInclude{tudscr-frontmatter}
\DocInclude{tudscr-misc}
\DocInclude{tudscr-comp}
\DocInclude{tudscr-fields}
\DocInclude{tudscr-localization}

\DocInclude{tudscr-color}
\DocInclude{tudscr-supervisor}

\ifdefined\tudfinalflag\else
\DocInclude{tudscr-manual}
\DocInclude{tudscr-doc}
\DocInclude{tudscr-texindy}
\DocInclude{tudscr-gitinfo}
\DocInclude{tudscr-twocolfix}
\DocInclude{tudscr-mathswap}
\DocInclude{install/tudscr-metrics}
\DocInclude{install/tudscr-scripts}
\fi

\PrintBackMatter*
\end{document}
