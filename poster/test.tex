\RequirePackage[ngerman=ngerman-x-latest]{hyphsubst}
\documentclass[%
  ngerman,
%  titlepage,
%  widehead,
%  ddcfoot,
  geometry=no,
%  DIV=12,
%  twoside,
%  draft,
%%  cd=color,
%  paper=a6,
%cdfont=no,
]{tudscrreprt}
\usepackage{selinput}\SelectInputMappings{adieresis={ä},germandbls={ß}}
\usepackage[T1]{fontenc}
\usepackage{babel}
\usepackage{blindtext}
\usepackage{showframe}
\usepackage{layout}
\usepackage{tudcolors}

\begin{document}
\pagestyle{tudheadings}
\clearpage
\layout*
%\layout
\makeatletter
\faculty{Juristische Fakultät}
\department{Fachrichtung Strafrecht}
\institute{Institut für Kriminologie}
\chair{Lehrstuhl für Kriminalprognose und noch etwas}
\extraheadline{Lehrstuhl für Kriminalprognose }
\date{16.12.2014}
\author{Mickey Mouse}
\title{Die Klasse tudscrposter}
%\subtitle{Eine \LaTeX klasse für die Evaluationsposter}
\maketitle

\makeatletter
\DeclareNewLayer[%
  background,align=tl,%
  contents={%
    \color{HKS41}%
%    \gdef\tud@head@color{white}%
    \rule{\layerwidth}{\tud@dim@headheight}%
  },%
]{tudheadings.head.back}%
\DeclareNewLayer[%
  background,align=tl,%
  voffset=\dimexpr\tud@dim@headheight-.0\tud@dim@line\relax,%
  contents={%
    \color{HKS41!80}%
%    \rule[\dimexpr\ht\strutbox\relax]{#1}{\tud@dim@line}%
    \rule{\layerwidth}{\tud@dim@barheight}
  },%
]{tudheadings.head.bar}%
\DeclareNewLayer[%
  background,
  %bottommargin,
  foot,
  %footskip,
  hoffset=0pt,
  contents={%
    \color{HKS41!80}%
    \rule{\paperwidth}{\layerheight}%
  },%
]{tudheadings.foot.back}%

\DeclareNewPageStyleByLayers{color.tudheadings}{%
  tudheadings.head.back,%
  tudheadings.head.bar,%
  tudheadings.head.mainlogo,%
  tudheadings.head.logo,%
%  tudheadings.head.topline,%
%  tudheadings.head.bottomline,%
  tudheadings.head.topwideline,%
  tudheadings.head.bottomwideline,%
  tudheadings.head.text,%
  tudheadings.foot.back,%
  tudheadings.foot.ddclogo,%
  scrheadings.foot.odd,%
  scrheadings.foot.even,%
  scrheadings.foot.oneside,%
  scrheadings.foot.above.line,%
  scrheadings.foot.below.line%
}%
\listadd\tud@pslist{color.tudheadings}
%\AddToLayerPageStyleOptions{color.tudheadings}{%
%  onselect={\tud@ps@init}%
%}%



\pagestyle{color.tudheadings}

%\maketitle

\section{Überschrift}
\blindtext
\subsection{Überschrift}
\blindtext
\subsubsection{Überschrift}
\blindtext
\clearpage


%\TUDoptions{ddcfoot}
\KOMAoptions{footheight=5cm}
\recalctypearea
\section{Überschrift}
\blindtext
\subsection{Überschrift}
\blindtext
\subsubsection{Überschrift}
\blindtext

\layout*
\end{document}
