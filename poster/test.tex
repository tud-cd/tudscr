\RequirePackage[ngerman=ngerman-x-latest]{hyphsubst}
\documentclass[%
  english,ngerman,%ddcfoot,%
%  paper=a3,pagesize,
  cdhead=bicolor,%cdfoot=bicolor,
  cdfoot,automark,%ddcfoot,
  cd=color,%
%  cdfont=no,
%  cdgeometry=no,
%  DIV=15,
%  fontsize=14pt,
]%
%{tudscrartcl}
%{tudscrreprt}
{tudscrreprtnew}
%{tudbook}
%{scrreprt}
%{report}
%
\usepackage{selinput}\SelectInputMappings{adieresis={ä},germandbls={ß}}
\usepackage[T1]{fontenc}
\usepackage{babel}
\usepackage{blindtext}
\usepackage{multicol}


%\usepackage[ddcfoot,cdstyle=full,cdfoot=1cm]{tudscrposter}
%\KOMAoptions{fontsize=32}
%\TUDoptions{cdfontsize=32}
%\usepackage[cam,b4]{crop}

% §§§ geht das doch irgendwie mit der farbe der Seitenzahl?
% §§§ evtl über letzte Ebene?!
%\addtokomafont{pagenumber}{\typeout{+++++}\color{red}}
%\addtokomafont{pageheadfoot}{\color{red}}

\begin{document}
\def\blubb{blubb}
%\recalctypearea
%\ifcsdef{tudcls@name}{%
\pagestyle{tudheadings}
%\tud@font@koma@set{pageheadfoot}{\color{red}}
%\tud@font@koma@set{pagenumber}{\color{red}}
%\blinddocument
\def\bla{%
  Dies hier ist ein Blindtext zum Testen von Textausgaben. Wer diesen Text 
  liest, ist selbst schuld. Der Text gibt lediglich den Grauwert der Schrift 
  an. Ist das wirklich so? Ist es gleichgültig, ob ich schreibe: „Dies ist ein 
  Blindtext“ oder „Huardest gefburn“? Kjift – mitnichten! Ein Blindtext bietet 
  mir wichtige Informationen. An ihm messe ich die Lesbarkeit einer Schrift, 
%  ihre Anmutung, wie harmonisch die Figuren zueinander stehen
}
\footcontent{%
%\begin{multicols}{2}
\bla\par
%\end{multicols}
}
\makeatletter
\footcontent{%
\begin{minipage}[t]{\dimexpr (\textwidth-\columnsep)/2\relax}
  Dies hier ist ein Blindtext zum Testen von Textausgaben. Wer diesen Text 
  liest, ist selbst schuld. Der Text gibt lediglich den Grauwert der Schrift 
  an. Ist das wirklich so? Ist es gleichgültig, ob ich schreibe:% „Dies ist ein 
%   Blindtext“ oder „Huardest gefburn“? Kjift
\end{minipage}%
\hspace*{\columnsep}%
\begin{minipage}[t]{\dimexpr (\textwidth-\columnsep)/2-2cm\relax}
  Blindtext“ oder „Huardest gefburn“? Kjift – mitnichten! Ein Blindtext bietet 
%  mir wichtige Informationen. An ihm messe ich die Lesbarkeit einer Schrift, 
%  ihre Anmutung
%  
\end{minipage}
%\typeout{+++++ \meaning\tud@foot@logocolor}
}
\makeatother
%\addtokomafont{pagenumber}{\color{red}}
\part{title}
\chapter{title}
\Blindtext%
\clearpage
\pagestyle{scrheadings}
\TUDoptions{cdfoot=nocolor}
\Blindtext%
\Blindtext%
\clearpage
\TUDoptions{cdhead=nocolor}
\Blindtext%
\Blindtext%
\clearpage
%\meaning\tud@font@koma@pageheadfoot

\clearpage
\footcontent{%
\begin{multicols}{2}
\bla
\end{multicols}
}
\begin{multicols}{2}
\Blindtext%
\end{multicols}
\clearpage

%\TUDoptions{cdstyle=lite}
%\pagestyle{empty.tudheadings}

\footcontent{%
\begin{minipage}[t]{\dimexpr (\textwidth-\columnsep)/2\relax}
  Dies hier ist ein Blindtext zum Testen von Textausgaben. Wer diesen Text 
  liest, ist selbst schuld. Der Text gibt lediglich den Grauwert der Schrift 
  an. Ist das wirklich so? Ist es gleichgültig, ob ich schreibe:% „Dies ist ein 
%   Blindtext“ oder „Huardest gefburn“? Kjift
\end{minipage}%
\hspace*{\columnsep}%
\begin{minipage}[t]{\dimexpr (\textwidth-\columnsep)/2-2cm\relax}
  Blindtext“ oder „Huardest gefburn“? Kjift – mitnichten! Ein Blindtext bietet 
%  mir wichtige Informationen. An ihm messe ich die Lesbarkeit einer Schrift, 
%  ihre Anmutung
%  
\end{minipage}
}
\begin{multicols}{2}
\Blindtext%
\Blindtext%
Dies hier ist ein Blindtext zum Testen.
\end{multicols}

\clearpage
\Blindtext%
\Blindtext%

\fmtversion
\end{document}
