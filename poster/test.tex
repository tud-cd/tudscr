\RequirePackage[ngerman=ngerman-x-latest]{hyphsubst}
\documentclass[%
  english,ngerman,%
%  paper=a3,pagesize,%
%  DIV=15,fontsize=14pt,%
%  cd=color,%
  cdhead=bicolor,
  cdfoot=bicolor,
%  ddcfoot=color,%
%  cdfoot,automark,%
]%
{tudscrreprt}
%{tudmathposter}
%{tudbook}
%{scrreprt}
%{report}
%
\usepackage{selinput}\SelectInputMappings{adieresis={ä},germandbls={ß}}
\usepackage[T1]{fontenc}
\usepackage{babel}
\usepackage{blindtext}
\usepackage{multicol}
\usepackage{etoolbox}
\usepackage{tudscrfonts}

%\usepackage[ddcfoot,cdstyle=full,cdfoot=1cm]{tudscrposter}
%\KOMAoptions{fontsize=32}
%\TUDoptions{cdfontsize=32}
%\usepackage[cam,b4]{crop}

% §§§
% ToDo: tudscrposter
% - Standardfußzeile über vorgegebene Felder definieren
%
% ToDo: tudmathposter
% Unterstützung der Befehle und Umgebungen:
% - \begin{farbtabellen}\end{farbtabellen} mit \grautabelle, \blautabelle
% - \telefon, \fax, \homepage, \email
% - \fusszeile
%   – Die linke Spalte enthält Hochschule, Einrichtung, Fachrichtung, Institut
%     und Professur. Institut und Professur sollten mit \institut und \professur
%     gesetzt werden. Die restlichen, Einrichtung und Fachrichtung, werden 
%     automatisch gesetzt.
%   - Die rechte Spalte ist frei wählbar, und kann alternativ mit den
%     vorgegebenen Variablen \author, \telefon, \email und \homepage oder mit 
%     einem frei gewählten Absatz (\footcolumn2) gefüllt werden. Hinweis: Wenn 
%     sich die Homepage mit dem Institutslogo überschneidet, kann jedes     
%     beliebige Feld mit Zeilenumbrüchen vertikal erweitert werden. Dazu können 
%     die üblichen Makrokombinationen wie \\ für den Zeilenumbruch und \strut 
%     zum Erzeugen von Inhalt für eine leere Zeile genutzt werden.

\begin{document}
\faculty{Juristische Fakultät}
\department{Fachrichtung Strafrecht}
\institute{Institut für Kriminologie}
\chair{Lehrstuhl für Kriminalprognose}
\extraheadline{textadline}
\pagestyle{empty.tudheadings}

%\renewcommand{\tudheadingsfootcontent}{def}

\def\bla{%
  Dies hier ist ein Blindtext zum Testen von Textausgaben. Wer diesen Text 
  liest, ist selbst schuld. Der Text gibt lediglich den Grauwert der Schrift 
  an. Ist das wirklich so? Ist es gleichgültig, ob ich schreibe: „Dies ist ein 
  Blindtext“ oder „Huardest gefburn“? Kjift – mitnichten! Ein Blindtext bietet 
  mir wichtige Informationen. An ihm messe ich die Lesbarkeit einer Schrift und 
  ihre Anmutung.
}
\tudheadingsfoot*{\bla}%[\bla]
\blindtext
\clearpage

\tudheadingsfoot{\bla}%[\bla]
\blindtext
\clearpage

\tudheadingsfoot{%
  Dies hier ist ein Blindtext zum Testen von Textausgaben. Wer diesen Text 
  liest, ist selbst schuld. Der Text gibt lediglich den Grauwert der Schrift 
  an. Ist das wirklich so? Ist es gleichgültig, ob ich schreibe:
}[%
  „Dies ist ein Blindtext“ oder „Huardest gefburn“?Kjift – mitnichten! Ein 
  Blindtext bietet mir wichtige Informationen. An ihm messe ich die Lesbarkeit 
  einer Schrift und ihre Anmutung.
]
\blindtext%

\LaTeXe-Version: \fmtversion
\end{document}
