\RequirePackage[ngerman=ngerman-x-latest]{hyphsubst}
\documentclass[%
  english,ngerman,%ddcfoot,%
%  paper=a3,pagesize,
%  cdhead,
  ddcfoot=color,%
  cdhead=bicolor,
%  twoside,open=right,clearcolor,
%  cdfoot=bicolor,
%  cdfoot,automark,%ddcfoot,
%  tudscrver=2.02,
  cd=color,%
%  titlepage=no,
%  cdtitle=lite,
%  cdfont=no,
  cdgeometry=no,
%  twocolumn,
%  titlepage=no,
  abstract=no,
%  DIV=15,
%  fontsize=14pt,
]%
{tudscrartcl}
%{tudscrreprt}
%{tudscrreprtnew}
%{tudbook}
%{scrreprt}
%{report}
%
\usepackage{selinput}\SelectInputMappings{adieresis={ä},germandbls={ß}}
\usepackage[T1]{fontenc}
%\usepackage{babel}
\usepackage{blindtext}
\usepackage{multicol}
\usepackage{etoolbox,quoting}

%\usepackage[ddcfoot,cdstyle=full,cdfoot=1cm]{tudscrposter}
%\KOMAoptions{fontsize=32}
%\TUDoptions{cdfontsize=32}
%\usepackage[cam,b4]{crop}

\begin{document}
%\section{title}
%bla bla bla bla bla bla bla bla bla bla bla bla bla bla bla bla bla bla 
%bla bla bla bla bla bla bla bla bla bla bla bla bla bla bla bla bla bla 
%bla bla bla bla bla bla bla bla bla bla bla bla bla bla bla bla bla bla 
%bla bla bla bla bla bla bla bla bla bla bla bla bla bla bla bla bla bla 
%\section{title}
%bla bla bla bla bla bla bla bla bla bla bla bla bla bla bla bla bla bla 
%bla bla bla bla bla bla bla bla bla bla bla bla bla bla bla bla bla bla 
%bla bla bla bla bla bla bla bla bla bla bla bla bla bla bla bla bla bla 
%bla bla bla bla bla bla bla bla bla bla bla bla bla bla bla bla bla bla 

\makeatletter
\TUDoption{abstract}{nofil}
\title{text\thanks{text}}
\maketitle
%\begin{quoting}
%bla bla bla bla bla bla bla bla bla bla bla bla bla bla bla bla bla bla 
%bla bla bla bla bla bla bla bla bla bla bla bla bla bla bla bla bla bla 
%bla bla bla bla bla bla bla bla bla bla bla bla bla bla bla bla bla bla 
%\end{quoting}
%\begin{quoting}
%bla bla bla bla bla bla bla bla bla bla bla bla bla bla bla bla bla bla 
%bla bla bla bla bla bla bla bla bla bla bla bla bla bla bla bla bla bla 
%bla bla bla bla bla bla bla bla bla bla bla bla bla bla bla bla bla bla 
%\end{quoting}

\begin{abstract}[1]
%\noindent
%\blindtext
bla bla bla bla bla bla bla bla bla bla bla bla bla bla bla bla bla bla 
bla bla bla bla bla bla bla bla bla bla bla bla bla bla bla bla bla bla 
bla bla bla bla bla bla bla bla bla bla bla bla bla bla bla bla bla bla 
%bla bla bla bla bla bla bla bla bla bla bla bla bla bla bla bla bla bla 
\nextabstract%[english]
%\noindent
%\blindtext
bla bla bla bla bla bla bla bla bla bla bla bla bla bla bla bla bla bla 
bla bla bla bla bla bla bla bla bla bla bla bla bla bla bla bla bla bla 
bla bla bla bla bla bla bla bla bla bla bla bla bla bla bla bla bla bla 
%bla bla bla bla bla bla bla bla bla bla bla bla bla bla bla bla bla bla 
\nextabstract%
%\noindent
bla bla bla bla bla bla bla bla bla bla bla bla bla bla bla bla bla bla 
bla bla bla bla bla bla bla bla bla bla bla bla bla bla bla bla bla bla 
bla bla bla bla bla bla bla bla bla bla bla bla bla bla bla bla bla bla 
%bla bla bla bla bla bla bla bla bla bla bla bla bla bla bla bla bla bla 
\end{abstract}
\blindtext
%\chapter{title}
%\headlogo{DDC-22}

%\meaning\columnsep
%\meaning\bigskipamount
%
%\ifisskip{\columnsep}{aaa}{bbb}
%\ifisdimen{\columnsep}{aaa}{bbb}
%\ifisskip{\bigskipamount}{aaa}{bbb}

\faculty{Juristische Fakultät}
\department{Fachrichtung Strafrecht}
\institute{Institut für Kriminologie}
\chair{Lehrstuhl für Kriminalprognose}
\extraheadline{textadline}
\pagestyle{tudheadings}
\part{title}
\blinddocument

%\renewcommand{\tudheadingsfootcontent}{def}

\def\bla{%
  Dies hier ist ein Blindtext zum Testen von Textausgaben. Wer diesen Text 
  liest, ist selbst schuld. Der Text gibt lediglich den Grauwert der Schrift 
  an. Ist das wirklich so? Ist es gleichgültig, ob ich schreibe:% „Dies 
  %ist ein 
  Blindtext“ oder „Huardest gefburn“? Kjift – mitnichten! Ein Blindtext bietet 
%  mir wichtige Informationen. An ihm messe ich die Lesbarkeit einer Schrift, 
%  ihre Anmutung, wie harmonisch die Figuren zueinander stehen
%  mir wichtige Informationen. An ihm messe ich die Lesbarkeit einer Schrift, 
%  ihre Anmutung, wie harmonisch die Figuren zueinander stehen
%    mir wichtige Informationen. An ihm messe ich die Lesbarkeit einer Schrift, 
%    ihre Anmutung, wie harmonisch die Figuren zueinander stehen
%      mir wichtige Informationen. An ihm messe ich die Lesbarkeit einer 
%      Schrift, 
%      ihre Anmutung, wie harmonisch die Figuren zueinander stehen
}
\tudheadingsfoot*{\bla}%[\bla]

\Blindtext\blindtext
\clearpage

\tudheadingsfoot{\bla}%[\bla]
\Blindtext\blindtext

%\TUDoptions{cdfoot=bicolor}
%\part[title]{title\footnote{text}}
%\blindtext
%\chapter[title]{title\footnote{text}}
%\blindtext
%\clearpage
%
%\TUDoptions{cdhead=bicolor,cd=fullcolor}
%\part[title]{title\footnote{text}}
%\blindtext
%\chapter[title]{title\footnote{text}}
%\blindtext

\end{document}

%\recalctypearea
%\ifcsdef{tudcls@name}{%
%\pagestyle{tudheadings}
%\tud@font@koma@set{pageheadfoot}{\color{red}}
%\tud@font@koma@set{pagenumber}{\color{red}}
%\blinddocument



\footcontent{%
\begin{minipage}[t]{\dimexpr (\textwidth-\columnsep)/2\relax}
  Dies hier ist ein Blindtext zum Testen von Textausgaben. Wer diesen Text 
  liest, ist selbst schuld. Der Text gibt lediglich den Grauwert der Schrift 
  an. Ist das wirklich so? Ist es gleichgültig, ob ich schreibe:% „Dies ist ein 
%   Blindtext“ oder „Huardest gefburn“? Kjift
\end{minipage}%
\hspace*{\columnsep}%
\begin{minipage}[t]{\dimexpr (\textwidth-\columnsep)/2-1cm\relax}
  Blindtext“ oder „Huardest gefburn“? Kjift – mitnichten! Ein Blindtext bietet 
%  mir wichtige Informationen. An ihm messe ich die Lesbarkeit einer Schrift, 
%  ihre Anmutung
%  
\end{minipage}

%\typeout{+++++ \meaning\tud@foot@logocolor}
}
%\addtokomafont{pagenumber}{\color{red}}

\Blindtext%
%\addtokomafont{tudheadings}{\color{red}}
\clearpage
%\pagestyle{scrheadings}
\TUDoptions{cdfoot=nocolor}
\Blindtext%
\Blindtext%
\clearpage
\TUDoptions{cdhead=nocolor}
\Blindtext%
\Blindtext%
\clearpage
%\meaning\tud@font@koma@pageheadfoot

\clearpage
\footcontent{%
\begin{multicols}{2}
\bla
\end{multicols}
}
\begin{multicols}{2}
\Blindtext%
\end{multicols}
\clearpage

%\TUDoptions{cdstyle=lite}
%\pagestyle{empty.tudheadings}

\footcontent{%
\begin{minipage}[t]{\dimexpr (\textwidth-\columnsep)/2\relax}
  Dies hier ist ein Blindtext zum Testen von Textausgaben. Wer diesen Text 
  liest, ist selbst schuld. Der Text gibt lediglich den Grauwert der Schrift 
  an. Ist das wirklich so? Ist es gleichgültig, ob ich schreibe:% „Dies ist ein 
%   Blindtext“ oder „Huardest gefburn“? Kjift
\end{minipage}%
\hspace*{\columnsep}%
\begin{minipage}[t]{\dimexpr (\textwidth-\columnsep)/2-0cm\relax}
  Blindtext“ oder „Huardest gefburn“? Kjift – mitnichten! Ein Blindtext bietet 
%  mir wichtige Informationen. An ihm messe ich die Lesbarkeit einer Schrift, 
%  ihre Anmutung
%  
\end{minipage}
}
\begin{multicols}{2}
\Blindtext%
\Blindtext%
Dies hier ist ein Blindtext zum Testen.
\end{multicols}

\clearpage
\Blindtext%
\Blindtext%

\fmtversion: \the\columnsep
\end{document}
