\documentclass[english,ngerman]%
{tudscrreprt}
%{scrreprt}
%

\PassOptionsToPackage{cdfont=heavy}{tudscrfonts}
\usepackage{ifluatex}
\ifluatex
\usepackage[no-math]{fontspec}
%\usepackage[cdmath=true]{tudscrfonts}
%\usepackage[cdmath=false]{tudscrfonts}
\usepackage{unicode-math}
%\setmathfont{Asana-Math.otf}
\setmathfont{xits-math.otf}
%\setmathfont{Univers}
%\usepackage[cdmath=true]{tudscrfonts}
%\usepackage[cdmath=false]{tudscrfonts}
\else
\usepackage{selinput}\SelectInputMappings{adieresis={ä},germandbls={ß}}
\usepackage[T1]{fontenc}
\usepackage[cdmath=true]{tudscrfonts}
%\usepackage[cdmath=false]{tudscrfonts}
\fi
\usepackage{babel}
\usepackage{blindtext}
\usepackage{siunitx}
\sisetup{detect-all}

\begin{document}
Das ist Fließtext mit ganz vielen Umlauten, wenn mir denn Wörter mit selbigen 
einfallen würden. Äh...

$1-2=3 - \alpha \cdot \Gamma$

$\mathsfup{abc \alpha\beta\Gamma 0123456789}$

$\mathsfit{abc \alpha\beta\Gamma 0123456789}$

$\mathbfsfup{abc \alpha\beta\Gamma 0123456789}$

$\mathbfsfit{abc \alpha\beta\Gamma 0123456789}$

\SI{-1984}{m^{-42}}

\section{title $abc$}

\makeatletter
\meaning\mv@normal

\meaning\mv@univers

\meaning\mv@dinbold

\LaTeXe-Version: \fmtversion
\end{document}
