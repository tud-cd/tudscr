\documentclass[english,ngerman]{tudscrreprt}
\usepackage{babel}
\usepackage{iftex}
\iftutex
  \usepackage{fontspec}
\else
  \usepackage[T1]{fontenc}
  \usepackage[ngerman=ngerman-x-latest]{hyphsubst}
\fi
\usepackage{scrhack}
\usepackage{tudscrsupervisor}

\AfterPackage*{hyperref}{%
\usepackage[%
  acronym,% Abkürzungen
  symbols,% Formelzeichen
  nomain,% kein Glossar
  nogroupskip,%
  toc,%
  section=chapter,%
  nostyles,%
  translate=babel,%
% mit Tex Live einfach verwendbar
  xindy={language=german-din},
]{glossaries}
\makeglossaries
}% Ende von AfterPackage

\AfterPackage*{glossaries}{%
\newglossarystyle{acrotabu}{%
  \renewenvironment{theglossary}{%
    \begin{tabu}{@{}lX<{\strut}l@{}}% 'spread 0pt' defekt in v2.9
  }{%
    \end{tabu}\par\bigskip%
  }%
  \renewcommand*{\glossaryheader}{}%
  \renewcommand*{\glsgroupheading}[1]{}%
  \renewcommand*{\glsgroupskip}{}%
  \renewcommand*{\glossentry}[2]{%
    \glsentryitem{##1}% Entry number if required
    \glstarget{##1}{\sffamily\bfseries\glossentryname{##1}} &
    \glsentrydesc{##1} &
    ##2\tabularnewline
  }
}

\newcommand*{\newformulasymbol}[5][]{%
  \newglossaryentry{#2}{%
    type=symbols,%
    name={#3},%
    description={\nopostdesc},%
    symbol={\ensuremath{#4}},%
    user1={\ensuremath{\mathrm{#5}}},%
    sort={#2},%
    #1%
  }%
}

\defglsentryfmt[symbols]{%
  \ifmmode%
    \glssymbol{\glslabel}%
  \else%
    \glsgenentryfmt~\glsentrysymbol{\glslabel}%
  \fi%
}
\newglossarystyle{symblongtabu}{%
  \renewenvironment{theglossary}{%
    \begin{longtabu}[l]{ccX<{\strut}l}% 'spread 0pt' defekt in v2.9
  }{%
    \end{longtabu}%
  }%
  \renewcommand*{\glsgroupheading}[1]{}%
  \renewcommand*{\glsgroupskip}{}%
  \renewcommand*{\glossaryheader}{%
    \toprule
    \bfseries Symbol & \bfseries Einheit &
    \bfseries Bezeichnung & \bfseries Seite(n)
    \tabularnewline\midrule\endhead%
    \bottomrule\endfoot%
  }%
  \renewcommand*{\glossentry}[2]{%
    \glsentryitem{##1}% Entry number if required
    \glstarget{##1}{\glossentrysymbol{##1}} &
    \glsentryuseri{##1} &
    \glossentryname{##1} &
    ##2\tabularnewline%
  }%
}
}% Ende von AfterPackage

\usepackage{csquotes}
\usepackage[backend=biber,style=alphabetic]{biblatex}

\usepackage{filecontents}
\begin{filecontents}{\jobname-temp.bib}
@book{goossens94,
  author    = {Goossens, Michel and Mittelbach, Frank
               and Samarin, Alexander},
  title     = {The LaTeX Companion},
  date      = {1994},
  publisher = {Addison-Wesley},
  location  = {Reading, Massachusetts},
  language  = {english},
}
@book{knuth84,
  author    = {Knuth, Donald E.},
  title     = {The \TeX book},
  date      = {1984},
  maintitle = {Computers \& Typesetting},
  volume    = {A},
  publisher = {Addison-Wesley},
  location  = {Reading, Massachusetts},
  language  = {english},
}
@manual{hanisch14,
  author    = {Hanisch, Falk},
  title     = {Ein \LaTeX"=Bundle für Dokumente im neuen Corporate
               Design der Technischen Universität Dresden},
  date      = {2014},
  subtitle  = {Benutzerhandbuch},
  location  = {Dresden},
  language  = {german},
}
\end{filecontents}
\addbibresource{\jobname-temp.bib}

\usepackage{caption}
\captionsetup{font=sf,labelfont=bf,labelsep=space}
\usepackage{floatrow}
\floatsetup{font=sf}
\floatsetup[table]{style=plaintop}
\captionsetup{singlelinecheck=off,format=hang,justification=raggedright}
\DeclareCaptionSubType[alph]{figure}
\DeclareCaptionSubType[alph]{table}
\captionsetup[subfloat]{labelformat=brace,list=off}

\usepackage{booktabs}
\usepackage{array}
\usepackage{tabularx}
\usepackage{tabulary}
\usepackage{tabu}
\usepackage{longtable}

\usepackage{quoting}

\usepackage[babel]{microtype}

\usepackage{enumitem}
\setlist[itemize]{noitemsep}

\usepackage{ellipsis}
\let\ellipsispunctuation\relax

\usepackage{xfrac}

\usepackage{isodate}

\usepackage[colorlinks,linkcolor=blue]{hyperref}

\begin{document}

\faculty{Juristische Fakultät}
\department{Fachrichtung Strafrecht}
\institute{Institut für Kriminologie}
\chair{Lehrstuhl für Kriminalprognose}
\title{%
  Entwicklung eines optimalen Verfahrens zur Eroberung des
  Geldspeichers in Entenhausen
}
\thesis{master}
\graduation[M.Sc.]{Master of Science}
\author{%
  Mickey Mouse%
  \matriculationnumber{12345678}%
  \dateofbirth{2.1.1990}%
  \placeofbirth{Dresden}%
  \course{Klinische Prognostik}%
  \discipline{Individualprognose}%
\and%
  Donald Duck%
  \matriculationnumber{87654321}%
  \dateofbirth{1.2.1990}%
  \placeofbirth{Berlin}%
  \course{Statistische Prognostik}%
  \discipline{Makrosoziologische Prognosen}%
}
\matriculationyear{2010}
\supervisor{Dagobert Duck \and Mac Moneysac}
\professor{Prof. Dr. Kater Karlo}
\date{10.09.2014}
\makecover
\maketitle

\newcommand{\taskcontent}{%
  Momentan ist das besagte Thema in aller Munde. Insbesondere wird es
  gerade in vielen~-- wenn nicht sogar in allen~-- Medien diskutiert.
  Es ist momentan noch nicht abzusehen, ob und wann sich diese Situation
  ändert. Eine kurzfristige Verlagerung aus dem Fokus der Öffentlichkeit
  wird nicht erwartet.

  Als Ziel dieser Arbeit soll identifiziert werden, warum das Thema
  gerade so omnipräsent ist und wie dieser Effekt abgeschwächt werden
  könnte. Zusätzlich sind Methoden zu entwickeln, mit denen sich ein
  ähnlicher Vorgang zukünftig vermeiden lässt.
}
\taskform[pagestyle=empty]{\taskcontent}{%
  \item Recherche
  \item Analyse
  \item Entwicklung eines Konzeptes
  \item Anwendung der entwickelten Methodik
  \item Dokumentation und grafische Aufbereitung der Ergebnisse
}

\TUDoption{abstract}{multiple,section}
\begin{abstract}
  Dies ist der deutschsprachige Teil der Zusammenfassung, in dem die
  Motivation sowie der Inhalt der nachfolgenden wissenschaftlichen
  Abhandlung kurz dargestellt werden.
\nextabstract[english]
  This is the english part of the summary, in which the motivation and
  the content of the following academic treatise are briefly presented.
\end{abstract}

\declaration[company=FIRMA]

\tableofcontents
\listoffigures
\listoftables

\printacronyms[style=acrotabu]
\printsymbols[style=symblongtabu]

\setchapterpreamble{%
  \renewcommand*{\dictumwidth}{.4\textwidth}%
  \dictum[Johann Wolfgang von Goethe]{%
    Es irrt der Mensch, solang er strebt.%
  }%
  \bigskip
}
\chapter{Einleitung}
Nachdem nun der Vorspann und~-- bis auf das Literaturverzeichnis am
Ende des Dokumentes auf Seite~\pageref{sec:bibliography}~-- alle
Verzeichnisse erfolgreich ausgegeben wurden, wird nun die Verwendung
der weiteren Umgebungen und Befehle demonstriert, welche im Tutorial
\texturn{treatise.pdf} vorgestellt wurden.

\section{Die Verwendung von Akronymen und Symbolen}
\newacronym{apsp}{APSP}{All-Pairs Shortest Path}
\newacronym{spsp}{SPSP}{Single-Pair Shortest Path}
\newacronym{sssp}{SSSP}{Single-Source Shortest Path}

In der Graphentheorie wird häufig die Lösung des Problems des kürzesten
Pfades zwischen zwei Knoten gesucht. Dieses Problem wird häufig auch
mit \gls{spsp} bezeichnet. Es lässt sich auf die Variationen \gls{sssp}
und \gls{apsp} erweitern. Für die Lösung von \gls{spsp}, \gls{sssp}
oder \gls{apsp} kommen unterschiedliche Algorithmen zum Einsatz.

\newformulasymbol{l}{Länge}{l}{m}
\newformulasymbol{m}{Masse}{m}{kg}
\newformulasymbol{a}{Beschleunigung}{a}{\sfrac{m}{s^2}}
\newformulasymbol{t}{Zeit}{t}{s}
\newformulasymbol{f}{Frequenz}{f}{s^{-1}}
\newformulasymbol{F}{Kraft}{F}{m \cdot kg \cdot s^{-2} = \sfrac{J}{m}}

Die Einheiten für die \gls{f} sowie die \gls{F} werden aus den
SI"=Einheiten der Basisgrößen \gls{l}, \gls{m} und \gls{t} abgeleitet.
Und dann gibt es noch die Grundgleichung der Mechanik, welche für den
Fall einer konstanten Kraftwirkung in die Bewegungsrichtung einer
Punktmasse lautet:
\[\gls{F} = \gls{m} \cdot \gls{a}\]

\end{document}
