\RequirePackage[cdfont=nomath]{tudscrfonts-fix}
\documentclass[color, ddc]{tudbook}
\usepackage{german}
%\usepackage[cdfont=true]{tudscrfonts}
\usepackage[math]{blindtext}

%\einrichtung{Fakult"at Sprach-, Literatur- und Kulturwissenschaften}
%\institut{Institut f"ur Germanistik}
\begin{document}
    \title{Novelle}
    \author{Johann Wolfgang von Goethe}
    \maketitle

    \tableofcontents
    
    \blindtext

    \chapter{Ein Kapitel}
    \section{Ein Unterabschnitt}

    Ein dichter Herbstnebel verh"ullte noch in der Fr"uhe die weiten R"aume des
    f"urstlichen Schlo"shofes, als man schon mehr oder weniger durch den sich
    lichtenden Schleier die ganze J"agerei zu Pferde und zu Fu"s durcheinander
    bewegt sah. Die eiligen Besch"aftigungen der N"achsten lie"sen sich erkennen:
    man verl"angerte, man verk"urzte die Steigb"ugel, man reichte sich B"uchse
    und Patront"aschchen, man schob die Dachsranzen zurecht, indes die Hunde
    ungeduldig am Riemen den Zur"uckhaltenden mit fortzuschleppen drohten. Auch
    hie und da geb"ardete ein Pferd sich mutiger, von feuriger Natur getrieben
    oder von dem Sporn des Reiters angeregt, der selbst hier in der Halbhelle
    eine gewisse Eitelkeit, sich zu zeigen, nicht verleugnen konnte. Alle jedoch
    warteten auf den F"ursten, der, von seiner jungen Gemahlin Abschied nehmend,
    allzulange zauderte.

%    \chapterpage
    \chapter{Ein anderes Kapitel}
    \section{Ein anderer Unterabschnitt}
    
    Erst vor kurzer Zeit zusammen getraut, empfanden sie schon das Gl"uck
    "ubereinstimmender Gem"uter; beide waren von t"atig lebhaftem Charakter, eines
    nahm gern an des andern Neigungen und Bestrebungen Anteil. Des F"ursten Vater
    hatte noch den Zeitpunkt erlebt und genutzt, wo es deutlich wurde, da"s alle
    Staatsglieder in gleicher Betriebsamkeit ihre Tage zubringen, in gleichem
    Wirken und Schaffen jeder nach seiner Art erst gewinnen und dann genie"sen sollte.

    Wie sehr dieses gelungen war, lie"s sich in diesen Tagen gewahr werden, als eben
    der Hauptmarkt sich versammelte, den man gar wohl eine Messe nennen konnte. Der
    F"urst hatte seine Gemahlin gestern durch das Gewimmel der aufgeh"auften Waren zu
    Pferde gef"uhrt und sie bemerken lassen, wie gerade hier das Gebirgsland mit dem
    flachen Lande einen gl"ucklichen Umtausch treffe; er wu"ste sie an Ort und Stelle
    auf die Betriebsamkeit seines L"anderkreises aufmerksam zu machen.

%    \begin{theglossary}
%        \glossitem{Goethe, Johann Wolfgang} Deutscher Dichter, geb. 1749, 
%gest. 
%        1832.
%        \glossitem{Novelle} Prosaerz"ahlung von J.W. Goethe, 1797 konzipiert 
%        und 1827 neu aufgeschrieben.
%    \end{theglossary}

\end{document}
