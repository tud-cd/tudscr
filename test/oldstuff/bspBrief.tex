\documentclass[finanz, rollstuhl, ddc, emas]{tudletter}
\usepackage{german}

\begin{document}

\einrichtung{Fakult"at Maschinenwesen}
\institut{Institut f"ur Verfahrenstechnik und Umwelttechnik}
\professur{Professur f"ur Thermische Verfahrenstechnik und Umwelttechnik}

\leitertitel{Prof. Dr. rer. nat.}
\leitername{Donald Duck}
\leiterfunktion{Vorsitzender}
\logo{erde}

\bearbeiter{Micky Maus}
\telefon{01 23}
\telefax{01 24}
\email{micky.maus@mailbox.tu-dresden.de}
\aktenzeichen{123456789}

\besucheradrB{Blumenstra"se 13}

\rollstuhlanmerkungen{Aufzug}
\rollstuhladresse{Blumenstra"se 15}

\subject{Novelle}

\begin{letter}{Herrn \\ Dagobert Duck \\ Geldspeicher \\ 01234 Entenhausen}
    \opening{Sehr geehrter Herr Duck,}
        Ein dichter Herbstnebel verh"ullte noch in der Fr"uhe die weiten R"aume des
        f"urstlichen Schlo"shofes, als man schon mehr oder weniger durch den sich
        lichtenden Schleier die ganze J"agerei zu Pferde und zu Fu"s durcheinander
        bewegt sah. Die eiligen Besch"aftigungen der N"achsten lie"sen sich erkennen:
        man verl"angerte, man verk"urzte die Steigb"ugel, man reichte sich B"uchse
        und Patront"aschchen, man schob die Dachsranzen zurecht, indes die Hunde
        ungeduldig am Riemen den Zur"uckhaltenden mit fortzuschleppen drohten. Auch
        hie und da geb"ardete ein Pferd sich mutiger, von feuriger Natur getrieben
        oder von dem Sporn des Reiters angeregt, der selbst hier in der Halbhelle
        eine gewisse Eitelkeit, sich zu zeigen, nicht verleugnen konnte. Alle jedoch
        warteten auf den F"ursten, der, von seiner jungen Gemahlin Abschied nehmend,
        allzulange zauderte.
        
        Erst vor kurzer Zeit zusammen getraut, empfanden sie schon das Gl"uck
        "ubereinstimmender Gem"uter; beide waren von t"atig lebhaftem Charakter, eines
        nahm gern an des andern Neigungen und Bestrebungen Anteil. Des F"ursten Vater
        hatte noch den Zeitpunkt erlebt und genutzt, wo es deutlich wurde, da"s alle
        Staatsglieder in gleicher Betriebsamkeit ihre Tage zubringen, in gleichem
        Wirken und Schaffen jeder nach seiner Art erst gewinnen und dann genie"sen sollte.
        [\ldots]
        \rightline{\footnotesize aus \itshape Goethe: Novelle. DB Sonderband: Meisterwerke
                   deutscher Dichter und Denker, S. 12111 (vgl. Goethe-HA Bd. 6, S. 491)}
    \closing{Mit freundlichen Gr"u"sen,}
    \encl{\LaTeX-Vorlage f"ur Briefe im neuen Corporate Design der TU-Dresden}
\end{letter}

\end{document}
