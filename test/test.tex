\documentclass%
%[10pt]
%[cdoldfont]
%[parskip=full]
[headings=small]%
%[headings=normal]%
%[paper=A0,fontsize=42pt]
{tudscrreprt}
%{scrreprt}
%{article}
\usepackage[T1]{fontenc}
\usepackage{lmodern}
\usepackage[ngerman]{babel}
\usepackage{blindtext}
\usepackage{textcase}
%\usepackage{showframe}

\KOMAoptions{parskip=full}

\begin{document}

\providecommand{\TUDoptions}[1]{}
\providecommand{\setpartsubtitle}[1]{}
\providecommand{\setchaptersubtitle}[1]{}


\makeatletter

%\renewcommand*\tud@cs@letltx[2]{%
%  \expandafter\expandafter\expandafter\LetLtxMacro%
%  \expandafter\csname #1\expandafter\endcsname\csname #2\endcsname%
%}

%  \tud@cs@check{chapterheadmidvskip}{%
%    \ifcsdef{@@tud@chapterheadmidvskip}{}{%
%      \tud@cs@letltx{@@tud@chapterheadmidvskip}{chapterheadmidvskip}%
%    }%
%  }%

%\expandafter\long\def\chapterheadmidvskip{\par bla}

%\DeclareCommandCopy\@@tud@chapterheadmidvskip\foo


\makeatother

%\def\blindtext{AAA BBB}

%\KOMAoptions{headings=normal}
%\KOMAoptions{headings=small}

\renewcaptionname{ngerman}{\partname}{Kapitelteil}

%\meaning\partheadstartvskip
%
%\meaning\chapterheadstartvskip
%

%\renewcommand*\partheadstartvskip{}
%\renewcommand*\chapterheadstartvskip{}

\renewcommand*\partpagestyle{plain}
%\renewcommand*\chapterpagestyle{plain.tudscrheadings}

\subject{text}
\subtitle{Untertitel}
\title{A Titel eines Dokumentes}
\titlehead{-}

\begingroup
\KOMAoptions{titlepage=false}
\maketitle
\endgroup

\maketitle

%\renewcommand*{\chapterheadmidvskip}{\vskip-\parskip}%

%\begingroup
%\TUDoptions{parttitle}
%\part*{A Titel eines Teils special}
%%bla
%\endgroup


\setpartsubtitle{Untertitel}
\part{Ä Titel eines Teils}
\setpartsubtitle{Untertitel}
\part*{Ä Titel eines Teils}
\part{A Titel eines Teils}
\setpartpreamble[o]{Something stupid\par\blindtext}
\part*{A Titel eines Teils}
\setpartpreamble[o]{Something stupid\par\blindtext}
\part{A Titel eines Teils}

\setchapterpreamble[o]{Something stupid\par\blindtext}
\begingroup
\KOMAoptions{chapterprefix}
\chapter{Ä Titel eines Kapitelsa}
\chapter{A Titel eines Kapitelsa}
\endgroup

\setchapterpreamble[o]{Something stupid\par\blindtext}
\chapter*{A Titel eines Kapitels}
\blindtext

\setchapterpreamble[o]{Something stupid\par\blindtext}
\chapter{A Titel eines Kapitels}
\chapter*{Ä Titel eines Kapitels}
\setchaptersubtitle{Untertitel}
\chapter*{A Titel eines Kapitels}
\setchaptersubtitle{Untertitel}
\chapter{Ä Titel eines Kapitels}
\clearpage


\section{Ä Titel eines Abschnitts g, p, q und y}
A Titel eines Abschnitts
\clearpage
\section{A Titel eines Abschnitts}
A Titel eines Abschnitts
\clearpage
\section{Titel eines Abschnitts g, p, q und y}
Ä Titel eines Abschnitts
\clearpage
\section{Titel eines Abschnitts g, p, q und y}
A Titel eines Abschnitts
\clearpage

\subsection{Ä Titel eines Abschnitts g, p, q und y}
A Titel eines Abschnitts
\clearpage
\subsection{A Titel eines Abschnitts}
A Titel eines Abschnitts
\clearpage
\subsection{Titel eines Abschnitts g, p, q und y}
Ä Titel eines Abschnitts
\clearpage
\subsection{Titel eines Abschnitts g, p, q und y}
A Titel eines Abschnitts
\clearpage

\subsubsection{Ä Titel eines Abschnitts g, p, q und y}
A Titel eines Abschnitts
\clearpage
\subsubsection{A Titel eines Abschnitts}
A Titel eines Abschnitts
\clearpage
\subsubsection{Titel eines Abschnitts g, p, q und y}
Ä Titel eines Abschnitts
\clearpage
\subsubsection{Titel eines Abschnitts g, p, q und y}
A Titel eines Abschnitts
\clearpage

\minisec{Ä Titel einer Miniüberschrift}
Ä Titel einer Miniüberschrift
\clearpage
\minisec{A Titel einer Miniüberschrift}
A Titel einer Miniüberschrift
\clearpage
\minisec{Titel einer Miniüberschrift g}
Ä Titel einer Miniüberschrift
\clearpage
\minisec{Titel einer Miniüberschrift g}
A Titel einer Miniüberschrifttts
\clearpage

\end{document}

\begin{document}

\meaning\AtBeginEnvironment

\meaning\AtEndEnvironment

\meaning\BeforeBeginEnvironment

\meaning\AfterEndEnvironment

\newenvironment{Bla}{begdef}{enddef}

\AtBeginEnvironment{Bla}{1}
\AtEndEnvironment{Bla}{2}
\BeforeBeginEnvironment{Bla}{3}
\AfterEndEnvironment{Bla}{4}

\begin{Bla}
aaaa
\end{Bla}

\end{document}


\def\testmathoutput{
$\alpha \not= \bm{\alpha}$
\par
$1 g \bm{g}$
$2 \mathrm{g \bm{g}}$
$3 {g} \bm{{g}}$
$4 \mathrm{{g} \bm{{g}}}$
$5 \mathrm{g} \bm{\mathrm{g}}$
\par
\blindtext
\par
1234567890
\par
$\mathbf{1234567890}$
\par
\boldmath
\blindtext
\par
1234567890
\par
$\mathbf{1234567890}$
}

\testmathoutput

\clearpage

\TUDoption{cdfont}{heavy,oldstylefigures,upgreek}
\testmathoutput

\end{document}

\documentclass[
ngerman,
%cdoldfont,
%lgrgreeks,
cdfont=false,
%cdfont=heavy,
%cdfont=ultrabold
%cdmath=false,
]{scrreprt}
\iftutex
  \usepackage{fontspec}
\else
  \usepackage[T1]{fontenc}
  \input glyphtounicode.tex
  \pdfgentounicode=1
  \usepackage[ngerman=ngerman-x-latest]{hyphsubst}
\fi
\usepackage{babel}
%\usepackage[math]{blindtext}

\usepackage{tudscrfonts}
\usepackage{libertine}
\usepackage{textcomp}
\usepackage{fonttable}

\begin{document}



\meaning\textrightarrow

aaa \textrightarrow bbb $\rightarrow$





%\blindtext
\end{document}
