% use pdflatex or lualatex for compilation
\documentclass[ngerman,cdfont=light]{tudscrposter}
\TUDoptions{backcolor={rgb,9:red,4;green,2;yellow,1}}
\usepackage{iftex}
\iftutex
  \usepackage{fontspec}
\else
  \usepackage[T1]{fontenc}
%  \usepackage[latin1]{inputenc}
%  \input glyphtounicode.tex
%  \pdfgentounicode=1
%  \usepackage[ngerman=ngerman-x-latest]{hyphsubst}
\fi
\usepackage{babel}
\usepackage[math]{blindtext}
\begin{document}
\blindtext

\clearpage

%\TUDoptions{}
\blindtext
\end{document}


\def\testmathoutput{
$\alpha \not= \bm{\alpha}$
\par
$1 g \bm{g}$
$2 \mathrm{g \bm{g}}$
$3 {g} \bm{{g}}$
$4 \mathrm{{g} \bm{{g}}}$
$5 \mathrm{g} \bm{\mathrm{g}}$
\par
\blindtext
\par
1234567890
\par
$\mathbf{1234567890}$
\par
\boldmath
\blindtext
\par
1234567890
\par
$\mathbf{1234567890}$
}

\testmathoutput

\clearpage

\TUDoption{cdfont}{heavy,oldstylefigures,upgreek}
\testmathoutput

\end{document}

\documentclass[
ngerman,
%cdoldfont,
%lgrgreeks,
%cdfont=false,
%cdfont=heavy,
%cdfont=ultrabold
%cdmath=false,
]{scrreprt}
\iftutex
  \usepackage{fontspec}
\else
  \usepackage[T1]{fontenc}
  \input glyphtounicode.tex
  \pdfgentounicode=1
  \usepackage[ngerman=ngerman-x-latest]{hyphsubst}
\fi
\usepackage{babel}
\usepackage{bm}
\usepackage[math]{blindtext}

\begin{document}
\def\testmathoutput{
$\alpha \italpha \upalpha \Psi \itPsi \upPsi$\par
$\alpha \not= \bm{\alpha}$\par
$1 g \bm{g}$
$2 \mathrm{g \bm{g}}$
$3 {g} \bm{{g}}$
$4 \mathrm{{g} \bm{{g}}}$
$5 \mathrm{g} \bm{\mathrm{g}}$
\par
\blindtext
\par
1234567890
\par
$\mathbf{1234567890}$
\par
\boldmath
\blindtext
\par
1234567890
\par
$\mathbf{1234567890}$
}

\testmathoutput

\clearpage

\TUDoption{cdfont}{heavy,oldstylefigures,upgreek}
\testmathoutput

\clearpage

\blinddocument
\end{document}
