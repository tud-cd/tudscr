\documentclass[ngerman]{article}
\usepackage[LGR,TS1,OML,OT1,OMS,T1]{fontenc} % load the desired encodings
\usepackage{fonttable}
\PassOptionsToPackage{}{tudscrfonts}
\usepackage{tudscrfonts}
\usepackage{babel}
\usepackage[math]{blindtext}

\begin{document}

\blindtext

%\xfonttable{OT1}{lun}{l}{n}
\clearpage
\xfonttable{OT1}{fos}{l}{n}
\clearpage
\xfonttable{T1}{fos}{l}{n}

%
%\clearpage
%\xfonttable{OML}{cmbrm}{m}{it}
\clearpage
\xfonttable{TS1}{fos}{l}{sl}
%
%
\clearpage
\xfonttable{LGR}{fos}{l}{it}

\clearpage
\xfonttable{OMS}{fos}{m}{n}
\clearpage
\xfonttable{OMS}{cmbrs}{m}{n}

\clearpage
\xfonttable{OML}{fos}{m}{it}
\clearpage
\xfonttable{OML}{cmbrm}{m}{it}
\clearpage
\xfonttable{LGR}{fos}{l}{it}
%\clearpage
%\xfonttable{LGR}{fos}{sb}{it}
%\clearpage
%\xfonttable{LGR}{fos}{b}{it}
%\clearpage
%\xfonttable{LGR}{fos}{eb}{it}



\end{document}

%\RequirePackage{fix-cm}
%\documentclass[paper=a0,fontsize=22pt,cdfoot=5ex,ddcfoot]{tudscrposter}
%
%\usepackage[T1]{fontenc}
%\usepackage{lmodern}
%\usepackage[english]{babel}
%\usepackage{amssymb}
%\usepackage{multicol}
%\usepackage{blindtext}
%%\usepackage{fancyhdr}
%%\usepackage{setspace}
%%\pagestyle{fancy} %eigener Seitenstil
%%\fancyhf{}
%%\fancyhead[R]{\today}
%%\fancyhead[L]{Arne Hansen-Dörr}
%\begin{document}
%%%%%%%%%%%%%%%%%%%%%%%%%%%%%%%%%%%%%%%%%%%%%%%%%%%%%%%%%%%%%%%%%%%%
%\faculty{Faculty of Mechanical Science and Engineering}
%\institute{Institute of Solid Mechanics}
%\chair[]{Chair of Computational and Experimental Solid Mechanics}
%\date{18.02.2015}
%\title{%
%	Phase Field Model for Interface Failure
%}
%
%
%\renewcaptionname{english}{\contactpersonname}{Corresponding Author}
%\contactperson{%
%	Arne Claus Hansen-Dörr%
%	\office{ZEU 355}%
%	\telephone{463--33450}%
%	\emailaddress{arne.hansen-doerr@tu-dresden.de}%
%}
%\professor{Prof. Dr.-Ing. habil. Markus Kästner}
%\maketitle
%
%
%\begin{abstract}%[columns=2]
%	\noindent\blindtext
%%	The functionality of engineering structures, machines and so forth may be 
%%compromised by
%%	cracks. Hence, it is of great interest to predict the cracking behaviour, 
%%i.e. the maximum
%%	loading capacity, the location at which the crack nucleates and its 
%%propagating direction,
%%	with the ultimate aim to prevent cracking. \par 
%%	This work focuses on a phase field model for fracture. It introduces a 
%%crack 
%%regularisation also referred to as `smeared crack approach' and covers 
%%certain 
%%points such as crack nucleation and propagation. The phase field model yields 
%%qualitatively good results for homogeneous materials. However, it only 
%%accounts 
%%for cohesive failure which makes it difficult to incorporate an adhesive 
%%interface such as glue layers without any major modification. To rule out 
%%this 
%%drawback, the critical fracture toughness which is introduced by the model is 
%%spatially adapted in the vicinity of the interface. Since the spatial 
%%adaptation is limited to the resolution of the mesh, the formerly discrete 
%%interface is regularised, too, and a second length scale characterising the 
%%interface can be introduced. \par
%%	At the best, a crack propagating along the interface is not influenced by 
%%how 
%%the spatial adaptation of the interface looks like. However, it can be 
%%observed, that it is influenced by the ratio of both length scales. The wider 
%%the interface is compared to the crack regularisation, the smaller the 
%%influence of the surrounding material gets. For reliable studies concerning 
%%crack nucleation and growth, it is of great importance to quantify the 
%%interaction. An approach is discussed which is based on an artificial 
%%adaptation of the interface fracture toughness to balance out the influence 
%%of 
%%the surrounding material.
%\end{abstract}
%
%
%\end{document}
%
%
%\documentclass[ngerman]{tudscrreprt}
%\usepackage[T1]{fontenc}
%\usepackage{selinput}
%\SelectInputMappings{adieresis={ä},germandbls={ß}}
%\usepackage{babel}
%\usepackage{isodate}
%\begin{document}
%\faculty{Meine Fakultät}
%\institute{Mein Institut}
%
%\author{Das bin dann wohl ich}
%\dateofbirth{1.1.1990}
%\title{Der Titel meiner Arbeit}
%
%\subject{Wissenschaftliche Arbeit im Fach >>\dots<<}
%\thesis{master}
%\graduation{Lehramt an >>\dots<<}
%
%\referee{Prof.-Dr. Dings \and Prof.-Dr. Bumms}
%\renewcaptionname{ngerman}{\datetext}{}
%\date*{11.8.2017}
%
%\maketitle
%
%\begin{titlepage}
%%\thispagestyle{empty.tudheadings}
%
%\usekomafont{titlepage}
%\centering
%\vspace*{4cm}
%
%\textbf{\Large\csuse{@title}}
%\vspace*{2cm}
%
%Wissenschaftliche Arbeit im Fach >>…<<
%\vspace*{1cm}
%
%Lehramt an >>…<<
%\vspace*{1cm}
%
%eingereicht von:\\[.5ex]
%\textbf{\csuse{@author}}\\
%geboren am \csuse{@dateofbirth}
%\vspace*{1cm}
%
%Technische Universität Dresden\\
%\csuse{@faculty}\\
%\csuse{@institute}
%\vspace*{2cm}
%
%Gutachter\\[.5ex]
%Prof.-Dr. Dings\\
%Prof.-Dr. Bumms
%\vspace*{1cm}
%
%Dresden, \csuse{@date}
%
%\end{titlepage}
%\end{document}


\documentclass[%
cdfont=false,
%cdmath=true,
cdmath=false,cd=false
]{tudscrbook}
\usepackage[%
  detect-all=true,
%detect-display-math=true, 
%detect-family=true,
%%%detect-inline-family=text,
%%%detect-inline-weight=text,
%detect-mode=true,
%%detect-shape=false,
%%detect-weight=false,  
%%  detect-all=false,
%%  mode=text
%%  mode=math
]{siunitx}
%\usepackage{libertine}


  %\TUDoptions{cdfont=true, cdmath=false}	%also problematic
%  \providecommand{\num}[1]{#1}
\begin{document}

\title{text}
\author{names}
\thesis{thesis}
\date{\today}

\makecover[cdcover]

\maketitle[cd=false]

%\TUDoptions{cdfont=true, cdmath=false}	%problematic
%  \TUDoptions{}
%    \sisetup{number-math-rm = \mathtt,}
  \[
  12340 + \num{12340}
  \]
  12340 + \num{12340} + 
  \(12340 + \num{12340}\) 
  (First couple and last couple should look the same)
\end{document}

\documentclass[%
%  cd=no,
  chapterprefix,
  headings=optiontoheadandtoc,
  ngerman
]{tudscrreprt}
\usepackage[T1]{fontenc}
\usepackage{selinput}
\SelectInputMappings{adieresis={ä},germandbls={ß}}
\usepackage{lmodern}
\usepackage{babel}
\usepackage{blindtext}
%\geometry{paper=b1,layout=a1,layoutoffset=1in,showcrop}

\makeatletter
%\tud@font@koma@set{chapter}{%
%  \tud@sec@fontface%
%  \tud@sec@fontsize%
%  \tud@color{\tud@chapter@fontcolor}%
%}%
\providecommand*\tud@chapterheadstartvskip{}%
\renewcommand*\tud@chapterheadstartvskip{%
%  \renewcommand*\tud@chapter@fontcolor{}%
%  \ifcase\tud@layout@chapter@num\relax\or\else% *color
%    \renewcommand*\tud@chapter@fontcolor{HKS41}%
%  \fi%
%
%  \vspace{\@tempskipa}%
%  \vspace*{%
%    \dimexpr\tud@len@areavskip+\tud@len@areaheadvskip\relax%
%  }%
  \if@tud@chapterpage%
    \TUD@deprecated@lengthcs{pageheadingsvskip}%
    \vspace*{\tud@dim@pageheadingsvskip}%
  \else%
    \TUD@deprecated@lengthcs{headingsvskip}%
    \vspace*{\tud@dim@headingsvskip}%
  \fi%
}
%\CheckCommand*{\scr@@makechapterhead}[2]{}%
\providecommand*\tud@@makechapterhead{}%
\renewcommand*{\tud@@makechapterhead}[2]{%
  \@tempskipa=\glueexpr \csname scr@#1@beforeskip\endcsname\relax\relax
  \ifdim\@tempskipa<\z@\@tempskipa-\@tempskipa\else
    \expandafter\ifnum\scr@v@is@ge{3.22}\@afterindenttrue\fi
  \fi
%  \tud@chapterheadstartvskip%
  \chapterheadstartvskip%
  {%
    \setlength{\parindent}{\z@}\setlength{\parfillskip}{\z@ plus 1fil}%
    \def\IfUseNumber{\ifnumbered{#1}}%
    \if@chapterprefix
      \let\IfUsePrefixLine\@firstoftwo
    \else
      \let\IfUsePrefixLine\@secondoftwo
    \fi
    \normalfont\usekomafont{disposition}{%
      \usekomafont{#1}{%
        \settoheight{\@tempskipa}{%
          {\usekomafont{#1prefix}{%
            \vrule \@width\z@ \@height\csname scr@#1@innerskip\endcsname%
          }}%
        }%
        \ifdin{\tud@raggedright}{\raggedchapter}%
        \IfUseNumber{%
          \IfUsePrefixLine{%
            \chapterlineswithprefixformat{#1}{%
              {%
%              \tud@headmidvskip@reverse
                \usekomafont{#1prefix}{%
                                \vskip-\@tempskipa
                  \tud@makeuppercase{\csname #1format\endcsname}%
                  \setlength{\@tempskipa}{\csname scr@#1@innerskip\endcsname}%
                  \chapterheadmidvskip%
                }%
              }%
            }{\interlinepenalty \@M\tud@makeuppercase{#2}\@@par}%
          }{%
            \chapterlinesformat{#1}%
            {\csname #1format\endcsname}%
            {\interlinepenalty \@M\tud@makeuppercase{#2}\@@par}%
          }%
        }{%
          \IfUsePrefixLine{%
            \chapterlineswithprefixformat{#1}%
            {}%
            {\interlinepenalty \@M\tud@makeuppercase{#2}\@@par}%
          }{%
            \chapterlinesformat{#1}%
            {}%
            {\interlinepenalty \@M\tud@makeuppercase{#2}\@@par}%
          }%
        }%
      }%
    }%
  }%
  \nobreak\par\nobreak
  \@tempskipa=\glueexpr \csname scr@#1@afterskip\endcsname\relax\relax
  \ifdim\@tempskipa<\z@\@tempskipa-\@tempskipa\fi
  \chapterheadendvskip
}
%  \renewcommand*\tud@chapter@fontcolor{}%
%  \ifcase\tud@layout@chapter@num\relax\or\else% *color
%    \renewcommand*\tud@chapter@fontcolor{HKS41}%
%  \fi%
%  \vspace*{%
%    \dimexpr\tud@len@areavskip+\tud@len@areaheadvskip\relax%
%  }%
%  \if@tud@chapterpage%
%    \TUD@deprecated@lengthcs{pageheadingsvskip}%
%    \vspace*{\tud@dim@pageheadingsvskip}%
%  \else%
%    \TUD@deprecated@lengthcs{headingsvskip}%
%    \vspace*{\tud@dim@headingsvskip}%
%  \fi%
\ifx\TUDScript\@undefined\else%
\let\scr@@makechapterhead\tud@@makechapterhead
\fi
\newcommand*{\tud@@makeschapterhead}[2]{%
  \@tempskipa=\glueexpr \csname scr@#1@beforeskip\endcsname\relax\relax
  \ifdim\@tempskipa<\z@\@tempskipa-\@tempskipa\else
    \expandafter\ifnum\scr@v@is@ge{3.22}\@afterindenttrue\fi
  \fi
%  \tud@chapterheadstartvskip%
  \chapterheadstartvskip%
  {%
    \setlength{\parindent}{\z@}\setlength{\parfillskip}{\z@ plus 1fil}%
    \let\IfUseNumber\secondoftwo
    \if@chapterprefix
      \let\IfUsePrefixLine\@firstoftwo
    \else
      \let\IfUsePrefixLine\@secondoftwo
    \fi
    \normalfont\usekomafont{disposition}{%
      \usekomafont{#1}{%
        \ifdin{\tud@raggedright}{\raggedchapter}%
        \IfUsePrefixLine{%
          \chapterlineswithprefixformat{#1}%
          {}%
          {\interlinepenalty \@M\tud@makeuppercase{#2}\@@par}%
        }{%
          \chapterlinesformat{#1}%
          {}%
          {\interlinepenalty \@M\tud@makeuppercase{#2}\@@par}%
        }%
      }%
    }%
  }%
  \nobreak\par\nobreak
  \@tempskipa=\glueexpr \csname scr@#1@afterskip\endcsname\relax\relax
  \ifdim\@tempskipa<\z@\@tempskipa-\@tempskipa\fi
  \chapterheadendvskip
}
\ifx\TUDScript\@undefined\else%
\let\scr@@makeschapterhead\tud@@makeschapterhead
\fi
\begin{document}
%\renewcommand{\chapterpagestyle}{empty}
%\meaning\prefacename
%\def\prefacename{Vorwort}
\tableofcontents
\chapter{\prefacename 1}
\chapter*{\prefacename 2}
\addchap{\prefacename 3}
\addchap*{\prefacename 4}
\meaning\SecDef

\meaning\chapter
\smallskip

\meaning\scr@startchapter
\smallskip

\meaning\@schapter
\smallskip

\meaning\scr@@startchapter
\smallskip
%
%\cleardoubleemptypage
%
%\DeclareSectionCommand[%
%  style=chapter,%
%  level=0,%
%  pagestyle=empty,%
%  beforeskip=3.9em,
%  afterskip=2.05em,
%%  indent=0pt,
%  innerskip=.6em,
%  font=\Huge\dinbn,
%  prefixfont=\huge,
%  tocstyle=chapter,%
%  tocindent=0pt,%
%  tocnumwidth=1.5em%
%]{tudchapter}
%\tudchapter{bla}
%\meaning\tudchapter
%\smallskip
%
%\meaning\scr@startchapter
%\smallskip
%
%\meaning\scr@@startchapter
%\smallskip
%
%\meaning\scr@@startschapter
\end{document}

folgende Makros sind zu definieren (vererben oder neu)
tudchapterpagestyle