\listfiles
\documentclass[ngerman]{tudscrreprt}
\usepackage{iftex}
\iftutex
  \usepackage{fontspec}
\else
  \usepackage[T1]{fontenc}
  \usepackage[ngerman=ngerman-x-latest]{hyphsubst}
\fi
\usepackage{babel}

\begin{document}
Test
\end{document}

\faculty{Juristische Fakultät}
\department{Fachrichtung Strafrecht}
\institute{Institut für Kriminologie}
\chair{Lehrstuhl für Kriminalprognose}
\title{%
  Entwicklung eines optimalen Verfahrens zur Eroberung des
  Geldspeichers in Entenhausen
}
\thesis{master}
\graduation[M.Sc.]{Master of Science}
\author{%
  Mickey Mouse%
  \matriculationnumber{12345678}%
  \dateofbirth{2.1.1990}%
  \placeofbirth{Dresden}%
  \course{Klinische Prognostik}%
  \discipline{Individualprognose}%
}
\matriculationyear{2010}
\supervisor{Dagobert Duck \and Mac Moneysac}
\professor{Prof. Dr. Kater Karlo}
\date{10.09.2014}
\makecover
\maketitle


\tableofcontents
\listoffigures
\listoftables

\printacronyms[style=acrotabu]
\printsymbols[style=symblongtabu]


\chapter{Einleitung}
\section{Die Verwendung von Akronymen und Symbolen}
\newacronym{apsp}{APSP}{All-Pairs Shortest Path}

In der Graphentheorie wird häufig ... \gls{apsp} zum Einsatz.
\newformulasymbol{l}{Länge}{l}{m}
Die Einheiten \gls{l} ... lautet...

\section{Referenzen und das Literaturverzeichnis}
Das Literaturverzeichnis wird auf Basis der nachfolgend verwendeten
Zitate erstellt und ist auf Seite~\pageref{sec:bibliography} zu finden.
In diesem Textabschnitt werden die zwei bekannten \LaTeX-Bücher
\cite{knuth84} und \cite{goossens94} sowie das Anwenderhandbuch
\cite{hanisch14} zitiert.

\printbibliography[heading=bibintoc]\label{sec:bibliography}%

\end{document}

\listfiles
\documentclass[ngerman,cd=true,cdtitle=true]{tudscrartcl}
\usepackage{iftex}
\iftutex
  \usepackage{fontspec}
\else
  \usepackage[T1]{fontenc}
  \input glyphtounicode.tex
  \pdfgentounicode=1
  \usepackage[ngerman=ngerman-x-latest]{hyphsubst}
\fi
%\TUDoptions{fontspec=false}
\usepackage{babel}
\usepackage[math]{blindtext}

\begin{document}

\end{document}

% use pdflatex or lualatex for compilation
\documentclass[ngerman,cdfont=light]{scrreprt}
\iftutex
  \usepackage{fontspec}
\else
  \usepackage[T1]{fontenc}
  \input glyphtounicode.tex
  \pdfgentounicode=1
  \usepackage[ngerman=ngerman-x-latest]{hyphsubst}
\fi
\usepackage{babel}
\usepackage[math]{blindtext}

\usepackage[]{tudscrfonts}
\usepackage{bm}

\begin{document}
\def\testmathoutput{
$\alpha \not= \bm{\alpha}$
\par
$1 g \bm{g}$
$2 \mathrm{g \bm{g}}$
$3 {g} \bm{{g}}$
$4 \mathrm{{g} \bm{{g}}}$
$5 \mathrm{g} \bm{\mathrm{g}}$
\par
\blindtext
\par
1234567890
\par
$\mathbf{1234567890}$
\par
\boldmath
\blindtext
\par
1234567890
\par
$\mathbf{1234567890}$
}

\testmathoutput

\clearpage

\TUDoption{cdfont}{heavy,oldstylefigures,upgreek}
\testmathoutput

\end{document}

\documentclass[
ngerman,
%cdoldfont,
%lgrgreeks,
cdfont=false,
%cdfont=heavy,
%cdfont=ultrabold
%cdmath=false,
]{scrreprt}
\iftutex
  \usepackage{fontspec}
\else
  \usepackage[T1]{fontenc}
  \input glyphtounicode.tex
  \pdfgentounicode=1
  \usepackage[ngerman=ngerman-x-latest]{hyphsubst}
\fi
\usepackage{babel}
%\usepackage[math]{blindtext}

\usepackage{tudscrfonts}
\usepackage{libertine}
\usepackage{textcomp}
\usepackage{fonttable}

\begin{document}



\meaning\textrightarrow

aaa \textrightarrow bbb $\rightarrow$





%\blindtext
\end{document}
