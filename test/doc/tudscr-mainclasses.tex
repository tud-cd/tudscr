\chapter[%
  Die Hauptklassen \Class{tudscrbook}, \Class{tudscrreprt} und 
  \Class{tudscrartcl}%
]{%
  Die Hauptklassen%
  \tudmarkuplabel[sec:mainclasses]{%
    \Class{tudscrbook},\Class{tudscrreprt},\Class{tudscrartcl}%
  }%
}
\begin{DeclareEntity*}{\Class{tudscrbook}}
\begin{DeclareEntity*}{\Class{tudscrreprt}}
\begin{DeclareEntity*}{\Class{tudscrartcl}}
\index{Klassen|!}%
Es werden die drei neuen Hauptklassen
%
\begin{description}
\item \Class{tudscrbook}
\item \Class{tudscrreprt}
\item \Class{tudscrartcl}
\end{description}
%
eingeführt, welche auf den \KOMAScript-Klassen basieren und grundsätzlich alle 
deren bereitgestellte Optionen~-- beispielsweise \Option{parskip=\PMisc} für 
die Absatzeinstellungen oder \Option{BCOR=\PLength} zur Festlegung der 
Bindekorrektur~-- sowie Umgebungen und Befehle unterstützen. Zusätzlich zu den 
\KOMAScript-Klassen werden weitere Pakete~-- beispielsweise für die Schriften, 
den Satzspiegel oder die Definition von Logos und Farben~-- zwingend benötigt, 
welche unter \autoref{sec:packages:needed} aufgeführt sind und durch \TUDScript 
geladen werden.

Es sei hier abermals auf die \scrguide[Anwenderdokumentation von \KOMAScript] 
hingewiesen, viele der folgend beschriebenen Befehle und Optionen beziehen sich 
auf die darin vorgestellten Einstellungsmöglichkeiten. Die Anpassungen und 
Erweiterungen der \KOMAScript-Klassen an das \CD und die neu definierten 
beziehungsweise geänderten oder erweiterten Befehle und Optionen werden im 
Folgenden erläutert.

\begin{Declaration}
  {\Macro{TUDoptions|\MPName{Optionenliste}}}
\begin{Declaration}
  {\Macro{TUDoption|\MPName{Option}\MPName{Werteliste}}}
\printdeclarationlist[%
  \index{Optionen|!}%
]
Mit diesen beiden Befehlen besteht für die meisten der neuen Klassen- respektive
Paketoptionen die Möglichkeit, den Wert der Optionen noch nach dem Laden der 
Klasse beziehungsweise des Paketes zu ändern. Sie sind das Äquivalent zu den 
von \KOMAScript bereitgestellten Befehlen \Macro{KOMAoptions} sowie 
\Macro{KOMAoption}.

Man kann wahlweise mit der Anweisung \Macro{TUDoptions} die Werte mehrerer 
Optionen gleichzeitig ändern, wobei diese durch Kommata zu trennen sind. Dabei 
muss innerhalb der Optionenliste jede zu ändernde Option in der Form 
\PName{Option}\PValue{=}\PName{Wert} angegeben werden. Die meisten der 
\TUDScript-Optionen besitzen auch einen Standard- respektive Säumniswert. 
Werden diese in der Form \PName{Option} ohne die Angabe eines Wertes genutzt, 
so wird der jeweiligen Option einfach der vorgesehene Säumniswert zugewiesen.

Einige Optionen können gleichzeitig mehrere, sich ergänzende Werte besitzen. 
Für diese besteht die Möglichkeit, mit \Macro{TUDoption} einer einzelnen Option 
nacheinander eine Reihe von Werten zuzuweisen, wobei diese in der Werteliste 
durch Komma voneinander zu trennen sind.

Diese beiden Befehlen erlauben es im Bedarfsfall das Verhalten von einer Option 
oder mehreren Optionen im Dokument zu ändern. Werden diese Befehle in einer 
Umgebung oder Gruppe genutzt, bleiben die vorgenommenen Einstellungen innerhalb 
dieser lokal begrenzt.
\end{Declaration}
\end{Declaration}



\section{%
  Die Schriften des \CDs%
  \label{sec:fonts}%
  \index{Schriftart|?(}%
}
%
Das \TUDCD gibt die Verwendung der Schriftfamilie \OpenSans für den Fließtext 
vor, was in der Standardkonfiguration durch \TUDScript auch so umgesetzt wird. 
Da jedoch in längeren Texten die Verwendung einer Serifenschriften aus 
typografischer Sicht zu empfehlen ist, gibt es allerdings die Möglichkeit, die 
ursprünglich vorgesehenen Schriften des \CDs nicht zu laden und stattdessen die 
\hologo{LaTeX}"=Standardschriften beziehungsweise ein anderes Schriftpaket zu 
verwenden. Die Erläuterungen dazu sind in \autoref{sec:text} zu finden. Zur 
Verwendung der Schriften mit \hologo{LaTeX}"~Klassen, welche nicht zum 
\TUDScript-Bundle gehören, lässt sich das Paket \Package{tudscrfonts} laden.

\ChangedAt{v2.02;v2.06}%
Durch das \CD werden keine Schriften für den Mathematiksatz festgelegt. Das ist 
insbesondere für mathematische Abhandlungen als auch ingenieur- und 
naturwissenschaftliche Dokumente nicht tragbar. Im Mathematikmodus werden 
deshalb die lateinischen Lettern mithilfe des Paketes \Package{mathastext}
sowie die griechischen Lettern der \OpenSans genutzt. Zur Ergänzung kann für 
weitere mathematische Symbole das Paket \Package{mdsymbol} geladen werden.
\ToDo{hinweis zu mdsymbol raus}[v2.07]

Diese Einstellung lässt sich deaktivieren, wodurch sich die Standardschriften 
oder gegebenenfalls die eines zusätzlichen Paketes für den mathematischen Satz 
nutzen lassen. Die dafür relevanten Einstellungen werden in \autoref{sec:math} 
erläutert. Weiterhin sind ergänzende Hinweise zu einem typografisch sauberen
Mathematiksatz in \autoref{sec:tut} zu finden.


\minisec{%
  Die Schriftformate Type1 und OpenType%
  \index{Type1-Schriften}%
  \index{OpenType-Schriften}%
}
Das \TUDScript-Bundle unterstützt die Nutzung der Schriften des \CDs sowohl 
im Type1- als auch im OpenType-Format. Beide Formate werden durch das Paket 
\Package{opensans} bereitgestellt, wobei unabhängig vom genutzten 
Textsatzsystem die Schriften im Type1-Format geladen werden. Die Verwendung der 
Schriften im OpenType-Format ist auch beim Einsatz von \Format{LuaLaTeX} oder 
\Format{XeLaTeX} nicht zwingend erforderlich. Dennoch ist dies problemlos 
möglich, indem einfach das Paket \Package{fontspec} eingebunden wird.

Ein Verzicht auf die Type1-Schriften ist dennoch nicht ohne Weiteres möglich. 
Denn einerseits sind diese für die Kompilierung eines Dokumentes über den 
klassischen Prozesspfad via \Path{latex\,>\,dvips\,>\,ps2pdf}~-- wie es 
beispielsweise für die Erstellung von Grafiken mit \Package{pstricks} notwendig 
ist~-- erforderlich. Andererseits werden für den Mathematiksatz die Glyphen den 
Type1-Schriften entnommen. Eine Verwendung der OpenType-Schriften für den 
Mathematiksatz ist leider nicht ohne einen erheblichen Aufwand möglich, weshalb 
\TUDScript~-- insbesondere aufgrund des nicht vorhandenen Mehrwerts~-- darauf 
verzichtet.%
\index{Schriftart|?)}%



\subsection{%
  Schriften für den Textsatz%
  \label{sec:text}%
  \index{Schriftstärke}%
}
%
\begin{Declaration}
  {\Option{cdfont=\PMisc}}
  (true,normalbold,liningfigures)
\printdeclarationlist
%
Mit der Option \Option{cdfont} können durch den Anwender alle zentralen 
Schrifteinstellungen für die \TUDScript-Klassen vorgenommen werden. Dies 
betrifft die Schriften sowohl für den Fließtext als auch den Mathematiksatz.
Dabei lassen sich insbesondere unterschiedliche Kombinationen von normaler und 
fetter Schriftstärke für die \OpenSans einstellen. Zu Beginn des Dokumentes 
sind diese maßgeblich für die Definition der Mathematikschriften. Die 
Schriftstärke im charakteristischen Querbalken der Kopfzeile lässt sich mit 
dieser Option ebenfalls einstellen.

\begin{DeclareValues}
\itemval=false=
  Es werden die \hologo{LaTeX}"=Standardschriften und nicht die Hausschrift 
  des \CDs verwendet. Der Anwender kann beliebige Schriftpakete nutzen.%
  \footnote{%
    Für die Verwendung der klassischen \hologo{LaTeX}"~Schriften, ist das Paket 
    \Package{lmodern} sehr empfehlenswert.%
  }
\itemval*=true,light,lightfont=
  Die Hausschrift \OpenSans des \CDs wird genutzt. Die Überschriften der 
  obersten Gliederungsebenen bis einschließlich \Macro||{subsubsection} 
  werden in (normal"~)fetter oder extra"~fetter Schriftstärke 
  (\seeplain{\Option'page'{headings=\PMisc}}) gesetzt, darunter liegende Ebenen%
  \footnote{\Macro{paragraph}, \Macro{subparagraph}} 
  verwenden immer den (normal"~)fetten Schriftschnitt. Standardmäßig nutzt 
  dieser \textcdrn{Open~Sans~Regular} und kann mit \Option{cdfont=heavybold} 
  stärker eingestellt werden. Im Fließtext kommt \textcdln{Open~Sans~Light} zum 
  Einsatz.
\itemval=heavy,heavyfont=
  Diese Einstellung unterscheidet sich von \Option{cdfont=true} insoweit als 
  die Schriftstärke der Hausschrift erhöht wird. Der Fließtext des Dokumentes 
  wird in \textcdrn{Open~Sans~Regular} gesetzt. Der fette Schriftschnitt ist im 
  Normalfall auf \textcdsn{Open~Sans~Semi"~Bold} festgelegt.
\end{DeclareValues}

Die Stärke der fetten Schriften lässt sich mit folgenden Einstellungen anpassen.
\begin{DeclareValues}
\itemval=normalbold=[v2.05]
  Für die fette Schriftstärke wird \textcdrn{Open~Sans~Regular} respektive bei 
  stärkerer Grundschrift (\Option{cdfont=heavy}) \textcdsn{Open~Sans~Semi"~Bold}
  verwendet. Dies ist die Voreinstellung.
\itemval=heavybold,ultrabold=[v2.05]
  Die fetten Schriften werden stärker hervorgehoben. Es kommt 
  \textcdsn{Open~Sans~Semi"~Bold} respektive \textcdbn{Open~Sans~Bold} bei 
  erhöhter Schriftstärke (\Option{cdfont=heavy}) zum Einsatz.
\end{DeclareValues}

Die folgende Tabelle verdeutlicht die Möglichkeiten zur Kombination der 
Einstellungen:
\begin{center}
\footnotesize%
\setlength{\tabcolsep}{.5em}%
\begin{tabular}{ccccc}
  \toprule
  & \multicolumn{2}{c}{\Option{cdfont=normalbold}}
  & \multicolumn{2}{c}{\Option{cdfont=heavybold}}
  \tabularnewline
  & \Macro{mdseries} & \Macro{bfseries} & \Macro{mdseries} & \Macro{bfseries}
  \tabularnewline\midrule
  \Option{cdfont=true} 
    & \textcdln{Open~Sans~Light} & \textcdrn{Open~Sans~Regular}
    & \textcdln{Open~Sans~Light} & \textcdsn{Open~Sans~Semi"~Bold}
  \tabularnewline\midrule
  \Option{cdfont=heavy}
    & \textcdrn{Open~Sans~Regular} & \textcdsn{Open~Sans~Semi"~Bold}
    & \textcdrn{Open~Sans~Regular} & \textcdbn{Open~Sans~Bold}
  \tabularnewline\bottomrule
\end{tabular}
\end{center}

\medskip\noindent%
\index{Ziffernform}%
Weiterhin lässt sich mit Versalziffern oder Mediävalziffern die Gestalt der 
Ziffern anpassen.
\begin{DeclareValues}
\itemval=liningfigures,normalfigures=[v2.06]
  Mit dieser Einstellung kommen dicktengleiche/äquidistant/proportionale 
  Versalziffern oder auch Majuskelziffern (\textbf{1234567890}) zum Einsatz.
  Mit \Macro{oldstylenums|\MPName{Ziffer(n)}} lassen sich Mediävalziffern oder 
  auch Minuskelziffern explizit setzen. Dieses Verhalten ist standardmäßig 
  aktiviert.
\itemval=oldstylefigures,osf,oldnumbers=[v2.06]
  Für den Fließtext werden nichtproportionale Mediävalziffern oder auch 
  Minuskelziffern (\textbf{\oldstylenums{0123456789}}) standardmäßig verwendet.
\end{DeclareValues}

Außerdem dient die Option \Option{cdfont=\PMisc} als erweiterte Schnittstelle 
für den Anwender, um zusätzliche Einstellungen für die Schriftnutzung vornehmen 
zu können, für welche normalerweise separate Optionen vorgesehen sind. Diese 
Möglichkeiten werden folgend kurz erläutert und auf die tatsächlich zugrunde 
liegende Option verweisen.

Die genutzten Mathematikschriften können mit folgenden Werten beeinflusst 
werden, wobei diese als Alias für die Optionen \Option||{cdmath} respektive 
\Option||{slantedgreek} aus \autoref{sec:math} fungieren.
\begin{DeclareValues}
\itemval=nomath,nocdmath=<\Option{cdmath=false}>
  Diese Einstellung deaktiviert die \OpenSans für den mathematischen Satz. Es 
  werden die \hologo{LaTeX}"=Standardschriften verwendet und es lassen sich  
  beliebige Schriftpakete für den Mathematikmodus nutzen.
\itemval=math,cdmath=<\Option{cdmath=true}>
  Die \OpenSans wird für die Mathematikschriften genutzt.
\itemval=upgreek,uprightgreek=<\Option{slantedgreek=false}>
  Griechische Majuskeln und Minuskeln sind aufrecht.
\itemval=slgreek,slantedgreek=<\Option{slantedgreek=true}>
  Die Ausgabe der griechischen Lettern erfolgt kursiv.
\itemval=texgreek,standardgreek=<\Option{slantedgreek=standard}>[v2.06]
  Es kommt der \hologo{LaTeX}"~Standard zum Einsatz, bei dem griechische 
  Majuskeln aufrecht und Minuskeln kursiv ausgegeben werden.
\end{DeclareValues}
%
Für die \TUDScript-Klassen gibt es bei den \PageStyle{tudheadings}"=Seitenstilen
folgende Einstellmöglichkeiten für die Schriftart im Querbalken, welche auch 
über die Option \Option||'fullref*'{cdhead} gesetzt werden können:
\begin{DeclareValues}
\itemval=nohead,noheadfont=<\Option{cdhead=false}>[v2.02]
  Bei deaktivierter Hausschrift für den Fließtext können diese ebenfalls im 
  Querbalken deaktiviert werden.
\itemval=head,lighthead,lightfonthead=<\Option{cdhead=true}>[v2.02]
  Für den Querbalken der Kopfzeile wird~-- unabhängig von der Verwendung der 
  Hausschrift im Fließtext~-- \OpenSans in normaler Schriftstärke verwendet.
\itemval=heavyhead,heavyfonthead=<\Option{cdhead=heavy}>[v2.02]
  Im Querbalken wird~-- unabhängig von der Dokumentschrift~-- \OpenSans in 
  erhöhter Stärke genutzt.
\end{DeclareValues}
\end{Declaration}

\begin{Declaration}
  {\Option{ttfont=\PMisc}}
  (true)
  [v2.06]
\printdeclarationlist[%
  \index{Schriftart}%
  \index{Schriftstärke}
]
Bei aktivierter Option \Option{cdfont} lässt sich zusätzlich die verwendete
\texttt{Schreibmaschinenschrift} einstellen, wofür selbige aus einem der Pakete 
\Package{roboto-mono} oder \Package{lmodern} genutzt wird und insbesondere die 
gewählten Einstellungen für die Schriftstärke der Option \Option{cdfont} 
beachtet werden. Ein direktes Laden des jeweiligen Paketes durch den Benutzer 
ist nicht notwendig.
\begin{DeclareValues}
\itemval=false=
  Es findet keine Anpassung Schreibmaschinenschrift (\Macro{ttfamily}) statt, 
  mögliche Änderungen können durch den Anwender erfolgen.
\itemval*=true,roboto,roboto-mono=
  Sollte das Paket \Package{roboto-mono} installiert sein, wird die darin 
  definierte Schreibmaschinenschrift genutzt. Diese ist serifenlos und liegt in 
  einer Vielzahl unterschiedlicher Schriftschnitte vor, weshalb sich diese 
  ideal mit der \OpenSans kombinieren lässt.
\itemval=lmodern,lmtt,lm=
  Es wird explizit die Schreibmaschinenschrift aus \Package{lmodern} aktiviert. 
\end{DeclareValues}
\end{Declaration}



\minisec{%
  Auszeichnungen im Text%
  \index{Schriftstärke}%
  \index{Schriftauszeichnung}%
}
Unabhängig davon, welche Schriftfamilie im Dokument verwendet wird, können die 
Schriften des \CDs jederzeit mit einem der hier aufgeführten Textschalter oder 
Textkommandos innerhalb des Dokumentes genutzt werden. Ein Textschalter wirkt 
sich~-- wenn er nicht lokal durch eine Gruppe oder Umgebung begrenzt wird~-- 
global auf das Dokument aus, wie etwa beispielsweise \Macro*{bfseries}. Bei 
einem Textkommando hingegen kommt die Schriftart nur für das nachfolgend 
angegebene Argument zum Einsatz, wie bei \Macro*{textbf|\MPName{Text}}. 
%
\begin{Declaration}
  {\Macro{cdfont|\MPName{Schriftart}}}
  [v2.04]
\begin{Declaration}
  {\Macro{textcd|\MPName{Schriftart}\MPName{Text}}}
  [v2.04]
\printdeclarationlist
%
Diese beiden Befehle dienen zur gezielten Aktivierung einer Schriftart des \CDs 
in Stärke und Schnitt. Hierbei entspricht \Macro{cdfont} einem Textschalter und 
ändert die verwendete Schriftart unverzüglich im aktuellen Geltungsbereich auf 
\PName{Schriftart}, wohingegen \Macro{textcd} als Textkommando fungiert und den 
im zweiten Argument gegebenen \PName{Text} in \PName{Schriftart} setzt, ohne 
dabei die Dokumentschriftart selbst zu ändern.

Für die Schriftauswahl muss im ersten Argument die Bezeichnung der gewünschten 
Schriftart angegeben werden. Mögliche Werte sind der nachfolgenden Tabelle zu 
entnehmen. Der Vorsatz \PValue{Open Sans} für die \PName{Schriftart} im 
Argument ist dabei optional. Ebenso sind weder Leerzeichen noch die passende 
Groß- und Kleinschreibung notwendig. Für die Wahl der Schriftstärke ist allein 
die Bezeichnung \PValue{Light/Regular/Semi-Bold/Bold/Extra-Bold} ausreichend. 
Anstelle des Suffix' \PValue{Italic} ist auch die Nutzung von \PValue{Oblique} 
oder \PValue{Slanted} als Alias für die kursiven Schriftlagen möglich.
%
\begin{center}%
  \newcommand*\listfonts[2]{%
    \csuse{textcd#2}{Open Sans #1} & \InlineDeclaration{\Macro*{cdfont#2}} & 
    \InlineDeclaration{\Macro*{textcd#2|\MPName{Text}}}\tabularnewline%
  }%
  \begin{tabular}{lll}%
    \toprule%
    \textbf{Schriftart/Bezeichnung} & \textbf{Schalter} & \textbf{Textkommando} 
    \tabularnewline
    \midrule
    \listfonts{Light}{ln}
    \listfonts{Regular}{rn}
    \listfonts{Semi-Bold}{sn}
    \listfonts{Bold}{bn}
    \listfonts{Extra-Bold}{xn}
    \listfonts{Light Italic}{li}
    \listfonts{Regular Italic}{ri}
    \listfonts{Semi-Bold Italic}{si}
    \listfonts{Bold Italic}{bi}
    \listfonts{Extra-Bold Italic}{xi}
    \bottomrule%
  \end{tabular}%
\end{center}%
%
Alternativ zu \Macro{cdfont} und \Macro{textcd} werden für jeden Schriftschnitt 
auch explizite Textschalter und "~kommandos bereitgestellt, bei denen das 
Argument für die Bezeichnung der Schriftart entfällt. Diese sind in der zweiten 
und dritten Spalte der Tabelle aufgeführt.
\end{Declaration}
\end{Declaration}



\subsection{%
  Schriften für den Mathematiksatz%
  \label{sec:math}%
  \index{Mathematiksatz|?(}%
  \ChangedAt*{%
    v2.07:Glyphen im Mathematiksatz werden passend zu den mit der Option 
      \Option{cdfont} gewählten normalen und fetten Schriftstärken gesetzt%
  }%
}
%
\begin{Declaration}
  {\Option{cdmath=\PBoolean}}
  (true|\Option{cdfont=false}:false)
  [v2.03]
\printdeclarationlist
%
Diese Option dient zur Anpassung der Mathematikschriften. Wird diese aktiviert, 
so werden zur Hausschrift passende Glyphen im Mathematikmodus genutzt. Normale 
sowie fette Schriftstärke werden zu \emph{Beginn des Dokuments} abhängig von 
der zu diesem Zeitpunkt aktiven Einstellung für die Schriften des Fließtexts 
(\seeplain{\Option'page'{cdfont=\PMisc}}) definiert. Fette Mathematikschriften 
können im Dokument mit \Macro{boldmath} aktiviert werden, eine Anpassung der 
Einstellungen ist durch \Macro{TUDoptions} möglich. Gültige Werte für die 
Option \Option{cdmath} sind:
\begin{DeclareValues}
\itemval=false=
  Es werden die normalen \hologo{LaTeX}"=Serifenschriften beziehungsweise die 
  Schriften beliebig nutzbarer Pakete für den Mathematiksatz verwendet.
\itemval*=true=
  Im Mathematikmodus kommt \OpenSans sowohl für lateinische als auch 
  griechische Lettern zum Einsatz.
\end{DeclareValues}
\end{Declaration}

\begin{Declaration}
  {\Option{slantedgreek=\PMisc}}
  (true)
\printdeclarationlist[%
  \index{Griechische Lettern|(}
]
Die Option ändert die standardmäßige Neigung der griechischen Majuskeln und 
Minuskeln (Groß- und Kleinbuchstaben) im Mathematikmodus bei der Verwendung 
der Standardbefehle für griechische Lettern.
\begin{DeclareValues}
\itemval=false=
  Die griechischen Lettern werden im Mathematiksatz allesamt aufrecht gesetzt.
\itemval*=true=
  Alle griechischen Lettern werden im Mathematikmodus kursiv ausgegeben.
\itemval=standard,latex=[v2.06]
  Die Ausgabe entspricht mit aufrechten Majuskeln und kursiven Minuskeln der 
  griechischen Lettern dem \hologo{LaTeX}"~Standard beim mathematischen Satz.
\end{DeclareValues}
\end{Declaration}

\ToDo{Deklaration mit Liste und Declaration'}[v2.07]
\begin{Declaration}{\Macro*{Gamma}}
\begin{Declaration}{\Macro*{Delta}}
\begin{Declaration}{\Macro*{Theta}}
\begin{Declaration}{\Macro*{Lambda}}
\begin{Declaration}{\Macro*{Xi}}
\begin{Declaration}{\Macro*{Pi}}
\begin{Declaration}{\Macro*{Sigma}}
\begin{Declaration}{\Macro*{Upsilon}}
\begin{Declaration}{\Macro*{Phi}}
\begin{Declaration}{\Macro*{Psi}}
\begin{Declaration}{\Macro*{Omega}}
\begin{Declaration}{\Macro*{alpha}}
\begin{Declaration}{\Macro*{beta}}
\begin{Declaration}{\Macro*{gamma}}
\begin{Declaration}{\Macro*{delta}}
\begin{Declaration}{\Macro*{epsilon}}
\begin{Declaration}{\Macro*{varepsilon}}
\begin{Declaration}{\Macro*{zeta}}
\begin{Declaration}{\Macro*{eta}}
\begin{Declaration}{\Macro*{theta}}
\begin{Declaration}{\Macro*{vartheta}}
\begin{Declaration}{\Macro*{iota}}
\begin{Declaration}{\Macro*{kappa}}
\begin{Declaration}{\Macro*{lambda}}
\begin{Declaration}{\Macro*{mu}}
\begin{Declaration}{\Macro*{nu}}
\begin{Declaration}{\Macro*{xi}}
\begin{Declaration}{\Macro*{pi}}
\begin{Declaration}{\Macro*{varpi}}
\begin{Declaration}{\Macro*{rho}}
\begin{Declaration}{\Macro*{varrho}}
\begin{Declaration}{\Macro*{sigma}}
\begin{Declaration}{\Macro*{varsigma}}
\begin{Declaration}{\Macro*{tau}}
\begin{Declaration}{\Macro*{upsilon}}
\begin{Declaration}{\Macro*{phi}}
\begin{Declaration}{\Macro*{varphi}}
\begin{Declaration}{\Macro*{chi}}
\begin{Declaration}{\Macro*{psi}}
\begin{Declaration}{\Macro*{omega}}
\printdeclarationlist(%
  \small%
  \newcommand\tablecontent{}%
  \newcommand*\greekLetters{%
    Gamma,Delta,Theta,Lambda,Xi,Pi,Sigma,Upsilon,Phi,Psi,Omega,%
    alpha,beta,gamma,delta,epsilon,varepsilon,zeta,eta,theta,vartheta,%
    iota,kappa,lambda,mu,nu,xi,pi,varpi,rho,varrho,sigma,varsigma,tau,%
    upsilon,phi,varphi,chi,psi,omega%
  }%
  \def\do#1{%
    \appto\tablecontent{%
      \InlineDeclaration{\Macro*{up#1}} & $\csuse{up#1}$ & & 
      \InlineDeclaration{\Macro*{it#1}} & $\csuse{it#1}$\tabularnewline
    }%
  }%
  \expandafter\docsvlist\expandafter{\greekLetters}%
  \centering%
  \begin{tabularm}{5}
    \toprule%
    \textbf{Befehl (aufrecht)} & \textbf{Symbol} & &
    \textbf{Befehl (kursiv)} & \textbf{Symbol}
    \tabularnewline\midrule\tablecontent\bottomrule%
  \end{tabularm}
)
%
Unabhängig von der Option \Option{slantedgreek} können sowohl kursive als auch 
aufrechte griechischen Lettern im Mathematikmodus mit diesen Befehlen direkt 
verwendet werden. Dies ist nützlich, um zwischen \emph{kursiven Variablen} und 
\emph{aufrechten Konstanten} zu unterscheiden. Weiterhin kann anstelle von 
\Macro*{up\PName{Letter}} oder \Macro*{it\PName{Letter}} auch der Präfix 
\Macro*{other\PName{Letter}} genutzt werden. Damit wird~-- abhängig von der 
Option \Option{slantedgreek}~-- die komplementäre Schriftlage der angegebenen 
Letter gesetzt.%
\index{Griechische Lettern|)}%
\end{Declaration}
\end{Declaration}
\end{Declaration}
\end{Declaration}
\end{Declaration}
\end{Declaration}
\end{Declaration}
\end{Declaration}
\end{Declaration}
\end{Declaration}
\end{Declaration}
\end{Declaration}
\end{Declaration}
\end{Declaration}
\end{Declaration}
\end{Declaration}
\end{Declaration}
\end{Declaration}
\end{Declaration}
\end{Declaration}
\end{Declaration}
\end{Declaration}
\end{Declaration}
\end{Declaration}
\end{Declaration}
\end{Declaration}
\end{Declaration}
\end{Declaration}
\end{Declaration}
\end{Declaration}
\end{Declaration}
\end{Declaration}
\end{Declaration}
\end{Declaration}
\end{Declaration}
\end{Declaration}
\end{Declaration}
\end{Declaration}
\end{Declaration}
\end{Declaration}



\minisec{Zusätzliche Hinweise zum Mathematiksatz}
%
\ToDo{hinweis zu mdsymbol raus}[v2.07]%
Unter Umständen werden zusätzliche Symbole für den Mathematiksatz benötigt. 
Sehr oft kommt hierfür das Paket \Package{amssymb} zum Einsatz. Dieses stellt 
zahlreiche zusätzliche für die \hologo{LaTeX}"=Standardschriften zur Verfügung. 
Diese sind in Verbindung mit der \OpenSans aus typografischer Sicht jedoch 
keine ideale Lösung. Die Symbole aus dem Paket \Package{mdsymbol} passen 
wesentlich besser zu besagter Schriftfamilie, weshalb diesem Paket der Vorzug 
gegeben werden sollte, falls die Schriften des \CDs für den mathematischen Satz 
genutzt werden. Weitere Hinweise zum typografisch guten Mathematiksatz sind 
außerdem in \autoref{sec:tut} zu finden.
\index{Mathematiksatz|?)}%



\subsection{%
  Abstände für vertikalen Leerraum in Abhängigkeit der Schriftgröße%
  \index{Längen}%
  \index{Leerraum}%
  \index{Schriftgröße}%
}
%
\ChangedAt{v2.05:Abstände für vertikalen Leerraum abhängig von Schriftgröße}%
Bei den \TUDScript-Klassen sind im Normalfall mehrere Längen von der für das 
Dokument gewählten Schriftgröße abhängig~-- im Gegensatz zu \KOMAScript. Dies 
hat den großen Vorteil, dass bei einer Änderung der Schriftgröße die folgend 
genannten Längen nicht zwingend separat durch den Anwender anzupassen sind, um 
weiterhin sinnvoll verwendet werden zu können. Da es sich dabei allerdings 
lediglich um generische Werte handelt, kann es für einen guten Satz dennoch 
sinnvoll sein, diese Längen \emph{nach} einer Änderung der Schriftgröße manuell 
zu adaptieren.

Eine automatische Anpassung an die Schriftgröße erfolgt für sowohl die 
dehnbaren Längen \Length{bigskipamount}, \Length{medskipamount} und 
\Length{smallskipamount}, welche unter anderem von den Befehlen 
\Macro{bigskip}, \Macro{medskip} beziehungsweise \Macro{smallskip} für das 
Einfügen vertikaler Abstände genutzt werden, als auch die beiden Längen 
\Length{abovecaptionskip} und \Length{belowcaptionskip}, welche den vertikalen 
Abstand vor und nach einer, für ein Gleitobjekt~-- beispielsweise Abbildung 
oder Tabelle~-- mit \Macro{caption} gesetzten Beschreibung bestimmen. Außerdem 
wird die Länge \Length{columnsep} als Maß für den horizontalen Abstand der 
einzelnen Textspalten im zwei- oder mehrspaltigen Layout, wie es beispielsweise 
mit dem Paket \Package{multicol} erzeugt werden kann, in Relation zur 
Schriftgröße sinnvoll festgelegt.

\ToDo{Option fontsize eigene Deklaration?}
Die verwendete Schriftgröße kann durch den Anwender über die \KOMAScript-Option 
\InlineDeclaration{\Option(*){fontsize=\PLength}} festgelegt werden.
\Attention{%
  Dabei ist zu beachten, dass diese immer als Klassenoption angegeben werden 
  sollte.%
}
Weitere Hinweise zur Wahl der passenden Schriftgröße sind außerdem in 
\autoref{sec:fontsize} zu finden.

\begin{Declaration}
  {\Option{relspacing=\PBoolean}}
  (true)
  [v2.05]
\printdeclarationlist
%
Mit der Option \Option{relspacing=\PBoolean} lässt sich die zuvor beschriebene 
Schriftgrößenabhängigkeit sowohl für vertikalen Leerraum zwischen zwei Absätzen 
oder bei Beschriftungen als auch für den horizontalen Abstand zwischen den 
Textspalten im mehrspaltigen Layout anpassen.
\begin{DeclareValues}
\itemval=false=
  Die besagten Längen werden nicht angepasst, passende Werte sollten bei einer 
  Änderung der Schriftgröße in jedem Fall durch den Anwender gewählt werden.
\itemval*=true=
  In Abhängigkeit von der gewählten Schriftgröße werden die zuvor genannten 
  Längen automatisch festgelegt.
\end{DeclareValues}
\end{Declaration}



\section{%
  Das Layout des \CDs%
  \index{Layout|(}%
}
\subsection{%
  Der Satzspiegel%
  \index{Satzspiegel|(}%
}
\subsubsection{Grundlegende Einstellungen für den Satzspiegel}
%
\begin{Declaration}
  {\Option(*){DIV=\PMisc}}
  (default)
\begin{Declaration}
  {\Option(*){paper=\PMisc}}
  (A4)
\begin{Declaration}
  {\Option(*){twoside=\PMisc}}=true,false,semi=
  (false|\Class{tudscrbook}:true)
\begin{Declaration}
  {\Option(*){twocolumn=\PBoolean}}
  (false)
\printdeclarationlist[%
  \index{Satzspiegel}%
  \index{Papierformat}%
]
Die Einstellung des Satzspiegels erfolgt durch \KOMAScript normalerweise mit 
dem Paket \Package{typearea}, wobei insbesondere die Klassenoption 
\Option{DIV=\PMisc} maßgeblichen Einfluss auf diesen hat. Mit 
\Option{paper=\PMisc} lässt sich das Papierformat und dessen Orientierung 
festlegen. Mit \Option{twocolumn=\PBoolean} wird das Dokument ein- oder 
zweispaltig gesetzt. Der doppelseitige Satz kann mit \Option{twoside=\PMisc} 
(de-)aktiviert werden. Ist dieser aktiv, sind weiterhin die Hinweise in 
\autoref{sec:vacat} von Relevanz. Detaillierte Anwendungshinweise für die 
genannten Optionen sind im \scrguide zu finden.
\end{Declaration}
\end{Declaration}
\end{Declaration}
\end{Declaration}



\subsubsection{Der Satzspiegel im \TUDCD}
%
\begin{Declaration}
  {\Option{cdgeometry=\PMisc}}
  (true,restricted|\Option{cd=false}:false)
  [v2.03]
\printdeclarationlist[%
  \index{Layout!Seitenstile}%
  \index{Satzspiegel!doppelseitig}%
]
Diese Option ist für die Wahl des Satzspiegels verantwortlich. Das Maß der 
Seitenränder ist im \CD fest vorgegeben und wird standardmäßig von den 
\TUDScript-Klassen eingehalten. Allerdings lassen sich die Seitenränder 
anpassen, um beispielsweise einen vernünftigen doppelseitigen Satz zu 
ermöglichen.%
\footnote{Hierbei sollte der innere Rand schmaler als der äußere sein}
Deshalb besteht die Möglichkeit, entweder auf das Standardverhalten von 
\KOMAScript zurückzufallen und den Satzspiegel durch das Paket 
\Package{typearea} zu berechnen oder aber diesen (fast) beliebig manuell 
festzulegen.
\begin{DeclareValues}
\itemval=false=
  Die Satzspiegelberechnung erfolgt via \Package{typearea}, die Vorgaben des 
  \CDs bezüglich der Seitenränder werden ignoriert.
  \Attention{Nur in diesem Fall wirkt die Option \Option{DIV}.}
\itemval*=true,asymmetric,cd=%
    <links:\,30\,mm, rechts:\,20\,mm, oben:\,20\,mm, unten:\,30\,mm>
  Die Seitenränder werden im asymmetrischen Stil des \CDs fest definiert und 
  auch für den doppelseitigen Satz (Klassenoption \Option{twoside=true}) 
  genutzt.%
\itemval=symmetric,centred,centered=%
    <links:\,25\,mm, rechts:\,25\,mm, oben:\,20\,mm, unten:\,30\,mm>
  Der Satzspiegel wird im einseitigen sowie doppelseitigen Satz auf der Seite 
  zentriert.%
\itemval=twoside,balanced=%
    <innen:\,20\,mm, außen:\,30\,mm, oben:\,20\,mm, unten:\,30\,mm>
  Diese Einstellung aktiviert den doppelseitigen Satz (\Option{twoside=true}) 
  und ändert den Satzspiegel derart, dass die Ränder der inneren Seiten 
  schmaler sind als die der äußeren.%
  \Attention{%
    Der so erzeugte Satzspiegel ist jedoch unvorteilhaft, da das Logo der \TnUD 
    sehr nah am inneren Seitenrand des Dokumentes gesetzt wird und folglich auf 
    rechten respektive ungeraden Seiten sehr weit an den Blattrand rückt.
  }
  Dieses Problem kann~-- bei \Class{tudscrbook} sowie \Class{tudscrreprt}~-- 
  prinzipiell gelöst werden, indem Titel, Teile und Kapitel über das Aktivieren 
  der \KOMAScript-Option \Option{open=left} immer auf einer linken Seite 
  beginnen, was allerdings aus typografischer Sicht eher unüblich ist.
\end{DeclareValues}

\ChangedAt*{v2.05:Neue Möglichkeiten bei der Satzspiegelberechnung im \CD}%
Mit den folgenden Werten lässt sich einstellen, in welcher Variante der 
Satzspiegel nach dem \TUDCD erstellt werden soll. 
\begin{DeclareValues}
\itemval=restricted=[v2.05]
  Der Satzspiegel entspricht den expliziten Vorgaben des \CDs.
\itemval=adapted=[v2.05]
  Laut dem Handbuch zum \CD werden für Papierformate zwischen DIN~A6 und DIN~A4 
  \enquote{im Interesse größter Einheitlichkeit die Maßverhältnisse über einen 
  größeren Formatbereich hinweg \enquote{eingefroren}.} Dies kann jedoch zu 
  schlecht nutzbaren Satzspiegeln führen. Mit dieser Einstellung kann die 
  Fixierung deaktiviert und der äquivalente Satzspiegel beispielsweise für das 
  Format~DIN~A5 aktiviert werden. Bei Formaten außerhalb des fixierten Bereichs 
  hat diese Einstellung keinen Einfluss. 
\itemval=calculated=[v2.05]
  Der Satzspiegel wird anhand der Referenzmaße für das Format~DIN~A4 für das 
  eingestellte Papierformat \emph{skaliert}. Die eigentlich definierten 
  diskreten Maße bei unterschiedlichen Gestaltungshöhen werden ignoriert. Es 
  wird empfohlen, diese Einstellung lediglich zu verwenden, wenn ein 
  Papierformat gewählt wird, welches nicht durch die Gestaltungshöhen des 
  \TUDCDs abgedeckt wird. Diese sind innerhalb der Papierhöhen von DIN~A6 bis 
  DIN~A0 definiert.
\end{DeclareValues}

Da es häufig sehr restriktive~-- wenn auch meistens völlig willkürliche~-- 
Vorgaben für die Seitenränder gibt, besteht außerdem die Möglichkeit, diese 
weitestgehend manuell einzustellen. 
\begin{DeclareValues}
\itemval=custom=[v2.05]
  Für die Festlegung der Seitenränder ist das Paket \Package{geometry} 
  maßgeblich verantwortlich. Der Satzspiegel lässt sich mit den beiden, durch 
  das Paket bereitgestellten Befehlen \Macro{geometry} und \Macro{newgeometry} 
  definieren. Für Hinweise zur weiteren Verwendung ist die Dokumentation des 
  Paketes zu konsultieren.
\end{DeclareValues}
\end{Declaration}


\ToDo[doc]{normale Deklaration (Tabelle) für Bindekorrektur (BCOR)}
\ToDo[doc]{verschieben in Grundlegende Einstellungen für den Satzspiegel?}
\subsubsection{%
  Bindekorrektur%
  \index{Satzspiegel!Bindekorrektur|!}%
}
%
Im Zusammenhang mit den Seitenrändern oder besser dem Satzspiegel ist die durch 
das Paket \Package{typearea} zur Verfügung gesellte \KOMAScript-Option 
\InlineDeclaration{\Option(*){BCOR=\PLength}} zu erwähnen. Mit dieser kann bei 
der Satzspiegelberechnung ein Heftrand respektive eine Bindekorrektur 
berücksichtigt werden. Durch die \TUDScript-Klassen wird der mit dieser Option 
angegebene Wert auch an das Paket \Package{geometry} weitergereicht, sodass der 
Benutzer unabhängig von der Satzspiegelgestaltung (\Option{cdgeometry}) die 
Option \Option{BCOR} nutzen kann. So lässt sich eine Bindekorrektur von 
beispielsweise \unit[5]{mm} mit der \emph{Klassenoption} \Option||{BCOR=5mm} 
festlegen.

Eine Anpassung der Bindekorrektur hat natürlich \emph{immer} eine Änderung der 
verfügbaren Breite des Textbereichs zur Folge hat und führt somit zwingend zu 
einer Anpassung des Satzspiegels. Da die Bindekorrektur jedoch abhängig von der 
Höhe des Buchblocks gewählt werden sollte, welche letztendlich erst mit dem 
Druck des fertiggestellten Dokumentes bestimmt werden kann, muss diese zu 
Beginn abgeschätzt werden.
%
\begin{Example}
Als Faustregel gilt, dass die erforderliche Bindekorrektur in etwa der halben 
Höhe des Buchblocks entsprechen sollte. Dessen Höhe wiederum ist abhängig von 
der Anzahl der Seiten sowie der Dichte des verwendeten Papiers. Wird normales 
Papier mit einer Dichte von \unit[80]{g/m²} verwendet, so entsprechen 100~Blatt 
in etwa einer Höhe von \unit[10]{mm}, bei \unit[100]{g/m²} ca. \unit[12]{mm}. 
Demzufolge wäre die Bindekorrektur mit \Option||{BCOR=5mm} beziehungsweise 
\Option||{BCOR=6mm} bei diesem Beispiel zu wählen.%
\index{Satzspiegel|)}%
\end{Example}



\subsubsection{%
  Kopf- und Fußzeile im Zusammenspiel mit dem Satzspiegel%
  \index{Layout!Kopfzeile|?}%
  \index{Layout!Fußzeile|?}%
}
%
Da im \CD nicht festgelegt ist, wie die Gestaltung der Kopf- und Fußzeilen in 
einer wissenschaftlichen Arbeit auszuführen ist, bleibt dem Nutzer dafür eine 
gewisse Freiheit. Dafür sollte idealerweise das zu \KOMAScript gehörige Paket 
\Package{scrlayer-scrpage} genutzt werden. 

In der Dokumentation zu \Package{typearea} wird auch darauf eingegangen, wann 
Kopf- und Fußzeile bei der Satzspiegelkonstruktion entweder dem Rand oder dem 
Textkörper zugeschlagen werden sollten. Dies sollte bei der Erstellung eigener 
Kopf- und Fußzeilen beachtet werden. Die Einstellung dafür erfolgt mit den 
\KOMAScript-Optionen \InlineDeclaration{\Option(*){headinclude=\PBoolean}} 
sowie \InlineDeclaration{\Option(*){footinclude=\PBoolean}}. Diese können~-- 
unabhängig von der Einstellung zur Satzspiegelgestaltung durch die Option 
\Option{cdgeometry}~-- verwendet werden.

\begin{Declaration}
  {\Option{extrabottommargin=\PLength}}
  (0pt)
  [v2.03]
\printdeclarationlist
%
Mit dieser Option kann die Größe des unteren Seitenrandes angepasst werden, 
wenn der Satzspiegel des \CDs (\seeplain{\Option'page'{cdgeometry=\PMisc}}) 
verwendet wird, wobei positive Werte diesen vergrößern und negative Werte 
selbigen verkleinern. Dies ist bei Seiten im Stil \PageStyle{tudheadings} 
nützlich, da deren Fußbereich unter Umständen zu klein ist, falls für diesen 
entweder mit \Macro{footcontent} ein übergroßer Inhalt angegeben wurde oder via 
\Macro{footlogo} Drittlogos verwendet werden, welche durch das optionale 
Argument oder die Option \Option{footlogoheight} über die Standardhöhe hinaus 
vergrößert wurden. 
\end{Declaration}



\subsection{%
  Die Gestalt von Titel, Umschlagseite, Teilen sowie Kapiteln \& Co.%
  \index{Layout!Farben|(}%
  \index{Layout!Seitenstile|(}%
  \index{Layout!Überschriften|(}%
}
%
Falls die nachfolgend beschriebene Option \Option{cd=\PMisc} aktiviert ist, 
werden einige, spezielle Seiten im prägnanten Stil mit dem Logo der \TnUD und 
der dazugehörigen Kopfzeile mit Querbalken gesetzt. Dies betrifft insbesondere 
\hyperref[sec:title]{Umschlagseite und Titel aus \autoref*{sec:title}} als auch 
die \hyperref[sec:part]{Teileseite in \autoref*{sec:part}} sowie die 
\hyperref[sec:chapter]{Kapitelseite in \autoref*{sec:chapter}}. Mit den 
\PageStyle{tudheadings}"=Seitenstilen oder der \Environment{tudpage}"=Umgebung 
können weitere Seiten in diesem Stil erzeugt werden. Wird das Paket 
\Package{tudscrsupervisor} verwendet und mit den bereitgestellten Befehlen oder 
Umgebungen eine Aufgabenstellung, ein Gutachten oder ein Aushang erstellt, so 
erscheinen auch diese in besagtem Seitenstil.
%
\begin{Declaration}
  {\Option{cd=\PMisc}}
  (true)
\printdeclarationlist
%
Mit dieser Option wird festgelegt, ob und wie das \TUDCD im gesamten Dokument 
verwendet wird. Sie hat Einfluss auf die Ausprägung von Titel"~, Teil"~, und 
Kapitelseiten sowie die Überschriften der weiteren Gliederungsebenen. 
Im Layout des \CDs wird auch bei \Option{cdfont=false} die Hausschrift in 
Überschriften verwendet.
\begin{DeclareValues}
\itemval=false=
  Es wird kein \CD sondern die Gestalt der \KOMAScript-Klassen genutzt.
\itemval*=true,nocolor,monochrome=
  Das Layout für Titel"~, Teil- und Kapitelseiten ist im \CD, es wird 
  schwarze Schrift für Titel sowie Teil- und Kapitelüberschriften verwendet.
  Die Ausprägung des Seitenkopfes ist abhängig von der Option \Option{cdhead}.
\itemval=lightcolor,pale=
  Die Einstellung entspricht weitestgehend der Option \Option{cd=true}, 
  allerdings wird die primäre Hausfarbe \Color{HKS41} für den Kopf des 
  \PageStyle{tudheadings}"=Seitenstils und Überschriften genutzt.
\itemval=barcolor=[v2.04:nur farbig abgesetzter Querbalken]
  Zusätzlich zur vorherigen Einstellung wird außerdem der Querbalken farbig 
  abgesetzt.
\itemval=bicolor,bichrome=%
    [v2.03:Farbeinsatz nur im Kopf mit farbig abgesetztem Querbalken]
  Der Kopf wird mit einem farbigen Hintergrund in der Hausfarbe gesetzt, auch 
  der Querbalken wird farbig hinterlegt. Für die Überschriften wird die 
  primären Hausfarbe verwendet.
\itemval=color=
  Der Titel sowie Teil- und Kapitelseiten werden allesamt farbig gestaltet, 
  der Seitenkopf wird in der primären Hausfarbe \Color{HKS41} gesetzt, der 
  Querbalken erhält Linien als Begrenzung.
\itemval=fullcolor,full=%
    [v2.03:voller Farbeinsatz mit farbig abgesetztem Querbalken]
  Entspricht der vorherigen Einstellung, allerdings wird der Querbalken nicht 
  durch Linien begrenzt sondern farbig hinterlegt.
\end{DeclareValues}
\end{Declaration}
\newcommand*\cdalias{false,true,lightcolor,barcolor,bicolor,color,fullcolor}

\begin{Declaration}
  {\Option{headings=\PMisc}}
  (heavy)
  [v2.06]
\printdeclarationlist[%
  \index{Schriftstärke}%
]
Die Option \Option{headings=\PMisc} wird bereits von \KOMAScript definiert und 
dient unter anderem zum Festlegen der Größe von Überschriften. Diese wird von 
\TUDScript erweitert, um im Layout des \CDs die Schriftstärke der Überschriften 
anpassen zu können. Die Einstellungen gelten für die Gliederungsebenen bis 
einschließlich \Macro||{subsubsection}.
\begin{DeclareValues}
\itemval=light,normalbold=
  Die Überschriften werden in fetter Schrift gesetzt.
\itemval=heavy,ultrabold=
  Für Überschriften wird die extra"~fette Schriftstärke verwendet.
\end{DeclareValues}
\end{Declaration}



\subsubsection{Individuelle Einstellungen für einzelne Elemente des Layouts}
%
Das Verhalten aller für das Layout relevanten Elemente wird von der eben zuvor 
erläuterten Option \Option{cd=\PMisc} bestimmt. Dies betrifft zum einen sowohl 
den Titel~(\Macro{maketitle}) als auch die Umschlagseite~(\Macro{makecover}) 
und zum anderen alle Teile"~ (\Macro{part}, \Macro{addpart}) und Kapitelseiten 
(\Macro{chapter} ,\Macro{addchap}) sowie darunter liegenden Gliederungsebenen.

Soll ein bestimmtes Element des Layouts abweichend von der allgemeinen 
Einstellung für das gesamte Dokument erscheinen, so kann eine der folgenden 
Optionen genutzt werden, um dieses individuell anzupassen und die mit 
\Option{cd=\PMisc} eingestellten Vorgaben für das jeweilige Element zu 
überschreiben.%
\footnote{%
  \Option{cdtitle} für den Titel, \Option{cdcover} für die Umschlagseite,
  \Option{cdpart} für Teile, \Option{cdchapter} für Kapitel sowie
  \Option{cdsection} für alle darunter liegenden Gliederungsebenen%
}
Die gültigen Wertzuweisungen für die einzelnen Elemente entsprechend dabei den 
möglichen Werten für die Option \Option{cd=\PMisc}. 

Zu beachten ist, dass die verwendete Schrift für die Elemente des Layouts 
abhängig von der Option \Option{cdfont=\PMisc} ist. Sowohl für Titel- als auch
Umschlagseite kann diese über das optionale Argument von \Macro{maketitle} 
respektive \Macro{makecover} geändert werden.

\begin{Declaration}
  {\Option{cdtitle=\PMisc}}=\cdalias=
  (\Option{cd=\PMisc})
\printdeclarationlist[%
  \index{Titel|?}%
]
Mit der Option \Option{cdtitle} kann die allgemeine Einstellung für den Titel 
überschrieben werden. Es kann zwischen dem normalen (\Option||{cdtitle=false}) 
und dem im \CD umgeschaltet werden. Die neue Titelseite unterstützt alle durch 
\KOMAScript definierten Befehle für den Titel.%
\footnote{%
  \Macro{extratitle|\MPName{Schmutztitel}}, 
  \Macro{frontispiece|\MPName{Frontispiz}} für eine Seite vor dem eigentlichen 
  Haupttitel sowie \Macro{titlehead|\MPName{Kopf}},  
  \Macro{subject|\MPName{Typisierung}}, \Macro{title|\MPName{Titel}},
  \Macro{subtitle|\MPName{Untertitel}}, \Macro{author|\MPName{Autor}},
  \Macro{and}, \Macro{thanks|\MPName{Fußnote}}, 
  \Macro{publishers|\MPName{Verlag}} und \Macro{date|\MPName{Datum}} für den 
  Titel selbst sowie \Macro{uppertitleback|\MPName{Titelrückseitenkopf}} und
  \Macro{lowertitleback|\MPName{Titelrückseitenfuß}} für die Rückseite des 
  Titels sowie abschließend \Macro{dedication|\MPName{Widmung}} für eine 
  Danksagung.
}
Zusätzlich werden viele neue Felder definiert, welche vor allem für eine 
wissenschaftliche Arbeit von Relevanz sind. Genaueres dazu ist in 
\autoref{sec:title} nachzulesen. Der Titel selbst wird~-- unabhängig von der 
gewählten Variante~-- immer mit \Macro{maketitle} erzeugt.
\end{Declaration}

\begin{Declaration}
  {\Option{cdcover=\PMisc}}=\cdalias=
  (true|\Option{cd=false}:false)
  [v2.02]
\printdeclarationlist[%
  \index{Umschlagseite|?}%
]
Die \TUDScript-Klassen führen zusätzlich den Befehl \Macro{makecover} ein, mit 
dem sich neben dem Titel eine separate Umschlagseite erzeugen lässt. Diese ist 
in ihrer Gestalt der Titelseite sehr ähnlich, wird normalerweise jedoch in 
einem anderen Satzspiegel als dem des Buchblocks gesetzt. Mit der Option 
\Option{cdcover} kann~-- unabhängig von \Option{cd=\PMisc}~-- das Aussehen der 
Umschlagseite geändert werden. Wird \Option||{cdcover=false} gewählt, so 
entspricht die Umschlagseite dem originalen \KOMAScript-Titel. Die Verwendung 
des Befehls \Macro{makecover} sowie die dazugehörigen Parameter werden 
detailliert in \autoref{sec:title} erläutert.
\end{Declaration}

\begin{Declaration}
  {\Option{cdpart=\PMisc}}=\cdalias=
  (\Option{cd=\PMisc})
\printdeclarationlist[%
  \index{Layout!Teileseiten|?}%
]
Für die Teileseiten kann der Wert des Schlüssels \Option{cd=\PMisc} separat 
überschrieben und somit deren Layout respektive Erscheinungsbild beeinflusst 
werden, welches bei der Benutzung der Befehle \Macro{part} und \Macro{addpart}
sowie deren Sternversionen verwendet wird. In \autoref{sec:part} sind weitere 
Hinweise zur Teileseite im 
\CD zu finden.
\end{Declaration}

\begin{Declaration}
  {\Option{cdchapter=\PMisc}}=\cdalias=
  (\Option{cd=\PMisc})
\printdeclarationlist[%
  \index{Layout!Kapitelseiten|?}%
]
Für Kapitelseiten kann der Schlüsselwert \Option{cd=\PMisc} ebenfalls angepasst 
und damit das Layout respektive Erscheinungsbild geändert werden, das bei der 
Verwendung von \Macro{chapter} beziehungsweise \Macro{addchap} und den 
dazugehörigen Sternversionen genutzt wird. Weitere Hinweise zur Kapitelseite im 
\CD sind in \autoref{sec:chapter} zu finden.
\end{Declaration}

\begin{Declaration}
  {\Option{cdsection=\PMisc}}=\cdalias=
  (\Option{cd=\PMisc})
  [v2.05]
\printdeclarationlist
%
Für Überschriften der weiteren Gliederungsebenen \Macro{section}, 
\Macro{subsection}, \Macro{subsubsection} sowie \Macro{paragraph} und 
\Macro{subparagraph} werden in der primären Hausfarbe \Color{HKS41} ausgegeben, 
falls über die Option \Option{cd=\PMisc} eine farbige Ausprägung des Layouts 
eingestellt wurde. Mit der Angabe von \Option||{cdsection=true} erscheinen die 
Überschriften der genannten Gliederungsebenen ohne Farbeinsatz.%
\index{Layout!Farben|)}%
\end{Declaration}
%
\begin{Example}
Soll die Titelseite in Farbe, der Rest des Dokumentes allerdings in schwarzer 
Schrift gesetzt werden, so kann dies folgendermaßen erreicht werden:
\begin{Code}[escapechar=§]
\documentclass[cd=true,cdtitle=color]{§\PName{Dokumentklasse}§}
\end{Code}
\end{Example}



\subsubsection{Die vertikale Position der Überschriften}
%
\begin{Declaration}
  {\Option{pageheadingsvskip=\PLength}}
  (0pt)
  [v2.05]
\begin{Declaration}
  {\Option{headingsvskip=\PLength}}
  (0pt)
  [v2.05]
\printdeclarationlist[%
  \index{Layout!Teileseiten|?}%
  \index{Layout!Kapitelseiten|?}%
]
Mit diesen beiden Optionen kann die vertikale Position spezieller Überschriften 
verändert werden. Mit der Option \Option{pageheadingsvskip} lässt sich sowohl 
der Titel auf einer Titelseite (\KOMAScript-Option \Option{titlepage=true}) als 
auch die Überschriften von separaten Kapitelseiten (\Option{chapterpage=true}) 
und Teilen vertikal verschieben. Demgegenüber kann mit der zweiten Option 
\Option{headingsvskip} sowohl der eigentliche Titel des Titelkopfes 
(\KOMAScript-Option \Option{titlepage=false}) als auch die Überschrift normaler 
Kapitel bei deaktivierter Kapitelseite (\Option{chapterpage=false}) in der 
vertikalen Position angepasst werden.

Die zuvor genannten Überschriften werden normalerweise im Layout relativ tief 
im Textbereich gesetzt. Mit negativen Werten werden die Überschriften nach oben 
verschoben, wobei darauf geachtet werden sollte, dass diese sich danach noch 
innerhalb des Satzspiegels befinden. Positive Werte setzen die Überschriften 
dementsprechend tiefer.%
\index{Layout!Überschriften|)}%
\end{Declaration}
\end{Declaration}



\subsection{%
  Seiten im Stil des \CDs%
  \label{sec:tudheadings}%
  \index{Layout!Seitenstile|?(}%
}
%
\begin{Declaration}
  {\Macro{faculty|\MPName{Fakultät}}}
\begin{Declaration}
  {\Macro{department|\MPName{Einrichtung}}}
\begin{Declaration}
  {\Macro{institute|\MPName{Institut}}}
\begin{Declaration}
  {\Macro{chair|\MPName{Lehrstuhl}}}
\begin{Declaration}
  {\Macro{extraheadline|\MPName{Textzeile}}}
\printdeclarationlist[%
  \index{Layout!Kopfzeile}%
]
Für den Seitenstil des \TUDCDs charakteristisch ist die Kopfzeile mit dem 
prägnanten Querbalken. In dieser wird~-- falls angegeben~-- in fetter Schrift 
die Fakultät ausgegeben, gefolgt von Einrichtung, Institut und Professur. 
Sollte hierfür eine einzelne Zeile nicht ausreichen, erfolgt an passender 
Stelle ein automatischer Zeilenumbruch.

In besonderen Ausnahmefällen erlaubt das \CD die Angabe einer zusätzlichen
zweiten beziehungsweise dritten Zeile unterhalb der Angaben des Bereichs an der 
\TnUD, welche weitere, frei wählbare Angaben enthält. Diese kann mit dem Befehl 
\Macro{extraheadline|\MPName{Textzeile}} definiert werden.
\end{Declaration}
\end{Declaration}
\end{Declaration}
\end{Declaration}
\end{Declaration}
%
\begin{Declaration}
  {\PageStyle{tudheadings}}
  [v2.02]
\begin{Declaration}
  {\PageStyle{plain.tudheadings}}
  [v2.02]
\begin{Declaration}
  {\PageStyle{empty.tudheadings}}
  [v2.02]
\printdeclarationlist[%
  \index{Layout!Kopfzeile}%
  \index{Layout!Zweitlogo}%
  \index{Layout!Fußzeile}%
]
\ChangedAt*{v2.02:Paket \Package{scrlayer-scrpage} zwingend erforderlich}%
Ein zentrales Element des \TUDCDs ist der prägnante Seitenkopf mit der Angabe 
von Fakultät~(\Macro{faculty}), Einrichtung~(\Macro{department}), 
Institut~(\Macro{institute}) und Lehrstuhl~(\Macro{chair}) im dazugehörigen 
Querbalken. Mithilfe der Einstellungsmöglichkeiten des \KOMAScript-Paketes 
\Package{scrlayer-scrpage} lassen sich entweder einzelne Seiten oder auch ganze 
Dokumente sehr einfach in diesem Stil setzen. Dazu muss lediglich mit 
\Macro{pagestyle|\MPName{Seitenstil}} einer der Seitenstile geladen werden. 

\ChangedAt{v2.03:Seitenstile um zweifarbigen Kopf und farbigen Fuß erweitert}%
Allen Seitenstilen gemein ist der typische Kopf mit dem charakteristischen 
Querbalken, dessen Gestalt für \emph{alle} Seitenstile gleichermaßen über die 
Option \Option{cdhead} angepasst werden kann. Mit dem Befehl \Macro{headlogo} 
lässt sich ein zusätzliches Zweitlogo im Kopfbereich ausgegeben.
\Attention{%
  Für die speziellen Layout-Elemente Titel und Umschlagseite sowie Teile- und 
  Kapitelseiten wird die Einstellung von \Option{cdhead} durch die Nutzung der 
  Option~\Option{cd=\PMisc} überschrieben.
}

Die Ausprägung des Fußes unterscheidet sich bei den einzelnen Seitenstilen. 
Dieser ist beim Seitenstil \PageStyle{empty.tudheadings} immer leer. Die beiden 
Stile~-- oder vielmehr das Seitenstil"~Paar~-- \PageStyle{tudheadings} und 
\PageStyle{plain.tudheadings} übernehmen die Einstellungen der Fußzeile aus der 
Anwenderschnittstelle von \Package{scrlayer-scrpage}.%
\footnote{%
  Es können die Befehle
  \Macro{lefoot}, \Macro{cefoot} und \Macro{refoot} sowie \Macro{lofoot}, 
  \Macro{cofoot} und \Macro{rofoot} respektive \Macro{ofoot}, \Macro{cfoot} und 
  \Macro{ifoot} genutzt werden.%
}
Wie die einzelnen Befehle zur Individualisierung der Fußzeile zu verwenden 
sind, kann der \scrguide[\KOMAScript-Anleitung] entnommen werden. Alternativ 
zu einer eigenen Definition der Fußzeile lässt sich außerdem die Option 
\Option{cdfoot} verwenden. Zusätzlich kann über \Macro{footcontent} ein freier 
Inhalt für den Fußbereich definiert werden, mit \Macro{footlogo} ist die 
Ausgabe von einem oder mehreren Logos in diesem möglich. Die verwendete Schrift 
im Fußbereich wird durch das Schriftelement~\Font{tudheadings} festgelegt.

Sobald mit \Macro{pagestyle|\MPName{Seitentil}} einer der hier beschriebenen 
Seitenstile aktiviert wurde, werden die eigentlichen Standardseitenstile 
\PageStyle{headings} beziehungsweise \PageStyle{plain} als Alias für 
\PageStyle{tudheadings} respektive \PageStyle{plain.tudheadings} definiert. 
Damit verwenden Optionen oder Befehle, welche normalerweise automatisch 
zwischen den Standardseitenstilen umschalten, nun die jeweils entsprechenden 
\PageStyle{tudheadings}"=Seitenstile.

Der Seitenstil \PageStyle{empty} erzeugt allerdings weiterhin eine komplett 
leere Seite. Soll stattdessen eine Seite erschienen, die zwar einen leeren 
Seitenfuß jedoch die prägnante Kopfzeile der \TnUD aufweist, so muss hierfür 
direkt \PageStyle{empty.tudheadings} als Seitenstil ausgewählt werden. Um auf 
das standardmäßige Verhalten von \KOMAScript zurückzuschalten, ist mit 
\Macro{pagestyle|\MPName{Seitentil}} einer der beiden Seitenstile 
\PageStyle{scrheadings} oder \PageStyle{plain.scrheadings} zu aktivieren.%
\index{Layout!Seitenstile|?)}%
\end{Declaration}
\end{Declaration}
\end{Declaration}

\begin{Declaration}
  {\Option{cdhead=\PMisc}}
  (true,nocolor,textwidth)
  [v2.03]
\printdeclarationlist[%
  \index{Layout!Kopfzeile}%
  \index{Layout!Querbalken}%
  \index{Datum}%
]
Mit dieser Option lassen sich für die \PageStyle{tudheadings}"=Seitenstile 
sowohl die Gestalt des Logos sowie des Querbalkens als auch die darin 
verwendete Schrift beeinflussen. Die folgenden Werte können für eine Anpassung 
der Schriftart im Balken verwendet werden:
\begin{DeclareValues}
\itemval=false=
  Sollte mit \Option{cdfont=false} die Verwendung der Hausschrift des \TUDCDs 
  deaktiviert worden sein, wird die Kopfzeile im Querbalken in den Serifenlosen 
  der aktiven Schrift gesetzt. Ist die Hausschrift aktiviert, hat diese 
  Einstellung keinen Einfluss.
\itemval*=true,light,lightfont=
  Im Querbalken wird für \Macro{faculty} \textcdsn{Open~Sans~Semi"~Bold} 
  verwendet, für \Macro{department}, \Macro{institute}, 
  \Macro{chair} und \Macro{extraheadline} kommt \textcdln{Open~Sans~Light} zum 
  Einsatz.
\itemval=heavy,heavyfont=
  Der Inhalt von \Macro{faculty} wird in \textcdbn{Open~Sans~Bold} gesetzt, für 
  die restlichen Felder wird \textcdrn{Open~Sans~Regular} genutzt.
\end{DeclareValues}

Bei der Ausprägung des Kopfes und des Querbalkens gibt es mehrere Varianten. 
Einerseits kann der Querbalken mit zwei Außenlinien dargestellt werden:
\begin{DeclareValues}
\itemval=nocolor,monochrome=
  Der Kopf und die Linien des Querbalkens erscheinen in schwarzer Farbe.
\itemval=lightcolor,pale=
  Sowohl Kopf als auch Querbalken werden in der primären Hausfarbe gesetzt.
\end{DeclareValues}

Andererseits ist auch eine Darstellung mit mehr Farbeinsatz möglich, bei 
welcher der Querbalken und gegebenenfalls der ganze Seitenkopf farbig 
abgesetzt wird. Dabei erstreckt sich der Kopfbereich über die komplette 
Seitenbreite, \seeplain{\Option'page'{cdhead=paperwidth}}.
\begin{DeclareValues}
\itemval=barcolor=[v2.04:nur farbig abgesetzter Querbalken]
  Im Gegensatz zur vorherigen Einstellung wird der Querbalken mit farbigem 
  Hintergrund verwendet, der darüber liegende Teil des Kopfes wird ohne 
  farbigen Hintergrund gesetzt.
\itemval=bicolor,bichrome=
  Die Kopfzeile wird farbig abgesetzt, wobei der Hintergrund des Logos und der 
  Querbalken unterschiedlich ausfallen. Die Außenlinien der Querbalkens 
  entfallen.
\end{DeclareValues}

Für den Fall, dass der Querbalken lediglich mit zwei Außenlinien dargestellt 
wird, kann zusätzlich dessen Laufweite festgelegt werden:
\begin{DeclareValues}
\itemval=textwidth,slim=
  Der Querbalken im Kopf erstreckt sich nur über den Textbereich. Diese 
  Einstellung ist insbesondere sinnvoll, wenn ein randloser Ausdruck technisch 
  nicht möglich ist. 
\itemval=paperwidth,wide=
  Die horizontale Ausdehnung des Querbalkens entspricht der kompletten 
  Seitenbreite bis an den Blattrand. Beim Kopf mit farbigem Hintergrund lässt 
  sich dies \emph{nicht} deaktivieren.
\end{DeclareValues}

Neben den zuvor beschriebenen Möglichkeiten zur Gestaltung des Kopfbereiches 
kann auf allen Seiten mit aktiviertem \PageStyle{tudheadings}"=Seitenstil 
unterhalb des Querbalkens das mit \Macro{date} angegebene Datum rechtsbündig 
eingeblendet werden.
\begin{DeclareValues}
\itemval=date,showdate=[v2.05:Datum zwischen Kopf- und Textbereich]
  Das eingestellte Datum wird rechts oberhalb vom Textbereich eingeblendet.
\itemval=nodate,hidedate=[v2.05:kein Datum zwischen Kopf- und Textbereich]
  Es erscheint kein Datum zwischen Kopf- und Textbereich.
\end{DeclareValues}
\end{Declaration}

\begin{Declaration}
  {\Option{cdfoot=\PMisc}}
  (false)%
\printdeclarationlist[%
  \index{Layout!Fußzeile}%
  \index{Layout!Kolumnentitel}%
  \index{Satzspiegel!doppelseitig}%
]
Die \TUDScript-Klassen sind~-- insbesondere aufgrund der Möglichkeit zur 
Verwendung des Paketes \Package{scrlayer-scrpage}~-- bei der Gestaltung der 
Kopf- und Fußzeilen sehr flexibel und individuell anpassbar. Die Ausprägung und 
der Inhalt dieser ist nicht explizit durch das \CD vorgegeben und können durch 
den Anwender beliebig gewählt und geändert werden. 

Eine Möglichkeit für deren Gestaltung zeigt das Handbuch für das \TUDCD. Dieses 
wird ohne Kopf- und mit einer einfachen Fußzeile gesetzt, welche den aktuellen 
Kolumnentitel sowie die Paginierung enthält. Mit \Option{cdfoot} kann diese 
Ausprägung aktiviert werden, was auch für dieses Anwenderhandbuch geschehen ist.
\begin{DeclareValues}
\itemval=false=
  Die Kopf- und Fußzeilen zeigen Standardverhalten, zur manuellen Änderung 
  ist \emph{zwingend} das Paket \Package{scrlayer-scrpage} zu verwenden.
\itemval*=true=
  Die Fußzeilen des Dokumentes werden äquivalent zum Handbuch des \TUDCDs mit 
  lebenden Kolumnentitel und Seitenzahl gesetzt, wobei im doppelseitigen Satz 
  (Klassenoption \Option{twoside=true}) die Paginierung am äußeren Rand des 
  Satzspiegels erfolgt.
\end{DeclareValues}

Wird beim Laden des Paketes \Package{scrlayer-scrpage} respektive einer 
\TUDScript-Klasse die Option \InlineDeclaration{\Option(*){manualmark}} 
\emph{nicht} explizit angegeben, so werden mit \Option{cdfoot=true} 
standardmäßig über die Option \InlineDeclaration{\Option(*){automark}} auch 
gleichzeitig die automatischen Kolumnentitel aktiviert, welche als Marken die 
Titel der Gliederungsebenen verwendet. Genaueres dazu und der Möglichkeit, die 
Kolumnentitel manuell festzulegen, ist dem \scrguide zu entnehmen.

Sollte einer der \PageStyle{tudheadings}"=Seitenstil aktiviert sein und es wird 
auf der erzeugten Seite ein farbiges Layout~--  beispielsweise der zweifarbige 
Kopf (\Option{cdhead=bicolor}) oder ein farbiger Seitenhintergrund~-- genutzt, 
so kann auch die Fußzeile einen farbigen Hintergrund erhalten.
\begin{DeclareValues}
\itemval=nocolor,monochrome=[v2.03:Fußzeile ohne farbigen Hintergrund]
  Der Fuß wird immer ohne farbigen Hintergrund gesetzt.
\itemval=color,bicolor,bichrome=[v2.03:farbiger Hintergrund der Fußzeile]
  Die Fußzeile wird farbig abgesetzt, falls entweder der Kopf in einer farbigen
  Variante genutzt wird (\seeplain{\Option'page'{cdhead=\PMisc}}) oder ein 
  farbiger Seitenhintergrund in der Hausfarbe (Titel oder Kapitelseite) aktiv 
  ist.
\end{DeclareValues}

Mit den Befehlen \Macro'page'{footlogo} und \Macro'page'{footcontent} können 
für den Fußbereich zusätzliche Inhalte definiert werden. Sollte der zur 
Verfügung stehende Platz hierfür nicht ausreichen, lässt sich dieser vergrößern.
\begin{DeclareValues}
\itemval=\PLength=[v2.03]
  Wird ein Längenwert übergeben, entspricht dies 
  \Option{extrabottommargin=\PLength}.
\end{DeclareValues}
\end{Declaration}

\begin{Declaration}
  {\Font{tudheadings}}
  [v2.04]
\printdeclarationlist[%
  \index{Schriftelemente}
]
Im Fußbereich der Seiten im \PageStyle{tudheadings}"=Seitenstil wird das 
Schriftelement~\Font{tudheadings} verwendet. Dieses wirkt sich auf die 
Seitenzahlen, den Kolumnentitel und die mit \Macro{footcontent} angegebenen 
Inhalte aus. Hierüber wird die Wahl der richtigen Schriftfarbe in Abhängigkeit 
vom Seitenhintergrund und den Einstellungen für die Optionen \Option{cdhead} 
sowie \Option{cdfoot} realisiert. Wie das Schriftelement~\Font{tudheadings} 
angepasst werden kann, ist in \autoref{sec:fonts:elements} zu finden.
\end{Declaration}

\begin{Declaration}
  {\Macro{headlogo|\OList\MPName{Dateiname}}}
\printdeclarationlist[%
  \index{Layout!Kopfzeile}%
  \index{Layout!Zweitlogo|?}%
  \index{Layout!Dresden-concept-Logo@\DDC-Logo}%
]
Neben dem Logo der \TnUD darf zusätzlich ein Zweitlogo im Kopf verwendet 
werden. Dieses lässt sich mit diesem Befehl einbinden. Normalerweise wird es 
auf die Höhe der Erstlogos skaliert. Über das optionale Argument können weitere 
Formatierungsparameter an den darunterliegend verwendeten Befehl 
\Macro{includegraphics} durchgereicht werden, um beispielsweise die Größe des 
Zweitlogos anzupassen. Welche Parameter angepasst werden können, ist der 
Dokumentation des Paketes \Package{graphicx} zu entnehmen.

Sollte die Option \Option{ddc} aktiviert sein, wird das \DDC-Logo nicht im Kopf 
sondern automatisch im Fuß gesetzt. Die Option \Option{ddchead} setzt dieses 
auf jeden Fall im Kopf und überschreibt damit das mit \Macro{headlogo} 
angegebene Zweitlogo.
\end{Declaration}

\begin{Declaration}
  {\Macro{footlogo|\OList\MPName{Dateinamenliste}}}
  [v2.03]
\begin{Declaration}
  {\Macro{footlogosep}}
  [v2.03]
\printdeclarationlist[%
  \index{Layout!Fußzeile}%
  \index{Layout!Drittlogo}%
  \index{Layout!Dresden-concept-Logo@\DDC-Logo}%
]
\ChangedAt*{v2.03:Seitenstile können zusätzliche Logos im Seitenfuß einbinden}%
Laut den Richtlinien des \CDs dürfen im Fußbereich weitere Logos erscheinen, 
beispielsweise von kooperierenden Unternehmen oder Sponsoren. Die Dateinamen 
der gewünschten Logos können als kommaseparierte Liste im obligatorischen 
Argument des Befehls \Macro{footlogo} angegeben werden. Sollte tatsächlich 
nicht nur ein Dateiname sondern eine Liste übergeben worden sein, so wird bei 
der Ausgabe der Drittlogos zwischen diesen jeweils der in \Macro{footlogosep} 
gespeicherte Separator~-- standardmäßig \Macro*{hfill}~-- gesetzt. Dieser kann 
mit \Macro*{renewcommand|*\MPValue{\Macro{footlogosep}}\MPValue{\dots}} 
beliebig durch den Anwender angepasst werden. Der Separator wird auch gesetzt, 
falls in \PName{Dateinamenliste} lediglich ein Komma verwendet wurde. Mit 
\Macro{footlogo|\MPValue{,\PName{Dateiname},}} kann so beispielsweise ein 
Logo zentriert im Fuß gesetzt werden.
\Attention{%
  Dabei ist zu beachten, dass ein mit der Option \Option{ddc} beziehungsweise 
  \Option{ddcfoot} gesetztes \DDC-Logo im Fußbereich~-- im Gegensatz zur 
  Verwendung von \Macro{footcontent}~-- überlagert werden könnte. Hier muss der 
  Anwender im Zweifel durch das Einfügen von Separatoren~-- sprich Kommas~-- im 
  Argument von \Macro{footlogo} etwas Formatierungsarbeit leisten.
}

Das optionale Argument von \Macro{footlogo} wird an \Macro{includegraphics} 
weitergereicht. Dies geschieht für alle angegeben Dateien aus der Liste 
gleichermaßen. Sollen für einzelne Logos individuelle Einstellungen vorgenommen 
werden, so sind die entsprechenden Parameter im obligatorischen Argument nach 
dem jeweiligen Dateinamen mit einem Doppelpunkt~\enquote{\PValue{:}} als 
Separator (\Macro{footlogo|\MPValue{\PName{Dateiname}:\PName{Parameter}}}) zu 
übergeben, wobei diese \emph{nach} den allgemeinen Einstellungen für alle Logos 
angewendet werden. Die möglichen Werte für die optionalen Parameter sind der 
Dokumentation des Paketes \Package{graphicx} zu entnehmen.
\end{Declaration}
\end{Declaration}

\begin{Declaration}
  {\Option{footlogoheight=\PLength}}
  (0pt)
  [v2.05]
\printdeclarationlist
%
Im Fußbereich wird sowohl für das \DDC-Logo als auch für Drittlogos, die ohne 
Größenangabe im optionalen Argument von \Macro{footlogo} definiert wurden, die 
Höhe mit der Option \Option{footlogoheight=\PLength} festgelegt. Mit der Angabe 
von \Option||{footlogoheight=0pt} werden diese auf die Höhe des Logos der 
\TnUD skaliert. Sollte der Platz im Fußbereich nicht ausreichen, um alle Logos 
in der gewünschten Größe darstellen zu können, kann dieser über die Option
\Option{extrabottommargin=\PLength} angepasst werden.
\end{Declaration}

\begin{Declaration}
  {\Macro{footcontent|\OPName{Anweisungen}\MPName{Inhalt}\OPName{Inhalt}}}
  [
    v2.04;
    v2.05:Änderung des Inhaltes nur einer Spalte möglich;
  ]
\begin{Declaration}
  {\Macro{footcontent*|\OPName{Anweisungen}\MPName{Inhalt}\OPName{Inhalt}}}
  [v2.04]
\printdeclarationlist[%
  \index{Layout!Fußzeile}%
]
Mit diesem Befehl kann beliebiger Inhalt entweder einspaltig oder zweispaltig 
im Fußbereich der \PageStyle{tudheadings}"=Seitenstile gesetzt werden. In der 
Form \Macro{footcontent|\MPName{Inhalt}} wird der Inhalt über die komplette 
Textbreite im Fuß ausgegeben. Wird der Befehl jedoch in der zweiten Variante 
\Macro{footcontent|\MPName{linker Inhalt}\OPName{rechter Inhalt}} mit einem 
optionalen \emph{nach} dem obligatorischen Argument verwendet, so erscheint der 
Fußbereich zweispaltig, wobei der Inhalt aus dem ersten, obligatorischen 
Argument in der linken und der Inhalt aus dem zweiten, optionalen Argument 
entsprechend in der rechten Fußspalte gesetzt wird. Dabei wird ein etwaiges 
\DDC-Logo, welches über die Option \Option{ddc} respektive \Option{ddcfoot} 
gesetzt wurde, beachtet und der für den Text zur Verfügung stehende Platz im 
Fuß reduziert. Mögliche Einstellungen des Paketes \Package{scrlayer-scrpage} 
für den Fuß werden nicht berücksichtigt, hier kann es zu Überlagerungen der 
Inhalte kommen. Gleiches gilt für die Verwendung der Werte \PValue{true} und 
\PValue{false} für die Option \Option{cdfoot}.

\ChangedAt{v2.05}%
Wird an das Argument für die linke oder die rechte Spalte lediglich ein Stern 
\PValue{*} übergeben, so bleibt der bis dahin definierte Inhalt in dieser 
Spalte erhalten. Beispielsweise kann die linke Fußbereichsspalte mit 
\Macro{footcontent|\MPName{Inhalt}\OPValue{*}} angepasst werden ohne dabei den 
Inhalt der rechten Spalte zu verändern oder es ließe sich lediglich die 
verwendete Schrift des Fußbereichs bei gleichbleibendem Inhalt mit 
\Macro{footcontent|\OPName{Anweisungen}\MPValue{*}\OPValue{*}} anpassen.

Im Fußbereich wird das Schriftelement \Font{tudheadings} verwendet. Dabei wird 
auch die Schriftgröße angepasst, wobei diese sich an der Kopfzeile orientiert. 
Zusätzlich können mit dem ersten optionalen Argument von \Macro{footcontent}~-- 
vor der eigentlichen Ausgabe des Inhaltes~-- zusätzliche Schrifteinstellungen 
respektive \PName{Anweisungen} ausgeführt werden. Soll die Definition des 
Inhalts im Fußbereich \emph{ohne} eine automatische Anpassung der Schriftgröße 
erfolgen, so ist die Sternversion \Macro{footcontent*} zu verwenden. Auch hier 
lässt sich das optionale Argument für die Schriftformatierung nutzen.
\ChangedAt{v2.06}%
Zu guter Letzt können bei der Änderung des Inhaltes im Fußbereich mit einem 
Stern im ersten optionalen Argument 
\Macro{footcontent|\OPValue{*}\MPName{Inhalt}\OPName{Inhalt}} auch zuvor 
durchgeführte Anpassungen der Schrift unverändert bleiben.
\end{Declaration}
\end{Declaration}

\begin{Declaration}
  {\Option{ddc=\PMisc}}
  (false)
  [v2.02:Logo von \DDC automatisch in Kopf/Fuß;]
\begin{Declaration}
  {\Option{ddchead=\PMisc}}
  [v2.02]
\begin{Declaration}
  {\Option{ddcfoot=\PMisc}}
  [v2.02]
\printdeclarationlist[%
  \index{Layout!Kopfzeile}%
  \index{Layout!Zweitlogo}%
  \index{Layout!Fußzeile}%
  \index{Layout!Dresden-concept-Logo@\DDC-Logo}%
]
Diese Option fügt das Logo von \DDC entweder im Kopf oder Fuß der Seiten mit 
dem Stil \PageStyle{tudheadings} ein. Diese wird automatisch entweder im Kopf 
oder~-- falls mit \Macro{headlogo} ein Zweitlogo angegeben wurde~-- im Fuß 
gesetzt.

Alternativ dazu können die Optionen \Option{ddchead} beziehungsweise 
\Option{ddcfoot} genutzt werden, welche das Logo explizit entweder im Kopf oder 
Fuß setzen. Ein mit \Macro{headlogo} angegebenes Zweitlogo wird durch 
\Option{ddchead=\PMisc} definitiv unterdrückt, \Option{ddcfoot=\PMisc} setzt 
das \DDC-Logo in jedem Fall in den Seitenfuß. Die Verwendung einer der drei 
Optionen führt folglich zur Deaktivierung der anderen beiden.

Das Logo von \DDC wird standardmäßig sowohl im Kopf als auch im Fuß in der 
gleichen Höhe wie das Logo der \TnUD gesetzt und lässt sich zumindest für den 
Kopf nicht ändern. Wird das Logo hingegen im Fuß verwendet, kann die Größe über 
die Option \Option{footlogoheight} angepasst werden. Sollte nach einer 
Vergrößerung des Logos die Höhe des Fußbereiches nicht ausreichen, so kann 
diese über \Option{extrabottommargin=\PLength} angepasst werden. Folgend sind 
die möglichen Werte für die Option \Option{ddc} aufgeführt, welche aber 
gleichermaßen für \Option{ddchead} und \Option{ddcfoot} gelten:
\begin{DeclareValues}
\itemval=false=
  Bei den \PageStyle{tudheadings}"=Seitenstile erscheint kein Logo von \DDC.
\itemval*=true=
  Das Logo von \DDC wird im Kopf beziehungsweise Fuß verwendet. Die Wahl der 
  Farbe des Logos geschieht passend zur farblichen Ausprägung der Seite selbst.
\end{DeclareValues}

Die Farbe des \DDC-Logos wird in Abhängigkeit der farblichen Ausprägung des 
Layouts (Option \Option{cd=\PMisc}) automatisch gewählt. Dies lässt sich 
manuell ändern:
\begin{DeclareValues}
\itemval=color=
  Im Kopf oder Fuß wird die achtfarbige 4C"~Variante des \DDC-Logos genutzt.
\itemval=colorblack=[v2.02]
  Es wird das achtfarbige Logo mit schwarzem \DDC-Schriftzug anstelle des 
  grauen verwendet. Für den Fuß wird der grüne Claim durch einen schwarzen 
  ersetzt. Dies ist insbesondere für kleine Darstellungen des Logos im Fuß 
  sinnvoll.
\itemval=gray,grey=[v2.02]
  Dies Ausgabe des \DDC-Logos erfolgt in Graustufen.
\itemval=black=[v2.02]
  Verwendung des Logos in Graustufen mit schwarzem Schriftzug.
\itemval=blue=[v2.02]
  Schriftzug und Logo werden in Abstufungen der Hausfarbe \Color{HKS41} gesetzt.
\itemval=white=[v2.02]
  Das \DDC-Logo sowie der dazugehörige Schriftzug sind vollständig weiß.
\end{DeclareValues}
\index{Layout!Seitenstile|)}%
\end{Declaration}
\end{Declaration}
\end{Declaration}

\begin{Declaration}
  {\Environment{tudpage|\OPList{Sprache}}}
\begin{Declaration}
  {\Environment{tudpage/language=\PSet{Sprache}}}
\begin{Declaration}
  {\Environment{tudpage/columns=\PSet{Anzahl}}}
\begin{Declaration}
  {\Environment{tudpage/pagestyle=\PSet{Seitenstil}}}
  [v2.02]
\begin{Declaration}
  {\Environment{tudpage/cdfont=\PMisc}}
  <\Option{cdfont}>
\begin{Declaration}
  {\Environment{tudpage/cdhead=\PMisc}}
  <\Option{cdhead}>
  [v2.03]
\begin{Declaration}
  {\Environment{tudpage/cdfoot=\PMisc}}
  <\Option{cdfoot}>
  [v2.03]
\begin{Declaration}
  {\Environment{tudpage/headlogo=\PSet{Dateiname}}}
  <\Macro{headlogo}>
\begin{Declaration}
  {\Environment{tudpage/footlogo=\PSet{Dateinamenliste}}}
  <\Macro{footlogo}>
  [v2.03]
\begin{Declaration}
  {\Environment{tudpage/ddc=\PMisc}}
  <\Option{ddc}>
  [v2.02]
\begin{Declaration}
  {\Environment{tudpage/ddchead=\PMisc}}
  <\Option{ddchead}>
  [v2.02]
\begin{Declaration}
  {\Environment{tudpage/ddcfoot=\PMisc}}
  <\Option{ddcfoot}>
  [v2.02]
\printdeclarationlist[%
  \index{Layout!Kopfzeile}%
  \index{Layout!Fußzeile}%
  \index{Layout!Seitenstile}%
]
Sind die Einstellungsmöglichkeiten der \PageStyle{tudheadings}"=Seitenstile 
nicht ausreichend, kann die \Environment{tudpage}"~Umgebung genutzt werden, 
wobei sich diese über die verschiedenen Parameter im optionalen Argument 
anpassen lässt. Mit dem Paket \Package{babel} kann die innerhalb der Umgebung 
angewandte Sprache mit \Environment{tudpage/language=\PSet{Sprache}} geändert 
werden, was zur Anpassung der sprachspezifischen Trennungsmuster und Bezeichner 
führt. Wurde das Paket \Package{multicol} geladen, wird der Inhalt entsprechend 
dem Parameter \Environment{tudpage/columns=\PSet{Anzahl}} mehrspaltig gesetzt. 
Mit dem Parameter \Environment{tudpage/pagestyle=\PSet{Seitenstil}} kann der 
gewünschte  \PageStyle{tudheadings}"=Seitenstil angepasst werden, wobei 
\PValue{headings}, \PValue{plain} und \PValue{empty} gültig sind. 

Die weiteren Parameter entsprechen in ihrem Verhalten den gleichnamigen 
Klassenoptionen oder Befehlen, wirken sich jedoch nur innerhalb der 
\Environment{tudpage}"~Umgebung aus. Das Verhalten sowie gültige 
Wertzuweisungen ist auf den angegebenen Seiten dokumentiert.
\end{Declaration}
\end{Declaration}
\end{Declaration}
\end{Declaration}
\end{Declaration}
\end{Declaration}
\end{Declaration}
\end{Declaration}
\end{Declaration}
\end{Declaration}
\end{Declaration}
\end{Declaration}



\subsection{%
  Der Titel und die Umschlagseite%
  \label{sec:title}%
  \index{Titel}%
  \index{Umschlagseite}%
}
%
\ChangedAt*{v2.03:Bugfix für Umschlagseite und Titel beim Satzspiegel}%
Für das Erstellen eines Titels mit dem Befehl \Macro{maketitle} wird mit der 
\KOMAScript-Option \Option{titlepage} festgelegt, ob dieser in Gestalt einer 
ganzen Titelseite oder nur als Titelkopf erscheinen soll. Für den Titel im 
\TUDCD werden alle Felder unterstützt, welche bereits durch \KOMAScript 
definiert sind. Darüber hinaus werden für die \TUDScript-Klassen weitere Felder 
bereitgestellt, welche Auswirkungen auf die Gestalt des Titels haben. Diese 
werden nachfolgend in diesem \autorefname erläutert. Der Titel~-- bestehend aus 
einem möglichen Schmutztitel (\Macro{extratitle}) und dazugehöriger Rückseite 
(\Macro{frontispiece|\MPName{Frontispiz}}), der eigentlichen Titelseite 
respektive des Titelkopfes und der nachgelagerten Elementen~-- kann mit 
\Macro{maketitle} ausgegeben werden. Außerdem lässt sich im zweispaltigen Satz 
\Macro{maketitleonecolumn} verwenden, womit eine einspaltige Ergänzung nach dem 
Titel selbst ermöglicht wird.

Zusätzlich zum Titel lässt sich mit \Macro{makecover} eine Umschlagseite 
erzeugen. Diese kann insbesondere für gebundene Arbeiten verwendet werden. Es 
wird~-- im Vergleich zum Titel~-- lediglich einer reduzierte Anzahl an Feldern 
auf dieser ausgegeben.

\begin{Declaration}
  {\Macro{maketitle|\OPList{Seitenzahl}}}
  [
    v2.01:Bugfix für Schriftstärke auf Titelseite für verschiedene Elemente;
    v2.02:Unterstützung der Schriftelemente \nosuffix{%
        \Font{titlepage} und \Font{thesis} sowie der bereits durch \KOMAScript 
        bereitgestellten \Font{titlehead}, \Font{subject}, \Font{title}, 
        \Font{subtitle}, \Font{author}, \Font{date}, \Font{publishers} und 
        \Font{dedication}};
  ]
\begin{Declaration}
  {\Macro{maketitle/pagenumber=\PSet{Seitenzahl}}}
  [v2.02]
\begin{Declaration}
  {\Macro{maketitle/cdgeometry=\PMisc}}
  <\Option{cdgeometry}>
  [v2.06]
\begin{Declaration}
  {\Macro{maketitle/cdfont=\PMisc}}
  <\Option{cdfont}>
  [v2.02]
\begin{Declaration}
  {\Macro{maketitle/cdhead=\PMisc}}
  <\Option{cdhead}>
  [v2.03]
\begin{Declaration}
  {\Macro{maketitle/cdfoot=\PMisc}}
  <\Option{cdfoot}>
  [v2.03]
\begin{Declaration}
  {\Macro{maketitle/headlogo=\PSet{Dateiname}}}
  <\Macro{headlogo}>
  [v2.03]
\begin{Declaration}
  {\Macro{maketitle/footlogo=\PSet{Dateinamenliste}}}
  <\Macro{footlogo}>
  [v2.03]
\begin{Declaration}
  {\Macro{maketitle/ddc=\PMisc}}
  <\Option{ddc}>
  [v2.03]
\begin{Declaration}
  {\Macro{maketitle/ddchead=\PMisc}}
  <\Option{ddchead}>
  [v2.03]
\begin{Declaration}
  {\Macro{maketitle/ddcfoot=\PMisc}}
  <\Option{ddcfoot}>
  [v2.03]
\printdeclarationlist[%
  \index{Titel|!(}%
  \index{Satzspiegel!doppelseitig}%
]
Der Befehl \Macro{maketitle} setzt für \Option{cdtitle=false} den normalen 
Titel von \KOMAScript, ansonsten wird die Titelseite im \TUDCD erzeugt. Die 
letztere Variante ist im Vergleich zum Standardtitel um eine Vielzahl von 
Feldern erweitert worden und erlaubt insbesondere die Angabe von Daten für das 
Deckblatt einer akademischen Abschlussarbeit. Die einzelnen Felder werden 
später in diesem \autorefname erläutert. Wird das Dokument doppelseitig und mit 
rechts öffnenden Kapiteln gesetzt,%
\footnote{%
  \KOMAScript-Optionen \Option{twoside=true} und \Option{open=right}, 
  Standard für \Class{tudscrbook}%
}
so wird zusätzlich die Option \Option{cleardoublespecialpage} einbezogen. Dies 
ist insbesondere bei den Befehlen \Macro{uppertitleback} beziehungsweise 
\Macro{lowertitleback} für die Titelrückseite zu beachten.

Das optionale Argument erlaubt~-- ebenso wie bei den \KOMAScript-Klassen~-- die
Änderung der Seitenzahl der Titelseite. Diese wird jedoch nicht ausgegeben, 
sondern beeinflusst lediglich die Zählung. Sie sollten hier unbedingt eine 
ungerade Zahl wählen, da sonst die gesamte Zählung durcheinander gerät. 
Wird eine Titelseite (\KOMAScript-Option \Option{titlepage=true}) im \TUDCD 
gesetzt (\Option{cdtitle=true}), können auch die weiterhin aufgeführten 
Parameter im optionalen Argument verwendet werden. Diese entsprechen in ihrem 
Verhalten den gleichnamigen Optionen respektive Befehlen, wirken sich jedoch 
nur lokal und einzig auf die Titelseite aus. So kann beispielsweise die Nutzung 
eines \DDC-Logos auf den Titel beschränkt bleiben.
\end{Declaration}
\end{Declaration}
\end{Declaration}
\end{Declaration}
\end{Declaration}
\end{Declaration}
\end{Declaration}
\end{Declaration}
\end{Declaration}
\end{Declaration}
\end{Declaration}

\begin{Declaration}
  {\Macro{maketitleonecolumn|\OPList{Seitenzahl}\MPName{Einspaltentext}}}
\printdeclarationlist[%
  \index{Satzspiegel!doppelseitig}%
  \index{Satzspiegel!zweispaltig}%
]
Im zweispaltigen Satz (Klassenoption~\Option{twocolumn}) wird mit 
\Macro{maketitle} die Titelseite selbst immer einspaltig gesetzt. Direkt nach 
dem Titel folgt normalerweise der zweispaltige Fließtext. Mit dem Befehl 
\Macro{maketitleonecolumn} kann nach dem Titel zusätzlich weiterer 
Inhalt~-- zum Beispiel eine Zusammenfassung respektive Kurzfassung~-- 
einspaltig gesetzt werden.

Wird der Befehl mit einem Titelkopf 
(\KOMAScript-Option~\Option{titlepage=false}) genutzt, so folgt diesem die 
einspaltige Textpassage aus dem obligatorischen Argument direkt, wobei 
gegebenenfalls bei entsprechendem Inhalt ein automatischer Seitenumbruch 
erfolgt. Danach wird direkt und ohne zusätzlichen Umbruch auf das zweispaltige 
Layout umgeschaltet. Bei einer aktivierten Titelseite (\Option{titlepage=true}) 
erfolgt die Ausgabe des Argumentes \PName{Einspaltentext} direkt nach dieser 
auf einer oder gegebenenfalls mehreren neuen Seiten ebenfalls einspaltig. 

Der optionale Parameter von \Macro{maketitleonecolumn} kann äquivalent zu 
\Macro{maketitle} für die Änderung der Seitenzahl, der verwendeten Schrift 
sowie zur Anpassung von Kopf und Fuß verwendet werden. Dabei ist zu beachten, 
dass ein Großteil der Parameter nur Auswirkungen haben, falls eine Titelseite
(\KOMAScript-Option \Option{titlepage=true}) verwendet wird.
\end{Declaration}

\begin{Declaration}
  {\Macro{makecover|\OPList{Seitenzahl}}}
  [
    v2.02:Umschlagseite für Layout ohne \CD hinzugefügt;
    v2.02:Unterstützung der Schriftelemente \nosuffix{%
        \Font{titlepage} und \Font{thesis} sowie der bereits durch \KOMAScript 
        bereitgestellten \Font{titlehead}, \Font{subject}, \Font{title}, 
        \Font{subtitle}, \Font{author} und \Font{publishers}};
  ]
\begin{Declaration}
  {\Macro{makecover/pagenumber=\PSet{Seitenzahl}}}
  [v2.02]
\begin{Declaration}
  {\Macro{makecover/cdgeometry=\PBoolean}}
\begin{Declaration}
  {\Macro{makecover/cdfont=\PMisc}}
  <\Option{cdfont}>
  [v2.02]
\begin{Declaration}
  {\Macro{makecover/cdhead=\PMisc}}
  <\Option{cdhead}>
  [v2.03]
\begin{Declaration}
  {\Macro{makecover/cdfoot=\PMisc}}
  <\Option{cdfoot}>
  [v2.03]
\begin{Declaration}
  {\Macro{makecover/headlogo=\PSet{Dateiname}}}
  <\Macro{headlogo}>
  [v2.03]
\begin{Declaration}
  {\Macro{makecover/footlogo=\PSet{Dateinamenliste}}}
  <\Macro{footlogo}>
  [v2.03]
\begin{Declaration}
  {\Macro{makecover/ddc=\PMisc}}
  <\Option{ddc}>
  [v2.03]
\begin{Declaration}
  {\Macro{makecover/ddchead=\PMisc}}
  <\Option{ddchead}>
  [v2.03]
\begin{Declaration}
  {\Macro{makecover/ddcfoot=\PMisc}}
  <\Option{ddcfoot}>
  [v2.03]
\printdeclarationlist[\index{Umschlagseite|!(}]
%
Eine Umschlagseite wird zumeist für gebundene Abschlussarbeiten verlangt, um 
diese beispielsweise für einen Prägedruck auf dem Buchdeckel zu verwenden. 
Deshalb ist die farbige Ausprägung der Umschlagseite auch deaktiviert, wenn 
diese für das restliche Dokument aktiv ist (\Option{cd=color}). Dies kann 
jedoch jederzeit mit \Option{cdcover=\PMisc} überschrieben werden.

Wird \Option{cdcover=true} gewählt, so wird die Umschlagseite im \TUDCD 
gesetzt. Auf dieser werden der Titel des Dokumentes, die Typisierung 
durch \Macro{thesis} und/oder \Macro{subject} sowie der Autor oder respektive 
die Autoren und gegebenenfalls der mit \Macro{publishers} angegebene Verlag 
ausgegeben.
\ChangedAt{v2.02}%
Für die Einstellung \Option{cdcover=false} wird lediglich der normale 
Titel von \KOMAScript als separate Umschlagseite ausgegeben. 

Die Titelseite selbst gehört immer zum Buchblock und sollte daher im gleichen 
Satzspiegel gesetzt werden. Dem entgegen steht die Umschlagseite, welche 
zumeist in einem anderen Layout erscheint. Normalerweise wird das Cover~-- 
unabhängig von der Option \Option{cdgeometry}~-- im asymmetrischen Satzspiegel 
des \CDs gesetzt. Mit \Macro||{makecover/cdgeometry=false} im optionalen 
Argument kann das Verhalten geändert werden. In diesem Fall erscheint auch die 
Umschlagseite im Buchblock des restlichen Dokumentes. Allerdings können für 
diese Einstellung die Seitenränder mit den Befehlen \Macro{coverpagetopmargin}, 
\Macro{coverpageleftmargin}, \Macro{coverpagerightmargin} sowie 
\Macro{coverpagebottommargin} durch den Nutzer frei angepasst werden. Mehr dazu 
ist im \scrguide zu finden.

Außerdem kann mit dem optionalen Argument die Seitenzahl der Umschlagseite 
geändert werden. Diese wird jedoch nicht ausgegeben, sondern beeinflusst 
lediglich die Zählung. Sie sollten hier unbedingt eine ungerade Zahl wählen, da 
sonst die gesamte Zählung durcheinander gerät. Die weiterhin aufgeführten 
Parameter entsprechen in ihrem Verhalten beziehungsweise ihrer Funktion den 
gleichnamigen Optionen respektive Befehlen, wirken sich jedoch nur lokal und 
einzig auf die Umschlagseite aus.%
\index{Umschlagseite|!)}%
\end{Declaration}
\end{Declaration}
\end{Declaration}
\end{Declaration}
\end{Declaration}
\end{Declaration}
\end{Declaration}
\end{Declaration}
\end{Declaration}
\end{Declaration}
\end{Declaration}

\begin{Declaration}
  {\Option(*){titlepage=\PMisc}}=true,false,firstiscover=
  (true|\Class{tudscrartcl}:false)
\begin{Declaration}
  {\Macro(*){title|\MPName{Titel}}}
\begin{Declaration}
  {\Macro(*){subtitle|\MPName{Untertitel}}}
\printdeclarationlist[%
  \index{Titel!Felder|(}%
]
Die Gestalt des Titels (alleinstehende Titelseite oder Titelkopf) wird mit der 
Option \Option{titlepage} gesetzt und ist im \scrguide dokumentiert. Darin sind 
ebenfalls die beiden, eigentlich selbsterklärenden Befehle \Macro{title} und 
\Macro{subtitle} erläutert. Anzumerken ist, dass die Schriftstärke des Titels 
von der Option \Option'full*'{headings} abhängt und der Untertitel \emph{immer} 
im (normal"~)fetten Schnitt erscheint. 

\end{Declaration}
\end{Declaration}
\end{Declaration}

\begin{Declaration}
  {\Font{titlepage}}
  [v2.02]
\begin{Declaration}
  {\Font{thesis}}
  [v2.02]
\begin{Declaration}
  {\Macro{raggedtitle}}
  [v2.06]
\printdeclarationlist[%
  \index{Schriftelemente}%
]%
Die \TUDScript-Klassen definieren diese neuen Schriftelemente. Dabei wird 
\Font{titlepage} auf der Titelseite für alle Felder verwendet, welche kein 
spezielles Schriftelement verwenden, welches ohnehin durch \KOMAScript 
bereitgestellt wird. Das mit \Macro{thesis} angegebene Feld, in welchem der Typ 
einer Abschlussarbeit angegeben wird, nutzt das Schriftelement~\Font{thesis}. 

\ChangedAt{v2.02}%
Für alle Felder des Titels und der Umschlagseite lassen sich die verwendeten
Schriften anpassen. In \autoref{sec:fonts:elements} lässt sich nachlesen, wie 
dies genau funktioniert. Dabei werden für Titel und Umschlagseite sowohl die 
bereits durch \KOMAScript bereitgestellten Schriftelemente \Font{titlehead}, 
\Font{subject}, \Font{title}, \Font{subtitle}, \Font{author}, \Font{date}, 
\Font{publishers} und \Font{dedication} als auch die neuen \Font{titlepage} 
sowie \Font{thesis} unterstützt.
\ChangedAt{v2.06}%
Der Befehl \Macro{raggedtitle} definiert die Ausrichtung des Titels, welcher 
standardmäßig linksbündig gesetzt wird.
%
\begin{Example}
Um die Einträge auf der Titelseite zentriert auszugeben, genügt folgende 
Definition:
\begin{Code}
\let\raggedtitle\centering
\end{Code}
\end{Example}
\end{Declaration}
\end{Declaration}
\end{Declaration}

\begin{Declaration}
  {\Macro(!){author|\MPName{Autor(en)}}}
\begin{Declaration}
  {\Macro{authormore|\MPName{Autorenzusatz}}}
\begin{Declaration}
  {\Macro{emailaddress|\OPName{Einstellungen}\MPName{E-Mail-Adresse}}}
  [
    v2.02;
    v2.05:optionales Argument zur Formatierung mit \Macro{hypersetup};
    v2.06:wird für alle Klassen bereitgestellt, Verwendung im Titel möglich;
  ]
\begin{Declaration}
  {\Macro{emailaddress*|\MPName{E-Mail-Adresse}}}
  [
    v2.05;
    v2.06:wird für alle Klassen bereitgestellt, Verwendung im Titel möglich;
  ]
\begin{Declaration}
  {\Macro{dateofbirth|\MPName{Geburtsdatum}}}
\begin{Declaration}
  {\Macro{placeofbirth|\MPName{Geburtsort}}}
\begin{Declaration}
  {\Macro{matriculationnumber|\MPName{Matrikelnummer}}}
\begin{Declaration}
  {\Macro{matriculationyear|\MPName{Immatrikulationsjahr}}}
\begin{Declaration}
  {\Macro{course|\MPName{Studiengang}}}
  [v2.05:Für Titel verwendbar;]
\begin{Declaration}
  {\Macro{discipline|\MPName{Studienrichtung}}}
  [
    v2.02;
    v2.05:Für Titel verwendbar
  ]
\printdeclarationlist[%
  \index{Autorenangaben|?}%
  \index{Kollaboratives Schreiben}%
  \index{Datum!Geburtsdatum|?}%
]
Mit dem Befehl \Macro{author} wird der Autor angegeben. Innerhalb des 
Argumentes können auch mehrere Autoren aufgeführt werden, wobei diese in diesem 
Fall jeweils mit \Macro{and} zu trennen sind. Alle weiteren hier vorgestellten 
Befehle können selbst im Argument von \Macro{author} verwendet werden, wodurch 
für jeden Autor individuelle Angaben möglich sind.

Mit \Macro{authormore} wird unter dem Autor eine Zeile ausgegeben, welche 
durch den Anwender frei belegt werden kann. Mit \Macro{emailaddress} kann für 
jeden Autor eine E"~Mail"=Adresse angegeben werden, welche als Hyperlink
definiert wird, falls das Paket \Package{hyperref} geladen wurde. Das optionale 
Argument wird an \Macro{hypersetup} aus besagtem Paket übergeben und kann somit 
für zusätzliche Einstellungen genutzt werden. Mit \Macro{emailaddress*} erfolgt 
keine Formatierung des Eintrags im Argument.

Sollte das Paket \Package{isodate} oder \Package{datetime2} geladen sein, wird 
die damit eingestellte Formatierung des Datums durch \Macro{dateofbirth}~-- wie 
übrigens bei jedem anderem Datumsfeld der \TUDScript-Klassen auch~-- für das 
Geburtsdatum auf dem Titel verwendet. Hierfür wird entweder \Macro{printdate} 
von \Package{isodate} oder \Macro{DTMDate} aus \Package{datetime2} genutzt. Mit 
dem Befehl \Macro{placeofbirth} lässt sich zusätzlich ein Geburtsort angeben.

Die weiteren Befehle als zusätzliche Angabe erklären sich quasi von selbst. 
Anzumerken ist, dass die mit den Befehlen \Macro{matriculationnumber},  
\Macro{matriculationyear}, \Macro{course} sowie \Macro{discipline} gemachten 
Angaben ebenfalls vom Paket \Package{tudscrsupervisor} innerhalb der 
Umgebung \Environment{task} genutzt werden, falls diese denn zum Einsatz kommt.
\end{Declaration}
\end{Declaration}
\end{Declaration}
\end{Declaration}
\end{Declaration}
\end{Declaration}
\end{Declaration}
\end{Declaration}
\end{Declaration}
\end{Declaration}

\begin{Declaration}
  {\Macro(!){and}}
\printdeclarationlist[%
  \index{Kollaboratives Schreiben|?}%
]
Dieser Befehl wird sowohl bei den \hologo{LaTeX}"=Standardklassen als auch bei 
den \KOMAScript-Klassen lediglich auf der Titelseite dazu verwendet, mehrere 
Autoren im Argument von \Macro{author} voneinander zu trennen.

Bei den \TUDScript-Klassen hingegen ist dieser Befehl derart in seiner Funktion 
erweitert worden, dass damit die Angabe einer kollaborativen Autorenschaft für 
Abschlussarbeiten innerhalb des Befehls \Macro{author} möglich ist. Weiterhin 
kann er noch im Argument von \Macro{supervisor}, \Macro{referee} sowie 
\Macro{advisor} verwendet werden, um mehrere Betreuer respektive Gutachter und 
Fachreferenten anzugeben. Er ist dabei nicht allein auf die Titelei beschränkt 
sondern lässt sich auch bei der Angabe von Personen in den entsprechenden 
Feldern der Umgebungen \Environment{task}, \Environment{evaluation} und 
\Environment{notice} aus dem Paket \Package{tudscrsupervisor} nutzen.
\end{Declaration}
%
\begin{Example}
Angenommen, es soll eine Abschlussarbeit von zwei unterschiedlichen Autoren in 
kollaborativer Gemeinschaft erstellt werden, so lässt sich die Autorenangaben 
folgendermaßen gestalten:
\begin{Code}
\author{%
  Mickey Mouse%
  \matriculationnumber{12345678}%
  \dateofbirth{2.1.1990}%
  \placeofbirth{Dresden}%
\and%
  Donald Duck%
  \matriculationnumber{87654321}%
  \dateofbirth{1.2.1990}%
  \placeofbirth{Berlin}%
}
\matriculationyear{2010}
\end{Code}
Alle zusätzlichen Angaben außerhalb des Argumentes von \Macro{author} werden 
für beide Autoren gleichermaßen übernommen. Angaben innerhalb des Argumentes 
von \Macro{author} werden den jeweiligen, mit \Macro{and} getrennten Autoren 
zugeordnet. Mehr dazu ist im Minimalbeispiel in \autoref{sec:exmpl:thesis}.
\end{Example}

\begin{Declaration}
  {\Macro(!){date|\OPList{Suffix}\MPName{Datum}}}
  [v2.05:Angabe von Parametern für Prä- und Suffix bei Datumsausgabe möglich;]
\begin{Declaration}
  {\Macro(!){date*|\OPList{Suffix}\MPName{Datum}}}
  [v2.05]
\begin{Declaration}
  {\Macro(!){date/before=\PSet{Präfix}}}
  [v2.05]
\begin{Declaration}
  {\Macro(!){date/after=\PSet{Suffix}}}
  [v2.05]
\begin{Declaration}
  {\Macro(!){date/place=\PSet{Ort}}}
  [v2.05]
\begin{Declaration}
  {\Macro{defensedate|\MPName{Verteidigungsdatum}}}
\printdeclarationlist[%
  \index{Datum|?}%
  \index{Datum!Abgabedatum|?}%
  \index{Datum!Verteidigungsdatum|?}%
]
Mit dem Befehl \Macro{date} lässt sich das Datum angegeben. 
\ChangedAt{v2.05}%
Über das optionale Argument können die beiden Parameter \Macro{date/before} und 
\Macro{date/after} genutzt werden, um ergänzende Angaben vor beziehungsweise 
nach dem eigentlichen Datum auszugeben. Die Sternversion \Macro{date*} setzt 
den mit \Macro{place} angegebenen Ort vor das Datum. Dies geschieht auch für 
die normale Version von \Macro{date}, wenn der Parameter \Macro{date/place} 
verwendet wird.

Das Datum wird bei normalen Dokumenten direkt nach dem Autor respektive den 
Autoren ausgegeben. Bei Abschlussarbeiten~-- aktiviert durch die Verwendung von 
\Macro{thesis} oder \Option{subjectthesis} in Verbindung mit \Macro{subject}~-- 
erscheint dieses am Ende der Titelseite als Abgabedatum. Außerdem lässt sich 
für diesen Fall mit dem Befehl \Macro{defensedate} das Datum der Verteidigung 
angeben, wie es beispielsweise bei dem Druck von Dissertationen üblich ist.

Sollte eines der Pakete \Package{isodate} oder \Package{datetime2} geladen 
sein, so wird mit \Macro{printdate} beziehungsweise \Macro{DTMDate} die durch 
das jeweilige Paket eingestellte Ausgabeformatierung des Datums für alle 
Datumsfelder des Dokumentes und folglich auch für die beiden Felder 
\Macro{date} und \Macro{defensedate} verwendet.
\end{Declaration}
\end{Declaration}
\end{Declaration}
\end{Declaration}
\end{Declaration}
\end{Declaration}

\begin{Declaration}
  {\Macro{thesis|\MPName{Typisierung}}}
\begin{Declaration}
  {\Macro(!){subject|\MPName{Typisierung}}}
\printdeclarationlist[%
  \index{Abschlussarbeit|!}%
  \index{Typisierung}%
]
Mit diesen Befehlen kann der Typ des Dokumentes respektive der Abschlussarbeit 
angegeben werden. Während \Macro{thesis} den Inhalt des Feldes unter dem Titel 
vertikal zentriert auf der Titelseite ausgibt, erscheint der Inhalt von 
\Macro{subject} oberhalb des Titels. Es können auch beide Befehle parallel mit 
unterschiedlichen Inhalten verwendet werden. Sie dienen den \TUDScript-Klassen 
außerdem zum Erkennen von Abschlussarbeiten, da für diese spezielle Felder 
bereitgehalten werden und auch die Titelseite in diesem Fall leicht geändert 
gesetzt wird.

Des Weiteren ist es bei beiden Befehlen möglich, spezielle Werte als Argument 
zur Typisierung des Dokumentes zu verwenden. Diese werden entsprechend der 
gewählten Dokumentensprache~-- entweder Deutsch oder Englisch~-- entschlüsselt 
und gesetzt. Die möglichen Werte sind \autoref{tab:thesis} zu entnehmen. Dabei 
ist zu beachten, dass das Setzen eines speziellen Wertes für \emph{entweder} 
\Macro{thesis} \emph{oder} \Macro{subject} möglich ist. Die Verwendung eines 
der genannten Werte führt immer dazu, dass das Dokument als Abschlussarbeiten 
erkannt und die erweiterte Titelseite aktiviert wird. Sollte vom Anwender kein 
explizites Verhalten für \Option{subjectthesis} definiert sein, so führt die 
Verwendung von \Macro{thesis|\MPName{Wert}} zu \Option{subjectthesis=false} 
und \Macro{subject|\MPName{Wert}} zu \Option{subjectthesis=true}.
\ToDo[doc]{thesis-Liste auslesen/verwenden, sobald implementiert}[v2.07]
%
\begin{table}
\caption{%
  Spezielle Werte zur Typisierung des Dokumentes für
  \Macro{thesis} und \Macro{subject}%
  \label{tab:thesis}%
  \index{Bezeichner}%
  \index{Typisierung}%
}%
\centering%
\newcommand*\typecast[2]{%
  \PValue{#1} & \Term{#2} & \csuse{#2} & \selectlanguage{english}\csuse{#2}
  \tabularnewline%
}%
\begin{tabular}{llll}
  \toprule
  \textbf{Wert} & \textbf{Bezeichner} & \textbf{Deutsch} & \textbf{Englisch}
  \tabularnewline
  \midrule
  \typecast{diss/phd}{dissertationname}
  \typecast{diploma}{diplomathesisname}
  \typecast{master}{masterthesisname}
  \typecast{bachelor}{bachelorthesisname}
  \typecast{student}{studentthesisname}
  \typecast{evidence}{studentresearchname}
  \typecast{project}{projectpapername}
  \typecast{seminar}{seminarpapername}
  \typecast{term}{termpapername}
  \typecast{research}{researchname}
  \typecast{log}{logname}
  \typecast{report}{reportname}
  \typecast{internship}{internshipname}
  \bottomrule
\end{tabular}
\end{table}
\end{Declaration}
\end{Declaration}

\begin{Declaration}
  {\Option{subjectthesis=\PBoolean}}
  (false|\Macro{subject|\MPName{\autoref{tab:thesis}}}:true)
\printdeclarationlist
%
Der Befehl \Macro{thesis} dient den \TUDScript-Hauptklassen zur Unterscheidung 
zwei unterschiedlicher Ausprägungen der Titelseite und ist speziell für 
Abschlussarbeiten gedacht. Außerdem kann bei der Nutzung spezieller Werte 
aus \autoref{tab:thesis} innerhalb des Argumentes von \Macro{subject} ebenfalls 
das Verhalten für Abschlussarbeiten aktiviert werden, wobei hierdurch die 
Einstellung \Option{subjectthesis=true} automatisch vorgenommen wird.

Für den Standardfall~-- bekanntlich \Option{subjectthesis=false}~-- wird der 
durch \Macro{thesis} gegebene Typ der Abschlussarbeit sowie der gegebenenfalls 
durch \Macro{graduation} gesetzte angestrebte Abschluss in großen Lettern und 
sehr zentral auf der Titelseite gesetzt. Die Verwendung von \Macro{subject} ist 
hierbei weiterhin möglich.
%
Wird die Option mit \Option{subjectthesis=true} aktiviert, so wird die mit 
\Macro{thesis} gesetzte Bezeichnung nicht unterhalb sondern oberhalb des Titels 
an der Stelle von \Macro{subject} ausgegeben. Der mit \Macro{graduation} 
angegebene Abschluss wird weiterhin unter dem Titel, allerdings in schlankerer 
Schrift gesetzt. Eine etwaige Verwendung des Befehls \Macro{subject} wird in 
diesem Fall ignoriert.
\begin{DeclareValues}
\itemval=false=
  Die Ausgabe des Typs der Abschlussarbeit (\Macro{thesis}) selbst sowie des 
  angestrebten Abschlusses (\Macro{graduation}) erfolgt in großen Lettern 
  zentral auf der Titelseite.
\itemval*=true=
  Der Typ der Abschlussarbeit (\Macro{thesis}) wird oberhalb des Titels in der 
  Betreffzeile gesetzt. Der angestrebte Abschluss (\Macro{graduation}) wird 
  zentral ausgegeben.
\end{DeclareValues}
\end{Declaration}

\begin{Declaration}
  {\Option{titlesignature=\PBoolean}}
  (false)
  [v2.06]
\printdeclarationlist
%
Einige Lehrstühle an der \TnUD verlangen eine Unterschrift des Autors einer 
Abschlussarbeit direkt auf der Titelseite. Mit dem Aktivieren dieser Option 
wird ein solches Feld am unteren Seitenrand des Titels erzeugt.
\end{Declaration}

\begin{Declaration}
  {\Macro{graduation|\OPName{Kurzform}\MPName{Grad}}}
  [v2.02]
\printdeclarationlist
%
Mit diesem Befehl wird der angestrebte akademische Grad auf der Titelseite 
ausgegeben. Da dies nur mit einer Abschlussarbeit erreicht werden kann erfolgt 
die Ausgabe nur, wenn entweder \Macro{thesis} oder \Macro{subject} verwendet 
wurde, wobei bei letzterem Befehl im Argument zwingend ein Wert aus 
\autoref{tab:thesis} verwendet werden muss.

Bei der Ausgabe des akademischen Grades hat die Option \Option{subjectthesis} 
Einfluss auf die Ausgabe auf der Titelseite. Bei \Option{subjectthesis=false} 
wird der Abschluss~-- ähnlich wie der Typ der Abschlussarbeit~-- zentral und in 
relativ großen Lettern gesetzt. Für \Option{subjectthesis=true} erfolgt die 
Ausgabe kleiner und in weniger starker Schriftstärke.
\end{Declaration}

\begin{Declaration}
  {\Macro{supervisor|\MPName{Name(n)}}}
\begin{Declaration}
  {\Macro{referee|\MPName{Name(n)}}}
\begin{Declaration}
  {\Macro{advisor|\MPName{Name(n)}}}
\begin{Declaration}
  {\Macro{professor|\MPName{Name}}}
\printdeclarationlist[%
  \index{Betreuer|?}%
  \index{Gutachter|?}%
  \index{Referent|?}%
]
Mit \Macro{supervisor}, \Macro{referee} und \Macro{advisor} werden die Betreuer 
einer Abschlussarbeit beziehungsweise die Gutachter und Fachreferenten einer 
Dissertation angegeben. Zusätzlich kann mit \Macro{professor} der betreuende 
Hochschullehrer beziehungsweise die betreuenden Professoren für studentische 
Arbeiten angegeben werden. Die Angabe mehrerer Person erfolgt wie beim Befehl 
\Macro{author} durch die Trennung mittels \Macro{and}.
\end{Declaration}
\end{Declaration}
\end{Declaration}
\end{Declaration}

\begin{Declaration}
  {\Macro{titledelimiter|\MPName{Trennzeichen}}}
\printdeclarationlist[%
  \index{Titel!Trennzeichen}%
]%
Für den Titel und die Umschlagseite werden durch die \TUDScript-Klassen eine 
Reihe von zusätzlichen Feldern bereitgestellt. Einigen dieser Felder wird eine 
Beschreibung (\autoref{sec:localization}) vorangestellt. Dazwischen 
wird bei der Ausgabe ein Trennzeichen eingefügt. Ein Doppelpunkt gefolgt von 
einem geschützten Leerzeichen (:\Macro*{nobreakspace}) ist hierfür die 
Voreinstellung. Mit dem Befehl \Macro{titledelimiter} lässt sich dieses 
Trennzeichen beliebig an die individuellen Wünsche des Anwenders anpassen.
\index{Titel!Felder|)}%
\end{Declaration}

\begin{Declaration}
  {\Macro(*){extratitle|\MPName{Schmutztitel}}}
\begin{Declaration}
  {\Macro(*){frontispiece|\MPName{Frontispiz}}}
\begin{Declaration}
  {\Macro(*){titlehead|\MPName{Kopf}}}
\begin{Declaration}
  {\Macro(*){publishers|\MPName{Verlag}}}
\begin{Declaration}
  {\Macro(*){thanks|\MPName{Fußnote}}}
\begin{Declaration}
  {\Macro(*){uppertitleback|\MPName{Titelrückseitenkopf}}}
\begin{Declaration}
  {\Macro(*){lowertitleback|\MPName{Titelrückseitenfuß}}}
\begin{Declaration}
  {\Macro(*){dedication|\MPName{Widmung}}}
\printdeclarationlist(\seeplain{\scrguide})
%
Diese Befehle entsprechen den in ihrem Verhalten den originalen Pendants der 
\KOMAScript-Klassen und sollen hier der Vollständigkeit halber erwähnt werden.

Die Ausgabe des mit \Macro{extratitle} definierten Schmutztitels~-- welcher 
beliebig gestaltet und formatiert werden kann~-- und der gegebenenfalls mit 
\ChangedAt<\Macro{frontispiece}>{v2.06:Schmutztitelrückseite wird unterstützt}%
\Macro{frontispiece} definierten Rückseite erfolgt als Bestandteil der Titelei 
mit \Macro{maketitle} vor der eigentlichen Titelseite. Mit dem Befehl 
\Macro{titlehead} kann zusätzlich ein beliebig formatierter Titelkopf oberhalb 
der Typisierung und des Titels ausgegeben werden. Da die vertikale Position des 
Dokumenttitels durch das \CD fixiert ist, kann es~-- im Gegensatz zu den 
\KOMAScript-Klassen~-- passieren, dass der Kopf des Haupttitels selbst in die 
Kopfzeile ragt. Dies wird durch die \TUDScript-Klassen nicht geprüft und muss 
gegebenenfalls vom Anwender kontrolliert werden. Mit \Macro{publishers} lassen 
sich ein Verlag oder auch andere Informationen am Ende der Titelseite ausgeben.

Fußnoten werden auf dem Titel nicht mit \Macro{footnote}, sondern mit der 
Anweisung \Macro{thanks} erzeugt. Diese dienen in der Regel für Anmerkungen bei 
Titel oder den Autoren. Als Fußnotenzeichen werden dabei Symbole statt Zahlen 
verwendet. Der Befehl \Macro{thanks} kann nur innerhalb des Arguments einer 
Anweisung für den Titel wie beispielsweise \Macro{author} oder \Macro{title} 
verwendet werden.
\index{Satzspiegel!doppelseitig}%
Im doppelseitigen Druck lässt sich die Rückseite der Haupttitelseite für 
weitere Angaben nutzen. Sowohl den Titelrückseitenkopf als auch den
Titelrückseitenfuß kann der Anwender mit \Macro{uppertitleback} und 
\Macro{lowertitleback} frei gestalten. Mit \Macro{dedication} lässt eine 
separate Widmungsseite zentriert und in etwas größerer Schrift setzen. Die 
Rückseite ist~-- wie auch die des Schmutztitels~-- grundsätzlich leer. Die 
Widmung wird mit der restlichen Titelei ausgegeben und muss daher vor der 
Nutzung von \Macro{maketitle} angegeben werden.
\index{Titel|!)}%
\end{Declaration}
\end{Declaration}
\end{Declaration}
\end{Declaration}
\end{Declaration}
\end{Declaration}
\end{Declaration}
\end{Declaration}



\subsection{%
  Die Teileseite%
  \label{sec:part}%
}
%
\ChangedAt<\Macro{partpagestyle}>{%
  v2.02:Seitenstil \nosuffix{\PageStyle{plain.tudheadings}} wird genutzt%
}%
Ist für die Teileseiten das Layout des \CDs aktiviert, so wird der Seitenstil 
dieser (\Macro{partpagestyle}) auf \PageStyle{plain.tudheadings} gesetzt. 
Möchten Sie stattdessen einen anderen Seitenstil nutzen, so kann dieser mit 
\Macro*{renewcommand|*\MPValue{\Macro{partpagestyle}}\MPName{Seitenstil}}
angepasst werden.

\begin{Declaration}
  {\Macro{setpartsubtitle|\MPName{Untertitel}}}
  [v2.06]
\begin{Declaration}
  {\Font{partsubtitle}}
  [v2.06]
\printdeclarationlist[%
  \index{Layout!Teileseiten|?}%
  \index{Layout!Überschriften}%
  \index{Schriftelemente}%
]
Mit \Macro{setpartsubtitle|\MPName{Untertitel}} kann für Teile nach der 
Überschrift selbst ein Untertitel gesetzt werden. Der Befehl muss in gleicher 
Weise wie \Macro{setpartpreamble} \emph{vor} der Verwendung von \Macro{part} 
oder den davon abgeleiteten Varianten angegeben werden. Mit dem Schriftelement 
\Font{partsubtitle} lässt sich die Schrift für den gegebenen Untertitel 
verändern. In \autoref{sec:fonts:elements} ist zu finden, wie es angepasst 
werden kann.
\end{Declaration}
\end{Declaration}

\begin{Declaration}
  {\Option{parttitle=\PBoolean}}
  (false)
\printdeclarationlist[%
  \index{Layout!Teileseiten|?}%
  \index{Layout!Überschriften}%
]
Diese Option ermöglicht es, den mit \Macro{title} gegebenen Titel des 
Dokumentes selbst in großer Schrift auf einer Teileseite auszugeben, die 
Bezeichnung des mit \Macro{part|\MPName{Bezeichnung}} erzeugten Teils wird in 
diesem Fall als Untertitel direkt darunter gesetzt. Diese Variante des Layouts 
findet sich im Handbuch für das \TUDCD. \notudscrartcl
\begin{DeclareValues}
\itemval=false=
  Die Bezeichnung des Teils erscheint in großer Schrift, der Titel des 
  Dokumentes wird nicht ausgegeben.
\itemval*=true=
  Der angegebene Titel wird in großer Auszeichnung auf der Teileseite gesetzt,
  die Bezeichnung des Teils selber als Untertitel.
\end{DeclareValues}
\end{Declaration}



\subsection{%
  Die Kapitelseite%
  \label{sec:chapter}%
  \index{Layout!Kapitelseiten|?}%
  \index{Layout!Überschriften}%
}
%
\begin{Declaration}
  {\Option{chapterpage=\PBoolean}}
  (false|\Option{cd=color}:true)
\printdeclarationlist[%
  \index{Satzspiegel!doppelseitig}%
  \index{Vakatseiten}%
]
Mit dieser Einstellung kann die Überschrift eines Kapitels separat auf einer 
Seite ausgegeben werden. Der nachfolgende Text wird auf der nächsten 
beziehungsweise bei doppelseitigem Satz und rechts öffnenden Kapiteln%
\footnote{%
  \KOMAScript-Optionen \Option{twoside=true} und \Option{open=right}, 
  Standard für \Class{tudscrbook}%
}
auf der übernächsten Seite ausgegeben. Die in diesem Fall erzeugte Rückseite 
wird in ihrer Ausprägung~-- wie auch Teileseiten~-- durch die Einstellung von 
\Option{cleardoublespecialpage} bestimmt. Beim farbigen Layout ist diese Option 
standardmäßig aktiviert. \notudscrartcl
\begin{DeclareValues}
\itemval=false=
  Es gibt keine Sonderstellung von Kapiteln, der nachfolgende Text wird direkt 
  unter der Überschrift respektive nach der mit \Macro{setchapterpreamble} 
  erzeugten Kapitelpräambel auf der gleichen Seite ausgegeben.
\itemval*=true=<!seealso!:\Option{cleardoublespecialpage=\PMisc}>
  Die Kapitelüberschrift~-- und gegebenenfalls auch die Kapitelpräambel~-- 
  werden auf einer separaten Seite gesetzt. Der auszugebende Text zum Beginn 
  des Kapitels erscheint auf der nächsten respektive übernächsten Seite.
\end{DeclareValues}

Beim Einsatz von Kapitelseiten (\Option{chapterpage=true}) ist das Aktivieren 
der \KOMAScript-Option \InlineDeclaration{\Option(*){chapterprefix=\PBoolean}} 
empfehlenswert. Damit werden die Kapitelüberschriften mit einer Vorsatzzeile 
gesetzt. Wird ein nummeriertes Kapitel erzeugt, so wird zunächst in einer Zeile 
\enquote{Kapitel} gefolgt von der aktuellen Kapitelnummer ausgegeben, in der 
nächsten Zeile wird anschließend die eigentliche Überschrift in linksbündigem 
Flattersatz ausgegeben. Mehr dazu ist der \scrguide[\KOMAScript-Dokumentation] 
zu entnehmen. Der Seitenstil von Kapiteln lässt sich übrigens mit 
\Macro*{renewcommand|*\MPValue{\Macro{chapterpagestyle}}\MPName{Seitenstil}}
anpassen~-- unabhängig von der Option \Option{chapterpage}.
\end{Declaration}

\begin{Declaration}
  {\Macro{setchaptersubtitle|\MPName{Untertitel}}}
  [v2.06]
\begin{Declaration}
  {\Font{chaptersubtitle}}
  [v2.06]
\printdeclarationlist[%
  \index{Layout!Kapitelseiten|?}%
  \index{Layout!Überschriften}%
  \index{Schriftelemente}%
]
Mit \Macro{setchaptersubtitle|\MPName{Untertitel}} kann für Kapitel nach der 
Überschrift selbst ein Untertitel gesetzt werden. Der Befehl muss in gleicher 
Weise wie \Macro{setchapterpreamble} \emph{vor} der Nutzung von \Macro{chapter} 
oder den davon abgeleiteten Varianten angegeben werden. Mit dem Schriftelement 
\Font{chaptersubtitle} lässt sich die Schrift für den gegebenen Untertitel 
verändern. In \autoref{sec:fonts:elements} ist zu finden, wie es angepasst 
werden kann.
\end{Declaration}
\end{Declaration}



\ToDo[doc]{normale Deklaration (Tabelle) für parskip und open=any,left,right?}
\ToDo[doc]{normale Deklaration (Tabelle) für cleardoublepage?}
\subsection{%
  Vakatseiten%
  \label{sec:vacat}%
  \index{Vakatseiten|(}%
}
%
Automatisch erzeugte Vakatseiten~-- auch absichtliche Leerseiten genannt~-- 
können nur im doppelseitigem Satz (\Option{twoside=true}) vor dem Beginn eines 
Teils oder Kapitels abhängig von der \KOMAScript-Option 
\InlineDeclaration{\Option(*){open=\PMisc}} auftreten. Wird ein neuer Teil oder 
ein neues Kapitel immer auf einer rechten Seite erzeugt (\Option{open=right}),%
\footnote{%
  Voreinstellungen
  \Class{tudscrbook}: 
  \Option{twoside=true}, \Option{open=right};
  \Class{tudscrartcl}, \Class{tudscrreprt}: 
  \Option{open=any}, \Option{twoside=false}%
}
so wird gegebenenfalls~-- der zuvor gesetzte Inhalt endete auf einer rechten 
Seite~-- eine Vakatseite erzeugt. Für diese kann mit der \KOMAScript-Option 
\InlineDeclaration{\Option(*){cleardoublepage=\PMisc}} der Seitenstil 
eingestellt werden.

\begin{Declaration}
  {\Option{cleardoublespecialpage=\PMisc}}
  (true)
  [v2.06:Einstellung für farbige Rückseiten]
\printdeclarationlist[%
  \index{Titel}%
  \index{Layout!Teileseiten}%
  \index{Layout!Kapitelseiten}%
  \index{Layout!Rückseiten}%
  \index{Satzspiegel!doppelseitig}%
]
Diese Option wirkt sich lediglich im gerade zuvor beschriebenen Fall des 
doppelseitigen Satzes und ausschließlich rechts eröffnenden Seiten für Teile 
beziehungsweise Kapitel aus.%
\footnote{%
  \KOMAScript-Option \Option{twoside=true} und \Option{open=right}%
}
Unter diesen Umständen kann der Stil der darauffolgenden, linken Seite~-- 
sprich der Rückseite~-- beeinflusst werden. Das Normalverhalten sieht vor, dass 
nach einem Teil die nachfolgende Rückseite unabhängig von der Einstellung für 
\Option{cleardoublepage} immer als vollständig leere Seite ohne Kopf- oder 
Fußzeilen gesetzt wird.

Diese Option erlaubt es, das Normalverhalten zu deaktivieren und für die Seite 
nach der Teileseite~-- und abhängig von \Option{chapterpage} auch nach einem 
Kapitelanfang auf einer separaten Seite~-- den Seitenstil der Option 
\Option{cleardoublepage} zu übernehmen. Des Weiteren kann auch ein anderer, 
bereits definierter Seitenstil gewählt werden. Außerdem kann im farbigen Layout 
die Rückseite in der gleichen Farbe wie die Vorderseite von Titel, Teil oder 
Kapitel gesetzt werden. \notudscrartcl
\begin{DeclareValues}
\itemval=false=
  Die Rückseiten sind vollständig leer, unabhängig von der Option
  \Option{cleardoublepage}.
\itemval*=true=
  Der Seitenstil der Rückseite von Teilen und gegebenenfalls Kapiteln entspricht
  der Einstellung von \Option{cleardoublepage} für Vakatseiten.
\itemval=current=
  Für die erzeugte Rückseite wird der aktuell definierte Seitenstil verwendet.
\itemval=\PName{Seitenstil}=
  Mit der Angabe von \Option{cleardoublespecialpage=\PName{Seitenstil}} 
  kann ein beliebiger, bereits definierter Seitenstil für die Rückseite nach 
  Teilen und Kapiteln verwendet werden.
\itemval=color=<\Macro{uppertitleback},\Macro{lowertitleback}>
  Im farbigen Layout (\seeplain{\Option'page'(!){cd=\PMisc}}) ist auch die 
  Rückseite von Teilen beziehungsweise Kapiteln farbig. Weiterhin wirkt sich 
  die Einstellung ebenfalls auf die Rückseite des Titels aus.
\itemval=nocolor=
  Es werden weiße Rückseiten bei Titel, Teilen und gegebenenfalls Kapiteln 
  erzeugt.
\end{DeclareValues}
\index{Vakatseiten|)}%
\end{Declaration}



\subsection{%
  Verwendung von Schriftelementen%
  \label{sec:fonts:elements}%
  \index{Schriftelemente|!}%
}
%
Für die \TUDScript-Klassen werden weitere Schriftelemente~-- in Ergänzung zu 
den bereits durch \KOMAScript bereitgestellten~-- definiert. Dies sind 
beispielsweise \Font{titlepage}, \Font{thesis}, \Font{tudheadings} sowie 
\Font{partsubtitle} und \Font{chaptersubtitle}. Sowohl die durch \KOMAScript 
definierten als auch die hier genannten und alle folgend erläuterten 
Schriftelemente lassen sich im Bedarfsfall mit 
\Macro{addtokomafont|\MPName{Schriftelement}\MPName{Einstellungen}} anpassen. 
Mehr dazu ist im \scrguide innerhalb des Abschnitts \emph{Textauszeichnungen} 
zu finden.

\Attention{%
  Da \TUDScript ebenfalls Gebrauch einigen dieser Schriftelemente macht, sollte 
  bei einer Verwendung in der Präambel das Ausführen bis zum Beginn des 
  Dokumentes verzögert werden.
}
\begin{quoting}[rightmargin=0pt]
\begin{Code}[escapechar=§]
\AtBeginDocument{\addtokomafont§\MPName{Schriftelement}\MPName{Einstellungen}§}
\end{Code}
\end{quoting}



\subsection{%
  Verwendung von Feldinhalten%
  \index{Felder|!}%
}
%
\begin{Declaration}
  {\Macro{getfield|\MPName{Feldname}}}
  [v2.06]
\printdeclarationlist
%
Mit diesem Makro kann auf den Inhalt eines zuvor angegebenen Feldes zugegriffen 
werden, welches normalerweise für Titel oder Seitenkopf verwendet wird. 
%
\begin{Example}
Das für dieses Dokument angegebene Datum lautet \enquote{\getfield{date}}. Es 
kann wie folgt ausgegeben werden:
\begin{Code}
\getfield{date}
\end{Code}
\end{Example}
%
Weitere mögliche Argumente sind beispielsweise \PValue{faculty}, 
\PValue{department}, \PValue{institute} und \PValue{chair} sowie \PValue{title},
\PValue{author}, \PValue{thesis}, \PValue{professor}, \PValue{supervisor} oder
\PValue{place}.
\end{Declaration}



\subsection{%
  Die Farben des \CDs%
  \index{Layout!Farben}%
}
%
Zur Verwendung der Farben des \CDs wird das Paket \Package{tudscrcolor} 
genutzt. Falls dieses nicht in der Präambel geladen wird~-- um beispielsweise 
zusätzliche Optionen aufzurufen~-- binden die \TUDScript-Klassen dieses 
automatisch ein. Detaillierte Informationen sind in der Dokumentation von 
\Package'full'{tudscrcolor} zu finden.%
\index{Layout|)}%



\section{Zusätzliche Optionen und Erweiterungen}
%
\ChangedAt*{%
  v2.03:Bugfix für \abstractname, \confirmationname\space und 
  \blockingname\space bei der Festlegung von Seitenstil und Kolumnentitel%
}%
Neben den Befehlen für die Anpassung des Layouts an das \TUDCD stellen die 
\TUDScript-Klassen weitere Befehle und Umgebungen zur Verfügung, um die 
Anwendung insbesondere für wissenschaftliche Arbeiten zu erleichtern.



\subsection{%
  Zusammenfassung/Kurzfassung%
  \index{Zusammenfassung|!(}%
}
%
\begin{Declaration}
  {\Option{abstract=\PMisc}}
  (false,notoc|\Class{tudscrbook}:chapter,toc)
\printdeclarationlist[%
  \index{Satzspiegel!zweispaltig}%
]
Diese Option wird für die Klassen \Class{scrartcl} und \Class{scrreprt} 
standardmäßig bereitgestellt. Für die Klasse \Class{scrbook} geschieht dies 
nicht. Dazu heißt es im Handbuch:
%
\begin{quoting}
Bei Büchern wird in der Regel eine andere Art der Zusammenfassung verwendet. 
Dort wird ein entsprechendes Kapitel an den Anfang oder Ende des Werks gesetzt. 
Oft wird diese Zusammenfassung entweder mit der Einleitung oder einem weiteren 
Ausblick verknüpft. Daher gibt es bei \Class{scrbook} generell keine 
\Environment{abstract}"~Umgebung. Bei Berichten im weiteren Sinne, etwa einer 
Studien- oder Diplomarbeit, ist ebenfalls eine Zusammenfassung in dieser Form 
zu empfehlen.
\end{quoting}
%
Durch die \TUDScript-Klassen wird die Option~\Option{abstract} erweitert. 
Neben den Auswahlmöglichkeit, welche bereits \KOMAScript für die Klassen 
\Class{tudscrartcl} und \Class{tudscrreprt} anbietet, kann die Überschrift für 
die Zusammenfassung außerdem in Gestalt eines Unterkapitels oder für 
\Class{tudscrreprt} und \Class{tudscrbook} in der Form eines Kapitels 
ausgegeben werden.
\begin{DeclareValues}
\itemval=false=<!not!:\Class'plain'{tudscrbook}>
  Es wird keine Überschrift für die \Environment{abstract}"~Umgebung ausgegeben.
\itemval*=true=<!not!:\Class'plain'{tudscrbook}>
  Wie bei den \KOMAScript-Klassen wird eine zentrierte Überschrift mit dem 
  Bezeichner \Term{abstractname} vor der eigentlichen Zusammenfassung gesetzt.
\itemval=chapter,addchap=(\Class{tudscrbook})<!not!:\Class'plain'{tudscrartcl}>
  Es wird der Befehl \Macro{chapter} für das Setzen der Überschrift 
  (\Term{abstractname}) genutzt. 
\itemval=section,addsec=
  Die Überschrift (\Term{abstractname}) verwendet den Gliederungsbefehl 
  \Macro{section}.
\itemval=heading=
  Es wird die höchstmögliche Gliederungsebene verwendet. Für 
  \Class{tudscrartcl} entspricht dies \Option{abstract=section}, bei 
  \Class{tudscrreprt} und \Class{tudscrbook} \Option{abstract=chapter}.
\end{DeclareValues}

Abhängig von der gewählten Gliederungsebene der Überschrift wird das Verhalten 
für das Setzen eines Eintrages ins Inhaltsverzeichnis festgelegt. Ohne oder mit 
zentrierter Überschrift wird per Voreinstellung kein Eintrag erzeugt. Wird die 
Überschrift jedoch in Form einer Gliederungsebene gewählt, so erscheint die 
Zusammenfassung für gewöhnlich im Inhaltsverzeichnis auf der obersten Ebene. 
Das voreingestellte Verhalten für die Einträge ins Inhaltsverzeichnis kann 
jederzeit mit folgenden Werten durch den Anwender überschrieben werden.
\begin{DeclareValues}
\itemval=notoc,nottotoc=
  Die Zusammenfassung wird definitiv nicht ins Inhaltsverzeichnis eingetragen.
\itemval=toc,totoc=
  Es wird auf der obersten Gliederungsebene der aktuell verwendeten 
  Dokumentklasse (\Macro||{chapter} oder \Macro||{section}) ein nicht 
  nummerierten Eintrag im Inhaltsverzeichnis für die Zusammenfassung gesetzt.
\itemval=tocleveldown,leveldown,totocleveldown=[v2.02]
  Der Inhaltsverzeichniseintrag wird eine Stufe unterhalb der obersten 
  Gliederungsebene (\Macro||{section} oder \Macro||{subsection}) erzeugt.
\itemval=tocmultiple,totocmultiple,tocaggregate,totocaggregate=[v2.04]
  Es wird ein \emph{einziger} Inhaltsverzeichniseintrag für \emph{alle} 
  Zusammenfassungen erstellt.
\end{DeclareValues}

Außerdem lässt sich das Verhalten für die Kolumnentitel durch den Nutzer 
beeinflussen. Diese werden normalerweise automatisch gesetzt, wenn diese über 
die Option \Option{automark} des Paketes \Package{scrlayer-scrpage} aktiviert 
wurden und sind von der aktuellen Gliederungsebene der Überschrift abhängig. 
Werden jedoch mit Hilfe der Option \Option{manualmark} manuelle Kolumnentitel 
genutzt, müssen diese normalerweise auch für die Zusammenfassung durch den 
Anwender gesetzt werden. Mit \Option{abstract=markboth} lässt sich allerdings 
das Setzen der Kolumnentitel unabhängig davon forcieren.
\begin{DeclareValues}
\itemval=markboth=[v2.02]
  Unabhängig von der Verwendung manueller oder automatischer Kolumnentitel 
  werden diese auf rechten sowie linken Seiten mit \Term{abstractname} gesetzt.
\itemval=nomarkboth=[v2.02]
  Die Einstellung für manuelle oder automatische Kolumnentitel werden beachtet 
  und abhängig von der verwendeten Gliederungsebene der Überschrift gesetzt.
\end{DeclareValues}

Mit dem optionalen Parameter \Environment{abstract/markboth} der 
\Environment{abstract}"~Umgebung kann der Kolumnentitel mit einem beliebigen 
Inhalt gesetzt werden.

Häufig wird für Abschlussarbeiten verlangt, neben einer deutschsprachigen auch 
noch eine englischsprachige Zusammenfassung zu erstellen. Mit der Einstellung 
\Option{abstract=multiple} werden~-- sofern denn genügend Platz für diese 
vorhanden ist~-- mehrere Kurzfassungen auf einer Seite ausgeben. Außerdem kann 
die standardmäßige vertikale Zentrierung der \Environment{abstract}"~Umgebung 
auf einer Seite deaktiviert werden. Es ist zu beachten, dass die nachfolgend 
beschriebenen Einstellungen zur Positionierung der Zusammenfassungen innerhalb 
der \Environment{abstract}"~Umgebung nur wirksam sind, wenn einerseits 
\emph{keine} Überschriften in Form von Kapiteln (\Option{abstract=chapter})
und andererseits~-- wie bei \Class{tudscrreprt} und \Class{tudscrbook} 
standardmäßig aktiviert~-- eine Titelseite 
(\KOMAScript-Option~\Option{titlepage=true}) verwendet werden.
\begin{DeclareValues}
\itemval=single,one,simple=
  Jede Zusammenfassung wird auf einer eigenen Seite
  beziehungsweise im zweispaltigen Satz in einer neuen Spalte ausgegeben.
\itemval=multiple,multi,all,aggregate=%
    [v2.02:ersetzt \nosuffix{\Option*{abstract=double}}]
  Zusammenfassungen, welche mit \Macro{nextabstract} getrennt wurden, werden 
  direkt nacheinander auf der gleichen Seite ausgegeben, wenn ausreichend Platz 
  auf dieser vorhanden sein sollte. Ist die Klassenoption \Option{twocolumn} 
  aktiviert, erfolgt die Ausgabe aller Zusammenfassungen ohne Spaltenumbruch.
\itemval=fill,fil,vfil,vfill=
  Alle Zusammenfassungen werden bei der Ausgabe auf einer Seite vertikal 
  zentriert. Diese Einstellung steht für den zweispaltigen Satz
  (Klassenoption \Option{twocolumn}) nicht zur Verfügung.
\itemval=nofill,nofil,novfil,novfill=
  Die Ausgabe erfolgt wie im normalen Fließtext auch.
\end{DeclareValues}
\end{Declaration}

\begin{Declaration}
  {\Environment{abstract|\OPList{Sprache}}}
  [v2.02:Trennung einzelner Abschnitte mit \Macro{nextabstract};]
\begin{Declaration}
  {\Macro{nextabstract|\OPList{Sprache}}}
\begin{Declaration}
  {\Environment{abstract/language=\PSet{Sprache}}}
\begin{Declaration}
  {\Environment{abstract/markboth=\PMisc}}
  [v2.02]
\begin{Declaration}
  {\Environment{abstract/pagestyle=\PSet{Seitenstil}}}
  [v2.02]
\begin{Declaration}
  {\Environment{abstract/columns=\PSet{Anzahl}}}
\begin{Declaration}
  {\Environment{abstract/option=\PMisc}}
  <\Option{abstract}>
\printdeclarationlist[%
  \index{Satzspiegel!zweispaltig}%
]
Die \Environment{abstract}"~Umgebung dient speziell zur Ausgabe einer 
Zusammenfassung, beispielsweise im Vorspann eines Dokumentes oder als Präambel 
von Teilen (\Macro{setpartpreamble}) und Kapiteln (\Macro{setchapterpreamble}). 
Wird ein Titelkopf (\KOMAScript-Option \Option{titlepage=false}) und keine 
Titelseite verwendet, wird der Inhalt einer Zusammenfassung~-- identisch zu den 
\KOMAScript-Klassen~-- in einer \Environment{quotation}"~Umgebung ausgegeben, 
wobei die Überschrift \emph{immer} zentriert in der aktuellen Schriftgröße und 
\emph{nicht} in der Form einer Gliederungsebene gesetzt wird. Gleiches gilt für 
die Präambeln von Teilen und Kapiteln. Die \Environment{quotation}"~Umgebung 
hat allerdings den Nachteil, dass die Einstellung zur Absatzauszeichnung 
(\Option{parskip=\PMisc}) \emph{keine Beachtung findet}. Um dies zu beheben, 
kann das Paket \Package{quoting} geladen werden, wodurch stattdessen die 
Umgebung \Environment{quoting} genutzt wird.

Mit der zuvor erläuterten Option \Option{abstract} kann die Gestalt der Ausgabe 
aller Zusammenfassungen eingestellt werden. Des Weiteren lassen sich diese  
Einstellungen spezifisch für jede \Environment{abstract}"~Umgebung über das 
optionale Argument anhand folgender Parameter anpassen. 
%
\ChangedAt{v2.02}%
Zur Festlegung der Kolumnentitel existieren für \Environment{abstract/markboth} 
folgende Einstellungen:
\begin{DeclareValues}[\Environment{abstract/markboth}]
\itemval=false=
  Die aktuellen Kolumnentitel~-- je nach gewählter Option für das Paket 
  \Package{scrlayer-scrpage} automatische (\Option{automark}) oder manuelle 
  (\Option{manualmark})~-- werden verwendet.
\itemval*=true=
  Die Kolumnentitel für linke und rechte Seiten werden auf \Term{abstractname} 
  gesetzt.
\itemval=\PName{Kolumnentitel}=
  Die Kolumnentitel werden manuell festgelegt. So lassen sich die Kolumnen  
  beispielsweise mit \Environment{abstract/markboth=\MPValue{}} auch 
  vollständig löschen. 
\end{DeclareValues}

Sollten durch \Environment{abstract/markboth} die Kolumnen aktiv gesetzt werden,
so wird für die Umgebung~-- falls eine Titelseite (\Option{titlepage=true}) 
verwendet wird~-- automatisch der Seitenstil \PageStyle{headings} genutzt, 
andernfalls findet keine Änderung des Seitenstils statt. Mit dem Parameter 
\Environment{abstract/pagestyle} lässt sich dieser auch direkt angeben, wobei 
die \PageStyle{tudheadings}"=Seitenstile ebenfalls unterstützt werden.

Wird das Paket \Package{babel} durch den Anwender geladen, kann mit dem 
optionalen Parameter \Environment{abstract/language=\PSet{Sprache}} die Sprache 
innerhalb der \Environment{abstract}"~Umgebung geändert werden. Dafür muss die 
gewünschte Sprache bereits mit dem Laden von \Package{babel} entweder als 
Paketoption oder besser noch als Klassenoption angegeben worden sein. Dadurch 
werden innerhalb der Umgebung die Bezeichnung \Term{abstractname} und die 
Trennungsmuster sprachspezifisch angepasst. Die gewünschte Sprache kann auch 
ohne die Verwendung des Parameters \Environment{abstract/language} direkt als 
optionales Argument übergeben werden.

Wurde das Paket \Package{multicol} geladen, kann mit dem Parameter 
\Environment{abstract/columns=\PSet{Anzahl}} die Zusammenfassung mehrspaltig 
gesetzt werden. An den Parameter \Environment{abstract/option} können alle 
gültigen, bereits erläuterten Werte der Option \Option{abstract} übergeben 
werden. Die damit gemachten Einstellungen wirken sich lediglich lokal auf die 
aktuell verwendete \Environment{abstract}"~Umgebung aus.

\ChangedAt{v2.02}%
Sollen mehrere Zusammenfassungen erzeugt und dabei die Einstellungen 
\Option{abstract=single} beziehungsweise \Option{abstract=multiple} sowie 
\Option{abstract=fill} respektive \Option{abstract=nofill} beachtet werden, so 
ist die \Environment{abstract}"~Umgebung nur einmal zu verwenden. Innerhalb 
dieser müssen die einzelnen Zusammenfassungen jeweils mit \Macro{nextabstract} 
voneinander getrennt werden. Der Befehl akzeptiert dabei im optionalen Argument 
alle Parameter, welche auch von der \Environment{abstract}"~Umgebung selbst 
unterstützt werden. Das Minimalbeispiel in \fullref{sec:exmpl:dissertation} 
zeigt hierfür das notwendige Vorgehen.
\end{Declaration}
\end{Declaration}
\end{Declaration}
\end{Declaration}
\end{Declaration}
\end{Declaration}
\end{Declaration}

\minisec{Umbenennung der \abstractname}
%
Mit dem \KOMAScript-Befehl \Macro{renewcaptionname} kann der Bezeichner~-- 
sprich der Wortlaut~-- für die von der \Environment{abstract}"~Umgebung 
verwendeten Überschrift verändert werden. In \autoref{sec:localization} ist
dazu mehr zu finden.
%
\begin{Example}
\index{Zusammenfassung|!)}%
Die Überschrift der \Environment{abstract}"~Umgebung soll für die Sprache 
\PValue{ngerman} von der ursprünglichen Bezeichnung \enquote{\abstractname} in 
\enquote{Kurzfassung} geändert werden. Hierfür ist 
\Macro{renewcaptionname|\MPName{Sprache}\MPName{Makro}\MPName{Inhalt}} zu 
verwenden:%
\begin{Code}[escapechar=§]
\renewcaptionname{ngerman}{\abstractname}{Kurzfassung}
\end{Code}
\end{Example}



\subsection{%
  Selbstständigkeitserklärung und Sperrvermerk%
  \index{Selbstständigkeitserklärung|!(}%
  \index{Sperrvermerk|!(}%
}
%
\begin{Declaration}
  {\Option{declaration=\PMisc}}
  (true,notoc|\Class{tudscrbook}:chapter,toc)
\printdeclarationlist
%
Mit \Option{declaration} kann äquivalent zur Option \Option{abstract} die 
Gestaltung von Selbstständigkeitserklärung und Sperrvermerk angepasst werden.
Zur Ausgabe der Erklärungen werden die Umgebung \Environment{declarations} 
sowie die Befehle \Macro{declaration} beziehungsweise \Macro{confirmation} und 
\Macro{blocking} bereitgestellt. 

Die beiden Optionen \Option{abstract} und \Option{declaration} ähneln sich sehr 
stark. Alle möglichen Wertzuweisungen für \Option{declaration} wurden bereits 
bei der Beschreibung von \Option{abstract} ausführlich erläutert. Deshalb 
geschieht dies hier in einer etwas kürzeren Ausführung. Sollte Ihnen eine 
Erläuterung etwas dürftig erscheinen, so hilft mit Sicherheit ein Blick zur 
Erklärung der Option \Option'full'{abstract}.

Die möglichen Werte für die Gestaltung der Überschrift werden nachfolgend 
genannt. Im Gegensatz zur Option \Option{abstract} stehen die beiden 
Einstellungen \Option{declaration=true} und \Option{declaration=false} auch für 
die Klasse \Class{tudscrbook} zur Verfügung.
\begin{DeclareValues}
\itemval=false=
  Es wird keine Überschrift über den Erklärungen selbst ausgegeben.
\itemval*=true=
  Eine zentrierte Überschrift mit dem Bezeichner \Term{confirmationname} vor 
  der Selbstständigkeitserklärung beziehungsweise \Term{blockingname} vor dem 
  Sperrvermerk wird gesetzt. 
\itemval=chapter,addchap=(\Class{tudscrbook})<!not!:\Class'plain'{tudscrartcl}>
  Es wird der Befehl \Macro{chapter} für das Setzen der Überschrift genutzt. 
\itemval=section,addsec=
  Die Überschrift verwendet den Gliederungsbefehl \Macro{section}.
\itemval=heading=
  Es wird die höchstmögliche Gliederungsebene verwendet. Für 
  \Class{tudscrartcl} entspricht dies \Option{declaration=section}, bei 
  \Class{tudscrreprt} und \Class{tudscrbook} \Option{declaration=chapter}.
\end{DeclareValues}

Abhängig von der gewählten Gliederungsebene der Überschrift wird das Verhalten 
für das Setzen eines Eintrages ins Inhaltsverzeichnis festgelegt. Normalerweise 
wird nur für Überschriften in Form einer Gliederungsebene ein Eintrag der 
Erklärung ins Inhaltsverzeichnis erstellt, für \Option{declaration=true} und 
\Option{declaration=false} geschieht dies standardmäßig nicht. Mit folgenden 
Werten kann das voreingestellte Verhalten überschrieben werden.
\begin{DeclareValues}
\itemval=notoc,nottotoc=
  Die Erklärung wird definitiv nicht ins Inhaltsverzeichnis eingetragen.
\itemval=toc,totoc=
  Unabhängig von der Wahl der Überschrift erhält jede Erklärung einen nicht
  nummerierten Eintrag im Inhaltsverzeichnis auf der obersten Gliederungsebene 
  der verwendeten Dokumentklasse (\Macro||{chapter} oder \Macro||{section}). 
\itemval=tocleveldown,leveldown,totocleveldown=[v2.02]
  Der Inhaltsverzeichniseintrag wird eine Stufer unter der obersten 
  Gliederungsebene (\Macro||{section} oder \Macro||{subsection}) erzeugt.
\itemval=tocmultiple,totocmultiple,tocaggregate,totocaggregate=[v2.04]
  Es wird ein \emph{einziger} Inhaltsverzeichniseintrag für \emph{alle} 
  Erklärungen erstellt.
\end{DeclareValues}

Normalerweise werden automatische Kolumnentitel basierend auf den obersten 
beiden Gliederungsebenen gesetzt, falls dies über die Option \Option{automark} 
des Paketes \Package{scrlayer-scrpage} aktiviert wurde. Werden mit der Option 
\Option{manualmark} manuelle Kolumnentitel genutzt, müssen diese auch für die 
Erklärungen manuell gesetzt werden. Mit \Option{declaration=markboth} lässt 
sich unabhängig davon das Setzen der Kolumnentitel auf linken und rechten 
Seiten forcieren, wobei hierfür der Titel der Überschrift genutzt wird.
\begin{DeclareValues}
\itemval=markboth=[v2.02]
  Die Kolumnentitel werden in jedem Fall auf rechten sowie linken Seiten mit 
  den Bezeichnern \Term{confirmationname} beziehungsweise \Term{blockingname} 
  gesetzt.
\itemval=nomarkboth=[v2.02]
  Die Einstellung für manuelle oder automatische Kolumnentitel werden beachtet.
\end{DeclareValues}

Weiterhin kann mit dem Parameter \Environment{declarations/markboth} ein 
freier Kolumnentitel für \Macro{declaration} respektive \Macro{confirmation} 
und \Macro{blocking} sowie die \Environment{declarations}"~Umgebung angegeben 
werden.

Die folgenden Einstellungen zur Positionierung der Erklärungen haben lediglich 
Auswirkungen, wenn die Überschrift der Erklärung \emph{nicht} im Form eines 
Kapitels ausgegeben und mit der \KOMAScript-Option \Option{titlepage=true} eine 
Titelseite verwendet wird.
\begin{DeclareValues}
\itemval=single,one,simple=
  Jede Erklärung wird auf einer separaten Seite
  beziehungsweise im zweispaltigen Satz in einer neuen Spalte ausgegeben.
\itemval=multiple,multi,all,aggregate=%
    [v2.02:ersetzt \nosuffix{\Option*{declaration=double}}]
  Erklärungen, welche in der \Environment{declarations}"~Umgebung mit den 
  Befehlen \Macro{confirmation}, \Macro{blocking} und \Macro{declaration} oder 
  außerhalb dieser mit \Macro{declaration} gesetzt wurden, werden direkt 
  nacheinander auf der gleichen Seite ausgegeben, wenn ausreichend Platz auf 
  dieser vorhanden sein sollte. Ist die Klassenoption \Option{twocolumn} 
  aktiviert, erfolgt die Ausgabe aller Erklärungen ohne Spaltenumbruch.
\itemval=fill,fil,vfil,vfill=
  Alle Erklärungen auf einer Ausgabeseite werden vertikal zentriert. Für 
  den zweispaltigen Satz (Klassenoption \Option{twocolumn}) steht diese 
  Einstellung nicht zur Verfügung.
\itemval=nofill,nofil,novfil,novfill=
  Die Ausgabe erfolgt wie im normalen Fließtext auch.
\end{DeclareValues}
\end{Declaration}

\begin{Declaration}
  {\Environment{declarations|\OPList{Sprache}}}
  [
    v2.02;
    v2.04:Trennung einzelner Abschnitte mit \Macro{nextdeclaration};
  ]
\begin{Declaration}
  {\Macro{nextdeclaration|\OPList{Sprache}\MPName{Überschrift}\MPName{Text}}}
\begin{Declaration}
  {\Environment{declarations/language=\PSet{Sprache}}}
\begin{Declaration}
  {\Environment{declarations/markboth=\PMisc}}
\begin{Declaration}
  {\Environment{declarations/pagestyle=\PSet{Seitenstil}}}
\begin{Declaration}
  {\Environment{declarations/columns=\PSet{Anzahl}}}
\begin{Declaration}
  {\Environment{declarations/option=\PMisc}}
\begin{Declaration}
  {\Environment{declarations/supporter=\PSet{Unterstützer}}}
\begin{Declaration}
  {\Environment{declarations/place=\PSet{Ort}}}
\begin{Declaration}
  {\Environment{declarations/closing=\PSet{Ende}}}
\begin{Declaration}
  {\Environment{declarations/company=\PSet{Firma}}}
\printdeclarationlist
%
Für Selbstständigkeitserklärung und Sperrvermerk sollte im einfachsten Fall 
\Macro{declaration} beziehungsweise \Macro{confirmation} und \Macro{blocking} 
verwendet werden. Sobald diese jedoch in anderer Reihenfolge, mehrfacher 
Ausführung, unterschiedlichen Sprachen oder um zusätzliche Erklärungen ergänzt 
werden, bietet die \Environment{declarations}"~Umgebung entsprechende 
Freiheiten.

Innerhalb dieser Umgebung können \Macro{declaration}, \Macro{confirmation} und 
\Macro{blocking} in beliebiger Kombination und Reihenfolge direkt angegeben 
werden, um beispielsweise mehrsprachige Erklärungen zu setzen.
\ChangedAt{v2.04}%
Des Weiteren gibt es mit \Macro{nextdeclaration} die Möglichkeit, eine 
Erklärung völlig frei zu verfassen. Dieser Befehl kann \emph{ausschließlich} 
innerhalb der \Environment{declarations}"~Umgebung genutzt werden, wobei im 
ersten Argument die gewünschte Überschrift und im zweiten der Inhalt angegeben 
werden muss.

Die folgend beschriebenen Parameter können sowohl für die Umgebung 
\Environment{declarations} selbst als auch für die zuvor genannten Befehle als 
optionales Argument verwendet werden. Wie die gleichnamigen Optionen sind auch 
die Umgebungen \Environment{abstract} und \Environment{declarations} sehr 
ähnlich zueinander. Deshalb werden die Erläuterungen relativ kurz gehalten. Ist 
ein Erklärung für einen Parameter etwas unverständlich, kann diese bei der 
Umgebung \Environment'full'{abstract} nachgelesen werden.
%
\ChangedAt{v2.02}%
Für \Environment{declarations/markboth} existieren folgende Einstellungen:
\begin{DeclareValues}[\Environment{declarations/markboth}]
\itemval=false=
  Die aktuellen Kolumnentitel~-- automatische respektive manuelle~-- werden 
  verwendet.
\itemval*=true=
  Für linke und rechte Seiten werden \Term{confirmationname} beziehungsweise 
  \Term{blockingname} als Kolumen gesetzt.
\itemval=\PName{Kolumnentitel}=
  Die Kolumnentitel werden manuell festgelegt.
\end{DeclareValues}

Wird der Parameter \Environment{declarations/markboth} genutzt, wird der 
Seitenstil \PageStyle{headings} automatisch aktiviert. Alternativ kann mit dem 
Parameter \Environment{declarations/pagestyle} auch ein beliebiger Seitenstil 
angegeben werden. 

Mit dem Parameter \Environment{declarations/language=\PSet{Sprache}} kann für 
die \Environment{declarations}"~Umgebung jede mit dem Paket \Package{babel} 
aktivierte Sprache eingestellt werden, wodurch die Bezeichner~-- unter anderem 
\Term{confirmationname} und \Term{blockingname}~-- sowie die Trennungsmuster 
innerhalb der Umgebung sprachspezifisch angepasst werden. Zusammen mit dem 
Paket \Package{multicol} wird der Inhalt der Umgebung durch die Verwendung von 
\Environment{declarations/columns=\PSet{Anzahl}} mehrspaltig gesetzt. Alle 
gültigen Werte der Option \Option{declaration} werden vom Parameter 
\Environment{declarations/option} akzeptiert. Die weiteren Parameter 
\Environment{declarations/supporter} sowie \Environment{declarations/place} und 
\Environment{declarations/closing} sind in der Dokumentation des Befehls 
\Macro{confirmation} zu finden, für \Macro{blocking} ist der Parameter 
\Environment{declarations/company} erläutert.
\end{Declaration}
\end{Declaration}
\end{Declaration}
\end{Declaration}
\end{Declaration}
\end{Declaration}
\end{Declaration}
\end{Declaration}
\end{Declaration}
\end{Declaration}
\end{Declaration}

\begin{Declaration}
  {\Macro{confirmation|\OPList{Unterstützer}}}
\begin{Declaration}
  {\Macro{confirmation*|\OList}}
  [v2.05]
\begin{Declaration}
  {\Macro{confirmation/supporter=\PSet{Unterstützer}}}
\begin{Declaration}
  {\Macro{confirmation/place=\PSet{Ort}}}
\begin{Declaration}
  {\Macro{confirmation/closing=\PSet{Ende}}}
\begin{Declaration}
  {\Macro{confirmation/language=\PSet{Sprache}}}
\begin{Declaration}
  {\Macro{confirmation/markboth=\PMisc}}
  [v2.02]
\begin{Declaration}
  {\Macro{confirmation/pagestyle=\PSet{Seitenstil}}}
  [v2.02]
\begin{Declaration}
  {\Macro{confirmation/columns=\PSet{Anzahl}}}
  [v2.02]
\begin{Declaration}
  {\Macro{confirmation/option=\PMisc}}
\printdeclarationlist[%
  \index{Datum}%
]
Mit diesem Befehl wird ein sprachspezifischer Standardtext für eine 
Selbstständigkeitserklärung ausgegeben, welcher in \Term{confirmationtext} 
gespeichert ist. Wie dieser angepasst beziehungsweise geändert werden kann, ist 
unter \autoref{sec:localization} zu finden. Er kann sowohl innerhalb der 
\Environment{declarations}"~Umgebung als auch außerhalb dieser direkt im 
Dokument verwendet werden. 

Der ausgegebene Standardtext lässt sich über Parameter im optionalen Argument 
von \Macro{confirmation} anpassen. In der Selbstständigkeitserklärung werden 
sowohl der Titel als auch der Typ der Abschlussarbeit~-- falls dieser mit 
\Macro{thesis}, \Macro{subject|\MPName{\autoref{tab:thesis}}} beziehungsweise 
mit der Option \Option{subjectthesis} angegeben wurde~-- aufgeführt. Über den 
Parameter \Macro{confirmation/supporter} oder \emph{zuvor} mit dem Befehl 
\Macro{supporter} können weitere an der Arbeit beteiligte Personen angegeben 
werden. Mehrere zu nennende Personen sind auch hier durch \Macro{and} zu 
trennen. Das Feld der Unterstützer kann auch mit dem bloßen optionalen Argument 
ohne die Angabe eines Parameters angepasst werden. 
\ChangedAt{v2.05}%
Mit der Sternversion \Macro{confirmation*} werden als Unterstützer die mit 
\Macro{supervisor|\MPName{Name(n)}} definierten Betreuer der Arbeit angegeben.

Nach dem eigentlichen Text der Selbstständigkeitserklärung wird der mit 
\Macro{confirmation/place} beziehungsweise \Macro{place} angegebene Ort sowie 
das mit \Macro{date} eingestellte Datum ausgegeben. Als Voreinstellung ist für 
den Ort \enquote{Dresden} gewählt. Danach folgen~-- mit etwas vertikalem 
Leerraum für die notwendige Unterschrift~-- der Autor oder die Autoren, 
angegeben durch den Befehl \Macro{author}. Soll anstelle dessen etwas anderes 
nach dem Text der Selbstständigkeitserklärung gesetzt werden, kann dies mit dem 
Parameter \Macro{confirmation/closing} oder zuvor mit dem Befehl 
\Macro{confirmationclosing} angepasst werden. Die Parameter 
\Environment{declarations/language}, \Environment{declarations/markboth}, 
\Environment{declarations/pagestyle}, \Environment{declarations/columns} und 
\Environment{declarations/option} entsprechen in ihrem Verhalten denen 
der \Environment{declarations}"~Umgebung.
\end{Declaration}
\end{Declaration}
\end{Declaration}
\end{Declaration}
\end{Declaration}
\end{Declaration}
\end{Declaration}
\end{Declaration}
\end{Declaration}
\end{Declaration}

\begin{Declaration}
  {\Macro{blocking|\OPList{Firma}}}
  [v2.02]
\begin{Declaration}
  {\Macro{blocking/company=\PSet{Firma}}}
\begin{Declaration}
  {\Macro{blocking/language=\PSet{Sprache}}}
\begin{Declaration}
  {\Macro{blocking/markboth=\PMisc}}
\begin{Declaration}
  {\Macro{blocking/pagestyle=\PSet{Seitenstil}}}
\begin{Declaration}
  {\Macro{blocking/columns=\PSet{Anzahl}}}
\begin{Declaration}
  {\Macro{blocking/option=\PMisc}}
\printdeclarationlist
%
Beim Sperrvermerk verhält es sich äquivalent zur Selbstständigkeitserklärung.
Es wird der in \Term{blockingtext} hinterlegte Standardtext in der gewählten 
Sprache ausgegeben. Dieser kann~-- wie in \autoref{sec:localization} 
beschrieben~-- geändert werden. Der Befehl \Macro{blocking} kann sowohl 
innerhalb der Umgebung \Environment{declarations} als auch direkt im Dokument 
verwendet werden. 

In seiner ursprünglichen Definition lässt sich der Sperrvermerk im optionalen 
Argument über die deklarierten Parameter anpassen. In dessen Standardtext 
werden sowohl der Titel als auch der Typ der Abschlussarbeit~-- falls dieser 
mit \Macro{thesis}, \Macro{subject|\MPName{\autoref{tab:thesis}}} respektive 
durch die Option \Option{subjectthesis} angegeben wurde~-- aufgeführt. Mit 
\Macro{blocking/company} oder \emph{vorher} mit \Macro{company} kann zusätzlich 
eine im Sperrvermerk zu nennende Firma oder ähnliches angegeben werden. Dieses 
Feld lässt sich auch direkt im optionalen Argument ohne die Verwendung eines 
Parameters definieren. Die weiteren Parameter 
\Environment{declarations/language}, \Environment{declarations/markboth}, 
\Environment{declarations/pagestyle}, \Environment{declarations/columns} und 
\Environment{declarations/option} entsprechen in ihrem Verhalten denen 
der \Environment{declarations}"~Umgebung.
\end{Declaration}
\end{Declaration}
\end{Declaration}
\end{Declaration}
\end{Declaration}
\end{Declaration}
\end{Declaration}

\begin{Declaration}
  {\Macro{declaration|\OList}}
\begin{Declaration}
  {\Macro{declaration*|\OList}}
  [v2.05]
\begin{Declaration}
  {\Macro{declaration/language=\PSet{Sprache}}}
\begin{Declaration}
  {\Macro{declaration/markboth=\PMisc}}
  [v2.02]
\begin{Declaration}
  {\Macro{declaration/pagestyle=\PSet{Seitenstil}}}
  [v2.02]
\begin{Declaration}
  {\Macro{declaration/columns=\PSet{Anzahl}}}
  [v2.02]
\begin{Declaration}
  {\Macro{declaration/option=\PMisc}}
\begin{Declaration}
  {\Macro{declaration/supporter=\PSet{Unterstützer}}}
\begin{Declaration}
  {\Macro{declaration/place=\PSet{Ort}}}
\begin{Declaration}
  {\Macro{declaration/closing=\PSet{Ende}}}
\begin{Declaration}
  {\Macro{declaration/company=\PSet{Firma}}}
\printdeclarationlist
%
Dieser Befehl gibt die Selbstständigkeitserklärung und den Sperrvermerk direkt 
aufeinanderfolgend aus. Dabei werden die Einstellungen zur Positionierung der 
einzelnen Erklärungen, welche über die Zuweisungen \Option{declaration=single} 
beziehungsweise \Option{declaration=multiple} sowie \Option{declaration=fill} 
respektive \Option{declaration=nofill} erfolgen, beachtet. Er kann sowohl 
innerhalb der \Environment{declarations}"~Umgebung als auch außerhalb dieser 
direkt im Dokument verwendet werden und akzeptiert im optionalen Argument dabei 
alle für die \Environment{declarations}"~Umgebung beschriebenen Parameter. 
\ChangedAt{v2.05}%
Die Sternversion erzwingt für die Selbstständigkeitserklärung eine Angabe der 
mit \Macro{supervisor|\MPName{Name(n)}} definierten Betreuer in dieser.
\end{Declaration}
\end{Declaration}
\end{Declaration}
\end{Declaration}
\end{Declaration}
\end{Declaration}
\end{Declaration}
\end{Declaration}
\end{Declaration}
\end{Declaration}
\end{Declaration}

\begin{Declaration}
  {\Macro{supporter|\MPName{Unterstützer}}}
\begin{Declaration}
  {\Macro{place|\MPName{Ort}}}
\begin{Declaration}
  {\Macro{confirmationclosing|\MPName{Ende}}}
\begin{Declaration}
  {\Macro{company|\MPName{Firma}}}
\printdeclarationlist
%
Diese Makros ändern~-- im Gegensatz zu den Parametern von 
\Macro{confirmation} und \Macro{blocking}~-- die entsprechenden Feldwerte 
global. Damit lässt sich die \emph{mehrfache} Angabe eines Parameters 
vermeiden, wenn beispielsweise eine Erklärung in unterschiedlichen Sprachen 
erzeugt wird.%
\index{Selbstständigkeitserklärung|!)}%
\index{Sperrvermerk|!)}%
\end{Declaration}
\end{Declaration}
\end{Declaration}
\end{Declaration}



\subsection{%
  Lesezeichen%
  \index{Lesezeichen}%
}
%
\begin{Declaration}
  {\Option{tudbookmarks=\PBoolean}}
  (true)
\printdeclarationlist[%
  \index{Titel}%
  \index{Umschlagseite}%
  \index{Inhaltsverzeichnis}%
  \index{Aufgabenstellung}%
  \index{Gutachten}%
  \index{Aushang}%
]
Diese Option wird wirksam, wenn \Package{hyperref} geladen wurde. Es werden für 
die Umschlag- und Titelseite, das Inhaltsverzeichnis sowie~-- bei der 
Verwendung des Paketes \Package{tudscrsupervisor}~-- die Aufgabenstellung 
Lesezeichen oder auch Outline-Einträge im PDF"~Dokument erzeugt.
\begin{DeclareValues}
\itemval=false=
  Es erfolgt kein Eintrag von ergänzenden Lesezeichen.
\itemval*=true=
  Es werden automatisch zusätzliche Lesezeichen eingetragen.
\end{DeclareValues}
\end{Declaration}

\begin{Declaration}
  {\Macro{tudbookmark|\OPName{Ebene}\MPName{Text}\MPName{Ankername}}}
\printdeclarationlist
%
Der Befehl \Macro{tudbookmark} arbeitet prinzipiell in der gleichen Weise wie 
\Macro{pdfbookmark} aus \Package{hyperref}. Die Lesezeichen werden jedoch nur 
bei aktivierte Option \Option{tudbookmarks} generiert.
\end{Declaration}



\section{%
  Sprachabhängige Bezeichner%
  \label{sec:localization}%
  \index{Bezeichner|!(}%
  \index{Titel!Felder|(}%
}
%
Durch \KOMAScript werden Befehle, mit denen sich sprachabhängige Bezeichner 
erzeugen oder ändern lassen, zur Verfügung gestellt. Diese werden durch das
\TUDScript-Bundle genutzt, um lokalisierte Begriffe für die Sprachen 
\emph{Englisch} und \emph{Deutsch} bereitzustellen. Ein Großteil davon betrifft 
Bezeichnungen für Felder auf der Titelseite (\autoref{sec:title}). Hierfür wird
\Macro{providecaptionname|\MPName{Sprache}\MPName{Makro}\MPName{Inhalt}} 
verwendet, wobei \PName{Sprache} dem geladenen Sprachpaket~-- normalerweise das 
Paket \Package{babel}~-- bekannt sein muss.

Sollte der Anwender die im Folgenden erläuterten oder auch andere Bezeichner, 
welche von einem beliebigen (Sprach"~)Paket bereitgestellt werden, ändern 
wollen, ist hierfür der Befehl
\Macro{renewcaptionname|\MPName{Sprache}\MPName{Makro}\MPName{Inhalt}} zu 
verwenden. Es sollte natürlich dabei eine \PName{Sprache} angegeben werden, 
welche im Dokument durch \Package{babel} oder ein anderes Sprachpaket verwendet 
wird, beispielsweise \PValue{ngerman} oder \PValue{english}. 

Die Makros der Bezeichner und deren Verwendung werden folgend kurz beschrieben 
und tabellarisch aufgeführt. Dabei wurde versucht, alle Befehle der Bezeichner 
für bestimmte Begriffe auf \Term*{\dots name} und beschreibende Texte auf 
\Term*{\dots text} enden zu lassen.

\begin{Declaration}
  {\Term{supervisorname}}
  [v2.05:Unterscheidung von einem und mehreren Betreuern]
\begin{Declaration}
  {\Term{supervisorothername}}
\begin{Declaration}
  {\Term{refereename}}
  [v2.02:Unterscheidung von einem und mehreren Gutachtern]
\begin{Declaration}
  {\Term{refereeothername}}
\begin{Declaration}
  {\Term{advisorname}}
  [v2.05:Unterscheidung von einem und mehreren Fachreferenten]
\begin{Declaration}
  {\Term{advisorothername}}
\begin{Declaration}
  {\Term{professorname}}
  [v2.02:Unterscheidung von einem und mehreren Professoren]
\begin{Declaration}
  {\Term{professorothername}}
  [v2.02]
\printdeclarationlist[%
  \index{Betreuer}%
  \index{Gutachter}%
  \index{Hochschullehrer}%
  \index{Referent}%
]
Diese sprachabhängigen Begriffe sind die Bezeichner für die Titelseitenfelder 
von Betreuer (\Macro{supervisor}), Gutachter (\Macro{referee}) und Fachreferent 
(\Macro{advisor}). Soll innerhalb eines dieser Felder mehr als eine Person 
angegeben werden, so sind die Einzelpersonen jeweils mit dem Befehl \Macro{and} 
voneinander zu trennen. In diesem Fall werden alle nach der erstgenannten 
folgenden Personen durch den Bezeichner \Term*{\dots othername} ergänzt.

\ChangedAt{v2.02;v2.05}%
Bei den Bezeichnung wird unterschieden, ob eine oder mehrere Personen angegeben 
wurden. Wird lediglich eine Person genannt, so ist eine Unterscheidung nicht 
notwendig und es wird der Singular genutzt. Werden jedoch zwei oder mehr 
Personen angegeben, so wird geprüft, ob der dazugehörige Bezeichner für die 
Zweitperson (\Term*{\dots othername}) definiert ist. Falls dies so ist, wird 
die alternative Bezeichnung für die erstgenannte Person verwendet, andernfalls 
wird der Plural des Bezeichners verwendet. Dies betrifft alle Felder, die über 
\Macro{referee}, \Macro{advisor}, \Macro{supervisor} oder \Macro{professor} 
angegeben wurden.

\renewcaptionname{ngerman}{\supervisorname}{(Erst-)Betreuer}
\renewcaptionname{english}{\supervisorname}{(First) Supervisors(s)}
\renewcaptionname{ngerman}{\refereename}{(Erst-)Gutachter}
\renewcaptionname{english}{\refereename}{(First) Referee(s)}
\renewcaptionname{ngerman}{\advisorname}{(Erster) Fachreferent(en)}
\renewcaptionname{english}{\advisorname}{(First) Advisor(s)}
\renewcaptionname{ngerman}{\professorname}{Betreuende(r) Hochschullehrer}
\renewcaptionname{english}{\professorname}{Supervising professor(s)}
\TermTable%
\end{Declaration}
\end{Declaration}
\end{Declaration}
\end{Declaration}
\end{Declaration}
\end{Declaration}
\end{Declaration}
\end{Declaration}

\begin{Declaration}
  {\Term{dissertationname}}
\begin{Declaration}
  {\Term{diplomathesisname}}
\begin{Declaration}
  {\Term{masterthesisname}}
\begin{Declaration}
  {\Term{bachelorthesisname}}
\begin{Declaration}
  {\Term{studentthesisname}}
\begin{Declaration}
  {\Term{studentresearchname}}
\begin{Declaration}
  {\Term{projectpapername}}
\begin{Declaration}
  {\Term{seminarpapername}}
\begin{Declaration}
  {\Term{termpapername}}
\begin{Declaration}
  {\Term{researchname}}
\begin{Declaration}
  {\Term{logname}}
\begin{Declaration}
  {\Term{reportname}}
\begin{Declaration}
  {\Term{internshipname}}
\printdeclarationlist[%
  \index{Abschlussarbeit}%
  \index{Typisierung}%
]
Diese Bezeichner dienen zur Typisierung einer Abschlussarbeit. Zur Nutzung wird 
auf die Erläuterung von \Macro{thesis} und \Macro'full'{subject} verwiesen.
\ToDo[doc]{thesis-Liste für Deklaration auslesen, sobald implementiert}[v2.07]
\TermTable%
\end{Declaration}
\end{Declaration}
\end{Declaration}
\end{Declaration}
\end{Declaration}
\end{Declaration}
\end{Declaration}
\end{Declaration}
\end{Declaration}
\end{Declaration}
\end{Declaration}
\end{Declaration}
\end{Declaration}

\begin{Declaration}
  {\Term{graduationtext}}
  [v2.02]
\printdeclarationlist[%
  \index{Abschlussarbeit}%
  \index{Typisierung}%
]
Für eine Abschlussarbeit kann mit dem Befehl \Macro{graduation} der zu 
erlangende akademische Grad angegeben werden. Auf dem Titel wird der 
aufgeführte Text entsprechend ergänzt.
\TermTable[.78\textwidth]%
\end{Declaration}

\begin{Declaration}
  {\Term{datetext}}
\begin{Declaration}
  {\Term{defensedatetext}}
\printdeclarationlist[%
  \index{Abschlussarbeit}%
  \index{Datum}%
  \index{Datum!Abgabedatum}%
  \index{Datum!Verteidigungsdatum}%
]
Wird mit \Macro{date} das (Abgabe"~)Datum und mit \Macro{defensedate} ein Datum 
der Verteidigung für eine Abschlussarbeit angegeben, so werden auch diese 
Felder durch einen Text beschrieben.
\TermTable%
\index{Titel!Felder|)}%
\end{Declaration}
\end{Declaration}

\begin{Declaration}
  {\Term{dateofbirthtext}}
\begin{Declaration}
  {\Term{placeofbirthtext}}
\begin{Declaration}
  {\Term{matriculationnumbername}}
\begin{Declaration}
  {\Term{matriculationyearname}}
\begin{Declaration}
  {\Term{coursename}}
\begin{Declaration}
  {\Term{disciplinename}}
  [v2.02]
\printdeclarationlist[%
  \index{Autorenangaben}%
  \index{Datum!Geburtsdatum}%
]
Werden für den Autor oder die Autoren mit dem entsprechenden Befehl das 
Geburtsdatum (\Macro{dateofbirth}), der Geburtsort (\Macro{placeofbirth}), der 
Studiengang (\Macro{course}), die Studienrichtung (\Macro{discipline}) oder 
auch die Matrikelnummer (\Macro{matriculationnumber}) und/oder das 
Immatrikulationsjahr (\Macro{matriculationyear}) angegeben, werden sowohl auf 
der Titelseite als auch auf der gegebenenfalls mit \Package{tudscrsupervisor} 
erstellten Aufgabenstellung die dazugehörigen Bezeichner vorangestellt. Auf 
dem Titel werden diese dabei mit dem durch \Macro{titledelimiter} gegebenen 
Trennzeichen vom eigentlichen Feld abgegrenzt.
\TermTable%
\end{Declaration}
\end{Declaration}
\end{Declaration}
\end{Declaration}
\end{Declaration}
\end{Declaration}

\begin{Declaration}
  {\Term{coverpagename}}
\begin{Declaration}
  {\Term{titlepagename}}
\printdeclarationlist[%
  \index{Lesezeichen}
]
Diese beiden Bezeichner werden bei aktivierter \Option{tudbookmarks} für das 
Eintragen von Lesezeichen in ein PDF"~Dokument genutzt.
\TermTable%
\end{Declaration}
\end{Declaration}

\begin{Declaration}
  {\Term{abstractname}}
\printdeclarationlist[%
  \index{Zusammenfassung}
]
Dieser Bezeichner wird für die Klasse \Class{tudscrbook} definiert, da selbiger 
von \KOMAScript für die Buchklasse nicht vorgesehen wird.
\TermTable%
\end{Declaration}

\begin{Declaration}
  {\Term{confirmationname}}
\begin{Declaration}
  {\Term{blockingname}}
  [v2.02]
\printdeclarationlist[%
  \index{Selbstständigkeitserklärung|(}%
  \index{Sperrvermerk|(}%
]
Es werden die Bezeichnungen für Selbstständigkeitserklärung und Sperrvermerk 
für die dazugehörigen Überschriften definiert.
\TermTable%
\end{Declaration}
\end{Declaration}

\begin{Declaration}
  {\Term{confirmationtext}}
\begin{Declaration}
  {\Term{blockingtext}}
  [v2.02]
\printdeclarationlist
%
Die Texte der Erklärungen selbst sind derart aufgebaut, dass sie in 
Abhängigkeit von den angegebenen Informationen unterschiedlich ausgeführt 
werden. Innerhalb der Selbstständigkeitserklärung (\Macro{confirmation}) werden 
gegebenenfalls die Felder für den Titel (\Macro{title}) und die Typisierung der 
Abschlussarbeit sowie die angegebenen Unterstützer%
\footnote{%
  \Macro{%
    confirmation|\OPValue{\Macro{confirmation/supporter=\PSet{Unterstützer}}}%
  } oder \Macro{supporter|\MPName{Unterstützer}}%
}
beachtet. Für den Sperrvermerk (\Macro{blocking}) wird neben dem Titel 
(\Macro{title}) optional außerdem noch das Feld der externen Firma%
\footnote{%
  \Macro{blocking|\OPValue{\Macro{blocking/company=\PSet{Firma}}}}
  oder \Macro{company|\MPName{Firma}}%
}
verwendet. Der Vollständigkeit halber werden im Folgenden noch die Texte für 
die Selbstständigkeitserklärung und den Sperrvermerk aufgeführt~-- allerdings 
lediglich die deutschsprachige Version. Dabei werden alle möglichen Felder 
angezeigt.

\begingroup%
  \makeatletter%
  \def\@@title{\PName{Titel}}%
  \def\@@thesis{\PName{Abschlussarbeit}}%
  \def\@supporter{\PName{Vorname Nachname} \and \PName{Vorname Nachname}}%
  \def\@company{\PName{Firma}}%
  \makeatother%
  \vskip\medskipamount\noindent%
  \textbf{Bezeichner}\quad\Term{confirmationtext}%
  \begin{quoting}
  \confirmationtext%
  \end{quoting}
  \textbf{Bezeichner}\quad\Term{blockingtext}%
  \begin{quoting}
  \blockingtext%
  \end{quoting}

  \makeatletter%
  \def\@@author{\PName{Vorname Nachname}}%
  \def\@supporter{\PName{Vorname Nachname}}%
  \makeatother%
  \newcommand*\showfield[1]{%
    \Macro*{getfield|\MPValue{#1}}~\textrightarrow~\getfield{#1}%
  }%
  \noindent%
  Soll eine der Erklärungen geändert und dabei der Inhalt eines Feldes genutzt 
  werden, lässt sich hierfür \Macro{getfield} verwenden.%
  \footnote{%
    Titel:~\showfield{title}, Art der Abschlussarbeit:~\showfield{thesis},
    Autor:~\showfield{author}, Firma:~\showfield{company} sowie ebenfalls
    Unterstützer:~\showfield{supporter}
  }
\endgroup%
Gegebenenfalls ist die Definition von \Macro{and} anzupassen.%
\index{Selbstständigkeitserklärung|)}%
\index{Sperrvermerk|)}%
\end{Declaration}
\end{Declaration}

\begin{Declaration}
  {\Term{listingname}}
\begin{Declaration}
  {\Term{listlistingname}}
\printdeclarationlist
%
Sollte ein Paket zur Einbindung von externem Quelltext~-- beispielsweise 
das Paket \Package{listings}~-- verwendet werden, so werden diese Bezeichnungen 
für Quelltextausschnitte und das Quelltextverzeichnis verwendet.
\TermTable%
\index{Bezeichner|!)}%
\end{Declaration}
\end{Declaration}



\section{%
  Kompatibilitätseinstellungen zu früheren Versionen%
  \index{Kompatibilität|!(}%
  \index{Änderungen|?(}%
}
%
Bei der Entwicklung von \TUDScript lässt es sich nicht immer vermeiden, dass 
Verbesserungen sowie Korrekturen an den Klassen und Paketen zu Änderungen am 
Ergebnis der Ausgabe führen, insbesondere bei Umbruch und Layout. Für bereits
archivierte Dokumente, welche mit einer früheren Version erstellt wurden ist 
dies unter Umständen jedoch eher unerwünscht.

\begin{Declaration}
  {\Option{tudscrver=\PMisc}}
  (last)
  [v2.03]
\printdeclarationlist
Mit dieser Option wird es möglich, auf das (Umbruch"~)Verhalten einer älteren 
respektive früheren Version von \TUDScript umzuschalten, um nach der 
Kompilierung das erwartete Ergebnis zu erhalten. Neue Möglichkeiten, die sich 
nicht auf den Umbruch oder das Layout auswirken, sind bei aktivierter 
Kompatibilität zu einer älteren Version dennoch verfügbar. 

Bei der Angabe einer unbekannten Version als Wert wird eine Warnung ausgegeben 
und \Option{tudscrver=first} angenommen. Mit \Option{tudscrver=last} wird die 
jeweils aktuell verfügbare Version ausgewählt und folglich auf die zukünftige 
Kompatibilität des Dokumentes zu der aktuell genutzten Version verzichtet. 
Dieses Verhalten entspricht der Voreinstellung. Es ist zu beachten, dass die 
Nutzung von \Option{tudscrver} nur als Klassenoption möglich ist.
\begin{DeclareValues}
\itemval=2.02,first=
  \ChangedAt{%
    v2.03:Satzspiegel im \CD geändert, Logo von \DDC im Fußbereich 
    wird ohne vergrößerten Seitenrand verwendet
  }%
  Der Satzspiegel (\seeplain{\Option'page'{cdgeometry=\PMisc}}) im Layout des 
  \CDs wurde mit der Version~v2.03 leicht geändert. Der obere Seitenrand wurde 
  verkleinert, der untere im gleichen Maße vergrößert. Der verfügbare 
  Textbereich blieb folglich unverändert. Bei der Aktivierung des \DDC-Logos im 
  Seitenfußbereich (\seeplain{\Option'page'{ddcfoot=\PMisc}}) wird der 
  identische Satzspiegel genutzt. Dieses Verhalten lässt sich mit 
  \Option{tudscrver=2.02} deaktivieren.
\itemval=2.03=
  \index{Leerraum}%
  \index{Schriftgröße}%
  Seit der Version~v2.04 werden mehrere Längen für vertikalen Leerraum in 
  Abhängigkeit der gewählten Schriftgröße definiert. Diese Funktionalität lässt 
  sich mit \Option{tudscrver=2.03} deaktivieren, wobei hierfür lediglich die 
  Option \Option{relspacing=false} gesetzt wird. 
\itemval=2.04=
  \index{Satzspiegel}%
  \ChangedAt{%
    v2.05:Satzspiegeleinstellungen für die jeweilige ISO/DIN"~Klasse 
    des Papierformates identisch%
  }%
  Mit der Version~v2.05 werden die vorgegebenen Einstellungen zum Satzspiegel 
  anhand der B"~ISO/DIN"~Reihe vorgenommen. Damit sind für alle Papierformate 
  einer spezifischen ISO/DIN"~Klasse die Seitenränder identisch. Mit der Wahl 
  \Option{tudscrver=2.04} ist der Satzspiegel von der A"~ISO/DIN"~Reihe 
  abhängig, sodass die B- und C"~Papierformate der gleichen Klasse größere 
  Seitenränder erhalten, als die D- und A"~Formate.
\itemval=2.05=
  \ChangedAt{v2.06:Griechische Lettern standardmäßig kursiv}%
  Mit dem Wechsel der Hausschrift zu \OpenSans ergaben sich einige Änderungen 
  bezüglich der Erscheinung des \CDs. Mit dieser Kompatibilitätseinstellung 
  werden die alten Schriftfamilien \Univers und \DIN 
  (\Option{cdoldfont=true}, \Option{headings=light}, \Option{ttfont=lmodern})
  aktiviert. Weiterhin wirkt sich für diese Schriftfamilien im Mathematikmodus 
  die Option \Option{slantedgreek=true} lediglich auf die griechischen 
  Majuskeln aus.
\itemval=2.06=
  \ChangedAt{%
    v2.07:Überschriften nutzen immer \OpenSans, wenn Layout des \CDs aktiviert;%
    v2.07:Abstände des vertikalen Leerraums in Abhängigkeit von der 
      Schriftgröße verbessert;%
  }%
  Wenn das Layout des \CDs nicht mit der Option \Option{cd=false} deaktiviert 
  wurde, wird ab der Version~v2.07 für Überschriften \OpenSans verwendet, auch 
  wenn für den Fließtext die Verwendung der Schriften des \CDs mit der 
  Einstellung \Option{cdfont=false} deaktiviert wurde. Weiterhin werden die 
  Abstände des vertikalen Leerraums (Option \Option{relspacing=true}) nicht 
  mehr in diskreten Stufen gesetzt, womit ein besseres Ergebnis für nicht 
  vordefinierte Schriftgrößen (Option \Option{fontsize}) erzielt wird.
\itemval=2.07=
  Dies ist Kompatibilitätseinstellung für \TUDScript~\vTUDScript{} und wird für 
  zukünftige Änderungen bereits vorgehalten. Soll ein mit dieser Version 
  erzeugtes Dokument auch mit einer späteren Version von \TUDScript nach einem 
  \hologo{LaTeX}"~Lauf das gleiche Ausgabeergebnis liefern, muss dies mit 
  \Option{tudscrver=\TUDScriptVersionNumber} angegeben werden.
\itemval=last=
  Es werden keine Kompatibilitätseinstellungen für das Dokument vorgenommen. 
  Mit einer späteren Version von \TUDScript kann ein anderes Umbruchverhalten 
  innerhalb des Dokumentes auftreten. Dies ist die Standardeinstellung.%
  \index{Änderungen|?)}%
  \index{Kompatibilität|!)}%
\end{DeclareValues}
\end{Declaration}
%
\ToDo[doc]{Wieder rein, falls vernünftig implementiert}[v2.08]
%\subsection{%
%  Fußnoten in Überschriften%
%  \index{Layout!Überschriften}%
%  \index{Fußnoten}%
%}
%%
%\begin{Declaration}
%  {\Option{footnotes=\PMisc}}
%  (nosymbolheadings)%
%  [v2.02:Fußnoten mit Symbolen in Überschriften möglich;]
%\begin{Declaration}
%  {\Counter{symbolheadings}}
%  [v2.02]
%\printdeclarationlist%
%%
%\ToDo[imp]{Fehler \Macro*{addchap} beheben, Paket \Package*{footmisc}}[v2.06]
%\ToDo[imp]{Zähler auch bei Sternversionen von Kapiteln zurücksetzen}[v2.06]
%\ToDo[imp]{Fußnoten nicht ins Inhaltsverzeichnis und in die Kopfzeile?!}[v2.06]
%\ToDo[imp]{Problem mit \Package*{hyperref} lösbar?}[v2.06]
%%
%Für die Überschriften wird die \KOMAScript-Option \Option{footnotes} erweitert.
%Normalerweise kann diese die Werte \PValue{multiple} und \PValue{nomultiple} 
%annehmen, wobei Letzteres der Standardfall ist. Die \TUDScript-Hauptklassen 
%erweitern die Option dahingehend, dass auf die Verwendung von Symbolen anstelle
%von Zahlen innerhalb der Überschriften umgeschaltet werden kann. Hierfür wird 
%der Zähler \Counter{symbolheadings} definiert, der mit dem Beginn eines neuen 
%Kapitels zurückgesetzt wird.
%
%\Attention{%
%  Die Option \Option{footnotes=symbolheadings} ist experimentell und kann unter
%  Umständen zu Fehlern respektive unerwünschten Ergebnissen führen.%
%}
%%
%\begin{DeclareValues}
%\itemval=nosymbolheadings,numberheadings=
%  Die Fußnoten der Überschriften werden fortlaufend mit denen des Fließtextes 
%  gesetzt.
%\itemval=symbolheadings=
%  Für die Überschriften werden symbolische Fußnoten mit einem eigenen Zähler 
%  verwendet.
%\end{DeclareValues}
%\end{Declaration}
%\end{Declaration}
\end{DeclareEntity*}
\end{DeclareEntity*}
\end{DeclareEntity*}
