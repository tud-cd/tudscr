\chapter[%
  Das Paket \Package{tudscrsupervisor} -- Studentische Betreuung%
]{%
  Betreuung studentischer Arbeiten%
  \tudmarkuplabel[sec:supervisor]{\Package{tudscrsupervisor}}%
}
\begin{DeclareEntity}{\Package{tudscrsupervisor}}
%
Das Paket \Package{tudscrsupervisor} stellt für das Erstellen von 
Aufgabenstellungen und Gutachten wissenschaftlicher Arbeiten sowie offiziellen 
Aushängen im \CD passende Umgebungen und Befehle für den Anwender bereit. 
Deshalb richtet es sich vornehmlich an Mitarbeiter der \TnUD, kann jedoch 
natürlich auch von Studenten genutzt werden, um beispielsweise die offiziell 
erhaltene Aufgabenstellung für eine Abschlussarbeit im gleichen Stil wie das 
restliche Dokument mit \hologo{LaTeX} zu setzen.

An der \TUD gibt es unterschiedlichste Vorlagen für Aufgabenstellungen, 
Aushänge und ähnlichem. Falls die folgend beschriebenen Umgebungen und Befehle 
nicht ausreichend sind, um die Ihnen auferlegten Vorgaben umzusetzen, dürfen 
Sie gerne das \Forum oder \GitHubRepo<issues> besuchen, um mögliche 
Anpassungen sowie Erweiterungen zu diskutieren.



\section{%
  Aufgabenstellung für eine wissenschaftliche Arbeit%
  \index{Aufgabenstellung|!(}%
}
%
\begin{Declaration}
  {\Environment{task|\OPList{Überschrift}}}
  <\Environment{tudpage}>
\begin{Declaration}
  {\Environment{task/headline=\PSet{Überschrift}}}
\begin{Declaration}
  {\Environment{task/style=\PSet{Stil}}}
  [v2.05]
\printdeclarationlist
%
Mit der \Environment{task}"~Umgebung kann ein Aufgabenstellung für eine 
wissenschaftliche Arbeit ausgegeben werden. Für diese wird normalerweise eine 
Überschrift gesetzt, welche sich aus \Term{taskname} und~-- falls der Typ der 
Abschlussarbeit angegeben wurde~-- noch aus \Term{tasktext} und \Macro{thesis} 
zusammensetzt. Der Parameter \Environment{task/headline} kann genutzt werden, 
um diese automatisch generierte Überschrift anzupassen. Als Basis fungiert 
dabei die \Environment{tudpage}"~Umgebung, weshalb sich im optionalen Argument 
zusätzlich auch deren zuvor beschriebenen Parameter nutzen lassen.

Als Kopf der Aufgabenstellung erscheint eine Tabelle mit den angegebenen 
Informationen zum Autor respektive zu den Autoren der Abschlussarbeit. Zwingend 
anzugeben sind dafür lediglich ein oder mehrere Verfasser der Abschlussarbeit 
(\Macro(\Bundle{tudscr}){author}) sowie der dazugehörige Titel (\Macro{title}), 
welcher am Ende der Tabelle in fetter Schrift aufgeführt wird. Optional werden 
noch die Felder für den Studiengang (\Macro(\Bundle{tudscr}){course}), die 
Fachrichtung (\Macro(\Bundle{tudscr}){discipline}) sowie Matrikelnummer und 
Immatrikulationsjahr (\Macro{matriculationnumber}, \Macro{matriculationyear}) 
angefügt, wobei nicht befüllte Felder ignoriert werden. Der eigentliche Inhalt 
der Umgebung~-- sprich die Aufgabenstellung selbst~-- wird nach dem generierten 
Kopf ausgegeben.

Dem Inhalt der Aufgabenstellung folgt eine zeilenweise Auflistung der 
angegebenen Gutachter respektive Prüfer (\Macro{referee}) sowie Betreuer 
(\Macro{supervisor}). Dabei wird vor dem jeweiligen Namen der dazugehörige 
Bezeichner (\Term{refereename}, \Term{refereeothername} beziehungsweise 
\Term{supervisorname}, \Term{supervisorothername}) gesetzt. 
Dies ist das voreingestellte Verhalten und kann über die Wahl des Parameters
\ChangedAt{v2.05}%
\Environment{task/style=table} aktiviert werden. Wird hingegen der Parameter 
\Environment{task/style=inline} gesetzt, erfolgt die Ausgabe von mehreren 
Prüfern und Betreuern in einer Zeile. Danach erscheinen das Ausgabedatum 
(\Macro{issuedate}) sowie der verpflichtende Abgabetermin (\Macro{duedate}). Am 
Ende wird für den Prüfungsausschussvorsitzenden (\Macro{chairman}) und den 
betreuenden Hochschullehrer (\Macro(\Bundle{tudscr}){professor}) eine 
Unterschriftzeile gesetzt, wobei unter den Namen selbst die dazugehörigen 
Bezeichner (\Term{chairmanname} und \Term{professorname}) ausgegeben werden. 
Die Möglichkeiten zur individuellen Anpassung der genutzten Bezeichner ist in 
\autoref{sec:localization} beschrieben. 
\end{Declaration}
\end{Declaration}
\end{Declaration}

\begin{Declaration}
  {\Macro{taskform|\OList\MPName{Ziele}\MPName{Schwerpunkte}}}
\printdeclarationlist
%
Zusätzlich zur der frei gestaltbaren Umgebung \Environment{task} zur Erstellung
einer Aufgabenstellung wird ein separater Befehl für eine standardisierte 
Ausgabe zur Verfügung gestellt. Dieser strukturiert die Aufgabenstellung in die 
zwei Bereiche \emph{Ziele} und \emph{Schwerpunkte} der Arbeit mit dazugehörigen 
Überschriften (\Term{objectivesname}, \Term{focusname}).

Im optionalen Argument können alle Parameter der Umgebung \Environment{task} 
verwendet werden. Im ersten obligatorischen Argument sollte ein Text mit einer 
kurzen thematischen Einordnung und dem eigentlichen Ziel der Arbeit erscheinen, 
im zweiten Argument sollen die thematischen Schwerpunkte in Stichpunkten 
benannt werden. Der Inhalt des zweiten notwendigen Argumentes wird in einer 
\Environment{itemize}"~Umgebung gesetzt. Deshalb \emph{muss} jedem Stichpunkt 
\Macro{item} vorangestellt werden.
\end{Declaration}
%
\begin{Example}
Die empfohlene Verwendung des Befehls \Macro{taskform} ist wie folgt:
\begin{Code}[escapechar=§]
\taskform{%
  Motivation der Arbeit im ersten Absatz§\dots§
  
  Ziele der Arbeit im zweiten Absatz§\dots§
}{%
  \item Schwerpunkt 1
  \item Schwerpunkt 2
}
\end{Code}
Hierzu sei auch auf das Minimalbeispiel in \autoref{sec:exmpl:task} verwiesen.%
\index{Aufgabenstellung|!)}%
\end{Example}

\begin{Declaration}
  {\Macro{chairman|\MPName{Prüfungsausschussvorsitzender}}}
\printdeclarationlist
%
Mit diesem Befehl wird am Ende der Aufgabenstellung zusätzlich zum betreuenden 
Hochschullehrer (\Macro(\Bundle{tudscr}){professor}) auch der Vorsitzende des 
Prüfungsausschusses aufgeführt. Dies wird zumeist für Abschlussarbeiten wie 
beispielsweise \masterthesisname{} oder \diplomathesisname{} benötigt.
\end{Declaration}

\begin{Declaration}
  {\Macro{issuedate|\MPName{Ausgabedatum}}}
\begin{Declaration}
  {\Macro{duedate|\MPName{Abgabetermin}}}
\printdeclarationlist
%
Mit diesen beiden Befehlen sollte das Datum der Ausgabe der Aufgabenstellung 
sowie der spätest mögliche Abgabetermin angegeben werden. Wurde das Paket 
\Package{isodate} oder \Package{datetime2} geladen, wird die damit eingestellte 
Ausgabeformatierung des Datums mit \Macro{printdate} respektive \Macro{DTMDate} 
für \Macro{duedate} und \Macro{issuedate} verwendet.
\end{Declaration}
\end{Declaration}



\section{%
  Gutachten für wissenschaftliche Arbeiten%
  \index{Gutachten|!(}%
}
%
\begin{Declaration}
  {\Environment{evaluation|\OPList{Überschrift}}}
  <\Environment{tudpage}>
\begin{Declaration}
  {\Environment{evaluation/headline=\PSet{Überschrift}}}
\begin{Declaration}
  {\Environment{evaluation/grade=\PSet{Note}}}
\printdeclarationlist
%
Diese Umgebung wird für das Erstellen eines Gutachtens einer wissenschaftlichen 
Arbeit bereitgestellt. Auch diese unterstützt alle Parameter, welche für die 
Umgebung \Environment{tudpage} beschrieben wurden. Für ein Gutachten wird 
gewöhnlich eine Überschrift aus \Term{evaluationname} und~-- falls der 
Abschlussarbeitstyp angegeben wurde~-- \Term{evaluationtext} sowie 
\Macro{thesis} generiert. Diese automatisch generierte Überschrift kann mit dem 
Parameter \Environment{evaluation/headline} ersetzt werden. Am Ende des 
Gutachtens wird die mit \Environment{evaluation/grade} gegebene Note in fetter 
Schrift ausgezeichnet.

Am Anfang der \Environment{evaluation}"~Umgebung wird die gleiche Tabelle mit 
Autorenangaben ausgegeben, wie dies bei der \Environment{task}"~Umgebung der 
Fall ist. Nach dem Tabellenkopf folgt auch hier der eigentliche Inhalt, sprich 
das Gutachten der Abschlussarbeit. Abgeschlossen wird die Umgebung mit der 
gegebenen Note~-- welche innerhalb von \Term{gradetext} ausgegeben wird~-- 
sowie der Orts- und Datumsangabe (\Macro{place}, \Macro{date}) und der 
darauffolgenden Unterschriftzeile für den oder die Gutachter (\Macro{referee}), 
welche wiederum mit den entsprechenden sprachabhängigen Bezeichner 
(\Term{refereename}, \Term{refereeothername}) ergänzt werden.
\end{Declaration}
\end{Declaration}
\end{Declaration}

\begin{Declaration}
  {\Macro{evaluationform|%
    \OList\MPName{Aufgabe}\MPName{Inhalt}\MPName{Bewertung}\MPName{Note}}}
\printdeclarationlist
%
Neben der individuell nutzbaren Umgebung \Environment{evaluation} wird ein 
separater Befehl zur Erstellung eines standardisierten Gutachtens 
bereitgestellt. Dieser strukturiert die Ausgabe in die vier Bereiche 
\emph{Aufgabe}, \emph{Inhalt}, \emph{Bewertung} und \emph{Note} und versieht 
diese jeweils mit der dazugehörigen Überschrift beziehungsweise Textausgabe 
(\Term{taskname}, \Term{contentname}, \Term{assessmentname} und 
\Term{gradetext}). Das optionale Argument unterstützt alle Parameter der 
\Environment{evaluation}"~Umgebung.
\end{Declaration}
%
\begin{Example}
Die empfohlene Verwendung des Befehls \Macro{evaluationform} ist wie folgt:
\begin{Code}[escapechar=§]
\evaluationform{%
  Kurzbeschreibung der Aufgabenstellung§\dots§
}{%
  Zusammenfassung von Inhalt und Struktur§\dots§
}{%
  Bewertung der schriftlichen Abschlussarbeit§\dots§
}{%
  Zahl (Note)
}
\end{Code}
Hierzu sei auch auf das Minimalbeispiel in \autoref{sec:exmpl:evaluation} 
verwiesen.%
\index{Gutachten|!)}%
\end{Example}

\begin{Declaration}
  {\Macro{grade|\MPName{Note}}}
\printdeclarationlist
%
Neben der Angabe der Note für ein Gutachten über den Parameter 
\Environment{evaluation/grade} der Umgebung \Environment{evaluation} kann dafür 
auch dieser global wirkende Befehl verwendet werden.
\end{Declaration}



\section{%
  Aushang%
  \index{Aushang|!(}%
}
%
\begin{Declaration}
  {\Environment{notice|\OPList{Überschrift}}}
  <\Environment{tudpage}>
\begin{Declaration}
  {\Environment{notice/headline=\PSet{Überschrift}}}
\printdeclarationlist
%
Für das Anfertigen eines Aushangs kann diese Umgebung verwendet werden, welche
abermals auf der \Environment{tudpage}"=Umgebung basiert und deren Parameter 
unterstützt. Wurde mit \Macro{date} ein Datum angegeben, wird dieses 
standardmäßig rechtsbündig oberhalb des Textbereiches angezeigt 
(\Environment||{tudpage/cdhead=date}). Danach erfolgt die Ausgabe der 
Überschrift, welche normalerweise \Term{noticename} entspricht und mit dem 
Parameter \Environment{notice/headline} geändert werden kann. Nach der 
Überschrift folgt der Inhalt der Umgebung. Wurde mit \Macro{contactperson} ein 
oder mehrere Ansprechpartner angegeben, werden diese Informationen am Ende der 
Umgebung gesetzt.
\end{Declaration}
\end{Declaration}

\begin{Declaration}
  {\Macro{noticeform|\OList\MPName{Inhalt}\MPName{Schwerpunkte}}}
\printdeclarationlist%
%
Auch für diese Umgebung gibt es einen Befehl für eine normierte Form. Diese 
soll vor allem Verwendung für den Aushang von Angeboten für studentische 
Arbeiten finden. Für das optionale Argument lassen sich sämtliche Parameter 
verwenden, die auch von der \Environment{notice}"~Umgebung unterstützt werden.

Das erste obligatorische Argument wird für eine kurze Inhaltsbeschreibung 
verwendet. Neben dem Text sollte hier~-- wenn möglich~-- eine thematisch 
passende Abbildung (\Macro{includegraphics}) eingefügt werden. Mit dem zweiten 
Argument erfolgt~-- wie schon bei \Macro{taskform}~-- die Ausgabe einiger 
Schwerpunkte, wobei hier ebenso eine \Environment{itemize}"~Umgebung direkt 
nach der Überschrift (\Term{focusname}) zum Einsatz kommt und deshalb allen 
Stichpunkten ein \Macro{item} vorangestellt werden \emph{muss}.
\ToDo{Declaration item und dabei suffix-Angabe für gültige Umgebungen?!}
\end{Declaration}
%
\begin{Example}
Die empfohlene Verwendung des Befehls \Macro{noticeform} ist wie folgt:
\begin{Code}[escapechar=§]
\noticeform{%
  Kurzbeschreibung des Inhaltes der studentischen Arbeit§\dots§
  
  Bild (optional), einzubinden mit:
    \includegraphics[§\PName{Einstellungen}§]{§\PName{Datei}§}
}{%
  \item Schwerpunkt 1
  \item Schwerpunkt 2
  \item Schwerpunkt 3
}
\end{Code}
Hierzu sei auch auf das Minimalbeispiel in \autoref{sec:exmpl:notice} 
verwiesen.%
\index{Aushang|!)}%
\end{Example}

\begin{Declaration}
  {\Macro{contactperson|\MPName{Kontaktperson(en)}}}
  [v2.02]
\begin{Declaration*}
  {\Macro{emailaddress|\OPName{Einstellungen}\MPName{E-Mail-Adresse}}}
\begin{Declaration*}
  {\Macro{emailaddress*|\MPName{E-Mail-Adresse}}}
\begin{Declaration}
  {\Macro{telephone|\MPName{Telefonnummer}}}
  [v2.02]
\begin{Declaration}
  {\Macro{telefax|\MPName{Telefaxnummer}}}
  [v2.05]
\begin{Declaration}
  {\Macro{office|\MPName{Dienstsitz}}}
\printdeclarationlist
%
\ToDo{Hinweise/Links auf Standardklassen (in Deklaration)?}[v2.07]
%  \Macro||'full'(\Bundle{tudscr}){author} bzw wenn umgesetzt mit
%  begin{Declaration}{...}(\Macro(\Bundle{tudscr}){author}),
Am Ende eines Aushangs lassen sich mit dem Befehl \Macro{contactperson} 
Kontaktinformationen für einen oder mehr Ansprechpartner angeben. Mehrere 
Kontaktpersonen müssen innerhalb dieses Befehls mit \Macro{and} voneinander 
getrennt werden. Für jede Person kann innerhalb des Argumentes der Dienstsitz 
(\Macro{office}), die dienstliche Telefon- (\Macro{telephone}) sowie "~faxnummer
(\Macro{telefax}) und die geschäftliche E"~Mail"=Adresse (%
  \Macro(\Bundle{tudscr}){emailaddress}%
  \footnote{Erläuterungen zu \Macro||'full'(\Bundle{tudscr}){emailaddress}}%
) angegeben werden. 
\end{Declaration}
\end{Declaration}
\end{Declaration}
\end{Declaration*}
\end{Declaration*}
\end{Declaration}

%\begin{Declaration}
%  {\Macro{webpage|\OPName{Einstellungen}\MPName{URL}}}
%\begin{Declaration}
%  {\Macro{webpage*|\MPName{URL}}}
%\printdeclarationlist%
%%
%Ganz zum Schluss kann für die rechte Spalte des Fußbereichs eine Homepage 
%angegeben werden. Wurde das Paket \Package{hyperref} geladen, wird diese in 
%einen Hyperlink gewandelt. Über das optionale Argument können beliebige 
%Einstellungen an \Macro{hypersetup} aus besagtem Paket übergeben werden. Soll 
%die Formatierung des Eintrags manuell erfolgen, so kann die Sternversion 
%\Macro{webpage*} verwendet werden, wobei alle gewünschten Einstellungen 
%innerhalb des Argumentes~-- gegebenenfalls in einer Gruppe~-- vorgenommen 
%werden müssen.
%\end{Declaration}
%\end{Declaration}



\section{%
  Zusätzliche sprachabhängige Bezeichner%
  \index{Bezeichner|!(}%
}
%
Für die zuvor erläuterten Befehle und Umgebungen das werden ergänzend zu den 
\TUDScript-Klassen weitere Bezeichner definiert. Für eine etwaige Anpassung 
dieser sei auf \autoref{sec:localization} verwiesen. Soll ein bestimmter 
Bezeichner lediglich einmalig für eine Umgebung oder einen Befehl geändert 
werden, sollten Sie weiterhin die Empfehlungen in \autoref{sec:tips:local} 
beachten.

\begin{Declaration}
  {\Term{taskname}}
\begin{Declaration}
  {\Term{tasktext}}
\printdeclarationlist
%
Die Bezeichnung der Aufgabenstellung selbst ist in \Term{taskname} enthalten. 
Für die Generierung einer Überschrift wird dieser verwendet. Wurde außerdem mit 
\Macro{thesis} oder \Macro{subject} der Typ der Abschlussarbeit%
\footnote{%
  spezieller Wert aus \autoref{tab:thesis} oder \Option||{subjectthesis=true}%
}
angegeben, wird die Überschrift zusammen mit dem Bezeichner \Term{tasktext}
um die Typisierung erweitert. Falls gewünscht, kann die automatisch generierte 
Überschrift mit dem Parameter \Environment{task/headline} der Umgebung 
\Environment{task} überschrieben werden.
\TermTable%
\end{Declaration}
\end{Declaration}

\begin{Declaration}
  {\Term{namesname}}
  [v2.04]
\begin{Declaration}
  {\Term{titlename}}
\printdeclarationlist
%
Diese beiden Bezeichner werden in der Tabelle mit den Autoreninformationen zu 
Beginn der Aufgabenstellung verwendet.
\TermTable%
\end{Declaration}
\end{Declaration}

\begin{Declaration}
  {\Term{issuedatetext}}
\begin{Declaration}
  {\Term{duedatetext}}
\printdeclarationlist
%
Am Ende der Aufgabenstellung wird nach dem oder der Betreuer das Ausgabedatum 
und der Abgabetermin (\Macro{issuedate}, \Macro{duedate}) der Abschlussarbeit 
mit folgenden Bezeichner erläutert.
\TermTable%
\end{Declaration}
\end{Declaration}

\begin{Declaration}
  {\Term{chairmanname}}
\printdeclarationlist
%
Wurde der Prüfungsausschussvorsitzende (\Macro{chairman}) angegeben, erfolgt 
unter dem Namen selbst die Ausgabe des Bezeichners.
\TermTable%
\end{Declaration}

\begin{Declaration}
  {\Term{focusname}}
\begin{Declaration}
  {\Term{objectivesname}}
\printdeclarationlist
%
Die Standardformen für Aufgabenstellung (\Macro{taskform}) respektive Aushang 
(\Macro{noticeform}) nutzen für die gesetzten Überschriften diese Bezeichner.
\TermTable%
\end{Declaration}
\end{Declaration}

\begin{Declaration}
  {\Term{evaluationname}}
\begin{Declaration}
  {\Term{evaluationtext}}
\printdeclarationlist
%
Die Bezeichnung des Gutachten selbst ist in \Term{evaluationname} enthalten. 
Für die Generierung der Überschrift wird der Bezeichner \Term{evaluationtext} 
sowie der mit \Macro{thesis} oder gegebenenfalls mit \Macro{subject} gegebenen 
Typ der Abschlussarbeit verwendet. Diese automatisch generierte Überschrift 
kann mit dem Parameter \Environment{evaluation/headline} der Umgebung 
\Environment{evaluation} durch den Anwender überschrieben werden.
\TermTable%
\end{Declaration}
\end{Declaration}

\begin{Declaration}
  {\Term{contentname}}
\begin{Declaration}
  {\Term{assessmentname}}
\printdeclarationlist
%
Bei der standardisierten Form des Gutachtens (\Macro{evaluationform}) werden 
die darin~-- für eine strukturierte Gliederung~-- erzeugten Überschriften mit 
den Bezeichnern \Term{taskname}, \Term{contentname} und \Term{assessmentname} 
gesetzt.
\TermTable%
\end{Declaration}
\end{Declaration}

\begin{Declaration}
  {\Term{gradetext}}
\printdeclarationlist
%
Wird für das Gutachten einer wissenschaftlichen Arbeit die erzielte Note 
entweder mit dem Befehl \Macro{grade|\MPName{Note}} oder alternativ dazu 
mit dem Parameter \Environment{evaluation/grade=\PSet{Note}} der Umgebung 
\Environment{evaluation} angegeben, so wird diese innerhalb von 
\Term{gradetext} verwendet.
\grade{\PName{Note}}
\TermTable[.8\textwidth]%
\end{Declaration}

\begin{Declaration}
  {\Term{noticename}}
\begin{Declaration}
  {\Term{contactpersonname}}
  [v2.02]
\printdeclarationlist
%
Die Bezeichnung des Aushangs selbst ist in \Term{noticename} enthalten. Für 
die Generierung einer Überschrift wird dieser verwendet. Falls gewünscht, kann 
diese mit dem Parameter \Environment{notice/headline} der Umgebung 
\Environment{notice} überschrieben werden. Wurde eine Kontaktperson mit dem 
Befehl \Macro{contactperson} angegeben, wird als Überschrift der Kontaktdaten 
der Bezeichner \Term{contactpersonname} verwendet.
\TermTable%
\end{Declaration}
\end{Declaration}
\index{Bezeichner|!)}%
\end{DeclareEntity}
