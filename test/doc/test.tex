\PassOptionsToPackage{check=none}{widows-and-orphans}%
\documentclass[english,ngerman,ToDo=inline,final,indexnote=false]{tudscrmanual}
\GitHubBase{\TUDScriptRepository}
\iftutex
  \usepackage{fontspec}
\else
  \usepackage[T1]{fontenc}
  \input glyphtounicode.tex
  \pdfgentounicode=1
  \usepackage[ngerman=ngerman-x-latest]{hyphsubst}
\fi
\lstset{%
  inputencoding=utf8,extendedchars=true,
  literate=%
    {ä}{{\"a}}1 {ö}{{\"o}}1 {ü}{{\"u}}1
    {Ä}{{\"A}}1 {Ö}{{\"O}}1 {Ü}{{\"U}}1
    {ß}{{\ss}}1 {~}{{\textasciitilde}}1
    {»}{{\guillemetright}}1 {«}{{\guillemetleft}}1
}
\usepackage{widows-and-orphans}
\usepackage{bookmark}
\KOMAoptions{headings=optiontoheadandtoc}

\tracinglabels*<created,matched>[all]
\tracinglabels[all]
%\tracingmarkup%[all]
\usepackage{blindtext}

\usepackage{lmodern}

\begin{document}

\begin{Declaration}
  {\Environment{abstract|\OPList{Sprache}}}
\begin{Declaration}
  {\Environment{abstract/markboth=\PMisc}}
\printdeclarationlist
Zur Festlegung der Kolumnentitel existieren für 
\Environment{abstract/markboth} 
folgende Einstellungen:%
\begin{DeclareValues}[\Environment{abstract/markboth}]
\itemval=false=
  Die aktuellen Kolumnentitel~-- je nach gewählter Option für das Paket 
  \Package{scrlayer-scrpage} automatische (\Option{automark}) oder manuelle 
  (\Option{manualmark})~-- werden verwendet.
\itemval*=true=
  Die Kolumnentitel für linke und rechte Seiten werden auf \Term{abstractname} 
  gesetzt.
\itemval=\PName{Kolumnentitel}=
  Die Kolumnentitel werden manuell festgelegt. So lassen sich die Kolumnen 
  beispielsweise mit \Environment{abstract/markboth=\MPValue{}} auch 
  vollständig löschen. 
\end{DeclareValues}
\end{Declaration}
\end{Declaration}

\end{document}
