\PassOptionsToPackage{check=none}{widows-and-orphans}%
\documentclass[english,ngerman,ttfont=roboto,ToDo=inline,final]{tudscrmanual}
\GitHubBase{\TUDScriptRepository}
\iftutex
  \usepackage{fontspec}
\else
  \usepackage[T1]{fontenc}
  \input glyphtounicode.tex
  \pdfgentounicode=1
  \usepackage[ngerman=ngerman-x-latest]{hyphsubst}
\fi
\lstset{%
  inputencoding=utf8,extendedchars=true,
  literate=%
    {ä}{{\"a}}1 {ö}{{\"o}}1 {ü}{{\"u}}1
    {Ä}{{\"A}}1 {Ö}{{\"O}}1 {Ü}{{\"U}}1
    {ß}{{\ss}}1 {~}{{\textasciitilde}}1
    {»}{{\guillemetright}}1 {«}{{\guillemetleft}}1
}
\usepackage{widows-and-orphans}
\usepackage{bookmark}
\KOMAoptions{headings=optiontoheadandtoc}

%\tracinglabels*<created,matched>[all]
\tracinglabels[all]
%\tracingmarkup%[all]
\usepackage{blindtext}


%\DefaultEntity{\Class{datetime2}}{
%  \DefaultEntity{\Class{daatetime2}}{
%    \Macro{DTMDate},
%  }
%}

%\DefaultEntity{\Bundle{enumitem}}{
%  \DefaultEntity{\Package{enumitem}}{
%    \DefaultEntity{\Environment{itemize}}{\Macro{item}},
%    \DefaultEntity{\Environment{enumerate}}{\Macro{item}},
%    \DefaultEntity{\Environment{description}}{\Macro{item}},
%    \Macro{setlist},
%  }
%}

\begin{document}

\makeatletter

\let\foo\@empty%
\listadd\foo{F}
\listadd\foo{A}
\listadd\foo{V}
\listadd\foo{C}
\listadd\foo{D}
\meaning\foo

\tud@list@sort\foo
\meaning\foo

\end{document}
\begin{Entity}{\Bundle{tudscr}}
\begin{Declaration}{\Option{declaration=\PMisc}}=hhh=
\begin{Declaration}{\Option{foo=\PMisc}}
\begin{Declaration}
  {\Environment{abstract|\OPList{Sprache}}}
%  [v2.02:Trennung einzelner Abschnitte mit \Macro{nextabstract};]
\begin{Declaration}
  {\Environment{abstract/markboth=\PMisc}}
  [v2.02]
\begin{Declaration}
  {\Environment{abstract/pagestyle=\PSet{Seitenstil}}}
%  [v2.02]
\printdeclarationlist
%
Mit \Option{declaration} kann äquivalent zur Option \Option{abstract} die 
%
%\begin{DeclareValues}%[\Option{declaration=\PMisc}]
%\itemval=bla=<foo>
%  Es wird keine Überschrift über den Erklärungen selbst ausgegeben.
%\itemval*=true=
%  Es wird eine Überschrift über den Erklärungen selbst ausgegeben.
%\end{DeclareValues}
%
%%\ChangedAt{v2.02}%
%%Mit dem Parameter \Environment{abstract/markboth} lassen sich die verwendeten 
%%Kolumnentitel anpassen. Die aktuellen~-- automatische respektive manuelle~-- 
%%werden mit \InlineDeclaration{\Environment{abstract/markboth=false}} 
%%verwendet. 
%%Die Einstellung \InlineDeclaration{\Environment{abstract/markboth=true}} 
%%wiederum setzt diese für linke und rechte Seiten auf \Term{abstractname}. Mit 
%%\InlineDeclaration{\Environment{abstract/markboth=\PName{Kolumnentitel}}} 
%%wird 
%%der Kolumnentitel manuell festgelegt. So lassen sich die Kolumnen mit 
%%\Environment{abstract/markboth=\MPValue{}} beispielsweise auch vollständig 
%%löschen. Sollte der Parameter \Environment{abstract/markboth} aktiviert 
%%werden, 
%%so wird in der Umgebung automatisch der Seitenstil \PageStyle{headings} 
%%genutzt~-- falls eine Titelseite (\KOMAScript-Option \Option{titlepage=true}) 
%%verwendet wird. Mit dem Parameter \Environment{abstract/pagestyle} lässt sich 
%%dieser auch direkt angeben, wobei die \PageStyle{tudheadings}"=Seitenstile 
%%ebenfalls unterstützt werden.
\end{Declaration}
\end{Declaration}
\end{Declaration}
\end{Declaration}
\end{Declaration}
\end{Entity}


%\InlineDeclaration{\Package{enumitem}}
%
%\InlineDeclaration{\Macro{setlist}}

\end{document}




\begin{Entity}{\Bundle{something}}
\begin{Declaration}{\Option{cdfont=\PMisc}}
\printdeclarationlist
%
Mit der Option \Option{cdfont} können durch den Anwender alle zentralen
\end{Declaration}
\end{Entity}

%\begin{Entity}{\Bundle{stuff}}
%\begin{Declaration}{\Option{cdfont=\PMisc}}
%\printdeclarationlist
%%
%Mit der Option \Option{cdfont} können durch den Anwender alle zentralen
%\end{Declaration}
%\end{Entity}


%\Option'auto'{cdfont}

\Package'GH:foo'{geometry}

\makeatletter
\tud@specialurl@resolve\tud@res@a{}{aaaaa}%

\meaning\tud@res@a


\end{document}

\ChangedAt{v2.02;v2.06}%
Durch das \CD werden keine Schriften für den Mathematiksatz festgelegt. Das ist 
insbesondere für mathematische Abhandlungen als auch ingenieur- und 
naturwissenschaftliche Dokumente nicht tragbar. Im Mathematikmodus werden 
deshalb die lateinischen Lettern mithilfe des Paketes \Package{mathastext}
sowie die griechischen Lettern der \OpenSans genutzt. Zur Ergänzung kann für 
weitere mathematische Symbole das Paket \Package{mdsymbol} geladen werden.
\ToDo{hinweis zu mdsymbol raus}[v2.07]

Diese Einstellung lässt sich deaktivieren, wodurch sich die Standardschriften 
oder gegebenenfalls die eines zusätzlichen Paketes für den mathematischen Satz 
nutzen lassen. Die dafür relevanten Einstellungen werden in \autoref{sec:math} 
erläutert. Weiterhin sind ergänzende Hinweise zu einem typografisch sauberen
Mathematiksatz in \autoref{sec:tut} zu finden.

\section{title}

\Macro{KOMAClassName}

\Macro{caption}

\end{document}

\begin{Declaration}
  {\Macro{foo}}
\printdeclarationlist
%
Mit dieser Option kann die Größe des unteren Seitenrandes angepasst werden, 
\end{Declaration}

\begin{DeclareEntity}{\Class{tudscrtest}}
\begin{Declaration}
  {\Macro{extrabottommargin}}
\printdeclarationlist
%
Mit dieser Option kann die Größe des unteren Seitenrandes angepasst werden, 
\end{Declaration}

\Macro'full'{extrabottommargin}

\Macro'full'{extrabottommargin/stuff}

\Macro'full'(\Class{tudscrtest},\Class{tudscrposter}){extrabottommargin/stuff}

\Macro'full'(\Class{tudscrtest}){extrabottommargin/stuff}

\Macro(\Class{tudscrposter}){extrabottommargin}

\end{DeclareEntity}

\begin{DeclareEntity}{\Class{tudscrposter}}
\begin{Declaration}
  {\Macro{extrabottommargin}}
\begin{Declaration}
  {\Macro{extrabottommargin/stuff}}
\printdeclarationlist
%
Mit dieser Option kann die Größe des unteren Seitenrandes angepasst werden, 
\end{Declaration}
\end{Declaration}

\Macro{extrabottommargin}

\Macro{extrabottommargin/stuff}

\Macro(\Class{tudscrposter},\Class{tudscrtest}){extrabottommargin}

\end{DeclareEntity}

\begin{DeclareEntity}{\Class{tudscrclass}}
\begin{Declaration}
  {\Macro{extrabottommargin}}
\begin{Declaration}
  {\Macro{extrabottommargin/stuff}}
\printdeclarationlist
%
Mit dieser Option kann die Größe des unteren Seitenrandes angepasst werden, 
\end{Declaration}
\end{Declaration}

\Macro{extrabottommargin}

\Macro{extrabottommargin/stuff}

\Macro(\Class{tudscrposter},\Class{tudscrtest}){extrabottommargin}

\end{DeclareEntity}

---------------------------------------

\Macro(\Bundle{tudscr}){extrabottommargin}

\Macro(\Bundle{tudscr},\Class{tudscrposter}){extrabottommargin}

\Macro(\Bundle{tudscr},\Class{tudscrposter},\Class{tudscrtest}){extrabottommargin}

\Macro{extrabottommargin/stuff}

\Macro{foo}

\Environment{itemize}

%\Macro(\Bundle{bar}){bar}
%
%\Macro(\Bundle{bar},\Bundle{blubb}){bar}

\Macro(\Class{tudscrposter},\Class{tudscrtest}){extrabottommargin}

\end{document}
