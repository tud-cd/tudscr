\PassOptionsToPackage{check=none}{widows-and-orphans}%
\documentclass[english,ngerman,ttfont=roboto,ToDo=inline,final]{tudscrmanual}
\GitHubBase{\TUDScriptRepository}
\iftutex
  \usepackage{fontspec}
\else
  \usepackage[T1]{fontenc}
  \input glyphtounicode.tex
  \pdfgentounicode=1
  \usepackage[ngerman=ngerman-x-latest]{hyphsubst}
\fi
\lstset{%
  inputencoding=utf8,extendedchars=true,
  literate=%
    {ä}{{\"a}}1 {ö}{{\"o}}1 {ü}{{\"u}}1
    {Ä}{{\"A}}1 {Ö}{{\"O}}1 {Ü}{{\"U}}1
    {ß}{{\ss}}1 {~}{{\textasciitilde}}1
    {»}{{\guillemetright}}1 {«}{{\guillemetleft}}1
}
\usepackage{widows-and-orphans}
\usepackage{bookmark}
\KOMAoptions{headings=optiontoheadandtoc}

\tracinglabels*<created,matched>[all]
\tracinglabels[all]
%\tracingmarkup%[all]
\usepackage{blindtext}

\DefaultEntity{\Bundle{letltxmacro}}'CTAN:foo'{\Package{letltxmacro}}

\begin{document}

bbb

\CrossIndex{Optionen}{options}%
\CrossIndex*{Befehle}{macros}%
\CrossIndex{Bezeichner}{terms}%
\CrossIndex{Seitenstile,Schriftelemente,Farben}{elements}%
\CrossIndex*{Längen}{misc}%
\CrossIndex{Zähler}{misc}%
\CrossIndex*{Klassen,Pakete,Dateien}[Index der Dateien etc.]{files}%
\CrossIndex*{Änderungen,Changelog,Version}[Änderungsliste]{changelog}%
%
\SeeRef*{Distribution}{\Distribution{Mac\hologo{TeX}}}
\SeeRef*{Distribution}{\Distribution{\hologo{TeX}~Live}}
\SeeRef*{Distribution}{\Distribution{\hologo{MiKTeX}}}
%
\index{Abbildungen|see{Grafiken}}%
\index{Abschlussarbeit|see{Typisierung}}%
\index{Aufzählungen|see{Listen}}%
\index{Autorenangaben|see{Titel}}%
\index{Bindekorrektur|see{Satzspiegel}}%
\index{Cover|see{Umschlagseite}}%
\index{Dezimaltrennzeichen|see{Zifferngruppierung}}%
\index{doppelseitiger Satz|see{Satzspiegel}}%
\index{Dresden-concept-Logo@\DDC-Logo|see{Layout}}%
\index{Drittlogo|see{Layout}}%
\index{Fachreferent|see{Referent}}%
\index{Farben|see{Layout}}%
\index{Fußzeile|see{Layout}}%
\index{Gliederung|see{Layout!Überschriften}}%
\index{Grafiken|see{Gleitobjekte}}%
\index{Kapitel|see{Layout}}%
\index{Klassenoptionen|see{Optionen}}%
\index{Kolumnentitel|see{Layout}}%
\index{Kopfzeile|see{Layout}}%
\index{Kurzfassung|see{Zusammenfassung}}%
\index{Layout!Seitenränder|see{Satzspiegel}}%
\index{Layout!Titel|see{Titel}}%
\index{Layout!Umschlagseite|see{Umschlagseite}}%
\index{Leerseiten|see{Vakatseiten}}%
\index{Lokalisierung|see{Bezeichner}}%
\index{Makros|see{Befehle}}%
\index{Mathematiksatz|see{Einheiten}}%
\index{Mathematiksatz|see{Griechische Buchstaben}}%
\index{Mathematiksatz|see{Zifferngruppierung}}%
\index{Nutzerinstallation|see{Installation}}%
\index{Outline-Eintrag|see{Lesezeichen}}%
\index{Parameter|see{Befehle}}%
\index{Professor|see{Hochschullehrer}}%
\index{Querbalken|see{Layout}}%
\index{Seitenränder|see{Satzspiegel}}%
\index{Seitenstil|see{Layout}}%
\index{Silbentrennung|see{Worttrennung}}%
\index{Sprachunterstützung|see{Bezeichner}}%
\index{Sprachunterstützung|see{Worttrennung}}%
\index{Sprungmarken|see{Lesezeichen}}%
\index{Tabellen|see{Gleitobjekte}}%
\index{Tausendertrennzeichen|see{Zifferngruppierung}}%
\index{Teil|see{Layout}}%
\index{Titel!Umschlagseite|see{Umschlagseite}}%
\index{Überfüllung|see{Beschnittzugabe}}%
\index{Umgebungen|see{Befehle}}%
\index{Trennungsmuster|see{Worttrennung}}%
\index{Vektorgrafiken|see{Grafiken}}%
\index{Versalien|see{Schriftauszeichnung}}%
\index{zweiseitiger Satz|see{Satzspiegel}}%
\index{zweispaltiger Satz|see{Satzspiegel}}%
\index{Zweitlogo|see{Layout}}%


\setchapterpreamble{%
  \addparttocentry{}{\indexname}%
  \begin{abstract}
    \noindent Die Formatierung der Einträge in allen aufgeführten Indexen ist 
    folgendermaßen aufzufassen: \textbf{Zahlen in fetter Schrift} verweisen auf 
    die \textbf{Erklärung} zu einem Stichwort, wobei in der digitalen Fassung 
    dieses Handbuchs dieser Eintrag selbst ein Hyperlink zu seiner Erläuterung 
    ist. Seitenzahlen in normaler Schriftstärke hingegen deuten auf zusätzliche 
    Informationen, wobei diese für \textit{kursiv hervorgehobene Zahlen} als 
    besonders \textit{wichtig} erachtet werden.
    
    Bei Einträgen für \hyperref[idx:options]{Klassen- und Paketoptionen}, 
    \hyperref[idx:macros]{Umgebungen und Befehlen} sowie 
    \hyperref[idx:elements]{Seitenstilen, Schriftelementen und Farben} 
    respektive \hyperref[idx:misc]{Längen und Zählern}, zu denen keine direkte 
    \textbf{Erklärung} existiert~-- und diese dementsprechend auch nicht als 
    Hyperlink dargestellt sind~-- handelt es sich um zusätzliche Hinweise für 
    Elemente, die nicht durch \TUDScript sondern von anderen Klassen oder 
    Paketen zur Verfügung gestellt werden, welche im Suffix des entsprechenden 
    Eintrages angegeben werden. 
  \end{abstract}
}
\chapter*{\indexname}
\PrintIndex
\PrintChangelog


\end{document}


%\Color*{cd\PName{Farbe}} genutzt werden.
%
%\begin{Example*}
%Die Grundfarbe \Color{HKS44} soll in der auf 20\% reduzierten, helleren 
%Abstufung genutzt werden. Innerhalb eines Befehls, der als Argument eine 
%gültige Farbe erwartet, muss lediglich \PValue{HKS44!20} angegeben werden. 
%Dies wird hier exemplarisch mit der folgenden \colorbox{HKS44!20}{%
%  Box \Macro{colorbox|\MPValue{HKS44!20}\MPValue{Box}}%
%}
%demonstriert.
%\end{Example*}
%
%Bei der farbigen Gestaltung des \CDs ist~-- je nach spezifisch gewählter
%
%\Color*{cd\PName{Farbe}} genutzt werden.
%
%\begin{Example}
%Die Grundfarbe \Color{HKS44} soll in der auf 20\% reduzierten, helleren 
%Abstufung genutzt werden. Innerhalb eines Befehls, der als Argument eine 
%gültige Farbe erwartet, muss lediglich \PValue{HKS44!20} angegeben werden. 
%Dies wird hier exemplarisch mit der folgenden \colorbox{HKS44!20}{%
%  Box \Macro{colorbox|\MPValue{HKS44!20}\MPValue{Box}}%
%}
%demonstriert.
%\end{Example}
%
%Bei der farbigen Gestaltung des \CDs ist~-- je nach spezifisch gewählter
%
%
%Sollen bestimmte Minuskeln erhalten bleiben, ist \Macro{NoCaseChange} im 
%Argument eines Gliederungsbefehls zu verwenden.
%%
%\begin{Example*}
%In einem Kapitel wird ein einzelnes Wort in Minuskeln geschrieben:
%\begin{Code}[escapechar=§]
%\chapter{Überschrift mit \NoCaseChange{kleinem} Wort}
%\end{Code}
%\end{Example*}
%%
%Die Schrift \DIN durfte laut \CD ausschließlich mit Majuskeln verwendet 
%werden, 
%weshalb das beschriebene Vorgehen lediglich im \emph{Ausnahmefall} anzuwenden 
%
%
%Sollen bestimmte Minuskeln erhalten bleiben, ist \Macro{NoCaseChange} im 
%Argument eines Gliederungsbefehls zu verwenden.
%%
%\begin{Example}
%In einem Kapitel wird ein einzelnes Wort in Minuskeln geschrieben:
%\begin{Code}[escapechar=§]
%\chapter{Überschrift mit \NoCaseChange{kleinem} Wort}
%\end{Code}
%\end{Example}
%%
%Die Schrift \DIN durfte laut \CD ausschließlich mit Majuskeln verwendet 
%werden, 
%weshalb das beschriebene Vorgehen lediglich im \emph{Ausnahmefall} anzuwenden 


%\begin{Entity}{\Bundle{tudscr}}
\begin{Cessations}{v2.00}
\begin{Cessation}
  {\Option{parskip=\PBoolean}}
  <\Option{fontsize}>
\printdeclarationlist
%
Die Einstellungen der farbigen Ausprägung des Dokumentes erfolgt über die 
\end{Cessation}
\end{Cessations}
%
%\begin{Declaration}
%  {\Environment{abstract|\OPList{Sprache}}}
%\begin{Declaration}
%  {\Environment{abstract/markboth=\PMisc}}
%\printdeclarationlist
%Zur Festlegung der Kolumnentitel existieren für 
%\Environment{abstract/markboth} 
%folgende Einstellungen:
%\begin{DeclareValues}[\Environment{abstract/markboth}]
%\itemval=false=
%  Die aktuellen Kolumnentitel~-- je nach gewählter Option für das Paket 
%  \Package{scrlayer-scrpage} automatische (\Option{automark}) oder manuelle 
%  (\Option{manualmark})~-- werden verwendet.
%\itemval*=true=
%  Die Kolumnentitel für linke und rechte Seiten werden auf \Term{abstractname} 
%  gesetzt.
%\itemval=\PName{Kolumnentitel}=
%  Die Kolumnentitel werden manuell festgelegt. So lassen sich die Kolumnen  
%  beispielsweise mit \Environment{abstract/markboth=\MPValue{}} auch 
%  vollständig löschen. 
%\end{DeclareValues}
%\end{Declaration}
%\end{Declaration}
%\end{Entity}
\end{document}

\section{title}
\begin{Changes}{v2.00}
\begin{Obsolete}
  {\Option{open=\PMisc}}
\printdeclarationlist
%
Die alternative Titelseite ist komplett aus dem \TUDScript-Bundle entfernt 
worden. Dementsprechend entfallen auch die dazugehörigen Optionen sowie Länge 
und Bezeichner.
\end{Obsolete}
\end{Changes}

%\InlineDeclaration{\Option|?|'autoref'{open=\PMisc}}

\begin{Declaration}
  {\Option{parskip=\PMisc}}
\printdeclarationlist
%
Die alternative Titelseite ist komplett aus dem \TUDScript-Bundle entfernt 
worden. Dementsprechend entfallen auch die dazugehörigen Optionen sowie Länge 
und Bezeichner.
\end{Declaration}



\begin{Declaration}
  {\Option{fontsize=\PMisc}}
\printdeclarationlist
%
Die alternative Titelseite ist komplett aus dem \TUDScript-Bundle entfernt 
worden. Dementsprechend entfallen auch die dazugehörigen Optionen sowie Länge 
und Bezeichner.
\end{Declaration}

\begin{DeclareEntity*}{\Bundle{tudscrbook}}
aaaa
\end{DeclareEntity*}

\end{document}
