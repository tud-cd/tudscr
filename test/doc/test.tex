\PassOptionsToPackage{check=none}{widows-and-orphans}%
\documentclass[english,ngerman,ttfont=roboto,ToDo=inline,final]{tudscrmanual}
\GitHubBase{\TUDScriptRepository}
\iftutex
  \usepackage{fontspec}
\else
  \usepackage[T1]{fontenc}
  \input glyphtounicode.tex
  \pdfgentounicode=1
  \usepackage[ngerman=ngerman-x-latest]{hyphsubst}
\fi
\lstset{%
  inputencoding=utf8,extendedchars=true,
  literate=%
    {ä}{{\"a}}1 {ö}{{\"o}}1 {ü}{{\"u}}1
    {Ä}{{\"A}}1 {Ö}{{\"O}}1 {Ü}{{\"U}}1
    {ß}{{\ss}}1 {~}{{\textasciitilde}}1
    {»}{{\guillemetright}}1 {«}{{\guillemetleft}}1
}
\usepackage{widows-and-orphans}
\usepackage{bookmark}
\KOMAoptions{headings=optiontoheadandtoc}

%\tracinglabels*<created,matched>[all]
\tracinglabels[all]
%\tracingmarkup
\usepackage{blindtext}

\begin{document}
Eine Umsetzung des \CDs für die \Class{beamer}"~Klasse sowie für Briefe und 
Geschäftsschreiben auf Basis von \KOMAScript ist bis jetzt leider noch nicht 
mit \TUDScript realisiert worden, soll jedoch langfristig erfolgen.

\Bundle{xyz}

\Package{xyz}

\Class{xyz}

\makeatletter

\the\glueexpr1\bigskipamount\@minus2\bigskipamount\relax

\PrintIndex

%\PrintChanges


\end{document}

\ChangedAt{v2.02;v2.06}%
Durch das \CD werden keine Schriften für den Mathematiksatz festgelegt. Das ist 
insbesondere für mathematische Abhandlungen als auch ingenieur- und 
naturwissenschaftliche Dokumente nicht tragbar. Im Mathematikmodus werden 
deshalb die lateinischen Lettern mithilfe des Paketes \Package{mathastext}
sowie die griechischen Lettern der \OpenSans genutzt. Zur Ergänzung kann für 
weitere mathematische Symbole das Paket \Package{mdsymbol} geladen werden.
\ToDo{hinweis zu mdsymbol raus}[v2.07]

Diese Einstellung lässt sich deaktivieren, wodurch sich die Standardschriften 
oder gegebenenfalls die eines zusätzlichen Paketes für den mathematischen Satz 
nutzen lassen. Die dafür relevanten Einstellungen werden in \autoref{sec:math} 
erläutert. Weiterhin sind ergänzende Hinweise zu einem typografisch sauberen
Mathematiksatz in \autoref{sec:tut} zu finden.

\section{title}

\Macro{KOMAClassName}

\Macro{caption}

\end{document}

\begin{Declaration}
  {\Macro{foo}}
\printdeclarationlist
%
Mit dieser Option kann die Größe des unteren Seitenrandes angepasst werden, 
\end{Declaration}

\begin{DeclareEntity}{\Class{tudscrtest}}
\begin{Declaration}
  {\Macro{extrabottommargin}}
\printdeclarationlist
%
Mit dieser Option kann die Größe des unteren Seitenrandes angepasst werden, 
\end{Declaration}

\Macro'full'{extrabottommargin}

\Macro'full'{extrabottommargin/stuff}

\Macro'full'(\Class{tudscrtest},\Class{tudscrposter}){extrabottommargin/stuff}

\Macro'full'(\Class{tudscrtest}){extrabottommargin/stuff}

\Macro(\Class{tudscrposter}){extrabottommargin}

\end{DeclareEntity}

\begin{DeclareEntity}{\Class{tudscrposter}}
\begin{Declaration}
  {\Macro{extrabottommargin}}
\begin{Declaration}
  {\Macro{extrabottommargin/stuff}}
\printdeclarationlist
%
Mit dieser Option kann die Größe des unteren Seitenrandes angepasst werden, 
\end{Declaration}
\end{Declaration}

\Macro{extrabottommargin}

\Macro{extrabottommargin/stuff}

\Macro(\Class{tudscrposter},\Class{tudscrtest}){extrabottommargin}

\end{DeclareEntity}

\begin{DeclareEntity}{\Class{tudscrclass}}
\begin{Declaration}
  {\Macro{extrabottommargin}}
\begin{Declaration}
  {\Macro{extrabottommargin/stuff}}
\printdeclarationlist
%
Mit dieser Option kann die Größe des unteren Seitenrandes angepasst werden, 
\end{Declaration}
\end{Declaration}

\Macro{extrabottommargin}

\Macro{extrabottommargin/stuff}

\Macro(\Class{tudscrposter},\Class{tudscrtest}){extrabottommargin}

\end{DeclareEntity}

---------------------------------------

\Macro(\Bundle{tudscr}){extrabottommargin}

\Macro(\Bundle{tudscr},\Class{tudscrposter}){extrabottommargin}

\Macro(\Bundle{tudscr},\Class{tudscrposter},\Class{tudscrtest}){extrabottommargin}

\Macro{extrabottommargin/stuff}

\Macro{foo}

\Environment{itemize}

%\Macro(\Bundle{bar}){bar}
%
%\Macro(\Bundle{bar},\Bundle{blubb}){bar}

\Macro(\Class{tudscrposter},\Class{tudscrtest}){extrabottommargin}

\end{document}
