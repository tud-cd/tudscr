\PassOptionsToPackage{check=none}{widows-and-orphans}%
\documentclass[english,ngerman,ttfont=roboto,ToDo=inline,final]{tudscrmanual}
\GitHubBase{\TUDScriptRepository}
\iftutex
  \usepackage{fontspec}
\else
  \usepackage[T1]{fontenc}
  \input glyphtounicode.tex
  \pdfgentounicode=1
  \usepackage[ngerman=ngerman-x-latest]{hyphsubst}
\fi
\lstset{%
  inputencoding=utf8,extendedchars=true,
  literate=%
    {ä}{{\"a}}1 {ö}{{\"o}}1 {ü}{{\"u}}1
    {Ä}{{\"A}}1 {Ö}{{\"O}}1 {Ü}{{\"U}}1
    {ß}{{\ss}}1 {~}{{\textasciitilde}}1
    {»}{{\guillemetright}}1 {«}{{\guillemetleft}}1
}
\usepackage{widows-and-orphans}
\usepackage{bookmark}
\KOMAoptions{headings=optiontoheadandtoc}

%\tracinglabels*<created,matched>[all]
\tracinglabels[all]
%\tracingmarkup%[all]
\usepackage{blindtext}


%\DefaultEntity{\Class{datetime2}}{
%  \DefaultEntity{\Class{daatetime2}}{
%    \Macro{DTMDate},
%  }
%}

%\DefaultEntity{\Bundle{enumitem}}{
%  \DefaultEntity{\Package{enumitem}}{
%    \DefaultEntity{\Environment{itemize}}{\Macro{item}},
%    \DefaultEntity{\Environment{enumerate}}{\Macro{item}},
%    \DefaultEntity{\Environment{description}}{\Macro{item}},
%    \Macro{setlist},
%  }
%}

\DefaultEntity{\Bundle{letltxmacro}}'CTAN:foo'{\Package{letltxmacro}}

\begin{document}

%\begin{Entity}{\Bundle{tudscr}}
\begin{Cessations}{v2.00}
\begin{Cessation}
  {\Option{parskip=\PBoolean}}
  <\Option{fontsize}>
\printdeclarationlist
%
Die Einstellungen der farbigen Ausprägung des Dokumentes erfolgt über die 
\end{Cessation}
\end{Cessations}
%
%\begin{Declaration}
%  {\Environment{abstract|\OPList{Sprache}}}
%\begin{Declaration}
%  {\Environment{abstract/markboth=\PMisc}}
%\printdeclarationlist
%Zur Festlegung der Kolumnentitel existieren für 
%\Environment{abstract/markboth} 
%folgende Einstellungen:
%\begin{DeclareValues}[\Environment{abstract/markboth}]
%\itemval=false=
%  Die aktuellen Kolumnentitel~-- je nach gewählter Option für das Paket 
%  \Package{scrlayer-scrpage} automatische (\Option{automark}) oder manuelle 
%  (\Option{manualmark})~-- werden verwendet.
%\itemval*=true=
%  Die Kolumnentitel für linke und rechte Seiten werden auf \Term{abstractname} 
%  gesetzt.
%\itemval=\PName{Kolumnentitel}=
%  Die Kolumnentitel werden manuell festgelegt. So lassen sich die Kolumnen  
%  beispielsweise mit \Environment{abstract/markboth=\MPValue{}} auch 
%  vollständig löschen. 
%\end{DeclareValues}
%\end{Declaration}
%\end{Declaration}
%\end{Entity}
\end{document}

\section{title}
\begin{Changes}{v2.00}
\begin{Obsolete}
  {\Option{open=\PMisc}}
\printdeclarationlist
%
Die alternative Titelseite ist komplett aus dem \TUDScript-Bundle entfernt 
worden. Dementsprechend entfallen auch die dazugehörigen Optionen sowie Länge 
und Bezeichner.
\end{Obsolete}
\end{Changes}

%\InlineDeclaration{\Option|?|'autoref'{open=\PMisc}}

\begin{Declaration}
  {\Option{parskip=\PMisc}}
\printdeclarationlist
%
Die alternative Titelseite ist komplett aus dem \TUDScript-Bundle entfernt 
worden. Dementsprechend entfallen auch die dazugehörigen Optionen sowie Länge 
und Bezeichner.
\end{Declaration}



\begin{Declaration}
  {\Option{fontsize=\PMisc}}
\printdeclarationlist
%
Die alternative Titelseite ist komplett aus dem \TUDScript-Bundle entfernt 
worden. Dementsprechend entfallen auch die dazugehörigen Optionen sowie Länge 
und Bezeichner.
\end{Declaration}

\begin{DeclareEntity*}{\Bundle{tudscrbook}}
aaaa
\end{DeclareEntity*}

\end{document}
