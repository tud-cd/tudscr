\PassOptionsToPackage{check=none}{widows-and-orphans}%
\documentclass[english,ngerman,ToDo=inline,indexnote=false]{tudscrmanual}
\GitHubBase{\TUDScriptRepository}

%\tracinglabels<created,matched>[all]
\tracinglabels[all]
%\tracingmarkup

\iftutex
  \usepackage{fontspec}
\else
  \usepackage[T1]{fontenc}
  \usepackage[ngerman=ngerman-x-latest]{hyphsubst}
\fi
\usepackage{babel}

\lstset{%
  inputencoding=utf8,extendedchars=true,
  literate=%
    {ä}{{\"a}}1 {ö}{{\"o}}1 {ü}{{\"u}}1
    {Ä}{{\"A}}1 {Ö}{{\"O}}1 {Ü}{{\"U}}1
    {ß}{{\ss}}1 {~}{{\textasciitilde}}1
    {»}{{\guillemetright}}1 {«}{{\guillemetleft}}1
}
\usepackage{widows-and-orphans}
\usepackage{bookmark}
\KOMAoptions{headings=optiontoheadandtoc}

\usepackage{blindtext}
\usepackage{lmodern}

\begin{document}

\begin{Entity}{\Bundle{tudscr}}

\begin{Declaration}
  {\Option{ddc=\PMisc}}
  (false)
\begin{Declaration}
  {\Option{ddchead=\PMisc}}
\begin{Declaration}
  {\Option{ddcfoot=\PMisc}}
\printdeclarationlist
Folgend sind die möglichen Werte für die Option \Option{ddc} aufgeführt, welche
aber gleichermaßen für \Option{ddchead} und \Option{ddcfoot} gelten:
\begin{DeclareValues}[\Option{ddc},\Option{ddchead},\Option{ddcfoot}]
\itemval=false=
  Bei den Seitenstile erscheint kein Logo von \DDC.
\itemval*=true=
  Das Logo von \DDC wird im Kopf beziehungsweise Fuß verwendet. Die Wahl der 
  Farbe des Logos geschieht passend zur farblichen Ausprägung der Seite selbst.
\end{DeclareValues}

Die Farbe des \DDC-Logos wird in Abhängigkeit der farblichen Ausprägung des 
Layouts automatisch gewählt. Dies lässt sich manuell ändern:
\begin{DeclareValues}[\Option{ddcfoot},\Option{ddc},\Option{ddchead}]
\itemval=color=
  Im Kopf oder Fuß wird die achtfarbige 4C"~Variante des \DDC-Logos genutzt.
\itemval=colorblack=
  Es wird das achtfarbige Logo mit schwarzem \DDC-Schriftzug anstelle des 
  grauen verwendet. Für den Fuß wird der grüne Claim durch einen schwarzen 
  ersetzt. Dies ist insbesondere für kleine Darstellungen des Logos im Fuß 
  sinnvoll.
\itemval=gray,grey=
  Dies Ausgabe des \DDC-Logos erfolgt in Graustufen.
\itemval=black=
  Verwendung des Logos in Graustufen mit schwarzem Schriftzug.
\itemval=blue=
  Schriftzug und Logo werden in Abstufungen der Hausfarbe gesetzt.
\itemval=white=
  Das \DDC-Logo sowie der dazugehörige Schriftzug sind vollständig weiß.
\end{DeclareValues}
\end{Declaration}
\end{Declaration}
\end{Declaration}




\begin{Declaration}
  {\Macro{maketitle}}
\begin{Declaration}
  {\Macro{maketitle/pagenumber=\PMisc}}
\begin{Declaration}
  {\Macro{maketitle/stuff=\PMisc}}
\printdeclarationlist
%
Diese Option dient zur Anpassung der Mathematikschriften. Wird diese aktiviert, 
so werden zur Hausschrift passende Glyphen im Mathematikmodus genutzt. Normale 
sowie fette Schriftstärke werden zu \emph{Beginn des Dokuments} abhängig von 
der zu diesem Zeitpunkt aktiven Einstellung für die Schriften des Fließtexts 
\begin{DeclareValues}%
%  [\Option{foo},\Option{bar},\Macro{maketitle/pagenumber=\PMisc}]
  [\Option{foo},\Option{bar}]
\itemval=blubb,false=
  Es werden die normalen \Logo{LaTeX}"=Serifenschriften beziehungsweise die 
  Schriften beliebig nutzbarer Pakete für den Mathematiksatz verwendet.
\itemval*=true=
  Im Mathematikmodus kommt \OpenSans sowohl für lateinische als auch 
  griechische Lettern zum Einsatz.
\itemval=\PLength=
  Wird ein Längenwert übergeben, entspricht dies 
\end{DeclareValues}
\end{Declaration}
\end{Declaration}
\end{Declaration}
\end{Entity}


\end{document}
