\PassOptionsToPackage{check=none}{widows-and-orphans}%
\documentclass[english,ngerman,ToDo=inline,final,indexnote=false]{tudscrmanual}
\GitHubBase{\TUDScriptRepository}
\iftutex
  \usepackage{fontspec}
\else
  \usepackage[T1]{fontenc}
  \input glyphtounicode.tex
  \pdfgentounicode=1
  \usepackage[ngerman=ngerman-x-latest]{hyphsubst}
\fi
\lstset{%
  inputencoding=utf8,extendedchars=true,
  literate=%
    {ä}{{\"a}}1 {ö}{{\"o}}1 {ü}{{\"u}}1
    {Ä}{{\"A}}1 {Ö}{{\"O}}1 {Ü}{{\"U}}1
    {ß}{{\ss}}1 {~}{{\textasciitilde}}1
    {»}{{\guillemetright}}1 {«}{{\guillemetleft}}1
}
\usepackage{widows-and-orphans}
\usepackage{bookmark}
\KOMAoptions{headings=optiontoheadandtoc}

\tracinglabels*<created,matched>[all]
\tracinglabels[all]
%\tracingmarkup%[all]
\usepackage{blindtext}

\begin{document}

%\begin{Entity}{\Bundle{tudscr}}
%\begin{Declaration}
%  {\Option{cd=\PMisc}}
%  (true)
%\printdeclarationlist
%Mit dieser Option wird festgelegt, ob und wie das \TUDCD im gesamten Dokument 
%verwendet wird. Sie hat Einfluss auf die Ausprägung von Titel"~, Teil"~, und 
%Kapitelseiten sowie die Überschriften der weiteren Gliederungsebenen. 
%Im Layout des \CDs wird auch bei \Option{cdfont=false} die Hausschrift in 
%Überschriften verwendet.
%\begin{DeclareValues}
%\itemval=barcolor=[v2.04:Layoutseiten verwenden für \PageStyle{tudheadings} 
%    solitären, farbig abgesetzten Querbalken]
%  Zusätzlich zur vorherigen Einstellung wird außerdem der Querbalken farbig 
%  abgesetzt.
%\end{DeclareValues}
%\index{Layout!Kopfzeile|)}%
%\end{Declaration}
%\end{Entity}

%\TeXLive

\DistributionGeneral

%\CrossIndex{terms}{Bezeichner,Sprachunterstützung}
%\SeeIndex{Sprachunterstützung}{Bezeichner}
%\SeeIndex{Lokalisierung}{Bezeichner,Sprachunterstützung}
%\SeeIndex{Silbentrennung,Sprachunterstützung,Trennungsmuster}{Worttrennung}


%\CrossIndex{elements}[Index der *]{Farben,Schriftelemente,Seitenstile}
%
%\SeeIndex{Cover}{Umschlagseite}

%\begin{Declaration}
%  {\Macro{TUDScript}}
%  [v2.04]
%\printdeclarationlist
%%
%Diese Anweisung setzt das Logo respektive die Wortmarke \enquote{\TUDScript{}} 
%in serifenloser Schrift und mit leichter Sperrung des in Versalien gesetzten 
%Teils. Dieser Befehl wird von allen Klassen und Paketen des \TUDScript-Bundles 
%definiert.
%\end{Declaration}

%\index{bla}
%
%\index{blubb}
%
%\index{foo|see{bla}}
%\index{foo|see{blubb}}
%
%\index{markup@markup (real)}
%
%\index{refmarkup|see{markup}}
%
%
%\index{refmarkup}
%
%\index{weird@stuff!really@shit!or@what|!}


%\TeXLive

%
%\DistributionGeneral

\section{Stub%
\label{idx:changes}%
\label{idx:macros}%
\label{idx:terms}%
\label{idx:files}%
\label{idx:elements}%
\label{idx:misc}%
\label{idx:options}%
}

\Application{Foo}

\index{bla|see{\Application{Foo}}}


%\DistributionGeneral aaa
%
%{\DistributionGeneral} bbb
%
%{\DistributionGeneral}ccc



%\begin{Declaration}
%  {\Macro(!){date|\OPList{Suffix}\MPName{Datum}}}
%  [v2.05:Angabe von Parametern für Prä- und Suffix bei Datumsausgabe möglich;]
%\begin{Declaration}
%  {\Macro(!){date/before=\PSet{Präfix}}}
%  [v2.05]
%\printdeclarationlist%
%Mit dem Befehl \Macro{date} lässt sich das Datum angegeben. 
%\ChangedAt{v2.05}%
%\end{Declaration}
%\end{Declaration}
%
%
%Über das optionale Argument können Parameter \Macro{date/before=\PSet{Präfix}} 


%\SeeIndex{Abstand}{Längen}
%\index{Längen}
%
%%\def\"a{ae}
%
%%\hypertarget{Blä}{Blä}
%%\hypertarget{Bla}{Bla}
%%und was kommt \hyperlink{Blä}{jetzt}?
%%und was kommt \hyperlink{Bla}{jetzt}?
%
\makeatletter
%
%\edef\foo{%
%  \noexpand\hypertarget{Blä}%
%}
%\meaning\foo
%
%\foo{Blä}
%
%
%\protected@edef\foo{%
%  \noexpand\hyperlink{Blä}%
%}
%\meaning\foo
%
%\foo{Blä ref}
%
%
%\let\hyperpage\@firstofone
\tud@index@mode@num=-2\relax%
%%\csuse{tud@index@set@@tudscr}%
%%\@input@{tudscr-tudscr.ind}%
\tud@index@print
%
\makeatother

\end{document}
