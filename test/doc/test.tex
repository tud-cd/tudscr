\PassOptionsToPackage{check=none}{widows-and-orphans}%
\documentclass[english,ngerman,ToDo=inline,final,indexnote=false]{tudscrmanual}
\GitHubBase{\TUDScriptRepository}
\iftutex
  \usepackage{fontspec}
\else
  \usepackage[T1]{fontenc}
  \input glyphtounicode.tex
  \pdfgentounicode=1
  \usepackage[ngerman=ngerman-x-latest]{hyphsubst}
\fi
\lstset{%
  inputencoding=utf8,extendedchars=true,
  literate=%
    {ä}{{\"a}}1 {ö}{{\"o}}1 {ü}{{\"u}}1
    {Ä}{{\"A}}1 {Ö}{{\"O}}1 {Ü}{{\"U}}1
    {ß}{{\ss}}1 {~}{{\textasciitilde}}1
    {»}{{\guillemetright}}1 {«}{{\guillemetleft}}1
}
\usepackage{widows-and-orphans}
\usepackage{bookmark}
\KOMAoptions{headings=optiontoheadandtoc}

\tracinglabels*<created,matched>[all]
\tracinglabels[all]
%\tracingmarkup%[all]
\usepackage{blindtext}

%\usepackage{showframe}

\begin{document}
\subsubsection{Der Satzspiegel im \TUDCD}
%
\begin{Declaration}
  {\Option{cdgeometry=\PMisc}}
  (true,restricted|\Option{cd=false}:false)
  [v2.03]
\printdeclarationlist[%
  \index{Satzspiegel!doppelseitig}%
]
Diese Option ist für die Wahl des Satzspiegels verantwortlich. Das Maß der 
Seitenränder ist im \CD fest vorgegeben und wird standardmäßig von den 
\TUDScript-Klassen eingehalten. Allerdings lassen sich die Seitenränder 
anpassen, um beispielsweise einen vernünftigen doppelseitigen Satz zu 
ermöglichen.%
\footnote{Hierbei sollte der innere Rand schmaler als der äußere sein}
Deshalb besteht die Möglichkeit, entweder auf das Standardverhalten von 
\KOMAScript zurückzufallen und den Satzspiegel durch das Paket 
\Package{typearea} zu berechnen oder aber diesen (fast) beliebig manuell 
festzulegen.
\begin{DeclareValues}
\NewDocumentCommand\marginformat{s m m m m}{%
  \unskip\hfill%
  \IfBooleanTF{#1}{innen}{links}:\,#2\,mm,\,%
  \IfBooleanTF{#1}{außen}{rechts}:\,#3\,mm,\,%
  oben:\,#4\,mm\,,unten:\,#5\,mm%
}%
\itemval=false=
  Die Satzspiegelberechnung erfolgt via \Package{typearea}, die Vorgaben des 
  \CDs bezüglich der Seitenränder werden ignoriert.
  \Attention{Nur in diesem Fall wirkt die Option \Option{DIV}.}
\itemval*=true,asymmetric,cd=<\marginformat{30}{20}{20}{30}>%
  Die Seitenränder werden im asymmetrischen Stil des \CDs fest definiert und 
  auch für den doppelseitigen Satz (Klassenoption \Option{twoside=true}) 
  genutzt.%
\itemval=symmetric,centred,centered=<\marginformat{25}{25}{20}{30}>%
  Der Satzspiegel wird im einseitigen sowie doppelseitigen Satz auf der Seite 
  zentriert.%
\itemval=twoside,balanced=<\marginformat*{20}{30}{20}{30}>%
  Diese Einstellung aktiviert den doppelseitigen Satz (\Option{twoside=true}) 
  und ändert den Satzspiegel derart, dass die Ränder der inneren Seiten 
  schmaler sind als die der äußeren.
  \Attention{%
    Der so erzeugte Satzspiegel ist jedoch unvorteilhaft, da das Logo der \TnUD 
    sehr nah am inneren Seitenrand des Dokumentes gesetzt wird und folglich auf 
    rechten respektive ungeraden Seiten sehr weit an den Blattrand rückt.
  }
  Dieses Problem kann~-- bei \Class{tudscrbook} sowie \Class{tudscrreprt}~-- 
  prinzipiell gelöst werden, indem Titel, Teile und Kapitel über das Aktivieren 
  der \KOMAScript-Option \Option{open=left} immer auf einer linken Seite 
  beginnen, was allerdings aus typografischer Sicht eher unüblich ist.
\end{DeclareValues}
\end{Declaration}

\end{document}
