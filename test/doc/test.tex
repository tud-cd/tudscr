\PassOptionsToPackage{check=none}{widows-and-orphans}%
\documentclass[english,ngerman,ttfont=roboto,ToDo=inline,final]{tudscrmanual}
\GitHubBase{\TUDScriptRepository}
\iftutex
  \usepackage{fontspec}
\else
  \usepackage[T1]{fontenc}
  \input glyphtounicode.tex
  \pdfgentounicode=1
  \usepackage[ngerman=ngerman-x-latest]{hyphsubst}
\fi
\lstset{%
  inputencoding=utf8,extendedchars=true,
  literate=%
    {ä}{{\"a}}1 {ö}{{\"o}}1 {ü}{{\"u}}1
    {Ä}{{\"A}}1 {Ö}{{\"O}}1 {Ü}{{\"U}}1
    {ß}{{\ss}}1 {~}{{\textasciitilde}}1
    {»}{{\guillemetright}}1 {«}{{\guillemetleft}}1
}
\usepackage{widows-and-orphans}
\usepackage{bookmark}
\KOMAoptions{headings=optiontoheadandtoc}

%\tracinglabels*<created,matched>[all]
\tracinglabels[all]
%\tracingmarkup%[all]
\usepackage{blindtext}


%\DefaultEntity{\Class{datetime2}}{
%  \DefaultEntity{\Class{daatetime2}}{
%    \Macro{DTMDate},
%  }
%}

%\DefaultEntity{\Bundle{enumitem}}{
%  \DefaultEntity{\Package{enumitem}}{
%    \DefaultEntity{\Environment{itemize}}{\Macro{item}},
%    \DefaultEntity{\Environment{enumerate}}{\Macro{item}},
%    \DefaultEntity{\Environment{description}}{\Macro{item}},
%    \Macro{setlist},
%  }
%}

\begin{document}

\makeatletter

%  
%\unexpanded{\Entity@Label@Seek\foo}\expandafter\\expandafter\@firstoftwo\tud@res@a
%}%

\InlineDeclaration{\Bundle{enumitem|}}

\InlineDeclaration{\Package{enumitem|}}



\Macro{setlist}


%\typeout{=====================================}
\Entity@Link@Seek\foo{Macro}{setlist}
\meaning\foo
%
%\Package{sansmath}
%
%\Format{pdfLaTeX}
%
%\Bundle{koma-script}
%
%\File{tudscr4texstudio.zip}

\makeatletter
%
\begin{Entity}{\Bundle{something}}
\begin{Declaration}
  {\Option{cdfont=\PMisc}}
  (true,normalbold,liningfigures)
\printdeclarationlist
%
\meaning\tud@declaration@list

\meaning\tud@declaration@label@list

Mit der Option \Option{cdfont} können durch den Anwender alle zentralen 
Schrifteinstellungen für die \TUDScript-Klassen vorgenommen werden. Dies 
betrifft die Schriften sowohl für den Fließtext als auch den Mathematiksatz.
Dabei lassen sich insbesondere unterschiedliche Kombinationen von normaler und 
fetter Schriftstärke für die \OpenSans einstellen. Zu Beginn des Dokumentes 
sind diese maßgeblich für die Definition der Mathematikschriften. Die 
Schriftstärke im charakteristischen Querbalken der Kopfzeile lässt sich mit 
dieser Option ebenfalls einstellen.
%
\begin{DeclareValues}
\itemval=false=
\meaning\tud@declaration@list

\meaning\tud@declaration@label@list

  Es werden die \hologo{LaTeX}"=Standardschriften und nicht die Hausschrift 
  des \CDs verwendet. Der Anwender kann beliebige Schriftpakete nutzen.%
  \footnote{%
    Für die Verwendung der klassischen \hologo{LaTeX}"~Schriften, ist das Paket 
    \Package{lmodern} sehr empfehlenswert.%
  }
\itemval*=true,light,lightfont=
  Die Hausschrift \OpenSans des \CDs wird genutzt. Die Überschriften der 
  obersten Gliederungsebenen bis einschließlich \Macro||{subsubsection} 
  werden in (normal"~)fetter oder extra"~fetter Schriftstärke 
  (\seeplain{\Option'page'{headings=\PMisc}}) gesetzt, darunter liegende Ebenen%
  \footnote{\Macro{paragraph}, \Macro{subparagraph}} 
  verwenden immer den (normal"~)fetten Schriftschnitt. Standardmäßig nutzt 
  dieser \textcdrn{Open~Sans~Regular} und kann mit \Option{cdfont=heavybold} 
  stärker eingestellt werden. Im Fließtext kommt \textcdln{Open~Sans~Light} zum 
  Einsatz.
\itemval=heavy,heavyfont=
  Diese Einstellung unterscheidet sich von \Option{cdfont=true} insoweit als 
  die Schriftstärke der Hausschrift erhöht wird. Der Fließtext des Dokumentes 
  wird in \textcdrn{Open~Sans~Regular} gesetzt. Der fette Schriftschnitt ist im 
  Normalfall auf \textcdsn{Open~Sans~Semi"~Bold} festgelegt.
\end{DeclareValues}
%
Die Stärke der fetten Schriften lässt sich mit folgenden Einstellungen anpassen.
%
\begin{DeclareValues}
\itemval=normalbold=[v2.05]
  Für die fette Schriftstärke wird \textcdrn{Open~Sans~Regular} respektive bei 
  stärkerer Grundschrift (\Option{cdfont=heavy}) \textcdsn{Open~Sans~Semi"~Bold}
  verwendet. Dies ist die Voreinstellung.
\itemval=heavybold,ultrabold=[v2.05]
  Die fetten Schriften werden stärker hervorgehoben. Es kommt 
  \textcdsn{Open~Sans~Semi"~Bold} respektive \textcdbn{Open~Sans~Bold} bei 
  erhöhter Schriftstärke (\Option{cdfont=heavy}) zum Einsatz.
\end{DeclareValues}
\end{Declaration}
\end{Entity}
%
%\Macro(\Bundle{something},\Bundle{stuff}){TUDoptions}
%
%
%\Macro{TUDoptions}
%
%
%\Macro{TUDoption}
%
%\Environment{enumerate}

%\begin{Entity}{\Bundle{tudscr}}
%\begin{Declaration}{\Environment{itemize}}
%\printdeclarationlist
%Mit diesen beiden Befehlen besteht...
%\end{Declaration}
%\end{Entity}

%\begin{Declaration}{\Environment(*){itemize}}
%\printdeclarationlist
%Mit diesen beiden Befehlen besteht...
%\end{Declaration}
%
%\Environment'full'{itemize}
%
%\Environment'CTAN:full'{itemize}

%\Environment{description}

%\Package'GH:bla'{datetimea2}

%\Macro'auto'{setlist}

%\InlineDeclaration{\Class(\Bundle{foo}){bla}}

%\Option{testoption}

%\Macro{minisec}

%\Package{pkg}

%\Distribution|?|{\hologo{TeX}~Live}, \Distribution|?|{Mac\hologo{TeX}} oder 
%\Distribution|?|{\hologo{MiKTeX}} sollte dies kein Problem darstellen. Wird 


%\PrintIndex



%\Macro{author|\MPName{Autor(en)}}

\end{document}

\ChangedAt{v2.02;v2.06}%
Durch das \CD werden keine Schriften für den Mathematiksatz festgelegt. Das ist 
insbesondere für mathematische Abhandlungen als auch ingenieur- und 
naturwissenschaftliche Dokumente nicht tragbar. Im Mathematikmodus werden 
deshalb die lateinischen Lettern mithilfe des Paketes \Package{mathastext}
sowie die griechischen Lettern der \OpenSans genutzt. Zur Ergänzung kann für 
weitere mathematische Symbole das Paket \Package{mdsymbol} geladen werden.
\ToDo{hinweis zu mdsymbol raus}[v2.07]

Diese Einstellung lässt sich deaktivieren, wodurch sich die Standardschriften 
oder gegebenenfalls die eines zusätzlichen Paketes für den mathematischen Satz 
nutzen lassen. Die dafür relevanten Einstellungen werden in \autoref{sec:math} 
erläutert. Weiterhin sind ergänzende Hinweise zu einem typografisch sauberen
Mathematiksatz in \autoref{sec:tut} zu finden.

\section{title}

\Macro{KOMAClassName}

\Macro{caption}

\end{document}

\begin{Declaration}
  {\Macro{foo}}
\printdeclarationlist
%
Mit dieser Option kann die Größe des unteren Seitenrandes angepasst werden, 
\end{Declaration}

\begin{DeclareEntity}{\Class{tudscrtest}}
\begin{Declaration}
  {\Macro{extrabottommargin}}
\printdeclarationlist
%
Mit dieser Option kann die Größe des unteren Seitenrandes angepasst werden, 
\end{Declaration}

\Macro'full'{extrabottommargin}

\Macro'full'{extrabottommargin/stuff}

\Macro'full'(\Class{tudscrtest},\Class{tudscrposter}){extrabottommargin/stuff}

\Macro'full'(\Class{tudscrtest}){extrabottommargin/stuff}

\Macro(\Class{tudscrposter}){extrabottommargin}

\end{DeclareEntity}

\begin{DeclareEntity}{\Class{tudscrposter}}
\begin{Declaration}
  {\Macro{extrabottommargin}}
\begin{Declaration}
  {\Macro{extrabottommargin/stuff}}
\printdeclarationlist
%
Mit dieser Option kann die Größe des unteren Seitenrandes angepasst werden, 
\end{Declaration}
\end{Declaration}

\Macro{extrabottommargin}

\Macro{extrabottommargin/stuff}

\Macro(\Class{tudscrposter},\Class{tudscrtest}){extrabottommargin}

\end{DeclareEntity}

\begin{DeclareEntity}{\Class{tudscrclass}}
\begin{Declaration}
  {\Macro{extrabottommargin}}
\begin{Declaration}
  {\Macro{extrabottommargin/stuff}}
\printdeclarationlist
%
Mit dieser Option kann die Größe des unteren Seitenrandes angepasst werden, 
\end{Declaration}
\end{Declaration}

\Macro{extrabottommargin}

\Macro{extrabottommargin/stuff}

\Macro(\Class{tudscrposter},\Class{tudscrtest}){extrabottommargin}

\end{DeclareEntity}

---------------------------------------

\Macro(\Bundle{tudscr}){extrabottommargin}

\Macro(\Bundle{tudscr},\Class{tudscrposter}){extrabottommargin}

\Macro(\Bundle{tudscr},\Class{tudscrposter},\Class{tudscrtest}){extrabottommargin}

\Macro{extrabottommargin/stuff}

\Macro{foo}

\Environment{itemize}

%\Macro(\Bundle{bar}){bar}
%
%\Macro(\Bundle{bar},\Bundle{blubb}){bar}

\Macro(\Class{tudscrposter},\Class{tudscrtest}){extrabottommargin}

\end{document}
