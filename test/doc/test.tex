\documentclass[english,ngerman,ttfont=roboto]{tudscrmanual}
\ifpdftex{
  \usepackage[T1]{fontenc}
  \input glyphtounicode
  \pdfgentounicode=1
  \usepackage[ngerman=ngerman-x-latest]{hyphsubst}
}{
  \usepackage{fontspec}
}
\lstset{%
  inputencoding=utf8,extendedchars=true,
  literate=%
    {ä}{{\"a}}1 {ö}{{\"o}}1 {ü}{{\"u}}1
    {Ä}{{\"A}}1 {Ö}{{\"O}}1 {Ü}{{\"U}}1
    {~}{{\textasciitilde}}1 {ß}{{\ss}}1
}

\usepackage{widows-and-orphans}

\usepackage{bookmark}
\KOMAoptions{headings=optiontoheadandtoc}

\tracinglabels[all]
%\tracingmarkup
%\tracingbundle
\usepackage{blindtext}

\begin{document}
\ChangedAt{%
  v2.02;%
  v2.04:Einfachere Verwendung des Paketes \Package{fontspec};%
  v2.06:\OpenSans als neue Schrift im \TUDCD;%
  v2.06f:Glyphen im Mathematiksatz werden passend zu den mit \Option{cdfont}
    gewählten normalen und fetten Schriftstärken gesetzt;%
}

bla
%\begin{Declaration}[%
%%  v2.03!\Option{cdfoot=\PValueName{Höhe}};%
%%  v2.03=\PValue{\PLength};%
%]{\Option{cdfoot=\PMisc}}[false]%
%\printdeclarationlist%
%%
%\begin{values}{\Option{cdfoot}}
%%\item[\PLength]
%%  Wird der Option ein Längenwert übergeben, entspricht dies exakt der 
%\item[\PName{bla:foo}]
%  Test
%\end{values}
%\end{Declaration}
%
%\Distribution{\hologo{MiKTeX}}|?|


\clearpage
%\PrintIndex
\PrintChangelog

\end{document}
