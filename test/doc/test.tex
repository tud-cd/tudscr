\documentclass[english,ngerman,ttfont=roboto]{tudscrmanual}
\ifpdftex{
  \usepackage[T1]{fontenc}
  \input glyphtounicode
  \pdfgentounicode=1
  \usepackage[ngerman=ngerman-x-latest]{hyphsubst}
}{
  \usepackage{fontspec}
}
\lstset{%
  inputencoding=utf8,extendedchars=true,
  literate=%
    {ä}{{\"a}}1 {ö}{{\"o}}1 {ü}{{\"u}}1
    {Ä}{{\"A}}1 {Ö}{{\"O}}1 {Ü}{{\"U}}1
    {~}{{\textasciitilde}}1 {ß}{{\ss}}1
}

\usepackage{bookmark}
\KOMAoptions{headings=optiontoheadandtoc}

\tracinglabels[all]
%\tracingmarkup
%\tracingbundle
\usepackage{blindtext}

\begin{document}
\newrobustcmd*\cdurl{%
  \begingroup%
    \hypersetup{hidelinks}%
    \href{https://tu-dresden.de/cd}{https://tu-dresden.de/cd}%
  \endgroup%
}
\faculty{\cdurl}
\subject{\TUDScript \vTUDScript{} basierend auf \KOMAScript}

\title{Ein {LaTeX}-Bundle für Dokumente im \TUDCD}
\ifdef{\tudprintflag}{%
  \subtitle{Benutzerhandbuch\thanks{\href{tudscr}{Online-Version}}}%
}{%
  \subtitle{Benutzerhandbuch\thanks{\href{tudscr_print}{Druckversion}}}%
}

\author{Falk Hanisch\TUDScriptContactNote}
\publishers{\GitHubRepo'{tudscr}[]}
\date{27.08.2019}


\makeatletter
\begingroup%
  \def\and{, }%
  \let\thanks\@gobble%
  \let\footnote\@gobble%
  \let\emailaddress\@gobble%
  \hypersetup{%
    pdfauthor = {\@author},%
    pdftitle = {\@title},%
    pdfsubject = {Benutzerhandbuch für \TUDScript},%
    pdfkeywords = {LaTeX, \TUDScript, Benutzerhandbuch},%
  }%
\endgroup%
\makeatother

\makeatletter
%\newcommand*\Markup@Get@Name[2]{%
%  \begingroup%
%    \expandafter\edef\expandafter\tud@res@a\expandafter{%
%      \expandafter\noexpand\@firstoftwo#2\@empty%
%    }%
%    \edef\tud@res@a{%
%      
%\expandafter\expandafter\expandafter\@gobble\expandafter\string\tud@res@a%
%    }%
%    \edef\tud@res@a{%
%      \endgroup%
%      \def\noexpand#1{\tud@res@a}%
%    }%
%  \tud@res@a%
%}
%\newcommand*\SetInlineFormat[2]{%
%  \Markup@Get@Name\tud@res@a{#1}%
%  \expandafter\newcommand\csname Output@Inline@\tud@res@a\endcsname{}%
%  \@namedef{Output@Inline@\tud@res@a}##1+##2+=##3=[##4]{#2}%
%}
%\SetInlineFormat\Option{1: #1 2: #2 3: #3 4: #4}%

%\NewDocumentCommand\TestMarkup{m d== o}{
%  \begingroup
%  \Markup@Inline@@Environment{#1}=#2=[#3]%
%  \par\Markup@Declare@@Environment{#1}=#2=[#3]%
%  \endgroup
%}
%
%\TestMarkup{env}[\Parameter{gucke}]
%
%\TestMarkup{env}

\Key*{\Macro{makecover}}{cdgeometry=false}

Das ist ein Paket \Length{testpaket} um etwas zu zeufen

\begin{Declaration}{\Color{HKS41}[cddarkblue]}
\printdeclarationlist%
bla
\end{Declaration}


\makeatother



\Key{\Macro{testmacro}}{language=\PName{Sprache}}%(\Package{testpackage})

\Key{\Environment{testenv}}{language=\PName{Sprache}}%(\Package{testpackage})

\Option{foo}=\PName{bar}=

\begin{Declaration}[v2.07]{\Macro{testmacro}}
\begin{Declaration}[v2.07]{\Macro{testmacro}+language=\PName{Sprache}+}
\printdeclarationlist
bbbbbbbbbbbbbbbbb
\end{Declaration}
\end{Declaration}

\begin{Declaration}[v2.07]{\Environment{testenv}[\LParameter]}
\begin{Declaration}[v2.07]{\Environment{testenv}+language=\PName{Sprache}+}
\printdeclarationlist
bbbbbbbbbbbbbbbbb
\end{Declaration}
\end{Declaration}

\bigskip

\Macro{testmacro}

\Macro{testmacro}+language=\PName{Sprache}+

\bigskip

\Environment{testenvA}

\Environment{testenvB}+language=\PName{Sprache}+


%\NewMacro{testmacroC}[\Parameter{gucke}]!language=\PName{Sprache}!
%
%\NewMacro{testmacroD}[\OParameter{gucke}]!language=\PName{Sprache}!
%
%\NewMacro{testmacroE}[\POParameter{gucke}]!language=\PName{Sprache}!

%\NewMacro{testmacroX}[\OParameter{gucke}]!language=\PName{Sprache}!(\Package{testpackage})


\clearpage
%\PrintIndex

\end{document}
