%\documentclass{tudscrartcl}
%\usepackage[T1]{fontenc}
%\usepackage{hologo}
%\begin{document}
%
%\hologo{LaTeX}
%
%\textbf{\hologo{LaTeX}}
%
%\textit{\hologo{LaTeX}}
%
%\textsf{\hologo{LaTeX}}
%
%\textsl{\hologo{LaTeX}}
%
%\meaning\hologo
%
%
%\end{document}
\PassOptionsToPackage{check=none}{widows-and-orphans}%
\documentclass[english,ngerman,ToDo=inline,final,indexnote=false]{tudscrmanual}
\GitHubBase{\TUDScriptRepository}
\iftutex
  \usepackage{fontspec}
\else
  \usepackage[T1]{fontenc}
  \input glyphtounicode.tex
  \pdfgentounicode=1
  \usepackage[ngerman=ngerman-x-latest]{hyphsubst}
\fi
\lstset{%
  inputencoding=utf8,extendedchars=true,
  literate=%
    {ä}{{\"a}}1 {ö}{{\"o}}1 {ü}{{\"u}}1
    {Ä}{{\"A}}1 {Ö}{{\"O}}1 {Ü}{{\"U}}1
    {ß}{{\ss}}1 {~}{{\textasciitilde}}1
    {»}{{\guillemetright}}1 {«}{{\guillemetleft}}1
}
\usepackage{widows-and-orphans}
\usepackage{bookmark}
\KOMAoptions{headings=optiontoheadandtoc}

\tracinglabels*<created,matched>[all]
\tracinglabels[all]
%\tracingmarkup%[all]
\usepackage{blindtext}

\usepackage{lmodern}
\begin{document}

%\TUDoptions{cdfont=false}

\makeatletter
\newlength\tud@logo@w
\setlength\tud@logo@w{0.9em}
% PENDING raus?
\newlength\tud@logo@h
\setlength\tud@logo@h{1ex}
\newcommand*\tud@logo@ic[1]{\ifdim\the\fontdimen\@ne\font>\z@\relax#1\fi}
\newrobustcmd*\sanslogo[1]{%
  \ifcsundef{tud@logo@#1}{%
    \csname#1\endcsname%
    \enskip\hologo{#1}%
    \ClassWarning{tudscrmanual}{warning text}%
  }{%
    \mbox{\csuse{tud@logo@#1}}%
    \enskip\hologo{#1}%
  }%
}
\newcommand*\tud@logo@scale[2]{%
  \begingroup%
    \setbox\z@\hbox{#1}%
    \vbox to\ht\z@{%
      \setbox\@ne\hbox{%
        \check@mathfonts%
        \fontsize\sf@size\z@%
        \math@fontsfalse\selectfont%
        #2%
      }%
      \hbox{%
        \copy\@ne\relax%
        \kern-.995\wd\@ne%
        \copy\@ne\relax%
      }%
      \vss%
    }%
  \endgroup%
}
\newcommand*\tud@logo@La{%
  L%
  \kern-.36\tud@logo@w\tud@logo@ic{\kern.1\tud@logo@w}%
  \tud@logo@scale{T}{A}%
}
\newcommand*\tud@logo@Xe{%
  X\kern-.05\tud@logo@w\tud@logo@ic{\kern-.015\tud@logo@w}%
 \lower.5\tud@logo@h\hbox{\rotatebox[origin=c]{180}{E}}%
}
\newcommand*\tud@logo@TeX{%
  T%
  \kern-.23\tud@logo@w\tud@logo@ic{\kern-.05\tud@logo@w}%
  \lower.5\tud@logo@h\hbox{E}%
  \kern-.075\tud@logo@w\tud@logo@ic{\kern.1\tud@logo@w}%
  X\@%
}
\newcommand*\tud@logo@LaTeX{%
  \tud@logo@La\kern-.15\tud@logo@w\tud@logo@TeX%
}

\newcommand*\tud@logo@LaTeXe{%
  \tud@logo@LaTeX\kern.1\tud@logo@w%
  \hbox{%
    \m@th%
    \if b\expandafter\@car\f@series x\@nil\csname boldmath\endcsname\fi%
    2$_{\textstyle\varepsilon}$%
  }\@%
}
\expandafter\newcommand\expandafter*\csname tud@logo@LaTeX3\endcsname{%
  \tud@logo@LaTeX\kern.1\tud@logo@w3\@%
}
\newcommand*\tud@logo@pdfLaTeX{%
  \tud@logo@scale{x}{PDF}\tud@logo@LaTeX%
}
\newcommand*\tud@logo@LuaLaTeX{%
  L\tud@logo@scale{x}{UA}\tud@logo@LaTeX%
}
\newcommand*\tud@logo@XeLaTeX{%
  \tud@logo@Xe%
  \kern-.15\tud@logo@w\tud@logo@ic{\kern.015\tud@logo@w}%
  \tud@logo@LaTeX%
}
\newcommand*\tud@logo@BibTeX{%
  B\tud@logo@scale{x}{IB}\kern-.1\tud@logo@w\tud@logo@TeX%
}
\newcommand*\tud@logo@MiKTeX{MiK\kern.05\tud@logo@w\tud@logo@TeX}
\newcommand*\tud@logo@LyX{%
  L\kern-.12\tud@logo@w\tud@logo@ic{\kern-.03\tud@logo@w}%
  \lower.5\tud@logo@h\hbox{Y}%
  \kern-.12\tud@logo@w\tud@logo@ic{\kern.07\tud@logo@w}%
  X\@%
}

\def\do#1{%
\clearpage
\par\noindent\sanslogo{#1}
\par\noindent\textsf{\sanslogo{#1}}
\par\noindent\textit{\sanslogo{#1}}
\par\noindent\textsl{\sanslogo{#1}}
\par\noindent\textbf{\sanslogo{#1}}
\par\noindent\textbf{\textsl{\sanslogo{#1}}}
}

\docsvlist{%
LyX,
MiKTeX,
BibTeX,
XeLaTeX,
LuaLaTeX,
pdfLaTeX,
LaTeX3,
LaTeXe,
LaTeX,
TeX,
}

\end{document}

\begin{Declaration}
  {\Environment{abstract|\OPList{Sprache}}}
\begin{Declaration}
  {\Environment{abstract/markboth=\PMisc}}
\printdeclarationlist
Zur Festlegung der Kolumnentitel existieren für 
\Environment{abstract/markboth} 
folgende Einstellungen:%
\begin{DeclareValues}[\Environment{abstract/markboth}]
\itemval=false=
  Die aktuellen Kolumnentitel~-- je nach gewählter Option für das Paket 
  \Package{scrlayer-scrpage} automatische (\Option{automark}) oder manuelle 
  (\Option{manualmark})~-- werden verwendet.
\itemval*=true=
  Die Kolumnentitel für linke und rechte Seiten werden auf \Term{abstractname} 
  gesetzt.
\itemval=\PName{Kolumnentitel}=
  Die Kolumnentitel werden manuell festgelegt. So lassen sich die Kolumnen 
  beispielsweise mit \Environment{abstract/markboth=\MPValue{}} auch 
  vollständig löschen. 
\end{DeclareValues}
\end{Declaration}
\end{Declaration}

\end{document}
