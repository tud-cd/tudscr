\PassOptionsToPackage{log-declarations=true}{xparse}
\documentclass[english,ngerman,ttfont=roboto]{tudscrmanual}
\ifpdftex{
  \usepackage[T1]{fontenc}
  \input glyphtounicode
  \pdfgentounicode=1
  \usepackage[ngerman=ngerman-x-latest]{hyphsubst}
}{
  \usepackage{fontspec}
}
\lstset{%
  inputencoding=utf8,extendedchars=true,
  literate=%
    {ä}{{\"a}}1 {ö}{{\"o}}1 {ü}{{\"u}}1
    {Ä}{{\"A}}1 {Ö}{{\"O}}1 {Ü}{{\"U}}1
    {~}{{\textasciitilde}}1 {ß}{{\ss}}1
}

\usepackage{widows-and-orphans}

\usepackage{bookmark}
\KOMAoptions{headings=optiontoheadandtoc}

\tracinglabels[all]
%\tracingmarkup
%\tracingbundle
\usepackage{blindtext}
\begin{document}
Das ist tet außerhalb\\
%Das ist \raisebox{2\baselineskip}[0pt][0pt]{\fbox{text}} außerhalb 

\end{document}

\begin{Declaration}{\Option{cdfoot=\PMisc}}[false]%
\printdeclarationlist%
\begin{DeclarationValues}
\itemval=\PLength=[v2.03]
%\itemval=\PValueName{Höhe}=[v2.03]
  Wird der Option ein Längenwert übergeben, entspricht dies exakt der 
\end{DeclarationValues}
\end{Declaration}

\end{document}

\begin{Bundle*}{\Class{tudscrposter}}
\begin{Declaration}[v2.01]{\Term{contactname}}
\printdeclarationlist%
%
Wie bereits zuvor erläutert, werden diese Bezeichner in der linken respektive 
rechten Spalte im Fuß vor der Ausgabe der eigentlichen Felder gesetzt.
\end{Declaration}
\end{Bundle*}
%
%\begin{Declaration}[v2.03]{\Term{contactname}}
%\printdeclarationlist%
%%
%Wie bereits zuvor erläutert, werden diese Bezeichner in der linken respektive 
%rechten Spalte im Fuß vor der Ausgabe der eigentlichen Felder gesetzt.
%\end{Declaration}
%
%\begin{Bundle}{\Package{tudscrsupervisor}}
%\begin{Obsolete}{v2.05}{\Term{contactname}}'\Term{contactpersonname}'
%\printdeclarationlist%
%%
%Wie bereits zuvor erläutert, werden diese Bezeichner in der linken respektive 
%rechten Spalte im Fuß vor der Ausgabe der eigentlichen Felder gesetzt.
%\end{Obsolete}
%\end{Bundle}
%
%
%\begin{Declaration'}{\Environment{tudpage}[\LParameter]}
%\begin{Obsolete}{v2.02}{\Key{\Environment{tudpage}}{head=\PMisc}}%
%  '\Key{\Environment{tudpage}}{pagestyle}'
%\begin{Obsolete}{v2.02}{\Key{\Environment{tudpage}}{foot=\PMisc}}%
%  '\Key{\Environment{tudpage}}{pagestyle}'
%\printdeclarationlist%
%%
%Diese beiden Parameter der Umgebung \Environment*{tudpage} wurden in ihrer 
%Funktionalität durch den Parameter \Key*{\Environment{tudpage}}{pagestyle} 
%ersetzt.
%\end{Obsolete}
%\end{Obsolete}
%\end{Declaration'}
%
%\end{document}

%\begin{Declaration}[%
%  v2.03!\Option{cd=bicolor}:%
%    Farbeinsatz nur im Kopf mit farbig abgesetztem Querbalken;
%  v2.03!\Option{cd=fullcolor}:%
%    voller Farbeinsatz mit farbig abgesetztem Querbalken;%
%  v2.04!\Option{cd=barcolor}:nur farbig abgesetzter Querbalken;%
%  v2.06f:Überschriften verwenden auch bei \Option{cdfont=false} die 
%Hausschrift,
%  falls das Layout des \CDs nicht mit \Option{cd=false} deaktiviert wurde%
%]{\Option{cd=\PMisc}}[true]
%\begin{Declaration}{\Option{cdvery=\PMisc}}[true]<aaa>
%\printdeclarationlist%
%%
%Mit dieser Option wird festgelegt, ob und wie das \TUDCD im gesamten Dokument 
%verwendet wird. Sie hat Einfluss auf die Ausprägung von Titel"~, Teil"~, und 
%Kapitelseiten sowie die Überschriften der weiteren Gliederungsebenen. Im 
%Layout 
%des \CDs wird auch bei \Option{cdfont=false} die Hausschrift in Überschriften 
%verwendet.
%%
%\begin{values}{\Option{cd}}
%\itemfalse
%  Es wird kein \CD sondern die Gestalt der \KOMAScript-Klassen genutzt.
%\itemtrue*[nocolor/monochrome]
%  Das Layout für Titel"~, Teil- und Kapitelseiten ist im \CD, es wird 
%  schwarze Schrift für Titel sowie Teil- und Kapitelüberschriften verwendet.
%  Die Ausprägung des Seitenkopfes ist abhängig von der Option \Option{cdhead}.
%\item[lightcolor/pale]
%  Die Einstellung entspricht weitestgehend der Option \Option{cd=true}, 
%  allerdings wird die primäre Hausfarbe \Color{HKS41} für den Kopf des 
%  \PageStyle{tudheadings}"=Seitenstils genutzt.
%\item[barcolor]
%  \ChangedAt{v2.04}
%  Zusätzlich zur vorherigen Einstellung wird außerdem der Querbalken farbig 
%  abgesetzt.
%\item[bicolor/bichrome]
%  \ChangedAt{v2.03}
%  Der Kopf wird mit einem farbigen Hintergrund in der Hausfarbe gesetzt, auch 
%  der Querbalken wird farbig hinterlegt. Für die Überschriften wird die 
%  primären Hausfarbe verwendet.
%\item[color]
%  Der Titel sowie Teil- und Kapitelseiten werden allesamt farbig gestaltet, 
%  der Seitenkopf wird in der primären Hausfarbe \Color{HKS41} gesetzt, der 
%  Querbalken erhält Linien als Begrenzung.
%\item[fullcolor/full]
%  \ChangedAt{v2.03}
%  Entspricht der vorherigen Einstellung, allerdings wird der Querbalken nicht 
%  durch Linien begrenzt sondern farbig hinterlegt.
%\end{values}
%\end{Declaration}
%\end{Declaration}

\clearpage

\begin{Declaration}[%
  v2.06f:Überschriften verwenden auch bei \Option{cdfont=false} die Hausschrift,
  falls das Layout des \CDs nicht mit \Option{cd=false} deaktiviert wurde%
]{\Option{cdnew=\PMisc}}[true]<aaa>
\begin{Declaration}{\Option{cdverynew=\PMisc}}[true]<aaa>
\printdeclarationlist%
%
Mit dieser Option wird festgelegt, ob und wie das \TUDCD im gesamten Dokument 
verwendet wird. Sie hat Einfluss auf die Ausprägung von Titel"~, Teil"~, und 
Kapitelseiten sowie die Überschriften der weiteren Gliederungsebenen. Im Layout 
des \CDs wird auch bei \Option{cdfont=false} die Hausschrift in Überschriften 
verwendet.
%
\begin{DeclarationValues}%
\itemval[v3.00]=false=
  Es wird kein \CD sondern die Gestalt der \KOMAScript-Klassen genutzt.
%\itemvaltrue*=nocolor/monochrome=
%  Das Layout für Titel"~, Teil- und Kapitelseiten ist im \CD, es wird 
%  schwarze Schrift für Titel sowie Teil- und Kapitelüberschriften verwendet.
%  Die Ausprägung des Seitenkopfes ist abhängig von der Option \Option{cdhead}.
%\itemval=lightcolor/pale=
%  Die Einstellung entspricht weitestgehend der Option \Option{cdnew=true}, 
%  allerdings wird die primäre Hausfarbe \Color{HKS41} für den Kopf des 
%  \PageStyle{tudheadings}"=Seitenstils genutzt.
%\itemval[%
%  v2.04:nur farbig abgesetzter Querbalken%
%]=barcolor=
%  Zusätzlich zur vorherigen Einstellung wird außerdem der Querbalken farbig 
%  abgesetzt.
%\itemval[%
%  v2.03:Farbeinsatz nur im Kopf mit farbig abgesetztem Querbalken%
%]=bicolor/bichrome=
%  Der Kopf wird mit einem farbigen Hintergrund in der Hausfarbe gesetzt, auch 
%  der Querbalken wird farbig hinterlegt. Für die Überschriften wird die 
%  primären Hausfarbe verwendet.
%\itemval=color=
%  Der Titel sowie Teil- und Kapitelseiten werden allesamt farbig gestaltet, 
%  der Seitenkopf wird in der primären Hausfarbe \Color{HKS41} gesetzt, der 
%  Querbalken erhält Linien als Begrenzung.
%\itemval[%
%  v2.03:voller Farbeinsatz mit farbig abgesetztem Querbalken
%]=fullcolor/full=
%  Entspricht der vorherigen Einstellung, allerdings wird der Querbalken nicht 
%  durch Linien begrenzt sondern farbig hinterlegt.
\end{DeclarationValues}
\end{Declaration}
\end{Declaration}

%\clearpage
%
%\Option{cdnew=fullcolor}
%
%\Option{cdverynew=fullcolor}

\clearpage
%\PrintIndex
\PrintChangelog

\end{document}
