\documentclass[english,ngerman,ttfont=roboto]{tudscrmanual}
\ifpdftex{
  \usepackage[T1]{fontenc}
  \input glyphtounicode
  \pdfgentounicode=1
  \usepackage[ngerman=ngerman-x-latest]{hyphsubst}
}{
  \usepackage{fontspec}
}
\lstset{%
  inputencoding=utf8,extendedchars=true,
  literate=%
    {ä}{{\"a}}1 {ö}{{\"o}}1 {ü}{{\"u}}1
    {Ä}{{\"A}}1 {Ö}{{\"O}}1 {Ü}{{\"U}}1
    {~}{{\textasciitilde}}1 {ß}{{\ss}}1
}

\usepackage{bookmark}
\KOMAoptions{headings=optiontoheadandtoc}

\tracinglabels[all]
%\tracingmarkup
%\tracingbundle
\usepackage{blindtext}

\begin{document}
\newrobustcmd*\cdurl{%
  \begingroup%
    \hypersetup{hidelinks}%
    \href{https://tu-dresden.de/cd}{https://tu-dresden.de/cd}%
  \endgroup%
}
\faculty{\cdurl}
\subject{\TUDScript \vTUDScript{} basierend auf \KOMAScript}

\title{Ein {LaTeX}-Bundle für Dokumente im \TUDCD}
\ifdef{\tudprintflag}{%
  \subtitle{Benutzerhandbuch\thanks{\href{tudscr}{Online-Version}}}%
}{%
  \subtitle{Benutzerhandbuch\thanks{\href{tudscr_print}{Druckversion}}}%
}

\author{Falk Hanisch\TUDScriptContactNote}
\publishers{\GitHubRepo'{tudscr}[]}
\date{27.08.2019}


\makeatletter
\begingroup%
  \def\and{, }%
  \let\thanks\@gobble%
  \let\footnote\@gobble%
  \let\emailaddress\@gobble%
  \hypersetup{%
    pdfauthor = {\@author},%
    pdftitle = {\@title},%
    pdfsubject = {Benutzerhandbuch für \TUDScript},%
    pdfkeywords = {LaTeX, \TUDScript, Benutzerhandbuch},%
  }%
\endgroup%
\makeatother

\maketitle

\section{title ä ö ü}

\Macro{caption}(\Package{koma-script},\Package{caption}) 

\end{document}
%
%\newcommand*\Kernel{\Package{lataex}}
%
%\label{ä:ö:ü}
%
%\Macro{foo}(\hologo{LaTeX})
%
%\Macro{bar}
%
%\Macro{blubb}(\Kernel)

%\Macro{check}(\Package{\LaTeX})

%\begin{Bundle*}{\Class{tudscrposter}}
%\begin{Declaration}{\Option{testoption=\PSet}}[true]
%\printdeclarationlist%
%\begin{values}{\Option{testoption}}
%\item[bicolor/bichrome]
%  Der Kopf wird mit einem farbigen Hintergrund in der Hausfarbe gesetzt, auch 
%  der Querbalken wird farbig hinterlegt. Für die Überschriften wird die 
%  primären Hausfarbe verwendet.
%\end{values}
%\Option{testoption}
%\end{Declaration}
%\Option{testoption}
%\end{Bundle*}

\begin{Bundle*}{\Class{foo}}
\begin{Declaration}{\Option{testoption=\PBoolean}}[true]
\printdeclarationlist%
\begin{values}{\Option{testoption}}
\itemtrue
  Der Kopf wird mit einem farbigen Hintergrund in der Hausfarbe gesetzt, auch 
  der Querbalken wird farbig hinterlegt. Für die Überschriften wird die 
  primären Hausfarbe verwendet.
\end{values}
\end{Declaration}
\end{Bundle*}


\Option{testoption=false}
\Option{testoption=true}

%\csmeaning{r@tudscr:opt:testoption=false}

\makeatletter
%\tud@lbl@get\bla{tudscr:opt:testoption}

%
%%\Option{backcolor}
%
%\Option{backcolor}(,\Package{xcolor})

\clearpage
%\PrintIndex

\end{document}
