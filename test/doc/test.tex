\PassOptionsToPackage{check=none}{widows-and-orphans}%
\documentclass[english,ngerman,ttfont=roboto,ToDo=inline,final]{tudscrmanual}
\GitHubBase{\TUDScriptRepository}
\iftutex
  \usepackage{fontspec}
\else
  \usepackage[T1]{fontenc}
  \input glyphtounicode.tex
  \pdfgentounicode=1
  \usepackage[ngerman=ngerman-x-latest]{hyphsubst}
\fi
\lstset{%
  inputencoding=utf8,extendedchars=true,
  literate=%
    {ä}{{\"a}}1 {ö}{{\"o}}1 {ü}{{\"u}}1
    {Ä}{{\"A}}1 {Ö}{{\"O}}1 {Ü}{{\"U}}1
    {ß}{{\ss}}1 {~}{{\textasciitilde}}1
    {»}{{\guillemetright}}1 {«}{{\guillemetleft}}1
}
\usepackage{widows-and-orphans}
\usepackage{bookmark}
\KOMAoptions{headings=optiontoheadandtoc}

%\tracinglabels*<created,matched>[all]
\tracinglabels[all]
%\tracingmarkup%[all]
\usepackage{blindtext}


%\DefaultEntity{\Class{datetime2}}{
%  \DefaultEntity{\Class{daatetime2}}{
%    \Macro{DTMDate},
%  }
%}

%\DefaultEntity{\Bundle{enumitem}}{
%  \DefaultEntity{\Package{enumitem}}{
%    \DefaultEntity{\Environment{itemize}}{\Macro{item}},
%    \DefaultEntity{\Environment{enumerate}}{\Macro{item}},
%    \DefaultEntity{\Environment{description}}{\Macro{item}},
%    \Macro{setlist},
%  }
%}

\DefaultEntity{\Bundle{letltxmacro}}'CTAN:foo'{\Package{letltxmacro}}

\begin{document}

\GitHubDownload*{tudscr_\vTUDScript.zip}

\GitHubDownload{atudscr_\vTUDScript.zip}

\end{document}
%
\noindent
\begin{quoting}
\renewcommand*\href[2]{\url{#1}~(#2)}%
\GitHubRepo<issues>\newline\Forum%
\end{quoting}

\begin{DeclarePackages}
\itempkg{scrbase,scrlayer-scrpage,typearea,scrletter,scrextend}<%
    \Bundle{koma-script},\Class*{koma-class},%
    \Class{scrbook},\Class{scrreprt},\Class{scrartcl}%
  >%
  Die zentrale Grundlage für \TUDScript bilden~-- zusammenfassend in diesem 
  Dokument gegebenenfalls als \Class||{koma-class} bezeichnet~-- die drei 
  Hauptklassen \Class{scrbook}, \Class{scrreprt} sowie \Class{scrartcl} aus dem 
  \scrguide[\KOMAScript-Bundle]. Das Paket \Package{scrbase} erlaubt die 
  Definition von Optionen oder Schlüsseln, die sich auch noch nach dem Laden 
  einer Klasse oder eines Paketes aus dem \TUDScript-Bundle mit den Befehlen 
  \Macro{TUDoption} und \Macro{TUDoptions} ändern lassen. Weiterhin werden die 
  \PageStyle{tudheadings}"=Seitenstile der \TUDScript-Klassen mithilfe des 
  Paketes \Package{scrlayer-scrpage} bereitgestellt. Wenn dieses nicht~-- mit 
  beliebigen Optionen~-- durch den Anwender geladen wird, erfolgt dies 
  automatisch am Ende der Präambel. Für die Festlegung des Satzspiegels wird
  das Paket \Package{typearea} genutzt.
\itempkg{opensans,mathastext,iwona}[v2.06]
  \index{Schriftart|!}%
  Das Paket \Package{opensans} stellt die Schriftfamilie \OpenSans sowohl für 
  den Fließtext als auch den mathematischen Satz zur Verfügung. Es enthält alle 
  nötigen Schriftschnitte sowohl im Type1- als auch im OpenType-Format. Da die 
  Schriftfamilie in der aktuellen Version keine mathematischen Glyphen 
  bereitstellt, werden die Pakete \Package{mathastext} und \Package{iwona} 
  zusätzlich genutzt, um zumindest einen halbwegs erträglichen mathematischen 
  Satz mit \OpenSans zu ermöglichen.
%  Werden dabei zusätzliche Symbole benötigt, wird empfohlen, auf das Paket 
%  \Package{amssymb} zu verzichten und anstelle dessen \Package{mdsymbol} zu 
%  laden.
  \ToDo{roboto-mono, hinweis zu mdsymbol raus}[v2.07]
  \ToDo{mweights raus, Text in Hinweise verschieben?!}[v2.07]
%\itempkg{mweights}
%  \index{Schriftstärke}%
%  In \hologo{LaTeX} existieren die Schriftfamilien für Serifenschriften 
%  (\Macro{rmfamily}), serifenlose Schriften (\Macro{sffamily}) sowie die 
%  Schreibmaschinenschriften (\Macro{ttfamily}). Deren Schriftstärke wird für 
%  gewöhnlich mit den Befehlen \Macro{mddefault} und \Macro{bfdefault} 
%  einheitlich festgelegt. Bei der Verwendung unterschiedlicher Schriftpakete 
%  kann es unter Umständen zu Problemen bei den Schriftstärken kommen. Diese 
%  Paket erlaubt die individuelle Definition der Schriftstärke für jede der 
%  drei Schriftfamilien.
\itempkg{geometry}
  \index{Satzspiegel}%
  Das Paket wird zum Festlegen der Seitenränder respektive des Satzspiegels 
  verwendet.
  \Attention{%
    Ein Weiterreichen zusätzlicher Optionen an das Paket wird dringlich nicht 
    empfohlen.
  }%
\itempkg{graphicx}
  \index{Grafiken}%
  Zum Einbinden des Logos der \TnUD im Kopf sowie aller weitere Abbildungen und 
  Logos wird \Macro{includegraphics} genutzt.
\itempkg{xcolor}
  \index{Farben}%
  Damit werden die Farben des \CDs zur Verwendung im Dokument definiert. 
  Genaueres ist bei der Beschreibung von \Package'ref'{tudscrcolor} zu finden. 
\itempkg{iftex,etoolbox,xpatch,letltxmacro}
  Diese Pakete stellen viele Funktionen zum Testen und zur Ablaufkontrolle 
  bereit. Weiterhin wird das Manipulieren vorhandener Makros ermöglicht.
  \ToDo{kvsetkeys, environ, nicht mehr verwenden, Doku raus!}[v2.07]
  \ToDo{text von environ to xparse?}[v2.07]
%\itempkg{kvsetkeys}
%  Das von \Package{scrbase} geladene Paket \Package{keyval} macht das 
%  Definieren von Klassen- und Paketoptionen sowie Parametern nach dem 
%  Schlüssel"=Wert"=Prinzip möglich. Mit diesem Paket kann das Verhalten für 
%  unbekannte Schlüssel festgelegt werden.
%\itempkg{environ}
%  \index{Befehlsdeklaration}%
%  Es wird eine verbesserte Deklaration von Umgebungen ermöglicht, bei der auch 
%  beim Abschluss der Umgebung auf die übergebenen Parameter zugegriffen werden 
%  kann. 
\itempkg{trimspaces}
  Bei mehreren Eingabefeldern für den Anwender werden die Argumente mithilfe 
  dieses Paketes um eventuell angegebene, unnötige Leerzeichen befreit.
\end{DeclarePackages}

\end{document}

\begin{Entity}{\Bundle{tudscr}}

\begin{Declaration}{\Option{declaration=\PMisc}}=hhh=
\begin{Declaration}{\Option{foo=\PValue{aaa}}}
\begin{Declaration}
  {\Environment{abstract|\OPList{Sprache}}}
%  [v2.02:Trennung einzelner Abschnitte mit \Macro{nextabstract};]
\begin{Declaration}
  {\Environment{abstract/markboth=\PMisc}}
  [v2.02]
\begin{Declaration}
  {\Environment{abstract/pagestyle=\PSet{Seitenstil}}}
%  [v2.02]
\printdeclarationlist
%
Mit \Option{declaration} kann äquivalent zur Option \Option{abstract} die 
%
\begin{DeclareValues}[\Environment{abstract/markboth}]
\itemval=bla=<foo>
  Es wird keine Überschrift über den Erklärungen selbst ausgegeben.
\itemval*=true=
  Es wird eine Überschrift über den Erklärungen selbst ausgegeben.
\end{DeclareValues}
%
%%\ChangedAt{v2.02}%
%%Mit dem Parameter \Environment{abstract/markboth} lassen sich die verwendeten 
%%Kolumnentitel anpassen. Die aktuellen~-- automatische respektive manuelle~-- 
%%werden mit \InlineDeclaration{\Environment{abstract/markboth=false}} 
%%verwendet. 
%%Die Einstellung \InlineDeclaration{\Environment{abstract/markboth=true}} 
%%wiederum setzt diese für linke und rechte Seiten auf \Term{abstractname}. Mit 
%%\InlineDeclaration{\Environment{abstract/markboth=\PName{Kolumnentitel}}} 
%%wird 
%%der Kolumnentitel manuell festgelegt. So lassen sich die Kolumnen mit 
%%\Environment{abstract/markboth=\MPValue{}} beispielsweise auch vollständig 
%%löschen. Sollte der Parameter \Environment{abstract/markboth} aktiviert 
%%werden, 
%%so wird in der Umgebung automatisch der Seitenstil \PageStyle{headings} 
%%genutzt~-- falls eine Titelseite (\KOMAScript-Option \Option{titlepage=true}) 
%%verwendet wird. Mit dem Parameter \Environment{abstract/pagestyle} lässt sich 
%%dieser auch direkt angeben, wobei die \PageStyle{tudheadings}"=Seitenstile 
%%ebenfalls unterstützt werden.
\end{Declaration}
\end{Declaration}
\end{Declaration}
\end{Declaration}
\end{Declaration}
\end{Entity}


%\InlineDeclaration{\Package{enumitem}}
%
%\InlineDeclaration{\Macro{setlist}}

\end{document}




\begin{Entity}{\Bundle{something}}
\begin{Declaration}{\Option{cdfont=\PMisc}}
\printdeclarationlist
%
Mit der Option \Option{cdfont} können durch den Anwender alle zentralen
\end{Declaration}
\end{Entity}

%\begin{Entity}{\Bundle{stuff}}
%\begin{Declaration}{\Option{cdfont=\PMisc}}
%\printdeclarationlist
%%
%Mit der Option \Option{cdfont} können durch den Anwender alle zentralen
%\end{Declaration}
%\end{Entity}


%\Option'auto'{cdfont}

\Package'GH:foo'{geometry}

\makeatletter
\tud@specialurl@resolve\tud@res@a{}{aaaaa}%

\meaning\tud@res@a


\end{document}

\ChangedAt{v2.02;v2.06}%
Durch das \CD werden keine Schriften für den Mathematiksatz festgelegt. Das ist 
insbesondere für mathematische Abhandlungen als auch ingenieur- und 
naturwissenschaftliche Dokumente nicht tragbar. Im Mathematikmodus werden 
deshalb die lateinischen Lettern mithilfe des Paketes \Package{mathastext}
sowie die griechischen Lettern der \OpenSans genutzt. Zur Ergänzung kann für 
weitere mathematische Symbole das Paket \Package{mdsymbol} geladen werden.
\ToDo{hinweis zu mdsymbol raus}[v2.07]

Diese Einstellung lässt sich deaktivieren, wodurch sich die Standardschriften 
oder gegebenenfalls die eines zusätzlichen Paketes für den mathematischen Satz 
nutzen lassen. Die dafür relevanten Einstellungen werden in \autoref{sec:math} 
erläutert. Weiterhin sind ergänzende Hinweise zu einem typografisch sauberen
Mathematiksatz in \autoref{sec:tut} zu finden.

\section{title}

\Macro{KOMAClassName}

\Macro{caption}

\end{document}

\begin{Declaration}
  {\Macro{foo}}
\printdeclarationlist
%
Mit dieser Option kann die Größe des unteren Seitenrandes angepasst werden, 
\end{Declaration}

\begin{DeclareEntity}{\Class{tudscrtest}}
\begin{Declaration}
  {\Macro{extrabottommargin}}
\printdeclarationlist
%
Mit dieser Option kann die Größe des unteren Seitenrandes angepasst werden, 
\end{Declaration}

\Macro'full'{extrabottommargin}

\Macro'full'{extrabottommargin/stuff}

\Macro'full'(\Class{tudscrtest},\Class{tudscrposter}){extrabottommargin/stuff}

\Macro'full'(\Class{tudscrtest}){extrabottommargin/stuff}

\Macro(\Class{tudscrposter}){extrabottommargin}

\end{DeclareEntity}

\begin{DeclareEntity}{\Class{tudscrposter}}
\begin{Declaration}
  {\Macro{extrabottommargin}}
\begin{Declaration}
  {\Macro{extrabottommargin/stuff}}
\printdeclarationlist
%
Mit dieser Option kann die Größe des unteren Seitenrandes angepasst werden, 
\end{Declaration}
\end{Declaration}

\Macro{extrabottommargin}

\Macro{extrabottommargin/stuff}

\Macro(\Class{tudscrposter},\Class{tudscrtest}){extrabottommargin}

\end{DeclareEntity}

\begin{DeclareEntity}{\Class{tudscrclass}}
\begin{Declaration}
  {\Macro{extrabottommargin}}
\begin{Declaration}
  {\Macro{extrabottommargin/stuff}}
\printdeclarationlist
%
Mit dieser Option kann die Größe des unteren Seitenrandes angepasst werden, 
\end{Declaration}
\end{Declaration}

\Macro{extrabottommargin}

\Macro{extrabottommargin/stuff}

\Macro(\Class{tudscrposter},\Class{tudscrtest}){extrabottommargin}

\end{DeclareEntity}

---------------------------------------

\Macro(\Bundle{tudscr}){extrabottommargin}

\Macro(\Bundle{tudscr},\Class{tudscrposter}){extrabottommargin}

\Macro(\Bundle{tudscr},\Class{tudscrposter},\Class{tudscrtest}){extrabottommargin}

\Macro{extrabottommargin/stuff}

\Macro{foo}

\Environment{itemize}

%\Macro(\Bundle{bar}){bar}
%
%\Macro(\Bundle{bar},\Bundle{blubb}){bar}

\Macro(\Class{tudscrposter},\Class{tudscrtest}){extrabottommargin}

\end{document}
