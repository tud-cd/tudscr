\chapter{Einleitung}
%
Zur fehlerfreien Verwendung der \TUDScript-Klassen in der Version~\vTUDScript{} 
werden sowohl die \KOMAScript-Klassen der Version~\vKOMAScript{} oder später 
als auch die Hausschrift des \CDs \OpenSans aus dem Paket \Package{opensans} 
zwingend benötigt. Zusätzlich müssen weitere Pakete verfügbar sein, welche 
unter \autoref{sec:packages:needed} aufgeführt sind. Beim Einsatz einer 
aktuellen Version von \TeXLive|?|, \MacTeX|?| oder \MiKTeX|?| sollte dies kein 
Problem darstellen. Wird jedoch eine \DistributionGeneral verwendet, die 
\TUDScript in der Version~\vTUDScript{} nicht zur Verfügung stellt und eine 
Aktualisierung dieser nicht möglich sein, so sollte \autoref{sec:install:ext} 
konsultiert werden. In diesem Fall ist der Anwender selbst dafür 
verantwortlich, alle benötigten Pakete in der jeweils notwendigen Version 
bereitzustellen, wobei sämtliche Paketabhängigkeiten zu beachten sind. Dieses 
Vorgehen ist jedoch äußerst fehleranfällig, weshalb dringlich dazu geraten 
wird, eine aktuelle \DistributionGeneral zu verwenden.



\section{Bestandteile und Nutzung des \TUDScript-Bundles}
%
\ChangedAt{v2.01:\TUDScript-Bundle über das \CTAN<pkg/tudscr> veröffentlicht}%
Das \TUDScript-Bundle wird über das \CTAN<pkg/tudscr> bereitgestellt und kann 
durch eine \DistributionGeneral wie \TeXLive, \MacTeX oder auch \MiKTeX genutzt 
werden. Es ist hauptsächlich für das Erstellen wissenschaftlicher Texte sowie 
Abhandlungen gedacht und stellt hierfür zum einen die drei Hauptklassen 
\Class{tudscrbook}, \Class{tudscrreprt} sowie \Class{tudscrartcl} und zum 
anderen die Klasse \Class{tudscrposter} zur Verfügung, welche in 
\autoref{sec:mainclasses} beziehungsweise \autoref{sec:poster} vorgestellt 
werden. Das Paket \Package{tudscrsupervisor}~-- in \autoref{sec:supervisor} 
dokumentiert~-- lässt sich zusätzlich in Verbindung mit diesen Klassen für die 
Erstellung von Aufgabenstellungen, Aushängen oder Gutachten zu studentischen 
Arbeiten nutzen. Weiterhin existieren auch eigenständige Pakete, welche in 
\autoref{sec:bundle} beschrieben sind. 

Für die Verwendung des \TUDScript-Bundles ist~-- neben \KOMAScript mindestens 
in der Version~\vKOMAScript{} sowie den in \autoref{sec:packages:needed} 
aufgeführten \hologo{LaTeX}"~Paketen~-- seit der Version~v2.06 lediglich die 
Schrift \OpenSans vonnöten, welche durch das Paket \Package{opensans} zur 
Verfügung gestellt wird. Eine lokale Nutzerinstallation der Schriften~-- wie in 
vorherigen Versionen~-- ist nicht notwendig. Lediglich für den Fall, dass 
gezielt die alten Schriften \Univers und \DIN eingesetzt werden sollen, müssen 
diese auch installiert sein. Weitere Hinweise zu deren Installation sowie 
Aktivierung sind in \autoref{sec:install:fonts} zu finden.

\minisec{Anmerkung zu Windows}
%
Unter Windows kann \TeXLive respektive \MiKTeX als \DistributionGeneral genutzt 
werden. Die Vorteile von \TeXLive liegen zum einen in der Wartung durch mehrere 
Autoren sowie der etwas früheren Verfügbarkeit aller Updates über das \CTAN. 
Zum anderen werden zusätzlich zu \hologo{LaTeX} ein \emph{Perl"~Interpreter} 
sowie \emph{Ghostscript} mitgeliefert, wodurch die Ad"~hoc"=Verwendung einiger 
Pakete wie beispielsweise \Package{glossaries} vereinfacht wird. Für \MiKTeX 
müssen diese externen Programme gegebenenfalls manuell installiert werden. 
Demgegenüber entfällt die alljährliche Neuinstallation, welche bei \TeXLive 
notwendig ist. Weiterhin werden zuvor nicht installierte jedoch benötigte 
\hologo{LaTeX}"~Pakete im Bedarfsfall~-- eine aktive Internetverbindung 
vorausgesetzt~-- automatisch nachinstalliert.

\minisec{Anmerkung zu Linux und OS~X}
%
Die Installation der \DistributionGeneral \TeXLive oder \MacTeX sollte direkt 
über die angebotenen Pakete (\mbox{\url{https://tug.org/texlive/}} oder 
\mbox{\url{https://tug.org/mactex/}}) und nicht über \Path{apt-get install} 
erfolgen. Damit wird sichergestellt, dass die aktuelle Variante der jeweiligen 
\DistributionGeneral genutzt wird.



\section{%
  Zur Verwendung dieses Handbuchs%
  \index{Optionen}%
  \index{Umgebungen}%
  \index{Befehle}%
}
%
Sämtliche neu definierten Optionen, Umgebungen und Befehle der Klassen und 
Pakete des \TUDScript-Bundles werden im Handbuch aufgeführt und beschrieben. Am 
Ende des Dokumentes befinden sich mehrere Indexe, die das Nachschlagen oder 
Auffinden dieser erleichtern sollen. Darin werden auch ausgewählte Optionen, 
Umgebungen und Befehle aufgeführt, welche nicht zu \TUDScript gehören und 
dennoch innerhalb dieses Handbuchs Erwähnung finden.

Die im Folgenden beschriebenen Optionen können~-- wie ein Großteil aller 
Einstellungen für \KOMAScript~-- als Schlüssel"=Wert"=Paare bei der Wahl der 
Dokumentklasse angegeben werden:
\Macro{documentclass|\OPValue{\PName{Schlüssel}=\PName{Wert}}\MPName{Klasse}}

Des Weiteren eröffnen die \KOMAScript-Klassen die Möglichkeit der späten 
Optionenwahl. Damit können Optionen nicht nur direkt beim Laden als sogenannte 
Klassenoptionen angegeben werden, sondern lassen sich auch noch innerhalb des 
Dokumentes nach dem Laden der Klasse ändern. Die \KOMAScript-Klassen sehen 
hierfür zwei Befehle vor. Mit \Macro{KOMAoptions|\MPName{Optionenliste}}
lassen sich beliebig viele Schlüsseln jeweils genau einen Wert zuweisen, 
\Macro{KOMAoption|\MPName{Option}\MPName{Werteliste}} erlaubt das gleichzeitige 
Setzen mehrere Werte für genau einen Schlüssel. Für die von \TUDScript 
\emph{zusätzlich} zur Verfügung gestellten Optionen werden äquivalent dazu die 
Befehle \Macro{TUDoptions|\MPName{Optionenliste}} und 
\Macro{TUDoption|\MPName{Option}\MPName{Werteliste}} definiert. Damit kann das 
Verhalten von Optionen im Dokument~-- innerhalb einer Gruppe auch lokal~-- 
geändert werden.

Bei der Beschreibung aller Optionen sind direkt neben dieser deren jeweilige 
Standardwerte mit \mbox{\enquote*{Voreinstellung: \PName*{Wert}}} angeführt. 
Einige dieser sind nicht immer gleich sondern werden in Abhängigkeit der 
genutzten Benutzereinstellungen und Optionen gesetzt. Diese bedingten 
Standardwerte werden mit 
\mbox{\enquote*{%
  Voreinstellung: \PName*{Wert}\,\textbar\,Bedingung: \PName*{bedingter Wert}%
}}
angegeben. Wird ein Schlüssel durch den Benutzer \emph{ohne} eine Wertzuweisung 
genutzt, so wird~-- falls vorhanden~-- ein vordefinierter Säumniswert gesetzt, 
welcher in der Beschreibung aller Optionen durch die~\PValue{\emph{kursive}} 
Schreibweise innerhalb der Werteliste gekennzeichnet ist. In den meisten Fällen 
ist der Säumniswert eines Schlüssels \PValue{true}, er entspricht folglich der 
Angabe \PName{Schlüssel}\PValue{=true}. Mit der expliziten Wertzuweisung eines 
Schlüssels werden sowohl einfache als auch bedingte Voreinstellungen in jedem 
Fall überschrieben. Die neben den Optionen neu eingeführten Umgebungen und 
Befehle der Klassen werden~-- gegebenenfalls zusammen mit den dazugehörigen 
optionalen Parametern~-- im gleichen Stil erläutert.


\ToDo{Deklarationen tud\dots ins Paket tudscrcomp verschieben}
\section{Weitere Klassen und Pakete für das \CD}
%
Für das Erstellen von Dokumenten im \TUDCD mit \hologo{LaTeX} existiert eine 
\hrfn{%
  https://tud.de/hilfe/kommunizieren-und-publizieren/cd/vorlagen/druck/latex%
}{Vielzahl von Paketen}. Das \TUDScript-Bundle soll diese nicht ersetzen, 
jedoch längerfristig sämtliche Variationen vereinheitlichen und mit einer 
konsistenten Benutzerschnittstelle ausstatten. 

Aktuell können ältere Dokumente, welche zuvor mit der Klasse \Class*{tudbook} 
und gegebenenfalls dem Paket \Package*{tudthesis} von Klaus~Bergmann gesetzt 
wurden, auf eine der \TUDScript-Klassen \Class{tudscrbook} beziehungsweise 
\Class{tudscrreprt} oder \Class{tudscrartcl} migriert werden. Bereits 
existierende Poster basierend auf \Class*{tudmathposter} und \Class*{tudposter} 
können ebenfalls auf die \TUDScript-Klasse \Class{tudscrposter} umgestellt 
werden. In allen Fällen kann dabei das Paket \Package'fullref*'{tudscrcomp} für 
einen einfacheren Umstieg zum Einsatz kommen. 

Für alle weiteren Klassen des Vorlagenbundles von Klaus~Bergmann~-- namentlich 
\Class*{tudletter}, \Class*{tudfax}, \Class*{tudhaus} und \Class*{tudform} 
sowie \Class*{tudbeamer}~-- werden durch \TUDScript \emph{momentan keine} 
äquivalenten Klassen angeboten. Sollten Sie auf deren Verwendung angewiesen 
sein, können Sie zumindest das Paket \Package'fullref*'{fix-tudscrfonts} 
verwenden, um die aktuellen Hausschriften zu aktivieren.

Eine anwenderfreundlich und vollständige Umsetzung des \CDs für Briefe und 
Geschäftsschreiben auf Basis von \KOMAScript ist bis jetzt leider noch nicht 
mit \TUDScript realisiert worden, soll jedoch langfristig erfolgen. Für das 
Erstellen von Briefen mit den \TUDScript-Klassen lässt sich allerdings relativ 
einfach das Paket \Package{scrletter} nutzen. Für Präsentationen im \TUDCD wird 
von \TUDScript bisher ebenfalls nichts angeboten, allerdings existieren für die 
\Class{beamer}"~Klasse bereits mehrere Stile, die im \GitHubRepo(tud-cd/tud-cd) 
zu finden sind. 



\section{Schnelleinstieg}
%
Das Handbuch gliedert sich in drei Teile. In \autoref{part:main} ist die 
Dokumentation von \TUDScript zu finden. Hier werden alle neuen Optionen, 
Umgebungen und Befehle, welche über die Funktionalität von \KOMAScript 
hinausgehen, erläutert. \autoref{part:additional} enthält zum einen einfache 
Minimalbeispiele, um den prinzipiellen Umgang und die Funktionalitäten von 
\TUDScript zu demonstrieren. Zum anderen werden hier auch ausführliche und 
dokumentierte Tutorials vor allem für \hologo{LaTeX}"=Neulinge angeboten. 
Insbesondere das Tutorial \Tutorial{treatise} ist mehr als einen Blick wert, 
wenn eine wissenschaftliche Arbeit mit \hologo{LaTeX} verfasst werden soll.
Abschließend werden verschiedenste Pakete vorgestellt, die nicht speziell für 
das \TUDScript-Bundle selber sondern auch für andere \hologo{LaTeX}"~Klassen
verwendet werden können und demzufolge für jeden Anwender interessant sein 
könnten. Außerdem werden hier einige Tipps \& Tricks beim Umgang mit 
\hologo{LaTeX} beschrieben, um kleinere oder größere Probleme zu lösen.

Die Klassen \Class{tudscrbook}, \Class{tudscrreprt} und \Class{tudscrartcl} 
sind Wrapper-Klassen der bekannten und korrelierenden \KOMAScript-Klassen 
\Class{scrbook}, \Class{scrreprt} sowie \Class{scrartcl} und können einfach 
anstelle deren verwendet werden, wobei ein Blick in das \scrguide sehr zu 
empfehlen ist~-- insbesondere wenn Sie noch keine oder nur wenig Erfahrung im 
Umgang mit den genannten Klassen haben. Auf diesen basierende Dokumente können 
durch das Umstellen der Dokumentklasse einfach in das \TUDCD überführt werden. 
Bei Fragestellungen bezüglich Layout, Schriften oder ähnlichem ist in jedem 
Fall ein weiterer Blick in das hier vorliegende Handbuch empfehlenswert.
